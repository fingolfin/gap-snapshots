% This file was created automatically from ctbl.msk.
% DO NOT EDIT!
%%%%%%%%%%%%%%%%%%%%%%%%%%%%%%%%%%%%%%%%%%%%%%%%%%%%%%%%%%%%%%%%%%%%%%%%%%%%%
%%
%A  ctbl.msk                    GAP documentation               Thomas Breuer
%A                                                           Alexander Hulpke
%%
%A  @(#)$Id: ctbl.msk,v 1.47.2.11 2006/11/02 10:22:25 jjm Exp $
%%
%Y  (C) 1998 School Math and Comp. Sci., University of St.  Andrews, Scotland
%Y  Copyright (C) 2002 The GAP Group
%%
%%  The documentation in this chapter corresponds to the declarations in the
%%  library files `ctbl.gd', `ctblauto.gd', `ctblgrp.gd', `ctblothe.gd',
%%  `ctblsolv.gd'.
%%
\Chapter{Character Tables}

\index{tables}

This chapter describes operations for *character tables of finite groups*.

Operations for *characters* (or, more general, *class functions*) are
described in Chapter~"Class Functions".

For a description of the {\GAP} Library of Character Tables,
see the separate manual for the {\GAP} package `ctbllib'.

Several examples in this chapter require the {\GAP} Character Table Library
to be available.
If it is not yet loaded then we load it now.

\beginexample
gap> LoadPackage( "ctbllib" );
true
\endexample


%%%%%%%%%%%%%%%%%%%%%%%%%%%%%%%%%%%%%%%%%%%%%%%%%%%%%%%%%%%%%%%%%%%%%%%%%%%%%
\Section{Some Remarks about Character Theory in GAP}

It seems to be necessary to state some basic facts --and maybe warnings--
at the beginning of the character theory package.
This holds for people who are familiar with character theory because
there is no global reference on computational character theory,
although there are many papers on this topic,
such as~\cite{NPP84} or~\cite{LP91}.
It holds, however, also for people who are familiar with {\GAP} because
the general concept of domains (see Chapter~"Domains") plays no important
role here --we  will justify this later in this section.

Intuitively, *characters* (or more generally, *class functions*) of a
finite group $G$ can be thought of as certain mappings defined on $G$,
with values in the complex number field;
the set of all characters of $G$ forms a semiring, with both addition
and multiplication defined pointwise, which is naturally embedded into
the ring of *generalized* (or *virtual*) *characters* in the natural way.
A ${\Z}$-basis of this ring, and also a vector space basis of the
complex vector space of class functions of $G$,
is given by the irreducible characters of $G$.

At this stage one could ask where there is a problem, since all these
algebraic structures are supported by {\GAP}.
But in practice, these structures are of minor importance,
compared to individual characters and the *character tables* themselves
(which are not domains in the sense of {\GAP}).

For computations with characters of a finite group $G$ with $n$ conjugacy
classes, say, we fix an ordering of the classes, and then identify each
class with its position according to this ordering.
Each character of $G$ can be represented by a list of length $n$ in which
the character value for elements of the $i$-th class is stored at
the $i$-th position.
Note that we need not know the conjugacy classes of $G$ physically,
even our knowledge of $G$ may be implicit in the sense that, e.g.,
we know how many classes of involutions $G$ has, and which length these
classes have, but we never have seen an element of $G$, or a presentation
or representation of $G$.
This allows us to work with the character tables of very large groups,
e.g., of the so-called monster, where {\GAP} has (currently) no chance
to deal with the group.

As a consequence, also other information involving characters is given
implicitly.  For example, we can talk about the kernel of a character not
as a group but as a list of classes (more exactly: a list of their
positions according to the chosen ordering of classes) forming this
kernel; we can deduce the group order, the contained cyclic subgroups
and so on, but we do not get the group itself.

So typical calculations with characters involve loops over lists of
character values.
For  example, the scalar product of two characters $\chi$, $\psi$ of $G$
given by
$$
[\chi,\psi] = \frac{1}{|G|} \sum_{g\in G} \chi(g) \psi(g^{-1})
$$
can be written as
\begintt
Sum( [ 1 .. n ], i -> SizesConjugacyClasses( t )[i] * chi[i]
                          * ComplexConjugate( psi[i] ) );
\endtt
where `t' is the character table of $G$, and `chi', `psi' are the lists
of values of $\chi$, $\psi$, respectively.

It is one of the advantages of character theory that after one has
translated a problem concerning groups into a problem concerning
only characters, the necessary calculations are mostly simple.
For example, one can often prove that a group is a Galois group over the
rationals using calculations with structure constants that can be
computed from the character table,
and information about (the character tables of) maximal subgroups.
When one deals with such questions,
the translation back to groups is just an interpretation by the user,
it does not take place in {\GAP}.

{\GAP} uses character *tables* to store information such as class
lengths, element orders, the irreducible characters of $G$ etc.~in a
consistent way;
in the example above, we have seen that `SizesConjugacyClasses( t )' is
the list of class lengths of the character table `t'.
Note that the values of these attributes rely on the chosen ordering
of conjugacy classes,
a character table is not determined by something similar to generators
of groups or rings in {\GAP} where knowledge could in principle be
recovered from the generators but is stored mainly for the sake of
efficiency.

Note that the character table of a group $G$ in {\GAP} must *not* be
mixed up with the list of complex irreducible characters of $G$.
The irreducible characters are stored in a character table via the
attribute `Irr' (see~"Irr").

Two further important instances of information that depends on the
ordering of conjugacy classes are *power maps* and *fusion maps*.
Both are represented as lists of integers in {\GAP}.
The $k$-th power map maps each class to the class of $k$-th powers
of its elements, the corresponding list contains at each position the
position of the image.
A class fusion map between the classes of a subgroup $H$ of $G$ and
the classes of $G$ maps each class $c$ of $H$ to that class of $G$ that
contains $c$, the corresponding list contains again the positions of
image classes;
if we know only the character tables of $H$ and $G$ but not the groups
themselves,
this means with respect to a fixed embedding of $H$ into $G$.
More about power maps and fusion maps can be found in
Chapter~"Maps Concerning Character Tables".

So class functions, power maps, and fusion maps are represented by lists
in {\GAP}.
If they are plain lists then they are regarded as class functions etc.~of
an appropriate character table when they are passed to {\GAP} functions
that expect class functions etc.
For example, a list with all entries equal to 1 is regarded as the
trivial character if it is passed to a function that expects a character.
Note that this approach requires the character table as an argument for
such a function.

One can construct class function objects that store their underlying
character table and other attribute values
(see Chapter~"Class Functions").
This allows one to omit the character table argument in many functions,
and it allows one to use infix operations for tensoring or inducing
class functions.




%%%%%%%%%%%%%%%%%%%%%%%%%%%%%%%%%%%%%%%%%%%%%%%%%%%%%%%%%%%%%%%%%%%%%%%%%%%%%
\Section{History of Character Theory Stuff in GAP}

{\GAP} provides functions for dealing with group characters since the version
{\GAP}~3.1, which was released in March 1992.
The reason for adding this branch of mathematics to the topics of {\GAP} was
(apart from the usefulness of character theoretic computations in general)
the insight that {\GAP} provides an ideal environment for developing the
algorithms needed.
In particular, it had been decided at Lehrstuhl D f{\accent127 u}r Mathematik
that the {\sf CAS} system (a standalone Fortran program together with a
database of character tables, see~\cite{NPP84})
should not be developed further and the functionality of {\sf CAS} should be
made available in {\GAP}.
The background was that extending {\sf CAS} (by new Fortran code) had turned
out to be much less flexible than writing analogous {\GAP} library code.

For integrating the existing character theory algorithms,
{\GAP}'s memory management and long integer arithmetic were useful as well as
the list handling --it is an important feature of character theoretic methods
that questions about groups are translated into manipulations of lists;
on the other hand, the datatype of cyclotomics (see Chapter~"Cyclotomics")
was added to the {\GAP} kernel because of the character theory algorithms.
For developing further code, also other areas of {\GAP}'s library became
interesting, such as permutation groups, finite fields, and polynomials.

The development of character theory code for {\GAP} has been supported
by several DFG grants,
in particular the project ``Representation Theory of Finite Groups and Finite
Dimensional Algebras'' (until 1991),
and the Schwerpunkt ``Algorithmische Zahlentheorie und Algebra''
(from 1991 until 1997).
Besides that, several Diploma theses at Lehrstuhl D were concerned with
the development and/or implementation of algorithms dealing with characters
in {\GAP}.

The major contributions can be listed as follows.
\beginlist%unordered
\item{$\bullet$}
    The arithmetic for the cyclotomics data type, following~\cite{Zum89},
    was first implemented by Marco van Meegen; an alternative approach was
    studied in the diploma thesis of Michael Scherner (see~\cite{Scherner92})
    but was not efficient enough;
    later Martin Sch{\accent127 o}nert replaced the implementation
    by a better one.
% and since then, there have been more than 30 local improvements ...

\item{$\bullet$}
    The basic routines for characters and character tables were written
    by Thomas Breuer and G{\accent127 o}tz Pfeiffer.

\item{$\bullet$}
    The lattice related functions, such as `LLL', `OrthogonalEmbeddings',
    and `DnLattice', were implemented by Ansgar Kaup (see~\cite{Kaup92}).

\item{$\bullet$}
    Functions for computing possible class fusions, possible power maps,
    and table automorphisms were written by Thomas Breuer (see~\cite{Bre91}).

\item{$\bullet$}
    Functions for computing possible permutation characters were written by
    Thomas Breuer (see~\cite{Bre91})
    and G{\accent127 o}tz Pfeiffer (see~\cite{Pfe91}).

\item{$\bullet$}
    Functions for computing character tables from groups were written by
    Alexander Hulpke (Dixon-Schneider algorithm, see~\cite{Hulpke93})
    and Hans Ulrich Besche (Baum algorithm and Conlon algorithm,
    see~\cite{Besche92}).

\item{$\bullet$}
    Functions for dealing with Clifford matrices were written by
    Ute Schiffer (see~\cite{Schiffer94}).

\item{$\bullet$}
    Functions for monomiality questions were written by Thomas Breuer and
    Erzs\'ebet Horv\'ath.
\endlist

Since then, the code has been maintained and extended further by
Alexander Hulpke (code related to his implementation of the Dixon-Schneider
algorithm) and Thomas Breuer.

% D. Bauten?

Currently {\GAP} does not provide special functionality for computing
Brauer character tables, but there is an interface to the {\sf MOC} system
(see~\cite{HJLP92}), and the {\GAP} Character Table Library contains many
known Brauer character tables.


%%%%%%%%%%%%%%%%%%%%%%%%%%%%%%%%%%%%%%%%%%%%%%%%%%%%%%%%%%%%%%%%%%%%%%%%%%%%%
\Section{Creating Character Tables}

\index{tables}
\index{character tables}
\index{library tables}
\index{character tables!access to}
\index{character tables!calculate}
\index{character tables!of groups}

There are in general five different ways to get a character table in
{\GAP}.
You can
\beginlist%ordered
\item{1.}
    compute the table from a group,
\item{2.}
    read a file that contains the table data,
\item{3.}
    construct the table using generic formulae,
\item{4.}
    derive it from known character tables, or
\item{5.}
    combine partial information about conjugacy classes, power maps
    of the group in question, and about (character tables of) some
    subgroups and supergroups.
\endlist

In 1., the computation of the irreducible characters is the hardest part;
the different algorithms available for this are described
in~"Computing the Irreducible Characters of a Group".
Possibility 2.~is used for the character tables in the {\GAP} Character 
Table Library, see the manual of this library.
Generic character tables --as addressed by 3.-- are described
in~"ctbllib:Generic Character Tables" in the manual of the {\GAP}
Character Table Library.
Several occurrences of 4.~are described
in~"Constructing Character Tables from Others".
The last of the above possibilities
*@is currently not supported and will be described in a chapter of its
own when it becomes available@*.

The operation `CharacterTable' (see~"CharacterTable") can be used for the
cases 1.--3.



\>CharacterTable( <G> ) O
\>CharacterTable( <G>, <p> ) O
\>CharacterTable( <ordtbl>, <p> ) O
\>CharacterTable( <name>[, <param>] ) O

Called with a group <G>, `CharacterTable' calls the attribute
`OrdinaryCharacterTable' (see~"OrdinaryCharacterTable").
Called with first argument a group <G> or an ordinary character table
<ordtbl>, and second argument a prime <p>, `CharacterTable' calls
the operation `BrauerTable' (see~"BrauerTable").
Called with a string <name> and perhaps optional parameters <param>,
`CharacterTable' delegates to `CharacterTableFromLibrary', which
tries to access the {\GAP} Character Table Library (see the manual of
this library for an overview of admissible strings <name>).

Probably the most interesting information about the character table is
its list of irreducibles, which can be accessed as the value of the
attribute `Irr' (see~"Irr").
If the argument of `CharacterTable' is a string <name> then the
irreducibles are just read from the library file,
therefore the returned table stores them already.
However, if `CharacterTable' is called with a group <G> or with an
ordinary character table <ordtbl>, the irreducible characters are *not*
computed by `CharacterTable'.
They are only computed when the `Irr' value is accessed for the first
time, for example when `Display' is called for the table
(see~"Printing Character Tables").
This means for example that `CharacterTable' returns its result very
quickly, and the first call of `Display' for this table may take some
time because the irreducible characters must be computed at that time
before they can be displayed together with other information stored on
the character table.
The value of the filter `HasIrr' indicates whether the irreducible
characters have been computed already.

The reason why `CharacterTable' does not compute the irreducible
characters is that there are situations where one only needs the
``table head'', that is, the information about class lengths, power maps
etc., but not the irreducibles.
For example, if one wants to inspect permutation characters of a group
then all one has to do is to induce the trivial characters of subgroups
one is interested in; for that, only class lengths and the class fusion
are needed.
Or if one wants to compute the Molien series (see~"MolienSeries") for a
given complex matrix group, the irreducible characters of this group are
in general of no interest.

For details about different algorithms to compute the irreducible
characters, see~"Computing the Irreducible Characters of a Group".

If the group <G> is given as an argument, `CharacterTable' accesses the
conjugacy classes of <G> and therefore causes that these classes are
computed if they were not yet stored
(see~"The Interface between Character Tables and Groups").


\>BrauerTable( <ordtbl>, <p> ) O
\>BrauerTable( <G>, <p> ) O
\>BrauerTableOp( <ordtbl>, <p> ) O
\>ComputedBrauerTables( <ordtbl> ) AM

Called with an ordinary character table <ordtbl> or a group <G>,
`BrauerTable' returns its <p>-modular character table
if {\GAP} can compute this table, and `fail' otherwise.
The <p>-modular table can be computed for <p>-solvable groups
(using the Fong-Swan Theorem) and in the case that <ordtbl> is a table
from the {\GAP} character table library for which also the <p>-modular
table is contained in the table library.

The default method for a group and a prime delegates to `BrauerTable' for
the ordinary character table of this group.
The default method for <ordtbl> uses the attribute
`ComputedBrauerTables' for storing the computed Brauer table
at position <p>, and calls the operation `BrauerTableOp' for
computing values that are not yet known.

So if one wants to install a new method for computing Brauer tables
then it is sufficient to install it for `BrauerTableOp'.

The `\\mod' operator for a character table and a prime
(see~"Operators for Character Tables") delegates to
`BrauerTable'.


\beginexample
gap> g:= SymmetricGroup( 4 );
Sym( [ 1 .. 4 ] )
gap> tbl:= CharacterTable( g );;  HasIrr( tbl );
false
gap> tblmod2:= CharacterTable( tbl, 2 );
BrauerTable( Sym( [ 1 .. 4 ] ), 2 )
gap> tblmod2 = CharacterTable( tbl, 2 );
true
gap> tblmod2 = BrauerTable( tbl, 2 );
true
gap> tblmod2 = BrauerTable( g, 2 );
true
gap> CharacterTable( "A5" );
CharacterTable( "A5" )
gap> CharacterTable( "Symmetric", 4 );
CharacterTable( "Sym(4)" )
gap> ComputedBrauerTables( tbl );
[ , BrauerTable( Sym( [ 1 .. 4 ] ), 2 ) ]
\endexample

\>`SupportedCharacterTableInfo' V

`SupportedCharacterTableInfo' is a list that contains at position $3i-2$
an attribute getter function, at position $3i-1$ the name of this
attribute, and at position $3i$ a list containing one or two of the
strings `\"class\"', `\"character\"',
depending on whether the attribute value relies on the ordering of
classes or characters.
This allows one to set exactly the components with these names in the
record that is later converted to the new table,
in order to use the values as attribute values.
So the record components that shall *not* be regarded as attribute values
can be ignored.
Also other attributes of the old table are ignored.

`SupportedCharacterTableInfo' is used when (ordinary or Brauer) character
table objects are created from records, using `ConvertToCharacterTable'
(see~"ConvertToCharacterTable").

New attributes and properties can be notified to
`SupportedCharacterTableInfo' by creating them with
`DeclareAttributeSuppCT' and `DeclarePropertySuppCT' instead of
`DeclareAttribute' and `DeclareProperty'.


\>ConvertToCharacterTable( <record> ) F
\>ConvertToCharacterTableNC( <record> ) F

Let <record> be a record.
`ConvertToCharacterTable' converts <record> into a component object
(see~"prg:Component Objects" in ``Programming in {\GAP}'')
representing a character table.
The values of those components of <record> whose names occur in
`SupportedCharacterTableInfo' (see~"SupportedCharacterTableInfo")
correspond to attribute values of the returned character table.
All other components of the record simply become components of the
character table object.

If inconsistencies in <record> are detected, `fail' is returned.
<record> must have the component `UnderlyingCharacteristic' bound
(see~"UnderlyingCharacteristic"),
since this decides about whether the returned character table lies in
`IsOrdinaryTable' or in `IsBrauerTable'
(see~"IsOrdinaryTable", "IsBrauerTable").

`ConvertToCharacterTableNC' does the same except that all checks of
<record> are omitted.

An example of a conversion from a record to a character table object
can be found in Section~"PrintCharacterTable".




%%%%%%%%%%%%%%%%%%%%%%%%%%%%%%%%%%%%%%%%%%%%%%%%%%%%%%%%%%%%%%%%%%%%%%%%%%%%%
\Section{Character Table Categories}

\>IsNearlyCharacterTable( <obj> ) C
\>IsCharacterTable( <obj> ) C
\>IsOrdinaryTable( <obj> ) C
\>IsBrauerTable( <obj> ) C
\>IsCharacterTableInProgress( <obj> ) C

Every ``character table like object'' in {\GAP} lies in the category
`IsNearlyCharacterTable'.
There are four important subcategories,
namely the *ordinary* tables in `IsOrdinaryTable',
the *Brauer* tables in `IsBrauerTable',
the union of these two in `IsCharacterTable',
and the *incomplete ordinary* tables in `IsCharacterTableInProgress'.

We want to distinguish ordinary and Brauer tables because a Brauer table
may delegate tasks to the ordinary table of the same group,
for example the computation of power maps.
A Brauer table is constructed from an ordinary table and stores this
table upon construction (see~"OrdinaryCharacterTable").

Furthermore,  `IsOrdinaryTable'  and  `IsBrauerTable'  denote   character
tables that provide enough information to  compute  all  power  maps  and
irreducible characters (and in the case  of  Brauer  tables  to  get  the
ordinary   table),   for   example   because   the    underlying    group
(see~"UnderlyingGroup!for character tables")  is  known  or  because  the
table is a library table
(see the manual of the {\GAP} Character Table Library).
We want to distinguish these tables from partially known ordinary tables
that cannot be asked for all power maps or all irreducible characters.

The character table objects in `IsCharacterTable' are always immutable
(see~"Mutability and Copyability").
This means mainly that the ordering of conjugacy classes used for the
various attributes of the character table cannot be changed;
see~"Sorted Character Tables" for how to compute a character table with a
different ordering of classes.

The {\GAP} objects in `IsCharacterTableInProgress' represent incomplete
ordinary character tables.
This means that not all irreducible characters, not all power maps are
known, and perhaps even the number of classes and the centralizer orders
are known.
Such tables occur when the character table of a group $G$ is constructed
using character tables of related groups and information about $G$ but
for example without explicitly computing the conjugacy classes of $G$.
An object in `IsCharacterTableInProgress' is first of all *mutable*,
so *nothing is stored automatically* on such a table,
since otherwise one has no control of side-effects when
a hypothesis is changed.
Operations for such tables may return more general values than for
other tables, for example class functions may contain unknowns
(see Chapter~"Unknowns") or lists of possible values in certain
positions,
the same may happen also for power maps and class fusions
(see~"Parametrized Maps").
*@Incomplete tables in this sense are currently not supported and will be
described in a chapter of their own when they become available.@*
Note that the term ``incomplete table'' shall express that {\GAP} cannot
compute certain values such as irreducible characters or power maps.
A table with access to its group is therefore always complete,
also if its irreducible characters are not yet stored.


\beginexample
gap> g:= SymmetricGroup( 4 );;
gap> tbl:= CharacterTable( g );  modtbl:= tbl mod 2;
CharacterTable( Sym( [ 1 .. 4 ] ) )
BrauerTable( Sym( [ 1 .. 4 ] ), 2 )
gap> IsCharacterTable( tbl );  IsCharacterTable( modtbl );
true
true
gap> IsBrauerTable( modtbl );  IsBrauerTable( tbl );
true
false
gap> IsOrdinaryTable( tbl );  IsOrdinaryTable( modtbl );
true
false
gap> IsCharacterTable( g );  IsCharacterTable( Irr( g ) );
false
false
\endexample

\>`InfoCharacterTable' V

is the info class (see~"Info Functions") for computations with
character tables.



\>`NearlyCharacterTablesFamily' V

Every character table like object lies in this family (see~"Families").




%%%%%%%%%%%%%%%%%%%%%%%%%%%%%%%%%%%%%%%%%%%%%%%%%%%%%%%%%%%%%%%%%%%%%%%%%%%%%
\Section{Conventions for Character Tables}

The following few conventions should be noted.
\beginlist%unordered
\item{--}
    The class of the *identity element* is expected to be the first one;
    thus the degree of a character is the character value at position $1$.
\item{--}
    The *trivial character* of a character table need not be the first in
    the list of irreducibles.
\item{--}
    Most functions that take a character table as an argument and work with
    characters expect these characters as an argument, too.
    For some functions, the list of irreducible characters serves as the
    default, i.e, the value of the attribute `Irr' (see~"Irr");
    in these cases, the `Irr' value is automatically computed if it was not
    yet known.
\item{--}
    For a stored class fusion, the image table is denoted by its `Identifier'
    value (see~"Identifier!for character tables");
    each library table has a unique identifier by which it can be accessed
    (see~"ctbllib:Access to Library Character Tables" in the manual for the
    {\GAP} Character Table Library),
    tables constructed from groups get an identifier that is unique in the
    current {\GAP} session.
\endlist


%%%%%%%%%%%%%%%%%%%%%%%%%%%%%%%%%%%%%%%%%%%%%%%%%%%%%%%%%%%%%%%%%%%%%%%%%%%%%
\Section{The Interface between Character Tables and Groups}

For a character table  with  underlying  group  (see~"UnderlyingGroup!for
character tables"), the interface between table  and  group  consists  of
three attribute values,  namely  the  *group*,  the  *conjugacy  classes*
stored  in   the   table   (see   `ConjugacyClasses'   below)   and   the
*identification*  of  the  conjugacy   classes   of   table   and   group
(see~`IdentificationOfConjugacyClasses' below).

Character tables constructed from groups know these values upon
construction,
and for character tables constructed without groups, these values are
usually not known and cannot be computed from the table.

However, given a group $G$ and a character table of a group isomorphic to
$G$ (for example a character table from the {\GAP} table library),
one can tell {\GAP} to use the given table as the character table of $G$
(see~"ConnectGroupAndCharacterTable").

Tasks may be delegated from a group to its character table or vice versa
only if these three attribute values are stored in the character table.



\>UnderlyingGroup( <ordtbl> )!{for character tables} A

For an ordinary character table <ordtbl> of a finite group,
the group can be stored as value of `UnderlyingGroup'.

Brauer tables do not store the underlying group,
they access it via the ordinary table (see~"OrdinaryCharacterTable").


\>ConjugacyClasses( <tbl> )!{for character tables} A

For a character table <tbl> with known underlying group $G$,
the `ConjugacyClasses' value of <tbl> is a list of conjugacy classes of
$G$.
All those lists stored in the table that are related to the ordering
of conjugacy classes (such as sizes of centralizers and conjugacy
classes, orders of representatives, power maps, and all class functions)
refer to the ordering of this list.

This ordering need *not* coincide with the ordering of conjugacy classes
as stored in the underlying group of the table
(see~"Sorted Character Tables").
One reason for this is that otherwise we would not be allowed to
use a library table as the character table of a group for which the
conjugacy classes are stored already.
(Another, less important reason is that we can use the same group as
underlying group of character tables that differ only w.r.t.~the ordering
of classes.)

The class of the identity element must be the first class
(see~"Conventions for Character Tables").

If <tbl> was constructed from $G$ then the conjugacy classes have been
stored at the same time when $G$ was stored.
If $G$ and <tbl> were connected later than in the construction of <tbl>,
the recommended way to do this is via `ConnectGroupAndCharacterTable'
(see~"ConnectGroupAndCharacterTable").
So there is no method for `ConjugacyClasses' that computes the value for
<tbl> if it is not yet stored.

Brauer tables do not store the ($p$-regular) conjugacy classes,
they access them via the ordinary table (see~"OrdinaryCharacterTable")
if necessary.


\>IdentificationOfConjugacyClasses( <tbl> ) A

For an ordinary character table <tbl> with known underlying group $G$,
`IdentificationOfConjugacyClasses' returns a list of positive integers
that contains at position $i$ the position of the $i$-th conjugacy class
of <tbl> in the list $`ConjugacyClasses'( G )$.


\beginexample
gap> g:= SymmetricGroup( 4 );;
gap> repres:= [ (1,2), (1,2,3), (1,2,3,4), (1,2)(3,4), () ];;
gap> ccl:= List( repres, x -> ConjugacyClass( g, x ) );;
gap> SetConjugacyClasses( g, ccl );
gap> tbl:= CharacterTable( g );;   # the table stores already the values
gap> HasConjugacyClasses( tbl );  HasUnderlyingGroup( tbl );
true
true
gap> UnderlyingGroup( tbl ) = g;
true
gap> HasIdentificationOfConjugacyClasses( tbl );
true
gap> IdentificationOfConjugacyClasses( tbl );
[ 5, 1, 2, 3, 4 ]
\endexample

\>ConnectGroupAndCharacterTable( <G>, <tbl>[, <arec>] ) F
\>ConnectGroupAndCharacterTable( <G>, <tbl>, <bijection> ) F

Let <G> be a group and <tbl> a character table of (a group isomorphic to)
<G>, such that <G> does not store its `OrdinaryCharacterTable' value
and <tbl> does not store its `UnderlyingGroup' value.
`ConnectGroupAndCharacterTable' calls `CompatibleConjugacyClasses',
trying to identify the classes of <G> with the columns of <tbl>.

If this identification is unique up to automorphisms of <tbl>
(see~"AutomorphismsOfTable") then <tbl> is stored as `CharacterTable'
value of <G>,
in <tbl> the values of `UnderlyingGroup', `ConjugacyClasses', and
`IdentificationOfConjugacyClasses' are set,
and `true' is returned.

Otherwise, i.e., if {\GAP} cannot identify the classes of <G> up to
automorphisms of <G>, `false' is returned.

If a record <arec> is present as third argument, its meaning is the
same as for `CompatibleConjugacyClasses'
(see~"CompatibleConjugacyClasses").

If a list <bijection> is entered as third argument,
it is used as value of `IdentificationOfConjugacyClasses',
relative to `ConjugacyClasses( <G> )',
without further checking, and `true' is returned.


\>CompatibleConjugacyClasses( <G>, <ccl>, <tbl>[, <arec>] ) O
\>CompatibleConjugacyClasses( <tbl>[, <arec>] ) O

In the first form, <ccl> must be a list of the conjugacy classes of the
group <G>, and <tbl> the ordinary character table of <G>.
Then `CompatibleConjugacyClasses' returns a list $l$ of positive integers
that describes an identification of the columns of <tbl> with the
conjugacy classes <ccl> in the sense that $l[i]$ is the position in <ccl>
of the class corresponding to the $i$-th column of <tbl>,
if this identification is unique up to automorphisms of <tbl>
(see~"AutomorphismsOfTable");
if {\GAP} cannot identify the classes, `fail' is returned.

In the second form, <tbl> must be an ordinary character table, and
`CompatibleConjugacyClasses' checks whether the columns of <tbl> can be
identified with the conjugacy classes of a group isomorphic to that for
which <tbl> is the character table;
the return value is a list of all those sets of class positions for which
the columns of <tbl> cannot be distinguished with the invariants used,
up to automorphisms of <tbl>.
So the identification is unique if and only if the returned list is
empty.

The usual approach is that one first calls `CompatibleConjugacyClasses'
in the second form for checking quickly whether the first form will be
successful, and only if this is the case the more time consuming
calculations with both group and character table are done.

The following invariants are used.
\beginlist%ordered
\item{1.} element orders (see~"OrdersClassRepresentatives"),
\item{2.} class lengths (see~"SizesConjugacyClasses"),
\item{3.} power maps (see~"PowerMap", "ComputedPowerMaps"),
\item{4.} symmetries of the table (see~"AutomorphismsOfTable").
\endlist

If the optional argument <arec> is present then it must be a record
whose components describe additional information for the class
identification.
The following components are supported.
\beginitems
`natchar' &
    if $G$ is a permutation group or matrix group then the value of this
    component is regarded as the list of values of the natural character
    (see~"NaturalCharacter") of <G>,
    w.r.t.~the ordering of classes in <tbl>,

`bijection' &
    a list describing a partial bijection; the $i$-th entry, if bound,
    is the position of the $i$-th conjugacy class of <tbl> in the list
    <ccl>.
\enditems


\beginexample
gap> g:= AlternatingGroup( 5 );
Alt( [ 1 .. 5 ] )
gap> tbl:= CharacterTable( "A5" );
CharacterTable( "A5" )
gap> HasUnderlyingGroup( tbl );  HasOrdinaryCharacterTable( g );
false
false
gap> CompatibleConjugacyClasses( tbl );   # unique identification
[  ]
gap> ConnectGroupAndCharacterTable( g, tbl );
true
gap> HasConjugacyClasses( tbl );  HasUnderlyingGroup( tbl );
true
true
gap> IdentificationOfConjugacyClasses( tbl );
[ 1, 2, 3, 4, 5 ]
gap> # Here is an example where the identification is not unique.
gap> CompatibleConjugacyClasses( CharacterTable( "J2" ) );
[ [ 17, 18 ], [ 9, 10 ] ]
\endexample


%%%%%%%%%%%%%%%%%%%%%%%%%%%%%%%%%%%%%%%%%%%%%%%%%%%%%%%%%%%%%%%%%%%%%%%%%%%
\Section{Operators for Character Tables}

\indextt{\*!for character tables}
\indextt{/!for character tables}
\indextt{mod!for character tables}
\index{character tables!infix operators}

The following infix operators are defined for character tables.
\beginitems
`<tbl1> \* <tbl2>' &
    the direct product of two character tables
    (see~"CharacterTableDirectProduct"),

`<tbl> / <list>' &
    the table of the factor group modulo the normal subgroup spanned by
    the classes in the list <list> (see~"CharacterTableFactorGroup"),

`<tbl> mod <p>' &
    the <p>-modular Brauer character table corresponding to the ordinary
    character table <tbl> (see~"CharacterTable"),

`<tbl>.<name>' &
    the position of the class with name <name> in <tbl>
    (see~"ClassNames").
\enditems




%%%%%%%%%%%%%%%%%%%%%%%%%%%%%%%%%%%%%%%%%%%%%%%%%%%%%%%%%%%%%%%%%%%%%%%%%%%%%
\Section{Attributes and Properties of Character Tables}

Several *attributes for groups* are valid also for character tables.
These are on one hand those that have the same meaning for both group and
character table, and whose values can be read off or computed,
respectively, from the character table,
such as `Size', `IsAbelian', or `IsSolvable'.
On the other hand, there are attributes whose meaning for character
tables is different from the meaning for groups, such as
`ConjugacyClasses'.



\>CharacterDegrees( <G> ) A
\>CharacterDegrees( <G>, <p> ) O
\>CharacterDegrees( <tbl> ) A

In the first two forms, `CharacterDegrees' returns a collected list of
the degrees of the absolutely irreducible characters of the group <G>;
the optional second argument <p> must be either zero or a prime integer
denoting the characteristic, the default value is zero.
In the third form, <tbl> must be an (ordinary or Brauer) character
table, and `CharacterDegrees' returns a collected list of the degrees of
the absolutely irreducible characters of <tbl>.

(The default method for the call with only argument a group is to call
the operation with second argument `0'.)

For solvable groups, the default method is based on~\cite{Con90b}.


\beginexample
gap> CharacterDegrees( SymmetricGroup( 4 ) );
[ [ 1, 2 ], [ 2, 1 ], [ 3, 2 ] ]
gap> CharacterDegrees( SymmetricGroup( 4 ), 2 );
[ [ 1, 1 ], [ 2, 1 ] ]
gap> CharacterDegrees( CharacterTable( "A5" ) );
[ [ 1, 1 ], [ 3, 2 ], [ 4, 1 ], [ 5, 1 ] ]
gap> CharacterDegrees( CharacterTable( "A5" ) mod 2 );
[ [ 1, 1 ], [ 2, 2 ], [ 4, 1 ] ]
\endexample

\>Irr( <G> ) A
\>Irr( <G>, <p> ) O
\>Irr( <tbl> ) A

Called with a group <G>, `Irr' returns the irreducible characters of the
ordinary character table of <G>.
Called with a group <G> and a prime integer <p>, `Irr' returns the
irreducible characters of the <p>-modular Brauer table of <G>.
Called with an (ordinary or Brauer) character table <tbl>,
`Irr' returns the list of all complex absolutely irreducible characters
of <tbl>.

For a character table <tbl> with underlying group,
`Irr' may delegate to the group.
For a group <G>, `Irr' may delegate to its character table only if the
irreducibles are already stored there.

(If <G> is <p>-solvable (see~"IsPSolvable") then the <p>-modular
irreducible characters can be computed by the Fong-Swan Theorem;
in all other cases, there may be no method.)

Note that the ordering of columns in the `Irr' matrix of the group <G>
refers to the ordering of conjugacy classes in `CharacterTable( <G> )',
which may differ from the ordering of conjugacy classes in <G>
(see~"The Interface between Character Tables and Groups").
As an extreme example, for a character table obtained from sorting the
classes of `CharacterTable( <G> )',
the ordering of columns in the `Irr' matrix respects the sorting of
classes (see~"Sorted Character Tables"),
so the irreducibles of such a table will in general not coincide with
the irreducibles stored as `Irr( <G> )' although also the sorted table
stores the group <G>.

The ordering of the entries in the attribute `Irr' of a group need *not*
coincide with the ordering of its `IrreducibleRepresentations'
(see~"IrreducibleRepresentations") value.



In the following example we temporarily increase the line length limit from
its default value 80 to 85 in order to get a nicer output format.

\beginexample
gap> Irr( SymmetricGroup( 4 ) );
[ Character( CharacterTable( Sym( [ 1 .. 4 ] ) ), [ 1, -1, 1, 1, -1 ] ), 
  Character( CharacterTable( Sym( [ 1 .. 4 ] ) ), [ 3, -1, -1, 0, 1 ] ), 
  Character( CharacterTable( Sym( [ 1 .. 4 ] ) ), [ 2, 0, 2, -1, 0 ] ), 
  Character( CharacterTable( Sym( [ 1 .. 4 ] ) ), [ 3, 1, -1, 0, -1 ] ), 
  Character( CharacterTable( Sym( [ 1 .. 4 ] ) ), [ 1, 1, 1, 1, 1 ] ) ]
gap> Irr( SymmetricGroup( 4 ), 2 );
[ Character( BrauerTable( Sym( [ 1 .. 4 ] ), 2 ), [ 1, 1 ] ), 
  Character( BrauerTable( Sym( [ 1 .. 4 ] ), 2 ), [ 2, -1 ] ) ]
gap> SizeScreen([ 85, ]);;
gap> Irr( CharacterTable( "A5" ) );
[ Character( CharacterTable( "A5" ), [ 1, 1, 1, 1, 1 ] ), 
  Character( CharacterTable( "A5" ), [ 3, -1, 0, -E(5)-E(5)^4, -E(5)^2-E(5)^3 ] ), 
  Character( CharacterTable( "A5" ), [ 3, -1, 0, -E(5)^2-E(5)^3, -E(5)-E(5)^4 ] ), 
  Character( CharacterTable( "A5" ), [ 4, 0, 1, -1, -1 ] ), 
  Character( CharacterTable( "A5" ), [ 5, 1, -1, 0, 0 ] ) ]
gap> SizeScreen([ 80, ]);;
gap> Irr( CharacterTable( "A5" ) mod 2 );
[ Character( BrauerTable( "A5", 2 ), [ 1, 1, 1, 1 ] ), 
  Character( BrauerTable( "A5", 2 ), [ 2, -1, E(5)+E(5)^4, E(5)^2+E(5)^3 ] ), 
  Character( BrauerTable( "A5", 2 ), [ 2, -1, E(5)^2+E(5)^3, E(5)+E(5)^4 ] ), 
  Character( BrauerTable( "A5", 2 ), [ 4, 1, -1, -1 ] ) ]
\endexample

\>LinearCharacters( <G> ) A
\>LinearCharacters( <G>, <p> ) O
\>LinearCharacters( <tbl> ) A

`LinearCharacters' returns the linear (i.e., degree $1$) characters in
the `Irr' (see~"Irr") list of the group <G> or the character table <tbl>,
respectively.
In the second form, `LinearCharacters' returns the <p>-modular linear
characters of the group <G>.

For a character table <tbl> with underlying group,
`LinearCharacters' may delegate to the group.
For a group <G>, `LinearCharacters' may delegate to its character table
only if the irreducibles are already stored there.

The ordering of linear characters in <tbl> need not coincide with the
ordering of linear characters in the irreducibles of <tbl> (see~"Irr").


\beginexample
gap> LinearCharacters( SymmetricGroup( 4 ) );
[ Character( CharacterTable( Sym( [ 1 .. 4 ] ) ), [ 1, 1, 1, 1, 1 ] ), 
  Character( CharacterTable( Sym( [ 1 .. 4 ] ) ), [ 1, -1, 1, 1, -1 ] ) ]
\endexample

\>OrdinaryCharacterTable( <G> ) A
\>OrdinaryCharacterTable( <modtbl> ) A

`OrdinaryCharacterTable' returns the ordinary character table of the
group <G> or the Brauer character table <modtbl>, respectively.

Since Brauer character tables are constructed from ordinary tables,
the attribute value for <modtbl> is already stored
(cf.~"Character Table Categories").


\beginexample
gap> OrdinaryCharacterTable( SymmetricGroup( 4 ) );
CharacterTable( Sym( [ 1 .. 4 ] ) )
gap> tbl:= CharacterTable( "A5" );;  modtbl:= tbl mod 2;
BrauerTable( "A5", 2 )
gap> OrdinaryCharacterTable( modtbl ) = tbl;
true
\endexample

The following operations for groups are applicable to character tables
and mean the same for a character table as for the group;
see the chapter about groups for the definition.
\beginitems
\indextt{AbelianInvariants!for character tables}
`AbelianInvariants'&
\indextt{CommutatorLength!for character tables}
`CommutatorLength'&
\indextt{Exponent!for character tables}
`Exponent'&
\indextt{IsAbelian!for character tables}
`IsAbelian'&
\indextt{IsCyclic!for character tables}
`IsCyclic'&
\indextt{IsElementaryAbelian!for character tables}
`IsElementaryAbelian'&
\indextt{IsFinite!for character tables}
`IsFinite'&
\indextt{IsMonomial!for character tables}
`IsMonomial'&
\indextt{IsNilpotent!for character tables}
`IsNilpotent'&
\indextt{IsPerfect!for character tables}
`IsPerfect'&
\indextt{IsSimple!for character tables}
`IsSimple'&
\indextt{IsSolvable!for character tables}
`IsSolvable'&
\indextt{IsSporadicSimple!for character tables}
`IsSporadicSimple'&
\indextt{IsSupersolvable!for character tables}
`IsSupersolvable'&
\indextt{NrConjugacyClasses!for character tables}
`NrConjugacyClasses'&
\indextt{Size!for character tables}
`Size'&
\enditems
These operations are mainly useful for selecting character tables with
certain properties, also for character tables without access to a group.


\beginexample
gap> tables:= [ CharacterTable( CyclicGroup( 3 ) ),
>               CharacterTable( SymmetricGroup( 4 ) ),
>               CharacterTable( AlternatingGroup( 5 ) ) ];;
gap> List( tables, AbelianInvariants );
[ [ 3 ], [ 2 ], [  ] ]
gap> List( tables, CommutatorLength );
[ 1, 1, 1 ]
gap> List( tables, Exponent );
[ 3, 12, 30 ]
gap> List( tables, IsAbelian );
[ true, false, false ]
gap> List( tables, IsCyclic );
[ true, false, false ]
gap> List( tables, IsFinite );
[ true, true, true ]
gap> List( tables, IsMonomial );
[ true, true, false ]
gap> List( tables, IsNilpotent );
[ true, false, false ]
gap> List( tables, IsPerfect );
[ false, false, true ]
gap> List( tables, IsSimple );
[ true, false, true ]
gap> List( tables, IsSolvable );
[ true, true, false ]
gap> List( tables, IsSupersolvable );
[ true, false, false ]
gap> List( tables, NrConjugacyClasses );
[ 3, 5, 5 ]
gap> List( tables, Size );
[ 3, 24, 60 ]
\endexample

The following three *attributes for character tables* would make sense
also for groups but are in fact *not* used for groups.
This is because the values depend on the ordering of conjugacy classes
stored as value of `ConjugacyClasses', and this value may differ for a
group and its character table
(see~"The Interface between Character Tables and Groups").
Note that for character tables, the consistency of attribute values must
be guaranteed,
whereas for groups, there is no need to impose such a consistency rule.



\>OrdersClassRepresentatives( <tbl> ) A

is a list of orders of representatives of conjugacy classes of the
character table <tbl>,
in the same ordering as the conjugacy classes of <tbl>.


\>SizesCentralizers( <tbl> ) A

is a list that stores at position $i$ the size of the centralizer of any
element in the $i$-th conjugacy class of the character table <tbl>.


\>SizesConjugacyClasses( <tbl> ) A

is a list that stores at position $i$ the size of the $i$-th conjugacy
class of the character table <tbl>.


\beginexample
gap> tbl:= CharacterTable( "A5" );;
gap> OrdersClassRepresentatives( tbl );
[ 1, 2, 3, 5, 5 ]
gap> SizesCentralizers( tbl );
[ 60, 4, 3, 5, 5 ]
gap> SizesConjugacyClasses( tbl );
[ 1, 15, 20, 12, 12 ]
\endexample

The following attributes apply only to character tables, not to groups.



\>AutomorphismsOfTable( <tbl> ) A

is the permutation group of all column permutations of the character
table <tbl> that leave the set of irreducibles and each power map of
<tbl> invariant (see also~"TableAutomorphisms").


\beginexample
gap> tbl:= CharacterTable( "Dihedral", 8 );;
gap> AutomorphismsOfTable( tbl );
Group([ (4,5) ])
gap> OrdersClassRepresentatives( tbl );
[ 1, 4, 2, 2, 2 ]
gap> SizesConjugacyClasses( tbl );
[ 1, 2, 1, 2, 2 ]
\endexample

\>UnderlyingCharacteristic( <tbl> ) A
\>UnderlyingCharacteristic( <psi> ) A

For an ordinary character table <tbl>, the result is `0',
for a $p$-modular Brauer table <tbl>, it is $p$.
The underlying characteristic of a class function <psi> is equal to
that of its underlying character table.

The underlying characteristic must be stored when the table is
constructed, there is no method to compute it.

We cannot use the attribute `Characteristic' (see~"Characteristic")
to denote this, since of course each Brauer character is an element
of characteristic zero in the sense of {\GAP}
(see Chapter~"Class Functions").


\beginexample
gap> tbl:= CharacterTable( "A5" );;
gap> UnderlyingCharacteristic( tbl );
0
gap> UnderlyingCharacteristic( tbl mod 17 );
17
\endexample

\>ClassNames( <tbl> ) A
\>ClassNames( <tbl>, \"ATLAS\" ) O
\>CharacterNames( <tbl> ) A

`ClassNames' and `CharacterNames' return lists of strings,
one for each conjugacy class or irreducible character, respectively,
of the character table <tbl>.
These names are used when <tbl> is displayed.

The default method for `ClassNames' computes class names consisting of
the order of an element in the class and at least one distinguishing
letter.

The default method for `CharacterNames' returns the list
`[ "X.1", "X.2", ... ]', whose length is the number of
irreducible characters of <tbl>.

The position of the class with name <name> in <tbl> can be accessed as
`<tbl>.<name>'.

When `ClassNames' is called with two arguments, the second being the
string `\"ATLAS\"', the class names returned obey the convention used in
Chapter~7, Section~5 of the {\ATLAS} of Finite Groups~\cite{CCN85}.
If one is interested in ``relative'' class names of almost simple
{\ATLAS} groups, one can use the function `AtlasClassNames' of the {\GAP}
package AtlasRep.


\beginexample
gap> tbl:= CharacterTable( "A5" );;
gap> ClassNames( tbl );
[ "1a", "2a", "3a", "5a", "5b" ]
gap> tbl.2a;
2
\endexample

\>Identifier( <tbl> )!{for character tables} A

is a string that identifies the character table <tbl> in the current
{\GAP} session.
It is used mainly for class fusions into <tbl> that are stored on other
character tables.
For character tables without group,
the identifier is also used to print the table;
this is the case for library tables,
but also for tables that are constructed as direct products, factors
etc.~involving tables that may or may not store their groups.

The default method for ordinary tables constructs strings of the form
`\"CT<n>\"', where <n> is a positive integer.
`LARGEST_IDENTIFIER_NUMBER' is a list containing the largest integer <n>
used in the current {\GAP} session.

The default method for Brauer tables returns the concatenation of the
identifier of the ordinary table, the string `\"mod\"',
and the (string of the) underlying characteristic.


\beginexample
gap> Identifier( CharacterTable( "A5" ) );
"A5"
gap> tbl:= CharacterTable( Group( () ) );;
gap> Identifier( tbl );  Identifier( tbl mod 2 );
"CT8"
"CT8mod2"
\endexample

\>InfoText( <tbl> ) A

is a mutable string with information about the character table <tbl>.
There is no default method to create an info text.

This attribute is used mainly for library tables (see the manual of the
{\GAP} Character Table Library).
Usual parts of the information are the origin of the table,
tests it has passed (`1.o.r.' for the test of orthogonality,
`pow[<p>]' for the construction of the <p>-th power map,
`DEC' for the decomposition of ordinary into Brauer characters,
`TENS' for the decomposition of tensor products of irreducibles),
and choices made without loss of generality.


\beginexample
gap> Print( InfoText( CharacterTable( "A5" ) ), "\n" );
origin: ATLAS of finite groups, tests: 1.o.r., pow[2,3,5]
\endexample

\>InverseClasses( <tbl> ) A

For a character table <tbl>, `InverseClasses' returns the list mapping
each conjugacy class to its inverse class.
This list can be regarded as $(-1)$-st power map of <tbl>
(see~"PowerMap").


\>RealClasses( <tbl> ) A

\index{classes!real}

For a character table <tbl>, `RealClasses' returns the strictly sorted
list of positions of classes in <tbl> that consist of real elements.

An element $x$ is *real* iff it is conjugate to its inverse
$x^{-1} = x^{o(x)-1}$.


\beginexample
gap> InverseClasses( CharacterTable( "A5" ) );
[ 1, 2, 3, 4, 5 ]
gap> InverseClasses( CharacterTable( "Cyclic", 3 ) );
[ 1, 3, 2 ]
gap> RealClasses( CharacterTable( "A5" ) );
[ 1, 2, 3, 4, 5 ]
gap> RealClasses( CharacterTable( "Cyclic", 3 ) );
[ 1 ]
\endexample

\>ClassOrbit( <tbl>, <cc> ) O

is the list of positions of those conjugacy classes
of the character table <tbl> that are Galois conjugate to the <cc>-th
class.
That is, exactly the classes at positions given by the list returned by
`ClassOrbit' contain generators of the cyclic group generated
by an element in the <cc>-th class.

This information is computed from the power maps of <tbl>.


\>ClassRoots( <tbl> ) A

For a character table <tbl>, `ClassRoots' returns a list
containing at position $i$ the list of positions of the classes
of all nontrivial $p$-th roots, where $p$ runs over the prime divisors
of `Size( <tbl> )'.

This information is computed from the power maps of <tbl>.


\beginexample
gap> ClassOrbit( CharacterTable( "A5" ), 4 );
[ 4, 5 ]
gap> ClassRoots( CharacterTable( "A5" ) );
[ [ 2, 3, 4, 5 ], [  ], [  ], [  ], [  ] ]
gap> ClassRoots( CharacterTable( "Cyclic", 6 ) );
[ [ 3, 4, 5 ], [  ], [ 2 ], [ 2, 6 ], [ 6 ], [  ] ]
\endexample

The following attributes for a character table <tbl> correspond to
attributes for the group $G$ of <tbl>.
But instead of a normal subgroup (or a list of normal subgroups) of $G$,
they return a strictly sorted list of positive integers (or a list of
such lists) which are the positions
--relative to `ConjugacyClasses( <tbl> )'--
of those classes forming the normal subgroup in question.



\>ClassPositionsOfNormalSubgroups( <ordtbl> ) A
\>ClassPositionsOfMaximalNormalSubgroups( <ordtbl> ) A
\>ClassPositionsOfMinimalNormalSubgroups( <ordtbl> ) A

correspond to `NormalSubgroups', `MaximalNormalSubgroups', and
`MinimalNormalSubgroups'
for the group of the ordinary character table <ordtbl>
(see~"NormalSubgroups", "MaximalNormalSubgroups",
"MinimalNormalSubgroups").

The entries of the result lists are sorted according to increasing
length.
(So this total order respects the partial order of normal subgroups
given by inclusion.)


\>ClassPositionsOfAgemo( <ordtbl>, <p> ) O

corresponds to `Agemo' (see~"Agemo")
for the group of the ordinary character table <ordtbl>.


\>ClassPositionsOfCentre( <ordtbl> )!{for character tables} A

corresponds to `Centre' (see~"Centre")
for the group of the ordinary character table <ordtbl>.


\>ClassPositionsOfDirectProductDecompositions( <tbl> ) A
\>ClassPositionsOfDirectProductDecompositions( <tbl>, <nclasses> ) O

Let <tbl> be the ordinary character table of the group $G$, say.
Called with the only argument <tbl>,
`ClassPositionsOfDirectProductDecompositions' returns the list of all
those pairs $[ l_1, l_2 ]$ where $l_1$ and $l_2$ are lists of
class positions of normal subgroups $N_1$, $N_2$ of $G$
such that $G$ is their direct product and $|N_1| \leq |N_2|$ holds.
Called with second argument a list <nclasses> of class positions of a
normal subgroup $N$ of $G$,
`ClassPositionsOfDirectProductDecompositions' returns the list of pairs
describing the decomposition of $N$ as a direct product of two
normal subgroups of $G$.


\>ClassPositionsOfDerivedSubgroup( <ordtbl> ) A

corresponds to `DerivedSubgroup' (see~"DerivedSubgroup")
for the group of the ordinary character table <ordtbl>.


\>ClassPositionsOfElementaryAbelianSeries( <ordtbl> ) A

corresponds to `ElementaryAbelianSeries' (see~"ElementaryAbelianSeries")
for the group of the ordinary character table <ordtbl>.


\>ClassPositionsOfFittingSubgroup( <ordtbl> ) A

corresponds to `FittingSubgroup' (see~"FittingSubgroup")
for the group of the ordinary character table <ordtbl>.


\>ClassPositionsOfLowerCentralSeries( <tbl> ) A

corresponds to `LowerCentralSeries' (see~"LowerCentralSeriesOfGroup")
for the group of the ordinary character table <ordtbl>.


\>ClassPositionsOfUpperCentralSeries( <ordtbl> ) A

corresponds to `UpperCentralSeries' (see~"UpperCentralSeriesOfGroup")
for the group of the ordinary character table <ordtbl>.


\>ClassPositionsOfSupersolvableResiduum( <ordtbl> ) A

corresponds to `SupersolvableResiduum' (see~"SupersolvableResiduum")
for the group of the ordinary character table <ordtbl>.


\>ClassPositionsOfNormalClosure( <ordtbl>, <classes> ) O

is the sorted list of the positions of all conjugacy classes of the
ordinary character table <ordtbl> that form the normal closure
(see~"NormalClosure") of the conjugacy classes at positions in the
list <classes>.


\beginexample
gap> tbla5:= CharacterTable( "A5" );;
gap> tbls4:= CharacterTable( "Symmetric", 4 );;
gap> tbld8:= CharacterTable( "Dihedral", 8 );;
gap> ClassPositionsOfNormalSubgroups( tbls4 );
[ [ 1 ], [ 1, 3 ], [ 1, 3, 4 ], [ 1 .. 5 ] ]
gap> ClassPositionsOfAgemo( tbls4, 2 );
[ 1, 3, 4 ]
gap> ClassPositionsOfCentre( tbld8 );
[ 1, 3 ]
gap> ClassPositionsOfDerivedSubgroup( tbld8 );
[ 1, 3 ]
gap> ClassPositionsOfElementaryAbelianSeries( tbls4 );
[ [ 1 .. 5 ], [ 1, 3, 4 ], [ 1, 3 ], [ 1 ] ]
gap> ClassPositionsOfElementaryAbelianSeries( tbla5 );
fail
gap> ClassPositionsOfFittingSubgroup( tbls4 );
[ 1, 3 ]
gap> ClassPositionsOfLowerCentralSeries( tbls4 );
[ [ 1 .. 5 ], [ 1, 3, 4 ] ]
gap> ClassPositionsOfLowerCentralSeries( tbld8 );
[ [ 1 .. 5 ], [ 1, 3 ], [ 1 ] ]
gap> ClassPositionsOfUpperCentralSeries( tbls4 );
[ [ 1 ] ]
gap> ClassPositionsOfUpperCentralSeries( tbld8 );
[ [ 1, 3 ], [ 1, 2, 3, 4, 5 ] ]
gap> ClassPositionsOfSupersolvableResiduum( tbls4 );
[ 1, 3 ]
gap> ClassPositionsOfNormalClosure( tbls4, [ 1, 4 ] );
[ 1, 3, 4 ]
\endexample


%%%%%%%%%%%%%%%%%%%%%%%%%%%%%%%%%%%%%%%%%%%%%%%%%%%%%%%%%%%%%%%%%%%%%%%%%%%%%
\Section{Operations Concerning Blocks}

\>PrimeBlocks( <ordtbl>, <p> ) O
\>PrimeBlocksOp( <ordtbl>, <p> ) O
\>ComputedPrimeBlockss( <tbl> ) A

For an ordinary character table <ordtbl> and a prime integer <p>,
`PrimeBlocks' returns a record with the following components.
\beginitems
`block' &
    a list, the value $j$ at position $i$ means that the $i$-th
    irreducible character of <ordtbl> lies in the $j$-th <p>-block
    of <ordtbl>,

`defect' &
    a list containing at position $i$ the defect of the $i$-th block,

`height' &
    a list containing at position $i$ the height of the $i$-th
    irreducible character of <ordtbl> in its block,

`relevant' &
    a list of class positions such that only the restriction to these
    classes need be checked for deciding whether two characters lie
    in the same block, and

`centralcharacter' &
    a list containing at position $i$ a list whose values at the
    positions stored in the component `relevant' are the values of
    a central character in the $i$-th block.
\enditems

The components `relevant' and `centralcharacters' are
used by `SameBlock' (see~"SameBlock").

If `InfoCharacterTable' has level at least 2,
the defects of the blocks and the heights of the characters are printed.

The default method uses the attribute
`ComputedPrimeBlockss' for storing the computed value at
position <p>, and calls the operation `PrimeBlocksOp' for
computing values that are not yet known.

Two ordinary irreducible characters $\chi, \psi$ of a group $G$ are said
to lie in the same $p$-*block* if the images of their central characters
$\omega_{\chi}, \omega_{\psi}$ (see~"CentralCharacter") under the
ring homomorphism $\ast \colon R \rightarrow R / M$ are equal,
where $R$ denotes the ring of algebraic integers in the complex number
field, and $M$ is a maximal ideal in $R$ with $pR \subseteq M$.
(The distribution to $p$-blocks is in fact independent of the choice of
$M$, see~\cite{Isa76}.)

For $|G| = p^a m$ where $p$ does not divide $m$, the *defect* of a block
is the integer $d$ such that $p^{a-d}$ is the largest power of $p$ that
divides the degrees of all characters in the block.

The *height* of a character $\chi$ in the block is defined as the largest
exponent $h$ for which $p^h$ divides $\chi(1) / p^{a-d}$.


\beginexample
gap> tbl:= CharacterTable( "L3(2)" );;
gap> pbl:= PrimeBlocks( tbl, 2 );
rec( block := [ 1, 1, 1, 1, 1, 2 ], defect := [ 3, 0 ], 
  height := [ 0, 0, 0, 1, 0, 0 ], relevant := [ 3, 5 ], 
  centralcharacter := [ [ ,, 56,, 24 ], [ ,, -7,, 3 ] ] )
\endexample

\>SameBlock( <p>, <omega1>, <omega2>, <relevant> ) F

Let <p> be a prime integer, <omega1> and <omega2> be two central
characters (or their values lists) of a character table,
and <relevant> be a list of positions as is stored in the component
`relevant' of a record returned by `PrimeBlocks' (see~"PrimeBlocks").

`SameBlock' returns `true' if <omega1> and <omega2> are equal modulo any
maximal ideal in the ring of complex algebraic integers containing the
ideal spanned by <p>, and `false' otherwise.


\beginexample
gap> omega:= List( Irr( tbl ), CentralCharacter );;
gap> SameBlock( 2, omega[1], omega[2], pbl.relevant );
true
gap> SameBlock( 2, omega[1], omega[6], pbl.relevant );
false
\endexample

\>BlocksInfo( <modtbl> ) A

For a Brauer character table <modtbl>, the value of `BlocksInfo'
is a list of (mutable) records, the $i$-th entry containing information
about the $i$-th block.
Each record has the following components.
\beginitems
`defect' &
     the defect of the block,

`ordchars' &
     the list of positions of the ordinary characters that belong to the
     block, relative to `Irr( OrdinaryCharacterTable( <modtbl> ) )',

`modchars' &
     the list of positions of the Brauer characters that belong to the
     block, relative to `IBr( <modtbl> )'.
\enditems
Optional components are
\beginitems
`basicset' &
     a list of positions of ordinary characters in the block whose
     restriction to <modtbl> is maximally linearly independent,
     relative to `Irr( OrdinaryCharacterTable( <modtbl> ) )',

`decmat' &
     the decomposition matrix of the block,
     it is stored automatically when `DecompositionMatrix' is called for
     the block (see~"DecompositionMatrix"),

`decinv' &
     inverse of the decomposition matrix of the block, restricted to the
     ordinary characters described by `basicset',

`brauertree' &
     a list that describes the Brauer tree of the block,
     in the case that the block is of defect $1$.
\enditems


\beginexample
gap> BlocksInfo( CharacterTable( "L3(2)" ) mod 2 );
[ rec( defect := 3, ordchars := [ 1, 2, 3, 4, 5 ], modchars := [ 1, 2, 3 ], 
      decinv := [ [ 1, 0, 0 ], [ 0, 1, 0 ], [ 0, 0, 1 ] ], 
      basicset := [ 1, 2, 3 ] ), 
  rec( defect := 0, ordchars := [ 6 ], modchars := [ 4 ], decinv := [ [ 1 ] ],
      basicset := [ 6 ] ) ]
\endexample

\>DecompositionMatrix( <modtbl> ) A
\>DecompositionMatrix( <modtbl>, <blocknr> ) O

Let <modtbl> be a Brauer character table.

In the first version `DecompositionMatrix' returns the decomposition
matrix of <modtbl>, where the rows and columns are indexed by the
irreducible characters of the ordinary character table of <modtbl>
and the irreducible characters of <modtbl>, respectively,

In the second version `DecompositionMatrix' returns the decomposition
matrix of the block of <modtbl> with number <blocknr>;
the matrix is stored as value of the `decmat' component of the
<blocknr>-th entry of the `BlocksInfo' list (see~"BlocksInfo") of
<modtbl>.

An ordinary irreducible character is in block $i$ if and only if all
characters before the first character of the same block lie in $i-1$
different blocks.
An irreducible Brauer character is in block $i$ if it has nonzero scalar
product with an ordinary irreducible character in block $i$.

`DecompositionMatrix' is based on the more general function
`Decomposition' (see~"Decomposition").


\beginexample
gap> modtbl:= CharacterTable( "L3(2)" ) mod 2;
BrauerTable( "L3(2)", 2 )
gap> DecompositionMatrix( modtbl );
[ [ 1, 0, 0, 0 ], [ 0, 1, 0, 0 ], [ 0, 0, 1, 0 ], [ 0, 1, 1, 0 ], 
  [ 1, 1, 1, 0 ], [ 0, 0, 0, 1 ] ]
gap> DecompositionMatrix( modtbl, 1 );
[ [ 1, 0, 0 ], [ 0, 1, 0 ], [ 0, 0, 1 ], [ 0, 1, 1 ], [ 1, 1, 1 ] ]
gap> DecompositionMatrix( modtbl, 2 );
[ [ 1 ] ]
\endexample

\>LaTeXStringDecompositionMatrix( <modtbl>[, <blocknr>][, <options>] ) F

is a string that contains La{\TeX} code to print a decomposition matrix
(see~"DecompositionMatrix") nicely.

The optional argument <options>, if present, must be a record with
components
`phi', `chi' (strings used in each label for columns and rows),
`collabels', `rowlabels' (subscripts for the labels).
The defaults for `phi' and `chi' are `\"{\\tt Y}\"' and `\"{\\tt X}\"',
the defaults for `collabels' and `rowlabels' are the lists of positions
of the Brauer characters and ordinary characters in the respective lists
of irreducibles in the character tables.

The optional components `nrows' and `ncols' denote the maximal number of
rows and columns per array;
if they are present then each portion of `nrows' rows and `ncols' columns
forms an array of its own which is enclosed in `\\[', `\\]'.

If the component `decmat' is bound in <options> then it must be the
decomposition matrix in question, in this case the matrix is not computed
from the information in <modtbl>.

For those character tables from the {\GAP} table library that belong to
the {\ATLAS} of Finite Groups~\cite{CCN85},
`AtlasLabelsOfIrreducibles' constructs character labels that are
compatible with those used in the {\ATLAS}
(see~"ctbllib:ATLAS Tables" and ~"ctbllib:AtlasLabelsOfIrreducibles"
in the manual of the {\GAP} Character Table Library).


\beginexample
gap> modtbl:= CharacterTable( "L3(2)" ) mod 2;;
gap> Print( LaTeXStringDecompositionMatrix( modtbl, 1 ) );
\[
\begin{array}{r||rrr} \hline
 & {\tt Y}_{1}
 & {\tt Y}_{2}
 & {\tt Y}_{3}
 \rule[-7pt]{0pt}{20pt} \\ \hline
{\tt X}_{1} & 1 & . & . \rule[0pt]{0pt}{13pt} \\
{\tt X}_{2} & . & 1 & . \\
{\tt X}_{3} & . & . & 1 \\
{\tt X}_{4} & . & 1 & 1 \\
{\tt X}_{5} & 1 & 1 & 1 \rule[-7pt]{0pt}{5pt} \\
\hline
\end{array}
\]
gap> options:= rec( phi:= "\\varphi", chi:= "\\chi" );;
gap> Print( LaTeXStringDecompositionMatrix( modtbl, 1, options ) );
\[
\begin{array}{r||rrr} \hline
 & \varphi_{1}
 & \varphi_{2}
 & \varphi_{3}
 \rule[-7pt]{0pt}{20pt} \\ \hline
\chi_{1} & 1 & . & . \rule[0pt]{0pt}{13pt} \\
\chi_{2} & . & 1 & . \\
\chi_{3} & . & . & 1 \\
\chi_{4} & . & 1 & 1 \\
\chi_{5} & 1 & 1 & 1 \rule[-7pt]{0pt}{5pt} \\
\hline
\end{array}
\]
\endexample


%%%%%%%%%%%%%%%%%%%%%%%%%%%%%%%%%%%%%%%%%%%%%%%%%%%%%%%%%%%%%%%%%%%%%%%%%%%%%
\Section{Other Operations for Character Tables}

In the following, we list operations for character tables that are not
attributes.

\>IsInternallyConsistent( <tbl> )!{for character tables} O

For an *ordinary* character table <tbl>, `IsInternallyConsistent'
checks the consistency of the following attribute values (if stored).
\beginlist%unordered
\item{--}
    `Size', `SizesCentralizers', and `SizesConjugacyClasses'.
\item{--}
    `SizesCentralizers' and `OrdersClassRepresentatives'.
\item{--}
    `ComputedPowerMaps' and `OrdersClassRepresentatives'.
\item{--}
    `SizesCentralizers' and `Irr'.
\item{--}
    `Irr' (first orthogonality relation).
\endlist

For a *Brauer* table <tbl>, `IsInternallyConsistent'
checks the consistency of the following attribute values (if stored).
\beginlist%unordered
\item{--}
    `Size', `SizesCentralizers', and `SizesConjugacyClasses'.
\item{--}
    `SizesCentralizers' and `OrdersClassRepresentatives'.
\item{--}
    `ComputedPowerMaps' and `OrdersClassRepresentatives'.
\item{--}
    `Irr' (closure under complex conjugation and Frobenius map).
\endlist

If no inconsistency occurs, `true' is returned,
otherwise each inconsistency is printed to the screen if the level of
`InfoWarning' is at least $1$ (see~"Info Functions"),
and `false' is returned at the end.



\>IsPSolvableCharacterTable( <tbl>, <p> ) O
\>IsPSolvableCharacterTableOp( <tbl>, <p> ) O
\>ComputedIsPSolvableCharacterTables( <tbl> ) A

`IsPSolvableCharacterTable' for the ordinary character table <tbl>
corresponds to `IsPSolvable' for the group of <tbl> (see~"IsPSolvable").
<p> must be either a prime integer or `0'.

The default method uses the attribute
`ComputedIsPSolvableCharacterTables' for storing the computed value at
position <p>, and calls the operation `IsPSolvableCharacterTableOp' for
computing values that are not yet known.


\beginexample
gap> tbl:= CharacterTable( "Sz(8)" );;
gap> IsPSolvableCharacterTable( tbl, 2 );
false
gap> IsPSolvableCharacterTable( tbl, 3 );
true
\endexample

\>IsClassFusionOfNormalSubgroup( <subtbl>, <fus>, <tbl> ) F

For two ordinary character tables <tbl> and <subtbl> of a group $G$ and
its subgroup $U$, say,
and a list <fus> of positive integers that describes the class fusion of
$U$ into $G$,
`IsClassFusionOfNormalSubgroup' returns `true'
if $U$ is a normal subgroup of $G$, and `false' otherwise.


\beginexample
gap> tblc2:= CharacterTable( "Cyclic", 2 );;
gap> tbld8:= CharacterTable( "Dihedral", 8 );;
gap> fus:= PossibleClassFusions( tblc2, tbld8 );
[ [ 1, 3 ], [ 1, 4 ], [ 1, 5 ] ]
gap> List( fus, map -> IsClassFusionOfNormalSubgroup( tblc2, map, tbld8 ) );
[ true, false, false ]
\endexample

\>Indicator( <tbl>, <n> ) O
\>Indicator( <tbl>[, <characters>], <n> ) O
\>Indicator( <modtbl>, 2 ) O
\>IndicatorOp( <tbl>, <characters>, <n> ) O
\>ComputedIndicators( <tbl> ) A

If <tbl> is an ordinary character table then `Indicator' returns the
list of <n>-th Frobenius-Schur indicators of the characters in the list
<characters>; the default of <characters> is `Irr( <tbl> )'.

The $n$-th Frobenius-Schur indicator $\nu_n(\chi)$ of an ordinary
character $\chi$ of the group $G$ is given by
$\nu_n(\chi) = \frac{1}{|G|} \sum_{g \in G} \chi(g^n)$.

If <tbl> is a Brauer table in characteristic $\not= 2$ and $<n> = 2$
then `Indicator' returns the second indicator.

The default method uses the attribute
`ComputedIndicators' for storing the computed value at
position <n>, and calls the operation `IndicatorOp' for
computing values that are not yet known.


\beginexample
gap> tbl:= CharacterTable( "L3(2)" );;
gap> Indicator( tbl, 2 );
[ 1, 0, 0, 1, 1, 1 ]
\endexample

\>NrPolyhedralSubgroups( <tbl>, <c1>, <c2>, <c3> ) F

\index{subgroups!polyhedral}

returns the number and isomorphism type of polyhedral subgroups of the
group with ordinary character table <tbl> which are generated by an
element $g$ of class <c1> and an element $h$ of class <c2> with the
property that the product $gh$ lies in class <c3>.

According to p.~233 in~\cite{NPP84}, the number of polyhedral subgroups
of isomorphism type $V_4$, $D_{2n}$, $A_4$, $S_4$, and $A_5$
can be derived from the class multiplication coefficient
(see~"ClassMultiplicationCoefficient!for character tables")
and the number of Galois
conjugates of a class (see~"ClassOrbit").

The classes <c1>, <c2> and <c3> in the parameter list must be ordered
according to the order of the elements in these classes.


\beginexample
gap> NrPolyhedralSubgroups( tbl, 2, 2, 4 );
rec( number := 21, type := "D8" )
\endexample

\>ClassMultiplicationCoefficient( <tbl>, <i>, <j>, <k> )!{for character tables} O

\index{class multiplication coefficient}
\index{structure constant}

returns the class multiplication coefficient of the classes <i>, <j>,
and <k> of the group $G$ with ordinary character table <tbl>.

The class multiplication coefficient $c_{i,j,k}$ of the classes <i>,
<j>, <k> equals the number of pairs $(x,y)$ of elements $x, y \in G$
such that $x$ lies in class <i>, $y$ lies in class <j>,
and their product $xy$ is a fixed element of class <k>.

In the center of the group algebra of $G$, these numbers are found as
coefficients of the decomposition of the product of two class sums $K_i$
and $K_j$ into class sums,
$$
K_i K_j = \sum_k c_{ijk} K_k\.
$$
Given the character table of a finite group $G$,
whose classes  are $C_1, \dots, C_r$ with representatives $g_i \in C_i$,
the class multiplication coefficient $c_{ijk}$ can be computed
by the following formula.
$$
c_{ijk} = \frac{\|C_i\|\|C_j\|}{\|G\|}
          \sum_{\chi \in Irr(G)}
          \frac{\chi(g_i) \chi(g_j) \overline{\chi(g_k)}}{\chi(1)}\.
$$
On the other hand the knowledge of the class multiplication coefficients
admits the computation of the irreducible characters of $G$.
(see~"IrrDixonSchneider").


\>ClassStructureCharTable( <tbl>, <classes> ) F

\index{class multiplication coefficient}
\index{structure constant}

returns the so-called class structure of the classes in the list
<classes>, for the character table <tbl> of the group $G$.
The length of <classes> must be at least 2.

Let $C = (C_1, C_2, \dots, C_n)$ denote the $n$-tuple of conjugacy
classes of $G$ that are indexed by <classes>.
The class structure $n(C)$ equals
the number of $n$-tuples $(g_1, g_2, \ldots, g_n)$ of elements
$g_i\in C_i$ with $g_1 g_2 \cdots g_n = 1$.
Note the difference to the definition of the class multiplication
coefficients in `ClassMultiplicationCoefficient'
(see~"ClassMultiplicationCoefficient!for character tables").

$n(C_1, C_2, \ldots, C_n)$ is computed using the formula
$$
n(C_1, C_2, \ldots, C_n) 
       = \frac{\|C_1\|\|C_2\|\cdots\|C_n\|}{\|G\|}
         \sum_{\chi \in Irr(G)}
         \frac{\chi(g_1)\chi(g_2)\cdots\chi(g_n)}{\chi(1)^{n-2}}.
$$


\>MatClassMultCoeffsCharTable( <tbl>, <i> ) F

\index{structure constant}
\index{class multiplication coefficient}

For an ordinary character table <tbl> and a class position <i>,
`MatClassMultCoeffsCharTable' returns the matrix
$[ a_{ijk} ]_{j,k}$ of structure constants
(see~"ClassMultiplicationCoefficient!for character tables").


\beginexample
gap> tbl:= CharacterTable( "L3(2)" );;
gap> ClassMultiplicationCoefficient( tbl, 2, 2, 4 );
4
gap> ClassStructureCharTable( tbl, [ 2, 2, 4 ] );
168
gap> ClassStructureCharTable( tbl, [ 2, 2, 2, 4 ] );
1848
gap> MatClassMultCoeffsCharTable( tbl, 2 );
[ [ 0, 1, 0, 0, 0, 0 ], [ 21, 4, 3, 4, 0, 0 ], [ 0, 8, 6, 8, 7, 7 ], 
  [ 0, 8, 6, 1, 7, 7 ], [ 0, 0, 3, 4, 0, 7 ], [ 0, 0, 3, 4, 7, 0 ] ]
\endexample


%%%%%%%%%%%%%%%%%%%%%%%%%%%%%%%%%%%%%%%%%%%%%%%%%%%%%%%%%%%%%%%%%%%%%%%%%%%%%
\Section{Printing Character Tables}

\indextt{ViewObj!for character tables}
The default  `ViewObj'  (see~"ViewObj")  method  for  ordinary  character
tables prints the string `\"CharacterTable\"', followed by the identifier
(see~"Identifier!for character tables") or, if known, the  group  of  the
character table enclosed in brackets. `ViewObj' for  Brauer  tables  does
the same, except that the first string is replaced by  `\"BrauerTable\"',
and that the characteristic is also shown.

\indextt{PrintObj!for character tables}
The default `PrintObj' (see~"PrintObj") method for character tables
does the same as `ViewObj',
except that the group is is `Print'-ed instead of `View'-ed.

\indextt{Display!for character tables}
There are various ways to customize the `Display' (see~"Display") output
for character tables.
First we describe the default behaviour,
alternatives are then described below.

The default `Display' method prepares the data in <tbl> for a columnwise
output.
The number of columns printed at one time depends on the actual
line length, which can be accessed and changed by the function
`SizeScreen' (see~"SizeScreen").

`Display' shows certain characters (by default all irreducible
characters) of <tbl>, together with the orders of the centralizers in
factorized form and the available power maps (see~"ComputedPowerMaps").
The <n>-th displayed character is given the name `X.<n>'.

The first lines of the output describe the order of the centralizer
of an element of the class factorized into its prime divisors.

The next line gives the name of each class.
If no class names are stored on <tbl>, `ClassNames' is called
(see~"ClassNames").

Preceded by a name `P<n>', the next lines show the <n>th power maps
of <tbl> in terms of the former shown class names.

Every ambiguous or unknown (see Chapter~"Unknowns") value of the table
is displayed as a question mark `?'.

Irrational character values are not printed explicitly because the
lengths of their printed representation might disturb the layout.
Instead of that every irrational value is indicated by a name,
which is a string of at least one capital letter.

Once a name for an irrational value is found, it is used all over the
printed table.
Moreover the complex conjugate (see~"ComplexConjugate", "GaloisCyc")
and the star of an irrationality (see~"StarCyc") are represented by
that very name preceded by a `/' and a `\*', respectively.

The printed character table is then followed by a legend,
a list identifying the occurring symbols with their actual values.
Occasionally this identification is supplemented by a quadratic
representation of the irrationality (see~"Quadratic") together with
the corresponding {\ATLAS} notation (see~\cite{CCN85}).

This default style can be changed by prescribing a record <arec> of
options, which can be given
\beginlist%unordered
\item{--}
    as an optional argument in the call to `Display',
\item{--}
    as the value of the attribute `DisplayOptions' (see~"DisplayOptions")
    if this value is stored in the table,
\item{--}
    as the value of the global variable
    `CharacterTableDisplayDefaults.User', or
\item{--}
    as the value of the global variable
    `CharacterTableDisplayDefaults.Global'
\endlist
(in this order of precedence).

The following components of <arec> are supported.

\beginitems
`centralizers' &
    `false' to suppress the printing of the orders of the centralizers,
    or the string `\"ATLAS\"' to force the printing of non-factorized
    centralizer orders in a style similar to that used in the
    {\ATLAS} of Finite Groups~\cite{CCN85},

`chars' &
    an integer or a list of integers to select a sublist of the
    irreducible characters of <tbl>,
    or a list of characters of <tbl>
    (in this case the letter `\"X\"' is replaced by `\"Y\"'),

`classes' &
    an integer or a list of integers to select a sublist of the
    classes of <tbl>,

`indicator' &
    `true' enables the printing of the second Frobenius Schur indicator,
    a list of integers enables the printing of the corresponding
    indicators (see~"Indicator"),

`letter' &
    a single capital letter (e.~g.~`\"P\"' for permutation characters)
    to replace the default `\"X\"' in character names,

`powermap' &
    an integer or a list of integers to select a subset of the
    available power maps,
    `false' to suppress the printing of power maps,
    or the string `\"ATLAS\"' to force a printing of class names and
    power maps in a style similar to that used in the
    {\ATLAS} of Finite Groups~\cite{CCN85},

`Display' &
    the function that is actually called in order to display the table;
    the arguments are the table and the optional record, whose components
    can be used inside the `Display' function,

`StringEntry' &
    a function that takes either a character value or a character value
    and the return value of `StringEntryData' (see below),
    and returns the string that is actually displayed;
    it is called for all character values to be displayed,
    and also for the displayed indicator values (see above),

`StringEntryData' &
    a unary function that is called once with argument <tbl> before the
    character values are displayed;
    it returns an object that is used as second argument of the function
    `StringEntry',

`Legend' &
    a function that takes the result of the `StringEntryData' call as its
    only argument, after the character table has been displayed;
    the return value is a string that describes the symbols used in the
    displayed table in a formatted way,
    it is printed below the displayed table.
\enditems



\>DisplayOptions( <tbl> ) A

There is no default method to compute a value,
one can set a value with `SetDisplayOptions'.


\beginexample
gap> tbl:= CharacterTable( "A5" );;
gap> Display( tbl );
A5

     2  2  2  .  .  .
     3  1  .  1  .  .
     5  1  .  .  1  1

       1a 2a 3a 5a 5b
    2P 1a 1a 3a 5b 5a
    3P 1a 2a 1a 5b 5a
    5P 1a 2a 3a 1a 1a

X.1     1  1  1  1  1
X.2     3 -1  .  A *A
X.3     3 -1  . *A  A
X.4     4  .  1 -1 -1
X.5     5  1 -1  .  .

A = -E(5)-E(5)^4
  = (1-ER(5))/2 = -b5
gap> CharacterTableDisplayDefaults.User:= rec(
>        powermap:= "ATLAS", centralizers:= "ATLAS", chars:= false );;
gap> Display( CharacterTable( "A5" ) );
A5

    60  4  3  5  5

 p      A  A  A  A
 p'     A  A  A  A
    1A 2A 3A 5A B*

gap> options:= rec( chars:= 4, classes:= [ tbl.3a .. tbl.5a ],
>                   centralizers:= false, indicator:= true,
>                   powermap:= [ 2 ] );;
gap> Display( tbl, options );
A5

          3a 5a
       2P 3a 5b
       2
X.4    +   1 -1
gap> SetDisplayOptions( tbl, options );  Display( tbl );
A5

          3a 5a
       2P 3a 5b
       2
X.4    +   1 -1
gap> Unbind( CharacterTableDisplayDefaults.User );
\endexample

\>PrintCharacterTable( <tbl>, <varname> ) F

Let <tbl> be a nearly character table, and <varname> a string.
`PrintCharacterTable' prints those values of the supported attributes
(see~"SupportedCharacterTableInfo") that are known for <tbl>;

The output of `PrintCharacterTable' is {\GAP} readable;
actually reading it into {\GAP} will bind the variable with name
<varname> to a character table that coincides with <tbl> for all
printed components.

This is used mainly for saving character tables to files.
A more human readable form is produced by `Display'.


\beginexample
gap> PrintCharacterTable( CharacterTable( "Cyclic", 2 ), "tbl" );
tbl:= function()
local tbl;
tbl:=rec();
tbl.Irr:=
[ [ 1, 1 ], [ 1, -1 ] ];
tbl.NrConjugacyClasses:=
2;
tbl.Size:=
2;
tbl.OrdersClassRepresentatives:=
[ 1, 2 ];
tbl.SizesCentralizers:=
[ 2, 2 ];
tbl.UnderlyingCharacteristic:=
0;
tbl.ClassParameters:=
[ [ 1, 0 ], [ 1, 1 ] ];
tbl.CharacterParameters:=
[ [ 1, 0 ], [ 1, 1 ] ];
tbl.Identifier:=
"C2";
tbl.InfoText:=
"computed using generic character table for cyclic groups";
tbl.ComputedPowerMaps:=
[ , [ 1, 1 ] ];
ConvertToLibraryCharacterTableNC(tbl);
return tbl;
end;
tbl:= tbl();
\endexample


%%%%%%%%%%%%%%%%%%%%%%%%%%%%%%%%%%%%%%%%%%%%%%%%%%%%%%%%%%%%%%%%%%%%%%%%%%%%%
\Section{Computing the Irreducible Characters of a Group}

Several algorithms are available for computing the irreducible characters of
a finite group $G$.
The default method for arbitrary finite groups is to use the Dixon-Schneider
algorithm (see~"IrrDixonSchneider").
For supersolvable groups, Conlon's algorithm can be used (see~"IrrConlon").
For abelian-by-supersolvable groups, the Baum-Clausen algorithm for computing
the irreducible representations (see~"IrreducibleRepresentations")
can be used to compute the irreducible characters (see~"IrrBaumClausen").

These functions are installed in methods for `Irr' (see~"Irr"),
but explicitly calling one of them will *not* set the `Irr' value of $G$.

\>IrrDixonSchneider( <G> ) A

computes the irreducible characters of the finite group <G>,
using the Dixon-Schneider method (see~"The Dixon-Schneider Algorithm").
It calls `DixonInit' and `DixonSplit',
and finally returns the list returned by `DixontinI'
(see~"Advanced Methods for Dixon-Schneider Calculations",
"Components of a Dixon Record",
"An Example of Advanced Dixon-Schneider Calculations").


\>IrrConlon( <G> ) A

For a finite solvable group <G>, `IrrConlon' returns a list of certain
irreducible characters of <G>, among those all irreducibles that have the
supersolvable residuum of <G> in their kernels;
so if <G> is supersolvable, all irreducible characters of <G> are
returned.
An error is signalled if <G> is not solvable.

The characters are computed using Conlon's algorithm
(see~\cite{Con90a} and~\cite{Con90b}).
For each irreducible character in the returned list,
the monomiality information (see~"TestMonomial") is stored.


\>IrrBaumClausen( <G> ) A

`IrrBaumClausen' returns the absolutely irreducible ordinary characters
of the factor group of the finite solvable group <G>
by the derived subgroup of its supersolvable residuum.

The characters are computed using the algorithm by Baum and Clausen
(see~\cite{BC94}).
An error is signalled if <G> is not solvable.



In the following example we temporarily increase the line length limit from
its default value 80 to 87 in order to get a nicer output format.

\beginexample
gap> g:= SL(2,3);;
gap> SizeScreen([ 87, ]);;
gap> irr1:= IrrDixonSchneider( g );
[ Character( CharacterTable( SL(2,3) ), [ 1, 1, 1, 1, 1, 1, 1 ] ), 
  Character( CharacterTable( SL(2,3) ), [ 1, E(3)^2, E(3), 1, E(3), E(3)^2, 1 ] ), 
  Character( CharacterTable( SL(2,3) ), [ 1, E(3), E(3)^2, 1, E(3)^2, E(3), 1 ] ), 
  Character( CharacterTable( SL(2,3) ), [ 2, 1, 1, -2, -1, -1, 0 ] ), 
  Character( CharacterTable( SL(2,3) ), [ 2, E(3)^2, E(3), -2, -E(3), -E(3)^2, 0 ] ),
  Character( CharacterTable( SL(2,3) ), [ 2, E(3), E(3)^2, -2, -E(3)^2, -E(3), 0 ] ),
  Character( CharacterTable( SL(2,3) ), [ 3, 0, 0, 3, 0, 0, -1 ] ) ]
gap> irr2:= IrrConlon( g );
[ Character( CharacterTable( SL(2,3) ), [ 1, 1, 1, 1, 1, 1, 1 ] ), 
  Character( CharacterTable( SL(2,3) ), [ 1, E(3), E(3)^2, 1, E(3)^2, E(3), 1 ] ), 
  Character( CharacterTable( SL(2,3) ), [ 1, E(3)^2, E(3), 1, E(3), E(3)^2, 1 ] ), 
  Character( CharacterTable( SL(2,3) ), [ 3, 0, 0, 3, 0, 0, -1 ] ) ]
gap> irr3:= IrrBaumClausen( g );
[ Character( CharacterTable( SL(2,3) ), [ 1, 1, 1, 1, 1, 1, 1 ] ), 
  Character( CharacterTable( SL(2,3) ), [ 1, E(3), E(3)^2, 1, E(3)^2, E(3), 1 ] ), 
  Character( CharacterTable( SL(2,3) ), [ 1, E(3)^2, E(3), 1, E(3), E(3)^2, 1 ] ), 
  Character( CharacterTable( SL(2,3) ), [ 3, 0, 0, 3, 0, 0, -1 ] ) ]
gap> SizeScreen([ 80, ]);;
gap> chi:= irr2[4];;  HasTestMonomial( chi );
true
\endexample

\>IrreducibleRepresentations( <G> ) A
\>IrreducibleRepresentations( <G>, <F> ) O

Called with a finite group <G> and a field <F>,
`IrreducibleRepresentations' returns a list of representatives of the
irreducible matrix representations of <G> over <F>, up to equivalence.

If <G> is the only argument then `IrreducibleRepresentations' returns a
list of representatives of the absolutely irreducible complex
representations of <G>, up to equivalence.

At the moment, methods are available for the following cases: If <G> is
abelian by supersolvable the method of~\cite{BC94} is used.

Otherwise, if <F> and <G> are both finite, the regular module of <G> is
split by MeatAxe methods which can make this an expensive operation.

Finally, if <F> is not given (i.e. it defaults to the cyclotomic
numbers) and <G> is a finite group, the method of \cite{Dix93} (see
"IrreducibleRepresentationsDixon") is used.

For other cases no methods are implemented yet.

See also `IrreducibleModules', which provides efficient
methods for solvable groups.


\beginexample
gap> g:= AlternatingGroup( 4 );;
gap> repr:= IrreducibleRepresentations( g );
[ Pcgs([ (2,4,3), (1,3)(2,4), (1,2)(3,4) ]) -> 
    [ [ [ 1 ] ], [ [ 1 ] ], [ [ 1 ] ] ], 
  Pcgs([ (2,4,3), (1,3)(2,4), (1,2)(3,4) ]) -> 
    [ [ [ E(3) ] ], [ [ 1 ] ], [ [ 1 ] ] ], 
  Pcgs([ (2,4,3), (1,3)(2,4), (1,2)(3,4) ]) -> 
    [ [ [ E(3)^2 ] ], [ [ 1 ] ], [ [ 1 ] ] ], 
  Pcgs([ (2,4,3), (1,3)(2,4), (1,2)(3,4) ]) -> 
    [ [ [ 0, 0, 1 ], [ 1, 0, 0 ], [ 0, 1, 0 ] ], 
      [ [ -1, 0, 0 ], [ 0, 1, 0 ], [ 0, 0, -1 ] ], 
      [ [ 1, 0, 0 ], [ 0, -1, 0 ], [ 0, 0, -1 ] ] ] ]
gap> ForAll( repr, IsGroupHomomorphism );
true
gap> Length( repr );
4
gap> gens:= GeneratorsOfGroup( g );
[ (1,2,3), (2,3,4) ]
gap> List( gens, x -> x^repr[1] );
[ [ [ 1 ] ], [ [ 1 ] ] ]
gap>  List( gens, x -> x^repr[4] );
[ [ [ 0, 0, -1 ], [ 1, 0, 0 ], [ 0, -1, 0 ] ], 
  [ [ 0, 1, 0 ], [ 0, 0, 1 ], [ 1, 0, 0 ] ] ]
\endexample

\>IrreducibleRepresentationsDixon( <G> ) F
\>IrreducibleRepresentationsDixon( <G>, <chi> ) F
\>IrreducibleRepresentationsDixon( <G>, <chilist> ) F

computes (reresentatives of) all irreducible complex representations for
the finite group <G>, using the method of~\cite{Dix93}, which computes
the character table and computes the representation as constituent of an
induced monomial representation of a subgroup.

This method can be quite expensive for larger groups, for example it
might involve calculation of the subgroup lattice of <G>.

If given, <chi> must be a character, in this case only a representation
for <chi> is returned.

If given, <chilist> must be a list of characters, in this case only
representations for characters in <chilist> are computed.

Note that this method might fail if for an irreducible representation
there is no subgroup in which its reduction has a linear constituent
with multiplicity one.

\beginexample
gap> a5:= AlternatingGroup( 5 );
Alt( [ 1 .. 5 ] )
gap> char:= First( Irr( a5 ), x -> x[1] = 4 );
Character( CharacterTable( Alt( [ 1 .. 5 ] ) ), [ 4, 0, 1, -1, -1 ] )
gap> hom:=IrreducibleRepresentationsDixon( a5, char );;
gap> Order( a5.1*a5.2 ) = Order( Image( hom, a5.1 )*Image( hom, a5.2 ) );
true
gap> reps:= List( ConjugacyClasses( a5 ), Representative );;
gap> List( reps, g -> TraceMat( Image( hom, g ) ) );
[ 4, 0, 1, -1, -1 ]
\endexample

%%%%%%%%%%%%%%%%%%%%%%%%%%%%%%%%%%%%%%%%%%%%%%%%%%%%%%%%%%%%%%%%%%%%%%%%%%%%%
\Section{Representations given by modules}

\>IrreducibleModules( <G>, <F>, <dim> ) O

returns a list of length 2. The first entry is a generating system of
<G>. The second entry is a list of all irreducible modules of
<G> over the field <F> in dimension <dim>, given as MeatAxe modules
(see~"GModuleByMats").

\>AbsoluteIrreducibleModules( <G>, <F>, <dim> ) O
\>AbsolutIrreducibleModules( <G>, <F>, <dim> ) O

returns a list of length 2. The first entry is a generating system of
<G>. The second entry is a list of all absolute irreducible modules of
<G> over the field <F> in dimension <dim>, given as MeatAxe modules
(see~"GModuleByMats").

\>RegularModule( <G>, <F> ) O

returns a list of length 2. The first entry is a generating system of
<G>. The second entry is the regular module of <G> over <F>, given as a
MeatAxe module (see~"GModuleByMats").


(Extensions by modules can be formed by the command `Extensions', see
"Extensions".)

%%%%%%%%%%%%%%%%%%%%%%%%%%%%%%%%%%%%%%%%%%%%%%%%%%%%%%%%%%%%%%%%%%%%%%%%%%%%%
\Section{The Dixon-Schneider Algorithm}

\atindex{Dixon-Schneider algorithm}{@Dixon-Schneider algorithm}

The {\GAP} library implementation of the Dixon-Schneider algorithm
first computes the linear characters, using the commutator factor group.
If irreducible characters are missing afterwards,
they are computed using the techniques described in~\cite{Dix67},
\cite{Sch90} and \cite{Hulpke93}.

Called with a group $G$, the function `CharacterTable'
(see~"CharacterTable") returns a character table object that stores
already information such as class lengths, but not the irreducible
characters.
The routines that compute the irreducibles may use the information that
is already contained in this table object.
In particular the ordering of classes in the computed characters
coincides with the ordering of classes in the character table of <G>
(see~"The Interface between Character Tables and Groups").
Thus it is possible to combine computations using the group
with character theoretic computations
(see~"Advanced Methods for Dixon-Schneider Calculations" for details),
for example one can enter known characters.
Note that the user is responsible for the correctness of the characters.
(There is little use in providing the trivial character to the routine.)

The computation of irreducible characters from the group needs to
identify the classes of group elements very often,
so it can be helpful to store a class list of all group elements.
Since this is obviously limited by the group order,
it is controlled by the global function `IsDxLargeGroup'
(see~"IsDxLargeGroup").

The routines compute in a prime field of size $p$,
such that the exponent of the group divides $(p-1)$ and such that
$2 \sqrt{|G|} \< p$.
Currently prime fields of size smaller than $65\,536$ are handled more
efficiently than larger prime fields,
so the runtime of the character calculation depends on how large the
chosen prime is.

The routine stores a Dixon record (see~"DixonRecord") in the group
that helps routines that identify classes,
for example `FusionConjugacyClasses', to work much faster.
Note that interrupting Dixon-Schneider calculations will prevent {\GAP}
from cleaning up the Dixon record;
when the computation by `IrrDixonSchneider' is complete,
the possibly large record is shrunk to an acceptable size.




%%%%%%%%%%%%%%%%%%%%%%%%%%%%%%%%%%%%%%%%%%%%%%%%%%%%%%%%%%%%%%%%%%%%%%%%%%%%%
\Section{Advanced Methods for Dixon-Schneider Calculations}

\index{irreducible characters!computation}

The computation of irreducible characters of very large groups may take quite
some time.
On the other hand, for the expert only a few irreducible characters may be
needed,
since the other ones can be computed using character theoretic methods
such as tensoring, induction, and restriction.
Thus {\GAP} provides also step-by-step routines for doing the calculations.
These routines allow one to compute some characters and to stop before all
are calculated.
Note that there is no ``safety net'':
The routines (being somehow internal) do no error checking,
and assume the information given is correct.

When the info level of `InfoCharacterTable' if positive,
information about the progress of splitting is printed.
(The default value is zero.)

\>DixonRecord( <G> ) AM

The `DixonRecord' of a group contains information used by the routines
to compute the irreducible characters and related information via the
Dixon-Schneider algorithm such as class arrangement and character spaces
split obtained so far.
Usually this record is passed as argument to all subfunctions to avoid a
long argument list.
It has a component `.conjugacyClasses' which contains the classes of <G>
*ordered as the algorithm needs them*.


\>DixonInit( <G> ) F

This function does all the initializations for the Dixon-Schneider
algorithm. This includes calculation of conjugacy classes, power maps,
linear characters and character morphisms.
It returns a record (see~"DixonRecord", "Components of a Dixon Record")
that can be used when calculating the irreducible characters of <G>
interactively.


\>DixontinI( <D> ) F

This function ends a Dixon-Schneider calculation.
It sorts the characters according to the degree and
unbinds components in the Dixon record that are not of use any longer.
It returns a list of irreducible characters.


\>DixonSplit( <D> ) F

This function performs one splitting step in the Dixon-Schneider
algorithm. It selects a class, computes the (partial) class sum matrix,
uses it to split character spaces and stores all the irreducible
characters obtained that way.

The class to use for splitting is chosen via `BestSplittingMatrix' and
the options described for this function apply here.

`DixonSplit' returns `true' if a split was performed and `fail'
otherwise.


\>BestSplittingMatrix( <D> ) F

returns the number of the class sum matrix that is assumed to yield the 
best (cost/earning ration) split. This matrix then will be the next one
computed and used.

The global option `maxclasslen' (defaulting to `infinity') is recognized
by `BestSplittingMatrix': Only classes whose length is limited by the
value of this option will be considered for splitting. If no usable class
remains, `fail' is returned.


\>DxIncludeIrreducibles( <D>, <new>[, <newmod>] ) F

This function takes a list of irreducible characters <new>, each given as
a list of values (corresponding to the class arrangement in <D>), and
adds these to a partial computed list of irreducibles as maintained by
the Dixon record <D>.
This permits one to add characters in interactive use obtained from other
sources and to continue the Dixon-Schneider calculation afterwards.
If the optional argument <newmod> is given, it must be a
list of reduced characters, corresponding to <new>.
(Otherwise the function has to reduce the characters itself.)

The function closes the new characters under the action of Galois
automorphisms and tensor products with linear characters.


\>SplitCharacters( <D>, <list> ) F

This routine decomposes the characters given in <list> according to the
character spaces found up to this point. By applying this routine to
tensor products etc., it may result in characters with smaller norm,
even irreducible ones. Since the recalculation of characters is only
possible if the degree is small enough, the splitting process is
applied only to characters of sufficiently small degree.


\>IsDxLargeGroup( <G> ) F

returns `true' if the order of the group <G> is smaller than the current
value of the global variable `DXLARGEGROUPORDER',
and `false' otherwise.
In Dixon-Schneider calculations, for small groups in the above sense a
class map is stored, whereas for large groups,
each occurring element is identified individually.




%%%%%%%%%%%%%%%%%%%%%%%%%%%%%%%%%%%%%%%%%%%%%%%%%%%%%%%%%%%%%%%%%%%%%%%%%%%%%
\Section{Components of a Dixon Record}

The ``Dixon record'' <D> returned by `DixonInit' (see~"DixonInit")
stores all the information that is used by the Dixon-Schneider routines
while computing the irreducible characters of a group.
Some entries, however, may be useful to know about when using the algorithm
interactively (see~"An Example of Advanced Dixon-Schneider Calculations").
\beginitems
`group':&
the group $G$ of which the character table is to be computed,

`conjugacyClasses':&
classes of $G$ (all characters stored in the Dixon record correspond to this
arrangement of classes),

`irreducibles':&
the already known irreducible characters
(given as lists of their values on the conjugacy classes),

`characterTable':&
the `CharacterTable' value of $G$ (whose characters are not yet known),

`ClassElement( <D>, <el> )':&
a function that returns the number of the class of $G$ that contains the
element <el>.
\enditems


%%%%%%%%%%%%%%%%%%%%%%%%%%%%%%%%%%%%%%%%%%%%%%%%%%%%%%%%%%%%%%%%%%%%%%%%%%%%%
\Section{An Example of Advanced Dixon-Schneider Calculations}

First, we set the appropriate info level higher
\beginexample
gap> SetInfoLevel( InfoCharacterTable, 1 );
\endexample
for printout of some internal results.
We now define our group, which is isomorphic to ${\rm PSL}_4(3)$.
\beginexample
gap> g:= PrimitiveGroup(40,5);
PSL(4,3)
gap> Size(g);
6065280
gap> d:= DixonInit( g );;
#I  29 classes
#I  choosing prime 65521
gap> c:= d.characterTable;;
\endexample
After the initialisation, one structure matrix is evaluated,
yielding smaller spaces and several irreducible characters.
\beginexample
gap> DixonSplit( d );
#I  Matrix 17,Representative of Order 3,Centralizer: 5832
#I  Dimensions: [ 1, 2, 1, 4, 12, 1, 1, 2, 1, 2, 1 ]
#I  Two-dim space split
#I  Two-dim space split
#I  Two-dim space split
17
\endexample
In this case spaces of the listed dimensions are a result of the
splitting process.
The three two dimensional spaces are split successfully by combinatoric
means.

We obtain several irreducible characters by tensor products and notify them
to the Dixon record.
\beginexample
gap> asp:= AntiSymmetricParts( c, d.irreducibles, 2 );;
gap> ro:= ReducedCharacters( c, d.irreducibles, asp );;
gap> Length( ro.irreducibles );
3
gap> DxIncludeIrreducibles( d, ro.irreducibles );
\endexample
The tensor products of the nonlinear characters among each other are reduced
with the irreducible characters.
The result is split according to the spaces found, which yields characters
of smaller norms, but no new irreducibles.
\beginexample
gap> nlc:= Filtered( d.irreducibles, i -> i[1] > 1 );;
gap> t:= Tensored( nlc, nlc );;
gap> ro:= ReducedCharacters( c, d.irreducibles, t );;  ro.irreducibles;
[  ]
gap> List( ro.remainders, i -> ScalarProduct( c, i, i) );
[ 2, 2, 4, 4, 4, 4, 13, 13, 18, 18, 19, 21, 21, 36, 36, 29, 34, 34, 42, 34, 
  48, 54, 62, 68, 68, 78, 84, 84, 88, 90, 159, 169, 169, 172, 172, 266, 271,
  271, 268, 274, 274, 280, 328, 373, 373, 456, 532, 576, 679, 683, 683, 754, 
  768, 768, 890, 912, 962, 1453, 1453, 1601, 1601, 1728, 1739, 1739, 1802, 
  2058, 2379, 2414, 2543, 2744, 2744, 2920, 3078, 3078, 4275, 4275, 4494, 
  4760, 5112, 5115, 5115, 5414, 6080, 6318, 7100, 7369, 7369, 7798, 8644, 
  10392, 12373, 12922, 14122, 14122, 18948, 21886, 24641, 24641, 25056, 
  38942, 44950, 78778 ]
gap> t:= SplitCharacters( d, ro.remainders );;
gap> List( t, i -> ScalarProduct( c, i, i ) );
[ 2, 2, 4, 2, 2, 4, 4, 3, 6, 5, 5, 9, 9, 4, 12, 13, 18, 18, 20, 18, 20, 24,
  26, 32, 32, 16, 42, 59, 69, 69, 72, 72, 36, 72, 78, 78, 84, 122, 127, 117,
  117, 127, 64, 132, 100, 144, 196, 256, 456, 532, 576, 679, 683, 683, 754,
  768, 768, 890, 912, 962, 1453, 1453, 1601, 1601, 1728, 1739, 1739, 1802, 
  2058, 2379, 2414, 2543, 2744, 2744, 2920, 3078, 3078, 4275, 4275, 4494, 
  4760, 5112, 5115, 5115, 5414, 6080, 6318, 7100, 7369, 7369, 7798, 8644, 
  10392, 12373, 12922, 14122, 14122, 18948, 21886, 24641, 24641, 25056, 
  38942, 44950, 78778 ]
\endexample
Finally we calculate the characters induced from all cyclic subgroups and
obtain the missing irreducibles by applying the LLL-algorithm to them.
\beginexample
gap> ic:= InducedCyclic( c, "all" );;
gap> ro:= ReducedCharacters( c, d.irreducibles, ic );;
gap> Length( ro.irreducibles );
0
gap> l:= LLL( c, ro.remainders );;
gap> Length( l.irreducibles );
13
\endexample
The LLL returns class function objects (see Chapter~"Class Functions"),
and the Dixon record works with character values lists.
So we convert them to a list of values
before feeding them in the machinery of the Dixon-algorithm.
\beginexample
gap> l.irreducibles[1];
Character( CharacterTable( PSL(4, 3) ), [ 640, E(13)^7+E(13)^8+E(13)^11,
  E(13)+E(13)^3+E(13)^9, E(13)^2+E(13)^5+E(13)^6, E(13)^4+E(13)^10+E(13)^12,
  0, -8, 0, 0, 0, 0, 0, -8, 0, 0, 0, -8, 0, 0, 0, 0, 0, 0, 0, 0, 1, 0, 1, 1
 ] )
gap> l:=List(l.irreducibles,ValuesOfClassFunction);;
gap> DxIncludeIrreducibles( d, l );
gap> Length( d.irreducibles );
29
gap> Length( d.classes );
29
\endexample
It turns out we have found all irreducible characters.
As the last step, we obtain the irreducible characters and tell them to the
group.
This makes them available also to the character table.
\beginexample
gap> irrs:= DixontinI( d );;
#I  Total:1 matrices,[ 17 ]
gap> SetIrr(g,irrs);
gap> Length(Irr(c));
29
gap> SetInfoLevel( InfoCharacterTable, 0 );
\endexample


%%%%%%%%%%%%%%%%%%%%%%%%%%%%%%%%%%%%%%%%%%%%%%%%%%%%%%%%%%%%%%%%%%%%%%%%%%%%%
\Section{Constructing Character Tables from Others}


The following operations take one or more character table arguments,
and return a character table.
This holds also for `BrauerTable' (see~"BrauerTable");
note that the return value of `BrauerTable' will in general not
know the irreducible Brauer characters,
and {\GAP} might be unable to compute these characters.

*Note* that whenever fusions between input and output tables occur in
these operations,
they are stored on the concerned tables,
and the `NamesOfFusionSources' values are updated.

(The interactive construction of character tables using character
theoretic methods and incomplete tables is not described here.)
*@Currently it is not supported and will be described in a chapter of its
own when it becomes available@*.



\>CharacterTableDirectProduct( <tbl1>, <tbl2> ) O

is the table of the direct product of the character tables <tbl1>
and <tbl2>.

The matrix of irreducibles of this table is the Kronecker product
(see~"KroneckerProduct") of the irreducibles of <tbl1> and <tbl2>.

Products of ordinary and Brauer character tables are supported.

In general, the result will not know an underlying group,
so missing power maps (for prime divisors of the result)
and irreducibles of the input tables may be computed in order to
construct the table of the direct product.

The embeddings of the input tables into the direct product are stored,
they can be fetched with `GetFusionMap' (see~"GetFusionMap");
if <tbl1> is equal to <tbl2> then the two embeddings are distinguished
by their `specification' components `"1"' and `"2"', respectively.

Analogously, the projections from the direct product onto the input
tables are stored, and can be distinguished by the `specification'
components.


The attribute `FactorsOfDirectProduct' (see~"FactorsOfDirectProduct")
is set to the lists of arguments.

The `\*' operator for two character tables
(see~"Operators for Character Tables") delegates to
`CharacterTableDirectProduct'.


\beginexample
gap> c2:= CharacterTable( "Cyclic", 2 );;
gap> s3:= CharacterTable( "Symmetric", 3 );;
gap> Display( CharacterTableDirectProduct( c2, s3 ) );
C2xSym(3)

     2  2  2  1  2  2  1
     3  1  .  1  1  .  1

       1a 2a 3a 2b 2c 6a
    2P 1a 1a 3a 1a 1a 3a
    3P 1a 2a 1a 2b 2c 2b

X.1     1 -1  1  1 -1  1
X.2     2  . -1  2  . -1
X.3     1  1  1  1  1  1
X.4     1 -1  1 -1  1 -1
X.5     2  . -1 -2  .  1
X.6     1  1  1 -1 -1 -1
\endexample

\>FactorsOfDirectProduct( <tbl> ) A

For an ordinary character table that has been constructed via
`CharacterTableDirectProduct' (see~"CharacterTableDirectProduct"),
the value of `FactorsOfDirectProduct' is the list of arguments in the
`CharacterTableDirectProduct' call.

Note that there is no default method for *computing* the value of
`FactorsOfDirectProduct'.



\>CharacterTableFactorGroup( <tbl>, <classes> ) O

is the character table of the factor group of the ordinary character
table <tbl> by the normal closure of the classes whose positions are
contained in the list <classes>.

The `\/' operator for a character table and a list of class positions
(see~"Operators for Character Tables") delegates to
`CharacterTableFactorGroup'.


\beginexample
gap> s4:= CharacterTable( "Symmetric", 4 );;
gap> ClassPositionsOfNormalSubgroups( s4 );
[ [ 1 ], [ 1, 3 ], [ 1, 3, 4 ], [ 1 .. 5 ] ]
gap> f:= CharacterTableFactorGroup( s4, [ 3 ] );
CharacterTable( "Sym(4)/[ 1, 3 ]" )
gap> Display( f );
Sym(4)/[ 1, 3 ]

     2  1  1  .
     3  1  .  1

       1a 2a 3a
    2P 1a 1a 3a
    3P 1a 2a 1a

X.1     1 -1  1
X.2     2  . -1
X.3     1  1  1
\endexample

\>CharacterTableIsoclinic( <tbl> ) A
\>CharacterTableIsoclinic( <tbl>, <classes> ) O
\>CharacterTableIsoclinic( <tbl>, <classes>, <centre> ) O

If <tbl> is the (ordinary or modular) character table of a group with the
structure $2\.G\.2$ with a central subgroup $Z$ of order $2$ and a normal
subgroup $N$ of index $2$ that contains $Z$
then `CharacterTableIsoclinic' returns the table of the isoclinic group
in the sense of the {\ATLAS} of Finite Groups~\cite{CCN85}, Chapter~6,
Section~7.
If $N$ is not uniquely determined then the positions of the classes
forming $N$ must be entered as list <classes>.
If $Z$ is not unique in $N$ then the position of the class consisting
of the involution in $Z$ must be entered as <centre>.

Note that also if <tbl> is a Brauer table then <classes> and <centre>
denote class numbers w.r.t.~the *ordinary* character table.


\beginexample
gap> d8:= CharacterTable( "Dihedral", 8 );;
gap> nsg:= ClassPositionsOfNormalSubgroups( d8 );
[ [ 1 ], [ 1, 3 ], [ 1 .. 3 ], [ 1, 3, 4 ], [ 1, 3 .. 5 ], [ 1 .. 5 ] ]
gap> q8:= CharacterTableIsoclinic( d8, nsg[3] );;
gap> Display( q8 );
Isoclinic(Dihedral(8))

     2  3  2  3  2  2

       1a 4a 2a 4b 4c
    2P 1a 2a 1a 2a 2a

X.1     1  1  1  1  1
X.2     1  1  1 -1 -1
X.3     1 -1  1  1 -1
X.4     1 -1  1 -1  1
X.5     2  . -2  .  .
\endexample

\>SourceOfIsoclinicTable( <tbl> ) A

For an ordinary character table that has been constructed via
`CharacterTableIsoclinic' (see~"CharacterTableIsoclinic"),
the value of `SourceOfIsoclinicTable' is the list of three arguments in
the `CharacterTableIsoclinic' call.

Note that there is no default method for *computing* the value of
`SourceOfIsoclinicTable'.


\beginexample
gap> SourceOfIsoclinicTable( q8 );
[ CharacterTable( "Dihedral(8)" ), [ 1, 2, 3 ], 3 ]
\endexample

%Declaration{CharacterTableOfNormalSubgroup}

\>CharacterTableWreathSymmetric( <tbl>, <n> ) F

returns the character table of the wreath product of a group $G$ with
the full symmetric group on $n$ points,
where <tbl> is the character table of $G$.

The result has values for `ClassParameters' and `CharacterParameters'
(see~"ctbllib:ClassParameters"
in the manual for the {\GAP} Character Table Library) stored,
the entries in these lists are sequences of partitions.
Note that this parametrization prevents the principal character from
being the first one in the list of irreducibles.


\beginexample
gap> c3:= CharacterTable( "Cyclic", 3 );;
gap> wr:= CharacterTableWreathSymmetric( c3, 2 );;
gap> Display( wr );
C3wrS2

     2  1   .   .   1  .   1  1   1   1
     3  2   2   2   2  2   2  1   1   1

       1a  3a  3b  3c 3d  3e 2a  6a  6b
    2P 1a  3b  3a  3e 3d  3c 1a  3c  3e
    3P 1a  1a  1a  1a 1a  1a 2a  2a  2a

X.1     1   1   1   1  1   1 -1  -1  -1
X.2     2   A  /A   B -1  /B  .   .   .
X.3     2  /A   A  /B -1   B  .   .   .
X.4     1 -/A  -A  -A  1 -/A -1  /A   A
X.5     2  -1  -1   2 -1   2  .   .   .
X.6     1  -A -/A -/A  1  -A -1   A  /A
X.7     1   1   1   1  1   1  1   1   1
X.8     1 -/A  -A  -A  1 -/A  1 -/A  -A
X.9     1  -A -/A -/A  1  -A  1  -A -/A

A = -E(3)^2
  = (1+ER(-3))/2 = 1+b3
B = 2*E(3)
  = -1+ER(-3) = 2b3
gap> CharacterParameters( wr )[1];
[ [ 1, 1 ], [  ], [  ] ]
\endexample


%%%%%%%%%%%%%%%%%%%%%%%%%%%%%%%%%%%%%%%%%%%%%%%%%%%%%%%%%%%%%%%%%%%%%%%%%%%%%
\Section{Sorted Character Tables}

\>CharacterTableWithSortedCharacters( <tbl> ) O
\>CharacterTableWithSortedCharacters( <tbl>, <perm> ) O

is a character table that differs from <tbl> only by the succession of
its irreducible characters.
This affects the values of the attributes `Irr' (see~"Irr") and
`CharacterParameters' (see~"ctbllib:CharacterParameters" in the manual
for the {\GAP} Character Table Library).
Namely, these lists are permuted by the permutation <perm>.

If no second argument is given then a permutation is used that yields
irreducible characters of increasing degree for the result.
For the succession of characters in the result, see~"SortedCharacters".

The result has all those attributes and properties of <tbl> that are
stored in `SupportedCharacterTableInfo' and do not depend on the
ordering of characters (see~"SupportedCharacterTableInfo").


\>SortedCharacters( <tbl>, <chars> ) O
\>SortedCharacters( <tbl>, <chars>, \"norm\" ) O
\>SortedCharacters( <tbl>, <chars>, \"degree\" ) O

is a list containing the characters <chars>, ordered as specified
by the other arguments.

There are three possibilities to sort characters:
They can be sorted according to ascending norms (parameter `\"norm\"'),
to ascending degree (parameter `\"degree\"'),
or both (no third parameter),
i.e., characters with same norm are sorted according to ascending degree,
and characters with smaller norm precede those with bigger norm.

Rational characters in the result precede other ones with same norm
and/or same degree.

The trivial character, if contained in <chars>, will always be sorted to
the first position.


\>CharacterTableWithSortedClasses( <tbl> ) O
\>CharacterTableWithSortedClasses( <tbl>, \"centralizers\" ) O
\>CharacterTableWithSortedClasses( <tbl>, \"representatives\" ) O
\>CharacterTableWithSortedClasses( <tbl>, <permutation> ) O

is a character table obtained by permutation of the classes of <tbl>.
If the second argument is the string `\"centralizers\"' then the classes
of the result are sorted according to descending centralizer orders.
If the second argument is the string `\"representatives\"' then the
classes of the result are sorted according to ascending representative
orders.
If no second argument is given then the classes of the result are sorted
according to ascending representative orders,
and classes with equal representative orders are sorted according to
descending centralizer orders.

If the second argument is a permutation <perm> then the classes of the
result are sorted by application of this permutation.

The result has all those attributes and properties of <tbl> that are
stored in `SupportedCharacterTableInfo' and do not depend on the
ordering of classes (see~"SupportedCharacterTableInfo").


\>SortedCharacterTable( <tbl>, <kernel> ) F
\>SortedCharacterTable( <tbl>, <normalseries> ) F
\>SortedCharacterTable( <tbl>, <facttbl>, <kernel> ) F

is a character table obtained on permutation of the classes and the
irreducibles characters of <tbl>.

The first form sorts the classes at positions contained in the list
<kernel> to the beginning, and sorts all characters in
`Irr( <tbl> )' such that the first characters are those that contain
<kernel> in their kernel.

The second form does the same successively for all kernels $k_i$ in
the list $<normalseries> = [ k_1, k_2, \ldots, k_n ]$ where
$k_i$ must be a sublist of $k_{i+1}$ for $1 \leq i \leq n-1$.

The third form computes the table $F$ of the factor group of <tbl>
modulo the normal subgroup formed by the classes whose positions are
contained in the list <kernel>;
$F$ must be permutation equivalent to the table <facttbl>,
in the sense of `TransformingPermutationsCharacterTables'
(see~"TransformingPermutationsCharacterTables"),
otherwise `fail' is returned.
The classes of <tbl> are sorted such that the preimages
of a class of $F$ are consecutive,
and that the succession of preimages is that of <facttbl>.
`Irr( <tbl> )' is sorted as with `SortCharTable( <tbl>, <kernel> )'.

(*Note* that the transformation is only unique up to table automorphisms
of $F$, and this need not be unique up to table automorphisms of <tbl>.)

All rearrangements of classes and characters are stable,
i.e., the relative positions of classes and characters that are not
distinguished by any relevant property is not changed.

The result has all those attributes and properties of <tbl> that are
stored in `SupportedCharacterTableInfo' and do not depend on the
ordering of classes and characters (see~"SupportedCharacterTableInfo").

The `ClassPermutation' value of <tbl> is changed if necessary,
see~"Conventions for Character Tables".

`SortedCharacterTable' uses `CharacterTableWithSortedClasses' and
`CharacterTableWithSortedCharacters'
(see~"CharacterTableWithSortedClasses",
"CharacterTableWithSortedCharacters").


\>ClassPermutation( <tbl> ) A

is a permutation $\pi$ of classes of the character table <tbl>.
If it is stored then class fusions into <tbl> that are stored on other
tables must be followed by $\pi$ in order to describe the correct
fusion.

This attribute value is bound only if <tbl> was obtained from another
table by permuting the classes, using
`CharacterTableWithSortedClasses' or `SortedCharacterTable',
(see~"CharacterTableWithSortedClasses", "SortedCharacterTable").

It is necessary because the original table and the sorted table have the
same identifier (and the same group if known),
and hence the same fusions are valid for the two tables.



\beginexample
gap> tbl:= CharacterTable( "Symmetric", 4 );
CharacterTable( "Sym(4)" )
gap> Display( tbl );
Sym(4)

     2  3  2  3  .  2
     3  1  .  .  1  .

       1a 2a 2b 3a 4a
    2P 1a 1a 1a 3a 2b
    3P 1a 2a 2b 1a 4a

X.1     1 -1  1  1 -1
X.2     3 -1 -1  .  1
X.3     2  .  2 -1  .
X.4     3  1 -1  . -1
X.5     1  1  1  1  1
\endexample

\beginexample
gap> srt1:= CharacterTableWithSortedCharacters( tbl );
CharacterTable( "Sym(4)" )
gap> List( Irr( srt1 ), Degree );
[ 1, 1, 2, 3, 3 ]
gap> srt2:= CharacterTableWithSortedClasses( tbl );
CharacterTable( "Sym(4)" )
gap> SizesCentralizers( tbl );
[ 24, 4, 8, 3, 4 ]
gap> SizesCentralizers( srt2 );
[ 24, 8, 4, 3, 4 ]
gap> nsg:= ClassPositionsOfNormalSubgroups( tbl );
[ [ 1 ], [ 1, 3 ], [ 1, 3, 4 ], [ 1 .. 5 ] ]
gap> srt3:= SortedCharacterTable( tbl, nsg );
CharacterTable( "Sym(4)" )
gap> nsg:= ClassPositionsOfNormalSubgroups( srt3 );
[ [ 1 ], [ 1, 2 ], [ 1 .. 3 ], [ 1 .. 5 ] ]
gap> Display( srt3 );
Sym(4)

     2  3  3  .  2  2
     3  1  .  1  .  .

       1a 2a 3a 2b 4a
    2P 1a 1a 3a 1a 2a
    3P 1a 2a 1a 2b 4a

X.1     1  1  1  1  1
X.2     1  1  1 -1 -1
X.3     2  2 -1  .  .
X.4     3 -1  . -1  1
X.5     3 -1  .  1 -1
\endexample

\beginexample
gap> ClassPermutation( srt3 );
(2,4,3)
\endexample


%%%%%%%%%%%%%%%%%%%%%%%%%%%%%%%%%%%%%%%%%%%%%%%%%%%%%%%%%%%%%%%%%%%%%%%%%%%%%
\Section{Automorphisms and Equivalence of Character Tables}

\>MatrixAutomorphisms( <mat>[, <maps>, <subgroup>] ) O

For a matrix <mat>, `MatrixAutomorphisms' returns the group of those
permutations of the columns of <mat> that leave the set of rows of <mat>
invariant.

If the arguments <maps> and <subgroup> are given,
only the group of those permutations is constructed that additionally
fix each list in the list <maps> under pointwise action `OnTuples',
and <subgroup> is a permutation group that is known to be a subgroup of
this group of automorphisms.

Each entry in <maps> must be a list of same length as the rows of <mat>.
For example, if <mat> is a list of irreducible characters of a group
then the list of element orders of the conjugacy classes
(see~"OrdersClassRepresentatives") may be an entry in <maps>.


\>TableAutomorphisms( <tbl>, <characters> ) O
\>TableAutomorphisms( <tbl>, <characters>, \"closed\" ) O
\>TableAutomorphisms( <tbl>, <characters>, <subgroup> ) O

`TableAutomorphisms' returns the permutation group of those matrix
automorphisms (see~"MatrixAutomorphisms") of the list <characters>
that leave the element orders (see~"OrdersClassRepresentatives")
and all stored power maps (see~"ComputedPowerMaps") of the character
table <tbl> invariant.

If <characters> is closed under Galois conjugacy --this is always
fulfilled for ordinary character tables--
the string `\"closed\"' may be entered as the third argument.
Alternatively, a known subgroup <subgroup> of the table automorphisms
can be entered as the third argument.

The attribute `AutomorphismsOfTable' (see~"AutomorphismsOfTable")
can be used to compute and store the table automorphisms for the case
that <characters> equals `Irr( <tbl> )'.


\beginexample
gap> tbld8:= CharacterTable( "Dihedral", 8 );;
gap> irrd8:= Irr( tbld8 );
[ Character( CharacterTable( "Dihedral(8)" ), [ 1, 1, 1, 1, 1 ] ), 
  Character( CharacterTable( "Dihedral(8)" ), [ 1, 1, 1, -1, -1 ] ), 
  Character( CharacterTable( "Dihedral(8)" ), [ 1, -1, 1, 1, -1 ] ), 
  Character( CharacterTable( "Dihedral(8)" ), [ 1, -1, 1, -1, 1 ] ), 
  Character( CharacterTable( "Dihedral(8)" ), [ 2, 0, -2, 0, 0 ] ) ]
gap> orders:= OrdersClassRepresentatives( tbld8 );
[ 1, 4, 2, 2, 2 ]
gap> MatrixAutomorphisms( irrd8 );
Group([ (4,5), (2,4) ])
gap> MatrixAutomorphisms( irrd8, [ orders ], Group( () ) );
Group([ (4,5) ])
gap> TableAutomorphisms( tbld8, irrd8 );
Group([ (4,5) ])
\endexample

\>TransformingPermutations( <mat1>, <mat2> ) O

Let <mat1> and <mat2> be matrices.
`TransformingPermutations' tries to construct
a permutation $\pi$ that transforms the set of rows of the matrix
<mat1> to the set of rows of the matrix <mat2>
by permuting the columns.

If such a permutation exists,
a record with components `columns', `rows', and `group' is returned,
otherwise `fail'.
For $`TransformingPermutations( <mat1>, <mat2> ) = <r>' \not= `fail'$,
we have `<mat2> =
 Permuted( List( <mat1>, x -> Permuted( x, <r>.columns ) ),<r>.rows )'.

`<r>.group' is the group of matrix automorphisms of <mat2>
(see~"MatrixAutomorphisms").
This group stabilizes the transformation in the sense that applying any
of its elements to the columns of <mat2>
preserves the set of rows of <mat2>.


\>TransformingPermutationsCharacterTables( <tbl1>, <tbl2> ) O

Let <tbl1> and <tbl2> be character tables.
`TransformingPermutationsCharacterTables' tries to construct
a permutation $\pi$ that transforms the set of rows of the matrix
`Irr( <tbl1> )' to the set of rows of the matrix `Irr( <tbl2> )'
by permuting the columns (see~"TransformingPermutations"),
such that $\pi$ transforms also the power maps and the element orders.

If such a permutation $\pi$ exists then a record with the components
`columns' ($\pi$),
`rows' (the permutation of `Irr( <tbl1> )' corresponding to $\pi$), and
`group' (the permutation group of table automorphisms of <tbl2>,
see~"AutomorphismsOfTable") is returned.
If no such permutation exists, `fail' is returned.


\beginexample
gap> tblq8:= CharacterTable( "Quaternionic", 8 );;
gap> irrq8:= Irr( tblq8 );
[ Character( CharacterTable( "Q8" ), [ 1, 1, 1, 1, 1 ] ), 
  Character( CharacterTable( "Q8" ), [ 1, 1, 1, -1, -1 ] ), 
  Character( CharacterTable( "Q8" ), [ 1, -1, 1, 1, -1 ] ), 
  Character( CharacterTable( "Q8" ), [ 1, -1, 1, -1, 1 ] ), 
  Character( CharacterTable( "Q8" ), [ 2, 0, -2, 0, 0 ] ) ]
gap> OrdersClassRepresentatives( tblq8 );
[ 1, 4, 2, 4, 4 ]
gap> TransformingPermutations( irrd8, irrq8 );
rec( columns := (), rows := (), group := Group([ (4,5), (2,4) ]) )
gap> TransformingPermutationsCharacterTables( tbld8, tblq8 );
fail
gap> tbld6:= CharacterTable( "Dihedral", 6 );;
gap> tbls3:= CharacterTable( "Symmetric", 3 );;
gap> TransformingPermutationsCharacterTables( tbld6, tbls3 );
rec( columns := (2,3), rows := (1,3,2), group := Group(()) )
\endexample

\>FamiliesOfRows( <mat>, <maps> ) F

distributes the rows of the matrix <mat> into families as follows.
Two rows of <mat> belong to the same family if there is a permutation
of columns that maps one row to the other row.
Each entry in the list <maps> is regarded to form a family of length 1.

`FamiliesOfRows( <mat>, <maps> )' returns a record with components
\beginitems
`famreps' &
     the list of representatives for each family,

`permutations' &
     the list that contains at position `i' a list of permutations
     that map the members of the family with representative `famreps[i]'
     to that representative,

`families' &
     the list that contains at position `i' the list of positions
     of members of the family of representative `famreps[i]';
     (for the element `<maps>[i]' the only member of the family
     will get the number `Length( <mat> ) + i').
\enditems




%%%%%%%%%%%%%%%%%%%%%%%%%%%%%%%%%%%%%%%%%%%%%%%%%%%%%%%%%%%%%%%%%%%%%%%%%%%%%
\Section{Storing Normal Subgroup Information}

\>NormalSubgroupClassesInfo( <tbl> ) AM

Let <tbl> be the ordinary character table of the group $G$.
Many computations for group characters of $G$ involve computations
in normal subgroups or factor groups of $G$.

In some cases the character table <tbl> is sufficient;
for example questions about a normal subgroup $N$ of $G$ can be answered
if one knows the conjugacy classes that form $N$,
e.g., the question whether a character of $G$ restricts
irreducibly to $N$.
But other questions require the computation of $N$ or
even more information, like the character table of $N$.

In order to do these computations only once, one stores in the group a
record with components to store normal subgroups, the corresponding lists
of conjugacy classes, and (if necessary) the factor groups, namely

\beginitems
`nsg': &
    list of normal subgroups of $G$, may be incomplete,

`nsgclasses': &
    at position $i$, the list of positions of conjugacy
    classes of <tbl> forming the $i$-th entry of the `nsg' component,

`nsgfactors': &
    at position $i$, if bound, the factor group
    modulo the $i$-th entry of the `nsg' component.
\enditems

`NormalSubgroupClasses',
`FactorGroupNormalSubgroupClasses', and
`ClassPositionsOfNormalSubgroup'
each use these components, and they are the only functions to do so.

So if you need information about a normal subgroup for that you know the
conjugacy classes, you should get it using `NormalSubgroupClasses'.  If
the normal subgroup was already used it is just returned, with all the
knowledge it contains.  Otherwise the normal subgroup is added to the
lists, and will be available for the next call.

For example, if you are dealing with kernels of characters using the
`KernelOfCharacter' function you make use of this feature
because `KernelOfCharacter' calls `NormalSubgroupClasses'.


\>ClassPositionsOfNormalSubgroup( <tbl>, <N> ) F

is the list of positions of conjugacy classes of the character table
<tbl> that are contained in the normal subgroup <N>
of the underlying group of <tbl>.


\>NormalSubgroupClasses( <tbl>, <classes> ) F

returns the normal subgroup of the underlying group $G$ of the ordinary
character table <tbl>
that consists of those conjugacy classes of <tbl> whose positions are in
the list <classes>.

If `NormalSubgroupClassesInfo( <tbl> ).nsg' does not yet contain
the required normal subgroup,
and if `NormalSubgroupClassesInfo( <tbl> ).normalSubgroups' is bound then
the result will be identical to the group in
`NormalSubgroupClassesInfo( <tbl> ).normalSubgroups'.


\>FactorGroupNormalSubgroupClasses( <tbl>, <classes> ) F

is the factor group of the underlying group $G$ of the ordinary character
table <tbl> modulo the normal subgroup of $G$ that consists of those
conjugacy classes of <tbl> whose positions are in the list <classes>.


\beginexample
gap> g:= SymmetricGroup( 4 );
Sym( [ 1 .. 4 ] )
gap> SetName( g, "S4" );
gap> tbl:= CharacterTable( g );
CharacterTable( S4 )
gap> irr:= Irr( g );
[ Character( CharacterTable( S4 ), [ 1, -1, 1, 1, -1 ] ), 
  Character( CharacterTable( S4 ), [ 3, -1, -1, 0, 1 ] ), 
  Character( CharacterTable( S4 ), [ 2, 0, 2, -1, 0 ] ), 
  Character( CharacterTable( S4 ), [ 3, 1, -1, 0, -1 ] ), 
  Character( CharacterTable( S4 ), [ 1, 1, 1, 1, 1 ] ) ]
gap> kernel:= KernelOfCharacter( irr[3] );
Group([ (1,2)(3,4), (1,4)(2,3) ])
gap> HasNormalSubgroupClassesInfo( tbl );
true
gap> NormalSubgroupClassesInfo( tbl );
rec( nsg := [ Group([ (1,2)(3,4), (1,4)(2,3) ]) ], nsgclasses := [ [ 1, 3 ] ],
  nsgfactors := [  ] )
gap> ClassPositionsOfNormalSubgroup( tbl, kernel );
[ 1, 3 ]
gap> FactorGroupNormalSubgroupClasses( tbl, [ 1, 3 ] );
Group([ f1, f2 ])
gap> NormalSubgroupClassesInfo( tbl );
rec( nsg := [ Group([ (1,2)(3,4), (1,4)(2,3) ]) ], nsgclasses := [ [ 1, 3 ] ],
  nsgfactors := [ Group([ f1, f2 ]) ] )
\endexample


%%%%%%%%%%%%%%%%%%%%%%%%%%%%%%%%%%%%%%%%%%%%%%%%%%%%%%%%%%%%%%%%%%%%%%%%%%%%%
%%
%E

