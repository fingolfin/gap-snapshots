% This file was created automatically from string.msk.
% DO NOT EDIT!
%%%%%%%%%%%%%%%%%%%%%%%%%%%%%%%%%%%%%%%%%%%%%%%%%%%%%%%%%%%%%%%%%%%%%%%%%%%%%
%%
%A  string.msk                  GAP documentation            Martin Schoenert
%A                                                           Alexander Hulpke
%%
%A  @(#)$Id: string.msk,v 1.32 2002/10/01 15:03:08 gap Exp $
%%
%Y  (C) 1998 School Math and Comp. Sci., University of St.  Andrews, Scotland
%Y  Copyright (C) 2002 The GAP Group
%%
\Chapter{Strings and Characters}

\index{type!strings}
\index{doublequotes}
\index{singlequotes}

\>IsChar( <obj> ) C
\>IsCharCollection( <obj> ) C

A *character* is simply an object in {\GAP} that represents an arbitrary
character from the character set of the operating system.
Character literals can be entered in {\GAP} by enclosing the character
in *singlequotes* `{'}'.


\beginexample
gap> x:= 'a';  IsChar( x );
'a'
true
gap> '*';
'*'
\endexample

\>IsString( <obj> ) C

A *string* is  a dense list (see~"IsList", "IsDenseList")
of characters (see~"IsChar"); thus strings are always homogeneous
(see~"IsHomogeneousList").

A string literal can either be entered  as the list of characters
or by writing the characters between *doublequotes* `\"'.
{\GAP} will always output strings in the latter format.
However, the input via the double quote syntax enables {\GAP} to store
the string in an efficient compact internal representation. See
"IsStringRep" below for more details.

Each character, in particular those which cannot be typed directly from the
keyboard, can also be typed in three digit octal notation. And for some
special characters (like the newline character) there is a further
possibility to type them, see section "Special Characters".


\beginexample
gap> s1 := ['H','e','l','l','o',' ','w','o','r','l','d','.'];
"Hello world."
gap> IsString( s1 );
true
gap> s2 := "Hello world.";
"Hello world."
gap> s1 = s2;
true
gap> s3 := "";  # the empty string
""
gap> s3 = [];
true
gap> IsString( [] );
true
gap> IsString( "123" );  IsString( 123 );
true
false
gap> IsString( [ '1', '2', '3' ] );
true
gap> IsString( [ '1', '2', , '4' ] );  # strings must be dense
false
gap> IsString( [ '1', '2', 3 ] );  # strings must only contain characters
false
\endexample

%notest
\beginexample
gap> s := "\007";
"\007"
gap> Print(s); # rings bell in many terminals
\endexample

Note that a string is just a special case of a list.
So everything that is possible for lists (see~"Lists") is also possible
for strings.
Thus you can access the characters in such a string (see~"List Elements"),
test for membership (see~"Membership Test for Collections"),
ask for the length, concatenate strings (see~"Concatenation"),
form substrings etc.
You can even assign to a mutable string (see~"List Assignment").
Of course unless you assign a character in such a way that the list stays
dense,
the resulting list will no longer be a string.

\beginexample
gap> Length( s2 );
12
gap> s2[2];
'e'
gap> 'a' in s2;
false
gap> s2[2] := 'a';;  s2;
"Hallo world."
gap> s1{ [1..4] };
"Hell"
gap> Concatenation( s1{ [ 1 .. 6 ] }, s1{ [ 1 .. 4 ] } );
"Hello Hell"
\endexample

If a string is  displayed by `View', for example as  result of an evaluation
(see~"Main  Loop"), or  by `ViewObj'  and `PrintObj',  it is  displayed with
enclosing doublequotes. (But  note that there is an ambiguity  for the empty
string which is also  an empty list of arbitrary {\GAP}  objects; it is only
printed like  a string if  it was  input as empty  string or converted  to a
string  with "ConvertToStringRep".)  The  difference  between `ViewObj'  and
`PrintObj'  is that  the  latter prints  *all*  non-printable and  non-ASCII
characters  in  three  digit  octal  notation,  while  `ViewObj'  sends  all
printable characters to  the output stream. The output of  `PrintObj' can be
read back into {\GAP}.

Strings  behave  differently  from  other {\GAP}  objects  with  respect  to
`Print', `PrintTo',  or `AppendTo'. These  commands *interpret* a  string in
the sense that  they essentially send the characters of  the string directly
to the output stream/file. (But depending on  the type of the stream and the
presence of some special characters used  as hints for line breaks there may
be sent some additional newline (or backslash and newline) characters.

\beginexample
gap> s4:= "abc\"def\nghi";;
gap> View( s4 );  Print( "\n" );
"abc\"def\nghi"
gap> ViewObj( s4 );  Print( "\n" );
"abc\"def\nghi"
gap> PrintObj( s4 );  Print( "\n" );
"abc\"def\nghi"
gap> Print( s4 );  Print( "\n" );
abc"def
ghi
gap> s := "German uses strange characters: \344\366\374\337\n";
"German uses strange characters: ����\n"
gap> Print(s);
German uses strange characters: ����
gap> PrintObj(s);
"German uses strange characters: \344\366\374\337\n"gap> 
\endexample

Note that only those line breaks are printed by `Print' that are contained
in the string (`\\n' characters, see~"Special Characters"),
as is shown in the example below.

%notest
\beginexample
gap> s1;
"Hello world."
gap> Print( s1 );
Hello world.gap> Print( s1, "\n" );
Hello world.
gap> Print( s1, "\nnext line\n" );
Hello world.
next line
\endexample


%%%%%%%%%%%%%%%%%%%%%%%%%%%%%%%%%%%%%%%%%%%%%%%%%%%%%%%%%%%%%%%%%%%%%%%%%%%%%
\Section{Special Characters}

\index{escaped characters}
\index{special character sequences}

There are a number of *special character sequences* that can be used
between the singlequotes of a character literal or between the
doublequotes of a string literal to specify characters. 
They consist of two characters. The first is
a backslash $\backslash$. The second may be any character. If it is an octal
digit (from `0' to `7') there must be two more such digits. The meaning
is given in the following list

\beginlist

\indextt{\\n}\index{newline character}
\item{`\\n'}
    *newline character*.
    This is the character that, at least on UNIX systems,
    separates lines in a text file.
    Printing of this character in a string has the effect of moving
    the cursor down one line and back to the beginning of the line.

\indextt{\\\"}\index{doublequote character}
\item{`\\\"'}
    *doublequote character*.
    Inside a string a doublequote must be escaped by the backslash,
    because it is otherwise interpreted as end of the string.

\atindex{\\'}{@\noexpand`\\\pif'}\index{singlequote character}
\item{`\\\pif'}
    *singlequote character*.
    Inside a character a singlequote must escaped by the backslash,
    because it is otherwise interpreted as end of the character.

\indextt{\\\\}\index{backslash character}
\item{`\\\\'}
    *backslash character*.
    Inside a string a backslash must be escaped by another backslash,
    because it is otherwise interpreted as first character of
    an escape sequence.

\indextt{\\b}\index{backspace character}
\item{`\\b'}
    *backspace character*.
    Printing this character should have the effect of moving the cursor
    back one character.  Whether it works or not is system dependent
    and should not be relied upon.

\indextt{\\r}\index{carriage return character}
\item{`\\r'}
    *carriage return character*.
    Printing this character should have the effect of moving the cursor
    back to the beginning of the same line.  Whether this works or not
    is again system dependent.

\indextt{\\c}\index{flush character}
\item{`\\c'}
    *flush character*.
    This character is not printed.
    Its purpose is to flush the output queue.
    Usually {\GAP} waits until it sees a <newline> before it prints a string.
    If you want to display a string that does not include this character
    use `\\c'.

\indextt{\\XYZ}\index{octal character codes}
\item{`\\XYZ'}
    with X, Y, Z three octal digits. This is translated to the character
    correponding to the number X*64+Y*8+Z modulo 256. This can be used to 
    specify and store  arbitrary binary data as a string in  {\GAP}.

\index{escaping non-special characters}
\item{other}
    For any other character the backslash is simply ignored.

\endlist

Again, if the line is displayed as result of an evaluation, those escape
sequences are displayed in the same way that they are input. 

Only  `Print', `PrintTo', or `AppendTo' send the characters directly to the
output stream.

% XXX Should the characters \< and \> and their use with `Print' be documented?

\beginexample
gap> "This is one line.\nThis is another line.\n";
"This is one line.\nThis is another line.\n"
gap> Print( last );
This is one line.
This is another line.
\endexample

Note in particular that it is not  allowed to enclose a <newline> inside the
string. You can use the  special character sequence `$\backslash$n' to write
strings that include  <newline> characters. If, however, an  input string is
too  long to  fit on  a single  line it  is possible  to *continue*  it over
several lines.  In this case the  last character of each  input line, except
the last line  must be a backslash. Both backslash  and <newline> are thrown
away by  {\GAP} while reading  the string.  Note that the  same continuation
mechanism is available for identifiers and integers.

\beginexample
gap> "This is a very long string that does not fit on a line \
gap> and is therefore continued on the next line.";
"This is a very long string that does not fit on a line and is therefore conti\
nued on the next line."
\endexample

Note that the output is also continued, but at a different place
that is determined by the value of `SizeScreen' (see~"SizeScreen").


%%%%%%%%%%%%%%%%%%%%%%%%%%%%%%%%%%%%%%%%%%%%%%%%%%%%%%%%%%%%%%%%%%%%%%%%%%%%%
\Section{Internally Represented Strings}

\index{convert!to a string}

\>IsStringRep( <obj> ) R

`IsStringRep' is a special (internal) representation of dense lists
of characters.
Dense lists of characters can be converted into this representation
using `ConvertToStringRep'.
Note that calling `IsString' does *not* change the representation.


\>ConvertToStringRep( <obj> ) F

If <obj> is a dense internally represented list of characters then
`ConvertToStringRep' changes the representation to `IsStringRep'.
This is useful in particular for converting the empty list `[]',
which usually is in `IsPlistRep', to `IsStringRep'.
If <obj> is not a string then `ConvertToStringRep' signals an error.


\>IsEmptyString( <str> ) F

`IsEmptyString' returns `true' if <str> is the empty string in the
representation `IsStringRep', and `false' otherwise.
Note that the empty list `[]' and the empty string `\"\"' have the same
type, the recommended way to distinguish them is via `IsEmptyString'.
For formatted printing, this distinction is sometimes necessary.


\beginexample
gap> l:= [];;  IsString( l );  IsEmptyString( l );  IsEmpty( l );
true
false
true
gap> l;  ConvertToStringRep( l );  l;
[  ]
""
gap> IsEmptyString( l );  IsEmptyString( "" );  IsEmptyString( "abc" );
true
true
false
gap> ll:= [ 'a', 'b' ];  IsStringRep( ll );  ConvertToStringRep( ll );
"ab"
false
gap> ll;  IsStringRep( ll );
"ab"
true
\endexample

\>`CharsFamily' V

Each character lies in the family `CharFamily',
each nonempty string lies in the collections family of this family.
Note the subtle differences between the empty list `[]' and the empty
string `\"\"' when both are printed.




%%%%%%%%%%%%%%%%%%%%%%%%%%%%%%%%%%%%%%%%%%%%%%%%%%%%%%%%%%%%%%%%%%%%%%%%%
\Section{Recognizing Characters}

\>IsDigitChar( <c> ) F

checks whether the character <c> is a digit, i.e., occurs in the string
`\"0123456789\"'.


\>IsLowerAlphaChar( <c> ) F

checks whether the character <c> is a lowercase alphabet letter, i.e.,
occurs in the string `\"abcdefghijklmnopqrstuvwxyz\"'.


\>IsUpperAlphaChar( <c> ) F

checks whether the character <c> is an uppercase alphabet letter, i.e.,
occurs in the string `\"ABCDEFGHIJKLMNOPQRSTUVWXYZ\"'.


\>IsAlphaChar( <c> ) F

checks whether the character <c> is either a lowercase or an uppercase
alphabet letter.



%%%%%%%%%%%%%%%%%%%%%%%%%%%%%%%%%%%%%%%%%%%%%%%%%%%%%%%%%%%%%%%%%%%%%%%%%
\Section{Comparisons of Strings}

\>`<string1> = <string2>'{strings!equality of}
\>`<string1> \<> <string2>'{strings!inequality of}

The equality operator `=' returns to `true' if the two strings
<string1> and <string2> are equal and `false' otherwise. The inequality
operator `\<>' returns `true' if the two strings <string1> and <string2>
are not equal and `false' otherwise.

\beginexample
gap> "Hello world.\n" = "Hello world.\n";
true
gap> "Hello World.\n" = "Hello world.\n";  # string comparison is case sensitive
false
gap> "Hello world." = "Hello world.\n";  # the first string has no <newline>
false
gap> "Goodbye world.\n" = "Hello world.\n";
false
gap> [ 'a', 'b' ] = "ab";
true
\endexample

\>`<string1> \< <string2>'{strings!lexicographic ordering of}

The ordering of strings is lexicographically according to the order
implied by the underlying, system dependent, character set.

\beginexample
gap> "Hello world.\n" < "Hello world.\n";  # the strings are equal
false
gap> "Hello World." < "Hello world.";  # in ASCII capitals range before small letters
true
gap> "Hello world." < "Hello world.\n";  # prefixes are always smaller
true
gap> "Goodbye world.\n" < "Hello world.\n";  # `G' comes before `H', in ASCII at least
true
\endexample

Strings can be compared via `\<' with certain {\GAP} objects that are not
strings, see~"Comparisons" for the details.


%%%%%%%%%%%%%%%%%%%%%%%%%%%%%%%%%%%%%%%%%%%%%%%%%%%%%%%%%%%%%%%%%%%%%%%%%%%%%
\Section{Operations to Produce or Manipulate Strings}

\>String( <obj> ) A
\>String( <obj>, <length> ) O

`String' returns a representation of <obj>,
which may be an object of arbitrary type, as a string.
This string should approximate as closely as possible the character
sequence you see if you print <obj>.

If <length> is given it must be an integer.
The absolute value gives the minimal length of the result.
If the string representation of <obj> takes less than that many
characters it is filled with blanks.
If <length> is positive it is filled on the left,
if <length> is negative it is filled on the right.

In the two argument case, the string returned is a new mutable
string (in particular not a part of any other object);
it can be modified safely,
and `MakeImmutable' may be safely applied to it.



\beginexample
gap> String(123);String([1,2,3]);
"123"
"[ 1, 2, 3 ]"
\endexample

\>HexStringInt( <int> ) F

returns a string which represents the integer <int> with hexa-decimal
digits (using `A-F' as digits `10-15'). The inverse translation can be achieved
with "IntHexString".

\>StringPP( <int> ) F

returns a string representing the prime factor decomposition
of the integer <int>.


\beginexample
gap> StringPP(40320);
"2^7*3^2*5*7"
\endexample

\>WordAlp( <alpha>, <nr> ) F

returns a string that is the <nr>-th word over the alphabet list
<alpha>, w.r.t. word length and lexicographical order.
The empty word is `WordAlp( <alpha>, 0 )'.


\beginexample
gap> List([0..5],i->WordAlp("abc",i));
[ "", "a", "b", "c", "aa", "ab" ]
\endexample

\>LowercaseString( <string> ) F

returns a lowercase version of the string <string>,
that is, a string in which each uppercase alphabet character is replaced
by the corresponding lowercase character.


\beginexample
gap> LowercaseString("This Is UpperCase");
"this is uppercase"
\endexample

\>SplitString( <string>, <seps>[, <wspace>] ) O

This function accepts a string <string> and lists <seps> and, optionally,
<wspace> of characters.  Now string is split into substrings at each
occurrence of a character in <seps> or <wspace>.  The characters in
<wspace> are interpreted as white space characters.  Substrings of
characters in <wspace> are treated as one white space character and they
are ignored at the beginning and end of a string.

Both arguments <seps> and <wspace> can be single characters.

Each string in the resulting list of substring does not contain any
characters in <seps> or <wspace>.

A character that occurs both in <seps> and <wspace> is treated as a
white space character.

A separator at the end of a string is interpreted as a terminator; in
this case, the separator does not produce a trailing empty string.
Also see~"Chomp".


\beginexample
gap> SplitString( "substr1:substr2::substr4", ":" );
[ "substr1", "substr2", "", "substr4" ]
gap> SplitString( "a;b;c;d;", ";" );
[ "a", "b", "c", "d" ]
gap> SplitString( "/home//user//dir/", "", "/" );
[ "home", "user", "dir" ]
\endexample

\>ReplacedString( <string>, <old>, <new> ) F

replaces occurrences of the string <old> in <string> by  <new>,  starting
from the left  and  always  replacing  the  first  occurrence.  To  avoid
infinite recursion, characters which have been replaced already, are  not
subject to renewed replacement.


\beginexample
gap> ReplacedString("abacab","a","zl");
"zlbzlczlb"
gap> ReplacedString("ababa", "aba","c");
"cba"
gap> ReplacedString("abacab","a","ba");
"babbacbab"
\endexample


\>NormalizeWhitespace( <string> ) F

This  function changes  the string  <string> in  place. The  characters `  '
(space), `\\n', `\\r' and `\\t' are considered as *white space*. Leading and
trailing white space characters in  <string> are removed. Sequences of white
space characters  between other  characters are replaced  by a  single space
character.

See "NormalizedWhitespace" for a non-destructive version.

\beginexample
gap> s := "   x y \n\n\t\r  z\n   \n";
"   x y \n\n\t\r  z\n   \n"
gap> NormalizeWhitespace(s);
gap> s;
"x y z"
\endexample

\>NormalizedWhitespace( <str> ) F

This function returns a copy of string <str> to which
"NormalizeWhitespace" was applied.



For the possibility to print {\GAP} objects to strings, see~"String Streams".

\>JoinStringsWithSeparator( <list>[, <sep>] ) F

joins <list> (a list of strings) after interpolating <sep> (or  `","'  if
the second argument is omitted) between each adjacent  pair  of  strings;
<sep> should be a string.

*Examples*

\beginexample
gap> list := List([1..10], String);                                  
[ "1", "2", "3", "4", "5", "6", "7", "8", "9", "10" ]
gap> JoinStringsWithSeparator(list);
"1,2,3,4,5,6,7,8,9,10"
gap> JoinStringsWithSeparator(["The", "quick", "brown", "fox"], " ");
"The quick brown fox"
gap> JoinStringsWithSeparator(["a", "b", "c", "d"], ",\n    ");    
"a,\n    b,\n    c,\n    d"
gap> Print("    ", last, "\n");
    a,
    b,
    c,
    d
\endexample

Recall, `last' is the last expression output by {\GAP}.


\>Chomp( <str> ) F

Like the similarly  named  Perl  function,  `Chomp'  removes  a  trailing
newline character (or carriage-return line-feed couplet)  from  a  string
argument <str> if present and returns the  result.  If  <str>  is  not  a
string or does  not  have  such  trailing  character(s)  it  is  returned
unchanged. This latter property means that `Chomp'  is  safe  to  use  in
cases where one is manipulating the  result  of  another  function  which
might sometimes return `fail', for example.

\beginexample
gap> Chomp("The quick brown fox jumps over the lazy dog.\n");
"The quick brown fox jumps over the lazy dog."
gap> Chomp("The quick brown fox jumps over the lazy dog.\r\n");
"The quick brown fox jumps over the lazy dog."
gap> Chomp("The quick brown fox jumps over the lazy dog.");
"The quick brown fox jumps over the lazy dog."
gap> Chomp(fail);
fail
gap> Chomp(32);
32
\endexample

*Note:*
`Chomp' only removes a trailing newline character  from  <str>.  If  your
string contains several newline characters and you really want  to  split
<str> into lines at the newline  characters  (and  remove  those  newline
characters) then you should use `SplitString' (see~"SplitString"), e.g.

\beginexample
gap> str := "The quick brown fox\njumps over the lazy dog.\n";
"The quick brown fox\njumps over the lazy dog.\n"
gap> SplitString(str, "", "\n");
[ "The quick brown fox", "jumps over the lazy dog." ]
gap> Chomp(str);
"The quick brown fox\njumps over the lazy dog."
\endexample



%%%%%%%%%%%%%%%%%%%%%%%%%%%%%%%%%%%%%%%%%%%%%%%%%%%%%%%%%%%%%%%%%%%%%%%%%%%%%
\Section{Character Conversion}

The following functions convert characters in their internal integer values
and vice versa. Note that the number corresponding to a particular character
might depend on the system used. While most systems use an extension of
ASCII, in particular character values outside the range 32-126 might differ
between architectures.

All functions in this section are internal and behaviour is undefined if
invarid arguments are given.

\>INT_CHAR(<char>) F

returns an integer value in the range 0-255 that corresponds to <char>.

\>CHAR_INT(<int>) F

returns a character which corresponds to the integer value <int>, which must
be in the range 0-255.

\beginexample
gap> c:=CHAR_INT(65);
'A'
gap> INT_CHAR(c);
65
\endexample

\>SINT_CHAR(<char>) F

returns a signed integer value in the range $-128$--127 that corresponds to
<char>.

\>CHAR_SINT(<int>) F

returns a character which corresponds to the signed integer value <int>,
which must be in the range $-128$--127.

The signed and unsigned integer functions behave the same for values in the
range from 0 to 127.

\beginexample
gap> SINT_CHAR(c);
65
gap> c:=CHAR_SINT(-20);;
gap> SINT_CHAR(c);
-20
gap> INT_CHAR(c);
236
gap> SINT_CHAR(CHAR_INT(255));
-1
\endexample

%%%%%%%%%%%%%%%%%%%%%%%%%%%%%%%%%%%%%%%%%%%%%%%%%%%%%%%%%%%%%%%%%%%%%%%%%%%%%
\Section{Operations to Evaluate Strings}

\index{evaluation!strings}
\>Int( <str> )!{for strings} A
\>Rat( <str> )!{for strings} A
\>IntHexString( <str> ) F

return either an integer (`Int' and `IntHexString'), or a rational (`Rat')
as represented by the string <str>.
`Int' returns `fail' if non-digit characters occur in <str>.
For `Rat', the argument string may start with the sign character `{'}-{'}',
followed by either a sequence of digits or by two sequences of digits
that are separated by one of the characters `{'}/{'}' or `{'}.{'}',
where the latter stands for a decimal dot.
(The methods only evaluate numbers but do *not* perform arithmetic!)

`IntHexString' evaluates an integer written with hexa-decimal digits. Here
the letters <a-f> or <A-F> are used as *digits* <10-15>. An error occurs
when a wrong character is found in the string. This function can be used
(together with "HexStringInt") for efficiently storing and reading large
integers from respectively into {\GAP}. Note that the translation between
integers and their hexa-decimal representation costs linear computation time
in terms of the number of digits, while translation from and into decimal
representation needs substantial computations.

\beginexample
gap> Int("12345")+1;
12346
gap> Int("123/45");
fail
gap> Int("1+2");
fail
gap> Int("-12");
-12
gap> Rat("123/45");
41/15
gap> Rat( "123.45" );
2469/20
gap> IntHexString("-abcdef0123456789");
-12379813738877118345
gap> HexStringInt(last);
"-ABCDEF0123456789"
\endexample

\>Ordinal( <n> ) F

returns the ordinal of the integer <n> as a string.

\beginexample
gap> Ordinal(2);  Ordinal(21);  Ordinal(33);  Ordinal(-33);
"2nd"
"21st"
"33rd"
"-33rd"
\endexample

\>EvalString( <expr> ) F

passes <expr> (a string) through  an  input text stream  so  that  {\GAP}
interprets it, and returns the  result.  The  following  trivial  example
demonstrates its use.

\beginexample
gap> a:=10;
10
gap> EvalString("a^2");
100
\endexample

`EvalString' is intended for *single* expressions. A sequence of commands
may   be   interpreted   by   using   the   functions   `InputTextString'
(see~"InputTextString")  and  `ReadAsFunction'   (see~"ReadAsFunction!for
streams") together; see "Operations for Input Streams" for an example.



%%%%%%%%%%%%%%%%%%%%%%%%%%%%%%%%%%%%%%%%%%%%%%%%%%%%%%%%%%%%%%%%%%%%%%%%%%%%%
\Section{Calendar Arithmetic}

All calendar functions use the Gregorian calendar.


\>DaysInYear( <year> ) F

returns the number of days in a year.


\>DaysInMonth( <month>, <year> ) F

returns the number of days in month number <month> of <year> (and `fail'
if `month' is integer not in valid range.


\beginexample
gap> DaysInYear(1998);
365
gap> DaysInMonth(3,1998);
31
\endexample

\>DMYDay( <day> ) F

converts a number of days, starting 1-Jan-1970 to a list
`[<day>,<month>,<year>]' in Gregorian calendar counting.


\>DayDMY( <dmy> ) F

returns the number of days from 01-Jan-1970 to the day given by <dmy>.
<dmy> must be a list of the form `[<day>,<month>,<year>]' in Gregorian
calendar counting. The result is `fail' on input outside valid ranges.

Note that this makes not much sense for early dates like: before 1582
(no Gregorian calendar at all), or before 1753 in many English countries
or before 1917 in Russia.


\>WeekDay( <date> ) F

returns the weekday of a day given by <date>. <date> can be a number of
days since 1-Jan-1970 or a list `[<day>,<month>,<year>]'.


\>StringDate( <date> ) F

converts <date> to a readable string.  <date> can be a number of days
since 1-Jan-1970 or a list `[<day>,<month>,<year>]'.


\beginexample
gap> DayDMY([1,1,1970]);DayDMY([2,1,1970]);
0
1
gap> DMYDay(12345);
[ 20, 10, 2003 ]
gap> WeekDay([11,3,1998]);
"Wed"
gap> StringDate([11,3,1998]);
"11-Mar-1998"
\endexample

\>HMSMSec( <msec> ) F

converts a number <msec> of milliseconds into a list
`[<hour>,<min>,<sec>,<milli>]'.


\>SecHMSM( <hmsm> ) F

is the reverse of `HMSMSec'.


\>StringTime( <time> ) F

converts <time> (given as a number of milliseconds or a list
`[<hour>, <min>, <sec>, <milli>]') to a readable string.


\beginexample
gap> HMSMSec(Factorial(10));
[ 1, 0, 28, 800 ]
gap> SecHMSM([1,10,5,13]);
4205013
gap> StringTime([1,10,5,13]);
" 1:10:05.013"
\endexample

\>SecondsDMYhms( <DMYhms> ) F

returns the number of seconds from 01-Jan-1970, 00:00:00,  to the time 
given by <DMYhms>.
<DMYhms> must be a list of the form
`[<day>,<month>,<year>,<hour>,<minute>,<second>]'. The remarks on the
Gregorian calendar in the section on "DayDMY" apply here as well. The
last three arguments must lie in the appropriate ranges. 


\>DMYhmsSeconds( <secs> ) F

This is the inverse function to "SecondsDMYhms".


\beginexample
gap> SecondsDMYhms([ 9, 9, 2001, 1, 46, 40 ]);
1000000000
gap> DMYhmsSeconds(-1000000000);
[ 24, 4, 1938, 22, 13, 20 ]
\endexample

%%%%%%%%%%%%%%%%%%%%%%%%%%%%%%%%%%%%%%%%%%%%%%%%%%%%%%%%%%%%%%%%%%%%%%%%%%%%%
%%
%E

