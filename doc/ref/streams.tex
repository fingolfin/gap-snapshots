% This file was created automatically from streams.msk.
% DO NOT EDIT!
%%%%%%%%%%%%%%%%%%%%%%%%%%%%%%%%%%%%%%%%%%%%%%%%%%%%%%%%%%%%%%%%%%%%%%%%%
%%
%W  streams.msk               GAP documentation              Frank Celler
%W                                                       Martin Schoenert
%W                                                         & Steve Linton
%%
%H  @(#)$Id: streams.msk,v 1.16 2002/06/24 14:03:16 gap Exp $
%%
%Y  Copyright 1997,  Lehrstuhl D fuer Mathematik,  RWTH Aachen,   Germany
%Y  Copyright 2000,  St Andrews
%%
%%  This file contains the description of streams.
%%


%%%%%%%%%%%%%%%%%%%%%%%%%%%%%%%%%%%%%%%%%%%%%%%%%%%%%%%%%%%%%%%%%%%%%%%%%
\Chapter{Streams}

*Streams* provide flexible access to {\GAP}'s input and output
processing. An *input stream* takes characters from some source and
delivers them to {\GAP} which *reads* them from the stream.  When an
input stream has delivered all characters it is at `end-of-stream'.  An
*output stream* receives characters from {\GAP} which *writes* them to
the stream, and delivers them to some destination.

A major use of streams is to provide efficient and flexible access to
files.  Files can be read and written using `Read' and `AppendTo',
however the former only allows a complete file to be read as {\GAP}
input and the latter imposes a high time penalty if many small pieces of
output are written to a large file. Streams allow input files in other
formats to be read and processed, and files to be built up efficiently
from small pieces of output. Streams may also be used for other purposes, 
for example to read from and print to {\GAP} strings, or to read input
directly from the user.

Any stream is either a *text stream*, which translates the `end-of-line'
character (`{'\\n'}') to or from the system's representation of
`end-of-line' (e.g., <new-line> under UNIX, <carriage-return> under
MacOS, <carriage-return>-<new-line> under DOS), or a *binary stream*,
which does not translate the `end-of-line' character. The processing of
other unprintable characters by text streams is undefined. Binary streams
pass them unchanged.

Note that binary streams are *@not yet implemented@*.

Whereas it is  cheap  to append  to a  stream, streams do  consume system
resources, and only a  limited number can  be open at any time, therefore
it is   necessary   to close   a  stream  as   soon as   possible  using
`CloseStream' described in Section~"CloseStream".   If creating  a stream
failed then `LastSystemError' (see "LastSystemError")  can be used to get
information about the failure. 




%%%%%%%%%%%%%%%%%%%%%%%%%%%%%%%%%%%%%%%%%%%%%%%%%%%%%%%%%%%%%%%%%%%%%%%%%
\Section{Categories for Streams and the StreamsFamily}

\>IsStream( <obj> ) C

Streams are {\GAP} objects and all open streams, input, output, text
and binary, lie in this category.


\>IsClosedStream( <obj> ) C

When a stream is closed, its type changes to lie in
'IsClosedStream'. This category is used to install methods that trap
accesses to closed streams.


\>IsInputStream( <obj> ) C

All input streams lie in this category, and support input
operations such as `ReadByte' (see "Operations for Input Streams")


\>IsInputTextStream( <obj> ) C

All *text* input streams lie in this category. They translate new-line
characters read.


\>IsInputTextNone( <obj> ) C

It is convenient to use a category to distinguish dummy streams
(see "Dummy Streams") from others. Other distinctions are usually
made using representations


\>IsOutputStream( <obj> ) C

All output streams lie in this category and support basic
operations such as `WriteByte' (see "Operations for Output Streams")


\>IsOutputTextStream( <obj> ) C

All *text* output streams lie in this category and translate
new-line characters on output.


\>IsOutputTextNone( <obj> ) C

It is convenient to use a category to distinguish dummy streams
(see "Dummy Streams") from others. Other distinctions are usually
made using representations


\>`StreamsFamily' V

All streams lie in the `StreamsFamily'



%%%%%%%%%%%%%%%%%%%%%%%%%%%%%%%%%%%%%%%%%%%%%%%%%%%%%%%%%%%%%%%%%%%%%%%%%
\Section{Operations applicable to All Streams}

\>CloseStream( <stream> ) O

In order  to preserve system resources  and to flush output streams every
stream should  be  closed  as soon   as  it is   no longer   used using
`CloseStream'.

It is an error to  try to read  characters from or  write characters to a
closed  stream.   Closing a  stream tells  the {\GAP}   kernel and/or the
operating system kernel  that the file is  no longer needed.  This may be
necessary  because  the {\GAP} kernel  and/or  the  operating  system may
impose a limit on how many streams may be open simultaneously.



\>FileDescriptorOfStream( <stream> ) O

returns the UNIX file descriptor of the underlying file. This is mainly
useful for the `UNIXSelect' function call (see~"UNIXSelect"). This is
as of now only available on UNIX-like operating systems and only for
streams to local processes and local files.



\>UNIXSelect( <inlist>, <outlist>, <exclist>, <timeoutsec>, <timeoutusec> ) F

makes the UNIX C-library function `select' accessible from {\GAP}
for streams. The functionality is as described in the man page (see
`man select'). The first three arguments must be lists containing
UNIX file descriptors (integers) for streams. They can be obtained via
`FileDescriptorOfStream' (see~"FileDescriptorOfStream") for streams
to local processes and to local files. The argument <timeoutsec> is a
timeout in seconds as in the `struct timeval' on the C level. The argument
<timeoutusec> is
analogously in microseconds. The total timeout is the sum of both. If
one of those timeout arguments is not a small integer then no timeout is
applicable (`fail' is allowed for the timeout arguments).

The return value is the number of streams that are ready, this may be
0 if a timeout was specified. All file descriptors in the three lists
that are not yet ready are replaced by `fail' in this function. So 
the lists are changed!

This function is not available on the Macintosh architecture and is
only available if your operating system has `select', which is detected
during compilation of {\GAP}.


%%%%%%%%%%%%%%%%%%%%%%%%%%%%%%%%%%%%%%%%%%%%%%%%%%%%%%%%%%%%%%%%%%%%%%%%%
\Section{Operations for Input Streams}

Three operations normally used to read files: `Read', `ReadAsFunction'
and `ReadTest' can also be used to read {\GAP} input from a
stream. The input is immediately parsed and executed. When reading
from a stream <str>, the {\GAP} kernel generates calls to
`ReadLine(<str>)' to supply text to the parser. 

Three further operations: `ReadByte', `ReadLine' and `ReadAll', support
reading characters from an input stream without parsing them. This can be
used to read data in any format and process it in {\GAP}.

Additional operations for input streams support detection of end of
stream, and (for those streams for which it is appropriate) random access 
to the data.


\>Read( <input-text-stream> )!{for streams} O

reads the input-text-stream as  input  until `end-of-stream' occurs.  See
"File Operations" for details.


\>ReadAsFunction( <input-text-stream> )!{for streams} O

reads the input-text-stream as function and returns this function. See
"File Operations" for details.

\>ReadTest( <input-text-stream> )!{for streams} O

reads the input-text-stream as  test input until `end-of-stream' occurs.
See "File Operations" for details.

*Example*

\beginexample
gap> # a function with local `a' does not change the global one
gap> a := 1;;
gap> i := InputTextString( "local a; a := 10; return a*10;" );;
gap> ReadAsFunction(i)();
100
gap> a;
1
\endexample

\beginexample
gap> # reading it via `Read' does
gap> i := InputTextString( "a := 10;" );;
gap> Read(i);
gap> a;
10
\endexample


\>ReadByte( <input-stream> ) O

`ReadByte' returns  one character (returned  as  integer) from  the input
stream <stream-in>.  `ReadByte' returns `fail' if there is no character
available, in particular if it is at the end of a file.

If <stream-in> is the input stream of  a input/output process, `ReadByte'
may also return `fail' if no byte is currently available.

`ReadByte' is the basic operation for input streams. If a `ReadByte'
method is installed for a user-defined type of stream which does
not block, then all the other
input stream operations will work (although possibly not at peak
efficiency).

`ReadByte' will wait (block) until a byte is available. For
instance if the stream is a connection to another process, it will
wait for the process to output a byte.


\>ReadLine( <input-stream> ) O

`ReadLine' returns one line (returned as string *with* the newline) from
the input stream <input-stream>.  `ReadLine' reads in the input until  a
newline is read or the end-of-stream is encountered.

If <input-stream> is the input stream of a input/output process, `ReadLine'
may also return `fail' or return an incomplete line if the other
process has not yet written any more. It will always wait (block) for at
least one byte to be available, but will then return as much input
as is available, up to a limit of one  line

A default method is supplied for `ReadLine' which simply calls `ReadByte'
repeatedly. This is only safe for streams that cannot block. The kernel 
uses calls to `ReadLine' to supply input to the
parser when reading from a stream.


\>ReadAll( <input-stream> ) O
\>ReadAll( <input-stream> , <limit> ) O

`ReadAll' returns all characters as string from the input stream
<stream-in>.  It waits (blocks) until at least one
character is available from the stream, or until there is evidence
that no characters will ever be available again. This last indicates
that the stream is at end-of-stream.
Otherwise, it reads as much input as it can from the stream without
blocking further and returns it to the user. If the stream is
already at end of file, so that no bytes are available, `fail' is
 returned. In the case of a file
stream connected to a normal file (not a pseudo-tty or named pipe
or similar), all the bytes should be immediately available and
this function will read the remainder of the file.

With a second argument, at most <limit> bytes will be
returned. Depending on the stream a bounded number of additional bytes
may have been read into an internal buffer.  

A default method is supplied for `ReadAll' which simply calls `ReadLine'
repeatedly. This is only really safe for streams which cannot
block. Other streams should install a method for ReadAll




*Example*

\beginexample
gap> i := InputTextString( "1Hallo\nYou\n1" );;
gap> ReadByte(i);
49
gap> CHAR_INT(last);
'1'
gap> ReadLine(i);
"Hallo\n"
gap> ReadLine(i);
"You\n"
gap> ReadLine(i);
"1"
gap> ReadLine(i);
fail
gap> ReadAll(i);
""
gap> RewindStream(i);;
gap> ReadAll(i);
"1Hallo\nYou\n1"
\endexample


\>IsEndOfStream( <input-stream> ) O

`IsEndOfStream' returns `true' if the input stream is at <end-of-stream>,
and `false' otherwise.  Note   that `IsEndOfStream' might  return `false'
even if the next `ReadByte' fails.


\>PositionStream( <input-stream> ) O

Some input streams, such as string streams and file streams attached to
disk files, support a form of random access by way of the operations
`PositionStream', `SeekPositionStream' and
`RewindStream'. `PositionStream' returns a non-negative integer denoting
the current position in the stream (usually the number of characters
*before* the next one to be read.

If this is not possible, for example for an input stream attached to
standard input (normally the keyboard), then `fail' is returned


\>RewindStream( <input-stream> ) O

`RewindStream' attempts to return an input stream to its starting
condition, so that all the same characters can be read again. It returns
`true' if the rewind succeeds and `fail' otherwise

A default method implements RewindStream using `SeekPositionStream'.


\>SeekPositionStream( <input-stream>, <pos> ) O

`SeekPositionStream' attempts to rewind or wind forward an input stream
to the specified position. This is not possible for all streams. It
returns `true' if the seek is successful and `fail' otherwise.



%%%%%%%%%%%%%%%%%%%%%%%%%%%%%%%%%%%%%%%%%%%%%%%%%%%%%%%%%%%%%%%%%%%%%%%%%
\Section{Operations for Output Streams}

\>WriteByte( <output-stream>, <byte> ) O

writes the  next  character  (given  as *integer*)  to the  output stream
<output-stream>.  The function  returns `true' if  the write succeeds and
`fail' otherwise.

`WriteByte' is the basic operation for output streams. If a `WriteByte'
method is installed for a user-defined type of stream, then all the other
output stream operations will work (although possibly not at peak
efficiency).


\>WriteLine( <output-stream>, <string> ) O

appends  <string> to <output-stream>.   A  final newline is written.
The function returns `true' if the write succeeds and `fail' otherwise.

A default method is installed which implements `WriteLine' by repeated
calls to `WriteByte'.


\>WriteAll( <output-stream>, <string> ) O

appends  <string> to <output-stream>.   No final  newline is written.
The function returns `true' if the write succeeds and `fail' otherwise.
It will block as long as necessary for the write operation to
complete (for example for a child process to clear its input buffer )

A default method is installed which implements `WriteAll' by repeated
calls to `WriteByte'.

When printing or appending to a stream (using `PrintTo', or `AppendTo' or
when logging to a stream), the kernel generates a call to `WriteAll' for
each line output.



*Example*

\beginexample
gap> str := "";; a := OutputTextString(str,true);;
gap> WriteByte(a,INT_CHAR('H'));
true
gap> WriteLine(a,"allo");
true
gap> WriteAll(a,"You\n");
true
gap> CloseStream(a);
gap> Print(str);
Hallo
You
\endexample

\>PrintTo( <output-stream>, <arg1>, ... )!{for streams} F
\>AppendTo( <output-stream>, <arg1>, ... )!{for streams} F

These functions work like `Print', except that the output is
appended to the output stream <output-stream>.

*Example*

\beginexample
gap> str := "";; a := OutputTextString(str,true);;
gap> AppendTo( a, (1,2,3), ":", Z(3) );
gap> CloseStream(a);
gap> Print( str, "\n" );
(1,2,3):Z(3)
\endexample

\>LogTo( <stream> )!{for streams} O

causes the subsequent interaction to  be  logged  to  the  output  stream
<stream>. It works in precisely  the  same  way  as  it  does  for  files
(see~"LogTo").


\>InputLogTo( <stream> )!{for streams} O

causes the subsequent input to be logged to the output  stream  <stream>.
It works just like it does for files (see~"InputLogTo").


\>OutputLogTo( <stream> )!{for streams} O

causes the subsequent output to be logged to the output stream  <stream>.
It works just like it does for files (see~"OutputLogTo").




When  text is being  sent to an output text   stream via `PrintTo', `AppendTo',
`LogTo', etc., it  is, by default formatted just  as it would  be were it being
printed to  the screen. Thus,  it is broken into lines  of reasonable length at
(where possible) sensible places, lines containing elements of lists or records
are indented, and so forth.  This is appropriate if the output is eventually to
be viewed by a human, and harmless if it to passed  as input to {\GAP}, but may
be unhelpful if the output is to be  passed as input to  another program. It is
possible    to turn      off   this behaviour     for   a   stream   using  the
`SetPrintFormattingStatus' operation, and to test whether it is on or off using
`PrintFormattingStatus'.

\>SetPrintFormattingStatus( <stream>, <newstatus> ) O

sets whether output sent to the output  stream  <stream>  via  `PrintTo',
`AppendTo', etc. (but not `WriteByte', `WriteLine' or `WriteAll') will be
formatted with line  breaks  and  indentation.  If  the  second  argument
<newstatus> is `true' then output will be so formatted,  and  if  `false'
then it will not. If the stream is not a text stream, only `false'
is allowed.


\>PrintFormattingStatus( <stream> ) O

returns `true' if output sent to the output text stream <stream> via
`PrintTo', `AppendTo', etc.  (but not `WriteByte', `WriteLine' or
`WriteAll') will be formatted with line breaks and indentation, and
`false' otherwise (see~"SetPrintFormattingStatus"). For non-text
streams, it returns `false'.



*Example*
\beginexample
gap> s := "";; str := OutputTextString(s,false);;
gap> PrintTo(str,Primes{[1..30]});
gap> s;
"[ 2, 3, 5, 7, 11, 13, 17, 19, 23, 29, 31, 37, 41, 43, 47, 53, 59, 61, 67, 71,\
 \n  73, 79, 83, 89, 97, 101, 103, 107, 109, 113 ]"
gap> Print(s,"\n");
[ 2, 3, 5, 7, 11, 13, 17, 19, 23, 29, 31, 37, 41, 43, 47, 53, 59, 61, 67, 71, 
  73, 79, 83, 89, 97, 101, 103, 107, 109, 113 ]
gap> SetPrintFormattingStatus(str, false);
gap> PrintTo(str,Primes{[1..30]});
gap> s;
"[ 2, 3, 5, 7, 11, 13, 17, 19, 23, 29, 31, 37, 41, 43, 47, 53, 59, 61, 67, 71,\
 \n  73, 79, 83, 89, 97, 101, 103, 107, 109, 113 ][ 2, 3, 5, 7, 11, 13, 17, 19\
, 23, 29, 31, 37, 41, 43, 47, 53, 59, 61, 67, 71, 73, 79, 83, 89, 97, 101, 103\
, 107, 109, 113 ]"
gap> Print(s,"\n");
[ 2, 3, 5, 7, 11, 13, 17, 19, 23, 29, 31, 37, 41, 43, 47, 53, 59, 61, 67, 71, 
  73, 79, 83, 89, 97, 101, 103, 107, 109, 113 ][ 2, 3, 5, 7, 11, 13, 17, 19, 2\
3, 29, 31, 37, 41, 43, 47, 53, 59, 61, 67, 71, 73, 79, 83, 89, 97, 101, 103, 1\
07, 109, 113 ]
\endexample

%%%%%%%%%%%%%%%%%%%%%%%%%%%%%%%%%%%%%%%%%%%%%%%%%%%%%%%%%%%%%%%%%%%%%%%%%
\Section{File Streams}

File streams  are  streams associated with  files.  An  input file stream
reads  the characters  it delivers from  a  file,  an output  file stream
prints the characters it receives to a file.  The following functions can
be used to create such streams.  They return `fail' if an error occurred,
in this case `LastSystemError' (see "LastSystemError") can be used to get
information about the error.

\>InputTextFile( <name-file> ) O

`InputTextFile( <name-file> )' returns an input stream in the category
`IsInputTextStream' that delivers the characters from the file
<name-file>.




\>OutputTextFile( <name-file>, <append> ) O

`OutputTextFile( <name-file>, <append> )' returns an output stream in the
category `IsOutputTextFile' that writes received characters to the file
<name-file>.  If <append> is `false', then the file is emptied first,
otherwise received characters are added at the end of the list.




*Example*

\beginexample
gap> # use a temporary directory
gap> name := Filename( DirectoryTemporary(), "test" );;
gap> # create an output stream, append output, and close again
gap> output := OutputTextFile( name, true );;
gap> AppendTo( output, "Hallo\n", "You\n" );
gap> CloseStream(output);
gap> # create an input, print complete contents of file, and close
gap> input := InputTextFile(name);;
gap> Print( ReadAll(input) );
Hallo
You
gap> CloseStream(input);
gap> # append a single line
gap> output := OutputTextFile( name, true );;
gap> AppendTo( output, "AppendLine\n" );
gap> # close output stream to flush the output
gap> CloseStream(output);
gap> # create an input, print complete contents of file, and close
gap> input := InputTextFile(name);;
gap> Print( ReadAll(input) );
Hallo
You
AppendLine
gap> CloseStream(input);
\endexample

%%%%%%%%%%%%%%%%%%%%%%%%%%%%%%%%%%%%%%%%%%%%%%%%%%%%%%%%%%%%%%%%%%%%%%%%%
\Section{User Streams}

The following two commands create streams which accept characters
from, or deliver characters to, the user, via the keyboard or the {\GAP} session
display.

\>InputTextUser( ) F

returns an input text stream which delivers characters typed by the user
(or from the standard input device if it has been redirected). In normal
circumstances, characters are delivered one by one as they are typed,
without waiting until the end of a line. No prompts are printed.



\>OutputTextUser( ) F

returns an output stream which delivers characters to the user's display
(or the standard output device if it has been redirected). Each character
is delivered immediately it is written, without waiting for a full line
of output. Text written in this way is *not* written to the session log
(see "LogTo").




%%%%%%%%%%%%%%%%%%%%%%%%%%%%%%%%%%%%%%%%%%%%%%%%%%%%%%%%%%%%%%%%%%%%%%%%%
\Section{String Streams}

String streams   are streams associated  with   strings.  An input string
stream reads  the characters it delivers  from a string, an output string
stream  appends the characters  it receives  to  a string.  The following
functions can be used to create such streams.

\>InputTextString( <string> ) O

`InputTextString( <string> )'returns an input stream that delivers the
characters from the string <string>.  The <string> is not changed when
reading characters from it and changing the <string> after the call to
`InputTextString' has no influence on the input stream.




\>OutputTextString( <list>, <append> ) O

returns an output stream that puts all received characters into the list
<list>.  If <append> is `false', then the list is emptied first,
otherwise received characters are added at the end of the list. 




*Example*

\beginexample
gap> # read input from a string
gap> input := InputTextString( "Hallo\nYou\n" );;
gap> ReadLine(input);
"Hallo\n"
gap> ReadLine(input);
"You\n"
gap> # print to a string
gap> str := "";;
gap> out := OutputTextString( str, true );;
gap> PrintTo( out, 1, "\n", (1,2,3,4)(5,6), "\n" );
gap> CloseStream(out);
gap> Print( str );
1
(1,2,3,4)(5,6)
\endexample

%%%%%%%%%%%%%%%%%%%%%%%%%%%%%%%%%%%%%%%%%%%%%%%%%%%%%%%%%%%%%%%%%%%%%%%%%
\Section{Input-Output Streams}

Input-output streams capture bidirectional 
communications between {\GAP} and another process, either locally
or (@as yet unimplemented@) remotely.

Such streams support the basic operations of both input and output 
streams. They should provide some buffering, allowing output data to be
written to the stream, even when input data is waiting to be read,
but the amount of this buffering is operating system dependent,
and the user should take care not to get too far ahead in writing, or 
behind in reading, or deadlock may occur.



\>IsInputOutputStream( <obj> ) C

`IsInputOutputStream' is the Category of Input-Output Streams; it returns
`true' if the <obj> is an input-output stream and `false' otherwise.



At present the only type of Input-Output streams that are
implemented provide communication with a local child process,
using a pseudo-tty.

Like other streams, write operations are blocking, read operations
will block to get the first character, but not thereafter. 

As far as possible, no translation is done on characters written
to, or read from the stream, and no control characters have special
effects, but the details of particular pseudo-tty implementations 
may effect this. 



\>InputOutputLocalProcess( <dir>, <executable>, <args> ) F


starts up a slave process, whose executable file is <executable>, with
``command line'' arguments <args> in the directory <dir>. (Suitable 
choices for <dir> are `DirectoryCurrent()' or `DirectoryTemporary()'
(see Section~"Directories"); `DirectoryTemporary()' may be a good choice
when <executable> generates output files that it doesn't itself remove
afterwards.) 
`InputOutputLocalProcess' returns an InputOutputStream object. Bytes
written to this stream are received by the slave process as if typed
at a terminal on standard input. Bytes written to standard output
by the slave process can be read from the stream. 

When the stream is closed, the signal SIGTERM is delivered to the child
process, which is expected to exit.



\beginexample
gap> d := DirectoryCurrent();
dir("./")
gap> f := Filename(DirectoriesSystemPrograms(), "rev");
"/usr/bin/rev"
gap> s := InputOutputLocalProcess(d,f,[]);
< input/output stream to rev >
gap> WriteLine(s,"The cat sat on the mat");
true
gap> Print(ReadLine(s));
tam eht no tas tac ehT
gap> x := ListWithIdenticalEntries(10000,'x');;
gap> ConvertToStringRep(x);
gap> WriteLine(s,x);
true
gap> WriteByte(s,INT_CHAR('\n'));
true
gap> y := ReadAll(s);;
gap> Length(y);
4095
gap> CloseStream(s);
gap> s;
< closed input/output stream to rev >
\endexample

\>ReadAllLine( <iostream>[, <nofail>][, <IsAllLine>] ) O

For an input/output stream <iostream> `ReadAllLine' reads until a newline
character if any input is found or returns `fail' if no input  is  found,
i.e.~if any input is found `ReadAllLine' is non-blocking.

If the argument <nofail> (which must be `false' or  `true')  is  provided
and it is set to `true' then `ReadAllLine' will wait, if  necessary,  for
input and never return `fail'.

If the argument <IsAllLine> (which must be a function that takes a string
argument and returns either  `true'  or  `false')  then  it  is  used  to
determine what  constitutes  a  whole  line.  The  default  behaviour  is
equivalent to passing the function

\begintt
line -> 0 < Length(line) and line[Length(line)] = '\n'
\endtt

for the <IsAllLine> argument. The purpose of the <IsAllLine> argument  is
to cater for the case where the input being  read  is  from  an  external
process that writes a ``prompt'' for data that does not terminate with  a
newline.

If the first argument is an input stream but not an  input/output  stream
then `ReadAllLine' behaves as if `ReadLine'  was  called  with  just  the
first argument and any additional arguments are ignored.



%%%%%%%%%%%%%%%%%%%%%%%%%%%%%%%%%%%%%%%%%%%%%%%%%%%%%%%%%%%%%%%%%%%%%%%%%
\Section{Dummy Streams}

The following  two commands create  dummy streams  which will consume all
characters and never deliver one.

\>InputTextNone( ) F

returns a dummy input text stream, which delivers no characters, i.e., it
is always at end of stream.  Its main use is for calls to `Process' (see
"Process") when the started program does not read anything.



\>OutputTextNone( ) F

returns a dummy output stream, which discards all received characters. 
Its main use is for calls to `Process' (see~"Process") when the started
program does not write anything.



%%%%%%%%%%%%%%%%%%%%%%%%%%%%%%%%%%%%%%%%%%%%%%%%%%%%%%%%%%%%%%%%%%%%%%%%%
\Section{Handling of Streams in the Background}

This section describes a feature of the {\GAP} kernel that can be used
to handle pending streams somehow ``in the background''. This is currently
not available on the Macintosh architecture and only on operating
systems that have `select'.

Right before {\GAP} reads a keypress from the keyboard it calls a little
subroutine that can handle streams that are ready to be read or ready to
be written. This means that {\GAP} can handle these streams during
user input on the command line. Note that this does not work when {\GAP}
is in the middle of some calculation.

This feature is used in the following way. One can install handler
functions for reading or writing streams. This is done via:

\>InstallCharReadHookFunc( <stream>, <mode>, <func> ) F

installs the function <func> as a handler function for the stream
<stream>. The argument <mode> decides, for what operations on the
stream this function is installed. <mode> must be a string, in which
a letter `r' means ``read'', `w' means ``write'' and `x' means
``exception'', according to the `select' function call in the UNIX
C-library (see `man select' and "UNIXSelect"). More than one letter 
is allowed in <mode>. As described above the function is called
in a situation when {\GAP} is reading a character from the keyboard.
Handler functions should not use much time to complete.

This functionality does not work on the Macintosh architecture and
only works if the operating system has a `select' function.



Handlers can be removed via:

\>UnInstallCharReadHookFunc( <stream>, <func> ) F

uninstalls the function <func> as a handler function for the stream
<stream>. All instances are deinstalled, regardless of the mode
of operation (read, write, exception).

This functionality does not work on the Macintosh architecture and
only works if the operating system has a `select' function.



Note that handler functions must not return anything and get one integer
argument, which refers to an index in one of the following arrays
(according to whether the function was installed for input, output or
exceptions on the stream). Handler functions usually should not output
anything on the standard output because this ruins the command line
during command line editing.

%%%%%%%%%%%%%%%%%%%%%%%%%%%%%%%%%%%%%%%%%%%%%%%%%%%%%%%%%%%%%%%%%%%%%%%%%
%%
%E  streams.tex . . . . . . . . . . . . . . . . . . . . . . . . ends here
