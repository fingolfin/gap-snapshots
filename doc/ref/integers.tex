% This file was created automatically from integers.msk.
% DO NOT EDIT!
%%%%%%%%%%%%%%%%%%%%%%%%%%%%%%%%%%%%%%%%%%%%%%%%%%%%%%%%%%%%%%%%%%%%%%%%%%%%%
%%
%A  integers.msk                GAP documentation            Martin Schoenert
%A                                                           Alexander Hulpke
%%
%A  @(#)$Id: integers.msk,v 1.20.2.1 2005/05/09 08:19:58 gap Exp $
%%
%Y  (C) 1998 School Math and Comp. Sci., University of St.  Andrews, Scotland
%Y  Copyright (C) 2002 The GAP Group
%%
\Chapter{Integers}

One of the most fundamental datatypes in every programming language is
the integer type.  {\GAP} is no exception.

{\GAP} integers are entered as a sequence of decimal digits
optionally preceded by a `+' sign for positive integers or a `-' sign for
negative integers.
The size of integers in {\GAP} is only limited by the amount of available
memory, so you can compute with integers having thousands of digits.

\beginexample
gap> -1234;
-1234
gap> 123456789012345678901234567890123456789012345678901234567890;
123456789012345678901234567890123456789012345678901234567890
\endexample


Many more functions that are mainly related to the prime residue group of
integers modulo an integer are described in chapter~"Number Theory",
and functions dealing with combinatorics can be found
in chapter~"Combinatorics".


\>`Integers' V
\>`PositiveIntegers' V
\>`NonnegativeIntegers' V

These global variables represent the ring of integers and the semirings
of positive and nonnegative integers, respectively.


\beginexample
gap> Size( Integers ); 2 in Integers;
infinity
true
\endexample

\>IsIntegers( <obj> ) C
\>IsPositiveIntegers( <obj> ) C
\>IsNonnegativeIntegers( <obj> ) C

are the defining categories for the domains `Integers',
`PositiveIntegers', and `NonnegativeIntegers'.


\beginexample
gap> IsIntegers( Integers );  IsIntegers( Rationals );  IsIntegers( 7 );
true
false
false
\endexample

`Integers' is a subset of `Rationals', which is a subset of `Cyclotomics'.
See Chapter~"Cyclotomic Numbers" for arithmetic operations and comparison of
integers.


%%%%%%%%%%%%%%%%%%%%%%%%%%%%%%%%%%%%%%%%%%%%%%%%%%%%%%%%%%%%%%%%%%%%%%%%%%%%%
\Section{Elementary Operations for Integers}

\>IsInt( <obj> ) C

Every rational integer lies in the category `IsInt',
which is a subcategory of `IsRat' (see~"Rational Numbers").


\>IsPosInt( <obj> ) C

Every positive integer lies in the category `IsPosInt'.


\>Int( <elm> ) A

`Int' returns an integer <int> whose meaning depends on the type
of <elm>.

If <elm> is a rational number (see~"Rational Numbers") then <int> is the
integer part of the quotient of numerator and denominator of <elm>
(see~"QuoInt").

If <elm> is an element of a finite prime field
(see Chapter~"Finite Fields") then <int> is the smallest
nonnegative integer such that `<elm> = <int> \* One( <elm> )'.

If <elm> is a string (see Chapter~"Strings and Characters") consisting of
digits `{'0'}', `{'1'}', $\ldots$, `{'9'}'
and `{'-'}' (at the first position) then <int> is the integer
described by this string.
The operation `String' (see~"String") can be used to compute a string for
rational integers, in fact for all cyclotomics.

\beginexample
gap> Int( 4/3 );  Int( -2/3 );
1
0
gap> int:= Int( Z(5) );  int * One( Z(5) );
2
Z(5)
gap> Int( "12345" );  Int( "-27" );  Int( "-27/3" );
12345
-27
fail
\endexample



\>IsEvenInt( <n> ) F

tests if the integer <n> is divisible by 2.


\>IsOddInt( <n> ) F

tests if the integer <n> is not divisible by 2.



\>AbsInt( <n> ) F

`AbsInt' returns the absolute value of the integer <n>, i.e., <n> if <n>
is positive, -<n> if <n> is negative and 0 if <n> is 0.

`AbsInt' is a special case of the general operation `EuclideanDegree'
see~"EuclideanDegree").


\index{absolute value of an integer}
See also "AbsoluteValue".
\beginexample
gap> AbsInt( 33 );
33
gap> AbsInt( -214378 );
214378
gap> AbsInt( 0 );
0
\endexample

\>SignInt( <n> ) F

`SignInt' returns the sign of the integer <n>, i.e., 1 if <n> is
positive, -1 if <n> is negative and 0 if <n> is 0.


\index{sign!of an integer}
\beginexample
gap> SignInt( 33 );
1
gap> SignInt( -214378 );
-1
gap> SignInt( 0 );
0
\endexample

\>LogInt( <n>, <base> ) F

`LogInt'   returns  the  integer part  of  the logarithm of  the positive
integer  <n> with  respect to   the positive integer   <base>, i.e.,  the
largest positive integer <exp> such  that $base^{exp} \leq  n$.  `LogInt'
will signal an error if either <n> or <base> is not positive.

For <base> $2$ this is very efficient because the internal binary
representation of the integer is used. 


\beginexample
gap> LogInt( 1030, 2 );
10
gap> 2^10;
1024
gap> LogInt( 1, 10 );
0
\endexample

\>RootInt( <n> ) F
\>RootInt( <n>, <k> ) F

`RootInt' returns the integer part of the <k>th root  of the integer <n>.
If the optional integer argument <k> is not given it defaults to 2, i.e.,
`RootInt' returns the integer part of the square root in this case.

If  <n> is positive, `RootInt' returns  the  largest positive integer $r$
such that $r^k \leq n$.  If <n>  is negative and  <k>  is  odd  `RootInt'
returns `-RootInt( -<n>,  <k> )'.  If  <n> is negative   and <k> is  even
`RootInt' will cause an error.  `RootInt' will also cause an error if <k>
is 0 or negative.


\index{root!of an integer}\index{square root!of an integer}
\beginexample
gap> RootInt( 361 );
19
gap> RootInt( 2 * 10^12 );
1414213
gap> RootInt( 17000, 5 );
7
gap> 7^5;
16807
\endexample

\>SmallestRootInt( <n> ) F

`SmallestRootInt' returns the smallest root of the integer <n>.

The  smallest  root of an  integer $n$  is  the  integer $r$  of smallest
absolute  value for which  a  positive integer $k$ exists such  that $n =
r^k$.


\index{root!of an integer, smallest}
\beginexample
gap> SmallestRootInt( 2^30 );
2
gap> SmallestRootInt( -(2^30) );
-4
\endexample

Note that $(-2)^{30} = +(2^{30})$.

\beginexample
gap> SmallestRootInt( 279936 );
6
gap> LogInt( 279936, 6 );
7
gap> SmallestRootInt( 1001 );
1001
\endexample

\>Random( Integers )!{for integers}

`Random' for integers returns
pseudo random integers between -10 and
10 distributed according to a binomial distribution.
To  generate  uniformly  distributed  integers from   a  range,  use the
construct 'Random( [ <low> .. <high> ] )'. (Also see~"Random".)



%%%%%%%%%%%%%%%%%%%%%%%%%%%%%%%%%%%%%%%%%%%%%%%%%%%%%%%%%%%%%%%%%%%%%%%%%%%%%
\Section{Quotients and Remainders}

\>QuoInt( <n>, <m> ) F

`QuoInt' returns the integer part of the quotient of its integer
operands.

If <n> and <m> are positive `QuoInt( <n>, <m> )' is the largest
positive integer <q> such that $<q> \* <m> \le <n>$.
If <n> or <m> or both are negative the absolute value of the integer part
of the quotient is the quotient of the absolute values of <n> and <m>,
and the sign of it is the product of the signs of <n> and <m>.

`QuoInt' is used in a method for the general operation
`EuclideanQuotient' (see~"EuclideanQuotient").


\index{integer part of a quotient}
\beginexample
gap> QuoInt(5,3);  QuoInt(-5,3);  QuoInt(5,-3);  QuoInt(-5,-3);
1
-1
-1
1
\endexample

\>BestQuoInt( <n>, <m> ) F

`BestQuoInt' returns the best quotient <q> of the integers <n> and <m>.
This is the quotient such that `<n>-<q>*<m>' has minimal absolute value.
If there are two quotients whose remainders have the same absolute value,
then the quotient with the smaller absolute value is chosen.


\beginexample
gap> BestQuoInt( 5, 3 );  BestQuoInt( -5, 3 );
2
-2
\endexample

\>RemInt( <n>, <m> ) F

`RemInt' returns the remainder of its two integer operands.

If <m> is not equal to zero
`RemInt( <n>, <m> ) = <n> - <m> * QuoInt( <n>, <m> )'.
Note that the rules given for `QuoInt' imply that `RemInt( <n>, <m> )'
has the same sign as <n> and its absolute value is strictly less than the
absolute value of <m>.
Note also that `RemInt( <n>, <m> ) = <n> mod <m>' when both <n> and <m>
are nonnegative.
Dividing by 0 signals an error.

`RemInt' is used in a method for the general operation
`EuclideanRemainder' (see~"EuclideanRemainder").


\index{remainder of a quotient}
\beginexample
gap> RemInt(5,3);  RemInt(-5,3);  RemInt(5,-3);  RemInt(-5,-3);
2
-2
2
-2
\endexample

\>GcdInt( <m>, <n> ) F

`GcdInt' returns the greatest common divisor of its two integer operands
<m> and <n>, i.e., the greatest integer that divides both <m> and <n>.
The greatest common divisor is never negative, even if the arguments are.
We define `GcdInt( <m>, 0 ) = GcdInt( 0, <m> ) = AbsInt( <m> )' and
`GcdInt( 0, 0 ) = 0'.

`GcdInt' is a method used by the general function `Gcd' (see~"Gcd").


\beginexample
gap> GcdInt( 123, 66 );
3
\endexample

\>Gcdex( <m>, <n> ) F

returns a record <g> describing the extended greatest common divisor of
<m> and <n>.
The component `gcd' is this gcd,
the components `coeff1' and `coeff2' are integer cofactors such that
`<g>.gcd =  <g>.coeff1 * <m> + <g>.coeff2 * <n>',
and the components `coeff3' and `coeff4' are integer cofactors such that
`0 = <g>.coeff3 * <m> + <g>.coeff4 * <n>'.

If <m> and <n> both are nonzero, `AbsInt( <g>.coeff1 )' is less than or
equal to `AbsInt(<n>) / (2 * <g>.gcd)' and `AbsInt( <g>.coeff2 )' is less
than or equal to `AbsInt(<m>) / (2 * <g>.gcd)'.

If <m> or <n> or both are zero `coeff3' is `-<n> / <g>.gcd' and
`coeff4' is `<m> / <g>.gcd'.

The coefficients always form a unimodular matrix, i.e.,
the determinant `<g>.coeff1 * <g>.coeff4 - <g>.coeff3 * <g>.coeff2'
is $1$ or $-1$.


\beginexample
gap> Gcdex( 123, 66 );
rec( gcd := 3, coeff1 := 7, coeff2 := -13, coeff3 := -22, coeff4 := 41 )
\endexample

This means $3 = 7 * 123 - 13 * 66$, $0 = -22 * 123 + 41 * 66$.

\beginexample
gap> Gcdex( 0, -3 );
rec( gcd := 3, coeff1 := 0, coeff2 := -1, coeff3 := 1, coeff4 := 0 )
gap> Gcdex( 0, 0 );
rec( gcd := 0, coeff1 := 1, coeff2 := 0, coeff3 := 0, coeff4 := 1 )
\endexample

\>LcmInt( <m>, <n> ) F

returns the least common multiple of the integers <m> and <n>.

`LcmInt' is a method used by the general function `Lcm'.


\beginexample
gap> LcmInt( 123, 66 );
2706
\endexample

\>CoefficientsQadic( <i>, <q> ) F

returns the <q>-adic representation of the integer <i> as a list <l> of
coefficients where $i = \sum_{j=0} q^j \cdot l[j+1]$.


\>CoefficientsMultiadic( <ints>, <int> ) F

returns the multiadic expansion of the integer <int> modulo the integers
given in <ints> (in ascending order).
It returns a list of coefficients in the *reverse* order to that in <ints>.



\>ChineseRem( <moduli>, <residues> ) F

`ChineseRem' returns the combination   of   the  <residues>  modulo   the
<moduli>, i.e., the  unique integer <c>  from `[0..Lcm(<moduli>)-1]' such
that  `<c>  = <residues>[i]' modulo `<moduli>[i]'   for  all  <i>, if  it
exists.  If no such combination exists `ChineseRem' signals an error.

Such a combination does exist if and only if
`<residues>[<i>]=<residues>[<k>]'  mod `Gcd(<moduli>[<i>],<moduli>[<k>])'
for every pair <i>, <k>.  Note  that this implies that such a combination
exists if the  moduli  are pairwise relatively prime.  This is called the
Chinese remainder theorem.


\atindex{Chinese remainder}{@Chinese remainder}
\beginexample
gap> ChineseRem( [ 2, 3, 5, 7 ], [ 1, 2, 3, 4 ] );
53
gap> ChineseRem( [ 6, 10, 14 ], [ 1, 3, 5 ] );
103
\endexample
%notest
\beginexample
gap> ChineseRem( [ 6, 10, 14 ], [ 1, 2, 3 ] );
Error, the residues must be equal modulo 2 called from
<function>( <arguments> ) called from read-eval-loop
Entering break read-eval-print loop ...
you can 'quit;' to quit to outer loop, or
you can 'return;' to continue
brk> gap> 
\endexample

\>PowerModInt( <r>, <e>, <m> ) F

returns $r^e\pmod{m}$ for integers <r>,<e> and <m> ($e\ge 0$).
Note that using `<r> ^ <e> mod <m>' will generally  be slower,
because it can not reduce intermediate results the way `PowerModInt'
does but would compute `<r>^<e>' first and then reduce the result
afterwards.

`PowerModInt' is a method for the general operation `PowerMod'.




%%%%%%%%%%%%%%%%%%%%%%%%%%%%%%%%%%%%%%%%%%%%%%%%%%%%%%%%%%%%%%%%%%%%%%%%%%%%%
\Section{Prime Integers and Factorization}

\>`Primes' V

`Primes' is a strictly sorted list of the 168 primes less than 1000.

This is used in `IsPrimeInt' and `FactorsInt' to cast out small primes
quickly.


\beginexample
gap> Primes[1];
2
gap> Primes[100];
541
\endexample

\>IsPrimeInt( <n> ) F
\>IsProbablyPrimeInt( <n> ) F

`IsPrimeInt' returns `false'  if it can  prove that <n>  is composite and
`true' otherwise.
By  convention `IsPrimeInt(0) = IsPrimeInt(1) = false'
and we define `IsPrimeInt( -<n> ) = IsPrimeInt( <n> )'.

`IsPrimeInt' will return  `true' for every prime $n$.  `IsPrimeInt'  will
return `false' for all composite $n \< 10^{13}$ and for all composite $n$
that have   a factor  $p \<  1000$.   So for  integers $n    \< 10^{13}$,
`IsPrimeInt' is  a    proper primality test.    It  is  conceivable  that
`IsPrimeInt' may  return `true' for some  composite $n > 10^{13}$, but no
such $n$ is currently known.  So for integers $n > 10^{13}$, `IsPrimeInt'
is a  probable-primality test. `IsPrimeInt' will issue a
warning when its argument is probably prime but not a proven prime.
(The function `IsProbablyPrimeInt' will do the same calculations but not 
issue a warning.) The warning can be switched off by 
`SetInfoLevel( InfoPrimeInt, 0 );', the default level is $1$.

If composites that  fool `IsPrimeInt' do exist, they  would be extremely
rare, and finding one by pure chance might be less likely than finding a
bug in {\GAP}. We would appreciate being informed about any example of a
composite number <n> for which `IsPrimeInt' returns `true'.

`IsPrimeInt' is a deterministic algorithm, i.e., the computations involve
no random numbers, and repeated calls will always return the same result.
`IsPrimeInt' first   does trial divisions  by the  primes less than 1000.
Then it tests  that  $n$  is a   strong  pseudoprime w.r.t. the base   2.
Finally it  tests whether $n$ is  a Lucas pseudoprime w.r.t. the smallest
quadratic nonresidue of  $n$.  A better  description can be found in  the
comment in the library file `integer.gi'.

The time taken by `IsPrimeInt' is approximately proportional to the third
power  of  the number  of  digits of <n>.   Testing numbers  with several
hundreds digits is quite feasible.

`IsPrimeInt' is a method for the general operation `IsPrime'.

Remark: In future versions of {\GAP} we hope to change the definition of 
`IsPrimeInt' to return `true' only for proven primes (currently, we lack
a sufficiently good primality proving function). In applications, use
explicitly `IsPrimeInt' or `IsProbablePrimeInt' with this change in
mind.


\beginexample
gap> IsPrimeInt( 2^31 - 1 );
true
gap> IsPrimeInt( 10^42 + 1 );
false
\endexample

\>IsPrimePowerInt( <n> ) F

`IsPrimePowerInt' returns `true' if the integer <n>  is a prime power and
`false' otherwise.

$n$ is a *prime power* if there exists a prime $p$ and a positive integer
$i$ such that $p^i = n$.  If $n$ is negative the  condition is that there
must exist a negative prime $p$ and an odd positive integer $i$ such that
$p^i = n$.  1 and -1 are not prime powers.

Note    that `IsPrimePowerInt'      uses       `SmallestRootInt'     (see
"SmallestRootInt") and a probable-primality test (see "IsPrimeInt").


\beginexample
gap> IsPrimePowerInt( 31^5 );
true
gap> IsPrimePowerInt( 2^31-1 );  # 2^31-1 is actually a prime
true
gap> IsPrimePowerInt( 2^63-1 );
false
gap> Filtered( [-10..10], IsPrimePowerInt );
[ -8, -7, -5, -3, -2, 2, 3, 4, 5, 7, 8, 9 ]
\endexample

\>NextPrimeInt( <n> ) F

`NextPrimeInt' returns the smallest prime  which is strictly larger  than
the integer <n>.

Note  that     `NextPrimeInt'  uses  a    probable-primality  test   (see
"IsPrimeInt").


\beginexample
gap> NextPrimeInt( 541 ); NextPrimeInt( -1 );
547
2
\endexample

\>PrevPrimeInt( <n> ) F

`PrevPrimeInt' returns the largest prime  which is  strictly smaller than
the integer <n>.

Note  that    `PrevPrimeInt'   uses   a  probable-primality    test  (see
"IsPrimeInt").


\beginexample
gap> PrevPrimeInt( 541 ); PrevPrimeInt( 1 );
523
-2
\endexample

\>FactorsInt( <n> ) F
\>FactorsInt( <n> : RhoTrials := <trials> ) F

`FactorsInt' returns a list of prime factors of the integer <n>.

If the <i>th power of a prime divides <n> this prime appears <i> times.
The list is sorted, that is the smallest prime factors come first.
The first element has the same sign as <n>, the others are positive.
For any integer <n> it holds that `Product( FactorsInt( <n> ) ) = <n>'.

Note that `FactorsInt' uses a probable-primality test (see~"IsPrimeInt").
Thus `FactorsInt' might return a list which contains composite integers.
In such a case you will get a warning about the use of a probable prime.
You can switch off these warnings by `SetInfoLevel(InfoPrimeInt, 0);'.

The time taken by   `FactorsInt'  is approximately  proportional to   the
square root of the second largest prime factor  of <n>, which is the last
one that `FactorsInt'  has to find,   since the largest  factor is simply
what remains when all others have been removed.  Thus the time is roughly
bounded by  the fourth  root of <n>.   `FactorsInt' is guaranteed to find
all factors   less than  $10^6$  and will find  most    factors less than
$10^{10}$.    If <n>    contains   multiple  factors   larger  than  that
`FactorsInt' may not be able to factor <n> and will then signal an error.

`FactorsInt' is used in a method for the general operation `Factors'.

In the second form, FactorsInt calls FactorsRho with a limit of <trials>
on the number of trials is performs. The  default is 8192.


\beginexample
gap> FactorsInt( -Factorial(6) );
[ -2, 2, 2, 2, 3, 3, 5 ]
gap> Set( FactorsInt( Factorial(13)/11 ) );
[ 2, 3, 5, 7, 13 ]
gap> FactorsInt( 2^63 - 1 );
[ 7, 7, 73, 127, 337, 92737, 649657 ]
gap> FactorsInt( 10^42 + 1 );
#I  IsPrimeInt: probably prime, but not proven: 4458192223320340849
[ 29, 101, 281, 9901, 226549, 121499449, 4458192223320340849 ]
\endexample

\>PrintFactorsInt( <n> ) F

prints the prime factorization of the integer <n> in human-readable
form.


\beginexample
gap> PrintFactorsInt( Factorial( 7 ) ); Print( "\n" );
2^4*3^2*5*7
\endexample

\>PrimePowersInt( <n> ) F

returns the prime factorization of the integer <n> as a list
$[ p_1, e_1, \ldots, p_n, e_n ]$ with $n = \prod_{i=1}^n p_i^{e_i}$.


\beginexample
gap> PrimePowersInt( Factorial( 7 ) );
[ 2, 4, 3, 2, 5, 1, 7, 1 ]
\endexample

\>DivisorsInt( <n> ) F

`DivisorsInt' returns a list of all divisors  of  the  integer  <n>.  The
list is sorted, so that it starts with 1 and  ends  with <n>.  We  define
that `Divisors( -<n> ) = Divisors( <n> )'.

Since the  set of divisors of 0 is infinite calling `DivisorsInt( 0 )'
causes an error.

`DivisorsInt' may call `FactorsInt' to obtain the prime factors.
`Sigma' and `Tau' (see~"Sigma" and "Tau") compute the sum and the
number of positive divisors, respectively.


\index{divisors!of an integer}
\beginexample
gap> DivisorsInt( 1 ); DivisorsInt( 20 ); DivisorsInt( 541 );
[ 1 ]
[ 1, 2, 4, 5, 10, 20 ]
[ 1, 541 ]
\endexample


%%%%%%%%%%%%%%%%%%%%%%%%%%%%%%%%%%%%%%%%%%%%%%%%%%%%%%%%%%%%%%%%%%%%%%%%%%%%%
\Section{Residue Class Rings}

\indextt{mod!residue class rings}
\>`<r> / <s> mod <n>'{modulo!residue class rings}

If <r>, <s> and <n> are integers, `<r> / <s>' as a  reduced  fraction  is
`<p> / <q>', and <q> and <n> are coprime, then `<r> /  <s>  mod  <n>'  is
defined to be the product of <p> and the inverse of <q> modulo  <n>.  See
Section~"Arithmetic Operators" for more details and definitions.

With the above definition, `4 / 6 mod 32' equals `2 / 3 mod 32' and hence
exists (and is equal to 22), despite the  fact  that  6  has  no  inverse
modulo 32.


\>ZmodnZ( <n> ) F
\>ZmodpZ( <p> ) F
\>ZmodpZNC( <p> ) F

`ZmodnZ' returns a ring $R$ isomorphic to the residue class ring of the
integers modulo the positive integer <n>.
The element corresponding to the residue class of the integer $i$ in this
ring can be obtained by $i \* `One'( R )$, and a representative of the
residue class corresponding to the element $x \in R$ can be computed by
$`Int'( x )$.

\index{mod!Integers}
`ZmodnZ( <n> )' is equivalent to `Integers mod <n>'.

`ZmodpZ' does the same if the argument <p> is a prime integer,
additionally the result is a field.
`ZmodpZNC' omits the check whether <p> is a prime.

Each ring returned by these functions contains the whole family of its
elements
if $n$ is not a prime, and is embedded into the family of finite field
elements of characteristic $n$ if $n$ is a prime.


\>ZmodnZObj( <Fam>, <r> ) O
\>ZmodnZObj( <r>, <n> ) O

If the first argument is a residue class family <Fam> then `ZmodnZObj'
returns the element in <Fam> whose coset is represented by the integer
<r>.
If the two arguments are an integer <r> and a positive integer <n> then
`ZmodnZObj' returns the element in `ZmodnZ( <n> )' (see~"ZmodnZ")
whose coset is represented by the integer <r>.


\beginexample
gap> r:= ZmodnZ(15);
(Integers mod 15)
gap> fam:=ElementsFamily(FamilyObj(r));;
gap> a:= ZmodnZObj(fam,9);
ZmodnZObj( 9, 15 )
gap> a+a;
ZmodnZObj( 3, 15 )
gap> Int(a+a);
3
\endexample

\>IsZmodnZObj( <obj> ) C
\>IsZmodnZObjNonprime( <obj> ) C
\>IsZmodpZObj( <obj> ) C
\>IsZmodpZObjSmall( <obj> ) C
\>IsZmodpZObjLarge( <obj> ) C

The elements in the rings $Z / n Z$ are in the category `IsZmodnZObj'.
If $n$ is a prime then the elements are of course also in the category
`IsFFE' (see~"IsFFE"), otherwise they are in `IsZmodnZObjNonprime'.
`IsZmodpZObj' is an abbreviation of `IsZmodnZObj and IsFFE'.  This
category is the disjoint union of `IsZmodpZObjSmall' and
`IsZmodpZObjLarge', the former containing all elements with $n$ at most
`MAXSIZE_GF_INTERNAL'.

The reasons to distinguish the prime case from the nonprime case are
\beginlist%unordered
\item{--}
  that objects in `IsZmodnZObjNonprime' have an external representation
  (namely the residue in the range $[ 0, 1, \ldots, n-1 ]$),
\item{--}
  that the comparison of elements can be defined as comparison of the
  residues, and
\item{--}
  that the elements lie in a family of type `IsZmodnZObjNonprimeFamily'
  (note that for prime $n$, the family must be an `IsFFEFamily').
\endlist

The reasons to distinguish the small and the large case are
that for small $n$ the elements must be compatible with the internal
representation of finite field elements, whereas we are free to define
comparison as comparison of residues for large $n$.

Note that we *cannot* claim that every finite field element of degree 1
is in `IsZmodnZObj', since finite field elements in internal
representation may not know that they lie in the prime field.



The residue class rings are rings, thus all operations for rings (see
Chapter~"Rings") apply.
See also Chapters~"Finite fields" and "Number theory".


%%%%%%%%%%%%%%%%%%%%%%%%%%%%%%%%%%%%%%%%%%%%%%%%%%%%%%%%%%%%%%%%%%%%%%%%%%%%%
%%
%E

