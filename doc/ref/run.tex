% This file was created automatically from run.msk.
% DO NOT EDIT!
%%%%%%%%%%%%%%%%%%%%%%%%%%%%%%%%%%%%%%%%%%%%%%%%%%%%%%%%%%%%%%%%%%%%%%%%%
%%
%W  run.msk                 GAP documentation                Frank Celler
%W                                                     & Martin Schoenert
%W                                                     & Alexander Hulpke
%W                                                    & Burkhard Hoefling
%%
%H  @(#)$Id: run.msk,v 1.37.2.1 2004/01/26 10:12:22 gap Exp $
%%
%Y  Copyright 1997,  Lehrstuhl D fuer Mathematik,  RWTH Aachen,   Germany
%%
%%  This file contains the  description of the installation procedure and
%%  command line options for various operating systems.
%%

%%%%%%%%%%%%%%%%%%%%%%%%%%%%%%%%%%%%%%%%%%%%%%%%%%%%%%%%%%%%%%%%%%%%%%%%%
\Chapter{Running GAP}

\index{options}

This chapter informs about command line options for {\GAP} under UNIX
and OS X
(see~"Command Line Options", "Advanced Features of GAP"),
and features of {\GAP} on the Macintosh (see 
"Running GAP under MacOS"),
the `.gaprc' file (see~"The .gaprc file"),
completion files (see~"Completion Files"),
the {\GAP} compiler (see~"The Compiler", "Suitability for Compilation",
"Compiling Library Code"),
and how to save and load a {\GAP} workspace
(see~"Saving and Loading a Workspace").


%%%%%%%%%%%%%%%%%%%%%%%%%%%%%%%%%%%%%%%%%%%%%%%%%%%%%%%%%%%%%%%%%%%%%%%%%
\Section{Command Line Options}

\atindex{features!under UNIX}{@features!under UNIX}
\atindex{UNIX!features}{@UNIX!features}
\atindex{options!under UNIX}{@options!under UNIX}
\atindex{UNIX!options}{@UNIX!options}
When you start {\GAP} under UNIX, you may specify a number of  options on
the command-line to change  the default behaviour  of {\GAP}.  All  these
options  start with a  hyphen `-', followed by a  single letter.  Options
must not be   grouped, e.g.,  `gap  -gq' is   illegal, use   `gap -g  -q'
instead.  Some options  require an argument, this  must follow the option
and  must be separated   by a <space>, e.g.,  `gap   -m 256k', it is  not
correct to say `gap -m256k' instead.

{\GAP} for UNIX will distinguish between upper and lower case options.

As is described in Chapter~"Installing GAP" (see "Installing GAP"),
usually you will not execute {\GAP} directly.  Instead  you  will
call a shell script, with the name `gap', which in turn executes  {\GAP}.
This shell script sets some options which are necessary  to  make  {\GAP}
work on your system. This means that the default settings mentioned below
may not be what you experience when you execute {\GAP} on your system.

\beginitems

\indextt{-h}
`-h' &
tells {\GAP} to print a summary of all available options (`-h' is mnemonic
for ``help''). {\GAP} exits after printing the summary, all other options
are ignored.

\indextt{-b}
`-b' &
tells {\GAP} to suppress the banner. That means that  {\GAP}  immediately
prints the prompt. This is useful when, after a while, you get  tired  of
the banner.

\indextt{-q}
`-q' &
tells {\GAP} to be quiet. This means that  {\GAP}  displays  neither  the
banner nor the prompt `gap>'. This is useful if you want to run {\GAP} as
a filter with input and output redirection and want to avoid  the  banner
and the prompts appearing in the output file.

\indextt{-e}
`-e' &
tells {\GAP} not to quit when receiving a `<ctr>-D'  on  an  empty  input
line (see~"quit"). This option should not be used when  the  input  is  a
file or pipe.

\indextt{-f}
`-f' &
tells {\GAP} to enable the line editing and history (see "Line Editing").

&
In general  line editing will  be enabled if the input  is connected to a
terminal.  There are rare circumstances, for  example when using a remote
session with a corrupted telnet implementation, when this detection fails.
Try using `-f' in this case to enable line editing.

\indextt{-n}
`-n' &
tells  {\GAP}  to  disable  the  line  editing  and  history  (see  "Line
Editing").

&
You may want to do this if the command line  editing is incompatible with
another program that is used to run {\GAP}.  For example if {\GAP} is run
from inside a GNU Emacs shell window, `-n' should be used since otherwise
every input line will be echoed twice, once by Emacs and once by {\GAP}.

\indextt{-x}
`-x <length>' &
With this option  you can tell  {\GAP}  how long lines  are.  {\GAP} uses
this value to decide when to split long lines. After starting {\GAP}  you
may use `SizeScreen' (see~"SizeScreen") to alter the line length.

&
The default value is 80, unless another value can be  obtained  from  the
Operating System, which is the right value if you have a  standard  ASCII
terminal. If you have a  larger  monitor,  or  use  a  smaller  font,  or
redirect the output to a printer, you may want to increase this value.

\indextt{-y}
`-y <length>' &
With this option you can tell {\GAP} how  many  lines  your  screen  has.
{\GAP} uses this value to decide after how many lines of on-line help  it
should  wait.  After   starting   {\GAP}   you   may   use   `SizeScreen'
(see~"SizeScreen") to alter the number of lines.

&
The default value is 24, unless another value can be  obtained  from  the
Operating System, which is the right value if you have a  standard  ASCII
terminal. If you have a  larger  monitor,  or  use  a  smaller  font,  or
redirect the output to a printer, you may want to increase this value.

\indextt{-g}
`-g' &
tells {\GAP} to print a information message every  time  a  full  garbage
collection is performed.

\begintt
#G  FULL 44580/2479kb live   57304/4392kb dead   734/4096kb free
\endtt

&
For example, this tells you  that  there  are  44580  live  objects  that
survived a full  garbage  collection,  that  57304  unused  objects  were
reclaimed by it, and that 734 KBytes from a  total  allocated  memory  of
4096 KBytes are available afterwards.

\indextt{-g -g}
`-g -g' &
If you give the option `-g' twice, {\GAP} prints  a  information  message
every time a  partial  or  full  garbage  collection  is  performed.  The
message,

\begintt
#G  PART 9405/961kb+live   7525/1324kb+dead   2541/4096kb free
\endtt

&
for example, tells you that 9405 objects  survived  the  partial  garbage
collection and 7525 objects were reclaimed, and that 2541 KBytes  from  a
total allocated memory of 4096 KBytes are available afterwards.

\indextt{-m}
`-m <memory>' &
tells {\GAP} to allocate <memory> bytes at  startup  time.  If  the  last
character of <memory> is `k' or `K' it is taken as KBytes, if the last
character is `m' or `M' <memory> is taken as MBytes and if it is 'g'
or 'G' it is taken as Gigabytes.

&
Under UNIX the default amount of memory allocated by {\GAP} is 24  MBytes.
The amount of memory should be large enough so that computations  do  not
require too many garbage  collections.  On  the  other  hand,  if  {\GAP}
allocates more virtual memory than is physically available, it will spend
most of the time paging.

\indextt{-o}
`-o <memory>' &
tells {\GAP} to allocate at most <memory> bytes. If the last character of
<memory> is `k' or `K' it is taken as KBytes, if the last character is
`m' or `M' <memory> is taken as MBytes and if it is 'g' or 'G' it is
taken as Gigabytes. 

&
Under UNIX the default amount is 256 MBytes.  If more than this amount is
required  during the {\GAP} session,  {\GAP} prints an error messages and
enters a break loop.

\indextt{-K} 
`-K <memory>' &
is like `-o' above. But while the latter actually allocates more memory if
the system allows it and then prints a warning inside a break loop the `-K' 
options tells {\GAP} not even to try to allocate more memory. Instead {\GAP}
just exits with an appropriate message. The default is that this feature is
switched off. You have to set it explicitly when you want to enable it.

\indextt{-l}
`-l <path_list>' &
can be used to modify {\GAP}'s list of root directories  (see  "GAP  Root
Directory"). Before the option `-l' is used for the first time, the  only
root directory is `./', i.e., {\GAP} has only one root directory which is
the current directory. Usually this  option  is  used  inside  a  startup
script to specify where {\GAP} is  installed  on  the  system.  The  `-l'
option can also  be  used  by  individual  users  to  tell  {\GAP}  about
privately  installed  modifications  of  the  library,  additional {\GAP}
packages and so on. Section "GAP Root  Directory"  explains  how  several
root paths can be used to do this.

\indextt{GAPInfo.RootPaths}
&
<path_list> should be a list of directories separated by  semicolons.  No
whitespace is permitted before or after a semicolon. Each directory  name
should end with a pathname separator, i.e., `/', but {\GAP} will silently
add one if it is missing. If <path_list> does not start  or  end  with  a
semicolon,  then  <path_list>  replaces  the  existing   list   of   root
directories. If <path_list> starts with a semicolon, then <path_list>  is
appended to the existing list of root directories.  If  <path_list>  ends
with a semicolon (and does not start with one), then the new list of root
directories is the concatenation of <path_list> and the existing list  of
root directories. After {\GAP} has completed its  startup  procedure  and
displays the prompt, the list of root directories can be  viewed  in  the
variable `GAPInfo.RootPaths'.

&
{\GAP} will attempt  to  read  the  file  `<root_dir>/lib/init.g'  during
startup where <root_dir> is one of the directories in its  list  of  root
directories. If {\GAP} cannot find `init.g' it will print  the  following
warning

\begintt
gap: hmm, I cannot find 'lib/init.g' maybe use option '-l <gaproot>'?
\endtt

&
It is not possible  to use {\GAP} without the  library files, so you must
not  ignore this warning.  You  should leave {\GAP}   and start it again,
specifying the correct root path using the `-l' option.

\indextt{-r}
`-r' &
The option `-r'  tells {\GAP}  not to  read the user  supplied `~/.gaprc'
files.

\indextt{-L}
`-L <filename>' &
The option `-L' tells {\GAP} to load a saved workspace. See
section~"Saving and Loading a Workspace".

\indextt{-R}
`-R' &
The option `-R' tells {\GAP} not to load a saved workspace previously 
specified via the `-L' option. 

\index{options!command line, filenames}
`<filename> ...' &
Further arguments are taken as filenames of files that are read by {\GAP}
during startup, after the system  and private init   files are read,  but
before the first prompt is  printed.  The files are read  in the order in
which they appear on the command line.   {\GAP} only accepts 14 filenames
on the command  line.  If a  file cannot  be opened {\GAP}  will print an
error message and will abort.

\enditems

%%%%%%%%%%%%%%%%%%%%%%%%%%%%%%%%%%%%%%%%%%%%%%%%%%%%%%%%%%%%%%%%%%%%%%%%%
\Section{Advanced Features of GAP}

The following options are in general not needed  for the normal operation
of {\GAP}.  They are mostly used for debugging.

\beginitems

\indextt{-a}
`-a <memory>' &
GASMAN, the storage manager of {\GAP} uses `sbrk' to get blocks of memory
from (certain) operating systems and it is required that subsequent calls
to `sbrk' produce adjacent blocks of memory in this case  because  {\GAP}
only wants to deal with one large block of  memory.  If  the  C  function
`malloc' is called for whatever reason, it is likely that `sbrk' will  no
longer produce adjacent blocks, therefore {\GAP} does  not  use  `malloc'
itself.

&
However some operating systems insist on calling  `malloc'  to  create  a
buffer when a file is opened, or for some other reason. In order to catch
these cases {\GAP} preallocates a block of memory with `malloc' which  is
immediately freed. The amount preallocated can  be  controlled  with  the
`-a' option. If the last character of <memory> is `k' or `K' it is  taken
as KBytes and if the last character is `m' or `M' <memory>  is  taken  as
MBytes.

\indextt{-A}
`-A' &
By default, some {\GAP} packages (see "GAP  Packages")  are  loaded,  if
present, into the {\GAP} session when it  starts.  This  option  disables
(actually toggles) this behaviour, which can be useful for  debugging  or
testing.

\indextt{-B}
`-B <architecture>' &
Executable binary files that form part of {\GAP} or of  a {\GAP}  package
are kept in a subdirectory of the `bin' directory with in  the  {\GAP} or
package root directory. The subdirectory  name  is  determined  from  the
operating system, processor and compiler details when {\GAP}  (resp.  the
package) is installed. Under rare circumstances, it may be  necessary  to
override this name, and this can be done using the `-B' option.

\indextt{-D}
`-D' &
The `-D' option tells {\GAP} to print short messages when it  is  reading
or completing files or loading modules. The message,

\begintt
#I  READ_GAP_ROOT: loading 'lib/kernel.g' as GAP file
\endtt

&
tells   you  that   {\GAP}  has  started    to  read the   library   file
`lib/kernel.g'.

\begintt
#I  READ_GAP_ROOT: loading 'lib/kernel.g' statically
\endtt

&
tells you that {\GAP} has used the compiled version of the  library  file
`lib/kernel.g'. This compiled module was statically linked to the  {\GAP}
kernel at the time the kernel was created.

\begintt
#I  READ_GAP_ROOT: loading 'lib/kernel.g' dynamically
\endtt

&
tells you that {\GAP} has loaded the compiled version of the library file
`lib/kernel.g'. This compiled module was dynamically loaded to the {\GAP}
kernel at runtime from a corresponding `.so' file.

\begintt
#I  completing 'lib/domain.gd'
\endtt

&
tells you  that  {\GAP}  has  completed  the  file  `lib/domain.gd'.  See
"Completion Files" for more information about completion of files.

\indextt{-M}
`-M' &
tells {\GAP} not to check for, nor to use, compiled versions  of  library
files.

\indextt{-N}
`-N' &
tells {\GAP} not  to  check  for,  nor  to  use,  completion  files,  see
"Completion Files".

\indextt{-O}
`-O' &
enables a {\GAP}~3 compatibility mode, in which (for instance) the values
`false' and `fail' are identified. Use of this mode  is  not  recommended
other than as a transitional step in porting {\GAP}~3 code  to  {\GAP}~4,
because the {\GAP}~4 library may not work reliably in this mode.

\indextt{-T}
`-T' &
suppresses the usual break loop behaviour of  {\GAP}.  With  this  option
{\GAP} behaves as if the user `quit' immediately from every  break  loop.
This is intended for automated testing of {\GAP}.

\indextt{-X}
`-X' &
tells {\GAP} to do a consistency  check  of  the  library  file  and  the
corresponding completion file when reading the completion file.

\indextt{-Y}
`-Y' &
tells {\GAP} to do a consistency  check  of  the  library  file  and  the
corresponding completion file when completing the library file.

\indextt{-i}
`-i <filename>' &
changes  the  name  of  the  init  file  from  the  default  `init.g'  to
<filename>.

\enditems

\index{options!command line, internal}
\indextt{-C}\indextt{-U}\indextt{-P}\indextt{-W}\indextt{-z}
Additional options, `-C', `-U', `-P', `-W' and `-z' are  used  internally
in the {\GAP} compiler and/or on specific operating systems.


%%%%%%%%%%%%%%%%%%%%%%%%%%%%%%%%%%%%%%%%%%%%%%%%%%%%%%%%%%%%%%%%%%%%%%%%
\Section{Running GAP under MacOS}

This sections describes the features of {\GAP} for MacOS that differ 
from those described earlier in this chapter. 


Since you cannot enter command line options directly when you launch the
{\GAP} application on a Macintosh, another mechanism is being used: Hold
down any of the command (apple), option, control or shift keys or space bar
when launching the {\GAP} application, e.g., by double-clicking on its icon.
Please note that some keys have side effects (e.g., pressing the option key usually
closes Findeer windows), and that System X behaves slightly differently from other
systems.

 A dialog
box will open, into which you can enter the desired {\GAP} command line
options.  as described in "Command Line Options". For example, 
if you want {\GAP} to start with a workspace of 32 megabytes,
the dialog box should contain the following text:

\begintt
-m 32m
\endtt

Note that the dialog box may already contain settings which you have 
previously saved. The `OK' button accepts the command line for 
the current {\GAP} session, and the `Save' button can be used to save these options 
for subsequent {\GAP} sessions. The
command line options will be saved in a text file called `GAP options' in the 
`Preferences' folder in the system folder. You may also modify the file `GAP
options' directly; note that changes only take effect the next time you launch
{\GAP}.

There are three additional command line option on the Mac.

\beginitems

\indextt{-z!on Macintosh}
`-z <n>' &
sets the time between checks for events (keystrokes, mouse  clicks  etc.)
to $n/60$ second. Lower values make {\GAP} more responsive but  computations
are somewhat slower. A value greater than  60  is  not  recommended,  the
default value for $n$ is $6$.

\indextt{-P!on Macintosh}
`-P <m>' &
sets the amount of memory required  for  printing.  The  reason  is  that
printer drivers may require quite a bit of memory, and may even crash  if
not enough is found. To prevent this, {\GAP} will  not  print  unless  at
least the specified amount of memory is available. The default  value  is
64 Kilobytes, which is enough for the Apple LaserWriter  printer  driver.
Setting the printing memory to 0 disables printing altogether.

\indextt{-W!on Macintosh}
`-W <m>' &
sets the size of the log window to <m> bytes. This means that if the text
in the log window exceeds this amount, then lines at the beginning of the
log are deleted. The default value is 32 Kilobytes.

\enditems

The following command line options work differently on the Mac.

\beginitems

\indextt{-a!on Macintosh}
`-a' &
On the Mac, the  `-a'  option  has  a  different  meaning  from  the  one
described in "Advanced Features of GAP". On the Mac, it must be  used  to
reserve memory for  loading  dynamic  libraries  into  {\GAP}.  See  "The
Compiler" for details about dynamic libraries  (and  note  that  the  PPC
version of {\GAP} for MacOS *can* use dynamic libraries).

\indextt{-f!on Macintosh}\indextt{-n!on Macintosh}
`-f', `-n' &
The `-f' and `-n' command line options do not have any effect on the Mac.

\indextt{-e!on Macintosh}
`-e' &
The `-e' command line option enables `<ctr>-D'.

\indextt{-o!on Macintosh}
`-o' &
The `-o' command line option should not normally be used on the Mac.  The
value set by the `-o' option is only used if it is lower than the size of
the workspace that would normally be available for {\GAP}.

\enditems

\indextt{gap.rc}
The file called `.gaprc' on UNIX systems (see "The .gaprc file") is called 
`gap.rc' on the Mac; it must be in the same folder as the {\GAP} application.


All interaction between {\GAP} and you takes place via the `GAP log' 
window: this is where {\GAP} prints its messages and waits for your input.
The amount of text in this window is limited (see the `-W' command line option
above), so don't be surprised if old {\GAP} messages are deleted from the 
beginning of the text when this limit is reached. The reason for deleting 
old lines is that otherwise {\GAP} may run out of memory just because of 
the messages it has printed.

{\GAP} for the Mac now remembers the font and
text size (which can be set choosing `Format...' in the `Edit' menu)
as well as the window position of the {\GAP} log window from one
session to the next.

Almost all of the {\GAP} editing keys described in Section~"Line Editing" work
on the Mac. In addition, {\GAP} for MacOS also supports the usual 
editing keys on the Mac, such as Copy and Paste, Undo, arrow keys (also
with `<shift>', `<option>' and `<command>'. Note that you can also move 
forward and backward in the command line history by pressing 
`<ctrl>-<arrow down>' and `<ctrl>-<arrow up>'.

Note that `Quit' in {\GAP}'s file menu works differently from the `quit'
{\GAP} command (see~"quit"):
`Quit' in the file menu always quits the {\GAP} application, it
cannot be used to quit from a break loop. 

{\GAP} for MacOS also contains a simple built-in text editor, which is mainly
intended to create {\GAP} files. `New', `Open...', `Save' and~`Close' from the
`File' menu work in the usual way.

The `Read...' and `LogTo' commands in the `File' menu work basically like the 
corresponding {\GAP} commands (see "File Operations"). The only difference
is that {\GAP} will prompt you for the file with a standard Mac file opening
dialog, so you do not have to enter the path name yourself. (You will see the
file's path name in the log window afterwards). Note that if a file you want 
to
read is open in {\GAP}'s built-in editor, then {\GAP} will read the file from
the edit window, not from the disk.

If you press the shift key while choosing `Read...' from the `File' menu, the
menu item will change to `Reread...' which will then use the GAP command `Reread' 
(see "Reread") to read the chosen file.

The `Read...' command in the `File' menu changes to `Read' if the front window
belongs to a file in {\GAP}'s built-in editor -- choosing `Read' then makes
{\GAP} read that file -- and while the file is being read, the `File' menu 
item
changes to `Abort Read'. You cannot close the file's window while it is being
read by {\GAP} -- choose `Abort Read' first.

Garbage collection messages, which are switched on and off by the `-g'
command line option (see "Command Line Options") can also be switched
on and off by choosing `Show garbage collections' and 
`Show partial collections' from the `Window' menu.

If `Always scroll to printout' is selected in the `Window' menu, {\GAP} will
always scroll the `GAP log' window so that you can see what {\GAP} is 
currently printing. Otherwise, the {\GAP} log window is only scrolled to
the current print position when {\GAP} prints its prompt and
waits for you to enter a command. Note that you may see text lines disappear
even if `Always scroll to printout' is off -- this happens if you are
viewing the text at the beginning of the log window and some lines are just
being deleted from the log because it has exceeded its 32000 character limit.

The contents of the `Help' menu should be quite self-explanatory. Note that, 
unlike in {\GAP}~3 for the Mac, the online help is not displayed in a separate 
window,
nor is the online help available while {\GAP} is computing.

Holding down the Command (Apple) key while selecting text does the same as
selecting the text and choosing `Find selection in table of contents' from
the `Help' menu, holding down both Command and Option keys while selecting
tries to find the selection in the index.

When you want to refer to files or folders in {\GAP} (for example in the
`Read', `PrintTo', `AppendTo', `LogTo' commands), or have to specify files
or folders for a command line option, these files must be identified by UNIX
style path names.  (Presently, {\GAP} for MacOS also supports Mac path
names, but this may change in the future.) 

Users who are familiar with UNIX path names may skip the rest of this
section, noting that the working directory (i.e., folder) is the one in 
which the {\GAP} application resides, and that file names on the Mac are 
*not* case sensitive. 

Paths are strings used to describe where a file 
is stored on a hard 
disk. There are two ways for specifying UNIX path names: absolute and 
relative paths. An absolute path starts with a `/', then the name of the 
disk where the file is located, another `/', then a list of folders, 
each containing the next one, separated by `/', and finally the name of 
the file, which resides in the last folder in the list. For instance, if 
your hard disk is called `My HD', and your file `program.g' resides (or 
should be created) in the folder `programs' in the folder `documents' on 
`My HD', the absolute path name to that file is

\begintt
/My HD/documents/programs/program.g
\endtt

Relative path names work similarly, except that the starting point is 
not a disk but the folder in which the {\GAP} application program resides.
Relative path names are formed like absolute ones, except that they do 
not start with a `/'. Thus, if you want to access the file `temp.g' in 
the folder `tmp' in the {\GAP} folder, you may use the following path name:
`tmp/temp.g'.
It is also possible to move upward to a parent folder: suppose that the 
folder containing {\GAP} is called `applications', which contains a folder 
`editor' which in turn contains the file 'program.g', then you could 
access this file by the path `../editor/program.g'. The path `./' refers
to the {\GAP} folder itself, and `../' refers to ``the folder above''.

Note also that {\GAP} for the Mac follows (resolves) aliases to folders and 
files.


%%%%%%%%%%%%%%%%%%%%%%%%%%%%%%%%%%%%%%%%%%%%%%%%%%%%%%%%%%%%%%%%%%%%%%%%%%%%%
\Section{The .gaprc file}

\indextt{gap.rc}\indextt{.gaprc}
When you start {\GAP}, it looks for the file with the name `.gaprc' in your
home directory (on UNIX systems). On a Macintosh or a Windows system the
equivalent to the `.gaprc' file is `gap.rc', and for it to be read it must
be in the same folder as the {\GAP} application.
(Note that the file must be called `gap.rc'. If you use a Windows text
editor, in particular if your default is not to show file suffixes, you
might accidentaly create a file `gap.rc.txt' or `gap.rc.doc' which GAP will
not recognize.)

If such a file is found it is read after
`<libname>/init.g', but before any of the files mentioned on the command
line are read. You can use this file for your private customizations.
For example, if you have a file containing functions or data that you
always need, you could read this from `.gaprc'. Or if you find some of
the names in the library too long, you could define abbreviations for
those names in `.gaprc'. The following sample `.gaprc' file does both.

\begintt
Read("/usr/you/dat/mygroups.grp");
Ac := Action;
AcHom := ActionHomomorphism;
RepAc := RepresentativeAction;
\endtt

\atindex{GAP3}{@GAP3}
If you have already a `.gaprc' file for {\GAP}~3, its settings might not be
compatible with {\GAP}~4. In this case it has to be removed.  On UNIX Systems
the following `.gaprc' file can be used to load alternatively a `.gap3rc' or
a `.gap4rc' file from your home directory.

\begintt
if IsBound(Permutations) then
  # GAP 3
  Exec("echo \"READ(\\\"`pwd ~`/.gap3rc\\\");\" > /tmp/jJj");
else
  # GAP 4
  Exec("echo \"READ(\\\"`pwd ~`/.gap4rc\\\");\" > /tmp/jJj");
fi;
Read("/tmp/jJj");
\endtt

%%%%%%%%%%%%%%%%%%%%%%%%%%%%%%%%%%%%%%%%%%%%%%%%%%%%%%%%%%%%%%%%%%%%%%%%%
\Section{Completion Files}

The standard distribution of {\GAP} already  contains completion files so
in general *you do not need to create these files by yourself*.

When starting, {\GAP} reads  in the  whole  library.  As this takes  some
time, library files are normally condensed into  completion files.  These
completion files contain the  basic skeleton of  the library but  not the
function bodies.   When a function  body is required, for example because
you  want   to  execute the  corresponding    function, the  library file
containing the function body is completed.

Completion files reduce the startup time of {\GAP} drastically.  However,
this technique also means that the information stored in  the  completion
files and the library must be consistent. If you change  a  library  file
without recreating the completion files disaster is bound to happen.

Bugfixes distributed for {\GAP} will also update  the  completion  files.
Therefore you only need to update them if you have changed the library by
yourself.

However, if you are modifying a library file a more convenient way is  to
use the `-X' option (see~"Command Line Options") that allows you (in most
cases) to use the completion files for the  unchanged  parts  of  library
files and avoids using the completion files for the changed parts.  After
you have finished modifying  the  library  files  you  can  recreate  the
completion files using:

\>CreateCompletionFiles( ) F
\>CreateCompletionFiles( <path> ) F

To create  completion files you must  have write permissions to `<path>',
which defaults to the  first root directory.   Start {\GAP} with the `-N'
option (to  suppress the reading  of any existing completion files), then
execute the command `CreateCompletionFiles( <path> );', where <path> is a
string giving a   path to the home   directory of  {\GAP} (the  directory
containing the `lib' directory).

This produces, in addition to lots of informational output,
the completion files.



\begintt
$ gap4 -N
gap> CreateCompletionFiles();
#I  converting "gap4/lib/read2.g" to "gap4/lib/read2.co"
#I    parsing "gap4/lib/process.gd"
#I    parsing "gap4/lib/listcoef.gi"
...
\endtt

%%%%%%%%%%%%%%%%%%%%%%%%%%%%%%%%%%%%%%%%%%%%%%%%%%%%%%%%%%%%%%%%%%%%%%%%
\Section{Testing for the System Architecture}

\>ARCH_IS_UNIX( ) F

tests whether {\GAP} is running on a UNIX system.

\>ARCH_IS_MAC( ) F

tests whether {\GAP} is running on a Macintosh under MacOS

\>ARCH_IS_WINDOWS( ) F

tests whether {\GAP} is running on a Windows system.



%%%%%%%%%%%%%%%%%%%%%%%%%%%%%%%%%%%%%%%%%%%%%%%%%%%%%%%%%%%%%%%%%%%%%%%%%%%%%
\Section{The Compiler}

\indextt{gac}
The {\GAP} compiler \package{GAC} creates C code from {\GAP} code and
then calls the system's C compiler to produce machine code from it. This can
result in a speedup (see section~"Suitability for Compilation" for more
details).

%%  The compiler works only on UNIX systems. It was written by
%%  Ferenc~R{\'a}k{\'o}czi and Martin Sch{\"o}nert.

To use the compiler to produce dynamically loadable modules,
call it with the `-d' option:
\begintt
M193 /home/ahulpke > gap4/bin/i386-ibm-linux-gcc2/gac -d test.g
gap4/bin/i386-ibm-linux-gcc2/gap -C /tmp/5827_test.c test.g Init_Dynamic
gcc -fpic -ansi -Wall -O2 -o /tmp/5827_test.o -I
gap4/bin/i386-ibm-linux-gcc2/../../src -c /tmp/5827_test.c
ld -Bshareable -x -o test.so /tmp/5827_test.o
rm -f /tmp/5827_test.o
rm -f /tmp/5827_test.c
\endtt
This produces a file `<file>.so'.

\>LoadDynamicModule( <filename> ) F
\>LoadDynamicModule( <filename>, <crc> ) F

To load a compiled file, the command `LoadDynamicModule' is used. This
command loads <filename> as module. If given, the CRC checksum <crc> must
match the value of the module (see "CRC Numbers").

\begintt
gap> LoadDynamicModule("./test.so");
gap> CrcFile("test.g");
2906458206
gap> LoadDynamicModule("./test.so",1);
Error, <crc> mismatch (or no support for dynamic loading) called from
<function>( <arguments> ) called from read-eval-loop
Entering break read-eval-print loop ...
you can 'quit;' to quit to outer loop, or
you can 'return;' to continue
brk> quit;
gap> LoadDynamicModule("./test.so",2906458206);
\endtt

If you want to see or modify the intermediate C code, you can also instruct
the compiler to produce only the C files by using the option `-C' instead of
`-d'. 

There are some known problems with C code produced with the
GAP compiler on 32 bit architectures and used on 64 bit architectures
(and vice versa).

On some operating systems, once you have loaded a dynamic module with
a certain filename, loading another with the same filename will have
no effect, even if the file on disk has changed.

%%%%%%%%%%%%%%%%%%%%%%%%%%%%%%%%%%%%%%%%%%%%%%%%%%%%%%%%%%%%%%%%%%%%%%%%%%%%%
\Section{Suitability for Compilation}

Typically algorithms spend large parts of their runtime only in small parts
of the code. The design of {\GAP} reflects this situation with kernel 
methods for many time critical calculations such as matrix or permutation
arithmetic.

Compiling an algorithm whose time critical parts are already in the kernel
of course will give disappointing results: Compilation will only speed up
the parts that are not already in the kernel and if they make us a small
part of the runtime, the overall gain is small.

Routines that benefit from compilation are those which do extensive
operations with basic data types, such as lists or small integers.

%%%%%%%%%%%%%%%%%%%%%%%%%%%%%%%%%%%%%%%%%%%%%%%%%%%%%%%%%%%%%%%%%%%%%%%%%%%%%
\Section{Compiling Library Code}

The most tempting code to compile is probably the library. This  section
describes the mechanism used to make {\GAP} recognize compiled versions of 
library files. Note however that there is no point in compiling the whole
library as typically only few functions benefit from compilation as
described in Section~"Suitability for Compilation".

All files that come with {\GAP} are read using the internal function
`READ_GAP_ROOT'. This function then checks whether a compiled version of the
file exists and if its CRC number (see~"CRC Numbers") matches the file.
If it does, the compiled version is loaded.  Otherwise the file is read.
You can start {\GAP} with the `-D -N' option to see information printed
about this process.

To make {\GAP} find the compiled versions, they must be
put in the `bin/<systemname>/compiled' directory (<systemname> is the name
you gave for compilation, for example `i386-ibm-linux-gcc2'). They have
to be called according to the following scheme: 
Suppose the file is `humpty/dumpty.gi' in the {\GAP} home
directory. Then the compiled version will be
`bin/<systemname>/compiled/humpty/gi/dumpty.so'. That is, the directory
hierarchy is mirrored under the `compiled' directory. A further directory
level is added for the suffix of the file,
and the suffix of the compiled version of the file is set to `.so'
(as produced by the compiler).

For example we show how
to compile the `combinat.gi' file on a Linux machine.  Suppose we are in
the home directory of the gap distribution.

\begintt
bin/gac -d lib/combinat.gi
\endtt

creates a file `combinat.so'. We now put it in the right place, creating
also the necessary directories:

\begintt
mkdir bin/i386-ibm-linux-gcc2/compiled
mkdir bin/i386-ibm-linux-gcc2/compiled/lib
mkdir bin/i386-ibm-linux-gcc2/compiled/lib/gi
mv combinat.so bin/i386-ibm-linux-gcc2/compiled/lib/gi
\endtt

If you now start {\GAP} and look, for example, at the function `Binomial',
defined in `combinat.gi', you see it is indeed compiled:
\begintt
gap> Print(Binomial);
function ( <<arg-1>>, <<arg-2>> )
    <<compiled code>>
end
\endtt

The command line option `-M'  disables the loading of compiled modules and
always reads code from the library.

%%%%%%%%%%%%%%%%%%%%%%%%%%%%%%%%%%%%%%%%%%%%%%%%%%%%%%%%%%%%%%%%%%%%%%%%%%%%%
\Section{CRC Numbers}

\atindex{CRC}{@CRC}\indextt{CrcFile!example}
CRC (cyclic redundancy check) numbers provide a certain method of doing
checksums. They are used by {\GAP} to check whether
files have changed.
Whenever files are ``condensed'' -- for example for completion files
(see Section~"Completion Files") or when compiling files 
(see Section~"The compiler") -- such a
checksum is computed implicitly and stored within the condensed file.

When reading a condensed version of the file instead of the original one,
the CRC checksum, which is computed via `CrcFile' (see~"CrcFile"), can be
used to check whether the original has been changed in the meantime, e.g.

\begintt
gap> CrcFile("lib/morpheus.gi");
2705743645
\endtt

%%%%%%%%%%%%%%%%%%%%%%%%%%%%%%%%%%%%%%%%%%%%%%%%%%%%%%%%%%%%%%%%%%%%%%%%%%%%%
\Section{Saving and Loading a Workspace}

\index{save}

\>SaveWorkspace( <filename> ) F

will save a ``snapshot'' image of the current {\GAP} workspace in the file
<filename>. This image then can be loaded by another copy of {\GAP} which
then will behave as at the point when `SaveWorkspace' was called. 

\begintt
gap> a:=1;
gap> SaveWorkspace("savefile");
true
gap> quit;
\endtt

`SaveWorkspace' can only be used at the main `gap>' prompt. It cannot
be included in the body of a loop or function, or called from a break loop.

\>`-L <filename>'{loading a saved workspace}

A saved workspace can be loaded by starting {\GAP} with the option `-L'
(see~"Command Line Options").
This will start {\GAP} and load the workspace.

\begintt
you@unix> gap -L savefile
gap> a;
1
\endtt

Please note that paths to workspaces have to be given in full, expansion of
the tilde to denote a home directory will *not* work.

Under UNIX, it is possible to compress savefiles using `gzip'. Compression
typically reduces the size of a workspace by a factor 3 or 4.
If {\GAP} is started with a compressed savefile (omit the `.gz' ending),
it will try to locate `gzip' on the system and uncompress the file
automatically while reading it.
\begintt
you@unix> gzip -9 savefile
you@unix> gap -L savefile
gap> a;
1
\endtt

We cannot guarantee that saved workspaces are portable between different
system architectures or over different versions of {\GAP} or its library.

If compiled modules had been loaded into {\GAP} before the workspace
was saved, they will be loaded into the new {\GAP} session during the
workspace loading process. If they are not available then the load
will fail. Additional compiled modules will *not* be used, even if
they are available, although they may be loaded later using `Reread'
(see "Reread").
`SaveWorkspace' may sometimes produce warning messages, as in

\begintt
gap> SaveWorkspace("b5");
#W bad bag id 4 found, 0 saved
#W bad bag id 20 found, 0 saved
true
\endtt

A small number of such messages can probably be ignored (they arise
because the garbage collector may not always collect all dead objects, 
and dead objects may contain data that `SaveWorkspace' does not know how 
to process). 

{\GAP} Packages which had been loaded before the workspace was saved
are loaded also when the workspace is loaded.
Packages which had been available but not loaded before the workspace
was saved are available also when the workspace is loaded,
provided these packages have not been upgraded.
Packages which have been newly installed *after* the workspace was saved
are *not* available when the workspace is loaded.
%T Is this acceptable?


%%%%%%%%%%%%%%%%%%%%%%%%%%%%%%%%%%%%%%%%%%%%%%%%%%%%%%%%%%%%%%%%%%%%%%%%%
%%
%E

