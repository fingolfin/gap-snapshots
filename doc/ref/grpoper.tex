% This file was created automatically from grpoper.msk.
% DO NOT EDIT!
%%%%%%%%%%%%%%%%%%%%%%%%%%%%%%%%%%%%%%%%%%%%%%%%%%%%%%%%%%%%%%%%%%%%%%%%%%%%%
%%
%A  grpoper.msk                GAP documentation            Alexander Hulpke
%A                                                          Heiko Theissen
%%
%A  @(#)$Id: grpoper.msk,v 1.46.2.1 2005/05/09 07:37:51 gap Exp $
%%
%Y  (C) 1998 School Math and Comp. Sci., University of St.  Andrews, Scotland
%Y  Copyright (C) 2002 The GAP Group
%%
\Chapter{Group Actions}

\index{group actions}\atindex{G-sets}{@$G$-sets}
A *group action* is a triple $(G,<Omega>,\mu)$, where $G$ is a group,
<Omega> a set and $\mu\colon<Omega>\times G\to<Omega>$ a function (whose
action is compatible with the group arithmetic). We call <Omega> the
*domain* of the action.

In {\GAP}, <Omega> can be a duplicate-free collection (an object that
permits access to its elements via the <Omega>[<n>] operation, for example a
list), it does not need to be sorted (see~"IsSet").

The acting function $\mu$ is a {\GAP} function of the form

\){\kernttindent}actfun(<pnt>,<g>)

that returns the image $\mu(<pnt>,<g>)$ for a point $<pnt>\in<Omega>$ and a
group element $<g>\in<G>$.

Groups always acts from the right, that is
$\mu(\mu(<pnt>,<g>),<h>)=\mu(<pnt>,<gh>)$.

{\GAP} does not test whether an acting function `actfun' satisfies the
conditions for a group operation but silently assumes that is does.
(If it does not, results are unpredictable.)

The first section of this chapter, "About Group Actions", describes the
various ways how operations for group actions can be called.

Functions for several commonly used action are already built into {\GAP}.
These are listed in section~"Basic Actions".

The sections "The Permutation Image of an Action" and 
"Action of a group on itself" describe homomorphisms and mappings associated
to group actions as well as the permutation group image of an action.

The other sections then describe operations to compute orbits,
stabilizers, as well as
properties of actions.

Finally section~"External Sets" describes the concept of ``external sets''
which represent the concept of a *$G$-set* and underly the actions mechanism.

%%%%%%%%%%%%%%%%%%%%%%%%%%%%%%%%%%%%%%%%%%%%%%%%%%%%%%%%%%%%%%%%%%%%%%%%%%%%%
\Section{About Group Actions}

\index{group actions!operations syntax}
The syntax which is used by the operations for group actions is quite
flexible. For example we can call the operation `OrbitsDomain' for the orbits
of the group <G> on the domain <Omega> in the following ways:

\)OrbitsDomain(<G>,<Omega>[,<actfun>])

The acting function <actfun> is optional. If it is not given, the built-in
action `OnPoints' (which defines an action via the caret operator `^') is
used as a default.

\)OrbitsDomain(<G>,<Omega>,<gens>,<acts>[,<actfun>])

This second version (of `OrbitsDomain') permits one to implement an action 
induced by a homomorphism:
If <H> acts on <Omega> via $\mu$ and $\varphi\colon G\to H$ is a
homomorphism, <G> acts on <Omega> via
$\mu'(\omega,g)=\mu(\omega,g^{\varphi})$:

Here <gens> must be a set of generators of <G> and <acts> the images of
<gens> under a homomorphism $\varphi\colon G\to H$.
<actfun> is the acting function for <H>, the call
to `ExampleActionFunction' implements the induced action of <G>.
Again, the acting function <actfun> is optional and `OnPoints' is used as a
default.

The advantage of this notation is that {\GAP} does not need to construct
this homomorphism $\varphi$ and the range group <H> as {\GAP} objects. (If a
small group <G> acts via complicated objects <acts> this otherwise could
lead to performance problems.)

{\GAP} does not test whether the mapping $<gens>\mapsto<acts>$
actually induces a homomorphism and the results are unpredictable if this is
not the case.

\)OrbitsDomain(<extset>) A

A third variant is to call the operation with an external set (which then
provides <G>, <Omega> and <actfun>. You will find more about external sets in
section~"External Sets".

For operations like `Stabilizer' of course the domain must be replaced by an
element of <Omega> which is to be acted on.

%%%%%%%%%%%%%%%%%%%%%%%%%%%%%%%%%%%%%%%%%%%%%%%%%%%%%%%%%%%%%%%%%%%%%%%%%%%%%
\Section{Basic Actions}

\index{group actions}
\index{actions}
\index{group operations}

{\GAP} already provides acting functions for the more common actions of a
group.  For built-in operations such as `Stabilizer' special methods are
available for many of these actions.

This section also shows how to implement different actions. (Note that every
action must be from the right.)

\>OnPoints( <pnt>, <g> ) F

\index{conjugation}\index{action!by conjugation}
returns `<pnt> ^ <g>'.
This is for example the action of a permutation group on points,
or the action of a group on its elements via conjugation.
The action of a matrix group on vectors from the right is described by
both `OnPoints' and `OnRight' (see~"OnRight").

\>OnRight( <pnt>, <g> ) F

returns `<pnt> \* <g>'.
This is for example the action of a group on its elements via right
multiplication,
or the action of a group on the cosets of a subgroup.
The action of a matrix group on vectors from the right is described by
both `OnPoints' (see~"OnPoints") and `OnRight'.

\>OnLeftInverse( <pnt>, <g> ) F

returns $<g>^{-1}$ `\* <pnt>'.
Forming the inverse is necessary to make this a proper action,
as in {\GAP} groups always act from the right.

(`OnLeftInverse' is used for example in the representation of a right
coset as an external set (see~"External Sets"), that is a right coset
$Ug$ is an external set for the group $U$ acting on it via
`OnLeftInverse'.)

\>OnSets( <set>, <g> ) F

\index{action!on sets}\index{action!on blocks}
Let <set> be a proper set (see~"Sorted Lists and Sets").
`OnSets' returns the proper set formed by the images
`OnPoints( <pnt>, <g> )' of all points <pnt> of <set>.

`OnSets' is for example used to compute the action of a permutation group
on blocks.

(`OnTuples' is an action on lists that preserves the ordering of entries,
see~"OnTuples".)

\>OnTuples( <tup>, <g> ) F

Let <tup> be a list.
`OnTuples' returns the list formed by the images
`OnPoints( <pnt>, <g> )' for all points <pnt> of <tup>.

(`OnSets' is an action on lists that additionally sorts the entries of
the result, see~"OnSets".)

\>OnPairs( <tup>, <g> ) F

is a special case of `OnTuples' (see~"OnTuples") for lists <tup>
of length 2.

\>OnSetsSets( <set>, <g> ) F

Action on sets of sets;
for the special case that the sets are pairwise disjoint,
it is possible to use `OnSetsDisjointSets' (see~"OnSetsDisjointSets").


\>OnSetsDisjointSets( <set>, <g> ) F

Action on sets of pairwise disjoint sets (see also~"OnSetsSets").


\>OnSetsTuples( <set>, <g> ) F

Action on sets of tuples.


\>OnTuplesSets( <set>, <g> ) F

Action on tuples of sets.


\>OnTuplesTuples( <set>, <g> ) F

Action on tuples of tuples


\beginexample
gap> g:=Group((1,2,3),(2,3,4));;
gap> Orbit(g,1,OnPoints);
[ 1, 2, 3, 4 ]
gap> Orbit(g,(),OnRight);
[ (), (1,2,3), (2,3,4), (1,3,2), (1,3)(2,4), (1,2)(3,4), (2,4,3), (1,4,2), 
  (1,4,3), (1,3,4), (1,2,4), (1,4)(2,3) ]
gap> Orbit(g,[1,2],OnPairs);
[ [ 1, 2 ], [ 2, 3 ], [ 1, 3 ], [ 3, 1 ], [ 3, 4 ], [ 2, 1 ], [ 1, 4 ], 
  [ 4, 1 ], [ 4, 2 ], [ 3, 2 ], [ 2, 4 ], [ 4, 3 ] ]
gap> Orbit(g,[1,2],OnSets);
[ [ 1, 2 ], [ 2, 3 ], [ 1, 3 ], [ 3, 4 ], [ 1, 4 ], [ 2, 4 ] ]
\endexample

\beginexample
gap> Orbit(g,[[1,2],[3,4]],OnSetsSets);
[ [ [ 1, 2 ], [ 3, 4 ] ], [ [ 1, 4 ], [ 2, 3 ] ], [ [ 1, 3 ], [ 2, 4 ] ] ]
gap> Orbit(g,[[1,2],[3,4]],OnTuplesSets);
[ [ [ 1, 2 ], [ 3, 4 ] ], [ [ 2, 3 ], [ 1, 4 ] ], [ [ 1, 3 ], [ 2, 4 ] ], 
  [ [ 3, 4 ], [ 1, 2 ] ], [ [ 1, 4 ], [ 2, 3 ] ], [ [ 2, 4 ], [ 1, 3 ] ] ]
gap> Orbit(g,[[1,2],[3,4]],OnSetsTuples);
[ [ [ 1, 2 ], [ 3, 4 ] ], [ [ 1, 4 ], [ 2, 3 ] ], [ [ 1, 3 ], [ 4, 2 ] ], 
  [ [ 2, 4 ], [ 3, 1 ] ], [ [ 2, 1 ], [ 4, 3 ] ], [ [ 3, 2 ], [ 4, 1 ] ] ]
gap> Orbit(g,[[1,2],[3,4]],OnTuplesTuples);
[ [ [ 1, 2 ], [ 3, 4 ] ], [ [ 2, 3 ], [ 1, 4 ] ], [ [ 1, 3 ], [ 4, 2 ] ], 
  [ [ 3, 1 ], [ 2, 4 ] ], [ [ 3, 4 ], [ 1, 2 ] ], [ [ 2, 1 ], [ 4, 3 ] ], 
  [ [ 1, 4 ], [ 2, 3 ] ], [ [ 4, 1 ], [ 3, 2 ] ], [ [ 4, 2 ], [ 1, 3 ] ], 
  [ [ 3, 2 ], [ 4, 1 ] ], [ [ 2, 4 ], [ 3, 1 ] ], [ [ 4, 3 ], [ 2, 1 ] ] ]
\endexample

\>OnLines( <vec>, <g> ) F

Let <vec> be a *normed* row vector, that is,
its first nonzero entry is normed to the identity of the relevant field,
`OnLines' returns the row vector obtained from normalizing
`OnRight( <vec>, <g> )' by scalar multiplication from the left.
This action corresponds to the projective action of a matrix group
on 1-dimensional subspaces.


\beginexample
gap> gl:=GL(2,5);;v:=[1,0]*Z(5)^0;
[ Z(5)^0, 0*Z(5) ]
gap> h:=Action(gl,Orbit(gl,v,OnLines),OnLines);
Group([ (2,3,5,6), (1,2,4)(3,6,5) ])
\endexample

\>OnIndeterminates( <poly>, <perm> )!{as a permutation action} F

A permutation <perm> acts on the multivariate polynomial <poly> by
permuting the indeterminates as it permutes points.


\>Permuted( <list>, <perm> )!{as a permutation action}

The following example demonstrates `Permuted' being used to implement a 
permutation action on a domain:

\beginexample
gap> g:=Group((1,2,3),(1,2));;
gap> dom:=[ "a", "b", "c" ];;
gap> Orbit(g,dom,Permuted);
[ [ "a", "b", "c" ], [ "c", "a", "b" ], [ "b", "a", "c" ], [ "b", "c", "a" ], 
  [ "a", "c", "b" ], [ "c", "b", "a" ] ]
\endexample

\>OnSubspacesByCanonicalBasis( <bas>, <mat> ) F

implements the operation of a matrix group on subspaces of a vector
space. <bas> must be a list of (linearly independent) vectors which
forms a basis of the subspace in Hermite normal form. <mat> is an
element of the acting matrix group. The function returns a mutable
matrix which gives the basis of the image of the subspace in Hermite
normal form. (In other words: it triangulizes the product of <bas> with
<mat>.)



\bigskip

If one needs an action for which no acting function is provided
by the library it can be implemented via a {\GAP} function that
conforms to the syntax

\)actfun(<omega>,<g>)

For example one could define the following function that acts on pairs of
polynomials via `OnIndeterminates':
\begintt
OnIndeterminatesPairs:=function(polypair,g)
  return [OnIndeterminates(polypair[1],g),
          OnIndeterminates(polypair[2],g)];
end;
\endtt

Note that this function *must* implement an action from the *right*. This is
not verified by {\GAP} and results are unpredicatble otherwise.

%%%%%%%%%%%%%%%%%%%%%%%%%%%%%%%%%%%%%%%%%%%%%%%%%%%%%%%%%%%%%%%%%%%%%%%%%%%%%
\Section{Orbits}

If <G> acts on <Omega> the set of all images of $\omega\in<Omega>$ under
elements of <G> is called the *orbit* of $\omega$. The set of orbits of <G>
is a partition of <Omega>.

Note that currently {\GAP} does *not* check whether a given point really
belongs to $\Omega$.
For example, consider the following example where the projective action
of a matrix group on a finite vector space shall be computed.

\beginexample
gap> Orbit( GL(2,3), [ -1, 0 ] * Z(3)^0, OnLines );
[ [ Z(3), 0*Z(3) ], [ Z(3)^0, 0*Z(3) ], [ Z(3)^0, Z(3) ], [ Z(3)^0, Z(3)^0 ], 
  [ 0*Z(3), Z(3)^0 ] ]
gap> Size( GL(2,3) ) / Length( last );
48/5
\endexample

The error is that `OnLines' (see~"OnLines") acts on the set of normed row
vectors (see~"NormedRowVectors") of the vector space in question,
but that the seed vector is itself not such a vector.

\>Orbit( <G>[, <Omega>], <pnt>, [<gens>, <acts>, ] <act> ) O

The orbit of the point <pnt> is the list of all images of <pnt> under
the action.

(Note that the arrangement of points in this list is not defined by the
operation.)

The orbit of <pnt> will always contain one element that is *equal* to
<pnt>, however for performance reasons this element is not necessarily
*identical* to <pnt>, in particular if <pnt> is mutable.


\beginexample
gap> g:=Group((1,3,2),(2,4,3));;
gap> Orbit(g,1);
[ 1, 3, 2, 4 ]
gap> Orbit(g,[1,2],OnSets);
[ [ 1, 2 ], [ 1, 3 ], [ 1, 4 ], [ 2, 3 ], [ 3, 4 ], [ 2, 4 ] ]
\endexample
(See Section~"Basic Actions" for information about specific actions.)

\>Orbits( <G>, <seeds>[, <gens>, <acts>][, <act>] )!{operation/attribute} O
\>Orbits( <xset> )!{operation/attribute} A

returns a duplicate-free list of the orbits of the elements in <seeds>
under the action <act> of <G>

(Note that the arrangement of orbits or of points within one orbit is
not defined by the operation.)


\>OrbitsDomain( <G>, <Omega>[, <gens>, <acts>][, <act>] ) O
\>OrbitsDomain( <xset> ) A

returns a list of the orbits of <G> on the domain <Omega> (given as
lists) under the action <act>.

This operation is often faster than `Orbits'.
The domain <Omega> must be closed under the action of <G>, otherwise an
error can occur.

(Note that the arrangement of orbits or of points within one orbit is
not defined by the operation.)


\beginexample
gap> g:=Group((1,3,2),(2,4,3));;
gap> Orbits(g,[1..5]);
[ [ 1, 3, 2, 4 ], [ 5 ] ]
gap> OrbitsDomain(g,Arrangements([1..4],3),OnTuples);
[ [ [ 1, 2, 3 ], [ 3, 1, 2 ], [ 1, 4, 2 ], [ 2, 3, 1 ], [ 2, 1, 4 ], 
      [ 3, 4, 1 ], [ 1, 3, 4 ], [ 4, 2, 1 ], [ 4, 1, 3 ], [ 2, 4, 3 ], 
      [ 3, 2, 4 ], [ 4, 3, 2 ] ], 
  [ [ 1, 2, 4 ], [ 3, 1, 4 ], [ 1, 4, 3 ], [ 2, 3, 4 ], [ 2, 1, 3 ], 
      [ 3, 4, 2 ], [ 1, 3, 2 ], [ 4, 2, 3 ], [ 4, 1, 2 ], [ 2, 4, 1 ], 
      [ 3, 2, 1 ], [ 4, 3, 1 ] ] ]
gap> OrbitsDomain(g,GF(2)^2,[(1,2,3),(1,4)(2,3)],
> [[[Z(2)^0,Z(2)^0],[Z(2)^0,0*Z(2)]],[[Z(2)^0,0*Z(2)],[0*Z(2),Z(2)^0]]]);
[ [ <an immutable GF2 vector of length 2> ], 
  [ <an immutable GF2 vector of length 2>, <an immutable GF2 vector of length 
        2>, <an immutable GF2 vector of length 2> ] ]
\endexample
(See Section~"Basic Actions" for information about specific actions.)

\>OrbitLength( <G>, <Omega>, <pnt>, [<gens>, <acts>, ] <act> ) O

computes the length of the orbit of <pnt>.


\>OrbitLengths( <G>, <seeds>[, <gens>, <acts>][, <act>] ) O
\>OrbitLengths( <xset> ) A

computes the lengths of all the orbits of the elements in <seegs> under
the action <act> of <G>.


\>OrbitLengthsDomain( <G>, <Omega>[, <gens>, <acts>][, <act>] ) O
\>OrbitLengthsDomain( <xset> ) A

computes the lengths of all the orbits of <G> on <Omega>.

This operation is often faster than `OrbitLengths'.
The domain <Omega> must be closed under the action of <G>, otherwise an
error can occur.


\beginexample
gap> g:=Group((1,3,2),(2,4,3));;
gap> OrbitLength(g,[1,2,3,4],OnTuples);
12
gap> OrbitLengths(g,Arrangements([1..4],4),OnTuples);
[ 12, 12 ]
\endexample

%%%%%%%%%%%%%%%%%%%%%%%%%%%%%%%%%%%%%%%%%%%%%%%%%%%%%%%%%%%%%%%%%%%%%%%%%%%%%
\Section{Stabilizers}

\index{point stabilizer}\index{set stabilizer}\index{tuple stabilizer}
The *Stabilizer* of an element $\omega$ is the set of all those $g\in G$
which fix $\omega$.


\>OrbitStabilizer( <G>, [<Omega>, ] <pnt>, [<gens>, <acts>, ] <act> ) O

computes the orbit and the stabilizer of <pnt> simultaneously in a
single Orbit-Stabilizer algorithm.

The stabilizer must have <G> as its parent.


\>Stabilizer( <G> [, <Omega>], <pnt> [, <gens>, <acts>] [, <act>] ) F

computes the stabilizer in <G> of the point <pnt>, that is the subgroup
of those elements of <G> that fix <pnt>.
The stabilizer will have <G> as its parent.


\beginexample
gap> g:=Group((1,3,2),(2,4,3));;
gap> Stabilizer(g,4);
Group([ (1,3,2) ])
\endexample

The stabilizer of a set or tuple of points can be computed by specifying an
action of sets or tuples of points.
\beginexample
gap> Stabilizer(g,[1,2],OnSets);
Group([ (1,2)(3,4) ])
gap> Stabilizer(g,[1,2],OnTuples);
Group(())
gap> OrbitStabilizer(g,[1,2],OnSets);
rec( orbit := [ [ 1, 2 ], [ 1, 3 ], [ 1, 4 ], [ 2, 3 ], [ 3, 4 ], [ 2, 4 ] ], 
  stabilizer := Group([ (1,2)(3,4) ]) )
\endexample
(See Section~"Basic Actions" for information about specific actions.)

The standard methods for all these actions are an Orbit-Stabilizer
algorithm. For permutation groups backtrack algorithms are used. For
solvable groups an orbit-stabilizer algorithm for solvable groups, which
uses the fact that the orbits of a normal subgroup form a block system (see
\cite{SOGOS}) is used.

\>OrbitStabilizerAlgorithm( <G>, <Omega>, <blist>, <gens>, <acts>, <pntact> ) F

This operation should not be called by a user. It is documented however
for purposes to extend or maintain the group actions package.

`OrbitStabilizerAlgorithm' performs an orbit stabilizer algorithm for
the group <G> acting with the generators <gens> via the generator images
<gens> and the group action <act> on the element <pnt>. (For
technical reasons <pnt> and <act> are put in one record with components
`pnt' and `act' respectively.)

The <pntact> record may carry a component <stabsub>. If given, this must
be a subgroup stabilizing *all* points in the domain and can be used to
abbreviate stabilizer calculations.

The argument <Omega> (which may be replaced by `false' to be ignored) is
the set within which the orbit is computed (once the orbit is the full
domain, the orbit calculation may stop).  If <blist> is given it must be
a bit list corresponding to <Omega> in which elements which have been found
already will be ``ticked off'' with `true'. (In particular, the entries
for the orbit of <pnt> still must be all set to `false'). Again the
remaining action domain (the bits set initially to `false') can be
used to stop if the orbit cannot grow any longer.
Another use of the bit list is if <Omega> is an enumerator which can
determine `PositionCanonical's very quickly. In this situation it can be
worth to search images not in the orbit found so far, but via their
position in <Omega> and use a the bit list to keep track whether the
element is in the orbit found so far.



%%%%%%%%%%%%%%%%%%%%%%%%%%%%%%%%%%%%%%%%%%%%%%%%%%%%%%%%%%%%%%%%%%%%%%%%%%%%%
\Section{Elements with Prescribed Images}

\index{transporter}
\>RepresentativeAction( <G> [, <Omega>], <d>, <e> [, <gens>, <acts>] [, <act>] ) O

computes an element of <G> that maps <d> to <e> under the given
action and returns `fail' if no such element exists.


\beginexample
gap> g:=Group((1,3,2),(2,4,3));;
gap> RepresentativeAction(g,1,3);
(1,3)(2,4)
gap> RepresentativeAction(g,1,3,OnPoints);
(1,3)(2,4)
gap> RepresentativeAction(g,(1,2,3),(2,4,3));
(1,2,4)
gap> RepresentativeAction(g,(1,2,3),(2,3,4));
fail
gap> RepresentativeAction(g,Group((1,2,3)),Group((2,3,4)));
(1,2,4)
gap>  RepresentativeAction(g,[1,2,3],[1,2,4],OnSets);
(2,4,3)
gap>  RepresentativeAction(g,[1,2,3],[1,2,4],OnTuples);
fail
\endexample
(See Section~"Basic Actions" for information about specific actions.)

Again the standard method for `RepresentativeAction' is an orbit-stabilizer
algorithm, for permutation groups and standard actions a backtrack algorithm
is used.

%%%%%%%%%%%%%%%%%%%%%%%%%%%%%%%%%%%%%%%%%%%%%%%%%%%%%%%%%%%%%%%%%%%%%%%%%%%%%
\Section{The Permutation Image of an Action}

If $G$ acts on a domain <Omega>, an enumeration of <Omega> yields a
homomorphism of $G$ into the symmetric group on $\{1,\ldots,|<Omega>|\}$. In
{\GAP}, the enumeration of the domain <Omega> is provided by the
`Enumerator' of <Omega> (see~"Enumerator") which of course is <Omega> itself
if it is a list.

\>ActionHomomorphism( <G>, <Omega> [, <gens>, <acts>] [, <act>] [, "surjective"] ) O
\>ActionHomomorphism( <xset> [, "surjective"] ) A
\>ActionHomomorphism( <action> ) A

computes a homomorphism from <G> into the symmetric group on $|<Omega>|$
points that gives the permutation action of <G> on <Omega>.

By default the homomorphism returned by `ActionHomomorphism' is not
necessarily surjective (its `Range' is the full symmetric group) to
avoid unnecessary computation of the image. If the optional string
argument `"surjective"' is given, a surjective homomorphism is created.

The third version (which is supported only for {\GAP}3 compatibility)
returns the action homomorphism that belongs to the image
obtained via `Action' (see "Action").


(See Section~"Basic Actions" for information about specific actions.)
\beginexample
gap> g:=Group((1,2,3),(1,2));;
gap> hom:=ActionHomomorphism(g,Arrangements([1..4],3),OnTuples);
<action homomorphism>
gap> Image(hom);
Group([ (1,9,13)(2,10,14)(3,7,15)(4,8,16)(5,12,17)(6,11,18)(19,22,23)(20,21,
    24), (1,7)(2,8)(3,9)(4,10)(5,11)(6,12)(13,15)(14,16)(17,18)(19,21)(20,
    22)(23,24) ])
gap> Size(Range(hom));Size(Image(hom));
620448401733239439360000
6
gap> hom:=ActionHomomorphism(g,Arrangements([1..4],3),OnTuples,
> "surjective");;
gap> Size(Range(hom));
6
\endexample

When acting on a domain, the operation `PositionCanonical' is used to
determine the position of elements in the domain.  This can be used to act
on a domain given by a list of representatives for which `PositionCanonical'
is implemented, for example a `RightTransversal' (see "RightTransversal").

\>Action( <G>, <Omega> [<gens>, <acts>] [, <act>] ) O
\>Action( <xset> ) A

returns the `Image' group of `ActionHomomorphism' called with the same
parameters.

Note that (for compatibility reasons to be able to get the
action homomorphism) this image group internally stores the action
homomorphism. If <G> or <Omega> are exteremly big, this can cause memory
problems. In this case compute only generator images and form the image
group yourself.

(See Section~"Basic Actions" for information about specific actions.)
\index{regular action}
The following code shows for example how to create the regular action of a
group:
\beginexample
gap> g:=Group((1,2,3),(1,2));;
gap> Action(g,AsList(g),OnRight);
Group([ (1,4,5)(2,3,6), (1,3)(2,4)(5,6) ])
\endexample

\>SparseActionHomomorphism( <G>, <Omega>, <start> [, <gens>, <acts>] [, <act>] ) O
\>SortedSparseActionHomomorphism( <G>, <Omega>, <start>[, <gens>, <acts>] [, <act>] ) O

`SparseActionHomomorphism' computes the
`ActionHomomorphism(<G>,<dom>[,<gens>,<acts>][,<act>])', where <dom>
is the union of the orbits `Orbit(<G>,<pnt>[,<gens>,<acts>][,<act>])'
for all points <pnt> from <start>. If <G> acts on a very large domain
<Omega> not surjectively this may yield a permutation image of
substantially smaller degree than by action on <Omega>.

The operation `SparseActionHomomorphism' will only use `=' comparisons
of points in the orbit. Therefore it can be used even if no good `\<'
comparison method exists. However the image group will depend on the
generators <gens> of <G>.

The operation `SortedSparseActionHomomorphism' in contrast
will sort the orbit and thus produce an image group which is not
dependent on these generators.


\beginexample
gap> h:=Group(Z(3)*[[[1,1],[0,1]]]);
Group([ [ [ Z(3), Z(3) ], [ 0*Z(3), Z(3) ] ] ])
gap> hom:=ActionHomomorphism(h,GF(3)^2,OnRight);;
gap> Image(hom);
Group([ (2,3)(4,9,6,7,5,8) ])
gap> hom:=SparseActionHomomorphism(h,[Z(3)*[1,0]],OnRight);;
gap> Image(hom);
Group([ (1,2,3,4,5,6) ])
\endexample

For an action homomorphism, the operation `UnderlyingExternalSet'
(see~"UnderlyingExternalSet") will return the external set on <Omega> which
affords the action.

%%%%%%%%%%%%%%%%%%%%%%%%%%%%%%%%%%%%%%%%%%%%%%%%%%%%%%%%%%%%%%%%%%%%%%%%%%%%%
\Section{Action of a group on itself}

Of particular importance is the action of a group on its elements or cosets
of a subgroup. These actions can be obtained by using `ActionHomomorphism'
for a suitable domain (for example a list of subgroups). For the following
(frequently used) types of actions however special (often particularly
efficient) functions are provided:

\>FactorCosetAction( <G>, <U>, [<N>] ) O

This command computes the action of <G> on the right cosets of the
subgroup <U>. If the normal subgroup <N> is given, it is stored as kernel
of this action.



\beginexample
gap> g:=Group((1,2,3,4,5),(1,2));;u:=SylowSubgroup(g,2);;Index(g,u);
15
gap> FactorCosetAction(g,u);
<action epimorphism>
gap> Range(last);
Group([ (1,9,13,10,4)(2,8,14,11,5)(3,7,15,12,6), 
  (1,7)(2,8)(3,9)(5,6)(10,11)(14,15) ])
\endexample

A special case is the regular action on all elements:
\>RegularActionHomomorphism( <G> ) A

returns an isomorphism from <G> onto the regular permutation
representation of <G>.


\>AbelianSubfactorAction( <G>, <M>, <N> ) O

Let <G> be a group and $<M>\ge<N>$ be subgroups of a common parent that
are normal under <G>, such that
the subfactor $<M>/<N>$ is elementary abelian. The operation
`AbelianSubfactorAction' returns a list `[<phi>,<alpha>,<bas>]' where
<bas> is a list of elements of <M> which are representatives for a basis
of $<M>/<N>$, <alpha> is a map from <M> into a $n$-dimensional row space
over $GF(p)$ where $[<M>:<N>]=p^n$ that is the
natural homomorphism of <M> by <N> with the quotient represented as an
additive group. Finally <phi> is a homomorphism from <G>
into $GL_n(p)$ that represents the action of <G> on the factor
$<M>/<N>$.

Note: If only matrices for the action are needed, `LinearActionLayer'
might be faster.


\beginexample
gap> g:=Group((1,8,10,7,3,5)(2,4,12,9,11,6),(1,9,5,6,3,10)(2,11,12,8,4,7));;
gap> c:=ChiefSeries(g);;List(c,Size);
[ 96, 48, 16, 4, 1 ]
gap> HasElementaryAbelianFactorGroup(c[3],c[4]);
true
gap> SetName(c[3],"my_group");;
gap> a:=AbelianSubfactorAction(g,c[3],c[4]);
[ [ (1,8,10,7,3,5)(2,4,12,9,11,6), (1,9,5,6,3,10)(2,11,12,8,4,7) ] -> 
    [ <an immutable 2x2 matrix over GF2>, <an immutable 2x2 matrix over GF2> ]
    , MappingByFunction( my_group, ( GF(2)^
    2 ), function( e ) ... end, function( r ) ... end ), 
  Pcgs([ (2,8,3,9)(4,10,5,11), (1,6,12,7)(4,10,5,11) ]) ]
gap> mat:=Image(a[1],g);
Group([ <an immutable 2x2 matrix over GF2>, 
  <an immutable 2x2 matrix over GF2> ])
gap> Size(mat);
3
gap> e:=PreImagesRepresentative(a[2],[Z(2),0*Z(2)]);
(2,8,3,9)(4,10,5,11)
gap> e in c[3];e in c[4];
true
false
\endexample

%%%%%%%%%%%%%%%%%%%%%%%%%%%%%%%%%%%%%%%%%%%%%%%%%%%%%%%%%%%%%%%%%%%%%%%%%%%%%
\Section{Permutations Induced by Elements and Cycles}

If only the permutation image of a single element is needed, it might not be
worth to create the action homomorphism, the following operations yield the
permutation image and cycles of a single element.

\>Permutation( <g>, <Omega>[, <gens>, <acts>][, <act>] ) F
\>Permutation( <g>, <xset> ) F

computes the permutation that corresponds to the action of <g> on the
permutation domain <Omega> (a list of objects that are permuted). If an
external set <xset> is given, the permutation domain is the `HomeEnumerator'
of this external set (see Section~"External Sets").
Note that the points of the returned permutation refer to the positions 
in <Omega>, even if <Omega> itself consists of integers.

If <g> does not leave the domain invariant, or does not map the domain 
injectively `fail' is returned.


\>PermutationCycle( <g>, <Omega>, <pnt> [, <act>] ) O

computes the permutation that represents the cycle of <pnt> under the
action of the element <g>.


\beginexample
gap> Permutation([[Z(3),-Z(3)],[Z(3),0*Z(3)]],AsList(GF(3)^2));
(2,7,6)(3,4,8)
gap> Permutation((1,2,3)(4,5)(6,7),[4..7]);
(1,2)(3,4)
gap> PermutationCycle((1,2,3)(4,5)(6,7),[4..7],4);
(1,2)
\endexample
\>Cycle( <g>, <Omega>, <pnt> [, <act>] ) O

returns a list of the points in the cycle of <pnt> under the action of the
element <g>.


\>CycleLength( <g>, <Omega>, <pnt> [, <act>] ) O

returns the length of the cycle of <pnt> under the action of the element
<g>.


\>Cycles( <g>, <Omega> [, <act>] ) O

returns a list of the cycles (as lists of points) of the action of the
element <g>.


\>CycleLengths( <g>, <Omega>, [, <act>] ) O

returns the lengths of all the cycles under the action of the element
<g> on <Omega>.


\beginexample
gap> Cycle((1,2,3)(4,5)(6,7),[4..7],4);
[ 4, 5 ]
gap> CycleLength((1,2,3)(4,5)(6,7),[4..7],4);
2
gap> Cycles((1,2,3)(4,5)(6,7),[4..7]);
[ [ 4, 5 ], [ 6, 7 ] ]
gap> CycleLengths((1,2,3)(4,5)(6,7),[4..7]);
[ 2, 2 ]
\endexample

%%%%%%%%%%%%%%%%%%%%%%%%%%%%%%%%%%%%%%%%%%%%%%%%%%%%%%%%%%%%%%%%%%%%%%%%%%%%%
\Section{Tests for Actions}

\>IsTransitive( <G>, <Omega>[, <gens>, <acts>][, <act>] )!{for group actions} O
\>IsTransitive( <xset> )!{for group actions} P

returns `true' if the action implied by the arguments is transitive, or
`false' otherwise.

\index{transitive}
We say that a  group <G> acts *transitively* on  a domain <D> if and
only if for every pair  of points <d>  and <e> there is  an element
<g> of <G> such that  $d^g = e$.


\>Transitivity( <G>, <Omega>[, <gens>, <acts>][, <act>] )!{for group actions} O
\>Transitivity( <xset> )!{for group actions} A

returns the degree $k$ (a non-negative integer) of transitivity of the
action implied by the arguments, i.e. the largest integer $k$ such that
the action is $k$-transitive. If the action is not transitive `0' is
returned.

An action is *$k$-transitive* if every $k$-tuple of points can be
mapped simultaneously to every other $k$-tuple.


\beginexample
gap> g:=Group((1,3,2),(2,4,3));;
gap> IsTransitive(g,[1..5]);
false
gap> Transitivity(g,[1..4]);
2
\endexample

*Note:* For permutation groups, the syntax `IsTransitive(<g>)' is also
permitted and tests whether the group is transitive on the points moved by
it, that is the group $\langle (2,3,4),(2,3)\rangle$ is transitive (on 3
points).

\>RankAction( <G>, <Omega>[, <gens>, <acts>][, <act>] ) O
\>RankAction( <xset> ) A

returns the rank of a transitive action, i.e. the number of orbits of
the point stabilizer.


\beginexample
gap> RankAction(g,Combinations([1..4],2),OnSets);
4
\endexample
\>IsSemiRegular( <G>, <Omega>[, <gens>, <acts>][, <act>] ) O
\>IsSemiRegular( <xset> ) P

returns `true' if the action implied by the arguments is semiregular, or
`false' otherwise.

\index{semiregular}
An action is *semiregular* is the stabilizer of each point is the
identity.


\>IsRegular( <G>, <Omega>[, <gens>, <acts>][, <act>] ) O
\>IsRegular( <xset> ) P

returns `true' if the action implied by the arguments is regular, or
`false' otherwise.

\index{regular}
An action is *regular* if it is  both  semiregular  (see~"IsSemiRegular")
and transitive (see~"IsTransitive!for group actions"). In this case every
point <pnt> of <Omega> defines a one-to-one  correspondence  between  <G>
and <Omega>.


\beginexample
gap> IsSemiRegular(g,Arrangements([1..4],3),OnTuples);
true
gap> IsRegular(g,Arrangements([1..4],3),OnTuples);
false
\endexample
\>Earns( <G>, <Omega>[, <gens>, <acts>][, <act>] ) O
\>Earns( <xset> ) A

returns a list of the elementary abelian regular (when acting on <Omega>)
normal subgroups of <G>.

At the moment only methods for a primitive group <G> are implemented.



\>IsPrimitive( <G>, <Omega>[, <gens>, <acts>][, <act>] ) O
\>IsPrimitive( <xset> ) P

returns `true' if the action implied by the arguments is primitive, or
`false' otherwise.

\index{primitive}
An action is *primitive* if it is transitive and the action admits no 
nontrivial block systems. See~"Block Systems".


\beginexample
gap> IsPrimitive(g,Orbit(g,(1,2)(3,4)));
true
\endexample
%\declaration{IsFixpointFree}


%%%%%%%%%%%%%%%%%%%%%%%%%%%%%%%%%%%%%%%%%%%%%%%%%%%%%%%%%%%%%%%%%%%%%%%%%%%%%
\Section{Block Systems}

A *block system* (system of imprimitivity) for the action of <G> on <Omega>
is a partition of <Omega> which -- as a partition -- remains invariant under
the action of <G>.

\>Blocks( <G>, <Omega>[, <seed>][, <gens>, <acts>][, <act>] ) O
\>Blocks( <xset>[, <seed>] ) A

computes a block system for the action. If
<seed> is not given and the action is imprimitive, a minimal nontrivial
block system will be found.
If <seed> is given, a block system in which <seed>
is the subset of one block is computed.
The action must be transitive.


\beginexample
gap> g:=TransitiveGroup(8,3);
E(8)=2[x]2[x]2
gap> Blocks(g,[1..8]);
[ [ 1, 8 ], [ 2, 3 ], [ 4, 5 ], [ 6, 7 ] ]
gap> Blocks(g,[1..8],[1,4]);
[ [ 1, 4 ], [ 2, 7 ], [ 3, 6 ], [ 5, 8 ] ]
\endexample
(See Section~"Basic Actions" for information about specific actions.)

\>MaximalBlocks( <G>, <Omega> [, <seed>] [, <gens>, <acts>] [, <act>] ) O
\>MaximalBlocks( <xset> [, <seed>] ) A

returns a block system that is maximal with respect to inclusion.
maximal with respect to inclusion) for the action of <G> on <Omega>.
If <seed> is given, a block system in which <seed>
is the subset of one block is computed.


\beginexample
gap> MaximalBlocks(g,[1..8]);
[ [ 1, 2, 3, 8 ], [ 4, 5, 6, 7 ] ]
\endexample

\>RepresentativesMinimalBlocks( <G>, <Omega>[, <gens>, <acts>][, <act>] ) O
\>RepresentativesMinimalBlocks( <xset> ) A

computes a list of block representatives for all minimal (i.e blocks are
minimal with respect to inclusion) nontrivial block systems for the
action. 


\beginexample
gap> RepresentativesMinimalBlocks(g,[1..8]);
[ [ 1, 2 ], [ 1, 3 ], [ 1, 4 ], [ 1, 5 ], [ 1, 6 ], [ 1, 7 ], [ 1, 8 ] ]
\endexample

\>AllBlocks( <G> ) A

computes a list of representatives of all block systems for a
permutation group <G> acting transitively on the points moved by the
group.


\beginexample
gap> AllBlocks(g);
[ [ 1, 8 ], [ 1, 2, 3, 8 ], [ 1, 4, 5, 8 ], [ 1, 6, 7, 8 ], [ 1, 3 ], 
  [ 1, 3, 5, 7 ], [ 1, 3, 4, 6 ], [ 1, 5 ], [ 1, 2, 5, 6 ], [ 1, 2 ], 
  [ 1, 2, 4, 7 ], [ 1, 4 ], [ 1, 7 ], [ 1, 6 ] ]
\endexample

The stabilizer of a block can be computed via the action
`OnSets' (see~"OnSets"):
\beginexample
gap> Stabilizer(g,[1,8],OnSets);
Group([ (1,8)(2,3)(4,5)(6,7) ])
\endexample

If <bs> is a partition of <omega>, given as a set of sets, the stabilizer
under the action `OnSetsDisjointSets' (see~"OnSetsDisjointSets") returns the
largest subgroup which preserves <bs> as a block system.
\beginexample
gap> g:=Group((1,2,3,4,5,6,7,8),(1,2));;
gap> bs:=[[1,2,3,4],[5,6,7,8]];;
gap> Stabilizer(g,bs,OnSetsDisjointSets);
Group([ (6,7), (5,6), (5,8), (2,3), (3,4)(5,7), (1,4), (1,5,4,8)(2,6,3,7) ])
\endexample

%%%%%%%%%%%%%%%%%%%%%%%%%%%%%%%%%%%%%%%%%%%%%%%%%%%%%%%%%%%%%%%%%%%%%%%%%%%%%
\Section{External Sets}

\atindex{G-sets}{@$G$-sets}
When considering group actions, sometimes the concept of a *$G$-set* is
used. This is the set <Omega> endowed with an action of $G$. The elements
of the $G$-set are the same as those of <Omega>, however concepts like
equality and equivalence of $G$-sets do not only consider the underlying
domain <Omega> but the group action as well.

This concept is implemented in {\GAP} via *external sets*.
\>IsExternalSet( <obj> ) C

An *external set*  specifies an action <act>:  $<Omega> \times <G> \to
<Omega>$  of a group <G> on a domain <Omega>. The external set knows the group,
the domain and the actual acting function.
Mathematically,  an external set  is the set~<Omega>, which is endowed with
the action of a group <G> via the group action <act>. For this reason
{\GAP} treats external sets as a domain whose elements are the  elements
of <Omega>. An external set is always a union of orbits.
Currently the domain~<Omega> must always be finite.
If <Omega> is not a list, an enumerator for <Omega> is automatically chosen.



\>ExternalSet( <G>, <Omega>[, <gens>, <acts>][, <act>] ) O

creates the external set for the action <act> of <G> on <Omega>.
<Omega> can be either a proper set  or a domain which is represented as
described in "Domains" and "Collections".


\beginexample
gap> g:=Group((1,2,3),(2,3,4));;
gap> e:=ExternalSet(g,[1..4]);
<xset:[ 1, 2, 3, 4 ]>
gap> e:=ExternalSet(g,g,OnRight);
<xset:<enumerator of perm group>>
gap> Orbits(e);
[ [ (), (1,2)(3,4), (1,3)(2,4), (1,4)(2,3), (2,4,3), (1,4,2), (1,2,3), 
      (1,3,4), (2,3,4), (1,3,2), (1,4,3), (1,2,4) ] ]
\endexample

The following three attributes of an external set hold its constituents.
\>ActingDomain( <xset> ) A

This attribute returns the group with which the external set <xset> was
defined.

\>FunctionAction( <xset> ) A

the acting function <act> of <xset>

\>HomeEnumerator( <xset> ) A

returns an enumerator of the domain <Omega> with which <xset> was defined.
For external subsets, this is  different  from `Enumerator(  <xset> )',
which enumerates only the subset.


\beginexample
gap> ActingDomain(e);
Group([ (1,2,3), (2,3,4) ])
gap> FunctionAction(e)=OnRight;
true
gap> HomeEnumerator(e);
<enumerator of perm group>
\endexample

Most operations for actions are applicable as an attribute for an external
set.

\>IsExternalSubset( <obj> ) R

An external subset is the restriction  of an external  set to a subset
of the domain (which must be invariant under the action). It is again an
external set.


The most prominent external subsets are orbits:
\>ExternalSubset( <G>, <xset>, <start>, [<gens>, <acts>, ]<act> ) O

constructs the external subset of <xset> on the union of orbits of the
points in <start>.


\>IsExternalOrbit( <obj> ) R

An external orbit is an external subset consisting of one orbit.


\>ExternalOrbit( <G>, <Omega>, <pnt>, [<gens>, <acts>, ] <act> ) O

constructs the external subset on the orbit of <pnt>. The
`Representative' of this external set is <pnt>.


\beginexample
gap> e:=ExternalOrbit(g,g,(1,2,3));
(1,2,3)^G
\endexample

Many subsets of a group, such as conjugacy classes or cosets
(see~"ConjugacyClass" and "RightCoset") are implemented as external orbits.

\>StabilizerOfExternalSet( <xset> ) A

computes the stabilizer of `Representative(<xset>)'
The stabilizer must have the acting group <G> of <xset> as its parent.


\beginexample
gap> Representative(e);
(1,2,3)
gap> StabilizerOfExternalSet(e);
Group([ (1,2,3) ])
\endexample

\>ExternalOrbits( <G>, <Omega>[, <gens>, <acts>][, <act>] ) O
\>ExternalOrbits( <xset> ) A

computes a list of `ExternalOrbit's that give the orbits of <G>.


\beginexample
gap> ExternalOrbits(g,AsList(g));
[ ()^G, (2,3,4)^G, (2,4,3)^G, (1,2)(3,4)^G ]
\endexample

\>ExternalOrbitsStabilizers( <G>, <Omega>[, <gens>, <acts>][, <act>] ) O
\>ExternalOrbitsStabilizers( <xset> ) A

In addition to `ExternalOrbits', this operation also computes the
stabilizers of the representatives of the external orbits at the same
time. (This can be quicker than computing the `ExternalOrbits' first and
the stabilizers afterwards.)


\beginexample
gap> e:=ExternalOrbitsStabilizers(g,AsList(g));
[ ()^G, (2,3,4)^G, (2,4,3)^G, (1,2)(3,4)^G ]
gap> HasStabilizerOfExternalSet(e[3]);
true
gap> StabilizerOfExternalSet(e[3]);
Group([ (2,4,3) ])
\endexample

\>CanonicalRepresentativeOfExternalSet( <xset> ) A

The canonical representative of an  external set may  only depend on <G>,
<Omega>, <act> and (in the case of  external subsets) `Enumerator( <xset> )'.
It must not depend, e.g., on the representative of an external orbit.
{\GAP} does not know methods for every external set to compute a
canonical representative . See
"CanonicalRepresentativeDeterminatorOfExternalSet".


\>CanonicalRepresentativeDeterminatorOfExternalSet( <xset> ) A

returns a function that
takes as arguments the acting group and the point. It returns a list
of length 3: [<canonrep>, <stabilizercanonrep>, <conjugatingelm>].
(List components 2 and 3 are optional and do not need to be bound.)
An external set is only guaranteed to be able to compute a canonical
representative if it has a
`CanonicalRepresentativeDeterminatorOfExternalSet'.

\>ActorOfExternalSet( <xset> ) A

returns an element mapping `Representative(<xset>)' to
`CanonicalRepresentativeOfExternalSet(<xset>)' under the given
action.


\beginexample
gap> u:=Subgroup(g,[(1,2,3)]);;
gap> e:=RightCoset(u,(1,2)(3,4));;
gap> CanonicalRepresentativeOfExternalSet(e);
(2,4,3)
gap> ActorOfExternalSet(e);
(1,3,2)
gap> FunctionAction(e)((1,2)(3,4),last);
(2,4,3)
\endexample

External sets also are implicitly underlying action homomorphisms:

\>UnderlyingExternalSet( <ohom> ) A

The underlying set of an action homomorphism is the external set on
which it was defined.

\beginexample
gap> g:=Group((1,2,3),(1,2));;
gap> hom:=ActionHomomorphism(g,Arrangements([1..4],3),OnTuples);;
gap> s:=UnderlyingExternalSet(hom);
<xset:[[ 1, 2, 3 ],[ 1, 2, 4 ],[ 1, 3, 2 ],[ 1, 3, 4 ],[ 1, 4, 2 ],
[ 1, 4, 3 ],[ 2, 1, 3 ],[ 2, 1, 4 ],[ 2, 3, 1 ],[ 2, 3, 4 ],[ 2, 4, 1 ],
[ 2, 4, 3 ],[ 3, 1, 2 ],[ 3, 1, 4 ],[ 3, 2, 1 ], ...]>
gap> Print(s,"\n");
[ [ 1, 2, 3 ], [ 1, 2, 4 ], [ 1, 3, 2 ], [ 1, 3, 4 ], [ 1, 4, 2 ], 
  [ 1, 4, 3 ], [ 2, 1, 3 ], [ 2, 1, 4 ], [ 2, 3, 1 ], [ 2, 3, 4 ], 
  [ 2, 4, 1 ], [ 2, 4, 3 ], [ 3, 1, 2 ], [ 3, 1, 4 ], [ 3, 2, 1 ], 
  [ 3, 2, 4 ], [ 3, 4, 1 ], [ 3, 4, 2 ], [ 4, 1, 2 ], [ 4, 1, 3 ], 
  [ 4, 2, 1 ], [ 4, 2, 3 ], [ 4, 3, 1 ], [ 4, 3, 2 ] ]
\endexample

\>SurjectiveActionHomomorphismAttr( <xset> ) A

returns an action homomorphism for <xset> which is surjective.
(As the `Image' of this homomorphism has to be computed to obtain the
range, this may take substantially longer
than `ActionHomomorphism'.)



%%%%%%%%%%%%%%%%%%%%%%%%%%%%%%%%%%%%%%%%%%%%%%%%%%%%%%%%%%%%%%%%%%%%%%%%%%%%%
%%
%E
