% This file was created automatically from vector.msk.
% DO NOT EDIT!
%%%%%%%%%%%%%%%%%%%%%%%%%%%%%%%%%%%%%%%%%%%%%%%%%%%%%%%%%%%%%%%%%%%%%%%%%%%%%
%%
%A  vector.msk                  GAP documentation            Martin Schoenert
%A                                                           Alexander Hulpke
%%
%A  @(#)$Id: vector.msk,v 1.23.2.1 2005/05/09 07:08:31 gap Exp $
%%
%Y  (C) 1998 School Math and Comp. Sci., University of St.  Andrews, Scotland
%Y  Copyright (C) 2002 The GAP Group
%%
\Chapter{Row Vectors}

Just as in mathematics, a vector in {\GAP} is any object which
supports appropriate addition and scalar multiplication operations
(see Chapter~"Vector Spaces"). As in mathematics, an especially important
class of vectors are those represented by a list of coefficients with
respect to some basis. These correspond roughly to the {\GAP} concept
of *row vectors*.

\>IsRowVector( <obj> ) C

A *row vector* is a vector (see~"IsVector") that is also a homogeneous
list of odd additive nesting depth
(see~"Filters Controlling the Arithmetic Behaviour of Lists").
Typical examples are lists of integers and rationals,
lists of finite field elements of the same characteristic,
and lists of polynomials from a common polynomial ring.
Note that matrices are *not* regarded as row vectors, because they have
even additive nesting depth.

The additive operations of the vector must thus be compatible with
that for lists, implying that the list entries are the
coefficients of the vector with respect to some basis.

Note that not all row vectors admit a multiplication via `\*'
(which is to be understood as a scalar product);
for example, class functions are row vectors but the product of two
class functions is defined in a different way.
For the installation of a scalar product of row vectors, the entries of
the vector must be ring elements; note that the default method expects
the row vectors to lie in `IsRingElementList', and this category may not
be implied by `IsRingElement' for all entries of the row vector
(see the comment for `IsVector' in~"IsVector").

Note that methods for special types of row vectors really must be
installed with the requirement `IsRowVector',
since `IsVector' may lead to a rank of the method below
that of the default method for row vectors (see file `lib/vecmat.gi').


\beginexample
gap> IsRowVector([1,2,3]);
true
\endexample

Because row vectors are just a special case of lists, all operations
and functions for lists are applicable to row vectors as well (see
Chapter~"Lists"). This especially includes accessing elements of a row
vector (see "List Elements"), changing elements of a mutable row
vector (see "List Assignment"), and comparing row vectors (see
"Comparisons of Lists").

Note that, unless your algorithms specifically require you to be able
to change entries of your vectors, it is generally better and faster
to work with immutable row vectors. See Section~"Mutability and
Copyability" for more details.

%%  The basic design of the row vector support in {\GAP} 4 is due to
%%  Martin Sch{\"o}nert. Frank Celler added the special support for
%%  vectors over the field of two elements; Steve Linton added special
%%  support for vectors over fields of sizes between 3 and 256; and Werner
%%  Nickel added special methods for vectors over large finite fields.


%%%%%%%%%%%%%%%%%%%%%%%%%%%%%%%%%%%%%%%%%%%%%%%%%%%%%%%%%%%%%%%%%%%%%%%%%%%%%
\Section{Operators for Row Vectors}

The rules for arithmetic operations involving row vectors are in fact
special cases of those for the arithmetic of lists,
as given in Section~"Arithmetic for Lists" and the following sections,
here we reiterate that definition, in the language of vectors.

Note that the additive behaviour sketched below is defined only for lists in
the category `IsGeneralizedRowVector',
and the multiplicative behaviour is defined only for lists in the category
`IsMultiplicativeGeneralizedRowVector'
(see~"Filters Controlling the Arithmetic Behaviour of Lists").


\>`<vec1> + <vec2>'{addition!vectors} O

returns the sum of the two row vectors <vec1> and <vec2>.
Probably the most usual situation is that <vec1> and <vec2> have the same
length and are defined over a common field;
in this case the sum is a new row vector over the same field where each entry
is the sum of the corresponding entries of the vectors.

In more general situations, the sum of two row vectors need not be a row
vector, for example adding an integer vector <vec1> and a vector <vec2> over
a finite field yields the list of pointwise sums,
which will be a mixture of finite field elements and integers if <vec1> is
longer than <vec2>.


\>`<scalar> + <vec>'{addition!scalar and vector} O
\>`<vec> + <scalar>'{addition!vector and scalar} O

returns the sum of the scalar <scalar> and the row vector <vec>.
Probably the most usual situation is that the elements of <vec> lie in a
common field with <scalar>;
in this case the sum is a new row vector over the same field where each entry
is the sum of the scalar and the corresponding entry of the vector.

More general situations are for example the sum of an integer scalar and a
vector over a finite field, or the sum of a finite field element and an
integer vector.

\beginexample
gap> [ 1, 2, 3 ] + [ 1/2, 1/3, 1/4 ];
[ 3/2, 7/3, 13/4 ]
gap>  [ 1/2, 3/2, 1/2 ] + 1/2;
[ 1, 2, 1 ]
\endexample


\>`<vec1> - <vec2>'{subtraction!vectors} O
\>`<scalar> - <vec>'{subtraction!scalar and vector} O
\>`<vec> - <scalar>'{subtraction!vector and scalar} O

Subtracting a vector or scalar is defined as adding its additive inverse,
so the statements for the addition hold likewise.

\beginexample
gap> [ 1, 2, 3 ] - [ 1/2, 1/3, 1/4 ];
[ 1/2, 5/3, 11/4 ]
gap> [ 1/2, 3/2, 1/2 ] - 1/2;
[ 0, 1, 0 ]
\endexample


\>`<scalar> * <vec>'{multiplication!scalar and vector} O
\>`<vec> * <scalar>'{multiplication!vector and scalar} O

returns the product of the scalar <scalar> and the row vector <vec>.
Probably the most usual situation is that the elements of <vec> lie in a
common field with <scalar>;
in this case the product is a new row vector over the same field where each
entry is the product of the scalar and the corresponding entry of the vector.

More general situations are for example the product of an integer scalar and
a vector over a finite field,
or the product of a finite field element and an integer vector.

\beginexample
gap> [ 1/2, 3/2, 1/2 ] * 2;
[ 1, 3, 1 ]
\endexample


\>`<vec1> * <vec2>'{multiplication!vectors} O

returns the standard scalar product of <vec1> and <vec2>,
i.e., the sum of the products of the corresponding entries of the vectors.
Probably the most usual situation is that <vec1> and <vec2> have the same
length and are defined over a common field;
in this case the sum is an element of this field.

More general situations are for example the inner product of an integer
vector and a vector over a finite field,
or the inner product of two row vectors of different lengths.

\beginexample
gap> [ 1, 2, 3 ] * [ 1/2, 1/3, 1/4 ];
23/12
\endexample

For the mutability of results of arithmetic operations,
see~"Mutability and Copyability".

Further operations with vectors as operands are defined by the matrix
operations (see~"Operators for Matrices").

\>NormedRowVector( <v> ) A

returns a scalar multiple `<w> = <c> \* <v>' of the row vector <v>
with the property that the first nonzero entry of <w> is an identity
element in the sense of `IsOne'.


\beginexample
gap> NormedRowVector([5,2,3]);
[ 1, 2/5, 3/5 ]
\endexample


%%%%%%%%%%%%%%%%%%%%%%%%%%%%%%%%%%%%%%%%%%%%%%%%%%%%%%%%%%%%%%%%%%%%%%%%%
\Section{Row Vectors over Finite Fields}

{\GAP} can use compact formats to store row vectors over fields of
order at most 256, based on those used by the Meat-Axe
\cite{Rin93}. This format also permits extremely efficient vector
arithmetic. On the other hand element access and assignment is
significantly slower than for plain lists.

The function `ConvertToVectorRep' is used to convert a list into a
compressed vector, or to rewrite a compressed vector over another
field. Note that this function is *much* faster when it is given a
field (or field size) as an argument, rather than having to scan the
vector and try to decide the field. Supplying the field can also
avoid errors and/or loss of performance, when one vector from some
collection happens to have all of its entries over a smaller field
than the \"natural\" field of the problem.

\>ConvertToVectorRep( <list> ) F
\>ConvertToVectorRep( <list> , <field> ) F
\>ConvertToVectorRep( <list> , <fieldsize> ) F
\>ConvertToVectorRepNC( <list> ) F
\>ConvertToVectorRepNC( <list> , <field> ) F
\>ConvertToVectorRepNC( <list> , <fieldsize> ) F

`ConvertToVectorRep( <list> )' converts <list> to an internal
vector representation if possible.

`ConvertToVectorRep( <list> , <field> )' converts <list> to an
internal vector representation appropriate for a vector over
<field>.

It is forbidden to call this function unless <list> is a plain
list or a vector, <field> a field, and all elements
of <list> lie in <field>, violation of this condition can lead to
unpredictable behaviour or a system crash. (Setting the assertion level
to at least 2 might catch some violations before a crash,
see~"SetAssertionLevel".)

Instead of a <field> also its size <fieldsize> may be given.

<list> may already be a compressed vector. In this case, if no
<field> or <fieldsize> is given, then nothing happens. If one is
given then the vector is rewritten as a compressed vector over the
given <field> unless it has the filter
`IsLockedRepresentationVector', in which case it is not changed.

The return value is the size of the field over which the vector
ends up written, if it is written in a compressed representation.




In this example, we first create a row vector and then ask {\GAP} to
rewrite it, first over GF(2) and then over GF(4).

\beginexample
gap> v := [Z(2)^0,Z(2),Z(2),0*Z(2)];
[ Z(2)^0, Z(2)^0, Z(2)^0, 0*Z(2) ]
gap> RepresentationsOfObject(v);
[ "IS_PLIST_REP", "IsInternalRep" ]
gap> ConvertToVectorRep(v);
2
gap> v;
<a GF2 vector of length 4>
gap> ConvertToVectorRep(v,4);
4
gap> v;
[ Z(2)^0, Z(2)^0, Z(2)^0, 0*Z(2) ]
gap> RepresentationsOfObject(v);
[ "IsDataObjectRep", "Is8BitVectorRep" ]
\endexample

A vector in the special representation over $GF(2)$ is always viewed
as `\<a GF2 vector of length ...>'. Over fields of orders 3 to 256, a
vector of length 10 or less is viewed as the list of its coefficients,
but a longer one is abbreviated.

Arithmetic operations (see~"Arithmetic for Lists" and the following
sections) preserve the compression status of row vectors in the sense that
if all arguments are compressed row vectors written over the same field and
the result is a row vector then also the result is a compressed row vector
written over this field.

\>NumberFFVector( <vec>, <sz> ) O

returns an integer that gives the position of the finite field row vector
(<vec>) in the sorted list of all row vectors over the field with <sz>
elements in the same dimension as <vec>. `NumberFFVector' returns `fail'
if the vector cannot be represented over the field with <sz> elements.



%%%%%%%%%%%%%%%%%%%%%%%%%%%%%%%%%%%%%%%%%%%%%%%%%%%%%%%%%%%%%%%%%%%%%%%%%%%%%
\Section{Coefficient List Arithmetic}

The following operations all perform arithmetic on row vectors.
given as homogeneous lists of the same length, containing
elements of a commutative ring.

There are two reasons for using  `AddRowVector'
in preference to arithmetic operators. Firstly, the three argument 
form has no single-step equivalent. Secondly
`AddRowVector' changes its first argument in-place, rather than allocating
a new vector to hold the result, and may thus produce less garbage.


\>AddRowVector( <dst>, <src>, [ <mul> [, <from>, <to>]] ) O

Adds the product of <src> and <mul> to <dst>, changing <dst>.
If <from> and <to> are given then only the index range `[<from>..<to>]' is
guaranteed to be affected. Other indices MAY be affected, if it is 
more convenient to do so. Even when <from> and <to> are given,
<dst> and <src> must be row vectors of the *same* length.

If <mul> is not given either then this Operation simply adds <src> to <dst>.



\>AddCoeffs( <list1>, <poss1>, <list2>, <poss2>, <mul> ) O
\>AddCoeffs( <list1>, <list2>, <mul> ) O
\>AddCoeffs( <list1>, <list2> ) O

`AddCoeffs' adds the entries  of `<list2>\{<poss2>\}', multiplied by the
scalar <mul>, to
`<list1>\{<poss1>\}'.  Non-existing  entries  in  <list1> are assumed  to  be
zero.  The position of the right-most non-zero element is returned.

If the ranges <poss1> and <poss2> are not given, they are assumed to
span the whole vectors. If the scalar <mul> is omitted, one is used as a
default.

Note  that it is  the responsibility  of the  caller  to ensure  that the
<list2> has elements at position <poss2> and that the result (in <list1>)
will be a dense list.

The function is free to remove trailing (right-most) zeros.


\beginexample
gap> l:=[1,2,3,4];;m:=[5,6,7];;AddCoeffs(l,m);
4
gap> l;
[ 6, 8, 10, 4 ]
\endexample

\>MultRowVector( <list1>, <poss1>, <list2>, <poss2>, <mul> ) O
\>MultRowVector( <list>, <mul> ) O

The five-argument version of this Operation replaces
`<list1>[<poss1>[<i>]]' by `<mul>*<list2>[<poss2>[<i>]]' for <i>
between 1 and `Length(<poss1>)'.

The two-argument version simply multiplies each element of <list>, 
in-place, by <mul>.


\>CoeffsMod( <list1>, [<len1>, ] <mod> ) O

returns the coefficient list obtained by reducing the entries in <list1>
modulo <mod>. After reducing it shrinks the list to remove trailing
zeroes.

\beginexample
gap> l:=[1,2,3,4];;CoeffsMod(l,2);
[ 1, 0, 1 ]
\endexample

%%%%%%%%%%%%%%%%%%%%%%%%%%%%%%%%%%%%%%%%%%%%%%%%%%%%%%%%%%%%%%%%%%%%%%%%%%%%%
\Section{Shifting and Trimming Coefficient Lists}

The following functions change coefficient lists by shifting or
trimming.

\>LeftShiftRowVector( <list>, <shift> ) O

changes <list> by assigning
`<list>[i]:=<list>[i+<shift>]' and removing the last <shift> entries of
the result.

\>RightShiftRowVector( <list>, <shift>, <fill> ) O

changes <list> by assigning
`<list>[i+<shift>]:=<list>[i]' and filling each of the <shift> first
entries with <fill>.

\>ShrinkRowVector( <list> ) O

removes trailing zeroes from the list <list>.



\>RemoveOuterCoeffs( <list>, <coef> ) O

removes <coef> at the beginning and at the end of <list> and returns the
number of elements removed at the beginning.

\beginexample
gap> l:=[1,1,2,1,2,1,1,2,1];;RemoveOuterCoeffs(l,1);
2
gap> l;
[ 2, 1, 2, 1, 1, 2 ]
\endexample

%%%%%%%%%%%%%%%%%%%%%%%%%%%%%%%%%%%%%%%%%%%%%%%%%%%%%%%%%%%%%%%%%%%%%%%%%%%%%
\Section{Functions for Coding Theory}

The following functions perform operations on Finite fields vectors
considered as code words in a linear code.


\>WeightVecFFE( <vec> ) O

returns the weight of the finite field vector <vec>, i.e. the number of
nonzero entries.

\>DistanceVecFFE( <vec1>, <vec2> ) O

returns the distance between the two vectors <vec1> and <vec2>, which
must have the same length and whose elements must lie in a common field.
The distance is the number of places where <vec1> and <vec2> differ.

\>DistancesDistributionVecFFEsVecFFE( <vecs>, <vec> ) O

returns the distances distribution of the vector <vec> to the vectors in
the list <vecs>. All vectors must have the same length, and all elements
must lie in a common field. The distances distribution is a list <d> of
length `Length(<vec>)+1', such that the value `<d>[<i>]' is the number
of vectors in <vecs> that have distance `<i>+1' to <vec>.

\>DistancesDistributionMatFFEVecFFE( <mat>, <f>, <vec> ) O

returns the distances distribution of the vector <vec> to the vectors in
the vector space generated by the rows of the matrix <mat> over the
finite field <f>. The length of the rows of <mat> and the length of
<vec> must be equal, and all elements must lie in <f>. The rows of <mat>
must be linearly independent. The distances distribution is a list <d>
of length `Length(<vec>)+1', such that the value `<d>[<i>]' is the
number of vectors in the vector space generated by the rows of <mat>
that have distance `<i>+1' to <vec>.

\>AClosestVectorCombinationsMatFFEVecFFE( <mat>, <f>, <vec>, <l>, <stop> ) O
\>AClosestVectorCombinationsMatFFEVecFFECoords( <mat>, <f>, <vec>, <l>, <stop> ) O

These functions run through the <f>-linear combinations of the
vectors in the rows of the matrix <mat> that can be written as
linear combinations of exactly <l> rows (that is without using
zero as a coefficient). The length of the rows of <mat> and the
length of <vec> must be equal, and all elements must lie in
<f>. The rows of <mat> must be linearly
independent. `AClosestVectorCombinationsMatFFEVecFFE' returns a
vector from these that is closest to the vector <vec>. If it finds
a vector of distance at most <stop>, which must be a nonnegative
integer, then it stops immediately and returns this vector.

`AClosestVectorCombinationsMatFFEVecFFECoords' returns a length 2
list
containing the same closest vector and also a vector <v> with exactly <l> non-zero
entries, such that <v> times <mat> is the closest vector.


\>CosetLeadersMatFFE( <mat>, <f> ) O

returns a list of representatives of minimal weight for the cosets of a
code. <mat> must be a *check matrix* for the code, the code is defined
over the finite field <f>.   All rows of <mat> must have the same
length, and all elements must lie in <f>. The rows of <mat> must be
linearly independent.


%%%%%%%%%%%%%%%%%%%%%%%%%%%%%%%%%%%%%%%%%%%%%%%%%%%%%%%%%%%%%%%%%%%%%%%%%%%%%
\Section{Vectors as coefficients of polynomials}


A list of ring elements can be interpreted as a row vector or the list of
coefficients of a polynomial. There are a couple of functions that implement
arithmetic operations based on these interpretations. {\GAP} contains proper
support for polynomials (see~"Polynomials and Rational Functions"), the
operations described in this section are on a lower level.

The following operations all perform arithmetic on univariate
polynomials given by their coefficient lists. These lists can have
different lengths but must be dense homogeneous lists containing
elements of a commutative ring.
Not all input lists may be empty.

In the following descriptions we will always assume that <list1> is the
coefficient list of the polynomial <pol1> and so forth.
If length parameter <leni> is not given, it is set to the length of
<listi> by default.


\>ValuePol( <coeff>, <x> ) F

Let <coeff> be the coefficients list of a univariate polynomial $f$,
and <x> a ring element. Then
`ValuePol' returns the value $f(<x>)$.

The coefficient of $x^i$ is assumed to be stored at position $i+1$ in
the coefficients list.


\beginexample
gap> ValuePol([1,2,3],4);
57
\endexample

%\ Declaration{ProductPol}
%\ beginexample
%gap> ProductPol([1,2,3],[4,5,6]);
%[ 4, 13, 28, 27, 18 ]
%\ endexample

%\ Declaration{MultCoeffs}
%\ beginexample
%gap> a:=[];;l:=[1,2,3,4];;m:=[5,6,7];;
%gap> MultCoeffs(a,l,4,m,3);
%6
%gap> a;
%[ 5, 16, 34, 52, 45, 28 ]
%\ endexample

\>ProductCoeffs( <list1>, [<len1>, ] <list2> [, <len2>] ) O

Let <pol1> (and <pol2>) be polynomials given by the first <len1> (<len2>)
entries of the coefficient list <list2> (<list2>).
If <len1> and <len2> are omitted, they default to the lengths of <list1>
and <list2>.
This operation returns the coefficient list of the product of <pol1> and
<pol2>.

\beginexample
gap> l:=[1,2,3,4];;m:=[5,6,7];;ProductCoeffs(l,m);
[ 5, 16, 34, 52, 45, 28 ]
\endexample

\>ReduceCoeffs( <list1> [, <len1>], <list2> [, <len2>] ) O

changes <list1> to the coefficient list of the remainder when dividing
<pol1> by <pol2>.
This operation changes <list1> which therefore must be a mutable list.
The operations returns the position of the last non-zero entry of the
result but is not guaranteed to remove trailing zeroes.

\beginexample
gap> l:=[1,2,3,4];;m:=[5,6,7];;ReduceCoeffs(l,m);
2
gap> l;
[ 64/49, -24/49, 0, 0 ]
\endexample

\>ReduceCoeffsMod( <list1>, [<len1>, ] <list2>, [<len2>, ] <mod> ) O

changes <list1> to the coefficient list of the remainder when dividing
<pol1> by <pol2> modulo <mod>. <mod> must be a positive integer.
This operation changes <list1> which therefore must be a mutable list.
The operations returns the position of the last non-zero entry of the
result but is not guaranteed to remove trailing zeroes.

\beginexample
gap> l:=[1,2,3,4];;m:=[5,6,7];;ReduceCoeffsMod(l,m,3);
1
gap> l;
[ 1, 0, 0, 0 ]
\endexample

\>PowerModCoeffs( <list1>[, <len1>], <exp>, <list2>[, <len2>] ) O

Let $p_1$ and $p_2$ be polynomials whose coefficients are given by the
first <len1> resp. <len2> entries of the lists <list1> and <list2>,
respectively.
If <len1> and <len2> are omitted, they default to the lengths of <list1>
and <list2>.
Let <exp> be a positive integer.
`PowerModCoeffs' returns the coefficient list of the remainder
when dividing the <exp>-th power of $p_1$ by $p_2$.
The coefficients are reduced already while powers are computed,
therefore avoiding an explosion in list length.


\beginexample
gap> l:= [1,2,3,4];; m:= [5,6,7];; PowerModCoeffs(l,5,m);
[ -839462813696/678223072849, -7807439437824/678223072849 ]
gap> EuclideanRemainder( UnivariatePolynomial( Rationals, l )^5,
>        UnivariatePolynomial( Rationals, m ) );
-7807439437824/678223072849*x_1-839462813696/678223072849
\endexample

\>ShiftedCoeffs( <list>, <shift> ) O

produces a new coefficient list <new> obtained by the rule
`<new>[i+<shift>]:=<list>[i]' and filling initial holes by the
appropriate zero.

\beginexample
gap> l:=[1,2,3];;ShiftedCoeffs(l,2);ShiftedCoeffs(l,-2);
[ 0, 0, 1, 2, 3 ]
[ 3 ]
\endexample

\>ShrinkCoeffs( <list> ) O

removes trailing zeroes from <list>. It returns the position of the last
non-zero entry, that is the length of <list> after the operation.

\beginexample
gap> l:=[1,0,0];;ShrinkCoeffs(l);l;
1
[ 1 ]
\endexample


%%%%%%%%%%%%%%%%%%%%%%%%%%%%%%%%%%%%%%%%%%%%%%%%%%%%%%%%%%%%%%%%%%%%%%%%%%%%%
%%
%E

