% This file was created automatically from README.msk.
% DO NOT EDIT!
                  README for GAP 4.4.10

These archives contain the current release of GAP. It is version
4.4.10.  They also contain a number of contributed
packages, found in the subdirectories pkg and small of the main
distribution directory. Each package is a separate work for copyright
and license purposes, included in this archive for the convenience of
the user. The main GAP system is distributed under the GPL version 2,
packages may have other licenses. Check the documentation and/or the
website for more details.

All files are available in the following formats:

  - extension .tar.gz  -  standard format for UNIX/Linux/OS X systems 
          or .tar.bz2     (use gunzip/bunzip2 and tar for unpacking)

  - extension .zoo     -  all systems, use the provided
                          util/unzoo.c, (UNIX, OS X)
                          bin/mac/unzoo4r4-PPC.sit  (MacOS PPC)
                          bin/win/unzoo.exe, (Windows)
                          for unpacking

  - extension -win.zip -  standard format for Windows/DOS systems
                          (here text files have DOS-style line
                           breaks, they might not work under other
			   architectures)

The following contains brief installation instructions aimed at experts.
More detailed informations can be found on the GAP web pages 
(http://www.gap-system.org) and in the `Installation'' chapter of
the GAP reference manual. You can find an HTML version of this manual also
on the GAP web pages:
http://www.gap-system.org/tmpsite/Manuals/docs/doc/htm/ref/CHAP073.htm


If after reading the information on the web pages you still face
unsurmountable problems, please send an email to
support@gap-system.org, describing your problems. (Include the
operating system you are using and the error messages or indications of
installation failure that you get.)

Once installation is complete, please send a short note to
support@gap-system.org to inform us of the installation.

For a complete installation fetch the files

   gap4r4p10    ....   The main GAP system
   packages....  ....   Contributed packages

(with the extensions of your choice, see above).
Note,  that certain  packages  will not  or  not fully  work  on Windows  or
Macintosh systems, see the package infos on our Web pages for details.
   
and  (optional additional tables of marks, need about 80 MB on disc)

   xtom4r4p10 ....

(In contrast to older versions, the installation archives provided will
always contain the most recent bugfix -- if you install GAP afresh you do not
need to fetch any bugfix.)

INSTALLATION UNDER UNIX AND OS X

While the `gap4r4p10' archive must be unpacked *in* the directory
in which you want GAP's root directory to sit (say, '/usr/local/lib/', a
subdirectory 'gap4r4' is created), the
`packages...' archive must be extracted into the
`gap4r4/pkg' subdirectory created by the first extraction.

To compile GAP on UNIX/Linux/OS X systems you can often just say

  ./configure
  make

inside  GAP's root  directory and  copy  the start  script bin/gap.sh  under
some  name (probably  'gap')  into  a generally  used  bin directory  (e.g.,
/usr/local/bin).

More information on compilation and optimization options can be found on the
webpage:

http://www.math.colostate.edu/~hulpke/osx

Note  that some  packages need  further installation/configuration.  See the
README.* files in the 'pkg' subdirectory.

INSTALLATION UNDER WINDOWS

Decompress  'gap4r4p10-win.zip'   on  the  top-level  directory   of  drive  C:\.
This  creates  a  directory  C:\GAP4R4,   the  'GAP  root  directory'.  Then
extract  'packages...-win.zip' into  C:\GAP4R4\pkg\.   
Click  on C:\gap4r4p10\bin\gapw95.exe to start GAP.
(It is possible that using the rxvt shell via
C:\GAP4R4\bin\gaprxvt.bat will give a nicer user interface. 
However, please note that we cannot provide support for this shell.)

To make  GAP accessible  from the  start menu,  create a  link 'GAP
4.4.10' to
C:\GAP4R4\bin\gapw95.exe in a submenu of the start menu of your choice.

(If you want to install GAP in  another directory, you will have to create a
suitable batch file  -- see the `Installation' chapter of the reference
manual on the web pages.)

For getting  access to the  documentation via the  start menu, you  can also
create such  links to the manual books, e.g. to C:\GAP4R4\doc\ref\manual.pdf
resp. C:\GAP4R4\doc\htm\ref\chapters.htm
for the  PDF- resp. HTML  version of the  reference manual.  Of course,
this requires viewers for the respective types of files to be installed.

INSTALLATION ON MAC OS

Decompress  'unzoo4r4-...sit' (using,  e. g.  , Stuffit
Expander). You should now have an application 'unzoo'.

Move this application and gap4r4p10.zoo  into the  folder in
which you want  to install  GAP and decompress  it by dragging it onto the
'unzoo' application icon.

.  This  should  have  created  a  folder  'gap4r4p10' with a 
'GAP 4 ...' application in this directory.

To install the GAP packages,  move 'packages....zoo' into the  
folder 'pkg' inside 'gap4r4' and decompress them there.

  The GAP Group

