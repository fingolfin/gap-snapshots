% This file was created automatically from options.msk.
% DO NOT EDIT!
%%%%%%%%%%%%%%%%%%%%%%%%%%%%%%%%%%%%%%%%%%%%%%%%%%%%%%%%%%%%%%%%%%%%%%%%%%%%%
%%
%A  options.msk                    GAP documentation             Steve Linton
%%
%A  @(#)$Id: options.msk,v 1.9 2002/06/07 14:04:42 gap Exp $
%%
%Y  (C) 1998 School Math and Comp. Sci., University of St.  Andrews, Scotland
%Y  Copyright (C) 2002 The GAP Group
%%
\Chapter{Options Stack}

{\GAP} Supports a global Options system. This is intended as a
way for the user to provide guidance to various algorithms that
might be used in a computation. Such guidance should not change
mathematically the specification of the computation to be
performed, although it may change the algorithm used. A typical
example is the selection of a strategy for the Todd-Coxeter coset
enumeration procedure. An example of something not suited to the
options mechanism is the imposition of exponent laws in the
$p$-Quotient algorithm.

The basis of this system is a global stack of records. All the
entries of each record are thought of as options settings, and the 
effective setting of an option is given by the topmost record
in which the relevant field is bound.

The reason for the choice of a stack is the intended pattern of use:

\begintt
PushOptions( rec( <stuff> ) );
DoSomething( <args> );
PopOptions();
\endtt

This can be abbreviated, to `DoSomething( <args> : <stuff> );' with
a small additional abbreviation of <stuff> permitted. See
"ref:function call!with options" for details. The full form
can be used where the same options are to run across several
calls, or where the `DoSomething' procedure is actually a binary
operation, or other function with special syntax. 

At some time, an options predicate or something of the kind may
be added to method selection.

An alternative to this system is the use of additional optional
arguments in procedure calls. This is not felt to be adequate
because many procedure calls might cause, for example, a coset
enumeration and each would need to make provision for the
possibility of extra arguments. In this system the options are
pushed when the user-level procedure is called, and remain in
effect (unless altered) for all procedures called by it.




\>PushOptions( <options record> ) F

This function pushes a record of options onto the global option
stack. Note that `PushOption(rec(<opt> := fail))' has the effect of
resetting option <opt>, since an option that has never been set
has the value `fail' returned by `ValueOptions'.

Note that there is no check for misspelt or undefined options.


\>PopOptions( ) F

This function removes the top-most options record from the options stack.



\>ResetOptionsStack( ) F

unbinds (i.e. removes) all the options records from the options stack.

*Note:*
`ResetOptionsStack' should *not* be used within a function. Its  intended
use is to clean up the options stack in  the  event  that  the  user  has
`quit' from a `break'  loop,  so  leaving  a  stack  of  no-longer-needed
options (see~"quit").



\>ValueOption( <opt> ) F

This function is the main method of accessing the Options Stack;
<opt> should be the name of an option, i.e.~a string. A 
function which makes decisions which might be affected by options should
examine the result of `ValueOption( <opt> )'. If <opt> has never
been set then `fail' is returned.



\>DisplayOptionsStack( ) F

This function prints a human-readable display of the complete
options stack.




\>`InfoOptions' V

This info class can be used to enable messages about options being 
changed (level 1) or accessed (level 2).




The example below shows simple manipulation of the Options Stack,
first using `PushOptions' and `PopOptions' and then using the special
function calling syntax. 

\beginexample
gap> foo := function()
> Print("myopt1 = ", ValueOption("myopt1"),
>       " myopt2 = ",ValueOption("myopt2"),"\n");
> end;
function(  ) ... end
gap> foo();
myopt1 = fail myopt2 = fail
gap> PushOptions(rec(myopt1 := 17));
gap> foo();
myopt1 = 17 myopt2 = fail
gap> DisplayOptionsStack();
[ rec(
      myopt1 := 17 ) ]
gap> PopOptions();
gap> foo();
myopt1 = fail myopt2 = fail
gap> foo( : myopt1, myopt2 := [Z(3),"aardvark"]);
myopt1 = true myopt2 = [ Z(3), "aardvark" ]
gap> DisplayOptionsStack();
[  ]
gap> 
\endexample

%%%%%%%%%%%%%%%%%%%%%%%%%%%%%%%%%%%%%%%%%%%%%%%%%%%%%%%%%%%%%%%%%%%%%%%%%%%%
%%
%E

