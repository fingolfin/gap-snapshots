% generated by GAPDoc2LaTeX from XML source (Frank Luebeck)
\documentclass[a4paper,11pt]{report}
\usepackage{a4wide}
\sloppy
\pagestyle{myheadings}
\usepackage{amssymb}
\usepackage[utf8]{inputenc}
\usepackage{makeidx}
\makeindex
\usepackage{color}
\definecolor{DarkOlive}{rgb}{0.1047,0.2412,0.0064}
\definecolor{FireBrick}{rgb}{0.5812,0.0074,0.0083}
\definecolor{RoyalBlue}{rgb}{0.0236,0.0894,0.6179}
\definecolor{RoyalGreen}{rgb}{0.0236,0.6179,0.0894}
\definecolor{RoyalRed}{rgb}{0.6179,0.0236,0.0894}
\definecolor{LightBlue}{rgb}{0.8544,0.9511,1.0000}
\definecolor{Black}{rgb}{0.0,0.0,0.0}
\definecolor{FuncColor}{rgb}{1.0,0.0,0.0}
%% strange name because of pdflatex bug:
\definecolor{Chapter }{rgb}{0.0,0.0,1.0}

\usepackage{fancyvrb}

\usepackage{pslatex}

\usepackage[pdftex=true,
        a4paper=true,bookmarks=false,pdftitle={Written with GAPDoc},
        pdfcreator={LaTeX with hyperref package / GAPDoc},
        colorlinks=true,backref=page,breaklinks=true,linkcolor=RoyalBlue,
        citecolor=RoyalGreen,filecolor=RoyalRed,
        urlcolor=RoyalRed,pagecolor=RoyalBlue]{hyperref}

% write page numbers to a .pnr log file for online help
\newwrite\pagenrlog
\immediate\openout\pagenrlog =\jobname.pnr
\immediate\write\pagenrlog{PAGENRS := [}
\newcommand{\logpage}[1]{\protect\write\pagenrlog{#1, \thepage,}}
%% were never documented, give conflicts with some additional packages


\newcommand{\GAP}{\textsf{GAP}}

%% nicer description environments, allows long labels
\usepackage{enumitem}
\setdescription{style=nextline}

\begin{document}

\logpage{[ 0, 0, 0 ]}
\begin{titlepage}
\begin{center}{\Huge \textbf{\textsf{ExamplesForHomalg}\mbox{}}}\\[1cm]
\hypersetup{pdftitle=\textsf{ExamplesForHomalg}}
\markright{\scriptsize \mbox{}\hfill \textsf{ExamplesForHomalg} \hfill\mbox{}}
{\Large \textbf{Examples for the \textsf{GAP} package \textsf{homalg}\mbox{}}}\\[1cm]
{Version 2011.12.15\mbox{}}\\[1cm]
{December 2011\mbox{}}\\[1cm]
\mbox{}\\[2cm]
{\large \textbf{Mohamed Barakat\\
    \mbox{}}}\\
{\large \textbf{Simon G{\"o}rtzen\\
    \mbox{}}}\\
{\large \textbf{Markus Lange-Hegermann\\
    \mbox{}}}\\
\hypersetup{pdfauthor=Mohamed Barakat\\
    ; Simon G{\"o}rtzen\\
    ; Markus Lange-Hegermann\\
    }
\mbox{}\\[2cm]
\begin{minipage}{12cm}\noindent
(\emph{this manual is still under construction}) \\
\\
 This manual is best viewed as an \textsc{HTML} document. The latest version is available \textsc{online} at: \\
\\
 \href{http://homalg.math.rwth-aachen.de/~barakat/ExamplesForHomalg/homalg-project/chap0.html} {\texttt{http://homalg.math.rwth-aachen.de/\texttt{\symbol{126}}barakat/ExamplesForHomalg/homalg-project/chap0.html}} \\
\\
 An \textsc{offline} version should be included in the documentation subfolder of the package. This
package is part of the \textsf{homalg}-project: \\
\\
 \href{http://homalg.math.rwth-aachen.de/index.php/core-packages/examplesforhomalg} {\texttt{http://homalg.math.rwth-aachen.de/index.php/core-packages/examplesforhomalg}} \end{minipage}

\end{center}\vfill

\mbox{}\\
{\mbox{}\\
\small \noindent \textbf{Mohamed Barakat\\
    } --- Email: \href{mailto://barakat@mathematik.uni-kl.de} {\texttt{barakat@mathematik.uni-kl.de}}\\
 --- Homepage: \href{http://www.mathematik.uni-kl.de/~barakat/} {\texttt{http://www.mathematik.uni-kl.de/\texttt{\symbol{126}}barakat/}}\\
 --- Address: \begin{minipage}[t]{8cm}\noindent
 Department of Mathematics, \\
 University of Kaiserslautern, \\
 67653 Kaiserslautern, \\
 Germany \end{minipage}
}\\
{\mbox{}\\
\small \noindent \textbf{Simon G{\"o}rtzen\\
    } --- Email: \href{mailto://simon.goertzen@rwth-aachen.de} {\texttt{simon.goertzen@rwth-aachen.de}}\\
 --- Homepage: \href{http://wwwb.math.rwth-aachen.de/goertzen/} {\texttt{http://wwwb.math.rwth-aachen.de/goertzen/}}\\
 --- Address: \begin{minipage}[t]{8cm}\noindent
 Lehrstuhl B f{\"u}r Mathematik, RWTH Aachen, Templergraben 64, 52056 Aachen,
Germany \end{minipage}
}\\
{\mbox{}\\
\small \noindent \textbf{Markus Lange-Hegermann\\
    } --- Email: \href{mailto://markus.lange.hegermann@rwth-aachen.de} {\texttt{markus.lange.hegermann@rwth-aachen.de}}\\
 --- Homepage: \href{http://wwwb.math.rwth-aachen.de/~markus/} {\texttt{http://wwwb.math.rwth-aachen.de/\texttt{\symbol{126}}markus/}}\\
 --- Address: \begin{minipage}[t]{8cm}\noindent
 Lehrstuhl B f{\"u}r Mathematik, RWTH Aachen, Templergraben 64, 52056 Aachen,
Germany \end{minipage}
}\\
\end{titlepage}

\newpage\setcounter{page}{2}
{\small 
\section*{Copyright}
\logpage{[ 0, 0, 1 ]}
 {\copyright} 2008-2011 by Mohamed Barakat, Simon Goertzen, Markus
Lange-Hegermann

 This package may be distributed under the terms and conditions of the GNU
Public License Version 2. \mbox{}}\\[1cm]
\newpage

\def\contentsname{Contents\logpage{[ 0, 0, 2 ]}}

\tableofcontents
\newpage

 \index{\textsf{ExamplesForHomalg}}   
\chapter{\textcolor{Chapter }{Introduction}}\label{intro}
\logpage{[ 1, 0, 0 ]}
\hyperdef{L}{X7DFB63A97E67C0A1}{}
{
  \cite{homalg-package} }

   
\chapter{\textcolor{Chapter }{Installation of the \textsf{ExamplesForHomalg} Package}}\label{install}
\logpage{[ 2, 0, 0 ]}
\hyperdef{L}{X802B637185D24732}{}
{
  To install this package just extract the package's archive file to the \textsf{GAP} \texttt{pkg} directory.

 By default the \textsf{ExamplesForHomalg} package is not automatically loaded by \textsf{GAP} when it is installed. You must load the package with \\
\\
 \texttt{LoadPackage("ExamplesForHomalg");} \\
\\
 before its functions become available.

 Please, send us an e-mail if you have any questions, remarks, suggestions,
etc. concerning this package. Also, I would be pleased to hear about
applications of this package. \\
\\
\\
 Mohamed Barakat and Simon G{\"o}rtzen.  }

  
\chapter{\textcolor{Chapter }{Examples}}\label{examples}
\logpage{[ 3, 0, 0 ]}
\hyperdef{L}{X7A489A5D79DA9E5C}{}
{
  
\section{\textcolor{Chapter }{Spectral Filtrations}}\label{SpectralFiltrations}
\logpage{[ 3, 1, 0 ]}
\hyperdef{L}{X7AD67ACA80E44216}{}
{
  
\subsection{\textcolor{Chapter }{ExtExt}}\label{ExtExt}
\logpage{[ 3, 1, 1 ]}
\hyperdef{L}{X7BB9DE017ECE6E86}{}
{
  This is Example B.2 in \cite{BaSF}. 
\begin{Verbatim}[fontsize=\small,frame=single,label=Example]
  gap> Qxyz := HomalgFieldOfRationalsInDefaultCAS( ) * "x,y,z";
  Q[x,y,z]
  gap> wmat := HomalgMatrix( "[ \
  > x*y,  y*z,    z,        0,         0,    \
  > x^3*z,x^2*z^2,0,        x*z^2,     -z^2, \
  > x^4,  x^3*z,  0,        x^2*z,     -x*z, \
  > 0,    0,      x*y,      -y^2,      x^2-1,\
  > 0,    0,      x^2*z,    -x*y*z,    y*z,  \
  > 0,    0,      x^2*y-x^2,-x*y^2+x*y,y^2-y \
  > ]", 6, 5, Qxyz );
  <A 6 x 5 matrix over an external ring>
  gap> W := LeftPresentation( wmat );
  <A left module presented by 6 relations for 5 generators>
  gap> Y := Hom( Qxyz, W );
  <A right module on 5 generators satisfying yet unknown relations>
  gap> F := InsertObjectInMultiFunctor( Functor_Hom_for_fp_modules, 2, Y, "TensorY" );
  <The functor TensorY for f.p. modules and their maps over computable rings>
  gap> G := LeftDualizingFunctor( Qxyz );;
  gap> II_E := GrothendieckSpectralSequence( F, G, W );
  <A stable homological spectral sequence with sheets at levels 
  [ 0 .. 4 ] each consisting of left modules at bidegrees [ -3 .. 0 ]x
  [ 0 .. 3 ]>
  gap> Display( II_E );
  The associated transposed spectral sequence:
  
  a homological spectral sequence at bidegrees
  [ [ 0 .. 3 ], [ -3 .. 0 ] ]
  ---------
  Level 0:
  
   * * * *
   * * * *
   . * * *
   . . * *
  ---------
  Level 1:
  
   * * * *
   . . . .
   . . . .
   . . . .
  ---------
  Level 2:
  
   s s s s
   . . . .
   . . . .
   . . . .
  
  Now the spectral sequence of the bicomplex:
  
  a homological spectral sequence at bidegrees
  [ [ -3 .. 0 ], [ 0 .. 3 ] ]
  ---------
  Level 0:
  
   * * * *
   * * * *
   . * * *
   . . * *
  ---------
  Level 1:
  
   * * * *
   * * * *
   . * * *
   . . . *
  ---------
  Level 2:
  
   * * s s
   * * * *
   . * * *
   . . . *
  ---------
  Level 3:
  
   * s s s
   * s s s
   . . s *
   . . . *
  ---------
  Level 4:
  
   s s s s
   . s s s
   . . s s
   . . . s
  gap> filt := FiltrationBySpectralSequence( II_E, 0 );
  <An ascending filtration with degrees [ -3 .. 0 ] and graded parts:
  
  0:	<A non-zero left module presented by yet unknown relations for 23 generator\
  s>
    -1:	<A non-zero left module presented by 37 relations for 22 generators>
    -2:	<A non-zero left module presented by 31 relations for 10 generators>
    -3:	<A non-zero left module presented by 32 relations for 5 generators>
  of
  <A non-zero left module presented by 111 relations for 37 generators>>
  gap> ByASmallerPresentation( filt );
  <An ascending filtration with degrees [ -3 .. 0 ] and graded parts:
     0:	<A non-zero left module presented by 25 relations for 16 generators>
    -1:	<A non-zero left module presented by 30 relations for 14 generators>
    -2:	<A non-zero left module presented by 18 relations for 7 generators>
    -3:	<A non-zero left module presented by 12 relations for 4 generators>
  of
  <A non-zero left module presented by 48 relations for 20 generators>>
  gap> m := IsomorphismOfFiltration( filt );
  <A non-zero isomorphism of left modules>
\end{Verbatim}
 }

 
\subsection{\textcolor{Chapter }{Purity}}\label{Purity}
\logpage{[ 3, 1, 2 ]}
\hyperdef{L}{X7EE63228803A04F1}{}
{
  This is Example B.3 in \cite{BaSF}. 
\begin{Verbatim}[fontsize=\small,frame=single,label=Example]
  gap> Qxyz := HomalgFieldOfRationalsInDefaultCAS( ) * "x,y,z";
  Q[x,y,z]
  gap> wmat := HomalgMatrix( "[ \
  > x*y,  y*z,    z,        0,         0,    \
  > x^3*z,x^2*z^2,0,        x*z^2,     -z^2, \
  > x^4,  x^3*z,  0,        x^2*z,     -x*z, \
  > 0,    0,      x*y,      -y^2,      x^2-1,\
  > 0,    0,      x^2*z,    -x*y*z,    y*z,  \
  > 0,    0,      x^2*y-x^2,-x*y^2+x*y,y^2-y \
  > ]", 6, 5, Qxyz );
  <A 6 x 5 matrix over an external ring>
  gap> W := LeftPresentation( wmat );
  <A left module presented by 6 relations for 5 generators>
  gap> filt := PurityFiltration( W );
  <The ascending purity filtration with degrees [ -3 .. 0 ] and graded parts:
  
  0:	<A codegree-[ 1, 1 ]-pure rank 2 left module presented by 3 relations for 4\
   generators>
  
  -1:	<A codegree-1-pure grade 1 left module presented by 4 relations for 3 gene\
  rators>
  
  -2:	<A cyclic reflexively pure grade 2 left module presented by 2 relations fo\
  r a cyclic generator>
  
  -3:	<A cyclic reflexively pure grade 3 left module presented by 3 relations fo\
  r a cyclic generator>
  of
  <A non-pure rank 2 left module presented by 6 relations for 5 generators>>
  gap> W;
  <A non-pure rank 2 left module presented by 6 relations for 5 generators>
  gap> II_E := SpectralSequence( filt );
  <A stable homological spectral sequence with sheets at levels
  [ 0 .. 4 ] each consisting of left modules at bidegrees [ -3 .. 0 ]x
  [ 0 .. 3 ]>
  gap> Display( II_E );
  The associated transposed spectral sequence:
  
  a homological spectral sequence at bidegrees
  [ [ 0 .. 3 ], [ -3 .. 0 ] ]
  ---------
  Level 0:
  
   * * * *
   * * * *
   . * * *
   . . * *
  ---------
  Level 1:
  
   * * * *
   . . . .
   . . . .
   . . . .
  ---------
  Level 2:
  
   s . . .
   . . . .
   . . . .
   . . . .
  
  Now the spectral sequence of the bicomplex:
  
  a homological spectral sequence at bidegrees
  [ [ -3 .. 0 ], [ 0 .. 3 ] ]
  ---------
  Level 0:
  
   * * * *
   * * * *
   . * * *
   . . * *
  ---------
  Level 1:
  
   * * * *
   * * * *
   . * * *
   . . . *
  ---------
  Level 2:
  
   s . . .
   * s . .
   . * * .
   . . . *
  ---------
  Level 3:
  
   s . . .
   * s . .
   . . s .
   . . . *
  ---------
  Level 4:
  
   s . . .
   . s . .
   . . s .
   . . . s
  
  gap> m := IsomorphismOfFiltration( filt );
  <A non-zero isomorphism of left modules>
  gap> IsIdenticalObj( Range( m ), W );
  true
  gap> Source( m );
  <A left module presented by 12 relations for 9 generators (locked)>
  gap> Display( last );
  0,  0,   x, -y,0,1, 0,    0,  0,
  x*y,-y*z,-z,0, 0,0, 0,    0,  0,
  x^2,-x*z,0, -z,1,0, 0,    0,  0,
  0,  0,   0, 0, y,-z,0,    0,  0,
  0,  0,   0, 0, x,0, -z,   0,  -1,
  0,  0,   0, 0, 0,x, -y,   -1, 0,
  0,  0,   0, 0, 0,-y,x^2-1,0,  0,
  0,  0,   0, 0, 0,0, 0,    z,  0,
  0,  0,   0, 0, 0,0, 0,    y-1,0,
  0,  0,   0, 0, 0,0, 0,    0,  z,
  0,  0,   0, 0, 0,0, 0,    0,  y,
  0,  0,   0, 0, 0,0, 0,    0,  x
  
  Cokernel of the map
  
  Q[x,y,z]^(1x12) --> Q[x,y,z]^(1x9),
  
  currently represented by the above matrix
  gap> Display( filt );
  Degree 0:
  
  0,  0,   x, -y,
  x*y,-y*z,-z,0, 
  x^2,-x*z,0, -z 
  
  Cokernel of the map
  
  Q[x,y,z]^(1x3) --> Q[x,y,z]^(1x4),
  
  currently represented by the above matrix
  ----------
  Degree -1:
  
  y,-z,0,   
  x,0, -z,  
  0,x, -y,  
  0,-y,x^2-1
  
  Cokernel of the map
  
  Q[x,y,z]^(1x4) --> Q[x,y,z]^(1x3),
  
  currently represented by the above matrix
  ----------
  Degree -2:
  
  Q[x,y,z]/< z, y-1 >
  ----------
  Degree -3:
  
  Q[x,y,z]/< z, y, x >
  gap> Display( m );
  1,   0,     0,  0,   0,
  0,   -1,    0,  0,   0,
  0,   0,     -1, 0,   0,
  0,   0,     0,  -1,  0,
  -x^2,-x*z,  0,  -z,  0,
  0,   0,     x,  -y,  0,
  0,   0,     0,  0,   -1,
  0,   0,     x^2,-x*y,y,
  -x^3,-x^2*z,0,  -x*z,z
  
  the map is currently represented by the above 9 x 5 matrix
\end{Verbatim}
 }

 
\subsection{\textcolor{Chapter }{A3{\textunderscore}Purity}}\label{A3_Purity}
\logpage{[ 3, 1, 3 ]}
\hyperdef{L}{X7816D6ED815ED641}{}
{
  This is Example B.4 in \cite{BaSF}. 
\begin{Verbatim}[fontsize=\small,frame=single,label=Example]
  gap> Qxyz := HomalgFieldOfRationalsInDefaultCAS( ) * "x,y,z";
  Q[x,y,z]
  gap> A3 := RingOfDerivations( Qxyz, "Dx,Dy,Dz" );
  Q[x,y,z]<Dx,Dy,Dz>
  gap> nmat := HomalgMatrix( "[ \
  > 3*Dy*Dz-Dz^2+Dx+3*Dy-Dz,           3*Dy*Dz-Dz^2,     \
  > Dx*Dz+Dz^2+Dz,                     Dx*Dz+Dz^2,       \
  > Dx*Dy,                             0,                \
  > Dz^2-Dx+Dz,                        3*Dx*Dy+Dz^2,     \
  > Dx^2,                              0,                \
  > -Dz^2+Dx-Dz,                       3*Dx^2-Dz^2,      \
  > Dz^3-Dx*Dz+Dz^2,                   Dz^3,             \
  > 2*x*Dz^2-2*x*Dx+2*x*Dz+3*Dx+3*Dz+3,2*x*Dz^2+3*Dx+3*Dz\
  > ]", 8, 2, A3 );
  <A 8 x 2 matrix over an external ring>
  gap> N := LeftPresentation( nmat );
  <A left module presented by 8 relations for 2 generators>
  gap> filt := PurityFiltration( N );
  <The ascending purity filtration with degrees [ -3 .. 0 ] and graded parts:
     0:	<A zero left module>
  
  -1:	<A cyclic reflexively pure grade 1 left module presented by 1 relation for\
   a cyclic generator>
  
  -2:	<A cyclic reflexively pure grade 2 left module presented by 2 relations fo\
  r a cyclic generator>
  
  -3:	<A cyclic reflexively pure grade 3 left module presented by 3 relations fo\
  r a cyclic generator>
  of
  <A non-pure grade 1 left module presented by 8 relations for 2 generators>>
  gap> II_E := SpectralSequence( filt );
  <A stable homological spectral sequence with sheets at levels 
  [ 0 .. 2 ] each consisting of left modules at bidegrees [ -3 .. 0 ]x
  [ 0 .. 3 ]>
  gap> Display( II_E );
  The associated transposed spectral sequence:
  
  a homological spectral sequence at bidegrees
  [ [ 0 .. 3 ], [ -3 .. 0 ] ]
  ---------
  Level 0:
  
   * * * *
   . * * *
   . . * *
   . . . *
  ---------
  Level 1:
  
   * * * *
   . . . .
   . . . .
   . . . .
  ---------
  Level 2:
  
   s . . .
   . . . .
   . . . .
   . . . .
  
  Now the spectral sequence of the bicomplex:
  
  a homological spectral sequence at bidegrees
  [ [ -3 .. 0 ], [ 0 .. 3 ] ]
  ---------
  Level 0:
  
   * * * *
   . * * *
   . . * *
   . . . *
  ---------
  Level 1:
  
   * * * *
   . * * *
   . . * *
   . . . .
  ---------
  Level 2:
  
   s . . .
   . s . .
   . . s .
   . . . .
  gap> m := IsomorphismOfFiltration( filt );
  <A non-zero isomorphism of left modules>
  gap> IsIdenticalObj( Range( m ), N );
  true
  gap> Source( m );
  <A left module presented by 6 relations for 3 generators (locked)>
  gap> Display( last );
  Dx,1/3,-1/9*x,
  0, Dy, 1/6,   
  0, Dx, -1/2,  
  0, 0,  Dz,    
  0, 0,  Dy,    
  0, 0,  Dx     
  
  Cokernel of the map
  
  R^(1x6) --> R^(1x3), ( for R := Q[x,y,z]<Dx,Dy,Dz> )
  
  currently represented by the above matrix
  gap> Display( filt );
  Degree 0:
  
  0
  ----------
  Degree -1:
  
  Q[x,y,z]<Dx,Dy,Dz>/< Dx > 
  ----------
  Degree -2:
  
  Q[x,y,z]<Dx,Dy,Dz>/< Dy, Dx >
  ----------
  Degree -3:
  
  Q[x,y,z]<Dx,Dy,Dz>/< Dz, Dy, Dx >
  gap> Display( m );
  1,                1,     
  3*Dz+3,           3*Dz,  
  -6*Dz^2+6*Dx-6*Dz,-6*Dz^2
  
  the map is currently represented by the above 3 x 2 matrix
\end{Verbatim}
 }

 
\subsection{\textcolor{Chapter }{TorExt-Grothendieck}}\label{TorExt-Grothendieck}
\logpage{[ 3, 1, 4 ]}
\hyperdef{L}{X812EF8147AE16E72}{}
{
  This is Example B.5 in \cite{BaSF}. 
\begin{Verbatim}[fontsize=\small,frame=single,label=Example]
  gap> Qxyz := HomalgFieldOfRationalsInDefaultCAS( ) * "x,y,z";
  Q[x,y,z]
  gap> wmat := HomalgMatrix( "[ \
  > x*y,  y*z,    z,        0,         0,    \
  > x^3*z,x^2*z^2,0,        x*z^2,     -z^2, \
  > x^4,  x^3*z,  0,        x^2*z,     -x*z, \
  > 0,    0,      x*y,      -y^2,      x^2-1,\
  > 0,    0,      x^2*z,    -x*y*z,    y*z,  \
  > 0,    0,      x^2*y-x^2,-x*y^2+x*y,y^2-y \
  > ]", 6, 5, Qxyz );
  <A 6 x 5 matrix over an external ring>
  gap> W := LeftPresentation( wmat );
  <A left module presented by 6 relations for 5 generators>
  gap> F := InsertObjectInMultiFunctor( Functor_TensorProduct_for_fp_modules, 2, W, "TensorW" );
  <The functor TensorW for f.p. modules and their maps over computable rings>
  gap> G := LeftDualizingFunctor( Qxyz );;
  gap> II_E := GrothendieckSpectralSequence( F, G, W );
  <A stable cohomological spectral sequence with sheets at levels
  [ 0 .. 4 ] each consisting of left modules at bidegrees [ -3 .. 0 ]x
  [ 0 .. 3 ]>
  gap> Display( II_E );
  The associated transposed spectral sequence:
  
  a cohomological spectral sequence at bidegrees
  [ [ 0 .. 3 ], [ -3 .. 0 ] ]
  ---------
  Level 0:
  
   * * * *
   * * * *
   . * * *
   . . * *
  ---------
  Level 1:
  
   * * * *
   . . . .
   . . . .
   . . . .
  ---------
  Level 2:
  
   s s s s
   . . . .
   . . . .
   . . . .
  
  Now the spectral sequence of the bicomplex:
  
  a cohomological spectral sequence at bidegrees
  [ [ -3 .. 0 ], [ 0 .. 3 ] ]
  ---------
  Level 0:
  
   * * * *
   * * * *
   . * * *
   . . * *
  ---------
  Level 1:
  
   * * * *
   * * * *
   . * * *
   . . . *
  ---------
  Level 2:
  
   * * s s
   * * * *
   . * * *
   . . . *
  ---------
  Level 3:
  
   * s s s
   . s s s
   . . s *
   . . . s
  ---------
  Level 4:
  
   s s s s
   . s s s
   . . s s
   . . . s
  gap> filt := FiltrationBySpectralSequence( II_E, 0 );
  <A descending filtration with degrees [ -3 .. 0 ] and graded parts:
  
  -3:	<A non-zero cyclic torsion left module presented by yet unknown relations \
  for a cyclic generator>
    -2:	<A non-zero left module presented by 17 relations for 6 generators>
    -1:	<A non-zero left module presented by 23 relations for 10 generators>
     0:	<A non-zero left module presented by 13 relations for 10 generators>
  of
  <A left module presented by yet unknown relations for 41 generators>>
  gap> ByASmallerPresentation( filt );
  <A descending filtration with degrees [ -3 .. 0 ] and graded parts:
  
  -3:	<A non-zero cyclic torsion left module presented by 3 relations for a cycl\
  ic generator>
    -2:	<A non-zero left module presented by 12 relations for 4 generators>
    -1:	<A non-zero left module presented by 18 relations for 8 generators>
     0:	<A non-zero left module presented by 11 relations for 10 generators>
  of
  <A non-zero left module presented by 21 relations for 12 generators>>
  gap> m := IsomorphismOfFiltration( filt );
  <A non-zero isomorphism of left modules>
\end{Verbatim}
 }

 
\subsection{\textcolor{Chapter }{TorExt}}\label{TorExt}
\logpage{[ 3, 1, 5 ]}
\hyperdef{L}{X784BC2567875830B}{}
{
  This is Example B.6 in \cite{BaSF}. 
\begin{Verbatim}[fontsize=\small,frame=single,label=Example]
  gap> Qxyz := HomalgFieldOfRationalsInDefaultCAS( ) * "x,y,z";
  Q[x,y,z]
  gap> wmat := HomalgMatrix( "[ \
  > x*y,  y*z,    z,        0,         0,    \
  > x^3*z,x^2*z^2,0,        x*z^2,     -z^2, \
  > x^4,  x^3*z,  0,        x^2*z,     -x*z, \
  > 0,    0,      x*y,      -y^2,      x^2-1,\
  > 0,    0,      x^2*z,    -x*y*z,    y*z,  \
  > 0,    0,      x^2*y-x^2,-x*y^2+x*y,y^2-y \
  > ]", 6, 5, Qxyz );
  <A 6 x 5 matrix over an external ring>
  gap> W := LeftPresentation( wmat );
  <A left module presented by 6 relations for 5 generators>
  gap> P := Resolution( W );
  <A right acyclic complex containing 3 morphisms of left modules at degrees 
  [ 0 .. 3 ]>
  gap> GP := Hom( P );
  <A cocomplex containing 3 morphisms of right modules at degrees [ 0 .. 3 ]>
  gap> FGP := GP * P;
  <A cocomplex containing 3 morphisms of left complexes at degrees [ 0 .. 3 ]>
  gap> BC := HomalgBicomplex( FGP );
  <A bicocomplex containing left modules at bidegrees [ 0 .. 3 ]x[ -3 .. 0 ]>
  gap> p_degrees := ObjectDegreesOfBicomplex( BC )[1];
  [ 0 .. 3 ]
  gap> II_E := SecondSpectralSequenceWithFiltration( BC, p_degrees );
  <A stable cohomological spectral sequence with sheets at levels 
  [ 0 .. 4 ] each consisting of left modules at bidegrees [ -3 .. 0 ]x
  [ 0 .. 3 ]>
  gap> Display( II_E );
  The associated transposed spectral sequence:
  
  a cohomological spectral sequence at bidegrees
  [ [ 0 .. 3 ], [ -3 .. 0 ] ]
  ---------
  Level 0:
  
   * * * *
   * * * *
   * * * *
   * * * *
  ---------
  Level 1:
  
   * * * *
   . . . .
   . . . .
   . . . .
  ---------
  Level 2:
  
   s s s s
   . . . .
   . . . .
   . . . .
  
  Now the spectral sequence of the bicomplex:
  
  a cohomological spectral sequence at bidegrees
  [ [ -3 .. 0 ], [ 0 .. 3 ] ]
  ---------
  Level 0:
  
   * * * *
   * * * *
   * * * *
   * * * *
  ---------
  Level 1:
  
   * * * *
   * * * *
   * * * *
   * * * *
  ---------
  Level 2:
  
   * * s s
   * * * *
   . * * *
   . . . *
  ---------
  Level 3:
  
   * s s s
   . s s s
   . . s *
   . . . s
  ---------
  Level 4:
  
   s s s s
   . s s s
   . . s s
   . . . s
  gap> filt := FiltrationBySpectralSequence( II_E, 0 );
  <A descending filtration with degrees [ -3 .. 0 ] and graded parts:
  
  -3:	<A non-zero cyclic torsion left module presented by yet unknown relations \
  for a cyclic generator>
    -2:	<A non-zero left module presented by 17 relations for 7 generators>
    -1:	<A non-zero left module presented by 29 relations for 13 generators>
     0:	<A non-zero left module presented by 13 relations for 10 generators>
  of
  <A left module presented by yet unknown relations for 24 generators>>
  gap> ByASmallerPresentation( filt );
  <A descending filtration with degrees [ -3 .. 0 ] and graded parts:
  
  -3:	<A non-zero cyclic torsion left module presented by 3 relations for a cycl\
  ic generator>
    -2:	<A non-zero left module presented by 12 relations for 4 generators>
    -1:	<A non-zero left module presented by 21 relations for 8 generators>
     0:	<A non-zero left module presented by 11 relations for 10 generators>
  of
  <A non-zero left module presented by 23 relations for 12 generators>>
  gap> m := IsomorphismOfFiltration( filt );
  <A non-zero isomorphism of left modules>
\end{Verbatim}
 }

 
\subsection{\textcolor{Chapter }{CodegreeOfPurity}}\label{CodegreeOfPurity}
\logpage{[ 3, 1, 6 ]}
\hyperdef{L}{X8021C33D85444081}{}
{
  This is Example B.7 in \cite{BaSF}. 
\begin{Verbatim}[fontsize=\small,frame=single,label=Example]
  gap> Qxyz := HomalgFieldOfRationalsInDefaultCAS( ) * "x,y,z";
  Q[x,y,z]
  gap> vmat := HomalgMatrix( "[ \
  > 0,  0,  x,-z, \
  > x*z,z^2,y,0,  \
  > x^2,x*z,0,y   \
  > ]", 3, 4, Qxyz );
  <A 3 x 4 matrix over an external ring>
  gap> V := LeftPresentation( vmat );
  <A non-torsion left module presented by 3 relations for 4 generators>
  gap> wmat := HomalgMatrix( "[ \
  > 0,  0,  x,-y, \
  > x*y,y*z,z,0,  \
  > x^2,x*z,0,z   \
  > ]", 3, 4, Qxyz );
  <A 3 x 4 matrix over an external ring>
  gap> W := LeftPresentation( wmat );
  <A non-torsion left module presented by 3 relations for 4 generators>
  gap> Rank( V );
  2
  gap> Rank( W );
  2
  gap> ProjectiveDimension( V );
  2
  gap> ProjectiveDimension( W );
  2
  gap> DegreeOfTorsionFreeness( V );
  1
  gap> DegreeOfTorsionFreeness( W );
  1
  gap> CodegreeOfPurity( V );
  [ 2 ]
  gap> CodegreeOfPurity( W );
  [ 1, 1 ]
  gap> filtV := PurityFiltration( V );
  <The ascending purity filtration with degrees [ -2 .. 0 ] and graded parts:
  
  0:	<A codegree-[ 2 ]-pure rank 2 left module presented by 3 relations for 4 ge\
  nerators>
    -1:	<A zero left module>
    -2:	<A zero left module>
  of
  <A codegree-[ 2 ]-pure rank 2 left module presented by 3 relations for 4 gener\
  ators>>
  gap> filtW := PurityFiltration( W );
  <The ascending purity filtration with degrees [ -2 .. 0 ] and graded parts:
  
  0:	<A codegree-[ 1, 1 ]-pure rank 2 left module presented by 3 relations for 4\
   generators>
    -1:	<A zero left module>
    -2:	<A zero left module>
  of
  <A codegree-[ 1, 1 ]-pure rank 2 left module presented by 3 relations for 4 ge\
  nerators>>
  gap> II_EV := SpectralSequence( filtV );
  <A stable homological spectral sequence with sheets at levels 
  [ 0 .. 4 ] each consisting of left modules at bidegrees [ -3 .. 0 ]x
  [ 0 .. 2 ]>
  gap> Display( II_EV );
  The associated transposed spectral sequence:
  
  a homological spectral sequence at bidegrees
  [ [ 0 .. 2 ], [ -3 .. 0 ] ]
  ---------
  Level 0:
  
   * * *
   * * *
   * * *
   . * *
  ---------
  Level 1:
  
   * * *
   . . .
   . . .
   . . .
  ---------
  Level 2:
  
   s . .
   . . .
   . . .
   . . .
  
  Now the spectral sequence of the bicomplex:
  
  a homological spectral sequence at bidegrees
  [ [ -3 .. 0 ], [ 0 .. 2 ] ]
  ---------
  Level 0:
  
   * * * *
   * * * *
   . * * *
  ---------
  Level 1:
  
   * * * *
   * * * *
   . . * *
  ---------
  Level 2:
  
   * . . .
   * . . .
   . . * *
  ---------
  Level 3:
  
   * . . .
   . . . .
   . . . *
  ---------
  Level 4:
  
   . . . .
   . . . .
   . . . s
  gap> II_EW := SpectralSequence( filtW );
  <A stable homological spectral sequence with sheets at levels 
  [ 0 .. 4 ] each consisting of left modules at bidegrees [ -3 .. 0 ]x
  [ 0 .. 2 ]>
  gap> Display( II_EW );                  
  The associated transposed spectral sequence:
  
  a homological spectral sequence at bidegrees
  [ [ 0 .. 2 ], [ -3 .. 0 ] ]
  ---------
  Level 0:
  
   * * *
   * * *
   . * *
   . . *
  ---------
  Level 1:
  
   * * *
   . . .
   . . .
   . . .
  ---------
  Level 2:
  
   s . .
   . . .
   . . .
   . . .
  
  Now the spectral sequence of the bicomplex:
  
  a homological spectral sequence at bidegrees
  [ [ -3 .. 0 ], [ 0 .. 2 ] ]
  ---------
  Level 0:
  
   * * * *
   . * * *
   . . * *
  ---------
  Level 1:
  
   * * * *
   . * * *
   . . . *
  ---------
  Level 2:
  
   * . . .
   . * . .
   . . . *
  ---------
  Level 3:
  
   * . . .
   . . . .
   . . . *
  ---------
  Level 4:
  
   . . . .
   . . . .
   . . . s
\end{Verbatim}
 }

 
\subsection{\textcolor{Chapter }{HomHom}}\label{HomHom}
\logpage{[ 3, 1, 7 ]}
\hyperdef{L}{X791E21F47805048A}{}
{
  This corresponds to the example of Section 2 in \cite{BREACA}. 
\begin{Verbatim}[fontsize=\small,frame=single,label=Example]
  gap> R := HomalgRingOfIntegersInExternalGAP( ) / 2^8;
  Z/( 256 )
  gap> Display( R );
  <A residue class ring>
  gap> M := LeftPresentation( [ 2^5 ], R );
  <A cyclic left module presented by an unknown number of relations for a cyclic\
   generator>
  gap> Display( M );
  Z/( 256 )/< |[ 32 ]| > 
  gap> M;
  <A cyclic left module presented by 1 relation for a cyclic generator>
  gap> _M := LeftPresentation( [ 2^3 ], R );
  <A cyclic left module presented by an unknown number of relations for a cyclic\
   generator>
  gap> Display( _M );
  Z/( 256 )/< |[ 8 ]| > 
  gap> _M;
  <A cyclic left module presented by 1 relation for a cyclic generator>
  gap> alpha2 := HomalgMap( [ 1 ], M, _M );
  <A "homomorphism" of left modules>
  gap> IsMorphism( alpha2 );
  true
  gap> alpha2;
  <A homomorphism of left modules>
  gap> Display( alpha2 );
  [ [  1 ] ]
  
  modulo [ 256 ]
  
  the map is currently represented by the above 1 x 1 matrix
  gap> M_ := Kernel( alpha2 );
  <A cyclic left module presented by yet unknown relations for a cyclic generato\
  r>
  gap> alpha1 := KernelEmb( alpha2 );
  <A monomorphism of left modules>
  gap> seq := HomalgComplex( alpha2 );
  <An acyclic complex containing a single morphism of left modules at degrees 
  [ 0 .. 1 ]>
  gap> Add( seq, alpha1 );
  gap> seq;
  <A sequence containing 2 morphisms of left modules at degrees [ 0 .. 2 ]>
  gap> IsShortExactSequence( seq );
  true
  gap> seq;
  <A short exact sequence containing 2 morphisms of left modules at degrees 
  [ 0 .. 2 ]>
  gap> Display( seq );
  -------------------------
  at homology degree: 2
  Z/( 256 )/< |[ 4 ]| > 
  -------------------------
  [ [  24 ] ]
  
  modulo [ 256 ]
  
  the map is currently represented by the above 1 x 1 matrix
  ------------v------------
  at homology degree: 1
  Z/( 256 )/< |[ 32 ]| > 
  -------------------------
  [ [  1 ] ]
  
  modulo [ 256 ]
  
  the map is currently represented by the above 1 x 1 matrix
  ------------v------------
  at homology degree: 0
  Z/( 256 )/< |[ 8 ]| > 
  -------------------------
  gap> K := LeftPresentation( [ 2^7 ], R );
  <A cyclic left module presented by an unknown number of relations for a cyclic\
   generator>
  gap> L := RightPresentation( [ 2^4 ], R );
  <A cyclic right module on a cyclic generator satisfying an unknown number of r\
  elations>
  gap> triangle := LHomHom( 4, seq, K, L, "t" );
  <An exact triangle containing 3 morphisms of left complexes at degrees 
  [ 1, 2, 3, 1 ]>
  gap> lehs := LongSequence( triangle );
  <A sequence containing 14 morphisms of left modules at degrees [ 0 .. 14 ]>
  gap> ByASmallerPresentation( lehs );
  <A non-zero sequence containing 14 morphisms of left modules at degrees 
  [ 0 .. 14 ]>
  gap> IsExactSequence( lehs );
  false
  gap> lehs;
  <A non-zero left acyclic complex containing 
  14 morphisms of left modules at degrees [ 0 .. 14 ]>
  gap> Assert( 0, IsLeftAcyclic( lehs ) );
  gap> Display( lehs );
  -------------------------
  at homology degree: 14
  Z/( 256 )/< |[ 4 ]| > 
  -------------------------
  [ [  4 ] ]
  
  modulo [ 256 ]
  
  the map is currently represented by the above 1 x 1 matrix
  ------------v------------
  at homology degree: 13
  Z/( 256 )/< |[ 8 ]| > 
  -------------------------
  [ [  2 ] ]
  
  modulo [ 256 ]
  
  the map is currently represented by the above 1 x 1 matrix
  ------------v------------
  at homology degree: 12
  Z/( 256 )/< |[ 8 ]| > 
  -------------------------
  [ [  2 ] ]
  
  modulo [ 256 ]
  
  the map is currently represented by the above 1 x 1 matrix
  ------------v------------
  at homology degree: 11
  Z/( 256 )/< |[ 4 ]| > 
  -------------------------
  [ [  4 ] ]
  
  modulo [ 256 ]
  
  the map is currently represented by the above 1 x 1 matrix
  ------------v------------
  at homology degree: 10
  Z/( 256 )/< |[ 8 ]| > 
  -------------------------
  [ [  2 ] ]
  
  modulo [ 256 ]
  
  the map is currently represented by the above 1 x 1 matrix
  ------------v------------
  at homology degree: 9
  Z/( 256 )/< |[ 8 ]| > 
  -------------------------
  [ [  2 ] ]
  
  modulo [ 256 ]
  
  the map is currently represented by the above 1 x 1 matrix
  ------------v------------
  at homology degree: 8
  Z/( 256 )/< |[ 4 ]| > 
  -------------------------
  [ [  4 ] ]
  
  modulo [ 256 ]
  
  the map is currently represented by the above 1 x 1 matrix
  ------------v------------
  at homology degree: 7
  Z/( 256 )/< |[ 8 ]| > 
  -------------------------
  [ [  2 ] ]
  
  modulo [ 256 ]
  
  the map is currently represented by the above 1 x 1 matrix
  ------------v------------
  at homology degree: 6
  Z/( 256 )/< |[ 8 ]| > 
  -------------------------
  [ [  2 ] ]
  
  modulo [ 256 ]
  
  the map is currently represented by the above 1 x 1 matrix
  ------------v------------
  at homology degree: 5
  Z/( 256 )/< |[ 4 ]| > 
  -------------------------
  [ [  4 ] ]
  
  modulo [ 256 ]
  
  the map is currently represented by the above 1 x 1 matrix
  ------------v------------
  at homology degree: 4
  Z/( 256 )/< |[ 8 ]| > 
  -------------------------
  [ [  2 ] ]
  
  modulo [ 256 ]
  
  the map is currently represented by the above 1 x 1 matrix
  ------------v------------
  at homology degree: 3
  Z/( 256 )/< |[ 8 ]| > 
  -------------------------
  [ [  2 ] ]
  
  modulo [ 256 ]
  
  the map is currently represented by the above 1 x 1 matrix
  ------------v------------
  at homology degree: 2
  Z/( 256 )/< |[ 4 ]| > 
  -------------------------
  [ [  8 ] ]
  
  modulo [ 256 ]
  
  the map is currently represented by the above 1 x 1 matrix
  ------------v------------
  at homology degree: 1
  Z/( 256 )/< |[ 16 ]| > 
  -------------------------
  [ [  1 ] ]
  
  modulo [ 256 ]
  
  the map is currently represented by the above 1 x 1 matrix
  ------------v------------
  at homology degree: 0
  Z/( 256 )/< |[ 8 ]| > 
  -------------------------
\end{Verbatim}
 }

 }

 
\section{\textcolor{Chapter }{Commutative Algebra}}\label{CommutativeAlgebra}
\logpage{[ 3, 2, 0 ]}
\hyperdef{L}{X85CF19B87D1C375F}{}
{
  
\subsection{\textcolor{Chapter }{Eliminate}}\label{Eliminate}
\logpage{[ 3, 2, 1 ]}
\hyperdef{L}{X781B1C0C80529B09}{}
{
  
\begin{Verbatim}[fontsize=\small,frame=single,label=Example]
  gap> R := HomalgFieldOfRationalsInDefaultCAS( ) * "x,y,z,l,m";
  Q[x,y,z,l,m]
  gap> var := Indeterminates( R );
  [ x, y, z, l, m ]
  gap> x := var[1];; y := var[2];; z := var[3];; l := var[4];; m := var[5];;
  gap> L := [ x*m+l-4, y*m+l-2, z*m-l+1, x^2+y^2+z^2-1, x+y-z ];
  [ x*m+l-4, y*m+l-2, z*m-l+1, x^2+y^2+z^2-1, x+y-z ]
  gap> e := Eliminate( L, [ l, m ] );
  <A ? x 1 matrix over an external ring>
  gap> Display( e );
  4*y+z,  
  4*x-5*z,
  21*z^2-8
  gap> I := LeftSubmodule( e );
  <A torsion-free (left) ideal given by 3 generators>
  gap> Display( I );
  4*y+z,  
  4*x-5*z,
  21*z^2-8
  
  A (left) ideal generated by the 3 entries of the above matrix
  gap> J := LeftSubmodule( "x+y-z, -2*z-3*y+x, x^2+y^2+z^2-1", R );
  <A torsion-free (left) ideal given by 3 generators>
  gap> I = J;
  true
\end{Verbatim}
 }

 }

  }

 \def\bibname{References\logpage{[ "Bib", 0, 0 ]}
\hyperdef{L}{X7A6F98FD85F02BFE}{}
}

\bibliographystyle{alpha}
\bibliography{ExamplesForHomalgBib.xml}

\def\indexname{Index\logpage{[ "Ind", 0, 0 ]}
\hyperdef{L}{X83A0356F839C696F}{}
}


\printindex

\newpage
\immediate\write\pagenrlog{["End"], \arabic{page}];}
\immediate\closeout\pagenrlog
\end{document}
