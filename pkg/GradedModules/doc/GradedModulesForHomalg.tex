% generated by GAPDoc2LaTeX from XML source (Frank Luebeck)
\documentclass[a4paper,11pt]{report}

\usepackage{a4wide}
\sloppy
\pagestyle{myheadings}
\usepackage{amssymb}
\usepackage[utf8]{inputenc}
\usepackage{makeidx}
\makeindex
\usepackage{color}
\definecolor{FireBrick}{rgb}{0.5812,0.0074,0.0083}
\definecolor{RoyalBlue}{rgb}{0.0236,0.0894,0.6179}
\definecolor{RoyalGreen}{rgb}{0.0236,0.6179,0.0894}
\definecolor{RoyalRed}{rgb}{0.6179,0.0236,0.0894}
\definecolor{LightBlue}{rgb}{0.8544,0.9511,1.0000}
\definecolor{Black}{rgb}{0.0,0.0,0.0}

\definecolor{linkColor}{rgb}{0.0,0.0,0.554}
\definecolor{citeColor}{rgb}{0.0,0.0,0.554}
\definecolor{fileColor}{rgb}{0.0,0.0,0.554}
\definecolor{urlColor}{rgb}{0.0,0.0,0.554}
\definecolor{promptColor}{rgb}{0.0,0.0,0.589}
\definecolor{brkpromptColor}{rgb}{0.589,0.0,0.0}
\definecolor{gapinputColor}{rgb}{0.589,0.0,0.0}
\definecolor{gapoutputColor}{rgb}{0.0,0.0,0.0}

%%  for a long time these were red and blue by default,
%%  now black, but keep variables to overwrite
\definecolor{FuncColor}{rgb}{0.0,0.0,0.0}
%% strange name because of pdflatex bug:
\definecolor{Chapter }{rgb}{0.0,0.0,0.0}
\definecolor{DarkOlive}{rgb}{0.1047,0.2412,0.0064}


\usepackage{fancyvrb}

\usepackage{mathptmx,helvet}
\usepackage[T1]{fontenc}
\usepackage{textcomp}


\usepackage[
            pdftex=true,
            bookmarks=true,        
            a4paper=true,
            pdftitle={Written with GAPDoc},
            pdfcreator={LaTeX with hyperref package / GAPDoc},
            colorlinks=true,
            backref=page,
            breaklinks=true,
            linkcolor=linkColor,
            citecolor=citeColor,
            filecolor=fileColor,
            urlcolor=urlColor,
            pdfpagemode={UseNone}, 
           ]{hyperref}

\newcommand{\maintitlesize}{\fontsize{50}{55}\selectfont}

% write page numbers to a .pnr log file for online help
\newwrite\pagenrlog
\immediate\openout\pagenrlog =\jobname.pnr
\immediate\write\pagenrlog{PAGENRS := [}
\newcommand{\logpage}[1]{\protect\write\pagenrlog{#1, \thepage,}}
%% were never documented, give conflicts with some additional packages

\newcommand{\GAP}{\textsf{GAP}}

%% nicer description environments, allows long labels
\usepackage{enumitem}
\setdescription{style=nextline}

%% depth of toc
\setcounter{tocdepth}{1}





%% command for ColorPrompt style examples
\newcommand{\gapprompt}[1]{\color{promptColor}{\bfseries #1}}
\newcommand{\gapbrkprompt}[1]{\color{brkpromptColor}{\bfseries #1}}
\newcommand{\gapinput}[1]{\color{gapinputColor}{#1}}


\begin{document}

\logpage{[ 0, 0, 0 ]}
\begin{titlepage}
\mbox{}\vfill

\begin{center}{\maintitlesize \textbf{\textsf{GradedModules}\mbox{}}}\\
\vfill

\hypersetup{pdftitle=\textsf{GradedModules}}
\markright{\scriptsize \mbox{}\hfill \textsf{GradedModules} \hfill\mbox{}}
{\Huge \textbf{A \textsf{homalg} based package for the Abelian category of finitely presented graded modules
over a computable graded ring\mbox{}}}\\
\vfill

{\Huge Version 2012.10.26\mbox{}}\\[1cm]
{October 2012\mbox{}}\\[1cm]
\mbox{}\\[2cm]
{\Large \textbf{Mohamed Barakat\\
    \mbox{}}}\\
{\Large \textbf{Sebastian Gutsche\\
    \mbox{}}}\\
{\Large \textbf{Markus Lange-Hegermann\\
    \mbox{}}}\\
{\Large \textbf{Oleksandr Motsak\\
    \mbox{}}}\\
\hypersetup{pdfauthor=Mohamed Barakat\\
    ; Sebastian Gutsche\\
    ; Markus Lange-Hegermann\\
    ; Oleksandr Motsak\\
    }
\mbox{}\\[2cm]
\begin{minipage}{12cm}\noindent
(\emph{this manual is still under construction}) \\
\\
 This manual is best viewed as an \textsc{HTML} document. The latest version is available \textsc{online} at: \\
\\
 \href{http://homalg.math.rwth-aachen.de/~markus/GradedModules/chap0.html} {\texttt{http://homalg.math.rwth-aachen.de/\texttt{\symbol{126}}markus/GradedModules/chap0.html}} \\
\\
 An \textsc{offline} version should be included in the documentation subfolder of the package. This
package is part of the \textsf{homalg}-project: \\
\\
 \href{http://homalg.math.rwth-aachen.de/index.php/unreleased/gradedmodules} {\texttt{http://homalg.math.rwth-aachen.de/index.php/unreleased/gradedmodules}} \end{minipage}

\end{center}\vfill

\mbox{}\\
{\mbox{}\\
\small \noindent \textbf{Mohamed Barakat\\
    }  Email: \href{mailto://barakat@mathematik.uni-kl.de} {\texttt{barakat@mathematik.uni-kl.de}}\\
  Homepage: \href{http://www.mathematik.uni-kl.de/~barakat} {\texttt{http://www.mathematik.uni-kl.de/\texttt{\symbol{126}}barakat}}\\
  Address: \begin{minipage}[t]{8cm}\noindent
 Department of Mathematics, \\
 University of Kaiserslautern, \\
 67653 Kaiserslautern, \\
 Germany \end{minipage}
}\\
{\mbox{}\\
\small \noindent \textbf{Sebastian Gutsche\\
    }  Email: \href{mailto://sebastian.gutsche@rwth-aachen.de} {\texttt{sebastian.gutsche@rwth-aachen.de}}\\
  Homepage: \href{http://wwwb.math.rwth-aachen.de/~gutsche/} {\texttt{http://wwwb.math.rwth-aachen.de/\texttt{\symbol{126}}gutsche/}}\\
  Address: \begin{minipage}[t]{8cm}\noindent
 Lehrstuhl B f{\"u}r Mathematik, RWTH Aachen, Templergraben 64, 52056 Aachen,
Germany \end{minipage}
}\\
{\mbox{}\\
\small \noindent \textbf{Markus Lange-Hegermann\\
    }  Email: \href{mailto://markus.lange.hegermann@rwth-aachen.de} {\texttt{markus.lange.hegermann@rwth-aachen.de}}\\
  Homepage: \href{http://wwwb.math.rwth-aachen.de/~markus} {\texttt{http://wwwb.math.rwth-aachen.de/\texttt{\symbol{126}}markus}}\\
  Address: \begin{minipage}[t]{8cm}\noindent
 Lehrstuhl B f{\"u}r Mathematik, RWTH Aachen, Templergraben 64, 52056 Aachen,
Germany \end{minipage}
}\\
{\mbox{}\\
\small \noindent \textbf{Oleksandr Motsak\\
    }  Email: \href{mailto://motsak@mathematik.uni-kl.de} {\texttt{motsak@mathematik.uni-kl.de}}\\
  Homepage: \href{http://www.mathematik.uni-kl.de/~motsak} {\texttt{http://www.mathematik.uni-kl.de/\texttt{\symbol{126}}motsak}}\\
  Address: \begin{minipage}[t]{8cm}\noindent
 Department of Mathematics, \\
 University of Kaiserslautern, \\
 67653 Kaiserslautern, \\
 Germany \end{minipage}
}\\
\end{titlepage}

\newpage\setcounter{page}{2}
{\small 
\section*{Copyright}
\logpage{[ 0, 0, 1 ]}
 {\copyright} 2008-2012 by Mohamed Barakat, Sebastian Gutsche, and Markus
Lange-Hegermann

 This package may be distributed under the terms and conditions of the GNU
Public License Version 2. \mbox{}}\\[1cm]
{\small 
\section*{Acknowledgements}
\logpage{[ 0, 0, 2 ]}
 \mbox{}}\\[1cm]
\newpage

\def\contentsname{Contents\logpage{[ 0, 0, 3 ]}}

\tableofcontents
\newpage

 \index{\textsf{GradedModules}}   
\chapter{\textcolor{Chapter }{Introduction}}\label{intro}
\logpage{[ 1, 0, 0 ]}
\hyperdef{L}{X7DFB63A97E67C0A1}{}
{
  \cite{homalg-project} }

   
\chapter{\textcolor{Chapter }{Installation of the \textsf{GradedModules} Package}}\label{install}
\logpage{[ 2, 0, 0 ]}
\hyperdef{L}{X7F2F1E0279991ADE}{}
{
  To install this package just extract the package's archive file to the \textsf{GAP} \texttt{pkg} directory.

 By default the \textsf{GradedModules} package is not automatically loaded by \textsf{GAP} when it is installed. You must load the package with \\
\\
 \texttt{LoadPackage("GradedModules");} \\
\\
 before its functions become available.

 Please, send me an e-mail if you have any questions, remarks, suggestions,
etc. concerning this package. Also, I would be pleased to hear about
applications of this package. \\
\\
\\
 Mohamed Barakat  }

   
\chapter{\textcolor{Chapter }{Quick Start}}\label{QuickStart}
\logpage{[ 3, 0, 0 ]}
\hyperdef{L}{X7EB860EC84DFC71E}{}
{
   }

   
\chapter{\textcolor{Chapter }{Ring Maps}}\label{RingMaps}
\logpage{[ 4, 0, 0 ]}
\hyperdef{L}{X7B222197819984A6}{}
{
  
\section{\textcolor{Chapter }{Ring Maps: Attributes}}\label{RingMaps:Attributes}
\logpage{[ 4, 1, 0 ]}
\hyperdef{L}{X7EBF1DD67BD0758F}{}
{
  

\subsection{\textcolor{Chapter }{KernelSubobject}}
\logpage{[ 4, 1, 1 ]}\nobreak
\hyperdef{L}{X87C00FFB79FA93A8}{}
{\noindent\textcolor{FuncColor}{$\triangleright$\ \ \texttt{KernelSubobject({\mdseries\slshape phi})\index{KernelSubobject@\texttt{KernelSubobject}}
\label{KernelSubobject}
}\hfill{\scriptsize (method)}}\\
\textbf{\indent Returns:\ }
a \textsf{homalg} submodule



 The kernel ideal of the ring map \mbox{\texttt{\mdseries\slshape phi}}. }

 }

 
\section{\textcolor{Chapter }{Ring Maps: Operations and Functions}}\label{RingMaps:Operations and Functions}
\logpage{[ 4, 2, 0 ]}
\hyperdef{L}{X7C7401BA7E2221CB}{}
{
  

\subsection{\textcolor{Chapter }{SegreMap}}
\logpage{[ 4, 2, 1 ]}\nobreak
\hyperdef{L}{X7B7DDDA17837AEF5}{}
{\noindent\textcolor{FuncColor}{$\triangleright$\ \ \texttt{SegreMap({\mdseries\slshape R, s})\index{SegreMap@\texttt{SegreMap}}
\label{SegreMap}
}\hfill{\scriptsize (method)}}\\
\textbf{\indent Returns:\ }
a \textsf{homalg} ring map



 The ring map corresponding to the Segre embedding of $MultiProj(\mbox{\texttt{\mdseries\slshape R}})$ into the projective space according to $P(W_1)\times P(W_2) \to P(W_1\otimes W_2)$. }

 

\subsection{\textcolor{Chapter }{PlueckerMap}}
\logpage{[ 4, 2, 2 ]}\nobreak
\hyperdef{L}{X78E0B36179C5646C}{}
{\noindent\textcolor{FuncColor}{$\triangleright$\ \ \texttt{PlueckerMap({\mdseries\slshape l, n, A, s})\index{PlueckerMap@\texttt{PlueckerMap}}
\label{PlueckerMap}
}\hfill{\scriptsize (method)}}\\
\textbf{\indent Returns:\ }
a \textsf{homalg} ring map



 The ring map corresponding to the Pl{\"u}cker embedding of the Grassmannian $G_l(P^{\mbox{\texttt{\mdseries\slshape n}}}(\mbox{\texttt{\mdseries\slshape A}}))=G_l(P(W))$ into the projective space $P(\bigwedge^l W)$, where $W=V^*$ is the $\mbox{\texttt{\mdseries\slshape A}}$-dual of the free module $V=A^{\mbox{\texttt{\mdseries\slshape n}}+1}$ of rank $\mbox{\texttt{\mdseries\slshape n}}+1$. }

 

\subsection{\textcolor{Chapter }{VeroneseMap}}
\logpage{[ 4, 2, 3 ]}\nobreak
\hyperdef{L}{X816B9AB287EEF9A5}{}
{\noindent\textcolor{FuncColor}{$\triangleright$\ \ \texttt{VeroneseMap({\mdseries\slshape n, d, A, s})\index{VeroneseMap@\texttt{VeroneseMap}}
\label{VeroneseMap}
}\hfill{\scriptsize (method)}}\\
\textbf{\indent Returns:\ }
a \textsf{homalg} ring map



 The ring map corresponding to the Veronese embedding of the projective space $P^{\mbox{\texttt{\mdseries\slshape n}}}(\mbox{\texttt{\mdseries\slshape A}})=P(W)$ into the projective space $P(S^d W)$, where $W=V^*$ is the $\mbox{\texttt{\mdseries\slshape A}}$-dual of the free module $V=A^{\mbox{\texttt{\mdseries\slshape n}}+1}$ of rank $\mbox{\texttt{\mdseries\slshape n}}+1$. }

 }

  }

   
\chapter{\textcolor{Chapter }{GradedModules}}\label{GradedModules}
\logpage{[ 5, 0, 0 ]}
\hyperdef{L}{X78B70E1D86624AC1}{}
{
  
\section{\textcolor{Chapter }{GradedModules: Category and Representations}}\label{GradedModules:Category}
\logpage{[ 5, 1, 0 ]}
\hyperdef{L}{X84BE86BD7CAFCA5F}{}
{
  }

 
\section{\textcolor{Chapter }{GradedModules: Constructors}}\label{GradedModules:Constructors}
\logpage{[ 5, 2, 0 ]}
\hyperdef{L}{X7B3AF789845366C0}{}
{
  }

 
\section{\textcolor{Chapter }{GradedModules: Properties}}\label{GradedModules:Properties}
\logpage{[ 5, 3, 0 ]}
\hyperdef{L}{X858BEC417BE013FE}{}
{
  For more properties see the corresponding section  (\textbf{Modules: Modules: Properties})) in the documentation of the \textsf{homalg} package. }

 
\section{\textcolor{Chapter }{GradedModules: Attributes}}\label{GradedModules:Attributes}
\logpage{[ 5, 4, 0 ]}
\hyperdef{L}{X7EEE66BA7E3A4CB8}{}
{
   

\subsection{\textcolor{Chapter }{BettiDiagram (for modules)}}
\logpage{[ 5, 4, 1 ]}\nobreak
\hyperdef{L}{X7C689B3E831DF3E7}{}
{\noindent\textcolor{FuncColor}{$\triangleright$\ \ \texttt{BettiDiagram({\mdseries\slshape M})\index{BettiDiagram@\texttt{BettiDiagram}!for modules}
\label{BettiDiagram:for modules}
}\hfill{\scriptsize (attribute)}}\\
\textbf{\indent Returns:\ }
a \textsf{homalg} diagram



 The Betti diagram of the \textsf{homalg} graded module \mbox{\texttt{\mdseries\slshape M}}. }

 

\subsection{\textcolor{Chapter }{CastelnuovoMumfordRegularity}}
\logpage{[ 5, 4, 2 ]}\nobreak
\hyperdef{L}{X854A879B8705130F}{}
{\noindent\textcolor{FuncColor}{$\triangleright$\ \ \texttt{CastelnuovoMumfordRegularity({\mdseries\slshape M})\index{CastelnuovoMumfordRegularity@\texttt{CastelnuovoMumfordRegularity}}
\label{CastelnuovoMumfordRegularity}
}\hfill{\scriptsize (attribute)}}\\
\textbf{\indent Returns:\ }
a non-negative integer



 The Castelnuovo-Mumford regularity of the \textsf{homalg} graded module \mbox{\texttt{\mdseries\slshape M}}. }

 

\subsection{\textcolor{Chapter }{CastelnuovoMumfordRegularityOfSheafification}}
\logpage{[ 5, 4, 3 ]}\nobreak
\hyperdef{L}{X7CB0AA408287A8E2}{}
{\noindent\textcolor{FuncColor}{$\triangleright$\ \ \texttt{CastelnuovoMumfordRegularityOfSheafification({\mdseries\slshape M})\index{CastelnuovoMumfordRegularityOfSheafification@\texttt{Castelnuovo}\-\texttt{Mumford}\-\texttt{Regularity}\-\texttt{Of}\-\texttt{Sheafification}}
\label{CastelnuovoMumfordRegularityOfSheafification}
}\hfill{\scriptsize (attribute)}}\\
\textbf{\indent Returns:\ }
a non-negative integer



 The Castelnuovo-Mumford regularity of the sheafification of \textsf{homalg} graded module \mbox{\texttt{\mdseries\slshape M}}. }

 For more attributes see the corresponding section  (\textbf{Modules: Modules: Attributes})) in the documentation of the \textsf{homalg} package. }

 
\section{\textcolor{Chapter }{\textsf{LISHV}: Logical Implications for GradedModules}}\label{GradedModules:LIGrMOD}
\logpage{[ 5, 5, 0 ]}
\hyperdef{L}{X803C06C287408984}{}
{
  }

 
\section{\textcolor{Chapter }{GradedModules: Operations and Functions}}\label{GradedModules:Operations}
\logpage{[ 5, 6, 0 ]}
\hyperdef{L}{X877CA99B7CB05AD2}{}
{
  

\subsection{\textcolor{Chapter }{MonomialMap}}
\logpage{[ 5, 6, 1 ]}\nobreak
\hyperdef{L}{X7E7BA9887C435CD4}{}
{\noindent\textcolor{FuncColor}{$\triangleright$\ \ \texttt{MonomialMap({\mdseries\slshape d, M})\index{MonomialMap@\texttt{MonomialMap}}
\label{MonomialMap}
}\hfill{\scriptsize (operation)}}\\
\textbf{\indent Returns:\ }
a \textsf{homalg} map



 The map from a free graded module onto all degree \mbox{\texttt{\mdseries\slshape d}} monomial generators of the finitely generated \textsf{homalg} module \mbox{\texttt{\mdseries\slshape M}}. 
\begin{Verbatim}[commandchars=!@|,fontsize=\small,frame=single,label=Example]
  !gapprompt@gap>| !gapinput@R := HomalgFieldOfRationalsInDefaultCAS( ) * "x,y,z";;|
  !gapprompt@gap>| !gapinput@S := GradedRing( R );;|
  !gapprompt@gap>| !gapinput@M := HomalgMatrix( "[ x^3, y^2, z,   z, 0, 0 ]", 2, 3, S );;|
  !gapprompt@gap>| !gapinput@M := LeftPresentationWithDegrees( M, [ -1, 0, 1 ] );|
  <A graded non-torsion left module presented by 2 relations for 3 generators>
  !gapprompt@gap>| !gapinput@m := MonomialMap( 1, M );|
  <A homomorphism of graded left modules>
  !gapprompt@gap>| !gapinput@Display( m );|
  x^2,0,0,
  x*y,0,0,
  x*z,0,0,
  y^2,0,0,
  y*z,0,0,
  z^2,0,0,
  0,  x,0,
  0,  y,0,
  0,  z,0,
  0,  0,1 
  
  the graded map is currently represented by the above 10 x 3 matrix
  
  (degrees of generators of target: [ -1, 0, 1 ])
\end{Verbatim}
 }

 

\subsection{\textcolor{Chapter }{RandomMatrix}}
\logpage{[ 5, 6, 2 ]}\nobreak
\hyperdef{L}{X86CB265786A878D8}{}
{\noindent\textcolor{FuncColor}{$\triangleright$\ \ \texttt{RandomMatrix({\mdseries\slshape S, T})\index{RandomMatrix@\texttt{RandomMatrix}}
\label{RandomMatrix}
}\hfill{\scriptsize (operation)}}\\
\textbf{\indent Returns:\ }
a \textsf{homalg} matrix



 A random matrix between the graded source module \mbox{\texttt{\mdseries\slshape S}} and the graded target module \mbox{\texttt{\mdseries\slshape T}}. 
\begin{Verbatim}[commandchars=!@|,fontsize=\small,frame=single,label=Example]
  !gapprompt@gap>| !gapinput@R := HomalgFieldOfRationalsInDefaultCAS( ) * "a,b,c";;|
  !gapprompt@gap>| !gapinput@S := GradedRing( R );;|
  !gapprompt@gap>| !gapinput@rand := RandomMatrix( S^1 + S^2, S^2 + S^3 + S^4 );|
  <A 2 x 3 matrix over a graded ring>
  !gapprompt@gap>| !gapinput@#Display( rand );|
  !gapprompt@gap>| !gapinput@#-3*a-b,                                                  -1,                   |
  !gapprompt@gap>| !gapinput@#-a^2+a*b+2*b^2-2*a*c+2*b*c+c^2,                          -a+c,                 |
  !gapprompt@gap>| !gapinput@#-2*a^3+5*a^2*b-3*b^3+3*a*b*c+3*b^2*c+2*a*c^2+2*b*c^2+c^3,-3*b^2-2*a*c-2*b*c+c^2|
\end{Verbatim}
 }

 

\subsection{\textcolor{Chapter }{GeneratorsOfHomogeneousPart}}
\logpage{[ 5, 6, 3 ]}\nobreak
\hyperdef{L}{X78127AB787A5C681}{}
{\noindent\textcolor{FuncColor}{$\triangleright$\ \ \texttt{GeneratorsOfHomogeneousPart({\mdseries\slshape d, M})\index{GeneratorsOfHomogeneousPart@\texttt{GeneratorsOfHomogeneousPart}}
\label{GeneratorsOfHomogeneousPart}
}\hfill{\scriptsize (operation)}}\\
\textbf{\indent Returns:\ }
a \textsf{homalg} matrix



 The resulting \textsf{homalg} matrix consists of a generating set (over $R$) of the \mbox{\texttt{\mdseries\slshape d}}-th homogeneous part of the finitely generated \textsf{homalg} $S$-module \mbox{\texttt{\mdseries\slshape M}}, where $R$ is the coefficients ring of the graded ring $S$ with $S_0=R$. 
\begin{Verbatim}[commandchars=!@|,fontsize=\small,frame=single,label=Example]
  !gapprompt@gap>| !gapinput@R := HomalgFieldOfRationalsInDefaultCAS( ) * "x,y,z";;|
  !gapprompt@gap>| !gapinput@S := GradedRing( R );;|
  !gapprompt@gap>| !gapinput@M := HomalgMatrix( "[ x^3, y^2, z,   z, 0, 0 ]", 2, 3, S );;|
  !gapprompt@gap>| !gapinput@M := LeftPresentationWithDegrees( M, [ -1, 0, 1 ] );|
  <A graded non-torsion left module presented by 2 relations for 3 generators>
  !gapprompt@gap>| !gapinput@m := GeneratorsOfHomogeneousPart( 1, M );|
  <An unevaluated non-zero 7 x 3 matrix over a graded ring>
  !gapprompt@gap>| !gapinput@Display( m );|
  x^2,0,0,
  x*y,0,0,
  y^2,0,0,
  0,  x,0,
  0,  y,0,
  0,  z,0,
  0,  0,1 
  (over a graded ring)
\end{Verbatim}
 Compare with \texttt{MonomialMap} (\ref{MonomialMap}). }

 

\subsection{\textcolor{Chapter }{SubmoduleGeneratedByHomogeneousPart}}
\logpage{[ 5, 6, 4 ]}\nobreak
\hyperdef{L}{X86E9CD307823CC52}{}
{\noindent\textcolor{FuncColor}{$\triangleright$\ \ \texttt{SubmoduleGeneratedByHomogeneousPart({\mdseries\slshape d, M})\index{SubmoduleGeneratedByHomogeneousPart@\texttt{SubmoduleGeneratedByHomogeneousPart}}
\label{SubmoduleGeneratedByHomogeneousPart}
}\hfill{\scriptsize (operation)}}\\
\textbf{\indent Returns:\ }
a \textsf{homalg} module



 The submodule of the \textsf{homalg} module \mbox{\texttt{\mdseries\slshape M}} generated by the image of the \mbox{\texttt{\mdseries\slshape d}}-th monomial map ($\to$ \texttt{MonomialMap} (\ref{MonomialMap})), or equivalently, by the generating set of the \mbox{\texttt{\mdseries\slshape d}}-th homogeneous part of \mbox{\texttt{\mdseries\slshape M}}. 
\begin{Verbatim}[commandchars=!@|,fontsize=\small,frame=single,label=Example]
  !gapprompt@gap>| !gapinput@R := HomalgFieldOfRationalsInDefaultCAS( ) * "x,y,z";;|
  !gapprompt@gap>| !gapinput@S := GradedRing( R );;|
  !gapprompt@gap>| !gapinput@M := HomalgMatrix( "[ x^3, y^2, z,   z, 0, 0 ]", 2, 3, S );;|
  !gapprompt@gap>| !gapinput@M := LeftPresentationWithDegrees( M, [ -1, 0, 1 ] );|
  <A graded non-torsion left module presented by 2 relations for 3 generators>
  !gapprompt@gap>| !gapinput@n := SubmoduleGeneratedByHomogeneousPart( 1, M );|
  <A graded left submodule given by 7 generators>
  !gapprompt@gap>| !gapinput@Display( M );|
  z,  0,    0,  
  0,  y^2*z,z^2,
  x^3,y^2,  z   
  
  Cokernel of the map
  
  Q[x,y,z]^(1x3) --> Q[x,y,z]^(1x3),
  
  currently represented by the above matrix
  (graded, degrees of generators: [ -1, 0, 1 ])
  !gapprompt@gap>| !gapinput@Display( n );|
  x^2,0,0,
  x*y,0,0,
  y^2,0,0,
  0,  x,0,
  0,  y,0,
  0,  z,0,
  0,  0,1 
  
  A left submodule generated by the 7 rows of the above matrix
  
  (graded, degrees of generators: [ 1, 1, 1, 1, 1, 1, 1 ])
  !gapprompt@gap>| !gapinput@N := UnderlyingObject( n );|
  <A graded left module presented by yet unknown relations for 7 generators>
  !gapprompt@gap>| !gapinput@Display( N );|
  z, 0, 0,0,    0,  0,0,   
  0, z, 0,0,    0,  0,0,   
  0, 0, z,0,    0,  0,0,   
  0, 0, 0,0,    -z, y,0,   
  x, 0, 0,0,    y,  0,z,   
  -y,x, 0,0,    0,  0,0,   
  0, -y,x,0,    0,  0,0,   
  0, 0, 0,-y,   x,  0,0,   
  0, 0, 0,-z,   0,  x,0,   
  0, 0, 0,0,    y*z,0,z^2, 
  0, 0, 0,y^2*z,0,  0,x*z^2
  
  Cokernel of the map
  
  Q[x,y,z]^(1x11) --> Q[x,y,z]^(1x7),
  
  currently represented by the above matrix
  
  (graded, degrees of generators: [ 1, 1, 1, 1, 1, 1, 1 ])
  !gapprompt@gap>| !gapinput@gens := GeneratorsOfModule( N );|
  <A set of 7 generators of a homalg left module>
  !gapprompt@gap>| !gapinput@Display( gens );|
  x^2,0,0,
  x*y,0,0,
  y^2,0,0,
  0,  x,0,
  0,  y,0,
  0,  z,0,
  0,  0,1 
  
  a set of 7 generators given by the rows of the above matrix
\end{Verbatim}
 }

 

\subsection{\textcolor{Chapter }{RepresentationMapOfRingElement}}
\logpage{[ 5, 6, 5 ]}\nobreak
\hyperdef{L}{X870CC71A801346E5}{}
{\noindent\textcolor{FuncColor}{$\triangleright$\ \ \texttt{RepresentationMapOfRingElement({\mdseries\slshape r, M, d})\index{RepresentationMapOfRingElement@\texttt{RepresentationMapOfRingElement}}
\label{RepresentationMapOfRingElement}
}\hfill{\scriptsize (operation)}}\\
\textbf{\indent Returns:\ }
a \textsf{homalg} matrix



 The graded map induced by the homogeneous degree \emph{$1$} ring element \mbox{\texttt{\mdseries\slshape r}} (of the underlying \textsf{homalg} graded ring $S$) regarded as a $R$-linear map between the \mbox{\texttt{\mdseries\slshape d}}-th and the $($\mbox{\texttt{\mdseries\slshape d}}$+1)$-st homogeneous part of the graded finitely generated \textsf{homalg} $S$-module $M$, where $R$ is the coefficients ring of the graded ring $S$ with $S_0=R$. The generating set of both modules is given by \texttt{GeneratorsOfHomogeneousPart} (\ref{GeneratorsOfHomogeneousPart}). The entries of the matrix presenting the map lie in the coefficients ring $R$. 
\begin{Verbatim}[commandchars=!@|,fontsize=\small,frame=single,label=Example]
  !gapprompt@gap>| !gapinput@R := HomalgFieldOfRationalsInDefaultCAS( ) * "x,y,z";;|
  !gapprompt@gap>| !gapinput@S := GradedRing( R );;|
  !gapprompt@gap>| !gapinput@x := Indeterminate( S, 1 );|
  x
  !gapprompt@gap>| !gapinput@M := HomalgMatrix( "[ x^3, y^2, z,   z, 0, 0 ]", 2, 3, S );;|
  !gapprompt@gap>| !gapinput@M := LeftPresentationWithDegrees( M, [ -1, 0, 1 ] );|
  <A graded non-torsion left module presented by 2 relations for 3 generators>
  !gapprompt@gap>| !gapinput@m := RepresentationMapOfRingElement( x, M, 0 );|
  <A "homomorphism" of graded left modules>
  !gapprompt@gap>| !gapinput@Display( m );|
  1,0,0,0,0,0,0,
  0,1,0,0,0,0,0,
  0,0,0,1,0,0,0 
  
  the graded map is currently represented by the above 3 x 7 matrix
  
  (degrees of generators of target: [ 1, 1, 1, 1, 1, 1, 1 ])
\end{Verbatim}
 }

 

\subsection{\textcolor{Chapter }{RepresentationMatrixOfKoszulId}}
\logpage{[ 5, 6, 6 ]}\nobreak
\hyperdef{L}{X797D315F87081C55}{}
{\noindent\textcolor{FuncColor}{$\triangleright$\ \ \texttt{RepresentationMatrixOfKoszulId({\mdseries\slshape d, M})\index{RepresentationMatrixOfKoszulId@\texttt{RepresentationMatrixOfKoszulId}}
\label{RepresentationMatrixOfKoszulId}
}\hfill{\scriptsize (operation)}}\\
\textbf{\indent Returns:\ }
a \textsf{homalg} matrix



 It is assumed that all indeterminates of the underlying \textsf{homalg} graded ring $S$ are of degree $1$. The output is the \textsf{homalg} matrix of the multiplication map $\mathrm{Hom}( A, M_d ) \to \mathrm{Hom}( A, M_{d+1} )$, where $A$ is the Koszul dual ring of $S$, defined using the operation \texttt{KoszulDualRing}. 
\begin{Verbatim}[commandchars=!@|,fontsize=\small,frame=single,label=Example]
  !gapprompt@gap>| !gapinput@R := HomalgFieldOfRationalsInDefaultCAS( ) * "x,y,z";;|
  !gapprompt@gap>| !gapinput@S := GradedRing( R );;|
  !gapprompt@gap>| !gapinput@A := KoszulDualRing( S, "a,b,c" );;|
  !gapprompt@gap>| !gapinput@M := HomalgMatrix( "[ x^3, y^2, z,   z, 0, 0 ]", 2, 3, S );;|
  !gapprompt@gap>| !gapinput@M := LeftPresentationWithDegrees( M, [ -1, 0, 1 ] );|
  <A graded non-torsion left module presented by 2 relations for 3 generators>
  !gapprompt@gap>| !gapinput@m := RepresentationMatrixOfKoszulId( 0, M );|
  <An unevaluated 3 x 7 matrix over a graded ring>
  !gapprompt@gap>| !gapinput@Display( m );|
  a,b,0,0,0,0,0,
  0,a,b,0,0,0,0,
  0,0,0,a,b,c,0 
  (over a graded ring)
\end{Verbatim}
 }

 

\subsection{\textcolor{Chapter }{RepresentationMapOfKoszulId}}
\logpage{[ 5, 6, 7 ]}\nobreak
\hyperdef{L}{X7DBC9F4F827B4F01}{}
{\noindent\textcolor{FuncColor}{$\triangleright$\ \ \texttt{RepresentationMapOfKoszulId({\mdseries\slshape d, M})\index{RepresentationMapOfKoszulId@\texttt{RepresentationMapOfKoszulId}}
\label{RepresentationMapOfKoszulId}
}\hfill{\scriptsize (operation)}}\\
\textbf{\indent Returns:\ }
a \textsf{homalg} map



 It is assumed that all indeterminates of the underlying \textsf{homalg} graded ring $S$ are of degree $1$. The output is the the multiplication map $\mathrm{Hom}( A, M_d ) \to \mathrm{Hom}( A, M_{d+1} )$, where $A$ is the Koszul dual ring of $S$, defined using the operation \texttt{KoszulDualRing}. 
\begin{Verbatim}[commandchars=!@|,fontsize=\small,frame=single,label=Example]
  !gapprompt@gap>| !gapinput@R := HomalgFieldOfRationalsInDefaultCAS( ) * "x,y,z";;|
  !gapprompt@gap>| !gapinput@S := GradedRing( R );;|
  !gapprompt@gap>| !gapinput@A := KoszulDualRing( S, "a,b,c" );;|
  !gapprompt@gap>| !gapinput@M := HomalgMatrix( "[ x^3, y^2, z,   z, 0, 0 ]", 2, 3, S );;|
  !gapprompt@gap>| !gapinput@M := LeftPresentationWithDegrees( M, [ -1, 0, 1 ] );|
  <A graded non-torsion left module presented by 2 relations for 3 generators>
  !gapprompt@gap>| !gapinput@m := RepresentationMapOfKoszulId( 0, M );|
  <A homomorphism of graded left modules>
  !gapprompt@gap>| !gapinput@Display( m );|
  a,b,0,0,0,0,0,
  0,a,b,0,0,0,0,
  0,0,0,a,b,c,0 
  
  the graded map is currently represented by the above 3 x 7 matrix
  
  (degrees of generators of target: [ 4, 4, 4, 4, 4, 4, 4 ])
\end{Verbatim}
 }

 

\subsection{\textcolor{Chapter }{KoszulRightAdjoint}}
\logpage{[ 5, 6, 8 ]}\nobreak
\hyperdef{L}{X7C2D50247FFA3704}{}
{\noindent\textcolor{FuncColor}{$\triangleright$\ \ \texttt{KoszulRightAdjoint({\mdseries\slshape M, degree{\textunderscore}lowest, degree{\textunderscore}highest})\index{KoszulRightAdjoint@\texttt{KoszulRightAdjoint}}
\label{KoszulRightAdjoint}
}\hfill{\scriptsize (operation)}}\\
\textbf{\indent Returns:\ }
a \textsf{homalg} cocomplex



 It is assumed that all indeterminates of the underlying \textsf{homalg} graded ring $S$ are of degree $1$. Compute the \textsf{homalg} $A$-cocomplex $C$ of Koszul maps of the \textsf{homalg} $S$-module \mbox{\texttt{\mdseries\slshape M}} ($\to$ \texttt{RepresentationMapOfKoszulId} (\ref{RepresentationMapOfKoszulId})) in the $[$ \mbox{\texttt{\mdseries\slshape degree{\textunderscore}lowest}} .. \mbox{\texttt{\mdseries\slshape degree{\textunderscore}highest}} $]$. The Castelnuovo-Mumford regularity of \mbox{\texttt{\mdseries\slshape M}} is characterized as the highest degree $d$, such that $C$ is not exact at $d$. $A$ is the Koszul dual ring of $S$, defined using the operation \texttt{KoszulDualRing}. 
\begin{Verbatim}[commandchars=!@|,fontsize=\small,frame=single,label=Example]
  !gapprompt@gap>| !gapinput@R := HomalgFieldOfRationalsInDefaultCAS( ) * "x,y,z";;|
  !gapprompt@gap>| !gapinput@S := GradedRing( R );;|
  !gapprompt@gap>| !gapinput@A := KoszulDualRing( S, "a,b,c" );;|
  !gapprompt@gap>| !gapinput@M := HomalgMatrix( "[ x^3, y^2, z,   z, 0, 0 ]", 2, 3, S );;|
  !gapprompt@gap>| !gapinput@M := LeftPresentationWithDegrees( M, [ -1, 0, 1 ], S );|
  <A graded non-torsion left module presented by 2 relations for 3 generators>
  !gapprompt@gap>| !gapinput@CastelnuovoMumfordRegularity( M );|
  1
  !gapprompt@gap>| !gapinput@R := KoszulRightAdjoint( M, -5, 5 );|
  <A cocomplex containing 10 morphisms of graded left modules at degrees
  [ -5 .. 5 ]>
  !gapprompt@gap>| !gapinput@R := KoszulRightAdjoint( M, 1, 5 );|
  <An acyclic cocomplex containing
  4 morphisms of graded left modules at degrees [ 1 .. 5 ]>
  !gapprompt@gap>| !gapinput@R := KoszulRightAdjoint( M, 0, 5 );|
  <A cocomplex containing 5 morphisms of graded left modules at degrees
  [ 0 .. 5 ]>
  !gapprompt@gap>| !gapinput@R := KoszulRightAdjoint( M, -5, 5 );|
  <A cocomplex containing 10 morphisms of graded left modules at degrees
  [ -5 .. 5 ]>
  !gapprompt@gap>| !gapinput@H := Cohomology( R );|
  <A graded cohomology object consisting of 11 graded left modules at degrees 
  [ -5 .. 5 ]>
  !gapprompt@gap>| !gapinput@ByASmallerPresentation( H );|
  <A non-zero graded cohomology object consisting of
  11 graded left modules at degrees [ -5 .. 5 ]>
  !gapprompt@gap>| !gapinput@Cohomology( R, -2 );|
  <A graded zero left module>
  !gapprompt@gap>| !gapinput@Cohomology( R, -3 );|
  <A graded zero left module>
  !gapprompt@gap>| !gapinput@Cohomology( R, -1 );|
  <A graded cyclic torsion-free non-free left module presented by 2 relations fo\
  r a cyclic generator>
  !gapprompt@gap>| !gapinput@Cohomology( R, 0 );|
  <A graded non-zero cyclic left module presented by 3 relations for a cyclic ge\
  nerator>
  !gapprompt@gap>| !gapinput@Cohomology( R, 1 );|
  <A graded non-zero cyclic left module presented by 2 relations for a cyclic ge\
  nerator>
  !gapprompt@gap>| !gapinput@Cohomology( R, 2 );|
  <A graded zero left module>
  !gapprompt@gap>| !gapinput@Cohomology( R, 3 );|
  <A graded zero left module>
  !gapprompt@gap>| !gapinput@Cohomology( R, 4 );|
  <A graded zero left module>
  !gapprompt@gap>| !gapinput@Display( Cohomology( R, -1 ) );|
  Q{a,b,c}/< b, a >
  
  (graded, degree of generator: 0)
  !gapprompt@gap>| !gapinput@Display( Cohomology( R, 0 ) );|
  Q{a,b,c}/< c, b, a >
  
  (graded, degree of generator: 0)
  !gapprompt@gap>| !gapinput@Display( Cohomology( R, 1 ) );|
  Q{a,b,c}/< b, a >
  
  (graded, degree of generator: 2)
\end{Verbatim}
 }

 

\subsection{\textcolor{Chapter }{HomogeneousPartOverCoefficientsRing}}
\logpage{[ 5, 6, 9 ]}\nobreak
\hyperdef{L}{X78E9B52D87FC5F3C}{}
{\noindent\textcolor{FuncColor}{$\triangleright$\ \ \texttt{HomogeneousPartOverCoefficientsRing({\mdseries\slshape d, M})\index{HomogeneousPartOverCoefficientsRing@\texttt{HomogeneousPartOverCoefficientsRing}}
\label{HomogeneousPartOverCoefficientsRing}
}\hfill{\scriptsize (operation)}}\\
\textbf{\indent Returns:\ }
a \textsf{homalg} module



 The degree $d$ homogeneous part of the graded $R$-module \mbox{\texttt{\mdseries\slshape M}} as a module over the coefficient ring or field of $R$. 
\begin{Verbatim}[commandchars=!@|,fontsize=\small,frame=single,label=Example]
  !gapprompt@gap>| !gapinput@R := HomalgFieldOfRationalsInDefaultCAS( ) * "x,y,z";;|
  !gapprompt@gap>| !gapinput@S := GradedRing( R );;|
  !gapprompt@gap>| !gapinput@M := HomalgMatrix( "[ x, y^2, z^3 ]", 3, 1, S );;|
  !gapprompt@gap>| !gapinput@M := Subobject( M, ( 1 * S )^0 );|
  <A graded torsion-free (left) ideal given by 3 generators>
  !gapprompt@gap>| !gapinput@CastelnuovoMumfordRegularity( M );|
  4
  !gapprompt@gap>| !gapinput@M1 := HomogeneousPartOverCoefficientsRing( 1, M );|
  <A graded left vector space of dimension 1 on a free generator>
  !gapprompt@gap>| !gapinput@gen1 := GeneratorsOfModule( M1 );|
  <A set consisting of a single generator of a homalg left module>
  !gapprompt@gap>| !gapinput@Display( M1 );|
  Q^(1 x 1)
  
  (graded, degree of generator: 1)
  !gapprompt@gap>| !gapinput@M2 := HomogeneousPartOverCoefficientsRing( 2, M );|
  <A graded left vector space of dimension 4 on free generators>
  !gapprompt@gap>| !gapinput@Display( M2 );|
  Q^(1 x 4)
  
  (graded, degrees of generators: [ 2, 2, 2, 2 ])
  !gapprompt@gap>| !gapinput@gen2 := GeneratorsOfModule( M2 );|
  <A set of 4 generators of a homalg left module>
  !gapprompt@gap>| !gapinput@M3 := HomogeneousPartOverCoefficientsRing( 3, M );|
  <A graded left vector space of dimension 9 on free generators>
  !gapprompt@gap>| !gapinput@Display( M3 );|
  Q^(1 x 9)
  
  (graded, degrees of generators: [ 3, 3, 3, 3, 3, 3, 3, 3, 3 ])
  !gapprompt@gap>| !gapinput@gen3 := GeneratorsOfModule( M3 );|
  <A set of 9 generators of a homalg left module>
  !gapprompt@gap>| !gapinput@Display( gen1 );|
  x
  
  a set consisting of a single generator given by (the row of) the above matrix
  !gapprompt@gap>| !gapinput@Display( gen2 );|
  x^2,
  x*y,
  x*z,
  y^2 
  
  a set of 4 generators given by the rows of the above matrix
  !gapprompt@gap>| !gapinput@Display( gen3 );|
  x^3,  
  x^2*y,
  x^2*z,
  x*y*z,
  x*z^2,
  x*y^2,
  y^3,  
  y^2*z,
  z^3   
  
  a set of 9 generators given by the rows of the above matrix
\end{Verbatim}
 }

 }

  }

   
\chapter{\textcolor{Chapter }{The Tate Resolution}}\label{Tate}
\logpage{[ 6, 0, 0 ]}
\hyperdef{L}{X7FE838537D4DF8E7}{}
{
  
\section{\textcolor{Chapter }{The Tate Resolution: Operations and Functions}}\label{Tate:Operations}
\logpage{[ 6, 1, 0 ]}
\hyperdef{L}{X83CE0B0785329667}{}
{
  

\subsection{\textcolor{Chapter }{TateResolution}}
\logpage{[ 6, 1, 1 ]}\nobreak
\hyperdef{L}{X7A9DCED27D5F0D67}{}
{\noindent\textcolor{FuncColor}{$\triangleright$\ \ \texttt{TateResolution({\mdseries\slshape M, degree{\textunderscore}lowest, degree{\textunderscore}highest})\index{TateResolution@\texttt{TateResolution}}
\label{TateResolution}
}\hfill{\scriptsize (operation)}}\\
\textbf{\indent Returns:\ }
a \textsf{homalg} cocomplex



 Compute the Tate resolution of the sheaf \mbox{\texttt{\mdseries\slshape M}}. 
\begin{Verbatim}[commandchars=!@|,fontsize=\small,frame=single,label=Example]
  !gapprompt@gap>| !gapinput@R := HomalgFieldOfRationalsInDefaultCAS( ) * "x0..x3";;|
  !gapprompt@gap>| !gapinput@S := GradedRing( R );;|
  !gapprompt@gap>| !gapinput@A := KoszulDualRing( S, "e0..e3" );;|
\end{Verbatim}
 In the following we construct the different exterior powers of the cotangent
bundle shifted by $1$. Observe how a single $1$ travels along the diagnoal in the window $[ -3 .. 0 ] x [ 0 .. 3 ]$. \\
\\
 First we start with the structure sheaf with its Tate resolution: 
\begin{Verbatim}[commandchars=!@C,fontsize=\small,frame=single,label=Example]
  !gapprompt@gap>C !gapinput@O := S^0;C
  <The graded free left module of rank 1 on a free generator>
  !gapprompt@gap>C !gapinput@T := TateResolution( O, -5, 5 );C
  <An acyclic cocomplex containing
  10 morphisms of graded left modules at degrees [ -5 .. 5 ]>
  !gapprompt@gap>C !gapinput@betti := BettiDiagram( T );C
  <A Betti diagram of <An acyclic cocomplex containing 
  10 morphisms of graded left modules at degrees [ -5 .. 5 ]>>
  !gapprompt@gap>C !gapinput@Display( betti );C
  total:   35  20  10   4   1   1   4  10  20  35  56   ?   ?   ?
  ----------|---|---|---|---|---|---|---|---|---|---|---|---|---|
      3:   35  20  10   4   1   .   .   .   .   .   .   0   0   0
      2:    *   .   .   .   .   .   .   .   .   .   .   .   0   0
      1:    *   *   .   .   .   .   .   .   .   .   .   .   .   0
      0:    *   *   *   .   .   .   .   .   1   4  10  20  35  56
  ----------|---|---|---|---|---|---|---|---S---|---|---|---|---|
  twist:   -8  -7  -6  -5  -4  -3  -2  -1   0   1   2   3   4   5
  ---------------------------------------------------------------
  Euler:  -35 -20 -10  -4  -1   0   0   0   1   4  10  20  35  56
\end{Verbatim}
 The Castelnuovo-Mumford regularity of the \emph{underlying module} is distinguished among the list of twists by the character \texttt{'V'} pointing to it. It is \emph{not} an invariant of the sheaf (see the next diagram). \\
\\
 The residue class field (i.e. S modulo the maximal homogeneous ideal): 
\begin{Verbatim}[commandchars=!@|,fontsize=\small,frame=single,label=Example]
  !gapprompt@gap>| !gapinput@k := HomalgMatrix( Indeterminates( S ), Length( Indeterminates( S ) ), 1, S );|
  <A 4 x 1 matrix over a graded ring>
  !gapprompt@gap>| !gapinput@k := LeftPresentationWithDegrees( k );|
  <A graded cyclic left module presented by 4 relations for a cyclic generator>
\end{Verbatim}
 Another way of constructing the structure sheaf: 
\begin{Verbatim}[commandchars=!@C,fontsize=\small,frame=single,label=Example]
  !gapprompt@gap>C !gapinput@U0 := SyzygiesObject( 1, k );C
  <A graded torsion-free left module presented by yet unknown relations for 4 ge\
  nerators>
  !gapprompt@gap>C !gapinput@T0 := TateResolution( U0, -5, 5 );C
  <An acyclic cocomplex containing
  10 morphisms of graded left modules at degrees [ -5 .. 5 ]>
  !gapprompt@gap>C !gapinput@betti0 := BettiDiagram( T0 );C
  <A Betti diagram of <An acyclic cocomplex containing 
  10 morphisms of graded left modules at degrees [ -5 .. 5 ]>>
  !gapprompt@gap>C !gapinput@Display( betti0 );C
  total:   35  20  10   4   1   1   4  10  20  35  56   ?   ?   ?
  ----------|---|---|---|---|---|---|---|---|---|---|---|---|---|
      3:   35  20  10   4   1   .   .   .   .   .   .   0   0   0
      2:    *   .   .   .   .   .   .   .   .   .   .   .   0   0
      1:    *   *   .   .   .   .   .   .   .   .   .   .   .   0
      0:    *   *   *   .   .   .   .   .   1   4  10  20  35  56
  ----------|---|---|---|---|---|---|---|---S---|---|---|---|---|
  twist:   -8  -7  -6  -5  -4  -3  -2  -1   0   1   2   3   4   5
  ---------------------------------------------------------------
  Euler:  -35 -20 -10  -4  -1   0   0   0   1   4  10  20  35  56
\end{Verbatim}
 The cotangent bundle: 
\begin{Verbatim}[commandchars=!@|,fontsize=\small,frame=single,label=Example]
  !gapprompt@gap>| !gapinput@cotangent := SyzygiesObject( 2, k );|
  <A graded torsion-free left module presented by yet unknown relations for 6 ge\
  nerators>
  !gapprompt@gap>| !gapinput@IsFree( UnderlyingModule( cotangent ) );|
  false
  !gapprompt@gap>| !gapinput@Rank( cotangent );|
  3
  !gapprompt@gap>| !gapinput@cotangent;|
  <A graded reflexive non-projective rank 3 left module presented by 4 relations\
   for 6 generators>
  !gapprompt@gap>| !gapinput@ProjectiveDimension( UnderlyingModule( cotangent ) );|
  2
\end{Verbatim}
 the cotangent bundle shifted by $1$ with its Tate resolution: 
\begin{Verbatim}[commandchars=!@C,fontsize=\small,frame=single,label=Example]
  !gapprompt@gap>C !gapinput@U1 := cotangent * S^1;C
  <A graded non-torsion left module presented by 4 relations for 6 generators>
  !gapprompt@gap>C !gapinput@T1 := TateResolution( U1, -5, 5 );C
  <An acyclic cocomplex containing
  10 morphisms of graded left modules at degrees [ -5 .. 5 ]>
  !gapprompt@gap>C !gapinput@betti1 := BettiDiagram( T1 );C
  <A Betti diagram of <An acyclic cocomplex containing 
  10 morphisms of graded left modules at degrees [ -5 .. 5 ]>>
  !gapprompt@gap>C !gapinput@Display( betti1 );C
  total:   120   70   36   15    4    1    6   20   45   84  140    ?    ?    ?
  -----------|----|----|----|----|----|----|----|----|----|----|----|----|----|
      3:   120   70   36   15    4    .    .    .    .    .    .    0    0    0
      2:     *    .    .    .    .    .    .    .    .    .    .    .    0    0
      1:     *    *    .    .    .    .    .    1    .    .    .    .    .    0
      0:     *    *    *    .    .    .    .    .    .    6   20   45   84  140
  -----------|----|----|----|----|----|----|----|----|----S----|----|----|----|
  twist:    -8   -7   -6   -5   -4   -3   -2   -1    0    1    2    3    4    5
  -----------------------------------------------------------------------------
  Euler:  -120  -70  -36  -15   -4    0    0   -1    0    6   20   45   84  140
\end{Verbatim}
 The second power $U^2$ of the shifted cotangent bundle $U=U^1$ and its Tate resolution: 
\begin{Verbatim}[commandchars=!@C,fontsize=\small,frame=single,label=Example]
  !gapprompt@gap>C !gapinput@U2 := SyzygiesObject( 3, k ) * S^2;C
  <A graded rank 3 left module presented by 1 relation for 4 generators>
  !gapprompt@gap>C !gapinput@T2 := TateResolution( U2, -5, 5 );C
  <An acyclic cocomplex containing
  10 morphisms of graded left modules at degrees [ -5 .. 5 ]>
  !gapprompt@gap>C !gapinput@betti2 := BettiDiagram( T2 );C
  <A Betti diagram of <An acyclic cocomplex containing 
  10 morphisms of graded left modules at degrees [ -5 .. 5 ]>>
  !gapprompt@gap>C !gapinput@Display( betti2 );C
  total:   140   84   45   20    6    1    4   15   36   70  120    ?    ?    ?
  -----------|----|----|----|----|----|----|----|----|----|----|----|----|----|
      3:   140   84   45   20    6    .    .    .    .    .    .    0    0    0
      2:     *    .    .    .    .    .    1    .    .    .    .    .    0    0
      1:     *    *    .    .    .    .    .    .    .    .    .    .    .    0
      0:     *    *    *    .    .    .    .    .    .    4   15   36   70  120
  -----------|----|----|----|----|----|----|----|----|----S----|----|----|----|
  twist:    -8   -7   -6   -5   -4   -3   -2   -1    0    1    2    3    4    5
  -----------------------------------------------------------------------------
  Euler:  -140  -84  -45  -20   -6    0    1    0    0    4   15   36   70  120
\end{Verbatim}
 The third power $U^3$ of the shifted cotangent bundle $U=U^1$ and its Tate resolution: 
\begin{Verbatim}[commandchars=!@C,fontsize=\small,frame=single,label=Example]
  !gapprompt@gap>C !gapinput@U3 := SyzygiesObject( 4, k ) * S^3;C
  <A graded free left module of rank 1 on a free generator>
  !gapprompt@gap>C !gapinput@Display( U3 );C
  Q[x0,x1,x2,x3]^(1 x 1)
  
  (graded, degree of generator: 1)
  !gapprompt@gap>C !gapinput@T3 := TateResolution( U3, -5, 5 );C
  <An acyclic cocomplex containing
  10 morphisms of graded left modules at degrees [ -5 .. 5 ]>
  !gapprompt@gap>C !gapinput@betti3 := BettiDiagram( T3 );C
  <A Betti diagram of <An acyclic cocomplex containing 
  10 morphisms of graded left modules at degrees [ -5 .. 5 ]>>
  !gapprompt@gap>C !gapinput@Display( betti3 );C
  total:   56  35  20  10   4   1   1   4  10  20  35   ?   ?   ?
  ----------|---|---|---|---|---|---|---|---|---|---|---|---|---|
      3:   56  35  20  10   4   1   .   .   .   .   .   0   0   0
      2:    *   .   .   .   .   .   .   .   .   .   .   .   0   0
      1:    *   *   .   .   .   .   .   .   .   .   .   .   .   0
      0:    *   *   *   .   .   .   .   .   .   1   4  10  20  35
  ----------|---|---|---|---|---|---|---|---|---S---|---|---|---|
  twist:   -8  -7  -6  -5  -4  -3  -2  -1   0   1   2   3   4   5
  ---------------------------------------------------------------
  Euler:  -56 -35 -20 -10  -4  -1   0   0   0   1   4  10  20  35
\end{Verbatim}
 Another way to construct $U^2=U^(3-1)$: 
\begin{Verbatim}[commandchars=!@C,fontsize=\small,frame=single,label=Example]
  !gapprompt@gap>C !gapinput@u2 := GradedHom( U1, S^(-1) );C
  <A graded torsion-free right module on 4 generators satisfying yet unknown rel\
  ations>
  !gapprompt@gap>C !gapinput@t2 := TateResolution( u2, -5, 5 );C
  <An acyclic cocomplex containing
  10 morphisms of graded right modules at degrees [ -5 .. 5 ]>
  !gapprompt@gap>C !gapinput@BettiDiagram( t2 );C
  <A Betti diagram of <An acyclic cocomplex containing 
  10 morphisms of graded right modules at degrees [ -5 .. 5 ]>>
  !gapprompt@gap>C !gapinput@Display( last );C
  total:   140   84   45   20    6    1    4   15   36   70  120    ?    ?    ?
  -----------|----|----|----|----|----|----|----|----|----|----|----|----|----|
      3:   140   84   45   20    6    .    .    .    .    .    .    0    0    0
      2:     *    .    .    .    .    .    1    .    .    .    .    .    0    0
      1:     *    *    .    .    .    .    .    .    .    .    .    .    .    0
      0:     *    *    *    .    .    .    .    .    .    4   15   36   70  120
  -----------|----|----|----|----|----|----|----|----|----S----|----|----|----|
  twist:    -8   -7   -6   -5   -4   -3   -2   -1    0    1    2    3    4    5
  -----------------------------------------------------------------------------
  Euler:  -140  -84  -45  -20   -6    0    1    0    0    4   15   36   70  120
\end{Verbatim}
 }

 }

  }

   
\chapter{\textcolor{Chapter }{Examples}}\label{examples}
\logpage{[ 7, 0, 0 ]}
\hyperdef{L}{X7A489A5D79DA9E5C}{}
{
  
\section{\textcolor{Chapter }{Betti Diagrams}}\label{BettiDiagram}
\logpage{[ 7, 1, 0 ]}
\hyperdef{L}{X81A7E0D380CE7F31}{}
{
  
\subsection{\textcolor{Chapter }{DE-2.2}}\label{DE-2.2}
\logpage{[ 7, 1, 1 ]}
\hyperdef{L}{X8441906E83F6845D}{}
{
  
\begin{Verbatim}[commandchars=!@|,fontsize=\small,frame=single,label=Example]
  !gapprompt@gap>| !gapinput@R := HomalgFieldOfRationalsInDefaultCAS( ) * "x0,x1,x2";;|
  !gapprompt@gap>| !gapinput@S := GradedRing( R );;|
  !gapprompt@gap>| !gapinput@mat := HomalgMatrix( "[ x0^2, x1^2, x2^2 ]", 1, 3, S ); |
  <A 1 x 3 matrix over a graded ring>
  !gapprompt@gap>| !gapinput@M := RightPresentationWithDegrees( mat, S );|
  <A graded cyclic right module on a cyclic generator satisfying 3 relations>
  !gapprompt@gap>| !gapinput@M := RightPresentationWithDegrees( mat );|
  <A graded cyclic right module on a cyclic generator satisfying 3 relations>
  !gapprompt@gap>| !gapinput@d := Resolution( M );|
  <A right acyclic complex containing
  3 morphisms of graded right modules at degrees [ 0 .. 3 ]>
  !gapprompt@gap>| !gapinput@betti := BettiDiagram( d );|
  <A Betti diagram of <A right acyclic complex containing
  3 morphisms of graded right modules at degrees [ 0 .. 3 ]>>
  !gapprompt@gap>| !gapinput@Display( betti );|
   total:  1 3 3 1
  ----------------
       0:  1 . . .
       1:  . 3 . .
       2:  . . 3 .
       3:  . . . 1
  ----------------
  degree:  0 1 2 3
  !gapprompt@gap>| !gapinput@## we are still below the Castelnuovo-Mumford regularity, which is 3:|
  !gapprompt@gap>| !gapinput@M2 := SubmoduleGeneratedByHomogeneousPart( 2, M );|
  <A graded torsion right submodule given by 3 generators>
  !gapprompt@gap>| !gapinput@d2 := Resolution( M2 );|
  <A right acyclic complex containing
  3 morphisms of graded right modules at degrees [ 0 .. 3 ]>
  !gapprompt@gap>| !gapinput@betti2 := BettiDiagram( d2 );|
  <A Betti diagram of <A right acyclic complex containing
  3 morphisms of graded right modules at degrees [ 0 .. 3 ]>>
  !gapprompt@gap>| !gapinput@Display( betti2 );|
   total:  3 8 6 1
  ----------------
       2:  3 8 6 .
       3:  . . . 1
  ----------------
  degree:  0 1 2 3
\end{Verbatim}
 }

 
\subsection{\textcolor{Chapter }{DE-Code}}\label{DE-Code}
\logpage{[ 7, 1, 2 ]}
\hyperdef{L}{X7E32106D7B13B8D9}{}
{
  
\begin{Verbatim}[commandchars=!@|,fontsize=\small,frame=single,label=Example]
  !gapprompt@gap>| !gapinput@R := HomalgFieldOfRationalsInDefaultCAS( ) * "x0,x1,x2";;|
  !gapprompt@gap>| !gapinput@S := GradedRing( R );;|
  !gapprompt@gap>| !gapinput@mat := HomalgMatrix( "[ x0^2, x1^2 ]", 1, 2, S );|
  <A 1 x 2 matrix over a graded ring>
  !gapprompt@gap>| !gapinput@M := RightPresentationWithDegrees( mat, S );|
  <A graded cyclic right module on a cyclic generator satisfying 2 relations>
  !gapprompt@gap>| !gapinput@d := Resolution( M );|
  <A right acyclic complex containing
  2 morphisms of graded right modules at degrees [ 0 .. 2 ]>
  !gapprompt@gap>| !gapinput@betti := BettiDiagram( d );|
  <A Betti diagram of <A right acyclic complex containing
  2 morphisms of graded right modules at degrees [ 0 .. 2 ]>>
  !gapprompt@gap>| !gapinput@Display( betti );|
   total:  1 2 1
  --------------
       0:  1 . .
       1:  . 2 .
       2:  . . 1
  --------------
  degree:  0 1 2
  !gapprompt@gap>| !gapinput@m := SubmoduleGeneratedByHomogeneousPart( 2, M );|
  <A graded torsion right submodule given by 4 generators>
  !gapprompt@gap>| !gapinput@d2 := Resolution( m );|
  <A right acyclic complex containing
  2 morphisms of graded right modules at degrees [ 0 .. 2 ]>
  !gapprompt@gap>| !gapinput@betti2 := BettiDiagram( d2 );|
  <A Betti diagram of <A right acyclic complex containing
  2 morphisms of graded right modules at degrees [ 0 .. 2 ]>>
  !gapprompt@gap>| !gapinput@Display( betti2 );|
       2:  4 8 4
  --------------
  degree:  0 1 2
\end{Verbatim}
 }

 
\subsection{\textcolor{Chapter }{Schenck-3.2}}\label{Schenck-3.2}
\logpage{[ 7, 1, 3 ]}
\hyperdef{L}{X793A69C4805C6819}{}
{
  This is an example from Section 3.2 in \cite{Sch}. 
\begin{Verbatim}[commandchars=!@|,fontsize=\small,frame=single,label=Example]
  !gapprompt@gap>| !gapinput@Qxyz := HomalgFieldOfRationalsInDefaultCAS( ) * "x,y,z";;|
  !gapprompt@gap>| !gapinput@mmat := HomalgMatrix( "[ x, x^3 + y^3 + z^3 ]", 1, 2, Qxyz );|
  <A 1 x 2 matrix over an external ring>
  !gapprompt@gap>| !gapinput@S := GradedRing( Qxyz );;|
  !gapprompt@gap>| !gapinput@M := RightPresentationWithDegrees( mmat, S );|
  <A graded cyclic right module on a cyclic generator satisfying 2 relations>
  !gapprompt@gap>| !gapinput@Mr := Resolution( M );|
  <A right acyclic complex containing
  2 morphisms of graded right modules at degrees [ 0 .. 2 ]>
  !gapprompt@gap>| !gapinput@bettiM := BettiDiagram( Mr );|
  <A Betti diagram of <A right acyclic complex containing
  2 morphisms of graded right modules at degrees [ 0 .. 2 ]>>
  !gapprompt@gap>| !gapinput@Display( bettiM );|
   total:  1 2 1
  --------------
       0:  1 1 .
       1:  . . .
       2:  . 1 1
  --------------
  degree:  0 1 2
  !gapprompt@gap>| !gapinput@R := GradedRing( CoefficientsRing( S ) * "x,y,z,w" );;|
  !gapprompt@gap>| !gapinput@nmat := HomalgMatrix( "[ z^2 - y*w, y*z - x*w, y^2 - x*z ]", 1, 3, R );|
  <A 1 x 3 matrix over a graded ring>
  !gapprompt@gap>| !gapinput@N := RightPresentationWithDegrees( nmat );|
  <A graded cyclic right module on a cyclic generator satisfying 3 relations>
  !gapprompt@gap>| !gapinput@Nr := Resolution( N );|
  <A right acyclic complex containing
  2 morphisms of graded right modules at degrees [ 0 .. 2 ]>
  !gapprompt@gap>| !gapinput@bettiN := BettiDiagram( Nr );|
  <A Betti diagram of <A right acyclic complex containing
  2 morphisms of graded right modules at degrees [ 0 .. 2 ]>>
  !gapprompt@gap>| !gapinput@Display( bettiN );|
   total:  1 3 2
  --------------
       0:  1 . .
       1:  . 3 2
  --------------
  degree:  0 1 2
\end{Verbatim}
 }

 
\subsection{\textcolor{Chapter }{Schenck-8.3}}\label{Schenck-8.3}
\logpage{[ 7, 1, 4 ]}
\hyperdef{L}{X7E8F44338461DC08}{}
{
  This is an example from Section 8.3 in \cite{Sch}. 
\begin{Verbatim}[commandchars=!@|,fontsize=\small,frame=single,label=Example]
  !gapprompt@gap>| !gapinput@R := HomalgFieldOfRationalsInDefaultCAS( ) * "x,y,z,w";;|
  !gapprompt@gap>| !gapinput@S := GradedRing( R );;|
  !gapprompt@gap>| !gapinput@jmat := HomalgMatrix( "[ z*w, x*w, y*z, x*y, x^3*z - x*z^3 ]", 1, 5, S );|
  <A 1 x 5 matrix over a graded ring>
  !gapprompt@gap>| !gapinput@J := RightPresentationWithDegrees( jmat );|
  <A graded cyclic right module on a cyclic generator satisfying 5 relations>
  !gapprompt@gap>| !gapinput@Jr := Resolution( J );|
  <A right acyclic complex containing
  3 morphisms of graded right modules at degrees [ 0 .. 3 ]>
  !gapprompt@gap>| !gapinput@betti := BettiDiagram( Jr );|
  <A Betti diagram of <A right acyclic complex containing
  3 morphisms of graded right modules at degrees [ 0 .. 3 ]>>
  !gapprompt@gap>| !gapinput@Display( betti );|
   total:  1 5 6 2
  ----------------
       0:  1 . . .
       1:  . 4 4 1
       2:  . . . .
       3:  . 1 2 1
  ----------------
  degree:  0 1 2 3
\end{Verbatim}
 }

 
\subsection{\textcolor{Chapter }{Schenck-8.3.3}}\label{Schenck-8.3.3}
\logpage{[ 7, 1, 5 ]}
\hyperdef{L}{X7B672C498385F92F}{}
{
  This is Exercise 8.3.3 in \cite{Sch}. 
\begin{Verbatim}[commandchars=!@|,fontsize=\small,frame=single,label=Example]
  !gapprompt@gap>| !gapinput@Qxyz := HomalgFieldOfRationalsInDefaultCAS( ) * "x,y,z";;|
  !gapprompt@gap>| !gapinput@S := GradedRing( Qxyz );;|
  !gapprompt@gap>| !gapinput@mat := HomalgMatrix( "[ x*y*z, x*y^2, x^2*z, x^2*y, x^3 ]", 1, 5, S );|
  <A 1 x 5 matrix over a graded ring>
  !gapprompt@gap>| !gapinput@M := RightPresentationWithDegrees( mat, S );|
  <A graded cyclic right module on a cyclic generator satisfying 5 relations>
  !gapprompt@gap>| !gapinput@Mr := Resolution( M );|
  <A right acyclic complex containing
  3 morphisms of graded right modules at degrees [ 0 .. 3 ]>
  !gapprompt@gap>| !gapinput@betti := BettiDiagram( Mr );|
  <A Betti diagram of <A right acyclic complex containing
  3 morphisms of graded right modules at degrees [ 0 .. 3 ]>>
  !gapprompt@gap>| !gapinput@Display( betti );|
   total:  1 5 6 2
  ----------------
       0:  1 . . .
       1:  . . . .
       2:  . 5 6 2
  ----------------
  degree:  0 1 2 3
\end{Verbatim}
 }

 }

 
\section{\textcolor{Chapter }{Commutative Algebra}}\label{CommutativeAlgebra}
\logpage{[ 7, 2, 0 ]}
\hyperdef{L}{X85CF19B87D1C375F}{}
{
  
\subsection{\textcolor{Chapter }{Saturate}}\label{Saturate}
\logpage{[ 7, 2, 1 ]}
\hyperdef{L}{X7EA4CC697C01E080}{}
{
  
\begin{Verbatim}[commandchars=!@|,fontsize=\small,frame=single,label=Example]
  !gapprompt@gap>| !gapinput@R := HomalgFieldOfRationalsInDefaultCAS( ) * "x,y,z";;|
  !gapprompt@gap>| !gapinput@S := GradedRing( R );;|
  !gapprompt@gap>| !gapinput@m := GradedLeftSubmodule( "x,y,z", S );|
  <A graded torsion-free (left) ideal given by 3 generators>
  !gapprompt@gap>| !gapinput@I := Intersect( m^3, GradedLeftSubmodule( "x", S ) );|
  <A graded torsion-free (left) ideal given by 6 generators>
  !gapprompt@gap>| !gapinput@NrRelations( I );|
  8
  !gapprompt@gap>| !gapinput@Im := SubobjectQuotient( I, m );|
  <A graded torsion-free rank 1 (left) ideal given by 3 generators>
  !gapprompt@gap>| !gapinput@I_m := Saturate( I, m );|
  <A graded principal (left) ideal of rank 1 on a free generator>
  !gapprompt@gap>| !gapinput@Is := Saturate( I );|
  <A graded principal (left) ideal of rank 1 on a free generator>
  !gapprompt@gap>| !gapinput@Assert( 0, Is = I_m );|
\end{Verbatim}
 }

 }

 
\section{\textcolor{Chapter }{Global Section Modules of the Induced Sheaves}}\label{Sheafs}
\logpage{[ 7, 3, 0 ]}
\hyperdef{L}{X86AF934C83004BF2}{}
{
  
\subsection{\textcolor{Chapter }{Examples of the ModuleOfGlobalSections Functor and Purity Filtrations}}\label{GlobalSectionsAndPurity}
\logpage{[ 7, 3, 1 ]}
\hyperdef{L}{X87EE931187E2226C}{}
{
  
\begin{Verbatim}[commandchars=!@|,fontsize=\small,frame=single,label=Example]
  !gapprompt@gap>| !gapinput@LoadPackage( "GradedRingForHomalg" );;|
  !gapprompt@gap>| !gapinput@Qxyzt := HomalgFieldOfRationalsInDefaultCAS( ) * "x,y,z,t";;|
  !gapprompt@gap>| !gapinput@S := GradedRing( Qxyzt );;|
  !gapprompt@gap>| !gapinput@|
  !gapprompt@gap>| !gapinput@wmat := HomalgMatrix( "[ \|
  !gapprompt@>| !gapinput@x*y,  y*z,    z*t,        0,           0,          0,\|
  !gapprompt@>| !gapinput@x^3*z,x^2*z^2,0,          x*z^2*t,     -z^2*t^2,   0,\|
  !gapprompt@>| !gapinput@x^4,  x^3*z,  0,          x^2*z*t,     -x*z*t^2,   0,\|
  !gapprompt@>| !gapinput@0,    0,      x*y,        -y^2,        x^2-t^2,    0,\|
  !gapprompt@>| !gapinput@0,    0,      x^2*z,      -x*y*z,      y*z*t,      0,\|
  !gapprompt@>| !gapinput@0,    0,      x^2*y-x^2*t,-x*y^2+x*y*t,y^2*t-y*t^2,0,\|
  !gapprompt@>| !gapinput@0,    0,      0,          0,           -1,         1 \|
  !gapprompt@>| !gapinput@]", 7, 6, Qxyzt );;|
  !gapprompt@gap>| !gapinput@|
  !gapprompt@gap>| !gapinput@LoadPackage( "GradedModules" );;|
  !gapprompt@gap>| !gapinput@wmor := GradedMap( wmat, "free", "free", "left", S );;|
  !gapprompt@gap>| !gapinput@IsMorphism( wmor );;|
  !gapprompt@gap>| !gapinput@W := LeftPresentationWithDegrees( wmat, S );;|
  !gapprompt@gap>| !gapinput@HW := ModuleOfGlobalSections( W );|
  <A graded left module presented by yet unknown relations for 6 generators>
  !gapprompt@gap>| !gapinput@LinearStrandOfTateResolution( W, 0,4 );|
  <A cocomplex containing 4 morphisms of graded left modules at degrees
  [ 0 .. 4 ]>
  !gapprompt@gap>| !gapinput@purity_iso := IsomorphismOfFiltration( PurityFiltration( W ) );|
  <A non-zero isomorphism of graded left modules>
  !gapprompt@gap>| !gapinput@Hpurity_iso := ModuleOfGlobalSections( purity_iso );|
  <An isomorphism of graded left modules>
  !gapprompt@gap>| !gapinput@ModuleOfGlobalSections( wmor );|
  <A homomorphism of graded left modules>
  !gapprompt@gap>| !gapinput@NaturalMapToModuleOfGlobalSections( W );|
  <A homomorphism of graded left modules>
\end{Verbatim}
 }

 
\subsection{\textcolor{Chapter }{Horrocks Mumford bundle}}\label{HorrocksMumford}
\logpage{[ 7, 3, 2 ]}
\hyperdef{L}{X7DD8F76D7A4206E3}{}
{
  This example computes the global sections module of the Horrocks-Mumford
bundle. 
\begin{Verbatim}[commandchars=@JN,fontsize=\small,frame=single,label=Example]
  @gappromptJgap>N @gapinputJLoadPackage( "GradedRingForHomalg" );;N
  @gappromptJgap>N @gapinputJR := HomalgFieldOfRationalsInDefaultCAS( ) * "x0..x4";;N
  @gappromptJgap>N @gapinputJS := GradedRing( R );;N
  @gappromptJgap>N @gapinputJA := KoszulDualRing( S, "e0..e4" );;N
  @gappromptJgap>N @gapinputJLoadPackage( "GradedModules" );;N
  @gappromptJgap>N @gapinputJmat := HomalgMatrix( "[ \N
  @gappromptJ>N @gapinputJe1*e4, e2*e0, e3*e1, e4*e2, e0*e3, \N
  @gappromptJ>N @gapinputJe2*e3, e3*e4, e4*e0, e0*e1, e1*e2  \N
  @gappromptJ>N @gapinputJ]",N
  @gappromptJ>N @gapinputJ2, 5, A );N
  <A 2 x 5 matrix over a graded ring>
  @gappromptJgap>N @gapinputJphi := GradedMap( mat, "free", "free", "left", A );;N
  @gappromptJgap>N @gapinputJIsMorphism( phi );N
  true
  @gappromptJgap>N @gapinputJM := GuessModuleOfGlobalSectionsFromATateMap( 2, phi );N
  #I  GuessModuleOfGlobalSectionsFromATateMap uses unproven assumptions.
   Do not trust the result.
  <A graded left module presented by yet unknown relations for 19 generators>
  @gappromptJgap>N @gapinputJIsPure( M );N
  true
  @gappromptJgap>N @gapinputJRank( M );N
  2 
  @gappromptJgap>N @gapinputJDisplay( BettiDiagram( Resolution( M ) ) );N
   total:  19 35 20  2
  --------------------
       3:   4  .  .  .
       4:  15 35 20  .
       5:   .  .  .  2
  --------------------
  degree:   0  1  2  3
  @gappromptJgap>N @gapinputJDisplay( BettiDiagram( TateResolution( M, -4, 6 ) ) );N
  total:   37  14  10   5   2   5  10  14  37 100 210   ?   ?   ?   ?
  ----------|---|---|---|---|---|---|---|---|---|---|---|---|---|---|
      4:   35   4   .   .   .   .   .   .   .   .   .   0   0   0   0
      3:    *   2  10  10   5   .   .   .   .   .   .   .   0   0   0
      2:    *   *   .   .   .   .   2   .   .   .   .   .   .   0   0
      1:    *   *   *   .   .   .   .   .   5  10  10   2   .   .   0
      0:    *   *   *   *   .   .   .   .   .   .   .   4  35 100 210
  ----------|---|---|---|---|---|---|---|---|---|---|---|---|---S---|
  twist:   -8  -7  -6  -5  -4  -3  -2  -1   0   1   2   3   4   5   6
  -------------------------------------------------------------------
  Euler:   35   2 -10 -10  -5   0   2   0  -5 -10 -10   2  35 100 210
  @gappromptJgap>N @gapinputJM;N
  <A graded reflexive non-projective rank 2 left module presented by 94 \
  relations for 19 generators>
  @gappromptJgap>N @gapinputJP := ElementOfGrothendieckGroup( M );N
  ( 2*O_{P^4} - 1*O_{P^3} - 4*O_{P^2} - 2*O_{P^1} ) -> P^4
  @gappromptJgap>N @gapinputJP!.DisplayTwistedCoefficients := true;N
  true
  @gappromptJgap>N @gapinputJP;N
  ( 2*O(-3) - 10*O(-2) + 15*O(-1) - 5*O(0) ) -> P^4
  @gappromptJgap>N @gapinputJchi := HilbertPolynomial( M );N
  1/12*t^4+2/3*t^3-1/12*t^2-17/3*t-5
  @gappromptJgap>N @gapinputJc := ChernPolynomial( M );N
  ( 2 | 1-h+4*h^2 ) -> P^4
  @gappromptJgap>N @gapinputJChernPolynomial( M * S^3 );N
  ( 2 | 1+5*h+10*h^2 ) -> P^4
  @gappromptJgap>N @gapinputJch := ChernCharacter( M );N
  [ 2-u-7*u^2/2!+11*u^3/3!+17*u^4/4! ] -> P^4
  @gappromptJgap>N @gapinputJHilbertPolynomial( ch );N
  1/12*t^4+2/3*t^3-1/12*t^2-17/3*t-5
  @gappromptJgap>N @gapinputJList( [ -8 .. 7 ], i -> Value( chi, i ) );N
  [ 35, 2, -10, -10, -5, 0, 2, 0, -5, -10, -10, 2, 35, 100, 210, 380 ]
  @gappromptJgap>N @gapinputJHF := HilbertFunction( M );N
  function( t ) ... end
  @gappromptJgap>N @gapinputJList( [ 0 .. 7 ], HF );N
  [ 0, 0, 0, 4, 35, 100, 210, 380 ]
  @gappromptJgap>N @gapinputJIndexOfRegularity( M );N
  4
  @gappromptJgap>N @gapinputJDataOfHilbertFunction( M );N
  [ [ [ 4 ], [ 3 ] ], 1/12*t^4+2/3*t^3-1/12*t^2-17/3*t-5 ]
\end{Verbatim}
 }

 }

  }

 

\appendix


\chapter{\textcolor{Chapter }{Overview of the \textsf{GradedModules} Package Source Code}}\label{FileOverview}
\logpage{[ "A", 0, 0 ]}
\hyperdef{L}{X7B0FCCEE7927815B}{}
{
  }

\def\bibname{References\logpage{[ "Bib", 0, 0 ]}
\hyperdef{L}{X7A6F98FD85F02BFE}{}
}

\bibliographystyle{alpha}
\bibliography{GradedModulesForHomalgBib.xml}

\addcontentsline{toc}{chapter}{References}

\def\indexname{Index\logpage{[ "Ind", 0, 0 ]}
\hyperdef{L}{X83A0356F839C696F}{}
}

\cleardoublepage
\phantomsection
\addcontentsline{toc}{chapter}{Index}


\printindex

\newpage
\immediate\write\pagenrlog{["End"], \arabic{page}];}
\immediate\closeout\pagenrlog
\end{document}
