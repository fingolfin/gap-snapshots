%%%%%%%%%%%%%%%%%%%%%%%%%%%%%%%%%%%%%%%%%%%%%%%%%%%%%%%%%%%%%%%%%%%%%%%%%
\Chapter{Introduction}

\index{cryst}

The {\Cryst} package, previously known as \package{CrystGAP}, provides 
functions for the computation with affine crystallographic groups, 
in particular space groups. For the definition of the standard 
crystallographic notions we refer to the International Tables 
\cite{Hah95}, in particular the chapter by Wondratschek \cite{Won95},
and to the introductory chapter in \cite{BBNWZ78}. The principal 
algorithms used in this package are described in \cite{EGN97}.

The present version for {\GAP}~4 has been considerably reworked from
an earlier version for {\GAP}~3.4.4. Most of the porting to {\GAP}~4
has been done by Franz G{\accent127 a}hler. Besides affine crystallographic
groups acting from the right, also affine crystallographic groups acting 
from the left are now fully supported. Many algorithms have been added, 
extended, or improved in other ways.

Our warmest thanks go the Max Neunh{\accent127 o}ffer, whose extensive
testing of the {\GAP} 4 version of {\Cryst} in connection with {\XGAP} 
uncovered several bugs and led to many performance improvements.

{\Cryst} is implemented in the {\GAP}~4 language, and runs on any 
system supporting {\GAP}~4. However, certain commands may require 
that other GAP packages such as {\CARAT} or {\XGAP} are installed.
In particular, the routines in Section~"Normalizers" are likely
to require {\CARAT}, and the function WyckoffGraph (see~"WyckoffGraph")
requires {\XGAP}. Both {\CARAT} and {\XGAP} are available only under Unix.

The {\Cryst} package is loaded with the command
\beginexample 
gap> LoadPackage( "cryst" ); 
true
\endexample

{\Cryst} has been developed by
\beginitems
Bettina Eick &
Fachbereich Mathematik und Informatik\hfil\break
Technische Universit\accent127at Braunschweig\hfil\break
Pockelsstr. 14, D-38106 Braunschweig, Germany\hfil\break
e-mail: \Mailto{b.eick@tu-bs.de}

Franz G{\accent127 a}hler &
Fakult\accent127at f\accent127ur Mathematik, 
Universit\accent127at Bielefeld\hfil\break
Postfach 10 01 31, D-33501 Bielefeld, Germany\hfil\break
e-mail: \Mailto{gaehler@math.uni-bielefeld.de}

Werner Nickel &
Fachbereich Mathematik, AG2,
Technische Universit{\accent127 a}t Darmstadt, \hfill\break
Schlossgartenstra{\ss}e 7, D-64289 Darmstadt, Germany \hfil\break
e-mail: \Mailto{nickel@mathematik.tu-darmstadt.de}
\enditems

Please send bug reports, suggestions and other comments to any of these
e-mail addresses.







