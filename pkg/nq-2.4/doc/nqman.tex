% generated by GAPDoc2LaTeX from XML source (Frank Luebeck)
\documentclass[a4paper,11pt]{report}

\usepackage{a4wide}
\sloppy
\pagestyle{myheadings}
\usepackage{amssymb}
\usepackage[latin1]{inputenc}
\usepackage{makeidx}
\makeindex
\usepackage{color}
\definecolor{FireBrick}{rgb}{0.5812,0.0074,0.0083}
\definecolor{RoyalBlue}{rgb}{0.0236,0.0894,0.6179}
\definecolor{RoyalGreen}{rgb}{0.0236,0.6179,0.0894}
\definecolor{RoyalRed}{rgb}{0.6179,0.0236,0.0894}
\definecolor{LightBlue}{rgb}{0.8544,0.9511,1.0000}
\definecolor{Black}{rgb}{0.0,0.0,0.0}

\definecolor{linkColor}{rgb}{0.0,0.0,0.554}
\definecolor{citeColor}{rgb}{0.0,0.0,0.554}
\definecolor{fileColor}{rgb}{0.0,0.0,0.554}
\definecolor{urlColor}{rgb}{0.0,0.0,0.554}
\definecolor{promptColor}{rgb}{0.0,0.0,0.589}
\definecolor{brkpromptColor}{rgb}{0.589,0.0,0.0}
\definecolor{gapinputColor}{rgb}{0.589,0.0,0.0}
\definecolor{gapoutputColor}{rgb}{0.0,0.0,0.0}

%%  for a long time these were red and blue by default,
%%  now black, but keep variables to overwrite
\definecolor{FuncColor}{rgb}{0.0,0.0,0.0}
%% strange name because of pdflatex bug:
\definecolor{Chapter }{rgb}{0.0,0.0,0.0}
\definecolor{DarkOlive}{rgb}{0.1047,0.2412,0.0064}


\usepackage{fancyvrb}

\usepackage{mathptmx,helvet}
\usepackage[T1]{fontenc}
\usepackage{textcomp}


\usepackage[
            pdftex=true,
            bookmarks=true,        
            a4paper=true,
            pdftitle={Written with GAPDoc},
            pdfcreator={LaTeX with hyperref package / GAPDoc},
            colorlinks=true,
            backref=page,
            breaklinks=true,
            linkcolor=linkColor,
            citecolor=citeColor,
            filecolor=fileColor,
            urlcolor=urlColor,
            pdfpagemode={UseNone}, 
           ]{hyperref}

\newcommand{\maintitlesize}{\fontsize{50}{55}\selectfont}

% write page numbers to a .pnr log file for online help
\newwrite\pagenrlog
\immediate\openout\pagenrlog =\jobname.pnr
\immediate\write\pagenrlog{PAGENRS := [}
\newcommand{\logpage}[1]{\protect\write\pagenrlog{#1, \thepage,}}
%% were never documented, give conflicts with some additional packages

\newcommand{\GAP}{\textsf{GAP}}

%% nicer description environments, allows long labels
\usepackage{enumitem}
\setdescription{style=nextline}

%% depth of toc
\setcounter{tocdepth}{1}





%% command for ColorPrompt style examples
\newcommand{\gapprompt}[1]{\color{promptColor}{\bfseries #1}}
\newcommand{\gapbrkprompt}[1]{\color{brkpromptColor}{\bfseries #1}}
\newcommand{\gapinput}[1]{\color{gapinputColor}{#1}}


\begin{document}

\logpage{[ 0, 0, 0 ]}
\begin{titlepage}
\mbox{}\vfill

\begin{center}{\maintitlesize \textbf{\textsf{NQ}\mbox{}}}\\
\vfill

\hypersetup{pdftitle=\textsf{NQ}}
\markright{\scriptsize \mbox{}\hfill \textsf{NQ} \hfill\mbox{}}
{\Huge \textbf{  A \textsf{GAP} 4 Package\\
 computing nilpotent factor groups of finitely presented groups\\
 {\nobreakspace}\\
 Based on the ANU Nilpotent Quotient Program \mbox{}}}\\
\vfill

{\Huge Version 2.4\mbox{}}\\[1cm]
{12 January 2012\mbox{}}\\[1cm]
\mbox{}\\[2cm]
{\Large \textbf{ Max Horn     \mbox{}}}\\
{\Large \textbf{ Werner Nickel     \mbox{}}}\\
\hypersetup{pdfauthor= Max Horn     ;  Werner Nickel     }
\end{center}\vfill

\mbox{}\\
{\mbox{}\\
\small \noindent \textbf{ Max Horn     }  Email: \href{mailto:// mhorn@tu-bs.de } {\texttt{ mhorn@tu-bs.de }}\\
  Homepage: \href{http://www.icm.tu-bs.de/~mhorn} {\texttt{http://www.icm.tu-bs.de/\texttt{\symbol{126}}mhorn}}\\
  Address: \begin{minipage}[t]{8cm}\noindent
 AG Algebra und Diskrete Mathematik\\
 Institut Computational Mathematics\\
 TU Braunschweig\\
 Pockelsstr. 14\\
 D-38106 Braunschweig\\
 Germany \end{minipage}
}\\
{\mbox{}\\
\small \noindent \textbf{ Werner Nickel     }\\
  Homepage: \href{http://www.mathematik.tu-darmstadt.de/~nickel} {\texttt{http://www.mathematik.tu-darmstadt.de/\texttt{\symbol{126}}nickel}}}\\
\end{titlepage}

\newpage\setcounter{page}{2}
{\small 
\section*{Copyright}
\logpage{[ 0, 0, 1 ]}
  {\copyright} 1992-2007 Werner Nickel. \mbox{}}\\[1cm]
{\small 
\section*{Acknowledgements}
\logpage{[ 0, 0, 2 ]}
  The author of ANU NQ is Werner Nickel. 

The development of this program was started while the author was supported by
an Australian National University PhD scholarship and an Overseas Postgraduate
Research Scholarship. 

Further development of this program was done with support from the
DFG-Schwerpunkt-Projekt "`Algorithmische Zahlentheorie und Algebra"'. 

Since then, maintenance of ANU NQ has been taken over by Max Horn. All credit
for creating ANU NQ still goes to Werner Nickel as sole author. However, bug
reports and other inquiries should be sent to Max Horn. 

The following are the original acknowledgements by Werner Nickel. 

Over the years a number of people have made useful suggestions that found
their way into the code: Mike Newman, Michael Vaughan-Lee, Joachim
Neub{\"u}ser, Charles Sims. 

Thanks to Volkmar Felsch and Joachim Neub{\"u}ser for their careful
examination of the package prior to its release for GAP 4. 

This documentation was prepared with the \textsf{GAPDoc} package by Frank L{\"u}beck and Max Neunh{\"o}ffer. \mbox{}}\\[1cm]
\newpage

\def\contentsname{Contents\logpage{[ 0, 0, 3 ]}}

\tableofcontents
\newpage

  
\chapter{\textcolor{Chapter }{Introduction}}\logpage{[ 1, 0, 0 ]}
\hyperdef{L}{X7DFB63A97E67C0A1}{}
{
  This package provides an interface between \textsf{GAP} 4 and the Australian National University Nilpotent Quotient Program (ANU NQ).
The ANU NQ was implemented as part of the author's work towards his PhD at the
Australian National University, hence the name of the program. The program
takes as input a finite presentation of a group and successively computes
factor groups modulo the terms of the lower central series of the group. These
factor groups are computed in terms of polycyclic presentations. 

 The ANU NQ is implemented in the programming language C. The implementation
has been developed in a Unix environment and Unix is currently the only
operating system supported. It runs on a number of different Unix versions,
e.g. Solaris and Linux. 

 For integer matrix computations it relies on the GNU MP \cite{GNUMP} package and requires this package to be installed on your system. 

 This package relies on the functionality for polycyclic groups provided by the \textsf{GAP} package \textsf{polycyclic} \cite{polycyclic} and requires the package \textsf{polycyclic} to be installed as a \textsf{GAP} package on your computer system. 

 Comments, bug reports and suggestions are very welcome. 

 This manual contains references to parts of the \textsf{GAP} Reference Manual which are typeset in a slightly idiosyncratic way. The
following example shows how such references are printed: 'For further
information on creating a free group see \texttt{FreeGroup} (\textbf{Reference: FreeGroup}).' The text in bold face refers to the \textsf{GAP} Reference Manual. 

Each item in the list of references at the end of this manual is followed by a
list of numbers that specify the pages of the manual where the reference
occurs. }

 
\chapter{\textcolor{Chapter }{General remarks}}\logpage{[ 2, 0, 0 ]}
\hyperdef{L}{X7A696C2A78E88D1A}{}
{
 In this chapter we define notation used throughout this manual and recollect
basic facts about nilpotent groups. We also provide some background
information about the functionality implemented in this package. 
\section{\textcolor{Chapter }{Commutators and the Lower Central Series}}\logpage{[ 2, 1, 0 ]}
\hyperdef{L}{X7E33A61A831C0068}{}
{
 \index{commutator} The \emph{commutator} of two elements $h_1$ and $h_2$ of a group $G$ is the element $h_1^{-1}h_2^{-1}h_1h_2$ and is denoted by $[h_1,h_2]$. It satisfies the equation $h_1h_2 = h_2h_1[h_1,h_2]$ and can be interpreted as the correction term that has to be introduced into a
word if two elements of a group are interchanged. Iterated commutators are
written in \emph{left-normed fashion}: \index{left-normed commutator} $[h_1,h_2,\ldots,h_{n-1},h_n]=[[h_1,h_2,\ldots,h_{n-1}],h_n]$. 

 \index{lower central series} The \emph{lower central series} of $G$ is defined inductively as $\gamma_1(G) = G, \gamma_i(G) = [\gamma_{i-1}(G),G]$ for $i \ge 2$. Each term in the lower central series is a normal (even fully invariant)
subgroup of $G$. The factors of the lower central series are abelian groups. On each factor
the induced action of $G$ via conjugation is the trivial action. 

The factor $\gamma_k(G)/\gamma_{k+1}(G)$ is generated by the elements $[g,h]\gamma_{k+1}(G),$ where $g$ runs through a set of (representatives of) generators for $G/\gamma_2(G)$ and $h$ runs through a set of (representatives of) generators for $\gamma_{k-1}(G)/\gamma_k(G).$ Therefore, each factor of the lower central series is finitely generated if $G$ is finitely generated. 

 If one factor of the lower central series is finite, then all subsequent
factors are finite. Then the exponent of the $k+1$-th factor is a divisor of the exponent of the $k$-th factor of the lower central series. In particular, the exponents of all
factors of the lower central series are bounded by the exponent of the first
finite factor of the lower central series. }

 
\section{\textcolor{Chapter }{Nilpotent groups}}\logpage{[ 2, 2, 0 ]}
\hyperdef{L}{X8463EF6A821FFB69}{}
{
 \index{nilpotent} A group $G$ is called \emph{nilpotent} if there is a positive integer $c$ such that all $(c+1)$-fold commutators are trivial in $G.$ The smallest integer with this property is called the \index{nilpotency class}\index{class} \emph{nilpotency class} of $G$. In terms of the lower central series a group $G \not= 1$ has nilpotency class $c$ if and only if $\gamma_{c}(G) \not= 1$ and $\gamma_{c+1}(G) = 1$. 

Examples of nilpotent groups are finite $p$-groups, the group of unitriangular matrices over a ring with one and the
factor groups of a free group modulo the terms of its lower central series. 

Finiteness of a nilpotent group can be decided by the group's commutator
factor group. A nilpotent group is finite if and only if its commutator factor
group is finite. A group whose commutator factor group is finite can only have
finite nilpotent quotient groups. 

 By refining the lower central series of a finitely generated nilpotent group
one can obtain a (sub)normal series $G_1>G_2>...>G_{k+1}=1$ with cyclic (central) factors. Therefore, every finitely generated nilpotent
group is \index{polycyclic}\emph{polycyclic}. Such a \emph{polycyclic series} gives rise to a \index{polycyclic generating sequence} polycyclic generating sequence by choosing a generator $a_i$ for each cyclic factor $G_i/G_{i+1}$. Let $I$ be the set of indices such that $G_i/G_{i+1}$ is finite. A simple induction argument shows that every element of the group
can be written uniquely as a \emph{normal word} $a_1^{e_1}\ldots a_n^{e_n}$ with integers $e_i$ and $0\leq e_i<m_i$ for $i\in I$. }

 
\section{\textcolor{Chapter }{Nilpotent presentations }}\logpage{[ 2, 3, 0 ]}
\hyperdef{L}{X8268F8197E6BD786}{}
{
 

From a polycyclic generating sequence one can obtain a \index{polycyclic presentation} \emph{polycyclic presentation} for the group. The following set of power and commutator relations is a
defining set of relations. The \index{power relation} \emph{power relations} express $a_i^{m_i}$ in terms of the generators $a_{i+1},\ldots,a_n$ whenever $G_i/G_{i+1}$ is finite with order $m_i$. The \index{commutator relation} \emph{commutator relations} are obtained by expressing $[a_j,a_i]$ for $j>i$ as a word in the generators $a_{i+1},\ldots,a_n$. If the polycyclic series is obtained from refining the lower central series,
then $[a_j,a_i]$ is even a word in $a_{j+1},\ldots,a_n$. In this case we obtain a nilpotent presentation. 

To be more precise, a \index{nilpotent presentation} \emph{nilpotent presentation} is given on a finite number of generators $a_1,\ldots,a_n$. Let $I$ be the set of indices such that $G_i/G_{i+1}$ is finite. Let $m_i$ be the order of $G_i/G_{i+1}$ for $i\in I$. Then a nilpotent presentation has the form 
\[ \langle a,\ldots,a_n | a_i^{m_i} = w_{ii}(a_{i+1},\ldots,a_n) \mbox{ for }
i\in I;\; [a_j,a_i] = w_{ij}(a_{j+1},\ldots,a_n) \mbox{ for } 1\leq i < j\leq
n\rangle \]
 Here, $w_{ij}(a_k,\ldots,a_n)$ denotes a group word in the generators $a_k,\ldots,a_n$. 

In a group given by a polycyclic presentation each element in the group can be
written as a \emph{normal word} $a_1^{e_1}\ldots a_n^{e_n}$ with $e_i \in \mathbb{Z}$ and $0 \leq e_i < m_i$ for $i \in I$. A procedure called \emph{collection} can be used to convert an arbitrary word in the generators into an equivalent
normal word. In general, the resulting normal word need not be unique. The
result of collecting a word may depend on the steps chosen during the
collection procedure. A polycyclic presentation with the property that two
different normal words are never equivalent is called \emph{consistent}\index{consistent}. A polycyclic presentation derived from a polycyclic series as above is
consistent. The following example shows an inconsistent polycyclic
presentation 
\[\langle a,b\mid a^2, b^a = b^2 \rangle \]
 as $b = baa = ab^2a = a^2b^4 = b^4$ which implies $b^3=1$. Here we have the equivalent normal words $b^3$ and the empty word. It can be proved that consistency can be checked by
collecting a finite number of words in the given generating set in two
essentially different ways and checking if the resulting normal forms are the
same in both cases. See Chapter 9 of the book \cite{Sims94} for an introduction to polycyclic groups and polycyclic presentations. 

For computations in a polycyclic group one chooses a consistent polycyclic
presentation as it offers a simple solution to the word problem: Equality
between two words is decided by collecting both words to their respective
normal forms and comparing the normal forms. Nilpotent groups and nilpotent
presentations are special cases of polycyclic groups and polycyclic
presentations. Nilpotent presentations allow specially efficient collection
methods. The package \textsf{Polycyclic} provides algorithms to compute with polycyclic groups given by a polycyclic
presentation. 

However, inconsistent nilpotent presentations arise naturally in the nilpotent
quotient algorithm. There is an algorithm based on the test words for
consistency mentioned above to modify the arising inconsistent presentations
suitably to obtain a consistent one for the same group. }

 
\section{\textcolor{Chapter }{A sketch of the algorithm}}\logpage{[ 2, 4, 0 ]}
\hyperdef{L}{X7DAF9CC17F6B868D}{}
{
 The input for the ANU NQ in its simplest form is a finite presentation $\langle X|R\rangle$ for a group $G$. The first step of the algorithm determines a nilpotent presentation for the
commutator quotient of $G$. This is a presentation of the class-$1$ quotient of $G$. Call its generators $a_1,...,a_d$. It also determines a homomorphism of $G$ onto the commutator quotient and describes it by specifying the image of each
generator in $X$ as a word in the $a_i$. 

For the general step assume that the algorithm has computed a nilpotent
presentation for the class-$c$ quotient of $G$ and that $a_1,...,a_d$ are the generators introduced in the first step of the algorithm. Furthermore,
there is a map from X into the class-$c$ quotient describing the epimorphism from $G$ onto $G/\gamma_{c+1}(G)$. 

Let $b_1,...b_k$ be the generators from the last step of the algorithm, the computation of $\gamma_c(G)/\gamma_{c+1}(G)$. This means that $b_1,...b_k$ generate $\gamma_c(G)/\gamma_{c+1}(G)$. Then the commutators $[b_j,a_i]$ generate $\gamma_{c+1}(G)/\gamma_{c+2}(G)$. The algorithm introduces new, central generators $c_{ij}$ into the presentation, adds the relations $[b_j,a_i] = c_{ij}$ and modifies the existing relations by appending suitable words in the $c_{ij}$, called \emph{tails}, to the right hand sides of the power and commutator relations. The resulting
presentation is a nilpotent presentation for the \emph{nilpotent cover} of $G/\gamma_{c+1}(G)$. The nilpotent cover is the largest central extension of $G/\gamma_{c+1}(G)$ generated by $d$ elements. It is is uniquely determined up to isomorphism. 

The resulting presentation of the nilpotent cover is in general inconsistent.
Consistency is achieved by running the consistency test. This results in
relations among the generators $c_{ij}$ which can be used to eliminate some of those generators or introduce power
relations. After this has been done we have a consistent nilpotent
presentation for the nilpotent cover of $G/\gamma_{c+1}(G)$. 

Furthermore, the nilpotent cover need not satisfy the relations of $G$. In other words, the epimorphism from $G$ onto $G/\gamma_{c+1}(G)$ cannot be lifted to an epimorphism onto the nilpotent cover. Applying the
epimorphism to each relator of $G$ and collecting the resulting words of the nilpotent cover yields a set of
words in the $c_{ij}$. This gives further relations between the $c_{ij}$ which leads to further eliminations or modifications of the power relations
for the $c_{ij}$. 

After this, the inductive step of the ANU NQ is complete and a consistent
nilpotent presentation for $G/\gamma_{c+2}(G)$ is obtained together with an epimorphism from $G$ onto the class-$(c+1)$ quotient. 

Chapter 11 of the book \cite{Sims94} discusses a nilpotent quotient algorithm. A description of the implementation
in the ANU NQ is contained in \cite{Nickel96} }

 
\section{\textcolor{Chapter }{Identical Relations}}\logpage{[ 2, 5, 0 ]}
\hyperdef{L}{X84EF796487BC1822}{}
{
\label{IdRels} Let $w$ be a word in free generators $x_1,\ldots,x_n$. A group $G$ satisfies the relation $w=1$ \emph{identically} if each map from $x_1,\ldots,x_n$ into $G$ maps $w$ to the identity element of $G$. We also say that $G$ satisfies the \index{identical relation} \index{law} \emph{identical relation} $w=1$ or satisfies the \emph{law} $w=1$. In slight abuse of notation, we call the elements $x_1,\ldots,x_n$ \index{identical generator} \emph{identical} generators. 

 Common examples of identical relations are: A group of nilpotency class at
most $c$ satisfies the law $[x_1,\ldots,x_{c+1}]=1$. A group that satisfies the law $[x,y,\ldots,y]=1$ where $y$ occurs $n$-times, is called an $n$-Engel group. A group that satisfies the law $x^d=1$ is a group of exponent $d$. 

 To describe finitely presented groups that satisfy one or more laws, we extend
a common notation for finitely presented groups by specifying the identical
generators as part of the generator list, separated from the group generators
by a semicolon: For example 
\[ \langle a,b,c; x,y | x^5, [x,y,y,y]\rangle \]
 is a group on 3 generators $a,b,c$ of exponent $5$ satisfying the 3rd Engel law. The presentation above is equivalent to a
presentation on 3 generators with an infinite set of relators, where the set
of relators consists of all fifth powers of words in the generators and all
commutators $[x,y,y,y]$ where $x$ and $y$ run through all words in the generators $a,b,c$. The standalone programme accepts the notation introduced above as a
description of its input. In \textsf{GAP 4} finitely presented groups are specified in a different way, see \texttt{NilpotentQuotient} (\ref{NilpotentQuotient}) for a description. 

 This notation can also be used in words that mix group and identical
generators as in the following example: 
\[ \langle a,b,c; x | [x,c], [a,x,x,x] \rangle \]
 The first relator specifies a law which says that $c$ commutes with all elements of the group. The second turns $a$ into a third right Engel element. 

An element $a$ is called \emph{a right $n$-th Engel element} or \emph{a right $n$-Engel element} \index{right Engel element} if it satisfies the commutator law $[a,x,...,x]=1$ where the identical generator $x$ occurs $n$-times. Likewise, an element $b$ is called an \emph{left $n$-th Engel element} or \emph{left $n$-Engel element} \index{left Engel element} if it satisfies the commutator law $[x,b,b,...b]=1$. 

Let $G$ be a nilpotent group. Then $G$ satisfies a given law if the law is satisfied by a certain finite set of
instances given by Higman's Lemma, see \cite{Higman59}. The ANU NQ uses Higman's Lemma to obtain a finite presentation for groups
that satisfy one or several identical relations. }

 
\section{\textcolor{Chapter }{Expression Trees}}\label{ExpTrees}
\logpage{[ 2, 6, 0 ]}
\hyperdef{L}{X861A2C6385F6BCF5}{}
{
 Expressions involving commutators play an important role in the context of
nilpotent groups. Expanding an iterated commutator produces a complicated and
long expression. For example, 
\[ [x,y,z] = y^{-1}x^{-1}yxz^{-1}x^{-1}y^{-1}xyz. \]
 Evaluating a commutator $[a,b]$ is done efficiently by computing the equation $(ba)^{-1}ab$. Therefore, for each commutator we need to perform two multiplications and
one inversion. Evaluating $[x,y,z]$ needs four multiplications and two inversions. Evaluation of an iterated
commutator with $n$ components takes $2n-1$ multiplications and $n-1$ inversions. The expression on the right hand side above needs $9$ multiplications and $5$ inversions which is clearly much more expensive than evaluating the commutator
directly. 

 Assuming that no cancellations occur, expanding an iterated commutator with n
components produces a word with $2^{n+1}-2^{n-1}-2$ factors half of which are inverses. A similar effect occurs whenever a compact
expression is expanded into a word in generators and inverses, for example $(ab)^{49}$. 

 Therefore, it is important not to expand expressions into a word in generators
and inverses. For this purpose we provide a mechanism which we call here \index{expression trees} \emph{expression trees}. An expression tree preserves the structure of a given expression. It is a
(binary) tree in which each node is assigned an operation and whose leaves are
generators of a free group or integers. For example, the expression $[(xy)^2, z]$ is stored as a tree whose top node is a commutator node. The right subtree is
just a generator node (corresponding to $z$). The left subtree is a power node whose subtrees are a product node on the
left and an integer node on the right. An expression tree can involve
products, powers, conjugates and commutators. However, the list of available
operations can be extended. 

Evaluation of an expression tree is done recursively and requires as many
operations as there are nodes in the tree. An expression tree can be evaluated
in a specific group by the function \texttt{EvaluateExpTree} (\ref{EvaluateExpTree}). 

A presentation specified by expression trees is a record with the components \texttt{.generators} and \texttt{.relations}. See section \ref{FunctionsExpTrees} for a description of the functions that produce and manipulate expression
trees. 
\begin{Verbatim}[commandchars=!@|,fontsize=\small,frame=single,label=Example]
  !gapprompt@gap>| !gapinput@RequirePackage( "nq" );|
  true
  !gapprompt@gap>| !gapinput@gens := ExpressionTrees( 2 );|
  [ x1, x2 ]
  !gapprompt@gap>| !gapinput@r1 := LeftNormedComm( [gens[1],gens[2],gens[2]] );|
  Comm( x1, x2, x2 )
  !gapprompt@gap>| !gapinput@r2 := LeftNormedComm( [gens[1],gens[2],gens[2],gens[1]] );|
  Comm( x1, x2, x2, x1 )
  !gapprompt@gap>| !gapinput@pres := rec( generators := gens, relations := [r1,r2] );|
  rec( generators := [ x1, x2 ], 
    relations := [ Comm( x1, x2, x2 ), Comm( x1, x2, x2, x1 ) ] )
    
\end{Verbatim}
 }

 
\section{\textcolor{Chapter }{A word about the implementation}}\logpage{[ 2, 7, 0 ]}
\hyperdef{L}{X7E27CA7F7E797520}{}
{
 The ANU NQ is written in C, but not in ANSI C. I hope to make one of the next
versions ANSI compliable. However, it uses a fairly restricted subset of the
language so that it should be easy to compile it in new environments. The code
is 64-bit clean. If you have difficulties with porting it to a new
environment, let me know and I'll be happy to assist if time permits. 

The program has two collectors: a simple collector from the left as described
in \cite{LS90} and a combinatorial from the left collector as described in \cite{VL90a}. The combinatorial collector is always faster than the simple collector,
therefore, it is the collector used by this package by default. This can be
changed by modifying the global variable \texttt{NqDefaultOptions} (\ref{NqDefaultOptions}). 

In a polycyclic group with generators that do not have power relations,
exponents may become arbitrarily large. Experience shows that this happens
rarely in the computations done by the ANU NQ. Exponents are represented by
32-bit integers. The collectors perform an overflow check and abort the
computation if an overflow occurred. In a GNU environment the program can be
compiled using the `long long' 64-bit integer type. For this uncomment the
relevant line in src/Makefile and recompile the program. 

As part of the step that enforces consistency and the relations of the group,
the ANU NQ performs computations with integer matrices and converts them to
Hermite Normal Form. The algorithm used here is a variation of the
Kanan-Bachem algorithm based on the GNU multiple precision package GNU MP \cite{GNUMP}. Experience shows that the integer matrices are usually fairly sparse and
Kanan-Bachem seems to be sufficient in this context. However, the
implementation might benefit from a more efficient strategy for computing
Hermite Normal Forms. This is a topic for further investigations. 

As the program does not compute the Smith Normal Form for each factor of the
lower central series but the Hermite Normal Form, it does not necessarily
obtain a minimal generating set for each factor of the lower central series.
The following is a simple example of this behaviour. We take the presentation 
\[ \langle x, y | x^2 = y \rangle \]
 The group is clearly isomorphic to the additive group of the integers.
Applying the ANU NQ to this presentation gives the following nilpotent
presentation: 
\[ \langle A,B | A^2 = B, [B,A] \rangle \]
 A nilpotent presentation on a minimal generating set would be the presentation
of the free group on one generator: 
\[ \langle A | \; \rangle \]
 }

 
\section{\textcolor{Chapter }{The input format of the standalone}}\logpage{[ 2, 8, 0 ]}
\hyperdef{L}{X79E150AA823439A8}{}
{
\label{InputForm} The input format for finite presentations resembles the way many people write
down a presentation on paper. Here are some examples of presentations that the
ANU NQ accepts: 
\begin{verbatim}  
  
      < a, b | >                       # free group of rank 2
  
      < a, b, c; x, y | 
                  [a,b,c],             # a left normed commutator
                  [b,c,c,c]^6,         # another one raised to a power
                  a^2 = c^-3*a^2*c^3,  # a relation
                  a^(b*c) = a,         # a conjugate relation
                  (a*[b,(a*c)])^6,     # something that looks complicated
                  [x,y,y,y,y],         # an identical relation
                  [c,x,x,x,x,x]        # c is a fifth right Engel element
      >
  
\end{verbatim}
 A presentation starts with '{\textless}' followed by a list of generators
separated by commas. Generator names are strings that contain only upper and
lower case letters, digits, dots and underscores and that do not start with a
digit. The list of generator names is separated from the list of
relators/relations by the symbol '$\mid$'. The list of generators can be followed by a list of identical generators
separated by a semicolon. Relators and relations are separated by commas and
can be mixed arbitrarily. Parentheses can be used in order to group
subexpressions together. Square brackets can be used in order to form left
normed commutators. The symbols '*' and '\texttt{\symbol{94}}' can be used to
form products and powers, respectively. The presentation finishes with the
symbol '{\textgreater}'. A comment starts with the symbol '\#' and finishes at
the end of the line. The file src/presentation.c contains a complete grammar
for the presentations accepted by the ANU NQ. 

Typically, the input for the standalone is put into a file by using a standard
text editor. The file can be passed as an argument to the function \texttt{NilpotentQuotient} (\ref{NilpotentQuotient}). It is also possible to put a presentation in the standalone's input format
into a string and use the string as argument for \texttt{NilpotentQuotient} (\ref{NilpotentQuotient}). }

 }

 
\chapter{\textcolor{Chapter }{The Functions of the Package}}\logpage{[ 3, 0, 0 ]}
\hyperdef{L}{X82738C527E6AC670}{}
{
  \index{Nilpotent Quotient Package} 
\section{\textcolor{Chapter }{Nilpotent Quotients of Finitely Presented Groups}}\logpage{[ 3, 1, 0 ]}
\hyperdef{L}{X7D147D4182F85244}{}
{
 

\subsection{\textcolor{Chapter }{NilpotentQuotient}}
\logpage{[ 3, 1, 1 ]}\nobreak
\hyperdef{L}{X8216791583DE512C}{}
{\noindent\textcolor{FuncColor}{$\triangleright$\ \ \texttt{NilpotentQuotient({\mdseries\slshape [output-file, ]fp-group[, id-gens][, c]})\index{NilpotentQuotient@\texttt{NilpotentQuotient}}
\label{NilpotentQuotient}
}\hfill{\scriptsize (function)}}\\
\noindent\textcolor{FuncColor}{$\triangleright$\ \ \texttt{NilpotentQuotient({\mdseries\slshape [output-file, ]input-file[, c]})\index{NilpotentQuotient@\texttt{NilpotentQuotient}}
\label{NilpotentQuotient}
}\hfill{\scriptsize (function)}}\\


 The parameter \texttt{fp-group} is either a finitely presented group or a record specifying a presentation by
expression trees (see section \ref{ExpTrees}). The parameter \texttt{input-file} is a string specifying the name of a file containing a finite presentation in
the input format (cf. section \ref{InputForm}) of the ANU NQ. Such a file can be prepared by a text editor or with the help
of the function \texttt{NqStringFpGroup} (\ref{NqStringFpGroup}). 

 Let $G$ be the group defined by \texttt{fp-group} or the group defined in \texttt{input-file}. The function computes a nilpotent presentation for $G/\gamma_{c+1}(G)$ if the optional parameter \texttt{c} is specified. If \texttt{c} is not given, then the function attempts to compute the largest nilpotent
quotient of $G$ and it will terminate only if $G$ has a largest nilpotent quotient. See section \ref{DiagnosticOutput} for a possibility to follow the progress of the computation. 

 The optional argument \texttt{id-gens} is a list of generators of the free group underlying the finitely presented
group \texttt{fp-group}. The generators in this list are treated as identical generators.
Consequently, all relations of the \texttt{fp-group} involving these generators are treated as identical relations for these
generators. 

 In addition to the arguments explained above, the function accepts the
following options as shown in the first example below: \index{options} 
\begin{itemize}
\item \texttt{group} \index{options!group} This option can be used instead of the parameter \texttt{fp-group}. 
\item \texttt{input\texttt{\symbol{92}}{\textunderscore}string} \index{options!input\_string} This option can be used to specify a finitely presented group by a string in
the input format of the standalone program. 
\item \texttt{input\texttt{\symbol{92}}{\textunderscore}file} \index{options!input\_file} This option specifies a file with input for the standalone program. 
\item \texttt{output\texttt{\symbol{92}}{\textunderscore}file} \index{options!ouput\_file} This option specifies a file for the output of the standalone. 
\item \texttt{idgens} \index{options!idgens} This options specifies a list of identical generators. 
\item \texttt{class} \index{options!class} This option specifies the nilpotency class up to which the nilpotent quotient
will be computed. 
\end{itemize}
 

The following example computes the class-5 quotient of the free group on two
generators. 
\begin{Verbatim}[commandchars=!@|,fontsize=\small,frame=single,label=Example]
  
  !gapprompt@gap>| !gapinput@F := FreeGroup( 2 );|
  <free group on the generators [ f1, f2 ]>
  !gapprompt@gap>| !gapinput@## Equivalent to:  NilpotentQuotient( : group := F, class := 5 );|
  !gapprompt@gap>| !gapinput@##                 NilpotentQuotient( F : class := 5 );          |
  !gapprompt@gap>| !gapinput@H := NilpotentQuotient( F, 5 );|
  Pcp-group with orders [ 0, 0, 0, 0, 0, 0, 0, 0, 0, 0, 0, 0, 0, 0 ]
  !gapprompt@gap>| !gapinput@lcs := LowerCentralSeries( H );;|
  !gapprompt@gap>| !gapinput@for i in [1..5] do Print( lcs[i] / lcs[i+1], "\n" ); od;|
  Pcp-group with orders [ 0, 0 ]
  Pcp-group with orders [ 0 ]
  Pcp-group with orders [ 0, 0 ]
  Pcp-group with orders [ 0, 0, 0 ]
  Pcp-group with orders [ 0, 0, 0, 0, 0, 0 ]
  
\end{Verbatim}
 Note that the lower central series in the example is part of the data returned
by the standalone program. Therefore, the execution of the function
LowerCentralSeries takes no time. 

The next example computes the class-4 quotient of the infinite dihedral group.
The group is soluble but not nilpotent. The first factor of its lower central
series is a Klein four group and all the other factors are cyclic or order $2$. 
\begin{Verbatim}[commandchars=!@|,fontsize=\small,frame=single,label=Example]
  
  !gapprompt@gap>| !gapinput@F := FreeGroup( 2 );|
  <free group on the generators [ f1, f2 ]>
  !gapprompt@gap>| !gapinput@G := F / [F.1^2, F.2^2];|
  <fp group on the generators [ f1, f2 ]>
  !gapprompt@gap>| !gapinput@H := NilpotentQuotient( G, 4 ); |
  Pcp-group with orders [ 2, 2, 2, 2, 2 ]
  !gapprompt@gap>| !gapinput@lcs := LowerCentralSeries( H );;|
  !gapprompt@gap>| !gapinput@for i in [1..Length(lcs)-1] do|
  !gapprompt@>| !gapinput@      Print( AbelianInvariants(lcs[i] / lcs[i+1]), "\n" );|
  !gapprompt@>| !gapinput@od;|
  [ 2, 2 ]
  [ 2 ]
  [ 2 ]
  [ 2 ]
  !gapprompt@gap>| !gapinput@|
  
\end{Verbatim}
 In the following example identical generators are used in order to express the
fact that the group is nilpotent of class $3$. A group is nilpotent of class $3$ if it satisfies the identical relation $[x_1,x_2,x_3,x_4]=1$ (cf. Section \ref{IdRels}). The result is the free nilpotent group of class $3$ on two generators. 
\begin{Verbatim}[commandchars=!@|,fontsize=\small,frame=single,label=Example]
  
  !gapprompt@gap>| !gapinput@F := FreeGroup( "a", "b", "w", "x", "y", "z" );|
  <free group on the generators [ a, b, w, x, y, z ]>
  !gapprompt@gap>| !gapinput@G := F / [ LeftNormedComm( [F.3,F.4,F.5,F.6] ) ];|
  <fp group of size infinity on the generators [ a, b, w, x, y, z ]>
  !gapprompt@gap>| !gapinput@## The following is equivalent to: |
  !gapprompt@gap>| !gapinput@##   NilpotentQuotient( G : idgens := [F.3,F.4,F.5,F.6] );|
  !gapprompt@gap>| !gapinput@H := NilpotentQuotient( G, [F.3,F.4,F.5,F.6] );|
  Pcp-group with orders [ 0, 0, 0, 0, 0 ]
  !gapprompt@gap>| !gapinput@NilpotencyClassOfGroup(H);|
  3
  !gapprompt@gap>| !gapinput@LowerCentralSeries(H);|
  [ Pcp-group with orders [ 0, 0, 0, 0, 0 ], Pcp-group with orders [ 0, 0, 0 ], 
    Pcp-group with orders [ 0, 0 ], Pcp-group with orders [  ] ]
  
\end{Verbatim}
 The following example uses expression trees in order to specify the third
Engel law for the free group on $3$ generators. 
\begin{Verbatim}[commandchars=!@|,fontsize=\small,frame=single,label=Example]
  
  !gapprompt@gap>| !gapinput@et := ExpressionTrees( 5 );                            |
  [ x1, x2, x3, x4, x5 ]
  !gapprompt@gap>| !gapinput@comm := LeftNormedComm( [et[1], et[2], et[2], et[2]] );|
  Comm( x1, x2, x2, x2 )
  !gapprompt@gap>| !gapinput@G := rec( generators := et, relations := [comm] );|
  rec( generators := [ x1, x2, x3, x4, x5 ], 
    relations := [ Comm( x1, x2, x2, x2 ) ] )
  !gapprompt@gap>| !gapinput@H := NilpotentQuotient( G : idgens := [et[1],et[2]] );|
  Pcp-group with orders [ 0, 0, 0, 0, 0, 0, 0, 0, 0, 0, 0, 0, 0, 0, 2, 4, 2, 2, 
    0, 6, 6, 0, 0, 2, 10, 10, 10 ]
  !gapprompt@gap>| !gapinput@TorsionSubgroup( H );|
  Pcp-group with orders [ 2, 2, 2, 2, 2, 2, 2, 10, 10, 10 ]
  !gapprompt@gap>| !gapinput@lcs := LowerCentralSeries( H );;|
  !gapprompt@gap>| !gapinput@NilpotencyClassOfGroup( H );|
  5
  !gapprompt@gap>| !gapinput@for i in [1..5] do Print( lcs[i] / lcs[i+1], "\n" ); od;|
  Pcp-group with orders [ 0, 0, 0 ]
  Pcp-group with orders [ 0, 0, 0 ]
  Pcp-group with orders [ 0, 0, 0, 0, 0, 0, 0, 0 ]
  Pcp-group with orders [ 2, 4, 2, 2, 0, 6, 6, 0, 0, 2 ]
  Pcp-group with orders [ 10, 10, 10 ]
  !gapprompt@gap>| !gapinput@for i in [1..5] do Print( AbelianInvariants(lcs[i]/lcs[i+1]), "\n" ); od;|
  [ 0, 0, 0 ]
  [ 0, 0, 0 ]
  [ 0, 0, 0, 0, 0, 0, 0, 0 ]
  [ 2, 2, 2, 2, 2, 2, 2, 0, 0, 0 ]
  [ 10, 10, 10 ]
  
\end{Verbatim}
 The example above also shows that the relative orders of an abelian polycyclic
group need not be the abelian invariants (elementary divisors) of the group.
Each zero corresponds to a generator of infinite order. The number of zeroes
is always correct. }

 

\subsection{\textcolor{Chapter }{NilpotentEngelQuotient}}
\logpage{[ 3, 1, 2 ]}\nobreak
\hyperdef{L}{X7ACCB6267C187AB0}{}
{\noindent\textcolor{FuncColor}{$\triangleright$\ \ \texttt{NilpotentEngelQuotient({\mdseries\slshape [output-file, ]fp-group, n[, id-gens][, c]})\index{NilpotentEngelQuotient@\texttt{NilpotentEngelQuotient}}
\label{NilpotentEngelQuotient}
}\hfill{\scriptsize (function)}}\\
\noindent\textcolor{FuncColor}{$\triangleright$\ \ \texttt{NilpotentEngelQuotient({\mdseries\slshape [output-file, ]input-file, n[, c]})\index{NilpotentEngelQuotient@\texttt{NilpotentEngelQuotient}}
\label{NilpotentEngelQuotient}
}\hfill{\scriptsize (function)}}\\


 This function is a special version of \texttt{NilpotentQuotient} (\ref{NilpotentQuotient}) which enforces the $n$-th Engel identity on the nilpotent quotients of the group specified by \texttt{fp-group} or by \texttt{input-file}. It accepts the same options as \texttt{NilpotentQuotient}. 

The Engel condition can also be enforced by using identical generators and the
Engel law and \texttt{NilpotentQuotient} (\ref{NilpotentQuotient}). See the examples there. 

 The following example computes the relatively free fifth Engel group on two
generators, determines its (normal) torsion subgroup and computes the
corresponding quotient group. The quotient modulo the torsion subgroup is
torsion-free. Therefore, there is a nilpotent presentation without power
relations. The example computes a nilpotent presentation for the torsion free
factor group through the upper central series. The factors of the upper
central series in a torsion free group are torsion free. In this way one
obtains a set of generators of infinite order and the resulting nilpotent
presentation has no power relations. 
\begin{Verbatim}[commandchars=!@|,fontsize=\small,frame=single,label=Example]
  !gapprompt@gap>| !gapinput@G := NilpotentEngelQuotient( FreeGroup(2), 5 );|
  Pcp-group with orders [ 0, 0, 0, 0, 0, 0, 0, 0, 0, 0, 0, 0, 0, 0, 3, 0, 10, 
    0, 0, 30, 0, 3, 3, 10, 2, 0, 6, 0, 0, 30, 2, 0, 9, 3, 5, 2, 6, 2, 10, 5, 5, 
    2, 0, 3, 3, 3, 3, 3, 5, 5, 3, 3 ]
  !gapprompt@gap>| !gapinput@NilpotencyClassOfGroup(G);|
  9
  !gapprompt@gap>| !gapinput@T := TorsionSubgroup( G );|
  Pcp-group with orders [ 3, 3, 2, 2, 3, 3, 2, 9, 3, 5, 2, 3, 2, 10, 5, 2, 3, 
    3, 3, 3, 3, 5, 5, 3, 3 ]
  !gapprompt@gap>| !gapinput@IsAbelian( T );|
  true
  !gapprompt@gap>| !gapinput@AbelianInvariants( T );|
  [ 3, 3, 3, 3, 3, 3, 3, 3, 30, 30, 30, 180, 180 ]
  !gapprompt@gap>| !gapinput@H := G / T;|
  Pcp-group with orders [ 0, 0, 0, 0, 0, 0, 0, 0, 0, 0, 0, 0, 0, 0, 3, 0, 10, 
    0, 0, 30, 0, 5, 0, 2, 0, 0, 10, 0, 2, 5, 0 ]
  !gapprompt@gap>| !gapinput@H := PcpGroupBySeries( UpperCentralSeries(H), "snf" );|
  Pcp-group with orders [ 0, 0, 0, 0, 0, 0, 0, 0, 0, 0, 0, 0, 0, 0, 0, 0, 0, 0, 
    0, 0, 0, 0, 0 ]
  !gapprompt@gap>| !gapinput@ucs := UpperCentralSeries( H );;|
  !gapprompt@gap>| !gapinput@for i in [1..NilpotencyClassOfGroup(H)] do|
  !gapprompt@>| !gapinput@	Print( ucs[i]/ucs[i+1], "\n" );|
  !gapprompt@>| !gapinput@od;|
  Pcp-group with orders [ 0, 0 ]
  Pcp-group with orders [ 0 ]
  Pcp-group with orders [ 0, 0 ]
  Pcp-group with orders [ 0, 0, 0 ]
  Pcp-group with orders [ 0, 0, 0, 0, 0, 0 ]
  Pcp-group with orders [ 0, 0, 0, 0 ]
  Pcp-group with orders [ 0, 0 ]
  Pcp-group with orders [ 0, 0, 0 ]
\end{Verbatim}
 }

 

\subsection{\textcolor{Chapter }{NqEpimorphismNilpotentQuotient}}
\logpage{[ 3, 1, 3 ]}\nobreak
\hyperdef{L}{X8758F663782AE655}{}
{\noindent\textcolor{FuncColor}{$\triangleright$\ \ \texttt{NqEpimorphismNilpotentQuotient({\mdseries\slshape [output-file, ]fp-group[, id-gens][, c]})\index{NqEpimorphismNilpotentQuotient@\texttt{NqEpimorphismNilpotentQuotient}}
\label{NqEpimorphismNilpotentQuotient}
}\hfill{\scriptsize (function)}}\\


 This function computes an epimorphism from the group $G$ given by the finite presentation \texttt{fp-group} onto $G/\gamma_{c+1}(G).$ If \texttt{c} is not given, then the largest nilpotent quotient of $G$ is computed and an epimorphism from $G$ onto the largest nilpotent quotient of $G$. If $G$ does not have a largest nilpotent quotient, the function will not terminate if $c$ is not given. 

 The optional argument \texttt{id-gens} is a list of generators of the free group underlying the finitely presented
group \texttt{fp-group}. The generators in this list are treated as identical generators.
Consequently, all relations of the \texttt{fp-group} involving these generators are treated as identical relations for these
generators. 

If identical generators are specified, then the epimorphism returned maps the
group generated by the `non-identical' generators onto the nilpotent factor
group. See the last example below. 

The function understands the same options as the function \texttt{NilpotentQuotient} (\ref{NilpotentQuotient}). 
\begin{Verbatim}[commandchars=!@|,fontsize=\small,frame=single,label=Example]
  
  !gapprompt@gap>| !gapinput@F := FreeGroup(3);                              |
  <free group on the generators [ f1, f2, f3 ]>
  !gapprompt@gap>| !gapinput@phi := NqEpimorphismNilpotentQuotient( F, 5 );|
  [ f1, f2, f3 ] -> [ g1, g2, g3 ]
  !gapprompt@gap>| !gapinput@Image( phi, LeftNormedComm( [F.3, F.2, F.1] ) );|
  g12
  !gapprompt@gap>| !gapinput@F := FreeGroup( "a", "b" ); |
  <free group on the generators [ a, b ]>
  !gapprompt@gap>| !gapinput@G := F / [ F.1^2, F.2^2 ];     |
  <fp group on the generators [ a, b ]>
  !gapprompt@gap>| !gapinput@phi := NqEpimorphismNilpotentQuotient( G, 4 );   |
  [ a, b ] -> [ g1, g2 ]
  !gapprompt@gap>| !gapinput@Image( phi, Comm(G.1,G.2) ); |
  g3*g4
  !gapprompt@gap>| !gapinput@F := FreeGroup( "a", "b", "u", "v", "x" );|
  <free group on the generators [ a, b, u, v, x ]>
  !gapprompt@gap>| !gapinput@a := F.1;; b := F.2;; u := F.3;; v := F.4;; x := F.5;;|
  !gapprompt@gap>| !gapinput@G := F / [ x^5, LeftNormedComm( [u,v,v,v] ) ];|
  <fp group of size infinity on the generators [ a, b, u, v, x ]>
  !gapprompt@gap>| !gapinput@phi := NqEpimorphismNilpotentQuotient( G : idgens:=[u,v,x], class:=5 );|
  [ a, b ] -> [ g1, g2 ]
  !gapprompt@gap>| !gapinput@U := Source(phi);                            |
  Group([ a, b ])
  !gapprompt@gap>| !gapinput@ImageElm( phi, LeftNormedComm( [U.1*U.2, U.2^-1,U.2^-1,U.2^-1,] ) );|
  id
  
\end{Verbatim}
 Note that the last epimorphism is a map from the group generated by $a$ and $b$ onto the nilpotent quotient. The identical generators are used only to
formulate the identical relator. They are not generators of the group $G$. Also note that the left-normed commutator above is mapped to the identity as $G$ satisfies the specified identical law. }

 

\subsection{\textcolor{Chapter }{LowerCentralFactors}}
\logpage{[ 3, 1, 4 ]}\nobreak
\hyperdef{L}{X827C2D4F78C982FC}{}
{\noindent\textcolor{FuncColor}{$\triangleright$\ \ \texttt{LowerCentralFactors({\mdseries\slshape ...})\index{LowerCentralFactors@\texttt{LowerCentralFactors}}
\label{LowerCentralFactors}
}\hfill{\scriptsize (function)}}\\


 This function accepts the same arguments and options as \texttt{NilpotentQuotient} (\ref{NilpotentQuotient}) and returns a list containing the abelian invariants of the central factors in
the lower central series of the specified group. 
\begin{Verbatim}[commandchars=!@|,fontsize=\small,frame=single,label=Example]
  !gapprompt@gap>| !gapinput@LowerCentralFactors( FreeGroup(2), 6 );|
  [ [ 0, 0 ], [ 0 ], [ 0, 0 ], [ 0, 0, 0 ], [ 0, 0, 0, 0, 0, 0 ], 
    [ 0, 0, 0, 0, 0, 0, 0, 0, 0 ] ]
\end{Verbatim}
 }

 }

 
\section{\textcolor{Chapter }{Expression Trees}}\label{FunctionsExpTrees}
\logpage{[ 3, 2, 0 ]}
\hyperdef{L}{X861A2C6385F6BCF5}{}
{
 

\subsection{\textcolor{Chapter }{ExpressionTrees}}
\logpage{[ 3, 2, 1 ]}\nobreak
\hyperdef{L}{X7CC7CDDD876BB8EB}{}
{\noindent\textcolor{FuncColor}{$\triangleright$\ \ \texttt{ExpressionTrees({\mdseries\slshape m[, prefix]})\index{ExpressionTrees@\texttt{ExpressionTrees}}
\label{ExpressionTrees}
}\hfill{\scriptsize (function)}}\\
\noindent\textcolor{FuncColor}{$\triangleright$\ \ \texttt{ExpressionTrees({\mdseries\slshape str1, str2, str3, ...})\index{ExpressionTrees@\texttt{ExpressionTrees}}
\label{ExpressionTrees}
}\hfill{\scriptsize (function)}}\\


 The argument \texttt{m} must be a positive integer. The function returns a list with \texttt{m} expression tree symbols named x1, x2,... The optional parameter \texttt{prefix} must be a string and is used instead of \texttt{x} if present. 

 Alternatively, the function can be executed with a list of strings \texttt{str1}, \texttt{str2}, .... It returns a list of symbols with these strings as names. 

 The following operations are defined for expression trees: multiplication,
inversion, exponentiation, forming commutators, forming conjugates. 
\begin{Verbatim}[commandchars=!@|,fontsize=\small,frame=single,label=Example]
  !gapprompt@gap>| !gapinput@t := ExpressionTrees( 3 );                      |
  [ x1, x2, x3 ]
  !gapprompt@gap>| !gapinput@tree := Comm( t[1], t[2] )^3/LeftNormedComm( [t[1],t[2],t[3],t[1]] );|
  Comm( x1, x2 )^3/Comm( x1, x2, x3, x1 )
  !gapprompt@gap>| !gapinput@t := ExpressionTrees( "a", "b", "x" );|
  [ a, b, x ]
  !gapprompt@gap>| !gapinput@tree := Comm( t[1], t[2] )^3/LeftNormedComm( [t[1],t[2],t[3],t[1]] );|
  Comm( a, b )^3/Comm( a, b, x, a )
       
\end{Verbatim}
 }

 

\subsection{\textcolor{Chapter }{EvaluateExpTree}}
\logpage{[ 3, 2, 2 ]}\nobreak
\hyperdef{L}{X879956307B67A136}{}
{\noindent\textcolor{FuncColor}{$\triangleright$\ \ \texttt{EvaluateExpTree({\mdseries\slshape tree, symbols, values})\index{EvaluateExpTree@\texttt{EvaluateExpTree}}
\label{EvaluateExpTree}
}\hfill{\scriptsize (function)}}\\


 The argument \texttt{tree} is an expression tree followed by the list of those symbols \texttt{symbols} from which the expression tree is built up. The argument \texttt{values} is a list containing a constant for each symbol. The function substitutes each
value for the corresponding symbol and computes the resulting value for \texttt{tree}. 
\begin{Verbatim}[commandchars=!@|,fontsize=\small,frame=single,label=Example]
  
  !gapprompt@gap>| !gapinput@F := FreeGroup( 3 );                               |
  <free group on the generators [ f1, f2, f3 ]>
  !gapprompt@gap>| !gapinput@t := ExpressionTrees( "a", "b", "x" );|
  [ a, b, x ]
  !gapprompt@gap>| !gapinput@tree := Comm( t[1], t[2] )^3/LeftNormedComm( [t[1],t[2],t[3],t[1]] );|
  Comm( a, b )^3/Comm( a, b, x, a )
  !gapprompt@gap>| !gapinput@EvaluateExpTree( tree, t, GeneratorsOfGroup(F) );|
  f1^-1*f2^-1*f1*f2*f1^-1*f2^-1*f1*f2*f1^-1*f2^-1*f1*f2*f1^-1*f3^-1*f2^-1*f1^
  -1*f2*f1*f3*f1^-1*f2^-1*f1*f2*f1*f2^-1*f1^-1*f2*f1*f3^-1*f1^-1*f2^-1*f1*f2*f3
  
       
\end{Verbatim}
 }

 }

 
\section{\textcolor{Chapter }{Auxiliary Functions}}\logpage{[ 3, 3, 0 ]}
\hyperdef{L}{X866E18057EF83F65}{}
{
 

\subsection{\textcolor{Chapter }{NqReadOutput}}
\logpage{[ 3, 3, 1 ]}\nobreak
\hyperdef{L}{X855407657CB86F40}{}
{\noindent\textcolor{FuncColor}{$\triangleright$\ \ \texttt{NqReadOutput({\mdseries\slshape stream})\index{NqReadOutput@\texttt{NqReadOutput}}
\label{NqReadOutput}
}\hfill{\scriptsize (function)}}\\


 The only argument \texttt{stream} is an output stream of the ANU NQ. The function reads the stream and returns a
record that has a component for each global variable used in the output of the
ANU NQ, see \texttt{NqGlobalVariables} (\ref{NqGlobalVariables}). }

 

\subsection{\textcolor{Chapter }{NqStringFpGroup}}
\logpage{[ 3, 3, 2 ]}\nobreak
\hyperdef{L}{X8443537679BC81D5}{}
{\noindent\textcolor{FuncColor}{$\triangleright$\ \ \texttt{NqStringFpGroup({\mdseries\slshape fp-group[, idgens]})\index{NqStringFpGroup@\texttt{NqStringFpGroup}}
\label{NqStringFpGroup}
}\hfill{\scriptsize (function)}}\\


 The function takes a finitely presented group \texttt{fp-group} and returns a string in the input format of the ANU NQ. If the list \texttt{idgens} is present, then it must contain generators of the free group underlying the
finitely presented group \texttt{FreeGroupOfFpGroup} (\textbf{Reference: FreeGroupOfFpGroup}). The generators in \texttt{idgens} are treated as identical generators. 
\begin{Verbatim}[commandchars=!@A,fontsize=\small,frame=single,label=Example]
  
  !gapprompt@gap>A !gapinput@F := FreeGroup(2);A
  <free group on the generators [ f1, f2 ]>
  !gapprompt@gap>A !gapinput@G := F / [F.1^2, F.2^2, (F.1*F.2)^4];A
  <fp group on the generators [ f1, f2 ]>
  !gapprompt@gap>A !gapinput@NqStringFpGroup( G );A
  "< x1, x2 |\n    x1^2,\n    x2^2,\n    x1*x2*x1*x2*x1*x2*x1*x2\n>\n"
  !gapprompt@gap>A !gapinput@Print( last );A
  < x1, x2 |
      x1^2,
      x2^2,
      x1*x2*x1*x2*x1*x2*x1*x2
  >
  !gapprompt@gap>A !gapinput@PrintTo( "dihedral", last );A
  !gapprompt@gap>A !gapinput@## The following is equivalent to: A
  !gapprompt@gap>A !gapinput@##     NilpotentQuotient( : input_file := "dihedral" );A
  !gapprompt@gap>A !gapinput@NilpotentQuotient( "dihedral" );A
  Pcp-group with orders [ 2, 2, 2 ]
  !gapprompt@gap>A !gapinput@Exec( "rm dihedral" );A
  !gapprompt@gap>A !gapinput@F := FreeGroup(3);A
  <free group on the generators [ f1, f2, f3 ]>
  !gapprompt@gap>A !gapinput@H := F / [ LeftNormedComm( [F.2,F.1,F.1] ),                               A
  !gapprompt@>A !gapinput@              LeftNormedComm( [F.2,F.1,F.2] ), F.3^7 ];A
  <fp group on the generators [ f1, f2, f3 ]>
  !gapprompt@gap>A !gapinput@str := NqStringFpGroup( H, [F.3] );                                  A
  "< x1, x2; x3 |\n    x1^-1*x2^-1*x1*x2*x1^-1*x2^-1*x1^-1*x2*x1^2,\n    x1^-1*x\
  2^-1*x1*x2^-1*x1^-1*x2*x1*x2,\n    x3^7\n>\n"
  !gapprompt@gap>A !gapinput@NilpotentQuotient( : input_string := str );A
  Pcp-group with orders [ 7, 7, 7 ]
  
       
\end{Verbatim}
 }

 

\subsection{\textcolor{Chapter }{NqStringExpTrees}}
\logpage{[ 3, 3, 3 ]}\nobreak
\hyperdef{L}{X82684F4D79A786F5}{}
{\noindent\textcolor{FuncColor}{$\triangleright$\ \ \texttt{NqStringExpTrees({\mdseries\slshape fp-group[, idgens]})\index{NqStringExpTrees@\texttt{NqStringExpTrees}}
\label{NqStringExpTrees}
}\hfill{\scriptsize (function)}}\\


 The function takes a finitely presented group \texttt{fp-group} given in terms of expression trees and returns a string in the input format of
the ANU NQ. If the list \texttt{idgens} is present, then it must contain a sublist of the generators of the
presentation. The generators in \texttt{idgens} are treated as identical generators. 
\begin{Verbatim}[commandchars=!@A,fontsize=\small,frame=single,label=Example]
  
  !gapprompt@gap>A !gapinput@x := ExpressionTrees( 2 );A
  [ x1, x2 ]
  !gapprompt@gap>A !gapinput@rels := [x[1]^2, x[2]^2, (x[1]*x[2])^5]; A
  [ x1^2, x2^2, (x1*x2)^5 ]
  !gapprompt@gap>A !gapinput@NqStringExpTrees( rec( generators := x, relations := rels ) );A
  "< x1, x2 |\n    x1^2,\n    x2^2,\n    (x1*x2)^5\n>\n"
  !gapprompt@gap>A !gapinput@Print( last );         A
  < x1, x2 |
      x1^2,
      x2^2,
      (x1*x2)^5
  >
  !gapprompt@gap>A !gapinput@x := ExpressionTrees( 3 );A
  [ x1, x2, x3 ]
  !gapprompt@gap>A !gapinput@rels := [LeftNormedComm( [x[2],x[1],x[1]] ),                              A
  !gapprompt@>A !gapinput@            LeftNormedComm( [x[2],x[1],x[2]] ), x[3]^7 ];A
  [ Comm( x2, x1, x1 ), Comm( x2, x1, x2 ), x3^7 ]
  !gapprompt@gap>A !gapinput@NqStringExpTrees( rec( generators := x, relations := rels ) );A
  "< x1, x2, x3 |\n    [ x2, x1, x1 ],\n    [ x2, x1, x2 ],\n    x3^7\n>\n"
  !gapprompt@gap>A !gapinput@Print( last );A
  < x1, x2, x3 |
      [ x2, x1, x1 ],
      [ x2, x1, x2 ],
      x3^7
  >
  
       
\end{Verbatim}
 }

 

\subsection{\textcolor{Chapter }{NqElementaryDivisors}}
\logpage{[ 3, 3, 4 ]}\nobreak
\hyperdef{L}{X7A28800579A2BB35}{}
{\noindent\textcolor{FuncColor}{$\triangleright$\ \ \texttt{NqElementaryDivisors({\mdseries\slshape int-mat})\index{NqElementaryDivisors@\texttt{NqElementaryDivisors}}
\label{NqElementaryDivisors}
}\hfill{\scriptsize (function)}}\\


 The function \texttt{ElementaryDivisorsMat} (\textbf{Reference: ElementaryDivisorsMat}) only returns the non-zero elementary divisors of an integer matrix. This
function computes the elementary divisors of \texttt{int-mat} and adds the appropriate number of zeroes in order to make it easier to
recognize the isomorphism type of the abelian group presented by the integer
matrix. At the same time ones are stripped from the list of elementary
divisors. }

 }

 
\section{\textcolor{Chapter }{Global Variables}}\logpage{[ 3, 4, 0 ]}
\hyperdef{L}{X7D9044767BEB1523}{}
{
 

\subsection{\textcolor{Chapter }{NqRuntime}}
\logpage{[ 3, 4, 1 ]}\nobreak
\hyperdef{L}{X87691A167A83FAF6}{}
{\noindent\textcolor{FuncColor}{$\triangleright$\ \ \texttt{NqRuntime\index{NqRuntime@\texttt{NqRuntime}}
\label{NqRuntime}
}\hfill{\scriptsize (global variable)}}\\


 This variable contains the number of milliseconds of runtime of the last call
of ANU NQ. 
\begin{Verbatim}[commandchars=!@|,fontsize=\small,frame=single,label=Example]
  !gapprompt@gap>| !gapinput@NilpotentEngelQuotient( FreeGroup(2), 5 );|
  Pcp-group with orders [ 0, 0, 0, 0, 0, 0, 0, 0, 0, 0, 0, 0, 0, 0, 3, 0, 10, 
    0, 0, 30, 0, 3, 3, 10, 2, 0, 6, 0, 0, 30, 2, 0, 9, 3, 5, 2, 6, 2, 10, 5, 5, 
    2, 0, 3, 3, 3, 3, 3, 5, 5, 3, 3 ]
  !gapprompt@gap>| !gapinput@NqRuntime;|
  18200
       
\end{Verbatim}
 }

 

\subsection{\textcolor{Chapter }{NqDefaultOptions}}
\logpage{[ 3, 4, 2 ]}\nobreak
\hyperdef{L}{X7DFBFD1580BF024A}{}
{\noindent\textcolor{FuncColor}{$\triangleright$\ \ \texttt{NqDefaultOptions\index{NqDefaultOptions@\texttt{NqDefaultOptions}}
\label{NqDefaultOptions}
}\hfill{\scriptsize (global variable)}}\\


 This variable contains a list of strings which are the standard command line
options passed to the ANU NQ in each call. Modifying this variable can be used
to pass additional options to the ANU NQ. 
\begin{Verbatim}[commandchars=!@|,fontsize=\small,frame=single,label=Example]
  !gapprompt@gap>| !gapinput@NqDefaultOptions;|
  [ "-g", "-p", "-C", "-s" ]
       
\end{Verbatim}
 The option \mbox{\texttt{\mdseries\slshape -g}} causes the ANU NQ to produce output in \textsf{GAP}-format. The option \mbox{\texttt{\mdseries\slshape -p}} prevents the ANU NQ from listing the pc-presentation of the nilpotent quotient
at the end of the calculation. The option \mbox{\texttt{\mdseries\slshape -C}} invokes the combinatorial collector. The option \mbox{\texttt{\mdseries\slshape -s}} is effective only in conjunction with options for computing with Engel
identities and instructs the ANU NQ to use only semigroup words in the
generators as instances of an Engel law. }

 

\subsection{\textcolor{Chapter }{NqGlobalVariables}}
\logpage{[ 3, 4, 3 ]}\nobreak
\hyperdef{L}{X83D1AFCB7EFF4380}{}
{\noindent\textcolor{FuncColor}{$\triangleright$\ \ \texttt{NqGlobalVariables\index{NqGlobalVariables@\texttt{NqGlobalVariables}}
\label{NqGlobalVariables}
}\hfill{\scriptsize (global variable)}}\\


 This variable contains a list of strings with the names of the global
variables that are used in the output stream of the ANU NQ. While the output
stream is read, these global variables are assigned new values. To avoid
overwriting these variables in case they contain values, their contents is
saved before reading the output stream and restored afterwards. }

 }

 
\section{\textcolor{Chapter }{Diagnostic Output}}\label{DiagnosticOutput}
\logpage{[ 3, 5, 0 ]}
\hyperdef{L}{X804DD7CE815D87C9}{}
{
 While the standalone program is running it can be asked to display progress
information. This is done by setting the info class \texttt{InfoNQ} to $1$ via the function \texttt{SetInfoLevel} (\textbf{Reference: SetInfoLevel}). 
\begin{Verbatim}[commandchars=!@|,fontsize=\small,frame=single,label=Example]
  !gapprompt@gap>| !gapinput@NilpotentQuotient(FreeGroup(2),5);|
  Pcp-group with orders [ 0, 0, 0, 0, 0, 0, 0, 0, 0, 0, 0, 0, 0, 0 ]
  !gapprompt@gap>| !gapinput@SetInfoLevel( InfoNQ, 1 );|
  !gapprompt@gap>| !gapinput@NilpotentQuotient(FreeGroup(2),5);|
  #I  Class 1: 2 generators with relative orders  0 0
  #I  Class 2: 1 generators with relative orders: 0
  #I  Class 3: 2 generators with relative orders: 0 0
  #I  Class 4: 3 generators with relative orders: 0 0 0
  #I  Class 5: 6 generators with relative orders: 0 0 0 0 0 0
  Pcp-group with orders [ 0, 0, 0, 0, 0, 0, 0, 0, 0, 0, 0, 0, 0, 0 ]
  !gapprompt@gap>| !gapinput@SetInfoLevel( InfoNQ, 0 );|
\end{Verbatim}
 }

 }

 
\chapter{\textcolor{Chapter }{Examples}}\logpage{[ 4, 0, 0 ]}
\hyperdef{L}{X7A489A5D79DA9E5C}{}
{
 
\section{\textcolor{Chapter }{Right Engel elements}}\logpage{[ 4, 1, 0 ]}
\hyperdef{L}{X8638E6CE7B5955FB}{}
{
 An old problem in the context of Engel elements is the question: Is a right $n$-Engel element left $n$-Engel? It is known that the answer is no. For details about the history of
the problem, see \cite{NewmanNickel94}. In this paper the authors show that for $n>4$ there are nilpotent groups with right $n$-Engel elements no power of which is a left $n$-Engel element. The insight was based on computations with the ANU NQ which we
reproduce here. We also show the cases $5>n$. 
\begin{Verbatim}[commandchars=!@|,fontsize=\small,frame=single,label=Example]
  !gapprompt@gap>| !gapinput@RequirePackage( "nq" );|
  true
  !gapprompt@gap>| !gapinput@##  SetInfoLevel( InfoNQ, 1 );|
  !gapprompt@gap>| !gapinput@##|
  !gapprompt@gap>| !gapinput@##  setup calculation|
  !gapprompt@gap>| !gapinput@##|
  !gapprompt@gap>| !gapinput@et := ExpressionTrees( "a", "b", "x" );|
  [ a, b, x ]
  !gapprompt@gap>| !gapinput@a := et[1];; b := et[2];; x := et[3];;|
  !gapprompt@gap>| !gapinput@|
  !gapprompt@gap>| !gapinput@##|
  !gapprompt@gap>| !gapinput@##  define the group for n = 2,3,4,5|
  !gapprompt@gap>| !gapinput@##|
  !gapprompt@gap>| !gapinput@|
  !gapprompt@gap>| !gapinput@rengel := LeftNormedComm( [a,x,x] );|
  Comm( a, x, x )
  !gapprompt@gap>| !gapinput@G := rec( generators := et, relations := [rengel] );|
  rec( generators := [ a, b, x ], relations := [ Comm( a, x, x ) ] )
  !gapprompt@gap>| !gapinput@## The following is equivalent to:|
  !gapprompt@gap>| !gapinput@##   NilpotentQuotient( : input_string := NqStringExpTrees( G, [x] ) )|
  !gapprompt@gap>| !gapinput@H := NilpotentQuotient( G, [x] );|
  Pcp-group with orders [ 0, 0, 0 ]
  !gapprompt@gap>| !gapinput@LeftNormedComm( [ H.2,H.1,H.1 ] );|
  id
  !gapprompt@gap>| !gapinput@LeftNormedComm( [ H.1,H.2,H.2 ] );|
  id
\end{Verbatim}
 This shows that each right 2-Engel element in a finitely generated nilpotent
group is a left 2-Engel element. Note that the group above is the largest
nilpotent group generated by two elements, one of which is right 2-Engel.
Every nilpotent group generated by an arbitrary element and a right 2-Engel
element is a homomorphic image of the group $H$. 
\begin{Verbatim}[commandchars=!@|,fontsize=\small,frame=single,label=Example]
  !gapprompt@gap>| !gapinput@rengel := LeftNormedComm( [a,x,x,x] );|
  Comm( a, x, x, x )
  !gapprompt@gap>| !gapinput@G := rec( generators := et, relations := [rengel] );|
  rec( generators := [ a, b, x ], relations := [ Comm( a, x, x, x ) ] )
  !gapprompt@gap>| !gapinput@H := NilpotentQuotient( G, [x] );|
  Pcp-group with orders [ 0, 0, 0, 0, 0, 4, 2, 2 ]
  !gapprompt@gap>| !gapinput@LeftNormedComm( [ H.1,H.2,H.2,H.2 ] );|
  id
  !gapprompt@gap>| !gapinput@h := LeftNormedComm( [ H.2,H.1,H.1,H.1 ] );|
  g6^2*g7*g8
  !gapprompt@gap>| !gapinput@Order( h );|
  4
\end{Verbatim}
 The element $h$ has order $4$. In a nilpotent group without $2$-torsion a right 3-Engel element is left 3-Engel. 
\begin{Verbatim}[commandchars=!@|,fontsize=\small,frame=single,label=Example]
  !gapprompt@gap>| !gapinput@rengel := LeftNormedComm( [a,x,x,x,x] );|
  Comm( a, x, x, x, x )
  !gapprompt@gap>| !gapinput@G := rec( generators := et, relations := [rengel] );|
  rec( generators := [ a, b, x ], relations := [ Comm( a, x, x, x, x ) ] )
  !gapprompt@gap>| !gapinput@H := NilpotentQuotient( G, [x] );|
  Pcp-group with orders [ 0, 0, 0, 0, 0, 0, 0, 0, 2, 0, 12, 0, 5, 10, 2, 0, 30, 
    5, 2, 5, 5, 5, 5 ]
  !gapprompt@gap>| !gapinput@LeftNormedComm( [ H.1,H.2,H.2,H.2,H.2 ] );|
  id
  !gapprompt@gap>| !gapinput@h := LeftNormedComm( [ H.2,H.1,H.1,H.1,H.1 ] );|
  g9*g10^2*g11^10*g12^5*g13^2*g14^8*g15*g16^6*g17^10*g18*g20^4*g21^4*g22^2*g23^2
  !gapprompt@gap>| !gapinput@Order( h );|
  60
\end{Verbatim}
 The previous calculation shows that in a nilpotent group without $2,3,5$-torsion a right 4-Engel element is left 4-Engel. 
\begin{Verbatim}[commandchars=!@|,fontsize=\small,frame=single,label=Example]
  !gapprompt@gap>| !gapinput@rengel := LeftNormedComm( [a,x,x,x,x,x] );|
  Comm( a, x, x, x, x, x )
  !gapprompt@gap>| !gapinput@G := rec( generators := et, relations := [rengel] );|
  rec( generators := [ a, b, x ], relations := [ Comm( a, x, x, x, x, x ) ] )
  !gapprompt@gap>| !gapinput@H := NilpotentQuotient( G, [x], 9 );|
  Pcp-group with orders [ 0, 0, 0, 0, 0, 0, 0, 0, 0, 0, 0, 0, 0, 0, 6, 0, 30, 
    0, 0, 30, 0, 3, 6, 0, 0, 10, 30, 0, 0, 0, 0, 30, 30, 0, 0, 3, 6, 5, 2, 0, 
    2, 408, 2, 0, 0, 0, 10, 10, 30, 10, 0, 0, 0, 3, 3, 3, 2, 204, 6, 6, 0, 10, 
    10, 10, 2, 2, 2, 0, 300, 0, 0, 18 ]
  !gapprompt@gap>| !gapinput@LeftNormedComm( [ H.1,H.2,H.2,H.2,H.2,H.2 ] );|
  id
  !gapprompt@gap>| !gapinput@h := LeftNormedComm( [ H.2,H.1,H.1,H.1,H.1,H.1 ] );;|
  !gapprompt@gap>| !gapinput@Order( h );|
  infinity
\end{Verbatim}
 Finally, we see that in a torsion-free group a right 5-Engel element need not
be a left 5-Engel element. }

 }

 
\chapter{\textcolor{Chapter }{Installation of the Package}}\logpage{[ 5, 0, 0 ]}
\hyperdef{L}{X79E1ED167D631DCC}{}
{
  Installation of the package requires the GNU multiple precision library (GNU
MP \cite{GNUMP}). If this library is not available on your system, it has to be installed
first. A copy of GNU MP can be obtained from \href{http://gmplib.org/} {\texttt{http://gmplib.org/}}. Installation of the ANU NQ is done in two steps. First the configure script
is run: 
\begin{Verbatim}[fontsize=\small,frame=single,label=Installation]
   ./configure  
\end{Verbatim}
 Among other things it will tell you whether it succeeded in locating the GNU
MP library and the corresponding include files. If it tells you that it could
not find the GNU MP include files or the GNU MP libraries, you have to run
configure with additional parameters. For example, if you have installed GNU
MP yourself in your home directory, you would type something like the
following 
\begin{Verbatim}[fontsize=\small,frame=single,label=Installation]
   ./configure --with-gmp=/home/me/gmp 
\end{Verbatim}
 If configure reports no problems, the next step is to start the compilation: 
\begin{Verbatim}[fontsize=\small,frame=single,label=Installation]
   make 
\end{Verbatim}
 

 A compiled version of the program named `nq' is then placed into the directory
`bin/{\textless}complicated name{\textgreater}'. The {\textless}complicated
name{\textgreater} component encodes the operating system and the compiler
used. This allows you to compile NQ on several architectures sharing the same
files system. 

 If there are any warnings or even fatal error messages during the compilation
process, please send a copy to the address at the end of this document
together with information about your operating system, the compiler you used
and any changes you might have made to the source code. We will have a look at
your problems and try to fix them. 

 After the compilation is finished you can check if the ANU NQ is running
properly on your system. Simply type 
\begin{Verbatim}[fontsize=\small,frame=single,label=Installation]
   make test 
\end{Verbatim}
 This runs some computations and compares their output with the output files in
the directory `examples'. If any errors are reported, please follow the
instruction that are printed out. 

The installation is completed by compiling the manual of the package. This is
done from within \textsf{GAP}. 
\begin{Verbatim}[fontsize=\small,frame=single,label=Installation]
  gap> NqBuildManual();
  Composing XML document . . .
  Parsing XML document . . .
  Checking XML structure . . .
  LaTeX version and calling latex and pdflatex:
      writing LaTeX file, 3 x latex, bibtex and makeindex, 2 x pdflatex, dvips
  Text version . . .
  #I  first run, collecting cross references, index, toc, bib and so on . . .
  #I  table of contents complete.
  #I  producing the index . . .
  #I  reading bibliography data files . . . 
  #I  writing bibliography . . .
  #I  second run through document . . .
  Writing manual.six file . . .
  And finally the HTML version . . .
  #I  first run, collecting cross references, index, toc, bib and so on . . .
  #I  table of contents complete.
  #I  producing the index . . .
  #I  reading bibliography data files . . . 
  #I  writing bibliography . . .
  #I  second run through document . . .
  gap> ?NilpotentQuotient
  Help: several entries match this topic - type ?2 to get match [2]
  
  [1] ANU NQ: NilpotentQuotient
  [2] ANU NQ: NilpotentQuotient
  [3] Reference: NilpotentQuotientOfFpLieAlgebra
  gap>
    
\end{Verbatim}
 }

 \def\bibname{References\logpage{[ "Bib", 0, 0 ]}
\hyperdef{L}{X7A6F98FD85F02BFE}{}
}

\bibliographystyle{alpha}
\bibliography{nqbib.xml}

\addcontentsline{toc}{chapter}{References}

\def\indexname{Index\logpage{[ "Ind", 0, 0 ]}
\hyperdef{L}{X83A0356F839C696F}{}
}

\cleardoublepage
\phantomsection
\addcontentsline{toc}{chapter}{Index}


\printindex

\newpage
\immediate\write\pagenrlog{["End"], \arabic{page}];}
\immediate\closeout\pagenrlog
\end{document}
