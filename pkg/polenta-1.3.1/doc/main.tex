% generated by GAPDoc2LaTeX from XML source (Frank Luebeck)
\documentclass[a4paper,11pt]{report}

\usepackage{a4wide}
\sloppy
\pagestyle{myheadings}
\usepackage{amssymb}
\usepackage[latin1]{inputenc}
\usepackage{makeidx}
\makeindex
\usepackage{color}
\definecolor{FireBrick}{rgb}{0.5812,0.0074,0.0083}
\definecolor{RoyalBlue}{rgb}{0.0236,0.0894,0.6179}
\definecolor{RoyalGreen}{rgb}{0.0236,0.6179,0.0894}
\definecolor{RoyalRed}{rgb}{0.6179,0.0236,0.0894}
\definecolor{LightBlue}{rgb}{0.8544,0.9511,1.0000}
\definecolor{Black}{rgb}{0.0,0.0,0.0}

\definecolor{linkColor}{rgb}{0.0,0.0,0.554}
\definecolor{citeColor}{rgb}{0.0,0.0,0.554}
\definecolor{fileColor}{rgb}{0.0,0.0,0.554}
\definecolor{urlColor}{rgb}{0.0,0.0,0.554}
\definecolor{promptColor}{rgb}{0.0,0.0,0.589}
\definecolor{brkpromptColor}{rgb}{0.589,0.0,0.0}
\definecolor{gapinputColor}{rgb}{0.589,0.0,0.0}
\definecolor{gapoutputColor}{rgb}{0.0,0.0,0.0}

%%  for a long time these were red and blue by default,
%%  now black, but keep variables to overwrite
\definecolor{FuncColor}{rgb}{0.0,0.0,0.0}
%% strange name because of pdflatex bug:
\definecolor{Chapter }{rgb}{0.0,0.0,0.0}
\definecolor{DarkOlive}{rgb}{0.1047,0.2412,0.0064}


\usepackage{fancyvrb}

\usepackage{mathptmx,helvet}
\usepackage[T1]{fontenc}
\usepackage{textcomp}


\usepackage[
            pdftex=true,
            bookmarks=true,        
            a4paper=true,
            pdftitle={Written with GAPDoc},
            pdfcreator={LaTeX with hyperref package / GAPDoc},
            colorlinks=true,
            backref=page,
            breaklinks=true,
            linkcolor=linkColor,
            citecolor=citeColor,
            filecolor=fileColor,
            urlcolor=urlColor,
            pdfpagemode={UseNone}, 
           ]{hyperref}

\newcommand{\maintitlesize}{\fontsize{50}{55}\selectfont}

% write page numbers to a .pnr log file for online help
\newwrite\pagenrlog
\immediate\openout\pagenrlog =\jobname.pnr
\immediate\write\pagenrlog{PAGENRS := [}
\newcommand{\logpage}[1]{\protect\write\pagenrlog{#1, \thepage,}}
%% were never documented, give conflicts with some additional packages

\newcommand{\GAP}{\textsf{GAP}}

%% nicer description environments, allows long labels
\usepackage{enumitem}
\setdescription{style=nextline}

%% depth of toc
\setcounter{tocdepth}{1}





%% command for ColorPrompt style examples
\newcommand{\gapprompt}[1]{\color{promptColor}{\bfseries #1}}
\newcommand{\gapbrkprompt}[1]{\color{brkpromptColor}{\bfseries #1}}
\newcommand{\gapinput}[1]{\color{gapinputColor}{#1}}


\begin{document}

\logpage{[ 0, 0, 0 ]}
\begin{titlepage}
\mbox{}\vfill

\begin{center}{\maintitlesize \textbf{\textsf{Polenta}\mbox{}}}\\
\vfill

\hypersetup{pdftitle=\textsf{Polenta}}
\markright{\scriptsize \mbox{}\hfill \textsf{Polenta} \hfill\mbox{}}
{\Huge \textbf{Polycyclic presentations for matrix groups\mbox{}}}\\
\vfill

{\Huge Version 1.3.1\mbox{}}\\[1cm]
{1 June 2012\mbox{}}\\[1cm]
\mbox{}\\[2cm]
{\Large \textbf{ Bj{\"o}rn Assmann   \mbox{}}}\\
{\Large \textbf{ Max Horn     \mbox{}}}\\
\hypersetup{pdfauthor= Bj{\"o}rn Assmann   ;  Max Horn     }
\end{center}\vfill

\mbox{}\\
{\mbox{}\\
\small \noindent \textbf{ Bj{\"o}rn Assmann   }\\
  Homepage: \href{http://www.cs.st-andrews.ac.uk/~bjoern/} {\texttt{http://www.cs.st-andrews.ac.uk/\texttt{\symbol{126}}bjoern/}}}\\
{\mbox{}\\
\small \noindent \textbf{ Max Horn     }  Email: \href{mailto:// mhorn@tu-bs.de } {\texttt{ mhorn@tu-bs.de }}\\
  Homepage: \href{http://www.icm.tu-bs.de/~mhorn/} {\texttt{http://www.icm.tu-bs.de/\texttt{\symbol{126}}mhorn/}}\\
  Address: \begin{minipage}[t]{8cm}\noindent
 AG Algebra und Diskrete Mathematik\\
 Institut Computational Mathematics\\
 TU Braunschweig\\
 Pockelsstr. 14\\
 D-38106 Braunschweig\\
 Germany \end{minipage}
}\\
\end{titlepage}

\newpage\setcounter{page}{2}
{\small 
\section*{Copyright}
\logpage{[ 0, 0, 1 ]}
 \index{License} {\copyright} 2003-2007 by Bj{\"o}rn Assmann

 The \textsf{Polenta} package is free software; you can redistribute it and/or modify it under the
terms of the \href{http://www.fsf.org/licenses/gpl.html} {GNU General Public License} as published by the Free Software Foundation; either version 2 of the License,
or (at your option) any later version. \mbox{}}\\[1cm]
{\small 
\section*{Acknowledgements}
\logpage{[ 0, 0, 2 ]}
 We appreciate very much all past and future comments, suggestions and
contributions to this package and its documentation provided by \textsf{GAP} users and developers. \mbox{}}\\[1cm]
\newpage

\def\contentsname{Contents\logpage{[ 0, 0, 3 ]}}

\tableofcontents
\newpage

 
\chapter{\textcolor{Chapter }{Introduction}}\label{Introduction}
\logpage{[ 1, 0, 0 ]}
\hyperdef{L}{X7DFB63A97E67C0A1}{}
{
  \index{Polenta} \index{Polycyclic}  
\section{\textcolor{Chapter }{The package}}\label{The package}
\logpage{[ 1, 1, 0 ]}
\hyperdef{L}{X79DE59997FDCF767}{}
{
  This package provides functions for computation with matrix groups. Let $G$ be a subgroup of $GL(d,R)$ where the ring $R$ is either equal to ${\ensuremath{\mathbb Q}},{\ensuremath{\mathbb Z}}$ or a finite field $\mathbb{F}_q$. Then: 
\begin{itemize}
\item  We can test whether $G$ is solvable. 
\item  We can test whether $G$ is polycyclic. 
\item  If $G$ is polycyclic, then we can determine a polycyclic presentation for $G$. 
\end{itemize}
 A group $G$ which is given by a polycyclic presentation can be largely investigated by
algorithms implemented in the \textsf{GAP}-package \textsf{Polycyclic} \cite{Polycyclic}. For example we can determine if $G$ is torsion-free and calculate the torsion subgroup. Further we can compute the
derived series and the Hirsch length of the group $G$. Also various methods for computations with subgroups, factor groups and
extensions are available. 

 As a by-product, the \textsf{Polenta} package provides some functionality to compute certain module series for
modules of solvable groups. For example, if $G$ is a rational polycyclic matrix group, then we can compute the radical series
of the natural ${\ensuremath{\mathbb Q}}[G]$-module ${\ensuremath{\mathbb Q}}^d$. }

  
\section{\textcolor{Chapter }{Polycyclic groups}}\label{Polycyclic groups}
\logpage{[ 1, 2, 0 ]}
\hyperdef{L}{X86007B0083F60470}{}
{
  A group $G$ is called polycyclic if it has a finite subnormal series with cyclic factors.
It is a well-known fact that every polycyclic group is finitely presented by a
so-called polycyclic presentation (see for example Chapter 9 in \cite{Sims} or Chapter 2 in \cite{Polycyclic} ). In \textsf{GAP}, groups which are defined by polycyclic presentations are called
polycyclically presented groups, abbreviated PcpGroups. 

 The overall idea of the algorithm implemented in this package was first
introduced by Ostheimer in 1996 \cite{Ostheimer}. In 2001 Eick presented a more detailed version \cite{Eick}. This package contains an implementation of Eick's algorithm. A description
of this implementation together with some refinements and extensions can be
found in \cite{AEi05} and \cite{Assmann}. }

 }

 
\chapter{\textcolor{Chapter }{Methods for matrix groups}}\label{Methods for matrix groups}
\logpage{[ 2, 0, 0 ]}
\hyperdef{L}{X829BA50B82FEC109}{}
{
   
\section{\textcolor{Chapter }{Polycyclic presentations of matrix groups}}\label{Polycyclic presentations of matrix groups}
\logpage{[ 2, 1, 0 ]}
\hyperdef{L}{X826C51B3825A2789}{}
{
  Groups defined by polycyclic presentations are called PcpGroups in \textsf{GAP}. We refer to the Polycyclic manual \cite{Polycyclic} for further background. 

 Suppose that a collection $X$ of matrices of $GL(d,R)$ is given, where the ring $R$ is either ${\ensuremath{\mathbb Q}},{\ensuremath{\mathbb Z}}$ or a finite field. Let $G= \langle X \rangle$. If the group $G$ is polycyclic, then the following functions determine a PcpGroup isomorphic to $G$. 

\subsection{\textcolor{Chapter }{PcpGroupByMatGroup}}
\logpage{[ 2, 1, 1 ]}\nobreak
\hyperdef{L}{X7A1BC4437FD92201}{}
{\noindent\textcolor{FuncColor}{$\triangleright$\ \ \texttt{PcpGroupByMatGroup({\mdseries\slshape G})\index{PcpGroupByMatGroup@\texttt{PcpGroupByMatGroup}}
\label{PcpGroupByMatGroup}
}\hfill{\scriptsize (operation)}}\\


 \mbox{\texttt{\mdseries\slshape G}} is a subgroup of $GL(d,R)$ where $R={\ensuremath{\mathbb Q}},{\ensuremath{\mathbb Z}} $ or $\mathbb{F}_q$. If \mbox{\texttt{\mdseries\slshape G}} is polycyclic, then this function determines a PcpGroup isomorphic to \mbox{\texttt{\mdseries\slshape G}}. If \mbox{\texttt{\mdseries\slshape G}} is not polycyclic, then this function returns \texttt{fail}. }

 

\subsection{\textcolor{Chapter }{IsomorphismPcpGroup}}
\logpage{[ 2, 1, 2 ]}\nobreak
\hyperdef{L}{X8771540F7A235763}{}
{\noindent\textcolor{FuncColor}{$\triangleright$\ \ \texttt{IsomorphismPcpGroup({\mdseries\slshape G})\index{IsomorphismPcpGroup@\texttt{IsomorphismPcpGroup}}
\label{IsomorphismPcpGroup}
}\hfill{\scriptsize (method)}}\\


 \mbox{\texttt{\mdseries\slshape G}} is a subgroup of $GL(d,R)$ where $R={\ensuremath{\mathbb Q}},{\ensuremath{\mathbb Z}} $ or $\mathbb{F}_q$. If \mbox{\texttt{\mdseries\slshape G}} is polycyclic, then this function determines an isomorphism onto a PcpGroup.
If \mbox{\texttt{\mdseries\slshape G}} is not polycyclic, then this function returns \texttt{fail}. 

 Note that the method \texttt{IsomorphismPcpGroup}, installed in this package, cannot be applied directly to a group given by
the function \texttt{AlmostCrystallographicGroup}. Please use \texttt{POL{\textunderscore}AlmostCrystallographicGroup} (with the same parameters as \texttt{AlmostCrystallographicGroup}) instead. }

 

\subsection{\textcolor{Chapter }{ImagesRepresentative}}
\logpage{[ 2, 1, 3 ]}\nobreak
\hyperdef{L}{X85ADB89B7C8DD7D0}{}
{\noindent\textcolor{FuncColor}{$\triangleright$\ \ \texttt{ImagesRepresentative({\mdseries\slshape map, elm})\index{ImagesRepresentative@\texttt{ImagesRepresentative}}
\label{ImagesRepresentative}
}\hfill{\scriptsize (method)}}\\
\noindent\textcolor{FuncColor}{$\triangleright$\ \ \texttt{ImageElm({\mdseries\slshape map, elm})\index{ImageElm@\texttt{ImageElm}}
\label{ImageElm}
}\hfill{\scriptsize (method)}}\\
\noindent\textcolor{FuncColor}{$\triangleright$\ \ \texttt{ImagesSet({\mdseries\slshape map, elms})\index{ImagesSet@\texttt{ImagesSet}}
\label{ImagesSet}
}\hfill{\scriptsize (method)}}\\


 Here \mbox{\texttt{\mdseries\slshape map}} is an isomorphism from a polycyclic matrix group \mbox{\texttt{\mdseries\slshape G}} onto a PcpGroup \mbox{\texttt{\mdseries\slshape H}} calculated by \texttt{IsomorphismPcpGroup} (\ref{IsomorphismPcpGroup}). These methods can be used to compute with such an isomorphism. If the input \mbox{\texttt{\mdseries\slshape elm}} is an element of \mbox{\texttt{\mdseries\slshape G}}, then the function \texttt{ImageElm} can be used to compute the image of \mbox{\texttt{\mdseries\slshape elm}} under \mbox{\texttt{\mdseries\slshape map}}. If \mbox{\texttt{\mdseries\slshape elm}} is not contained in \mbox{\texttt{\mdseries\slshape G}} then the function \texttt{ImageElm} returns \texttt{fail}. The input \mbox{\texttt{\mdseries\slshape pcpelm}} is an element of \mbox{\texttt{\mdseries\slshape H}}. }

 

\subsection{\textcolor{Chapter }{IsSolvableGroup}}
\logpage{[ 2, 1, 4 ]}\nobreak
\hyperdef{L}{X809C78D5877D31DF}{}
{\noindent\textcolor{FuncColor}{$\triangleright$\ \ \texttt{IsSolvableGroup({\mdseries\slshape G})\index{IsSolvableGroup@\texttt{IsSolvableGroup}}
\label{IsSolvableGroup}
}\hfill{\scriptsize (method)}}\\


 \mbox{\texttt{\mdseries\slshape G}} is a subgroup of $GL(d,R)$ where $R={\ensuremath{\mathbb Q}},{\ensuremath{\mathbb Z}} $ or $\mathbb{F}_q$. This function tests if \mbox{\texttt{\mdseries\slshape G}} is solvable and returns \texttt{true} or \texttt{false}. }

 

\subsection{\textcolor{Chapter }{IsTriangularizableMatGroup}}
\logpage{[ 2, 1, 5 ]}\nobreak
\hyperdef{L}{X7EE01C207C214C1F}{}
{\noindent\textcolor{FuncColor}{$\triangleright$\ \ \texttt{IsTriangularizableMatGroup({\mdseries\slshape G})\index{IsTriangularizableMatGroup@\texttt{IsTriangularizableMatGroup}}
\label{IsTriangularizableMatGroup}
}\hfill{\scriptsize (property)}}\\


 \mbox{\texttt{\mdseries\slshape G}} is a subgroup of $GL(d,{\ensuremath{\mathbb Q}})$. This function tests if \mbox{\texttt{\mdseries\slshape G}} is triangularizable (possibly over a finite field extension) and returns \texttt{true} or \texttt{false}. }

 

\subsection{\textcolor{Chapter }{IsPolycyclicGroup}}
\logpage{[ 2, 1, 6 ]}\nobreak
\hyperdef{L}{X7D7456077D3D1B86}{}
{\noindent\textcolor{FuncColor}{$\triangleright$\ \ \texttt{IsPolycyclicGroup({\mdseries\slshape G})\index{IsPolycyclicGroup@\texttt{IsPolycyclicGroup}}
\label{IsPolycyclicGroup}
}\hfill{\scriptsize (method)}}\\


 \mbox{\texttt{\mdseries\slshape G}} is a subgroup of $GL(d,R)$ where $R={\ensuremath{\mathbb Q}},{\ensuremath{\mathbb Z}} $ or $\mathbb{F}_q$. This function tests if \mbox{\texttt{\mdseries\slshape G}} is polycyclic and returns \texttt{true} or \texttt{false}. }

 }

  
\section{\textcolor{Chapter }{Module series}}\label{Module series}
\logpage{[ 2, 2, 0 ]}
\hyperdef{L}{X80D1E9E07DB87F97}{}
{
  Let $G$ be a finitely generated solvable subgroup of $GL(d,{\ensuremath{\mathbb Q}})$. The vector space ${\ensuremath{\mathbb Q}}^d$ is a module for the algebra ${\ensuremath{\mathbb Q}}[G]$. The following functions provide the possibility to compute certain module
series of ${\ensuremath{\mathbb Q}}^d$. Recall that the radical $Rad_G({\ensuremath{\mathbb Q}}^d)$ is defined to be the intersection of maximal ${\ensuremath{\mathbb Q}}[G]$-submodules of ${\ensuremath{\mathbb Q}}^d$. Also recall that the radical series 
\[ 0=R_n < R_{n-1} < \dots < R_1 < R_0={\ensuremath{\mathbb Q}}^d \]
 is defined by $R_{i+1}:= Rad_G(R_i)$. 

\subsection{\textcolor{Chapter }{RadicalSeriesSolvableMatGroup}}
\logpage{[ 2, 2, 1 ]}\nobreak
\hyperdef{L}{X84472FDC863322BD}{}
{\noindent\textcolor{FuncColor}{$\triangleright$\ \ \texttt{RadicalSeriesSolvableMatGroup({\mdseries\slshape G})\index{RadicalSeriesSolvableMatGroup@\texttt{RadicalSeriesSolvableMatGroup}}
\label{RadicalSeriesSolvableMatGroup}
}\hfill{\scriptsize (operation)}}\\


 This function returns a radical series for the ${\ensuremath{\mathbb Q}}[G]$-module ${\ensuremath{\mathbb Q}}^d$, where \mbox{\texttt{\mdseries\slshape G}} is a solvable subgroup of $GL(d,{\ensuremath{\mathbb Q}})$. 

 A radical series of ${\ensuremath{\mathbb Q}}^d$ can be refined to a homogeneous series. }

 

\subsection{\textcolor{Chapter }{HomogeneousSeriesAbelianMatGroup}}
\logpage{[ 2, 2, 2 ]}\nobreak
\hyperdef{L}{X8524F992828B6A71}{}
{\noindent\textcolor{FuncColor}{$\triangleright$\ \ \texttt{HomogeneousSeriesAbelianMatGroup({\mdseries\slshape G})\index{HomogeneousSeriesAbelianMatGroup@\texttt{HomogeneousSeriesAbelianMatGroup}}
\label{HomogeneousSeriesAbelianMatGroup}
}\hfill{\scriptsize (function)}}\\


 A module is said to be homogeneous if it is the direct sum of pairwise
irreducible isomorphic submodules. A homogeneous series of a module is a
submodule series such that the factors are homogeneous. This function returns
a homogeneous series for the ${\ensuremath{\mathbb Q}}[G]$-module ${\ensuremath{\mathbb Q}}^d$, where \mbox{\texttt{\mdseries\slshape G}} is an abelian subgroup of $GL(d,{\ensuremath{\mathbb Q}})$. }

 

\subsection{\textcolor{Chapter }{HomogeneousSeriesTriangularizableMatGroup}}
\logpage{[ 2, 2, 3 ]}\nobreak
\hyperdef{L}{X87D9F67C7CBB1499}{}
{\noindent\textcolor{FuncColor}{$\triangleright$\ \ \texttt{HomogeneousSeriesTriangularizableMatGroup({\mdseries\slshape G})\index{HomogeneousSeriesTriangularizableMatGroup@\texttt{Homogeneous}\-\texttt{Series}\-\texttt{Triangularizable}\-\texttt{Mat}\-\texttt{Group}}
\label{HomogeneousSeriesTriangularizableMatGroup}
}\hfill{\scriptsize (function)}}\\


 A module is said to be homogeneous if it is the direct sum of pairwise
irreducible isomorphic submodules. A homogeneous series of a module is a
submodule series such that the factors are homogeneous. This function returns
a homogeneous series for the ${\ensuremath{\mathbb Q}}[G]$-module ${\ensuremath{\mathbb Q}}^d$, where \mbox{\texttt{\mdseries\slshape G}} is a triangularizable subgroup of $GL(d,{\ensuremath{\mathbb Q}})$. 

 A homogeneous series can be refined to a composition series. }

 

\subsection{\textcolor{Chapter }{CompositionSeriesAbelianMatGroup}}
\logpage{[ 2, 2, 4 ]}\nobreak
\hyperdef{L}{X86FB6E9B801A37D4}{}
{\noindent\textcolor{FuncColor}{$\triangleright$\ \ \texttt{CompositionSeriesAbelianMatGroup({\mdseries\slshape G})\index{CompositionSeriesAbelianMatGroup@\texttt{CompositionSeriesAbelianMatGroup}}
\label{CompositionSeriesAbelianMatGroup}
}\hfill{\scriptsize (function)}}\\


 A composition series of a module is a submodule series such that the factors
are irreducible. This function returns a composition series for the ${\ensuremath{\mathbb Q}}[G]$-module ${\ensuremath{\mathbb Q}}^d$, where \mbox{\texttt{\mdseries\slshape G}} is an abelian subgroup of $GL(d,{\ensuremath{\mathbb Q}})$. }

 

\subsection{\textcolor{Chapter }{CompositionSeriesTriangularizableMatGroup}}
\logpage{[ 2, 2, 5 ]}\nobreak
\hyperdef{L}{X78DE110C7E2A493C}{}
{\noindent\textcolor{FuncColor}{$\triangleright$\ \ \texttt{CompositionSeriesTriangularizableMatGroup({\mdseries\slshape G})\index{CompositionSeriesTriangularizableMatGroup@\texttt{Composition}\-\texttt{Series}\-\texttt{Triangularizable}\-\texttt{Mat}\-\texttt{Group}}
\label{CompositionSeriesTriangularizableMatGroup}
}\hfill{\scriptsize (function)}}\\


 A composition series of a module is a submodule series such that the factors
are irreducible. This function returns a composition series for the ${\ensuremath{\mathbb Q}}[G]$-module ${\ensuremath{\mathbb Q}}^d$, where \mbox{\texttt{\mdseries\slshape G}} is a triangularizable subgroup of $GL(d,{\ensuremath{\mathbb Q}})$. }

 }

  
\section{\textcolor{Chapter }{Subgroups}}\label{Subgroups}
\logpage{[ 2, 3, 0 ]}
\hyperdef{L}{X7BA181CA81D785BB}{}
{
        

\subsection{\textcolor{Chapter }{SubgroupsUnipotentByAbelianByFinite}}
\logpage{[ 2, 3, 1 ]}\nobreak
\hyperdef{L}{X79273B8581D15356}{}
{\noindent\textcolor{FuncColor}{$\triangleright$\ \ \texttt{SubgroupsUnipotentByAbelianByFinite({\mdseries\slshape G})\index{SubgroupsUnipotentByAbelianByFinite@\texttt{SubgroupsUnipotentByAbelianByFinite}}
\label{SubgroupsUnipotentByAbelianByFinite}
}\hfill{\scriptsize (operation)}}\\


 \mbox{\texttt{\mdseries\slshape G}} is a subgroup of $GL(d,R)$ where $R={\ensuremath{\mathbb Q}}$ or ${\ensuremath{\mathbb Z}}$. If \mbox{\texttt{\mdseries\slshape G}} is polycyclic, then this function returns a record containing two normal
subgroups $T$ and $U$ of $G$. The group $T$ is unipotent-by-abelian (and thus triangularizable) and of finite index in \mbox{\texttt{\mdseries\slshape G}}. The group $U$ is unipotent and is such that $T/U$ is abelian. If \mbox{\texttt{\mdseries\slshape G}} is not polycyclic, then the algorithm returns \texttt{fail}. }

 }

  
\section{\textcolor{Chapter }{Examples}}\label{Examples}
\logpage{[ 2, 4, 0 ]}
\hyperdef{L}{X7A489A5D79DA9E5C}{}
{
  

\subsection{\textcolor{Chapter }{PolExamples}}
\logpage{[ 2, 4, 1 ]}\nobreak
\hyperdef{L}{X7C7C3EFA7E49F932}{}
{\noindent\textcolor{FuncColor}{$\triangleright$\ \ \texttt{PolExamples({\mdseries\slshape l})\index{PolExamples@\texttt{PolExamples}}
\label{PolExamples}
}\hfill{\scriptsize (function)}}\\


 Returns some examples for polycyclic rational matrix groups, where \mbox{\texttt{\mdseries\slshape l}} is an integer between 1 and 24. These can be used to test the functions in
this package. Some of the properties of the examples are summarised in the
following table.  
\begin{Verbatim}[commandchars=!@|,fontsize=\small,frame=single,label=Example]
  PolExamples      number generators      subgroup of      Hirsch length
            1                      3           GL(4,Z)                 6
            2                      2           GL(5,Z)                 6
            3                      2           GL(4,Q)                 4
            4                      2           GL(5,Q)                 6
            5                      9          GL(16,Z)                 3
            6                      6           GL(4,Z)                 3
            7                      6           GL(4,Z)                 3
            8                      7           GL(4,Z)                 3
            9                      5           GL(4,Q)                 3
           10                      4           GL(4,Q)                 3
           11                      5           GL(4,Q)                 3
           12                      5           GL(4,Q)                 3
           13                      5           GL(5,Q)                 4
           14                      6           GL(5,Q)                 4
           15                      6           GL(5,Q)                 4
           16                      5           GL(5,Q)                 4
           17                      5           GL(5,Q)                 4
           18                      5           GL(5,Q)                 4
           19                      5           GL(5,Q)                 4
           20                      7          GL(16,Z)                 3
           21                      5          GL(16,Q)                 3
           22                      4          GL(16,Q)                 3
           23                      5          GL(16,Q)                 3
           24                      5          GL(16,Q)                 3
  
\end{Verbatim}
 }

 }

 }

 
\chapter{\textcolor{Chapter }{An example application}}\label{An example application}
\logpage{[ 3, 0, 0 ]}
\hyperdef{L}{X81CAD2F27B2066C4}{}
{
  In this section we outline three example computations with functions from the
previous chapter.  
\section{\textcolor{Chapter }{Presentation for rational matrix groups}}\label{Presentation for rational matrix groups}
\logpage{[ 3, 1, 0 ]}
\hyperdef{L}{X7DAC33E37B977087}{}
{
  
\begin{Verbatim}[commandchars=!@|,fontsize=\small,frame=single,label=Example]
  !gapprompt@gap>| !gapinput@mats :=|
  [ [ [ 1, 0, -1/2, 0 ], [ 0, 1, 0, 1 ], [ 0, 0, 1, 0 ], [ 0, 0, 0, 1 ] ],
    [ [ 1, 1/2, 0, 0 ], [ 0, 1, 0, 0 ], [ 0, 0, 1, 1 ], [ 0, 0, 0, 1 ] ],
    [ [ 1, 0, 0, 1 ], [ 0, 1, 0, 0 ], [ 0, 0, 1, 0 ], [ 0, 0, 0, 1 ] ],
    [ [ 1, -1/2, -3, 7/6 ], [ 0, 1, -1, 0 ], [ 0, 1, 0, 0 ], [ 0, 0, 0, 1 ] ],
    [ [ -1, 3, 3, 0 ], [ 0, 0, 1, 0 ], [ 0, 1, 0, 0 ], [ 0, 0, 0, 1 ] ] ];
  
  !gapprompt@gap>| !gapinput@G := Group( mats );|
  <matrix group with 5 generators>
  
  # calculate an isomorphism from G to a pcp-group
  !gapprompt@gap>| !gapinput@nat := IsomorphismPcpGroup( G );;|
  
  !gapprompt@gap>| !gapinput@H := Image( nat );|
  Pcp-group with orders [ 2, 2, 3, 5, 5, 5, 0, 0, 0 ]
  
  !gapprompt@gap>| !gapinput@h := GeneratorsOfGroup( H );|
  [ g1, g2, g3, g4, g5, g6, g7, g8, g9]
  
  !gapprompt@gap>| !gapinput@mats2 := List( h, x -> PreImage( nat, x ) );;|
  
  # take a random element of G
  !gapprompt@gap>| !gapinput@exp :=  [ 1, 1, 1, 1, 0, 0, 0, 0, 1 ];;|
  !gapprompt@gap>| !gapinput@g := MappedVector( exp, mats2 );|
  [ [ -1, 17/2, -1, 233/6 ],
    [ 0, 1, 0, -2 ],
    [ 0, 1, -1, 2 ],
    [ 0, 0, 0, 1 ] ]
  
  # map g into the image of nat
  !gapprompt@gap>| !gapinput@i := ImageElm( nat, g );|
  g1*g2*g3*g4*g9
  
  # exponent vector
  !gapprompt@gap>| !gapinput@Exponents( i );|
  [ 1, 1, 1, 1, 0, 0, 0, 0, 1 ]
  
  # compare the preimage with g
  !gapprompt@gap>| !gapinput@PreImagesRepresentative( nat, i );|
  [ [ -1, 17/2, -1, 233/6 ],
    [ 0, 1, 0, -2 ],
    [ 0, 1, -1, 2 ],
    [ 0, 0, 0, 1 ] ]
  
  
  !gapprompt@gap>| !gapinput@last = g;|
  true
  
\end{Verbatim}
 }

  
\section{\textcolor{Chapter }{Modules series}}\label{Modules series}
\logpage{[ 3, 2, 0 ]}
\hyperdef{L}{X79CF643081B3FB26}{}
{
  
\begin{Verbatim}[commandchars=!@|,fontsize=\small,frame=single,label=Example]
  !gapprompt@gap>| !gapinput@gens :=|
  [ [ [ 1746/1405, 524/7025, 418/1405, -77/2810 ],
      [ 815/843, 899/843, -1675/843, 415/281 ],
      [ -3358/4215, -3512/21075, 4631/4215, -629/1405 ],
      [ 258/1405, 792/7025, 1404/1405, 832/1405 ] ],
    [ [ -2389/2810, 3664/21075, 8942/4215, -35851/16860 ],
      [ 395/281, 2498/2529, -5105/5058, 3260/2529 ],
      [ 3539/2810, -13832/63225, -12001/12645, 87053/50580 ],
      [ 5359/1405, -3128/21075, -13984/4215, 40561/8430 ] ] ];
  
  !gapprompt@gap>| !gapinput@H := Group( gens );|
  <matrix group with 2 generators>
  
  !gapprompt@gap>| !gapinput@RadicalSeriesSolvableMatGroup( H );|
  [ [ [ 1, 0, 0, 0 ], [ 0, 1, 0, 0 ], [ 0, 0, 1, 0 ], [ 0, 0, 0, 1 ] ],
    [ [ 1, 0, 0, 79/138 ], [ 0, 1, 0, -275/828 ], [ 0, 0, 1, -197/414 ] ],
    [ [ 1, 0, -3, 2 ], [ 0, 1, 55/4, -55/8 ] ],
    [ [ 1, 4/15, 2/3, 1/6 ] ],
    [  ] ]
\end{Verbatim}
 }

  
\section{\textcolor{Chapter }{Triangularizable subgroups}}\label{Triangularizable subgroups}
\logpage{[ 3, 3, 0 ]}
\hyperdef{L}{X7BA34CD28059D6CD}{}
{
  
\begin{Verbatim}[commandchars=@|C,fontsize=\small,frame=single,label=Example]
  @gapprompt|gap>C @gapinput|G := PolExamples(3);C
  <matrix group with 2 generators>
  
  @gapprompt|gap>C @gapinput|GeneratorsOfGroup( G );C
  [ [ [ 73/10, -35/2, 42/5, 63/2 ],
      [ 27/20, -11/4, 9/5, 27/4 ],
      [ -3/5, 1, -4/5, -9 ],
      [ -11/20, 7/4, -2/5, 1/4 ] ],
    [ [ -42/5, 423/10, 27/5, 479/10 ],
      [ -23/10, 227/20, 13/10, 231/20 ],
      [ 14/5, -63/5, -4/5, -79/5 ],
      [ -1/10, 9/20, 1/10, 37/20 ] ] ]
  
  @gapprompt|gap>C @gapinput|subgroups := SubgroupsUnipotentByAbelianByFinite( G );C
  rec( T := <matrix group with 2 generators>,
    U := <matrix group with 4 generators> )
  
  @gapprompt|gap>C @gapinput|GeneratorsOfGroup( subgroups.T );C
  [ [ [ 73/10, -35/2, 42/5, 63/2 ],
      [ 27/20, -11/4, 9/5, 27/4 ],
      [ -3/5, 1, -4/5, -9 ],
      [ -11/20, 7/4, -2/5, 1/4 ] ],
    [ [ -42/5, 423/10, 27/5, 479/10 ],
      [ -23/10, 227/20, 13/10, 231/20 ],
      [ 14/5, -63/5, -4/5, -79/5 ],
      [ -1/10, 9/20, 1/10, 37/20 ] ] ]
  
  # so G is triangularizable!
\end{Verbatim}
 }

 }

 
\chapter{\textcolor{Chapter }{Installation}}\label{Installation}
\logpage{[ 4, 0, 0 ]}
\hyperdef{L}{X8360C04082558A12}{}
{
  \index{Installation}  
\section{\textcolor{Chapter }{Installing this package}}\label{Installing this package}
\logpage{[ 4, 1, 0 ]}
\hyperdef{L}{X81746D7285808409}{}
{
  The \textsf{Polenta} package is part of the standard distribution of \textsf{GAP} and so normally there should be no need to install it separately. If by any
chance it is not part of your \textsf{GAP} distribution, then the standard method is to unpack the package into the \texttt{pkg} directory of your \textsf{GAP} distribution. This will create a \texttt{polenta} subdirectory. For other non-standard options please see Chapter  (\textbf{Reference: Installing a GAP Package}) of the \textsf{GAP} Reference Manual. 

 Note that the GAP-Packages \textsf{Alnuth} and \textsf{Polycyclic} are needed for this package. Normally they should be contained in your
distribution. If not, they can be obtained at \href{http://www.gap-system.org/Packages/packages.html} {\texttt{http://www.gap-system.org/Packages/packages.html}}.   }

  
\section{\textcolor{Chapter }{Loading the \textsf{Polenta} package}}\label{Loading the Polenta package}
\logpage{[ 4, 2, 0 ]}
\hyperdef{L}{X7B5D69ED82E9E5BD}{}
{
  \index{Loading the \textsf{Polenta} package} If the \textsf{Polenta} package is not already loaded then you have to request it explicitly. This can
be done via the \texttt{LoadPackage} (\textbf{Reference: LoadPackage}) command. }

  
\section{\textcolor{Chapter }{Running the test suite}}\label{Running the test suite}
\logpage{[ 4, 3, 0 ]}
\hyperdef{L}{X796DF52483B61C74}{}
{
  Once the package is installed, it is possible to check the correct
installation by running the test suite of the package. 
\begin{Verbatim}[commandchars=!@|,fontsize=\small,frame=single,label=Example]
      gap> ReadPackage( "Polenta", "tst/testall.g" );
\end{Verbatim}
 For more details on Test Files see Section  (\textbf{Reference: Test Files}) of the \textsf{GAP} Reference Manual. 

 If the test suite runs into an error, even though the packages Polycyclic and
Alnuth and their depdendencies have been correctly installed, then please send
a message to \texttt{mhorn@tu-bs.de} including the error message. }

 }

 
\chapter{\textcolor{Chapter }{Information Messages}}\label{Information Messages}
\logpage{[ 5, 0, 0 ]}
\hyperdef{L}{X85677704855596EB}{}
{
  It is possible to get informations about the status of the computation of the
functions of Chapter \ref{Methods for matrix groups} of this manual.  
\section{\textcolor{Chapter }{Info Class}}\label{Info Class}
\logpage{[ 5, 1, 0 ]}
\hyperdef{L}{X7EBFC26F83EB9F72}{}
{
  

\subsection{\textcolor{Chapter }{InfoPolenta}}
\logpage{[ 5, 1, 1 ]}\nobreak
\hyperdef{L}{X809F2CFB87393CE0}{}
{\noindent\textcolor{FuncColor}{$\triangleright$\ \ \texttt{InfoPolenta\index{InfoPolenta@\texttt{InfoPolenta}}
\label{InfoPolenta}
}\hfill{\scriptsize (info class)}}\\


 is the Info class of the \textsf{Polenta} package (for more details on the Info mechanism see Section  (\textbf{Reference: Info Functions}) of the \textsf{GAP} Reference Manual). With the help of the function \texttt{SetInfoLevel(InfoPolenta,\mbox{\texttt{\mdseries\slshape level}})} you can change the info level of \texttt{InfoPolenta}. 
\begin{itemize}
\item  If \texttt{InfoLevel( InfoPolenta )} is equal to 0 then no information messages are displayed. 
\item  If \texttt{InfoLevel( InfoPolenta )} is equal to 1 then basic informations about the process are provided. For
further background on the displayed informations we refer to \cite{Assmann} (publicly available via the Internet address \href{http://www.icm.tu-bs.de/ag_algebra/software/assmann/diploma.pdf} {\texttt{http://www.icm.tu-bs.de/ag{\textunderscore}algebra/software/assmann/diploma.pdf}}). 
\item  If \texttt{InfoLevel( InfoPolenta )} is equal to 2 then, in addition to the basic information, the generators of
computed subgroups and module series are displayed. 
\end{itemize}
 }

 }

  
\section{\textcolor{Chapter }{Example}}\label{Example}
\logpage{[ 5, 2, 0 ]}
\hyperdef{L}{X85861B017AEEC50B}{}
{
  
\begin{Verbatim}[commandchars=!@|,fontsize=\small,frame=single,label=Example]
  !gapprompt@gap>| !gapinput@SetInfoLevel( InfoPolenta, 1 );|
  
  !gapprompt@gap>| !gapinput@PcpGroupByMatGroup( PolExamples(11) );|
  #I  Determine a constructive polycyclic sequence
      for the input group ...
  #I
  #I  Chosen admissible prime: 3
  #I
  #I  Determine a constructive polycyclic sequence
      for the image under the p-congruence homomorphism ...
  #I  finished.
  #I  Finite image has relative orders [ 3, 2, 3, 3, 3 ].
  #I
  #I  Compute normal subgroup generators for the kernel
      of the p-congruence homomorphism ...
  #I  finished.
  #I
  #I  Compute the radical series ...
  #I  finished.
  #I  The radical series has length 4.
  #I
  #I  Compute the composition series ...
  #I  finished.
  #I  The composition series has length 5.
  #I
  #I  Compute a constructive polycyclic sequence
      for the induced action of the kernel to the composition series ...
  #I  finished.
  #I  This polycyclic sequence has relative orders [  ].
  #I
  #I  Calculate normal subgroup generators for the
      unipotent part ...
  #I  finished.
  #I
  #I  Determine a constructive polycyclic  sequence
      for the unipotent part ...
  #I  finished.
  #I  The unipotent part has relative orders
  #I  [ 0, 0, 0 ].
  #I
  #I  ... computation of a constructive
      polycyclic sequence for the whole group finished.
  #I
  #I  Compute the relations of the polycyclic
      presentation of the group ...
  #I  Compute power relations ...
  #I  ... finished.
  #I  Compute conjugation relations ...
  #I  ... finished.
  #I  Update polycyclic collector ...
  #I  ... finished.
  #I  finished.
  #I
  #I  Construct the polycyclic presented group ...
  #I  finished.
  #I
  Pcp-group with orders [ 3, 2, 3, 3, 3, 0, 0, 0 ]
  
  
  !gapprompt@gap>| !gapinput@SetInfoLevel( InfoPolenta, 2 );|
  
  !gapprompt@gap>| !gapinput@PcpGroupByMatGroup( PolExamples(11) );|
  #I  Determine a constructive polycyclic sequence
      for the input group ...
  #I
  #I  Chosen admissible prime: 3
  #I
  #I  Determine a constructive polycyclic sequence
      for the image under the p-congruence homomorphism ...
  #I  finished.
  #I  Finite image has relative orders [ 3, 2, 3, 3, 3 ].
  #I
  #I  Compute normal subgroup generators for the kernel
      of the p-congruence homomorphism ...
  #I  finished.
  #I  The normal subgroup generators are
  #I  [ [ [ 1, -3/2, 0, 0 ], [ 0, 1, 0, 0 ], [ 0, 0, 1, 3 ], [ 0, 0, 0, 1 ] ],
    [ [ 1, 0, 0, 24 ], [ 0, 1, 0, 0 ], [ 0, 0, 1, 0 ], [ 0, 0, 0, 1 ] ],
    [ [ 1, 3, 3, 15 ], [ 0, 1, 0, 6 ], [ 0, 0, 1, -6 ], [ 0, 0, 0, 1 ] ],
    [ [ 1, 3, 3, 9 ], [ 0, 1, 0, 6 ], [ 0, 0, 1, -6 ], [ 0, 0, 0, 1 ] ],
    [ [ 1, 3/2, 3/2, 3/2 ], [ 0, 1, 0, 3 ], [ 0, 0, 1, -3 ], [ 0, 0, 0, 1 ] ],
    [ [ 1, -3/2, 9/2, -69/2 ], [ 0, 1, 0, 9 ], [ 0, 0, 1, 3 ], [ 0, 0, 0, 1 ] ]
      , [ [ 1, 0, 0, -24 ], [ 0, 1, 0, 0 ], [ 0, 0, 1, 0 ], [ 0, 0, 0, 1 ] ],
    [ [ 1, -3, -3, -9 ], [ 0, 1, 0, -6 ], [ 0, 0, 1, 6 ], [ 0, 0, 0, 1 ] ],
    [ [ 1, -3, -3, -15 ], [ 0, 1, 0, -6 ], [ 0, 0, 1, 6 ], [ 0, 0, 0, 1 ] ],
    [ [ 1, -3, 0, 9 ], [ 0, 1, 0, 0 ], [ 0, 0, 1, 6 ], [ 0, 0, 0, 1 ] ],
    [ [ 1, -3, -3, -9 ], [ 0, 1, 0, -6 ], [ 0, 0, 1, 6 ], [ 0, 0, 0, 1 ] ],
    [ [ 1, -3, 0, 9 ], [ 0, 1, 0, 0 ], [ 0, 0, 1, 6 ], [ 0, 0, 0, 1 ] ],
    [ [ 1, -3/2, -3/2, -9/2 ], [ 0, 1, 0, -3 ], [ 0, 0, 1, 3 ], [ 0, 0, 0, 1 ]
       ],
    [ [ 1, -3, -3, -12 ], [ 0, 1, 0, -6 ], [ 0, 0, 1, 6 ], [ 0, 0, 0, 1 ] ],
    [ [ 1, 3, -3/2, -21 ], [ 0, 1, 0, -3 ], [ 0, 0, 1, -6 ], [ 0, 0, 0, 1 ] ],
    [ [ 1, 3/2, 3/2, 9/2 ], [ 0, 1, 0, 3 ], [ 0, 0, 1, -3 ], [ 0, 0, 0, 1 ] ] ]
  #I
  #I  Compute the radical series ...
  #I  finished.
  #I  The radical series has length 4.
  #I  The radical series is
  #I  [ [ [ 1, 0, 0, 0 ], [ 0, 1, 0, 0 ], [ 0, 0, 1, 0 ], [ 0, 0, 0, 1 ] ],
    [ [ 0, 1, 0, 0 ], [ 0, 0, 1, 0 ], [ 0, 0, 0, 1 ] ], [ [ 0, 0, 0, 1 ] ],
    [  ] ]
  #I
  #I  Compute the composition series ...
  #I  finished.
  #I  The composition series has length 5.
  #I  The composition series is
  #I  [ [ [ 1, 0, 0, 0 ], [ 0, 1, 0, 0 ], [ 0, 0, 1, 0 ], [ 0, 0, 0, 1 ] ],
    [ [ 0, 1, 0, 0 ], [ 0, 0, 1, 0 ], [ 0, 0, 0, 1 ] ],
    [ [ 0, 0, 1, 0 ], [ 0, 0, 0, 1 ] ], [ [ 0, 0, 0, 1 ] ], [  ] ]
  #I
  #I  Compute a constructive polycyclic sequence
      for the induced action of the kernel to the composition series ...
  #I  finished.
  #I  This polycyclic sequence has relative orders [  ].
  #I
  #I  Calculate normal subgroup generators for the
      unipotent part ...
  #I  finished.
  #I  The normal subgroup generators for the unipotent part are
  #I  [ [ [ 1, -3/2, 0, 0 ], [ 0, 1, 0, 0 ], [ 0, 0, 1, 3 ], [ 0, 0, 0, 1 ] ],
    [ [ 1, 0, 0, 24 ], [ 0, 1, 0, 0 ], [ 0, 0, 1, 0 ], [ 0, 0, 0, 1 ] ],
    [ [ 1, 3, 3, 15 ], [ 0, 1, 0, 6 ], [ 0, 0, 1, -6 ], [ 0, 0, 0, 1 ] ],
    [ [ 1, 3, 3, 9 ], [ 0, 1, 0, 6 ], [ 0, 0, 1, -6 ], [ 0, 0, 0, 1 ] ],
    [ [ 1, 3/2, 3/2, 3/2 ], [ 0, 1, 0, 3 ], [ 0, 0, 1, -3 ], [ 0, 0, 0, 1 ] ],
    [ [ 1, -3/2, 9/2, -69/2 ], [ 0, 1, 0, 9 ], [ 0, 0, 1, 3 ], [ 0, 0, 0, 1 ] ]
      , [ [ 1, 0, 0, -24 ], [ 0, 1, 0, 0 ], [ 0, 0, 1, 0 ], [ 0, 0, 0, 1 ] ],
    [ [ 1, -3, -3, -9 ], [ 0, 1, 0, -6 ], [ 0, 0, 1, 6 ], [ 0, 0, 0, 1 ] ],
    [ [ 1, -3, -3, -15 ], [ 0, 1, 0, -6 ], [ 0, 0, 1, 6 ], [ 0, 0, 0, 1 ] ],
    [ [ 1, -3, 0, 9 ], [ 0, 1, 0, 0 ], [ 0, 0, 1, 6 ], [ 0, 0, 0, 1 ] ],
    [ [ 1, -3, -3, -9 ], [ 0, 1, 0, -6 ], [ 0, 0, 1, 6 ], [ 0, 0, 0, 1 ] ],
    [ [ 1, -3, 0, 9 ], [ 0, 1, 0, 0 ], [ 0, 0, 1, 6 ], [ 0, 0, 0, 1 ] ],
    [ [ 1, -3/2, -3/2, -9/2 ], [ 0, 1, 0, -3 ], [ 0, 0, 1, 3 ], [ 0, 0, 0, 1 ]
       ],
    [ [ 1, -3, -3, -12 ], [ 0, 1, 0, -6 ], [ 0, 0, 1, 6 ], [ 0, 0, 0, 1 ] ],
    [ [ 1, 3, -3/2, -21 ], [ 0, 1, 0, -3 ], [ 0, 0, 1, -6 ], [ 0, 0, 0, 1 ] ],
    [ [ 1, 3/2, 3/2, 9/2 ], [ 0, 1, 0, 3 ], [ 0, 0, 1, -3 ], [ 0, 0, 0, 1 ] ] ]
  #I
  #I  Determine a constructive polycyclic  sequence
      for the unipotent part ...
  #I  finished.
  #I  The unipotent part has relative orders
  #I  [ 0, 0, 0 ].
  #I
  #I  ... computation of a constructive
      polycyclic sequence for the whole group finished.
  #I
  #I  Compute the relations of the polycyclic
      presentation of the group ...
  #I  Compute power relations ...
  .....
  #I  ... finished.
  #I  Compute conjugation relations ...
  ..............................................
  #I  ... finished.
  #I  Update polycyclic collector ...
  #I  ... finished.
  #I  finished.
  #I
  #I  Construct the polycyclic presented group ...
  #I  finished.
  #I
  Pcp-group with orders [ 3, 2, 3, 3, 3, 0, 0, 0 ]
\end{Verbatim}
 }

 }

 \def\bibname{References\logpage{[ "Bib", 0, 0 ]}
\hyperdef{L}{X7A6F98FD85F02BFE}{}
}

\bibliographystyle{alpha}
\bibliography{polentabib.xml}

\addcontentsline{toc}{chapter}{References}

\def\indexname{Index\logpage{[ "Ind", 0, 0 ]}
\hyperdef{L}{X83A0356F839C696F}{}
}

\cleardoublepage
\phantomsection
\addcontentsline{toc}{chapter}{Index}


\printindex

\newpage
\immediate\write\pagenrlog{["End"], \arabic{page}];}
\immediate\closeout\pagenrlog
\end{document}
