\newcommand{\QPAIntroPartNumber}{1}
\documentclass[usenames,dvipsnames]{beamer}

%\usetheme{ntnu}
\usetheme{Warsaw}

\usepackage{times}
\usepackage[T1]{fontenc}
\usepackage[all]{xy}
\usepackage{textpos}
\usepackage{tikz}

\newcommand{\defn}[1]{\textit{#1}}
\newcommand{\Q}{\mathbb{Q}}
\newcommand{\equivalence}{\simeq}
\newcommand{\VV}[2]{\begin{pmatrix} #1 & #2 \end{pmatrix}}
\newcommand{\vv}[2]{\left( \begin{smallmatrix} #1 & #2 \end{smallmatrix} \right)}
\newcommand{\into}{\hookrightarrow}
\newcommand{\iso}{\cong}
\newcommand{\dsum}{\oplus}
\DeclareMathOperator{\fmod}{mod}
\DeclareMathOperator{\Rep}{Rep}
\DeclareMathOperator{\Hom}{Hom}
\DeclareMathOperator{\coker}{coker}
\DeclareMathOperator{\im}{im}
\DeclareMathOperator{\rad}{rad}
\DeclareMathOperator{\soc}{soc}
\DeclareMathOperator{\Top}{top}

\title[Introduction to QPA, part \QPAIntroPartNumber]
      {Introduction to QPA}
\subtitle{Part \QPAIntroPartNumber}

\author{\O{}ystein~Skarts\ae{}terhagen \and \O{}yvind~Solberg}
\institute{
Department of Mathematical Sciences\\
Norwegian University of Science and Technology}

\date{Third GAP Days}

% If you have a file called "university-logo-filename.xxx", where xxx
% is a graphic format that can be processed by latex or pdflatex,
% resp., then you can add a logo as follows:

% \pgfdeclareimage[height=0.5cm]{university-logo}{university-logo-filename}
% \logo{\pgfuseimage{university-logo}}



% Delete this, if you do not want the table of contents to pop up at
% the beginning of each subsection:
% \AtBeginSubsection[]
% {
%   \begin{frame}<beamer>{Outline}
%     \tableofcontents[currentsection,currentsubsection]
%   \end{frame}
% }


\begin{document}

\begin{frame}
  \titlepage
\end{frame}

\begin{frame}{Outline}
  \tableofcontents
  % You might wish to add the option [pausesections]
\end{frame}

\section{About QPA}

\subsection{What is QPA?}

\begin{frame}{What is QPA?}
\begin{itemize}
\item QPA = Quivers and Path Algebras
\item GAP package for computations with quotients of path algebras
and their modules
\end{itemize}
\end{frame}

\subsection{Obtaining QPA}

\begin{frame}[fragile]{Obtaining QPA}
\begin{itemize}
\item QPA is distributed with GAP (from version~4.7.8)
\item Can also clone git repository from
\begin{center}\texttt{https://github.com/gap-system/qpa}\end{center}
to follow QPA development
\item Loading QPA in GAP:
\begin{verbatim}
gap> LoadPackage("qpa");
\end{verbatim}
\end{itemize}
\end{frame}


\section{Basic structures}

\begin{frame}{}
\huge Basic structures
\Large
\begin{itemize}
\item Quivers
\item Path algebras (modulo relations)
\item Modules (representations)\\and homomorphisms
\end{itemize}
\end{frame}

\subsection{Quivers}

\begin{frame}{Quivers}
\[
\xymatrix{1\ar[r]^a & 2\ar[r]^b & 3} \qquad\quad
\xymatrix{1\ar@<.5ex>[r]^a\ar@<-.5ex>[r]_b & 2} \qquad\quad
\xymatrix{1\ar@(ul,ur)^a\ar[rr]^b & & 2\ar[dl]^c \\ & 3\ar[ul]^d & }
\]
\pause \defn{Quiver}: oriented graph (loops and multiple edges allowed)
\end{frame}

\begin{frame}{Paths}

{\Large
\[
Q \colon \xymatrix{1\ar[r]^a & 2\ar[r]^b & 3}
\]
}
Paths in $Q$:\pause
\begin{itemize}
\item Length $0$: $e_1, e_2, e_3$ (vertices/trivial paths)
\pause
\item Length $1$: $a, b$ (arrows)
\pause
\item Length $2$: $ab$ (concatenation of $a$ and $b$)
\end{itemize}
\end{frame}

\begin{frame}{Paths}

{\Large
\[
Q \colon \xymatrix{1\ar@<.5ex>[r]^a\ar@<-.5ex>[r]_b & 2}
\]
}
Paths in $Q$:\pause
\begin{itemize}
\item Length $0$: $e_1, e_2$
\pause
\item Length $1$: $a, b$
\end{itemize}
\end{frame}

\begin{frame}{Paths}

{\Large
\[
Q \colon \xymatrix{1\ar@(ul,ur)^a\ar[rr]^b & & 2\ar[dl]^c \\ & 3\ar[ul]^d & }
\]
}
Paths in $Q$:\pause
\begin{itemize}
\item Length $0$: $e_1, e_2, e_3$
\pause
\item Length $1$: $a, b, c, d$
\pause
\item Length $2$: $a^2, ab, bc, cd, da, db$
\pause
\item Length $3$: $a^3, a^2b, abc, bcd, cda, cdb, da^2, dab, dbc$
\pause
\item \ldots
\end{itemize}
\end{frame}


\begin{frame}[fragile]{Constructing a quiver in QPA}

{\Large
\[
Q \colon \xymatrix{1\ar[r]^a & 2\ar[r]^b & 3}
\]
}
\begin{verbatim}
gap> Q := Quiver(3, [[1,2,"a"],[2,3,"b"]]); 
<quiver with 3 vertices and 2 arrows>
\end{verbatim}
\end{frame}

% show:
% VerticesOfQuiver
% ArrowsOfQuiver
% [show multiplication of arrows]


\subsection{Path algebras}

\begin{frame}{Path algebras}
% Path algebra

\begin{itemize}
\item Given quiver $Q$ and field $k$
\item Define \defn{path algebra} $kQ$
\item Basis: paths in $Q$
\item Multiplication: concatenation of paths
\end{itemize}
\end{frame}

\begin{frame}{Path algebras}{}
\[
\left.
\begin{aligned}
Q &\colon \xymatrix{1\ar[r]^a & 2\ar[r]^b & 3} \\
k &\ \text{a field}
\end{aligned}
\pause
\right\}
\rightsquigarrow
\text{path algebra $kQ$}
\]
\begin{columns}
\begin{column}{0.5\textwidth}
\begin{itemize}
\pause
\item Basis: $\{ e_1, e_2, e_3, a, b, ab \}$\\
\pause
\item Multiplication:
\begin{align*}
e_1 \cdot e_1 &= e_1 \\
e_1 \cdot e_2 &= 0 \\
e_1 \cdot a &= a \\
e_2 \cdot a &= 0 \\
a \cdot b &= ab &
\end{align*}
\pause
\end{itemize}
\end{column}
\begin{column}{0.5\textwidth}
\begin{tabular}{c|cccccc}
$\cdot$ & $e_1$ & $e_2$ & $e_3$ & $a$   & $b$   & $ab$ \\
\hline
$e_1$   & $e_1$ & $0$   & $0$   & $a$   & $0$   & $ab$ \\
$e_2$   & $0$   & $e_2$ & $0$   & $0$   & $b$   & $0$  \\
$e_3$   & $0$   & $0$   & $e_3$ & $0$   & $0$   & $0$  \\
$a$     & $0$   & $a$   & $0$   & $0$   & $ab$  & $0$  \\
$b$     & $0$   & $0$   & $b$   & $0$   & $0$   & $0$  \\
$ab$    & $0$   & $0$   & $ab$  & $0$   & $0$   & $0$
\end{tabular}
\end{column}
\end{columns}
\end{frame}

\begin{frame}[fragile]{Constructing a path algebra in QPA}{}
{\Large
\[
Q \colon \xymatrix{1\ar[r]^a & 2\ar[r]^b & 3}
\]
}
\begin{verbatim}
gap> Q := Quiver(3, [[1,2,"a"],[2,3,"b"]]); 
<quiver with 3 vertices and 2 arrows>
gap> kQ := PathAlgebra(Rationals, Q);
<Rationals[<quiver with 3 vertices and 2 arrows>]>
\end{verbatim}

% [show in QPA]

% AssignGeneratorVariables

% [show in QPA]
\end{frame}

\begin{frame}{Relations}
\begin{itemize}
\item \defn{Relation}: linear combination of paths with common source and target
\end{itemize}
\[
Q \colon
\vcenter{\xymatrix{
1\ar[r]^a\ar[d]^c & 2\ar[d]^b\\
3\ar[r]^d & 4
}}
\qquad
\sigma = \underbrace{ab - 2cd}_{
\footnotesize
\begin{array}{l}
\text{source: $1$}\\
\text{target: $4$}
\end{array}
} \in kQ
\]
% \begin{overprint}
% \onslide<1>
% \[
% Q \colon
% \vcenter{\xymatrix{
% 1\ar[r]^a\ar[d]^c & 2\ar[d]^b\\
% 3\ar[r]^d & 4
% }}
% \qquad
% \sigma = ab - 2cd \in kQ
% \]
% \onslide<2->
% \[
% Q \colon
% \vcenter{\xymatrix{
% 1\ar[r]^a\ar[d]^c & 2\ar[d]^b\\
% 3\ar[r]^d & 4
% }}
% \qquad
% \sigma = \underbrace{ab - 2cd}_{
% \footnotesize
% \begin{array}{l}
% \text{source: $e_1$}\\
% \text{target: $e_4$}
% \end{array}
% } \in kQ
% \]
% \end{overprint}
\end{frame}

\begin{frame}{Quotient of path algebra modulo relations}
\begin{itemize}
\item Given path algebra $kQ$ and a set $\rho \subseteq kQ$ of relations.
\item Can create quotient algebra $kQ / \langle \rho \rangle$.
\end{itemize}
\[
Q \colon
\vcenter{\xymatrix{
1\ar[r]^a\ar[d]^c & 2\ar[d]^b\\
3\ar[r]^d & 4
}}
\qquad
A = kQ/\langle ab - 2cd \rangle
\]
\end{frame}

\begin{frame}[fragile]{Quotients in QPA}
\[
Q \colon
\vcenter{\xymatrix{
1\ar[r]^a\ar[d]^c & 2\ar[d]^b\\
3\ar[r]^d & 4
}}
\qquad
A = kQ/\langle ab - 2cd \rangle
\]
\begin{verbatim}
gap> Q := Quiver(4,[[1,2,"a"],[2,4,"b"],
                    [1,3,"c"],[3,4,"d"]]);;
gap> kQ := PathAlgebra(Rationals,Q);;
gap> A := kQ/[kQ.a*kQ.b - 2*kQ.c*kQ.d];;
gap> A.a * A.b;
[(1)*a*b]
gap> A.c * A.d;
[(1/2)*a*b]
\end{verbatim}
% [show]
\end{frame}


\begin{frame}{Admissible ideals}
\begin{itemize}
\item $J \subseteq kQ$ ideal generated by the arrows
\item Ideal $I \subseteq kQ$ \defn{admissible} if $J^t \subseteq I
\subseteq J^2$ for some $t \ge 2$
\item If $I$ admissible, then $kQ/I$ finite-dimensional
%\item Only care about algebras $kQ/I$ with $I$ admissible
\end{itemize}
\end{frame}


\begin{frame}{Admissible ideals -- what does it mean?}{}
{\huge
\[
J^t \subseteq I \subseteq J^2
\]
}
\pause
\vspace{-1em}
%\begin{textblock}{5}(0,-.5)
\begin{columns}
\begin{column}{.5\textwidth}
\begin{tikzpicture}
\node [below left] at (0,0) {long paths must die};
\draw [->] (0,0) -- (1,1);
\end{tikzpicture}
\\
\color{OliveGreen}
\[
Q\colon
\vcenter{\xymatrix@R=1em@C=.7em{
& 2 \ar[dr]^b \\
1 \ar[ur]^a &&
3 \ar[ll]^c
}}
\quad
\rho = \{ abcab \}
\]
%\end{textblock}
\end{column}
\begin{column}{.5\textwidth}
\pause
%\begin{textblock}{5}(7.3,-.5)
\begin{tikzpicture}
\node [below right] at (0,0) {no single arrows in relations};
\draw [->] (0,0) -- (-1,1);
\end{tikzpicture}
\\
\color{red}
\[
Q\colon
\vcenter{\xymatrix@R=1em@C=.7em{
& 2 \ar[dr]^b \\
1 \ar[ur]^a \ar[rr]^c &&
3
}}
\quad
\rho = \{ ab - c \}
\]
%\end{textblock}
\end{column}
\end{columns}
\end{frame}


\begin{frame}{Admissible ideals -- why?}
Algebras of the form $kQ/I$ with $I$ admissible \ldots
\begin{itemize}
\item \ldots are ``almost all'' finite-dimensional algebras, and
\item \ldots have a nice theory.
\end{itemize}
\end{frame}


\begin{frame}[fragile]{Checking ideals for admissibility in QPA}
\begin{verbatim}
gap> Q := Quiver(4,[[1,2,"a"],[2,4,"b"],
                    [1,3,"c"],[3,4,"d"]]);;
gap> kQ := PathAlgebra(Rationals,Q);;
gap> I := Ideal(kQ, [kQ.a*kQ.b - 2*kQ.c*kQ.d]);;
gap> IsAdmissibleIdeal(I);
true
\end{verbatim}
\end{frame}


\begin{frame}{Algebras: summary}
\begin{columns}
\begin{column}{.4\textwidth}
Take ...
\begin{itemize}
\onslide<2->
\item a \textcolor{BrickRed}{QUIVER $Q$},
\onslide<3->
\item a \textcolor{OliveGreen}{FIELD $k$},
\onslide<4->
\item a set $\textcolor{blue}{\rho} \subseteq kQ$ of \textcolor{blue}{RELATIONS}
\onslide<5->
      generating an \textcolor{violet}{ADMISSIBLE} ideal
\end{itemize}
\end{column}
\begin{column}{.6\textwidth}
\onslide<2->
\color{BrickRed}
\[
Q \colon
\vcenter{\xymatrix{
1 \ar[r]^a \ar[d]_c &
2 \ar[d]^b \\
3 \ar[r]_d &
4 \ar[ul]_x
}}
\onslide<3->
\color{OliveGreen}
\qquad
k = \Q
\]
\onslide<4->
\color{blue}
\[
\rho = \{ ab - cd, dxa \}
\]
\onslide<5->
\color{violet}
\[
J^t \subseteq \langle \rho \rangle \subseteq J^2
\]
\end{column}
\end{columns}
\vspace{1em}
\onslide<6->
{\huge
\[
\text{\normalsize Let ...}\quad
A =
\textcolor{OliveGreen}{k}
\textcolor{BrickRed}{Q}
  /
\langle \textcolor{blue}{\rho} \rangle
\quad\quad
\]
}
\end{frame}


\subsection{Modules}

\begin{frame}{Modules and representations}{}
{\huge
\[
\fmod kQ \equivalence \Rep_k Q
\]
}
\pause
\begin{textblock}{5}(0,-.5)
\begin{tikzpicture}
\node [below] at (0,0) {finitely generated $kQ$-modules};
\draw [->] (0,0) -- (1,1);
\end{tikzpicture}
\end{textblock}
\pause
\begin{textblock}{5}(7.3,-.5)
\begin{tikzpicture}
\node [below] at (0,0) {representations of $Q$ over $k$};
\draw [->] (0,0) -- (-1,1);
\end{tikzpicture}
\end{textblock}
\end{frame}

\begin{frame}{Representations}
Given
\[
Q \colon
\xymatrix{1 \ar[r]^a & 2 \ar[r]^b & 3}
\]
Want to make a representation $R$ of $Q$.
\onslide<2->
\[
\onslide<2->{R \colon}
\xymatrix{
\temporal<2>{}{\bullet}{V_1} \ar[r]^{\onslide<4->{f_a}} &
\temporal<2>{}{\bullet}{V_2} \ar[r]^{\onslide<4->{f_b}} &
\temporal<2>{}{\bullet}{V_3}
}
\]
Start with the quiver, and put
\begin{itemize}
\onslide<3->
\item a vector space at each vertex,
\onslide<4->
\item a linear transformation on each arrow.
\end{itemize}
\end{frame}

\begin{frame}{A representation}
\Large
\[
Q \colon
\xymatrix{1 \ar[r]^a & 2 \ar[r]^b & 3}
\]
\vspace{.5em}
\[
R \colon
\xymatrix{
k   \ar[r]^{\left( \begin{smallmatrix} 2 & 0 \end{smallmatrix} \right)} &
k^2 \ar[r]^{\left( \begin{smallmatrix} 4 \\ -1 \end{smallmatrix} \right)} &
k
}
\]
\end{frame}

\begin{frame}[fragile]{Creating a representation in QPA}
\[
R \colon
\xymatrix{
k   \ar[r]^{\left( \begin{smallmatrix} 2 & 0 \end{smallmatrix} \right)} &
k^2 \ar[r]^{\left( \begin{smallmatrix} 4 \\ -1 \end{smallmatrix} \right)} &
k
}
\]
\begin{verbatim}
gap> Q := Quiver(3, [[1,2,"a"],[2,3,"b"]]);;
gap> kQ := PathAlgebra(Rationals, Q);;
gap> M := RightModuleOverPathAlgebra
          (kQ, [1,2,1],
           [["a", [[2,0]]], ["b", [[4],[-1]]]]);
<[ 1, 2, 1 ]>
\end{verbatim}
% [show: basis]
\end{frame}

\begin{frame}{Module structure of a representation}
\[
R \colon
\xymatrix{
k   \ar[r]^{\left( \begin{smallmatrix} 2 & 0 \end{smallmatrix} \right)} &
k^2 \ar[r]^{\left( \begin{smallmatrix} 4 \\ -1 \end{smallmatrix} \right)} &
k
}
\]
An element $e$ of $R$:
\begin{align*}
e &\colon
\xymatrix{
3   \ar[r] &
{\VV{5}{7}} \ar[r] &
4
}
\\
\intertext{Multiplying $e$ with elements of $kQ$:}
e \cdot v_1 &\colon
\xymatrix{
3   \ar[r] &
{\VV{0}{0}} \ar[r] &
0
}
\\
e \cdot a &\colon
\xymatrix{
0   \ar[r] &
{\VV{6}{0}} \ar[r] &
0
}
\\
e \cdot b &\colon
\xymatrix{
0   \ar[r] &
{\VV{0}{0}} \ar[r] &
13
}
\end{align*}
% [show: basis, action of algebra elements]
\end{frame}

\begin{frame}{Representations and relations}
Given quiver with relations:
\[
Q \colon
\vcenter{\xymatrix{
1\ar[r]^a\ar[d]_c & 2\ar[d]^b\\
3\ar[r]^d & 4
}}
\qquad
\rho = \{ ab - 2cd \}
\]
A representation $M$ of $Q$ \defn{respects the relation}
$ab - 2cd$ if $f_a f_b - 2 f_c f_d = 0$.
\[
M \colon
\vcenter{\xymatrix{
V_1 \ar[r]^{f_a} \ar[d]_{f_c} &
V_2 \ar[d]^{f_b} \\
V_3 \ar[r]^{f_d} &
V_4
}}
\]
\end{frame}

\begin{frame}{Representations and relations}
Representations respecting relation $ab - 2cd$?
\begin{align*}
&
\vcenter{\xymatrix@C=30pt{
k\ar[r]^{f_a = 1}\ar[d]_{f_c = 1} & k\ar[d]^{f_b
  = \left(\begin{smallmatrix} 1 & 0\end{smallmatrix}\right)}\\
k\ar[r]^{f_d = \left(\begin{smallmatrix} 0 & 1\end{smallmatrix}\right)} & k^2
}}
&&
\onslide<2->{
\text{\textcolor{red}{NO}
($f_a f_b - 2 f_c f_d
  % = 1 \cdot \begin{pmatrix} 1&0 \end{pmatrix}
  %   - 2 \cdot 1 \cdot \begin{pmatrix} 0 & 1 \end{pmatrix}
  = \begin{pmatrix} 1 & -1 \end{pmatrix} \ne 0$)}}
\\
&
\onslide<3->{
\vcenter{\xymatrix@C=30pt{
k\ar[r]^{f_a = 1}\ar[d]_{f_c = 1} & k\ar[d]^{f_b = 2}\\
k\ar[r]^{f_d = 1} & k
}}}
&&
\onslide<4->{
\text{\textcolor{green}{YES}
($f_a f_b - 2 f_c f_d
  % = 1 \cdot \begin{pmatrix} 1&0 \end{pmatrix}
  %   - 2 \cdot 1 \cdot \begin{pmatrix} 0 & 1 \end{pmatrix}
  = 0$)}}
\end{align*}
\end{frame}

\begin{frame}{Representations and relations}

For $A = kQ/\langle \rho \rangle$:

{\huge
\[
\fmod A \equivalence \Rep_k (Q,\rho)
\]
}
\pause
\begin{textblock}{5}(0,-.5)
\begin{tikzpicture}
\node [below] at (0,0) {finitely generated $A$-modules};
\draw [->] (0,0) -- (1,1);
\end{tikzpicture}
\end{textblock}
\pause
\begin{textblock}{5}(7.3,-.5)
\begin{tikzpicture}
\node[align=left,anchor=north] at (0,0) {representations of $Q$ over $k$\\respecting $\rho$};
\draw [->] (0,0) -- (-1,1);
\end{tikzpicture}
\end{textblock}
\end{frame}

% [show modules over kQ/I in QPA]

\begin{frame}{Module homomorphisms}
For two modules $M$ and $N$ given as representations:
\begin{overprint}
\onslide<1-2>
\[
\xymatrix{
M \colon &
V_1 \ar[r]^{f_a} &
V_2 \ar[r]^{f_b} &
V_3
\\
N \colon &
W_1 \ar[r]^{g_a} &
W_2 \ar[r]^{g_b} &
W_3
}
\]
\onslide<3>
\[
\xymatrix{
M \colon \ar@<-.2em>[d]^{h} &
V_1 \ar[r]^{f_a} \ar[d]_{h_1} &
V_2 \ar[r]^{f_b} \ar[d]_{h_2} &
V_3              \ar[d]_{h_3}
\\
N \colon &
W_1 \ar[r]^{g_a} &
W_2 \ar[r]^{g_b} &
W_3
}
\]
\onslide<4->
\[
\xymatrix{
M \colon \ar@<-.2em>[d]^{h} &
V_1 \ar[r]^{f_a} \ar[d]_{h_1} \ar@{}[dr]|{\circ} &
V_2 \ar[r]^{f_b} \ar[d]_{h_2} \ar@{}[dr]|{\circ} &
V_3              \ar[d]_{h_3}
\\
N \colon &
W_1 \ar[r]^{g_a} &
W_2 \ar[r]^{g_b} &
W_3
}
\]
\end{overprint}
\vspace{1em}

\onslide<2->
... a homomorphism $h \colon M \to N$ is given by:
\onslide<3->
\begin{itemize}
\item linear maps $h_i$ for every vertex $i$,
\onslide<4->
\item commuting with the linear maps for the arrows.
\end{itemize}
\end{frame}

\begin{frame}[fragile]{Module homomorphisms in QPA}
\begin{overprint}
\onslide<1>
\[
\xymatrix{
M \colon &
0 \ar[r]     &
k \ar[r]^{5} &
k \ar@{}[d]^{\phantom{1}}
\\
N \colon &
k   \ar[r]_{\vv{0}{3}} &
k^2 \ar[r]^{\left( \begin{smallmatrix} 1 \\ 1 \end{smallmatrix} \right)} &
k
}
\]
\begin{verbatim}
gap> Q := Quiver(3, [[1,2,"a"], [2,3,"b"]]);;
gap> kQ := PathAlgebra(Rationals, Q);;
gap> M := RightModuleOverPathAlgebra
          (kQ, [0,1,1], [["b", [[5]]]]);;
gap> N := RightModuleOverPathAlgebra
          (kQ, [1,2,1], [["a", [[0,3]]],
                         ["b", [[1],[1]]]]);;
\end{verbatim}
\onslide<2>
\[
\xymatrix{
M \colon \ar@<-.2em>[d]^{h} &
0 \ar[r]     \ar[d] &
k \ar[r]^{5} \ar[d]_{\vv{3}{2}} &
k            \ar[d]^{1}
\\
N \colon &
k   \ar[r]_{\vv{0}{3}} &
k^2 \ar[r]^{\left( \begin{smallmatrix} 1 \\ 1 \end{smallmatrix} \right)} &
k
}
\]
\begin{verbatim}
gap> h := RightModuleHomOverAlgebra
          (M, N, [ [[0]], [[3,2]], [[1]] ]);
<<[ 0, 1, 1 ]> ---> <[ 1, 2, 1 ]>>
\end{verbatim}
\end{overprint}
\end{frame}

\end{document}
