%%%%%%%%%%%%%%%%%%%%%%%%%%%%%%%%%%%%%%%%%%%%%%%%%%%%%%%%%%%%%%%%%%%%%%%%%
%%
%W  install.tex          Radiroot documentation           Andreas Distler
%%
%Y  2005
%%

%%%%%%%%%%%%%%%%%%%%%%%%%%%%%%%%%%%%%%%%%%%%%%%%%%%%%%%%%%%%%%%%%%%%%%%%%
\Chapter{Installation}

%%%%%%%%%%%%%%%%%%%%%%%%%%%%%%%%%%%%%%%%%%%%%%%%%%%%%%%%%%%%%%%%%%%%%%%%%
\Section{Getting and Installing this Package}

This package is available at

\begintt
https://gap-packages.github.io/radiroot/
\endtt

in form of a gzipped tar-archive. For installation instructions see
Chapter~"ref:Installing a GAP Package" in the {\GAP} reference manual. 
Normally you will unpack the archive in the `pkg' directory of your
{\GAP} version by typing:

\beginexample
    bash> tar xfz radiroot-2.8.tar.gz        # for the gzipped tar-archive
\endexample

%%%%%%%%%%%%%%%%%%%%%%%%%%%%%%%%%%%%%%%%%%%%%%%%%%%%%%%%%%%%%%%%%%%%%%%%%
\Section{Loading and Testing the Package}

To use the {\Radiroot} package you have to request it explicitly. This  is
done by calling

\beginexample
gap> LoadPackage("radiroot");
-----------------------------------------------------------------------------
Loading  RadiRoot 2.8 (Roots of a Polynomial as Radicals)
by Andreas Distler (a.distler@tu-bs.de).
Homepage: https://gap-packages.github.io/radiroot/
-----------------------------------------------------------------------------
true
\endexample

The `LoadPackage' command is described  in  Section~"ref:LoadPackage"  in
the {\GAP} reference manual.

If you want to load the {\Radiroot} package by default, you  can  put  the
`LoadPackage' command  into  your  `gaprc'  file  (see  Section~"ref:The
gaprc file" in the {\GAP} reference manual).

Once the package is loaded, it is possible to check the correct
    installation by running the test suite of the package with the command

\beginexample
    gap> ReadPackage( "radiroot", "tst/testall.g" );
\endexample

%%%%%%%%%%%%%%%%%%%%%%%%%%%%%%%%%%%%%%%%%%%%%%%%%%%%%%%%%%%%%%%%%%%%%%%%%
\Section{Additional Requirements}

To use {\Radiroot} the package {\Alnuth} in version 3.0 or higher
has to be loaded with its interface fully functional.

In the standard mode a dvi file is created to display the roots of a
polynomial. As default the package uses the command `latex' searched
for in your system programs to create the dvi file and the command
`xdvi' to start the dvi viewer. If you can not use this settings you
will have to change the function `RR_Display' in the file `Strings.gi'
in the subdirectory `lib' of the package.

%%%%%%%%%%%%%%%%%%%%%%%%%%%%%%%%%%%%%%%%%%%%%%%%%%%%%%%%%%%%%%%%%%%%%%%%%
%%
%E
