% generated by GAPDoc2LaTeX from XML source (Frank Luebeck)
\documentclass[a4paper,11pt]{report}

\usepackage{a4wide}
\sloppy
\pagestyle{myheadings}
\usepackage{amssymb}
\usepackage[latin1]{inputenc}
\usepackage{makeidx}
\makeindex
\usepackage{color}
\definecolor{FireBrick}{rgb}{0.5812,0.0074,0.0083}
\definecolor{RoyalBlue}{rgb}{0.0236,0.0894,0.6179}
\definecolor{RoyalGreen}{rgb}{0.0236,0.6179,0.0894}
\definecolor{RoyalRed}{rgb}{0.6179,0.0236,0.0894}
\definecolor{LightBlue}{rgb}{0.8544,0.9511,1.0000}
\definecolor{Black}{rgb}{0.0,0.0,0.0}

\definecolor{linkColor}{rgb}{0.0,0.0,0.554}
\definecolor{citeColor}{rgb}{0.0,0.0,0.554}
\definecolor{fileColor}{rgb}{0.0,0.0,0.554}
\definecolor{urlColor}{rgb}{0.0,0.0,0.554}
\definecolor{promptColor}{rgb}{0.0,0.0,0.589}
\definecolor{brkpromptColor}{rgb}{0.589,0.0,0.0}
\definecolor{gapinputColor}{rgb}{0.589,0.0,0.0}
\definecolor{gapoutputColor}{rgb}{0.0,0.0,0.0}

%%  for a long time these were red and blue by default,
%%  now black, but keep variables to overwrite
\definecolor{FuncColor}{rgb}{0.0,0.0,0.0}
%% strange name because of pdflatex bug:
\definecolor{Chapter }{rgb}{0.0,0.0,0.0}
\definecolor{DarkOlive}{rgb}{0.1047,0.2412,0.0064}


\usepackage{fancyvrb}

\usepackage{mathptmx,helvet}
\usepackage[T1]{fontenc}
\usepackage{textcomp}


\usepackage[
            pdftex=true,
            bookmarks=true,        
            a4paper=true,
            pdftitle={Written with GAPDoc},
            pdfcreator={LaTeX with hyperref package / GAPDoc},
            colorlinks=true,
            backref=page,
            breaklinks=true,
            linkcolor=linkColor,
            citecolor=citeColor,
            filecolor=fileColor,
            urlcolor=urlColor,
            pdfpagemode={UseNone}, 
           ]{hyperref}

\newcommand{\maintitlesize}{\fontsize{50}{55}\selectfont}

% write page numbers to a .pnr log file for online help
\newwrite\pagenrlog
\immediate\openout\pagenrlog =\jobname.pnr
\immediate\write\pagenrlog{PAGENRS := [}
\newcommand{\logpage}[1]{\protect\write\pagenrlog{#1, \thepage,}}
%% were never documented, give conflicts with some additional packages

\newcommand{\GAP}{\textsf{GAP}}

%% nicer description environments, allows long labels
\usepackage{enumitem}
\setdescription{style=nextline}

%% depth of toc
\setcounter{tocdepth}{1}





%% command for ColorPrompt style examples
\newcommand{\gapprompt}[1]{\color{promptColor}{\bfseries #1}}
\newcommand{\gapbrkprompt}[1]{\color{brkpromptColor}{\bfseries #1}}
\newcommand{\gapinput}[1]{\color{gapinputColor}{#1}}


\begin{document}

\logpage{[ 0, 0, 0 ]}
\begin{titlepage}
\mbox{}\vfill

\begin{center}{\maintitlesize \textbf{\textsf{Citrus}\mbox{}}}\\
\vfill

\hypersetup{pdftitle=\textsf{Citrus}}
\markright{\scriptsize \mbox{}\hfill \textsf{Citrus} \hfill\mbox{}}
{\Huge \textbf{Computing with Semigroups of Transformations and Partial Permutations\mbox{}}}\\
\vfill

{\Huge Version 0.999\mbox{}}\\[1cm]
\mbox{}\\[2cm]
{\Large \textbf{J. D. Mitchell   \mbox{}}}\\
\hypersetup{pdfauthor=J. D. Mitchell   }
\end{center}\vfill

\mbox{}\\
{\mbox{}\\
\small \noindent \textbf{J. D. Mitchell   }  Email: \href{mailto://jdm3@st-and.ac.uk} {\texttt{jdm3@st-and.ac.uk}}\\
  Homepage: \href{http://tinyurl.com/jdmitchell} {\texttt{http://tinyurl.com/jdmitchell}}}\\
\end{titlepage}

\newpage\setcounter{page}{2}
{\small 
\section*{Abstract}
\logpage{[ 0, 0, 2 ]}
 The \textsf{Citrus} package is a \textsf{GAP} package for computing with semigroups of transformations and partial
permutations. Citrus contains more efficient methods than those available in
the GAP library (and in many cases more efficient than any other software) for
creating semigroups of transformations and partial permutations, calculating
their Green's classes, size, elements, group of units, minimal ideal, small
generating sets, testing membership, finding the inverses of a regular
element, factorizing elements over the generators, and many more. It is also
possible to test if a semigroup satisfies a particular property, such as if it
is regular, simple, inverse, completely regular, and a variety of further
properties. Several catalogues of examples are provided, such as generators
for the endomorphism monoids of every connected graph with at most 8 vertices,
and generators of the endomorphism monoids of the non-abelian groups with
order at most 64. \mbox{}}\\[1cm]
{\small 
\section*{Copyright}
\logpage{[ 0, 0, 1 ]}
{\copyright} 2011-12 by J. D. Mitchell.

 \textsf{Citrus} is free software; you can redistribute it and/or modify it under the terms of
the \href{ http://www.fsf.org/licenses/gpl.html} {GNU General Public License} as published by the Free Software Foundation; either version 2 of the License,
or (at your option) any later version. \mbox{}}\\[1cm]
{\small 
\section*{Acknowledgements}
\logpage{[ 0, 0, 3 ]}
 I would like to thank P. von Bunau, A. Distler, S. Linton, J. Neubueser, M. R.
Quick, E. F. Robertson, and N. Ruskuc for their help and suggestions. Special
thanks go to J. Araujo for his mathematical suggestions and to M. Neunhoeffer
for his invaluable help in improving the efficiency of the package. 

 I would also like to acknowledge the support of the Centre of Algebra at the
University of Lisbon, and of EPSRC grant number GR/S/56085/01. \mbox{}}\\[1cm]
\newpage

\def\contentsname{Contents\logpage{[ 0, 0, 4 ]}}

\tableofcontents
\newpage

 
\chapter{\textcolor{Chapter }{The \textsf{Citrus} package}}\label{citrus}
\logpage{[ 1, 0, 0 ]}
\hyperdef{L}{X78519F458386696A}{}
{
  \index{Citrus@\textsf{Citrus} package overview} This is the manual for the \textsf{Citrus} package version 0.999 for computing with semigroups of transformations and
partial permutations. \textsf{Citrus} 0.999 is an updated and expanded version of the \href{ http://schmidt.nuigalway.ie/monoid/index.html} {Monoid package for GAP 3} by Goetz Pfeiffer, Steve A. Linton, Edmund F. Robertson, and Nik Ruskuc and
the \href{ http://www-history.mcs.st-and.ac.uk/~jamesm/monoid/index.html} {Monoid package for GAP 4} by J. D. Mitchell.

 Some of the theory behind the algorithms in \textsf{Citrus} is described in \cite{pfeiffer1} and described in \cite{pfeiffer2}. Another reference is \cite{lallement}.

 The functionality of \textsf{Monoid} 3.1.4 for \textsf{GAP} 4 has been split between the \textsf{GAP} 4.5 packages \textsf{Citrus} and \href{ http://www-history.mcs.st-and.ac.uk/~jamesm/comatose/index.html } {Comatose}. \textsf{Citrus} 0.999 retains all the functionality of the original \textsf{Monoid} package for \textsf{GAP} 3; and many of those functions from \textsf{Monoid} 3.1.4 not involved in the computation of automorphism groups of semigroups.
The \href{ http://www-history.mcs.st-and.ac.uk/~jamesm/comatose/index.html } {Comatose} package retains those functions from \textsf{Monoid} 3.1.4 used to compute automorphism groups of transformation semigroups. 

 \noindent \textsf{Citrus} 0.999 contains more efficient methods than those available in the \textsf{GAP} library (and in many cases more efficient than any other software) for
creating semigroups of transformations and partial permutations, calculating
their Green's classes, size, elements, group of units, minimal ideal, and
testing membership, finding the inverses of a regular element, and factorizing
elements over the generators, and many more; see Chapters \ref{create} and \ref{green}. There are also methods for testing if a semigroup satisfies a particular
property, such as if it is regular, simple, inverse, completely regular, and a
variety of further properties; see Chapter \ref{green}. A range of functions is provided for creating and determining properties of
individual transformations and partial permutations such as the index and
period or the least idempotent power; see Chapter \ref{Transformations and partial permutations}. A large catalogue of examples is provided; see Sections \ref{Examples} and \ref{Catalogues}. \textsf{Citrus} also provides abbreviated names for many of the commonly used \textsf{GAP} library functions related to semigroups, and functions to read and write large
collections of transformations or partial permutations to a file; see \texttt{ReadCitrus} (\ref{ReadCitrus}) and \texttt{WriteCitrus} (\ref{WriteCitrus}). 

 \noindent The \textsf{Citrus} package is written in C and \textsf{GAP} code and requires the \href{ http://www-groups.mcs.st-and.ac.uk/~neunhoef/Computer/Software/Gap/orb.html } {Orb} and \href{ http://www-groups.mcs.st-and.ac.uk/~neunhoef/Computer/Software/Gap/io.html } {IO} packages. The \href{ http://www-groups.mcs.st-and.ac.uk/~neunhoef/Computer/Software/Gap/orb.html } {Orb} package is used to efficiently compute orbits in semigroups of transformations
and partial permutations, and these methods underpin many of the features of \textsf{Citrus}. The \href{ http://www-groups.mcs.st-and.ac.uk/~neunhoef/Computer/Software/Gap/io.html } {IO} package is used to read and write transformations and partial permutations to
a file. The \href{http://www.maths.qmul.ac.uk/~leonard/grape/} {Grape} package is used in the function \texttt{MunnSemigroup} (\ref{MunnSemigroup}) but nowhere else in \textsf{Citrus}. If \href{http://www.maths.qmul.ac.uk/~leonard/grape/} {Grape} is not fully installed, then \textsf{Citrus} can be used as normal with the except that the operation \texttt{MunnSemigroup} (\ref{MunnSemigroup}) does not work. 

 \textsf{Citrus} is still under development, and so some features may not function as expected.
At the present time, I do not know of any errors or serious issues with \textsf{Citrus}. If you find a bug or an issue with the package, then please let me know by
emailing me the details at: \href{mailto://jdm3@st-and.ac.uk} {\texttt{jdm3@st-and.ac.uk}}.

 For more details about semigroups in \textsf{GAP} or Green's relations in particular, see  (\textbf{Reference: Semigroups}) or  (\textbf{Reference: Green's Relations}). 
\section{\textcolor{Chapter }{Installing \textsf{Citrus}}}\label{install}
\logpage{[ 1, 1, 0 ]}
\hyperdef{L}{X7A7CA4CE84B39273}{}
{
  In this section we give a brief description of how to start using \textsf{Citrus}. If you have any problems getting \textsf{Citrus} working, then you could try emailing me at \href{mailto://jdm3@st-and.ac.uk} {\texttt{jdm3@st-and.ac.uk}}. 

 It is assumed that you have a working copy of \textsf{GAP} with version number 4.5 or higher. The most up-to-date version of \textsf{GAP} and instructions on how to install it can be obtained from the main \textsf{GAP} webpage \vspace{\baselineskip}

\noindent\vspace{\baselineskip} \href{http://www.gap-system.org} {\texttt{http://www.gap-system.org}}

 \noindent The following is a summary of the steps that should lead to a successful
installation of \textsf{Citrus}: 
\begin{itemize}
\item  download and install the \href{ http://www-groups.mcs.st-and.ac.uk/~neunhoef/Computer/Software/Gap/orb.html } {Orb} package version 4.2 or higher. For more details go to: \vspace{\baselineskip}

\noindent \href{http://www-groups.mcs.st-and.ac.uk/~neunhoef/Computer/Software/Gap/orb.html } {\texttt{http://www-groups.mcs.st-and.ac.uk/\texttt{\symbol{126}}neunhoef/Computer/Software/Gap/orb.html }}

 \noindent Note that \href{ http://www-groups.mcs.st-and.ac.uk/~neunhoef/Computer/Software/Gap/orb.html } {Orb} and \textsf{Citrus} both perform better if \href{ http://www-groups.mcs.st-and.ac.uk/~neunhoef/Computer/Software/Gap/orb.html } {Orb} is compiled. 
\item  download and install the \href{ http://www-groups.mcs.st-and.ac.uk/~neunhoef/Computer/Software/Gap/io.html } {IO} package version 4.1 or higher. For more details go to: \vspace{\baselineskip}

\noindent \href{http://www-groups.mcs.st-and.ac.uk/~neunhoef/Computer/Software/Gap/ io.html } {\texttt{http://www-groups.mcs.st-and.ac.uk/\texttt{\symbol{126}}neunhoef/Computer/Software/Gap/ io.html }} 
\item  if you want to be able to calculate the Munn semigroup of a semilattice \texttt{MunnSemigroup} (\ref{MunnSemigroup}), then download and fully install the \href{http://www.maths.qmul.ac.uk/~leonard/grape/} {Grape} package version 4.5 or higher. For more details go to:\vspace{\baselineskip}

\noindent \href{http://www.maths.qmul.ac.uk/~leonard/grape/} {\texttt{http://www.maths.qmul.ac.uk/\texttt{\symbol{126}}leonard/grape/}}

 If \href{http://www.maths.qmul.ac.uk/~leonard/grape/} {Grape} is not fully installed, then \textsf{Citrus} can be used as normal with the except that the operation \texttt{MunnSemigroup} (\ref{MunnSemigroup}) does not work. 
\item  download the package archive \texttt{citrus-0.999.tar.gz} from \vspace{\baselineskip}

\noindent\vspace{\baselineskip} \href{http://www-history.mcs.st-and.ac.uk/~jamesm/citrus/index.html } {\texttt{http://www-history.mcs.st-and.ac.uk/\texttt{\symbol{126}}jamesm/citrus/index.html }} 
\item  unzip and untar the file, this should create a directory called \texttt{citrus-0.999}.
\item  locate the \texttt{pkg} directory of your \textsf{GAP} directory, which contains the directories \texttt{lib}, \texttt{doc} and so on. Move the directory \texttt{citrus-0.999} into the \texttt{pkg} directory. 
\item  to use the functions in \textsf{Citrus} for partial permutations and inverse semigroups, you must compile the C part
of the package (in the "pkg" directory) by doing: 
\begin{Verbatim}[fontsize=\small,frame=single,label=]
  cd citrus
  ./configure
  make
\end{Verbatim}
 Further information about this step can be found in Section \ref{Compiling}. 
\item  start \textsf{GAP} in the usual way.
\item  type \texttt{LoadPackage("citrus");}
\item  compile the documentation by using \texttt{CitrusMakeDoc} (\ref{CitrusMakeDoc}) 
\end{itemize}
 \emph{\textsc{Please note that} \textsf{Citrus} can only be used to compute with semigroups of partial permutations if it has
been compiled. }

 Below is an example of an installation of \textsf{Citrus} in a Unix environment where \texttt{home} should be substituted with the main \textsf{GAP} directory (the one containing the folders \texttt{bin}, \texttt{lib}, and so on) in your installation of \textsf{GAP}.

 
\begin{Verbatim}[commandchars=!@|,fontsize=\small,frame=single,label=Example]
  !gapprompt@>| !gapinput@gunzip citrus-0.999.tar.gz |
  !gapprompt@>| !gapinput@tar -xf citrus-0.999.tar |
  !gapprompt@>| !gapinput@mv citrus-0.999 home/pkg|
  !gapprompt@>| !gapinput@gap |
  
  [ ... ]
  
  !gapprompt@gap>| !gapinput@LoadPackage("citrus");|
  ----------------------------------------------------------------------
  Loading  orb 4.2 (orb - Methods to enumerate orbits)
  by Juergen Mueller (http://www.math.rwth-aachen.de/~Juergen.Mueller),
     Max Neunhoeffer (http://www-groups.mcs.st-and.ac.uk/~neunhoef), and
     Felix Noeske (http://www.math.rwth-aachen.de/~Felix.Noeske).
  ----------------------------------------------------------------------
  ----------------------------------------------------------------------
  Loading  Citrus 0.999 (Citrus - ComputIng wiTh semigRUopS)
  by J. D. Mitchell (http://tinyurl.com/jdmitchell).
  ----------------------------------------------------------------------
  true
  gap>
\end{Verbatim}
 Presuming that the above steps can be completed successfully you will be
running the \textsf{Citrus} package!

 If you want to check that the package is working correctly, you should run
some the tests described in Section \ref{testing}.

 }

 
\section{\textcolor{Chapter }{Compiling the kernel component of \textsf{Citrus}}}\label{Compiling}
\logpage{[ 1, 2, 0 ]}
\hyperdef{L}{X80C19DDE78E77C1E}{}
{
  As of version 0.7, the \textsf{Citrus} package has a \textsf{GAP} kernel component in C which should be compiled. This component contains
low-level functions relating to partial permutations and it is not possible to
use these features or any related features in \textsf{Citrus} without compiling the package. It is possible to run \textsf{Citrus} without compiling it, but \textsf{Citrus} is limited to computing with transformation semigroups in this case.

 To compile the GAP kernel component in the 'pkg' directory do: 
\begin{Verbatim}[fontsize=\small,frame=single,label=]
  cd citrus
  ./configure
  make
\end{Verbatim}


 If you installed the package in another 'pkg' directory than the standard
'pkg' directory in your \textsf{GAP} installation, then you have to do two things. Firstly during compilation you
have to use the option '--with-gaproot=PATH' of the 'configure' script where
'PATH' is a path to the main GAP root directory (if not given the default
'../..' is assumed).

 If you installed \textsf{GAP} on several architectures, you must execute the configure/make step for each of
the architectures. You can either do this immediately after configuring and
compiling GAP itself on this architecture, or alternatively (when using
version 4.5 of GAP or newer) set the environment variable 'CONFIGNAME' to the
name of the configuration you used when compiling GAP before running
'./configure'. Note however that your compiler choice and flags (environment
variables 'CC' and 'CFLAGS') need to be chosen to match the setup of the
original GAP compilation. For example you have to specify 32-bit or 64-bit
mode correctly! }

 
\section{\textcolor{Chapter }{Compiling the documentation}}\label{doc}
\logpage{[ 1, 3, 0 ]}
\hyperdef{L}{X7E61798C7D949C4E}{}
{
 To compile the documentation use \texttt{CitrusMakeDoc} (\ref{CitrusMakeDoc}). If you want to use the help system, it is essential that you compile the
documentation. 

\subsection{\textcolor{Chapter }{CitrusMakeDoc}}
\logpage{[ 1, 3, 1 ]}\nobreak
\hyperdef{L}{X818E429B7D0BD9C1}{}
{\noindent\textcolor{FuncColor}{$\triangleright$\ \ \texttt{CitrusMakeDoc({\mdseries\slshape })\index{CitrusMakeDoc@\texttt{CitrusMakeDoc}}
\label{CitrusMakeDoc}
}\hfill{\scriptsize (function)}}\\
\textbf{\indent Returns:\ }
Nothing.



 This function should be called with no argument to compile the \textsf{Citrus} documentation. }

 }

 
\section{\textcolor{Chapter }{Testing the installation}}\label{testing}
\logpage{[ 1, 4, 0 ]}
\hyperdef{L}{X7AE7A7077F513655}{}
{
 In this section we describe how to test that \textsf{Citrus} is working as intended. To test that \textsf{Citrus} is installed correctly use \texttt{CitrusTestInstall} (\ref{CitrusTestInstall}) or for more extensive tests use \texttt{CitrusTestAll} (\ref{CitrusTestAll}). Please note that it will take a few seconds for \texttt{CitrusTestInstall} (\ref{CitrusTestInstall}) to finish and it will take no more than 1 minute for \texttt{CitrusTestAll} (\ref{CitrusTestAll}) to finish.

 Note that after calling \texttt{CitrusTestAll} (\ref{CitrusTestAll}), \texttt{CitrusTestAll} (\ref{CitrusTestAll}), or \texttt{CitrusTestManualExamples} (\ref{CitrusTestManualExamples}), the message \texttt{gzip: stdout: Broken pipe} might be displayed (several times). While this is unfortunate, it is not an
error and should simply be ignored. We hope to resolve this issue in the
future. If something goes wrong, then please review the instructions in
Section \ref{install} and ensure that \textsf{Citrus} has been properly installed. If you continue having problems, please email me
at \href{mailto://jdm3@st-and.ac.uk} {\texttt{jdm3@st-and.ac.uk}}. 

\subsection{\textcolor{Chapter }{CitrusTestAll}}
\logpage{[ 1, 4, 1 ]}\nobreak
\hyperdef{L}{X8745FD2B8587CEAC}{}
{\noindent\textcolor{FuncColor}{$\triangleright$\ \ \texttt{CitrusTestAll({\mdseries\slshape })\index{CitrusTestAll@\texttt{CitrusTestAll}}
\label{CitrusTestAll}
}\hfill{\scriptsize (function)}}\\
\textbf{\indent Returns:\ }
Nothing.



 This function should be called with no argument to comprehensively test that \textsf{Citrus} is working correctly. These tests should take no more than a few minutes to
complete. To quickly test that \textsf{Citrus} is installed correctly use \texttt{CitrusTestInstall} (\ref{CitrusTestInstall}). }

 

\subsection{\textcolor{Chapter }{CitrusTestInstall}}
\logpage{[ 1, 4, 2 ]}\nobreak
\hyperdef{L}{X845A72B47BD5900E}{}
{\noindent\textcolor{FuncColor}{$\triangleright$\ \ \texttt{CitrusTestInstall({\mdseries\slshape })\index{CitrusTestInstall@\texttt{CitrusTestInstall}}
\label{CitrusTestInstall}
}\hfill{\scriptsize (function)}}\\
\textbf{\indent Returns:\ }
Nothing.



 This function should be called with no argument to test your installation of \textsf{Citrus} is working correctly. These tests should take no more than a fraction of a
second to complete. To more comprehensively test that \textsf{Citrus} is installed correctly use \texttt{CitrusTestAll} (\ref{CitrusTestAll}). }

 

\subsection{\textcolor{Chapter }{CitrusTestManualExamples}}
\logpage{[ 1, 4, 3 ]}\nobreak
\hyperdef{L}{X81589CE97F199853}{}
{\noindent\textcolor{FuncColor}{$\triangleright$\ \ \texttt{CitrusTestManualExamples({\mdseries\slshape })\index{CitrusTestManualExamples@\texttt{CitrusTestManualExamples}}
\label{CitrusTestManualExamples}
}\hfill{\scriptsize (function)}}\\
\textbf{\indent Returns:\ }
Nothing.



 This function should be called with no argument to test the examples in the \textsf{Citrus} manual. These tests should take no more than a few minutes to complete. To
more comprehensively test that \textsf{Citrus} is installed correctly use \texttt{CitrusTestAll} (\ref{CitrusTestAll}). See also \texttt{CitrusTestInstall} (\ref{CitrusTestInstall}). }

 }

 
\section{\textcolor{Chapter }{More information during a computation}}\logpage{[ 1, 5, 0 ]}
\hyperdef{L}{X798CBC46800AB80F}{}
{
 

\subsection{\textcolor{Chapter }{InfoCitrus}}
\logpage{[ 1, 5, 1 ]}\nobreak
\hyperdef{L}{X7F5178348556E6B3}{}
{\noindent\textcolor{FuncColor}{$\triangleright$\ \ \texttt{InfoCitrus\index{InfoCitrus@\texttt{InfoCitrus}}
\label{InfoCitrus}
}\hfill{\scriptsize (info class)}}\\


 \texttt{InfoCitrus} is the info class of the \textsf{Citrus} package. The info level is initially set to 0 and no info messages are
displayed. We recommend that you set the level to 1 so that basic info
messages are displayed. To increase the amount of information displayed during
a computation increase the info level to 2 or 3. To stop all info messages
from being displayed, set the info level to 0. See also  (\textbf{Reference: Info Functions}) and \texttt{SetInfoLevel} (\textbf{Reference: SetInfoLevel}). }

 }

 
\section{\textcolor{Chapter }{Reading and writing transformations and partial permutations to a file}}\logpage{[ 1, 6, 0 ]}
\hyperdef{L}{X875277447E3DF377}{}
{
 The functions \texttt{ReadCitrus} (\ref{ReadCitrus}) and \texttt{WriteCitrus} (\ref{WriteCitrus}) can be used to read or write transformations or partial permutations to a
file. 

\subsection{\textcolor{Chapter }{CitrusDir}}
\logpage{[ 1, 6, 1 ]}\nobreak
\hyperdef{L}{X833D52497EAC6D92}{}
{\noindent\textcolor{FuncColor}{$\triangleright$\ \ \texttt{CitrusDir({\mdseries\slshape })\index{CitrusDir@\texttt{CitrusDir}}
\label{CitrusDir}
}\hfill{\scriptsize (function)}}\\
\textbf{\indent Returns:\ }
A string.



 This function returns the absolute path to the \textsf{Citrus} package directory as a string. The same result can be obtained typing: 
\begin{Verbatim}[commandchars=@|A,fontsize=\small,frame=single,label=Example]
  PackageInfo("citrus")[1]!.InstallationPath;
\end{Verbatim}
 at the \textsf{GAP} prompt. }

 

\subsection{\textcolor{Chapter }{ReadCitrus}}
\logpage{[ 1, 6, 2 ]}\nobreak
\hyperdef{L}{X7A0CE3A67EEB1EA9}{}
{\noindent\textcolor{FuncColor}{$\triangleright$\ \ \texttt{ReadCitrus({\mdseries\slshape filename[, nr]})\index{ReadCitrus@\texttt{ReadCitrus}}
\label{ReadCitrus}
}\hfill{\scriptsize (function)}}\\
\textbf{\indent Returns:\ }
A list of: transformations, partial permutations, lists of partial
permutations or lists of transformations. 



 If \mbox{\texttt{\mdseries\slshape filename}} is a file created using \texttt{WriteCitrus} (\ref{WriteCitrus}), then \texttt{ReadCitrus} returns the contents of this file as a list of lists of transformations and
partial permutations. If the optional second argument \mbox{\texttt{\mdseries\slshape nr}} is present, then \texttt{ReadCitrus} returns the transformations or partial permutations stored in the \mbox{\texttt{\mdseries\slshape nr}}th line of \mbox{\texttt{\mdseries\slshape filename}}. 

 Note that if the file \mbox{\texttt{\mdseries\slshape filename}} is gzipped, then the message \texttt{gzip: stdout: Broken pipe} might be displayed when \texttt{ReadCitrus} is used. While this is unfortunate, it not an error that affect the output of \texttt{ReadCitrus}. We hope to resolve this issue in the future. }

 

\subsection{\textcolor{Chapter }{WriteCitrus}}
\logpage{[ 1, 6, 3 ]}\nobreak
\hyperdef{L}{X7AB7775378DE8585}{}
{\noindent\textcolor{FuncColor}{$\triangleright$\ \ \texttt{WriteCitrus({\mdseries\slshape filename, list})\index{WriteCitrus@\texttt{WriteCitrus}}
\label{WriteCitrus}
}\hfill{\scriptsize (function)}}\\
\textbf{\indent Returns:\ }
Nothing.



 This function provides a method for writing transformations and partial
permutations to a file that uses a relatively small amount of disk space. The
resulting file can be further compressed using \texttt{gzip}.

 \mbox{\texttt{\mdseries\slshape list}} should be a list of: transformations of equal degree, partial permutations, a
semigroup of transformations or partial permutations, or a list of lists of
transformations, partial permutations, or semigroups of transformations or
partial permutations; and \mbox{\texttt{\mdseries\slshape filename}} should be a string containing the name of a file where the entries in \mbox{\texttt{\mdseries\slshape list}} will be written. 

 \texttt{WriteCitrus} appends a line to the file \mbox{\texttt{\mdseries\slshape filename}} for every entry in \mbox{\texttt{\mdseries\slshape list}}. If any element of \mbox{\texttt{\mdseries\slshape list}} is a semigroup, then the generators of that semigroup are written to \mbox{\texttt{\mdseries\slshape filename}}. 

 The first character \texttt{m} of the appended line is the number of characters in the degree of the
transformations to be written, the next \texttt{m} characters are the degree \texttt{n} of the transformations to be written, and the transformations themselves are
written in blocks of \texttt{m*n} in the remainder of the line. For example, the transformations: 
\begin{Verbatim}[commandchars=!@|,fontsize=\small,frame=single,label=Example]
  [ Transformation( [ 2, 6, 7, 2, 6, 9, 9, 1, 1, 5 ] ), 
    Transformation( [ 3, 8, 1, 9, 9, 4, 10, 5, 10, 6 ] )]
\end{Verbatim}
 are written as: 
\begin{Verbatim}[commandchars=!@|,fontsize=\small,frame=single,label=Example]
  210 2 2 6 7 2 6 9 9 1 1 5 3 8 1 9 9 410 510 6
\end{Verbatim}
 The file \mbox{\texttt{\mdseries\slshape filename}} can be read using \texttt{ReadCitrus} (\ref{ReadCitrus}). 

 
\begin{Verbatim}[commandchars=!@|,fontsize=\small,frame=single,label=Example]
  !gapprompt@gap>| !gapinput@file:=Concatenation(CitrusDir(), "/examples/graph7c.citrus.gz");;|
  !gapprompt@gap>| !gapinput@ReadCitrus(file, 453);|
  [ Transformation( [ 1, 2, 2, 4, 5, 6, 7 ] ), 
    Transformation( [ 1, 2, 3, 4, 5, 6, 7 ] ), 
    Transformation( [ 1, 2, 3, 4, 5, 7, 7 ] ), 
    Transformation( [ 1, 3, 2, 4, 7, 6, 7 ] ), 
    Transformation( [ 4, 2, 1, 1, 6, 5, 7 ] ), 
    Transformation( [ 4, 3, 2, 1, 6, 7, 7 ] ), 
    Transformation( [ 4, 4, 5, 7, 6, 1, 1 ] ), 
    Transformation( [ 7, 6, 6, 1, 2, 4, 4 ] ), 
    Transformation( [ 7, 7, 5, 4, 3, 1, 1 ] ) ]
\end{Verbatim}
 }

 }

 
\section{\textcolor{Chapter }{What does \textsf{Citrus} do best?}}\logpage{[ 1, 7, 0 ]}
\hyperdef{L}{X826BB2CB7CA0F6B2}{}
{
 Due to inherent difficulties with computing Green's $\mathcal{L}$-, $\mathcal{D}$-, and $\mathcal{H}$-classes in transformation semigroups, the methods used to compute with
Green's $\mathcal{R}$-classes are the most efficient in \textsf{Citrus}. Thus, if you are computing with a transformation semigroup, wherever
possible it is advisable to use the commands relating to Green's $\mathcal{R}$-classes rather than those relating to Green's $\mathcal{L}$-, $\mathcal{D}$-, or $\mathcal{H}$-classes. No such difficulties are present when computing with semigroups of
partial permutations. 

 The methods in \textsf{Citrus} allow the computation of individual Green's classes without the need to
compute all the elements of the underlying semigroup; see \texttt{GreensRClassOfElementNC} (\ref{GreensRClassOfElementNC}). It is also possible to compute all the $\mathcal{R}$-classes, the number of elements and test membership in a transformation
semigroup without computing all the elements; see, for example, \texttt{GreensRClasses} (\ref{GreensRClasses}), \texttt{RClassReps} (\ref{RClassReps}), \texttt{IteratorOfRClassReps} (\ref{IteratorOfRClassReps}), \texttt{IteratorOfRClasses} (\ref{IteratorOfRClasses}), or \texttt{NrRClasses} (\ref{NrRClasses}). This may be useful if you want to study a very large semigroup where
computing all the elements of the semigroup is infeasible. }

 
\section{\textcolor{Chapter }{New methods for existing functions}}\logpage{[ 1, 8, 0 ]}
\hyperdef{L}{X7F3F95577FB2396F}{}
{
 In this section we list the functions from the \textsf{GAP} library (those available when \textsf{Citrus} is not loaded) which have new methods in \textsf{Citrus}. Some of these new methods apply to a wider class of objects than the
existing methods. Where this is the case, we give references to both the \textsf{Citrus} documentation and the \textsf{GAP} manual; otherwise we may only give references to the \textsf{GAP} manual.

 \textsf{Citrus} contains special methods for the following \textsf{GAP} functions where the argument is an arbitrary Green's class of a semigroup of
transformations or partial permutations (specifically a Green's class
satisfying \texttt{IsGreensClassOfTransSemigp} (\ref{IsGreensClassOfTransSemigp}) or \texttt{IsGreensClassOfPartialPermSemigp} (\ref{IsGreensClassOfPartialPermSemigp})) or a semigroup itself (where appropriate). For more details about semigroups
in \textsf{GAP} or Green's relations in particular, see  (\textbf{Reference: Semigroups}) or  (\textbf{Reference: Green's Relations}). 
\begin{itemize}
\item  \texttt{\texttt{\symbol{92}}{\textless}} (\ref{<:for Green's classes}), \texttt{=} and \texttt{in} (see  (\textbf{Reference: Comparisons})), 
\item  \texttt{AsSSortedList} (\textbf{Reference: AsSSortedList}), 
\item  \texttt{Enumerator} (\textbf{Reference: Enumerator}), 
\item  \texttt{GreensHClasses} (\textbf{Reference: GreensHClasses}) and \texttt{GreensHClasses} (\ref{GreensHClasses}), 
\item  \texttt{GreensHClassOfElement} (\textbf{Reference: GreensHClassOfElement}), 
\item  \texttt{GreensLClasses} (\textbf{Reference: GreensLClasses}) and \texttt{GreensLClasses} (\ref{GreensLClasses}), 
\item  \texttt{GreensLClassOfElement} (\textbf{Reference: GreensLClassOfElement}), 
\item  \texttt{GreensRClasses} (\textbf{Reference: GreensRClasses}) and \texttt{GreensRClasses} (\ref{GreensRClasses}), 
\item  \texttt{GreensRClassOfElement} (\textbf{Reference: GreensRClassOfElement}), 
\item  \texttt{GreensDClasses} (\textbf{Reference: GreensDClasses}) and \texttt{GreensDClasses} (\ref{GreensDClasses}), 
\item  \texttt{GreensDClassOfElement} (\textbf{Reference: GreensDClassOfElement}), 
\item  \texttt{Idempotents} (\textbf{Reference: Idempotents}) and \texttt{Idempotents} (\ref{Idempotents}), 
\item  \texttt{Size} (\textbf{Reference: Size}). 
\end{itemize}
 

 In addition to those functions mentioned above that can be applied to
arbitrary Green's classes, \textsf{Citrus} contains special methods for the following \textsf{GAP} functions: 
\begin{itemize}
\item \texttt{IsRegularDClass} (\textbf{Reference: IsRegularDClass}) and \texttt{IsRegularDClass} (\ref{IsRegularDClass}),
\item \texttt{GroupHClassOfGreensDClass} (\textbf{Reference: GroupHClassOfGreensDClass}) and \texttt{GroupHClass} (\ref{GroupHClass}),
\item \texttt{IsGroupHClass} (\textbf{Reference: IsGroupHClass}),
\item \texttt{IsomorphismPermGroup} (\textbf{Reference: IsomorphismPermGroup}) (for a group $\mathcal{H}$-class),
\item \texttt{StructureDescription} (\textbf{Reference: StructureDescription}) (for a group $\mathcal{H}$-class),
\item \texttt{IsomorphismTransformationMonoid} and \texttt{IsomorphismTransformationSemigroup} (\textbf{Reference: IsomorphismTransformationSemigroup}) and (for a permutation group).
\end{itemize}
 

 \textsf{Citrus} contains functions synonymous to some of the commands mentioned in this
section but, for the sake of convenience with abbreviated names; for further
details see below. }

 }

 
\chapter{\textcolor{Chapter }{Transformations and partial permutations }}\label{Transformations and partial permutations}
\logpage{[ 2, 0, 0 ]}
\hyperdef{L}{X8139A4C17BDD5F3F}{}
{
 A \emph{transformation} is just a function from the set $\{1,\ldots, n\}$ to itself, and a \emph{partial permutation} is an injective function from a subset of $\{1,\ldots, n\}$ to $\{1,\ldots, n\}$ where $n$ is a positive integer. In this chapter, we describe functions for creating and
determining fundamental properties of transformations and partial
permutations. Semigroups in the \textsf{Citrus} package are semigroups of transformations and partial permutations, and so we
describe how to create and manipulate these objects first.

 The functions described in this section relating to transformations extend the
functionality of \textsf{GAP} library; see also  (\textbf{Reference: Transformations}).

  
\section{\textcolor{Chapter }{Creating transformations}}\logpage{[ 2, 1, 0 ]}
\hyperdef{L}{X80F3086F87E93DF8}{}
{
 In this section we describe the functions available in \textsf{Citrus} for creating transformations. The following functions from the \textsf{GAP} reference manual are particularly relevant to this section: 
\begin{itemize}
\item  \texttt{Transformation} (\textbf{Reference: Transformation}), 
\item  \texttt{IdentityTransformation} (\textbf{Reference: IdentityTransformation}), 
\item  \texttt{RandomTransformation} (\textbf{Reference: RandomTransformation}), 
\item  \texttt{AsTransformation} (\textbf{Reference: AsTransformation}). 
\end{itemize}
 

\subsection{\textcolor{Chapter }{ConstantTransformation}}
\logpage{[ 2, 1, 1 ]}\nobreak
\hyperdef{L}{X7F1E4B5184210D2B}{}
{\noindent\textcolor{FuncColor}{$\triangleright$\ \ \texttt{ConstantTransformation({\mdseries\slshape m, n})\index{ConstantTransformation@\texttt{ConstantTransformation}}
\label{ConstantTransformation}
}\hfill{\scriptsize (function)}}\\
\textbf{\indent Returns:\ }
A constant transformation of degree \mbox{\texttt{\mdseries\slshape m}} with value \mbox{\texttt{\mdseries\slshape n}}. 



 
\begin{Verbatim}[commandchars=!@|,fontsize=\small,frame=single,label=Example]
  !gapprompt@gap>| !gapinput@ConstantTransformation(5, 1);|
  Transformation( [ 1, 1, 1, 1, 1 ] )
  !gapprompt@gap>| !gapinput@ConstantTransformation(6, 4);|
  Transformation( [ 4, 4, 4, 4, 4, 4 ] )
\end{Verbatim}
 }

 
\subsection{\textcolor{Chapter }{Idempotent}}\logpage{[ 2, 1, 2 ]}
\hyperdef{L}{X85D1071484CE004C}{}
{
\noindent\textcolor{FuncColor}{$\triangleright$\ \ \texttt{Idempotent({\mdseries\slshape ker, img})\index{Idempotent@\texttt{Idempotent}}
\label{Idempotent}
}\hfill{\scriptsize (function)}}\\
\noindent\textcolor{FuncColor}{$\triangleright$\ \ \texttt{IdempotentNC({\mdseries\slshape ker, img})\index{IdempotentNC@\texttt{IdempotentNC}}
\label{IdempotentNC}
}\hfill{\scriptsize (function)}}\\
\textbf{\indent Returns:\ }
An idempotent transformation.



 This function returns an idempotent with \texttt{CanonicalTransSameKernel} (\ref{CanonicalTransSameKernel}) equal to \mbox{\texttt{\mdseries\slshape ker}} and image set equal to \mbox{\texttt{\mdseries\slshape img}} after first checking that \texttt{IsInjectiveTransOnList} (\ref{IsInjectiveTransOnList}) holds with argument \mbox{\texttt{\mdseries\slshape ker, img}}. 

 \texttt{IdempotentNC} returns an idempotent with \texttt{CanonicalTransSameKernel} (\ref{CanonicalTransSameKernel}) equal to \mbox{\texttt{\mdseries\slshape ker}} and image set equal to \mbox{\texttt{\mdseries\slshape img}} without checking that \texttt{IsInjectiveTransOnList} (\ref{IsInjectiveTransOnList}) holds. 
\begin{Verbatim}[commandchars=!@|,fontsize=\small,frame=single,label=Example]
  !gapprompt@gap>| !gapinput@f:=Transformation( [ 10, 2, 3, 10, 5, 10, 7, 2, 5, 6 ] );;|
  !gapprompt@gap>| !gapinput@ker:=CanonicalTransSameKernel(f);|
  [ 1, 2, 3, 1, 4, 1, 5, 2, 4, 6 ]
  !gapprompt@gap>| !gapinput@img:=ImageSetOfTransformation(f);|
  [ 2, 3, 5, 6, 7, 10 ]
  !gapprompt@gap>| !gapinput@Idempotent(ker, img);|
  Transformation( [ 6, 2, 3, 6, 5, 6, 7, 2, 5, 10 ] )
\end{Verbatim}
 }

 
\subsection{\textcolor{Chapter }{RandomIdempotent}}\logpage{[ 2, 1, 3 ]}
\hyperdef{L}{X7C5BED247F770ECB}{}
{
\noindent\textcolor{FuncColor}{$\triangleright$\ \ \texttt{RandomIdempotent({\mdseries\slshape img, m})\index{RandomIdempotent@\texttt{RandomIdempotent}}
\label{RandomIdempotent}
}\hfill{\scriptsize (operation)}}\\
\noindent\textcolor{FuncColor}{$\triangleright$\ \ \texttt{RandomIdempotentNC({\mdseries\slshape img, m})\index{RandomIdempotentNC@\texttt{RandomIdempotentNC}}
\label{RandomIdempotentNC}
}\hfill{\scriptsize (operation)}}\\
\textbf{\indent Returns:\ }
An idempotent transformation.



 Returns a random idempotent with image set \mbox{\texttt{\mdseries\slshape img}} of degree \mbox{\texttt{\mdseries\slshape m}} after checking that the maximum value in \mbox{\texttt{\mdseries\slshape img}} is at most \mbox{\texttt{\mdseries\slshape m}}. 

 \texttt{RandomIdempotentNC} returns a random idempotent with image set \mbox{\texttt{\mdseries\slshape img}} of degree \mbox{\texttt{\mdseries\slshape m}} without checking that the maximum value in \mbox{\texttt{\mdseries\slshape img}} is at most \mbox{\texttt{\mdseries\slshape m}}. 
\begin{Verbatim}[commandchars=!@|,fontsize=\small,frame=single,label=Example]
  !gapprompt@gap>| !gapinput@RandomIdempotent([ 1, 2, 3 ], 5);|
  Transformation( [ 1, 2, 3, 1, 3 ] )
  !gapprompt@gap>| !gapinput@RandomIdempotent([ 1, 2, 3 ], 4);|
  Transformation( [ 1, 2, 3, 3 ] )
  !gapprompt@gap>| !gapinput@RandomIdempotent([ 1, 2, 3 ], 3);            |
  Transformation( [ 1, 2, 3 ] )
  !gapprompt@gap>| !gapinput@RandomIdempotent([ 1, 2, 4, 5, 6 ], 10);|
  Transformation( [ 1, 2, 6, 4, 5, 6, 2, 5, 1, 6 ] )
\end{Verbatim}
 }

 
\subsection{\textcolor{Chapter }{RandomTransformation}}\logpage{[ 2, 1, 4 ]}
\hyperdef{L}{X8475448F87E8CB8A}{}
{
\noindent\textcolor{FuncColor}{$\triangleright$\ \ \texttt{RandomTransformation({\mdseries\slshape arg})\index{RandomTransformation@\texttt{RandomTransformation}}
\label{RandomTransformation}
}\hfill{\scriptsize (operation)}}\\
\noindent\textcolor{FuncColor}{$\triangleright$\ \ \texttt{RandomTransformationNC({\mdseries\slshape arg})\index{RandomTransformationNC@\texttt{RandomTransformationNC}}
\label{RandomTransformationNC}
}\hfill{\scriptsize (operation)}}\\
\textbf{\indent Returns:\ }
A transformation.



 If \mbox{\texttt{\mdseries\slshape arg}} is a positive integer, then a random transformation of degree \mbox{\texttt{\mdseries\slshape arg}} is returned; see \texttt{RandomTransformation} (\textbf{Reference: RandomTransformation}).

 If \mbox{\texttt{\mdseries\slshape arg}}\texttt{[1]} is a list of positive integers and \mbox{\texttt{\mdseries\slshape arg}}\texttt{[2]} is a positive integer, then \texttt{RandomTransformation} returns a random transformation with degree \mbox{\texttt{\mdseries\slshape arg}}\texttt{[2]} and image contained in \mbox{\texttt{\mdseries\slshape arg}}\texttt{[1]}. 

 The no check version \texttt{RandomTransformationNC} does not check that the arguments can define a transformation. 
\begin{Verbatim}[commandchars=!@|,fontsize=\small,frame=single,label=Example]
  !gapprompt@gap>| !gapinput@RandomTransformation([1,2,3], 6);             |
  Transformation( [ 2, 1, 2, 1, 1, 2 ] )
  !gapprompt@gap>| !gapinput@RandomTransformationNC([1,2,3], 6);|
  Transformation( [ 3, 1, 2, 2, 1, 2 ] )
\end{Verbatim}
 }

 }

 
\section{\textcolor{Chapter }{Creating partial permutations}}\logpage{[ 2, 2, 0 ]}
\hyperdef{L}{X7B9D451D7FDA1DD8}{}
{
 In this section we describe the functions available in \textsf{Citrus} for creating partial permutations. In addition to the functions described in
this section, partial permutations can also be created from permutations and
certain transformations; see Section \ref{Changing representation}. 

 \emph{Please note that the functions for partial permutations and inverse semigroups
are only available if you have compiled \textsf{Citrus}.} 

\subsection{\textcolor{Chapter }{PartialPerm (for a domain and range)}}
\logpage{[ 2, 2, 1 ]}\nobreak
\hyperdef{L}{X80531F067E9E1EB9}{}
{\noindent\textcolor{FuncColor}{$\triangleright$\ \ \texttt{PartialPerm({\mdseries\slshape dom, ran})\index{PartialPerm@\texttt{PartialPerm}!for a domain and range}
\label{PartialPerm:for a domain and range}
}\hfill{\scriptsize (function)}}\\
\noindent\textcolor{FuncColor}{$\triangleright$\ \ \texttt{PartialPerm({\mdseries\slshape list})\index{PartialPerm@\texttt{PartialPerm}!for a dense range}
\label{PartialPerm:for a dense range}
}\hfill{\scriptsize (function)}}\\
\noindent\textcolor{FuncColor}{$\triangleright$\ \ \texttt{PartialPermNC({\mdseries\slshape dom, ran})\index{PartialPermNC@\texttt{PartialPermNC}!for a domain and range}
\label{PartialPermNC:for a domain and range}
}\hfill{\scriptsize (function)}}\\
\noindent\textcolor{FuncColor}{$\triangleright$\ \ \texttt{PartialPermNC({\mdseries\slshape list})\index{PartialPermNC@\texttt{PartialPermNC}!for a dense range}
\label{PartialPermNC:for a dense range}
}\hfill{\scriptsize (function)}}\\
\textbf{\indent Returns:\ }
A partial permutation.



 Partial permutations can be created in two ways: by giving the domain and the
range, or the dense range list. 

 The partial permutation defined by a domain \mbox{\texttt{\mdseries\slshape dom}} and range \mbox{\texttt{\mdseries\slshape ran}}, where \mbox{\texttt{\mdseries\slshape dom}} is a set of positive integers and \mbox{\texttt{\mdseries\slshape ran}} is a duplicate free list of positive integers, maps \mbox{\texttt{\mdseries\slshape dom}}\texttt{[i]} to \mbox{\texttt{\mdseries\slshape ran}}\texttt{[i]}. For example, the partial permutation mapping $1$ and $5$ to $20$ and $2$ can be created using: 
\begin{Verbatim}[commandchars=!@|,fontsize=\small,frame=single,label=Example]
  PartialPermNC([1,5],[20,2]); 
\end{Verbatim}
 

 The partial permutation defined by a dense range list \mbox{\texttt{\mdseries\slshape list}}, maps the positive integer \texttt{i} to \mbox{\texttt{\mdseries\slshape list}}\texttt{[i]} if \mbox{\texttt{\mdseries\slshape list}}\texttt{[i]{\textless}{\textgreater}{\textgreater}0} and is undefined at \texttt{i} if \mbox{\texttt{\mdseries\slshape list}}\texttt{[i]=0}. For example, the partial permutation mapping $1$ and $5$ to $20$ and $2$ can be created using: 
\begin{Verbatim}[commandchars=!@|,fontsize=\small,frame=single,label=Example]
  PartialPermNC([20,0,0,0,2]);
\end{Verbatim}


 Regardless of which of these two methods are used to create a partial
permutation in \textsf{GAP} the internal representation is the same. 

 If the largest point in the domain is larger than the rank of the partial
permutation, then using the dense range list to define the partial permutation
will require less typing; otherwise using the domain and the range will
require less typing. For example, the partial permutation mapping $10000$ to $1$ can be defined using: 
\begin{Verbatim}[commandchars=!@|,fontsize=\small,frame=single,label=Example]
  PartialPermNC([10000], [1]);
\end{Verbatim}
 but using the dense range list would require a list with $9999$ entries equal to $0$ and the final entry equal to $1$. On the other hand, the identity on \texttt{[1,2,3,4,6]} can be defined using: 
\begin{Verbatim}[commandchars=!@|,fontsize=\small,frame=single,label=Example]
  PartialPermNC([1,2,3,4,0,6]);
\end{Verbatim}
 

 \texttt{PartialPerm} checks that the argument defines a partial permatutation whereas \texttt{PartialPermNC} does not. \texttt{PartialPermNC} is a kernel function written in C, which performs no checks on it argument,
and so if the argument does not define a partial permutation, then the results
can be unpredictable, and it could even potentially cause \textsf{GAP} to crash. 

 It is currently only possible to create partial permutations acting on
positive integers not greater than $65535$. Note that this might cause some unexpected behaviour when multiplying
partial permutations by permutations. For example, if \texttt{f:=PartialPerm([1,2,3]);}, then \texttt{f*(1,100000)*(1,100000)} will return an error since the result of \texttt{f*(1,100000)} would be a partial permutation acting on too many points. However, the product \texttt{f*((1,100000)*(1,100000))} is just \texttt{f} and no error is given. 

 Please note that a partial permutation in \textsf{GAP} is never a permutation nor is a permutation ever a partial permutation. For
example, the permutation \texttt{(1,4,2)} fixes \texttt{3} but the partial permutation \texttt{PartialPerm([4,1,0,2]);} is not defined on \texttt{3}. }

 

\subsection{\textcolor{Chapter }{RestrictedPartialPerm}}
\logpage{[ 2, 2, 2 ]}\nobreak
\hyperdef{L}{X80ABBF4883C79060}{}
{\noindent\textcolor{FuncColor}{$\triangleright$\ \ \texttt{RestrictedPartialPerm({\mdseries\slshape f, set})\index{RestrictedPartialPerm@\texttt{RestrictedPartialPerm}}
\label{RestrictedPartialPerm}
}\hfill{\scriptsize (operation)}}\\
\noindent\textcolor{FuncColor}{$\triangleright$\ \ \texttt{RestrictedPartialPermNC({\mdseries\slshape f, set})\index{RestrictedPartialPermNC@\texttt{RestrictedPartialPermNC}}
\label{RestrictedPartialPermNC}
}\hfill{\scriptsize (operation)}}\\
\textbf{\indent Returns:\ }
A partial permutation.



 \texttt{RestrictedPartialPerm} returns a new partial permutation that acts on the points in the set of
positive integers \mbox{\texttt{\mdseries\slshape set}} in the same way as the partial permutation \mbox{\texttt{\mdseries\slshape f}}, and that is undefined on those points that are not in \mbox{\texttt{\mdseries\slshape set}}.

 \texttt{RestrictedPartialPermNC} does not check whether \mbox{\texttt{\mdseries\slshape set}} is a set of positive integers, if it is not, then the results of this function
are unpredictable. 
\begin{Verbatim}[commandchars=!@|,fontsize=\small,frame=single,label=Example]
  !gapprompt@gap>| !gapinput@f:=PartialPermNC( [ 1, 3, 4, 7, 8, 9 ], [ 9, 4, 1, 6, 2, 8 ] );;|
  !gapprompt@gap>| !gapinput@RestrictedPartialPerm(f, [2,3,6,10]);|
  [ 3 ] -> [ 4 ]
\end{Verbatim}
 }

 
\subsection{\textcolor{Chapter }{One}}\logpage{[ 2, 2, 3 ]}
\hyperdef{L}{X8129A6877FFD804B}{}
{
\noindent\textcolor{FuncColor}{$\triangleright$\ \ \texttt{One({\mdseries\slshape f})\index{One@\texttt{One}!for a partial perm}
\label{One:for a partial perm}
}\hfill{\scriptsize (method)}}\\
\textbf{\indent Returns:\ }
A partial permutation.



 As described in the reference manual \texttt{OneImmutable} (\textbf{Reference: OneImmutable}), \texttt{One} returns the multiplicative neutral element of the partial permutation \mbox{\texttt{\mdseries\slshape f}}, which is the identity partial permutation on the union of the domain and
range of \mbox{\texttt{\mdseries\slshape f}}; see \texttt{DomainOfPartialPerm} (\ref{DomainOfPartialPerm}) and \texttt{RangeOfPartialPerm} (\ref{RangeOfPartialPerm}). 
\begin{Verbatim}[commandchars=!@|,fontsize=\small,frame=single,label=Example]
  !gapprompt@gap>| !gapinput@f:=PartialPermNC([ 1, 2, 3, 4, 5, 7, 10 ], [ 3, 7, 9, 6, 1, 10, 2 ]);;|
  !gapprompt@gap>| !gapinput@One(f);|
  <identity on [ 1, 2, 3, 4, 5, 6, 7, 9, 10 ]>
\end{Verbatim}
 }

 

\subsection{\textcolor{Chapter }{LeftOne}}
\logpage{[ 2, 2, 4 ]}\nobreak
\hyperdef{L}{X7C8CCA647AF7A2BC}{}
{\noindent\textcolor{FuncColor}{$\triangleright$\ \ \texttt{LeftOne({\mdseries\slshape f})\index{LeftOne@\texttt{LeftOne}}
\label{LeftOne}
}\hfill{\scriptsize (function)}}\\
\noindent\textcolor{FuncColor}{$\triangleright$\ \ \texttt{RightOne({\mdseries\slshape f})\index{RightOne@\texttt{RightOne}}
\label{RightOne}
}\hfill{\scriptsize (function)}}\\
\textbf{\indent Returns:\ }
The idempotent partial permutation \mbox{\texttt{\mdseries\slshape f}}\texttt{*}\mbox{\texttt{\mdseries\slshape f\texttt{\symbol{94}}-1}} or \mbox{\texttt{\mdseries\slshape f\texttt{\symbol{94}}-1}}\texttt{*}\mbox{\texttt{\mdseries\slshape f}}. 



 \texttt{LeftOne} returns the identity partial permutation on \texttt{DomainOfPartialPerm(\mbox{\texttt{\mdseries\slshape f}})}, which is equal to \mbox{\texttt{\mdseries\slshape f}}\texttt{*}\mbox{\texttt{\mdseries\slshape f\texttt{\symbol{94}}-1}}; see \texttt{DomainOfPartialPerm} (\ref{DomainOfPartialPerm}).

 \texttt{RightOne} returns the idenpartial permutation on \texttt{RangeOfPartialPerm(\mbox{\texttt{\mdseries\slshape f}})}, which is equal to \mbox{\texttt{\mdseries\slshape f\texttt{\symbol{94}}-1}}\texttt{*}\mbox{\texttt{\mdseries\slshape f}}; see \texttt{RangeOfPartialPerm} (\ref{RangeOfPartialPerm}).

 The methods for \texttt{LeftOne} and \texttt{RightOne} are more efficient than composing \mbox{\texttt{\mdseries\slshape f}} with \mbox{\texttt{\mdseries\slshape f\texttt{\symbol{94}}-1}} or vice versa. 
\begin{Verbatim}[commandchars=!@|,fontsize=\small,frame=single,label=Example]
  !gapprompt@gap>| !gapinput@f:=PartialPermNC( [ 1, 2, 3, 4, 5, 6, 7, 8, 10, 11, 12, 16, 17, 18, 19 ], |
  !gapprompt@>| !gapinput@[ 3, 12, 14, 4, 11, 18, 17, 2, 9, 5, 15, 8, 20, 10, 19 ] );;|
  !gapprompt@gap>| !gapinput@LeftOne(f);|
  <identity on [ 1, 2, 3, 4, 5, 6, 7, 8, 10, 11, 12, 16, 17, 18, 19 ]>
  !gapprompt@gap>| !gapinput@RightOne(f);|
  <identity on [ 2, 3, 4, 5, 8, 9, 10, 11, 12, 14, 15, 17, 18, 19, 20 ]>
\end{Verbatim}
 }

 

\subsection{\textcolor{Chapter }{RandomPartialPerm}}
\logpage{[ 2, 2, 5 ]}\nobreak
\hyperdef{L}{X7E6ADC8583C31530}{}
{\noindent\textcolor{FuncColor}{$\triangleright$\ \ \texttt{RandomPartialPerm({\mdseries\slshape n})\index{RandomPartialPerm@\texttt{RandomPartialPerm}}
\label{RandomPartialPerm}
}\hfill{\scriptsize (function)}}\\
\textbf{\indent Returns:\ }
A partial permutation.



 \texttt{RandomPartialPerm} returns a randomly chosen partial permutation where points in the domain and
range are bounded above by the positive integer \mbox{\texttt{\mdseries\slshape n}}. Points are chosen in the dense range list of the returned partial
permutation with the probability of a value being undefined equal to the
probability of it being defined. Thus the expected rank of the returned
partial permutation is approximately \texttt{n/2}. 
\begin{Verbatim}[commandchars=!@|,fontsize=\small,frame=single,label=Example]
  !gapprompt@gap>| !gapinput@f:=RandomPartialPerm(10);  |
  [ 1, 2, 3, 4, 7, 8, 9 ] -> [ 5, 9, 1, 7, 2, 8, 4 ]
\end{Verbatim}
 }

 }

 
\section{\textcolor{Chapter }{Displaying partial permutations}}\logpage{[ 2, 3, 0 ]}
\hyperdef{L}{X7849595B81D063EE}{}
{
 Partial permutations \texttt{f} with rank less than 20 are displayed as: 
\begin{Verbatim}[commandchars=!@|,fontsize=\small,frame=single,label=Example]
  DomainOfPartialPerm(f) -> RangeOfPartialPerm(f)
\end{Verbatim}
 see \texttt{DomainOfPartialPerm} (\ref{DomainOfPartialPerm}) and \texttt{RangeOfPartialPerm} (\ref{RangeOfPartialPerm}). The empty mapping is displayed as: 
\begin{Verbatim}[commandchars=!@|,fontsize=\small,frame=single,label=Example]
  <empty mapping>
\end{Verbatim}
 and a partial identity is displayed as, for example, 
\begin{Verbatim}[commandchars=!@|,fontsize=\small,frame=single,label=Example]
  <identity on [ 1, 2, 3, 4, 5, 8, 12, 19, 20 ]>
\end{Verbatim}
 

 If a partial permutation in \textsf{GAP} has rank 20 points or more, then to make the display more readable the domain
and the range of the partial permutatation are not printed, for example, a
partial permutation with rank 27 will be displayed as: 
\begin{Verbatim}[commandchars=!@|,fontsize=\small,frame=single,label=Example]
  <partial perm on 27 pts>
\end{Verbatim}
 If you want to display a partial permutation in a form that can be copied and
pasted back into \textsf{GAP}, then use \texttt{Display}: 
\begin{Verbatim}[commandchars=!@|,fontsize=\small,frame=single,label=Example]
  !gapprompt@gap>| !gapinput@Display(f);               |
  PartialPermNC( [ 1, 2, 3, 5, 6, 7, 8 ], [ 5, 9, 10, 6, 3, 8, 4 ] )
\end{Verbatim}
 Likewise, you can display a collection of partial permutations using \texttt{Display}: 
\begin{Verbatim}[commandchars=!@|,fontsize=\small,frame=single,label=Example]
  !gapprompt@gap>| !gapinput@Display(Generators(s));|
  [ PartialPermNC( [ 1, 2, 3, 4, 5, 8, 10 ], [ 3, 1, 4, 2, 5, 6, 7 ] ),
  PartialPermNC( [ 1, 2, 4, 7, 8, 9 ], [ 10, 7, 8, 5, 9, 1 ] ) ]
\end{Verbatim}
 

\subsection{\textcolor{Chapter }{PrettyPrintPP}}
\logpage{[ 2, 3, 1 ]}\nobreak
\hyperdef{L}{X7F3367CF7A2DDEEF}{}
{\noindent\textcolor{FuncColor}{$\triangleright$\ \ \texttt{PrettyPrintPP({\mdseries\slshape f})\index{PrettyPrintPP@\texttt{PrettyPrintPP}}
\label{PrettyPrintPP}
}\hfill{\scriptsize (function)}}\\
\textbf{\indent Returns:\ }
Nothing.



 \texttt{PrettyPrintPP} will print the partial permutation \mbox{\texttt{\mdseries\slshape f}} as a product of disjoint permutations and chains. A \emph{chain} is a list \texttt{X} of length \texttt{n} such that: 
\begin{itemize}
\item  \texttt{X[1]} is an element of the domain of \mbox{\texttt{\mdseries\slshape f}} but not the range
\item \texttt{X[i]\texttt{\symbol{94}}\mbox{\texttt{\mdseries\slshape f}}=X[i+1]} for all \texttt{n} in $\{1,\ldots, n-1\}$
\item \texttt{X[n]} is in the range of \mbox{\texttt{\mdseries\slshape f}} but not the domain.
\end{itemize}
 In the display, permutations are displayed as they usually are in \textsf{GAP}, fixed points are displayed enclosed in round brackets, and chains are
displayed enclosed in square brackets. 
\begin{Verbatim}[commandchars=!@|,fontsize=\small,frame=single,label=Example]
  !gapprompt@gap>| !gapinput@f:=PartialPermNC( [ 1, 2, 3, 4, 5, 6, 7, 8, 10, 11, 12, 16, 17, 18, 19 ],|
  !gapprompt@>| !gapinput@[ 3, 12, 14, 4, 11, 18, 17, 2, 9, 5, 15, 8, 20, 10, 19 ]);;|
  !gapprompt@gap>| !gapinput@PrettyPrintPP(f);|
  [1,3,14][16,8,2,12,15](4)(5,11)[6,18,10,9][7,17,20](19)
\end{Verbatim}
 }

 }

 
\section{\textcolor{Chapter }{Properties of transformations}}\logpage{[ 2, 4, 0 ]}
\hyperdef{L}{X7A7A6FDE7D85111C}{}
{
 In this section we describe the functions available in \textsf{Citrus} for finding various properties of transformations. 

 In addition to those functions described below, \textsf{Citrus} also contains a special method for \texttt{RankOfTransformation} (\textbf{Reference: RankOfTransformation}). 

\subsection{\textcolor{Chapter }{CanonicalTransSameKernel}}
\logpage{[ 2, 4, 1 ]}\nobreak
\hyperdef{L}{X7CA4FE7A7C7EBBA3}{}
{\noindent\textcolor{FuncColor}{$\triangleright$\ \ \texttt{CanonicalTransSameKernel({\mdseries\slshape obj})\index{CanonicalTransSameKernel@\texttt{CanonicalTransSameKernel}}
\label{CanonicalTransSameKernel}
}\hfill{\scriptsize (function)}}\\
\textbf{\indent Returns:\ }
A list of positive integers.



 The argument \mbox{\texttt{\mdseries\slshape obj}} should be a transformation or the list of images of a transformation. If \mbox{\texttt{\mdseries\slshape obj}} is a transformation, then we define \texttt{f:=\mbox{\texttt{\mdseries\slshape obj}}} and if \mbox{\texttt{\mdseries\slshape obj}} is the image list of a transformation we define \texttt{f:=Transformation(\mbox{\texttt{\mdseries\slshape obj}})}. \texttt{CanonicalTransSameKernel} returns the image list of a transformation \texttt{g} such that the kernel of \texttt{g} equals the kernel of \texttt{f} and \texttt{j\texttt{\symbol{94}}g=i} for all \texttt{j} in \texttt{KernelOfTransformation(g)[i]}. For a given transformation \texttt{f}, there is a unique transformation \texttt{g} with this property. 

 See also \texttt{ImageListOfTransformation} (\textbf{Reference: ImageListOfTransformation}) and \texttt{KernelOfTransformation} (\textbf{Reference: KernelOfTransformation}). 
\begin{Verbatim}[commandchars=!@|,fontsize=\small,frame=single,label=Example]
  !gapprompt@gap>| !gapinput@f:=Transformation( [ 10, 3, 7, 10, 1, 5, 9, 2, 6, 10 ] );;|
  !gapprompt@gap>| !gapinput@CanonicalTransSameKernel(f);|
  [ 1, 2, 3, 1, 4, 5, 6, 7, 8, 1 ]
  !gapprompt@gap>| !gapinput@f:=[ 10, 6, 7, 9, 9, 9, 4, 1, 4, 1 ];;|
  !gapprompt@gap>| !gapinput@CanonicalTransSameKernel(f);|
  [ 1, 2, 3, 4, 4, 4, 5, 6, 5, 6 ]
\end{Verbatim}
 }

 

\subsection{\textcolor{Chapter }{IndexPeriodOfTransformation}}
\logpage{[ 2, 4, 2 ]}\nobreak
\hyperdef{L}{X863216CB7AF88BED}{}
{\noindent\textcolor{FuncColor}{$\triangleright$\ \ \texttt{IndexPeriodOfTransformation({\mdseries\slshape f})\index{IndexPeriodOfTransformation@\texttt{IndexPeriodOfTransformation}}
\label{IndexPeriodOfTransformation}
}\hfill{\scriptsize (function)}}\\
\textbf{\indent Returns:\ }
A pair of positive integers.



 Returns the least positive integers \texttt{m, r} such that \texttt{\mbox{\texttt{\mdseries\slshape f}}\texttt{\symbol{94}}(m+r)=\mbox{\texttt{\mdseries\slshape f}}\texttt{\symbol{94}}m}, which are known as the \emph{index} and \emph{period} of the transformation \mbox{\texttt{\mdseries\slshape f}}. 
\begin{Verbatim}[commandchars=!@|,fontsize=\small,frame=single,label=Example]
  !gapprompt@gap>| !gapinput@f:=Transformation( [ 3, 4, 4, 6, 1, 3, 3, 7, 1 ] );;|
  !gapprompt@gap>| !gapinput@IndexPeriodOfTransformation(f);|
  [ 2, 3 ]
  !gapprompt@gap>| !gapinput@f^2=f^5;|
  true
\end{Verbatim}
 }

 
\subsection{\textcolor{Chapter }{InversesOfTransformation}}\logpage{[ 2, 4, 3 ]}
\hyperdef{L}{X846B9EBB86A69BDC}{}
{
\noindent\textcolor{FuncColor}{$\triangleright$\ \ \texttt{InversesOfTransformation({\mdseries\slshape S, f})\index{InversesOfTransformation@\texttt{InversesOfTransformation}}
\label{InversesOfTransformation}
}\hfill{\scriptsize (operation)}}\\
\noindent\textcolor{FuncColor}{$\triangleright$\ \ \texttt{InversesOfTransformationNC({\mdseries\slshape S, f})\index{InversesOfTransformationNC@\texttt{InversesOfTransformationNC}}
\label{InversesOfTransformationNC}
}\hfill{\scriptsize (operation)}}\\
\textbf{\indent Returns:\ }
A list of transformations.



 \texttt{InversesOfTransformation} returns a list of the inverses of the transformation \mbox{\texttt{\mdseries\slshape f}} in the transformation semigroup \mbox{\texttt{\mdseries\slshape S}} after first checking that \mbox{\texttt{\mdseries\slshape f}} is an element of \mbox{\texttt{\mdseries\slshape S}}. 

 A transformation \texttt{g} in \mbox{\texttt{\mdseries\slshape S}} is an \emph{inverse} of \mbox{\texttt{\mdseries\slshape f}} if \texttt{\mbox{\texttt{\mdseries\slshape f}}*g*\mbox{\texttt{\mdseries\slshape f}}=\mbox{\texttt{\mdseries\slshape f}}} and \texttt{g*\mbox{\texttt{\mdseries\slshape f}}*g=g}.

 The function \texttt{InversesOfTransformationNC} does not check that \mbox{\texttt{\mdseries\slshape f}} is an element of \mbox{\texttt{\mdseries\slshape S}}. 
\begin{Verbatim}[commandchars=!@|,fontsize=\small,frame=single,label=Example]
  !gapprompt@gap>| !gapinput@S:=Semigroup([ Transformation( [ 3, 1, 4, 2, 5, 2, 1, 6, 1 ] ), |
  !gapprompt@>| !gapinput@ Transformation( [ 5, 7, 8, 8, 7, 5, 9, 1, 9 ] ), |
  !gapprompt@>| !gapinput@ Transformation( [ 7, 6, 2, 8, 4, 7, 5, 8, 3 ] ) ]);;|
  !gapprompt@gap>| !gapinput@f:=Transformation( [ 3, 1, 4, 2, 5, 2, 1, 6, 1 ] );;|
  !gapprompt@gap>| !gapinput@InversesOfTransformationNC(S, f);|
  [  ]
  !gapprompt@gap>| !gapinput@IsRegularTransformation(S, f);|
  false
  !gapprompt@gap>| !gapinput@f:=Transformation( [ 1, 9, 7, 5, 5, 1, 9, 5, 1 ] );;|
  !gapprompt@gap>| !gapinput@InversesOfTransformation(S, f);|
  [ Transformation( [ 1, 5, 1, 2, 5, 1, 3, 2, 2 ] ), 
    Transformation( [ 1, 2, 3, 5, 5, 1, 3, 5, 2 ] ), 
    Transformation( [ 1, 5, 1, 1, 5, 1, 3, 1, 2 ] ) ]
  !gapprompt@gap>| !gapinput@IsRegularTransformation(S, f);|
  true
\end{Verbatim}
 }

 

\subsection{\textcolor{Chapter }{IsInjectiveTransOnList}}
\logpage{[ 2, 4, 4 ]}\nobreak
\hyperdef{L}{X86C0902D811E917B}{}
{\noindent\textcolor{FuncColor}{$\triangleright$\ \ \texttt{IsInjectiveTransOnList({\mdseries\slshape obj, list})\index{IsInjectiveTransOnList@\texttt{IsInjectiveTransOnList}}
\label{IsInjectiveTransOnList}
}\hfill{\scriptsize (operation)}}\\
\textbf{\indent Returns:\ }
\texttt{true} or \texttt{false}. 



 The argument \mbox{\texttt{\mdseries\slshape obj}} should be a transformation or the list of images of a transformation and \mbox{\texttt{\mdseries\slshape list}} should be a list of positive integers. If \mbox{\texttt{\mdseries\slshape obj}} is a transformation, then we define \texttt{f:=\mbox{\texttt{\mdseries\slshape obj}}} and if \mbox{\texttt{\mdseries\slshape obj}} is the image list of a transformation we define \texttt{f:=Transformation(\mbox{\texttt{\mdseries\slshape obj}})}. \texttt{IsInjectiveTransOnList} returns \texttt{true} if \texttt{f} is injective on \mbox{\texttt{\mdseries\slshape list}} and \texttt{false} if it is not. If \mbox{\texttt{\mdseries\slshape list}} is not duplicate free, then \texttt{false} is returned. 

 See also \texttt{ImageListOfTransformation} (\textbf{Reference: ImageListOfTransformation}) and \texttt{Transformation} (\textbf{Reference: Transformation}). 
\begin{Verbatim}[commandchars=!@|,fontsize=\small,frame=single,label=Example]
  !gapprompt@gap>| !gapinput@f:=Transformation( [ 2, 6, 7, 2, 6, 9, 9, 1, 1, 5 ] );;|
  !gapprompt@gap>| !gapinput@IsInjectiveTransOnList(f, [1,5]);|
  true
  !gapprompt@gap>| !gapinput@IsInjectiveTransOnList(f, [5,1]);|
  true
  !gapprompt@gap>| !gapinput@IsInjectiveTransOnList(f, [5,1,5,1,1,]);|
  false
  !gapprompt@gap>| !gapinput@IsInjectiveTransOnList([1,2,3,4,5], [5,1,2,3]);   |
  true
\end{Verbatim}
 }

 

\subsection{\textcolor{Chapter }{IsRegularTransformation}}
\logpage{[ 2, 4, 5 ]}\nobreak
\hyperdef{L}{X7856A73E80DAF56F}{}
{\noindent\textcolor{FuncColor}{$\triangleright$\ \ \texttt{IsRegularTransformation({\mdseries\slshape S, f})\index{IsRegularTransformation@\texttt{IsRegularTransformation}}
\label{IsRegularTransformation}
}\hfill{\scriptsize (operation)}}\\
\textbf{\indent Returns:\ }
\texttt{true} or \texttt{false}. 



 This function returns \texttt{true} if \mbox{\texttt{\mdseries\slshape f}} is a regular element of the transformation semigroup \mbox{\texttt{\mdseries\slshape S}} and \texttt{false} if it is not.

 A transformation \mbox{\texttt{\mdseries\slshape f}} is \emph{regular} in a transformation semigroup \mbox{\texttt{\mdseries\slshape S}} if it lies inside a regular $\mathcal{D}$-class; see \texttt{IsRegularDClass} (\textbf{Reference: IsRegularDClass}) or \texttt{IsRegularDClass} (\ref{IsRegularDClass}). Equivalently \mbox{\texttt{\mdseries\slshape f}} is regular if the orbit of the image of \mbox{\texttt{\mdseries\slshape f}} containing a transversal of the kernel of \mbox{\texttt{\mdseries\slshape f}}; see \texttt{Transformation} (\textbf{Reference: Transformation}) and \texttt{ImageSetOfTransformation} (\textbf{Reference: ImageSetOfTransformation}). 
\begin{Verbatim}[commandchars=!@|,fontsize=\small,frame=single,label=Example]
  !gapprompt@gap>| !gapinput@S:=Monoid(Transformation([2,2,4,4,5,6]),Transformation([5,3,4,4,6,6]));;|
  !gapprompt@gap>| !gapinput@f:=Generators(S)[1];;|
  !gapprompt@gap>| !gapinput@IsRegularTransformation(S, f);|
  true
  !gapprompt@gap>| !gapinput@img:=ImageSetOfTransformation(f);|
  [ 2, 4, 5, 6 ]
  !gapprompt@gap>| !gapinput@o:=Orb(S, img, OnSets);; Enumerate(o);|
  <closed orbit, 3 points>
  !gapprompt@gap>| !gapinput@ForAny(o, x-> IsInjectiveTransOnList(f, x));|
  true
  !gapprompt@gap>| !gapinput@IsRegularTransformation(S, Generators(S)[2]);|
  false
  !gapprompt@gap>| !gapinput@IsRegularTransformation(FullTransformationSemigroup(6), Generators(S)[2]);|
  true
\end{Verbatim}
 }

 

\subsection{\textcolor{Chapter }{DegreeOfTransformation}}
\logpage{[ 2, 4, 6 ]}\nobreak
\hyperdef{L}{X78A209C87CF0E32B}{}
{\noindent\textcolor{FuncColor}{$\triangleright$\ \ \texttt{DegreeOfTransformation({\mdseries\slshape f})\index{DegreeOfTransformation@\texttt{DegreeOfTransformation}}
\label{DegreeOfTransformation}
}\hfill{\scriptsize (function)}}\\
\noindent\textcolor{FuncColor}{$\triangleright$\ \ \texttt{Degree({\mdseries\slshape f})\index{Degree@\texttt{Degree}!for a transformation}
\label{Degree:for a transformation}
}\hfill{\scriptsize (function)}}\\
\noindent\textcolor{FuncColor}{$\triangleright$\ \ \texttt{Degree({\mdseries\slshape C})\index{Degree@\texttt{Degree}!for a transformation coll}
\label{Degree:for a transformation coll}
}\hfill{\scriptsize (function)}}\\
\textbf{\indent Returns:\ }
A positive integer.



 The \emph{degree} of a transformation \mbox{\texttt{\mdseries\slshape f}} is the length of \texttt{ImageListOfTransformation} (\textbf{Reference: ImageListOfTransformation}). 

 The degree of a transformation collection \mbox{\texttt{\mdseries\slshape C}} is the degree of any (and all) transformations in \mbox{\texttt{\mdseries\slshape C}}. 

 See also \texttt{DegreeOfTransformation} (\textbf{Reference: DegreeOfTransformation}). }

 

\subsection{\textcolor{Chapter }{RankOfTransformation}}
\logpage{[ 2, 4, 7 ]}\nobreak
\hyperdef{L}{X85F22DDD84C28583}{}
{\noindent\textcolor{FuncColor}{$\triangleright$\ \ \texttt{RankOfTransformation({\mdseries\slshape f})\index{RankOfTransformation@\texttt{RankOfTransformation}}
\label{RankOfTransformation}
}\hfill{\scriptsize (function)}}\\
\noindent\textcolor{FuncColor}{$\triangleright$\ \ \texttt{Rank({\mdseries\slshape f})\index{Rank@\texttt{Rank}!for a transformation}
\label{Rank:for a transformation}
}\hfill{\scriptsize (function)}}\\
\textbf{\indent Returns:\ }
A positive integer.



 \texttt{RankOfTransformation} returns the length of the set of image points of the transformation \mbox{\texttt{\mdseries\slshape f}}; see \texttt{RankOfTransformation} (\textbf{Reference: RankOfTransformation}).

 \texttt{Rank(\mbox{\texttt{\mdseries\slshape f}})} returns the same result as \texttt{RankOfTransformation} and is included for the sake of having a shorter name. \texttt{Rank} can also be applied to partial permutations; see \texttt{RankOfPartialPerm} (\ref{RankOfPartialPerm}) 
\begin{Verbatim}[commandchars=!@|,fontsize=\small,frame=single,label=Example]
  !gapprompt@gap>| !gapinput@f:=Transformation( [ 8, 5, 8, 2, 2, 8, 4, 7, 3, 1 ] );;|
  !gapprompt@gap>| !gapinput@RankOfTransformation(f);|
  7
  !gapprompt@gap>| !gapinput@Rank(f);|
  7
\end{Verbatim}
 }

 

\subsection{\textcolor{Chapter }{SmallestIdempotentPower (for a transformation)}}
\logpage{[ 2, 4, 8 ]}\nobreak
\hyperdef{L}{X85FE9F20810BCC70}{}
{\noindent\textcolor{FuncColor}{$\triangleright$\ \ \texttt{SmallestIdempotentPower({\mdseries\slshape f})\index{SmallestIdempotentPower@\texttt{SmallestIdempotentPower}!for a transformation}
\label{SmallestIdempotentPower:for a transformation}
}\hfill{\scriptsize (operation)}}\\
\textbf{\indent Returns:\ }
A positive integer.



 This function returns the least positive integer \texttt{n} such that the transformation \texttt{\mbox{\texttt{\mdseries\slshape f}}\texttt{\symbol{94}}n} is an idempotent. See also \texttt{SmallestIdempotentPower} (\ref{SmallestIdempotentPower:for a partial perm}). 
\begin{Verbatim}[commandchars=!@|,fontsize=\small,frame=single,label=Example]
  !gapprompt@gap>| !gapinput@f:=Transformation( [ 6, 7, 4, 1, 7, 4, 6, 1, 3, 4 ] );;|
  !gapprompt@gap>| !gapinput@SmallestIdempotentPower(f);|
  3
  !gapprompt@gap>| !gapinput@f:=Transformation( [ 6, 6, 6, 2, 7, 1, 5, 3, 10, 6 ] );;|
  !gapprompt@gap>| !gapinput@SmallestIdempotentPower(f);|
  2
\end{Verbatim}
 }

 }

 
\section{\textcolor{Chapter }{Properties of partial permutations}}\logpage{[ 2, 5, 0 ]}
\hyperdef{L}{X80FCAB5A7CA59D69}{}
{
 In this section we describe the functions available in \textsf{Citrus} for finding various properties of partial permutations. 

 

\subsection{\textcolor{Chapter }{IsPartialPerm}}
\logpage{[ 2, 5, 1 ]}\nobreak
\hyperdef{L}{X7EECE133792B30FC}{}
{\noindent\textcolor{FuncColor}{$\triangleright$\ \ \texttt{IsPartialPerm({\mdseries\slshape obj})\index{IsPartialPerm@\texttt{IsPartialPerm}}
\label{IsPartialPerm}
}\hfill{\scriptsize (category)}}\\
\noindent\textcolor{FuncColor}{$\triangleright$\ \ \texttt{IsPartialPermCollection({\mdseries\slshape obj})\index{IsPartialPermCollection@\texttt{IsPartialPermCollection}}
\label{IsPartialPermCollection}
}\hfill{\scriptsize (category)}}\\
\textbf{\indent Returns:\ }
\texttt{true} or \texttt{false}.



 Each \emph{partial permutation} in \textsf{GAP} lies in the category \texttt{IsPartialPerm}. Basic operations for partial permutations are \texttt{DomainOfPartialPerm} (\ref{DomainOfPartialPerm}), \texttt{RangeOfPartialPerm} (\ref{RangeOfPartialPerm}), \texttt{RangeSetOfPartialPerm} (\ref{RangeSetOfPartialPerm}), \texttt{RankOfPartialPerm} (\ref{RankOfPartialPerm}), \texttt{DegreeOfPartialPerm} (\ref{DegreeOfPartialPerm}), multiplication of two partial permutations is via \texttt{*}, and exponentiation with the first argument a positive integer \texttt{i} and second argument a partial permutation \texttt{f} where the result is the image \texttt{i\texttt{\symbol{94}}f} of the point \texttt{i} under \texttt{f}.

 \texttt{IsPartialPermCollection} is the category of collections of partial permutations. For example, a
semigroup of partial permutations belongs in \texttt{IsPartialPermCollection}. }

 

\subsection{\textcolor{Chapter }{DomainOfPartialPerm}}
\logpage{[ 2, 5, 2 ]}\nobreak
\hyperdef{L}{X784A14F787E041D7}{}
{\noindent\textcolor{FuncColor}{$\triangleright$\ \ \texttt{DomainOfPartialPerm({\mdseries\slshape f})\index{DomainOfPartialPerm@\texttt{DomainOfPartialPerm}}
\label{DomainOfPartialPerm}
}\hfill{\scriptsize (operation)}}\\
\textbf{\indent Returns:\ }
A set of positive integers.



 The \emph{domain} of a partial permutation \mbox{\texttt{\mdseries\slshape f}} is the set of positive integers where \mbox{\texttt{\mdseries\slshape f}} is defined. 
\begin{Verbatim}[commandchars=!@|,fontsize=\small,frame=single,label=Example]
  !gapprompt@gap>| !gapinput@f:=PartialPermNC( [ 1, 2, 3, 6, 8, 10 ], [ 2, 6, 7, 9, 1, 5 ] );;|
  !gapprompt@gap>| !gapinput@DomainOfPartialPerm(f);|
  [ 1, 2, 3, 6, 8, 10 ]
\end{Verbatim}
 }

 

\subsection{\textcolor{Chapter }{RangeOfPartialPerm}}
\logpage{[ 2, 5, 3 ]}\nobreak
\hyperdef{L}{X858638A27E681E1A}{}
{\noindent\textcolor{FuncColor}{$\triangleright$\ \ \texttt{RangeOfPartialPerm({\mdseries\slshape f})\index{RangeOfPartialPerm@\texttt{RangeOfPartialPerm}}
\label{RangeOfPartialPerm}
}\hfill{\scriptsize (operation)}}\\
\textbf{\indent Returns:\ }
A duplicate-free list of positive integers.



 The \emph{range} of a partial permutation \mbox{\texttt{\mdseries\slshape f}} is the list of images of the elements of the domain \mbox{\texttt{\mdseries\slshape f}} where \texttt{RangeOfPartialPerm(f)[i]=DomainOfPartialPerm(f)[i]\texttt{\symbol{94}}f.} 
\begin{Verbatim}[commandchars=!@|,fontsize=\small,frame=single,label=Example]
  !gapprompt@gap>| !gapinput@f:=PartialPermNC( [ 1, 2, 3, 4, 5, 8, 10 ], [ 7, 1, 4, 3, 2, 6, 5 ] );;|
  !gapprompt@gap>| !gapinput@RangeOfPartialPerm(f);|
  [ 7, 1, 4, 3, 2, 6, 5 ]
\end{Verbatim}
 }

 

\subsection{\textcolor{Chapter }{RangeSetOfPartialPerm}}
\logpage{[ 2, 5, 4 ]}\nobreak
\hyperdef{L}{X875D378578CC10EA}{}
{\noindent\textcolor{FuncColor}{$\triangleright$\ \ \texttt{RangeSetOfPartialPerm({\mdseries\slshape f})\index{RangeSetOfPartialPerm@\texttt{RangeSetOfPartialPerm}}
\label{RangeSetOfPartialPerm}
}\hfill{\scriptsize (operation)}}\\
\textbf{\indent Returns:\ }
A set of positive integers.



 The range of a partial permutation \mbox{\texttt{\mdseries\slshape f}} is the list of images of the elements of the domain \mbox{\texttt{\mdseries\slshape f}}. \texttt{RangeSetOfPartialPerm} returns the range of a partial permutation sorted into increasing order. 
\begin{Verbatim}[commandchars=!@|,fontsize=\small,frame=single,label=Example]
  !gapprompt@gap>| !gapinput@f:=PartialPermNC( [ 1, 2, 3, 5, 7, 10 ], [ 10, 2, 3, 5, 7, 6 ] );;|
  !gapprompt@gap>| !gapinput@RangeSetOfPartialPerm(f);|
  [ 2, 3, 5, 6, 7, 10 ]
\end{Verbatim}
 }

 

\subsection{\textcolor{Chapter }{FixedPointsOfPartialPerm}}
\logpage{[ 2, 5, 5 ]}\nobreak
\hyperdef{L}{X84DAE12780CE4D4A}{}
{\noindent\textcolor{FuncColor}{$\triangleright$\ \ \texttt{FixedPointsOfPartialPerm({\mdseries\slshape f})\index{FixedPointsOfPartialPerm@\texttt{FixedPointsOfPartialPerm}}
\label{FixedPointsOfPartialPerm}
}\hfill{\scriptsize (operation)}}\\
\textbf{\indent Returns:\ }
A set of positive integers.



 A positive integer \texttt{i} is fixed by a partial permutation \mbox{\texttt{\mdseries\slshape f}} if \texttt{i\texttt{\symbol{94}}}\mbox{\texttt{\mdseries\slshape f}}\texttt{=i}. \texttt{FixedPointsOfPartialPerm} returns the set of points fixed by the partial permutation \mbox{\texttt{\mdseries\slshape f}}. 
\begin{Verbatim}[commandchars=!@|,fontsize=\small,frame=single,label=Example]
  !gapprompt@gap>| !gapinput@f:=PartialPermNC( |
  !gapprompt@>| !gapinput@[ 1, 2, 3, 5, 6, 8, 10, 11, 14, 17, 20, 21, 22, 24, 25, 26, 32 ], |
  !gapprompt@>| !gapinput@[ 30, 20, 29, 24, 9, 14, 26, 5, 25, 15, 11, 6, 35, 2, 10, 19, 23 ] );;|
  !gapprompt@gap>| !gapinput@FixedPointsOfPartialPerm(f);|
  [  ]
  !gapprompt@gap>| !gapinput@f:=PartialPermNC([1..10]);|
  <identity on [ 1 .. 10 ]>
  !gapprompt@gap>| !gapinput@FixedPointsOfPartialPerm(f);|
  [ 1, 2, 3, 4, 5, 6, 7, 8, 9, 10 ]
\end{Verbatim}
 }

 

\subsection{\textcolor{Chapter }{MovedPoints (for a partial perm)}}
\logpage{[ 2, 5, 6 ]}\nobreak
\hyperdef{L}{X82FE981A87FAA2DC}{}
{\noindent\textcolor{FuncColor}{$\triangleright$\ \ \texttt{MovedPoints({\mdseries\slshape f})\index{MovedPoints@\texttt{MovedPoints}!for a partial perm}
\label{MovedPoints:for a partial perm}
}\hfill{\scriptsize (operation)}}\\
\noindent\textcolor{FuncColor}{$\triangleright$\ \ \texttt{MovedPoints({\mdseries\slshape C})\index{MovedPoints@\texttt{MovedPoints}!for a partial perm coll}
\label{MovedPoints:for a partial perm coll}
}\hfill{\scriptsize (operation)}}\\
\textbf{\indent Returns:\ }
A set of positive integers.



 A positive integer \texttt{i} is moved by a partial permutation \mbox{\texttt{\mdseries\slshape f}} if \texttt{i\texttt{\symbol{94}}}\mbox{\texttt{\mdseries\slshape f}}\texttt{{\textless}{\textgreater}i}. \texttt{MovedPoints} returns the set of points moved by: the partial permutation \mbox{\texttt{\mdseries\slshape f}} or at least one partial permutation in the collection \mbox{\texttt{\mdseries\slshape C}}, respectively.

 The operation \texttt{Points} (\ref{Points:for a partial perm coll}) returns a list of all the points on which the collection \mbox{\texttt{\mdseries\slshape C}} acts, not only those points that are moved. 
\begin{Verbatim}[commandchars=!@|,fontsize=\small,frame=single,label=Example]
  !gapprompt@gap>| !gapinput@f:=PartialPermNC( |
  !gapprompt@>| !gapinput@[ 1, 2, 3, 5, 6, 8, 10, 11, 14, 17, 20, 21, 22, 24, 25, 26, 32 ], |
  !gapprompt@>| !gapinput@[ 30, 20, 29, 24, 9, 14, 26, 5, 25, 15, 11, 6, 35, 2, 10, 19, 23 ] );;|
  !gapprompt@gap>| !gapinput@MovedPoints(f);|
  [ 1, 2, 3, 5, 6, 8, 10, 11, 14, 17, 20, 21, 22, 24, 25, 26, 32 ]
  !gapprompt@gap>| !gapinput@f:=PartialPermNC([1..10]);|
  <identity on [ 1 .. 10 ]>
  !gapprompt@gap>| !gapinput@MovedPoints(f);|
  [  ]
\end{Verbatim}
 }

 

\subsection{\textcolor{Chapter }{NrMovedPoints (for a partial perm)}}
\logpage{[ 2, 5, 7 ]}\nobreak
\hyperdef{L}{X81F5C64E7DAD27A7}{}
{\noindent\textcolor{FuncColor}{$\triangleright$\ \ \texttt{NrMovedPoints({\mdseries\slshape f})\index{NrMovedPoints@\texttt{NrMovedPoints}!for a partial perm}
\label{NrMovedPoints:for a partial perm}
}\hfill{\scriptsize (operation)}}\\
\noindent\textcolor{FuncColor}{$\triangleright$\ \ \texttt{NrMovedPoints({\mdseries\slshape C})\index{NrMovedPoints@\texttt{NrMovedPoints}!for a partial perm coll}
\label{NrMovedPoints:for a partial perm coll}
}\hfill{\scriptsize (operation)}}\\
\textbf{\indent Returns:\ }
A positive integer.



 A positive integer \texttt{i} is moved by a partial permutation \mbox{\texttt{\mdseries\slshape f}} if \texttt{i\texttt{\symbol{94}}}\mbox{\texttt{\mdseries\slshape f}}\texttt{{\textless}{\textgreater}i}. \texttt{NrMovedPoints} returns the number of points moved by: the partial permutation \mbox{\texttt{\mdseries\slshape f}} or the by at least one partial permutation in the collection \mbox{\texttt{\mdseries\slshape C}}, respectively. 
\begin{Verbatim}[commandchars=!@|,fontsize=\small,frame=single,label=Example]
  !gapprompt@gap>| !gapinput@f:=PartialPermNC( |
  !gapprompt@>| !gapinput@[ 1, 2, 3, 5, 6, 8, 10, 11, 14, 17, 20, 21, 22, 24, 25, 26, 32 ], |
  !gapprompt@>| !gapinput@[ 30, 20, 29, 24, 9, 14, 26, 5, 25, 15, 11, 6, 35, 2, 10, 19, 23 ] );;|
  !gapprompt@gap>| !gapinput@NrMovedPoints(f);|
  17
  !gapprompt@gap>| !gapinput@f:=PartialPermNC([1..10]);|
  <identity on [ 1 .. 10 ]>
  !gapprompt@gap>| !gapinput@NrMovedPoints(f);|
  0
\end{Verbatim}
 }

 

\subsection{\textcolor{Chapter }{LargestMovedPoint (for a partial perm)}}
\logpage{[ 2, 5, 8 ]}\nobreak
\hyperdef{L}{X7D4290A785ABC86D}{}
{\noindent\textcolor{FuncColor}{$\triangleright$\ \ \texttt{LargestMovedPoint({\mdseries\slshape f})\index{LargestMovedPoint@\texttt{LargestMovedPoint}!for a partial perm}
\label{LargestMovedPoint:for a partial perm}
}\hfill{\scriptsize (operation)}}\\
\noindent\textcolor{FuncColor}{$\triangleright$\ \ \texttt{LargestMovedPoint({\mdseries\slshape C})\index{LargestMovedPoint@\texttt{LargestMovedPoint}!for a partial perm coll}
\label{LargestMovedPoint:for a partial perm coll}
}\hfill{\scriptsize (operation)}}\\
\textbf{\indent Returns:\ }
A non-negative integer.



 A positive integer \texttt{i} is moved by a partial permutation \mbox{\texttt{\mdseries\slshape f}} if \texttt{i\texttt{\symbol{94}}}\mbox{\texttt{\mdseries\slshape f}}\texttt{{\textless}{\textgreater}i}. 

 
\begin{description}
\item[{For a partial permutation}]  \texttt{LargestMovedPoint} returns the largest point moved by the partial permutation \mbox{\texttt{\mdseries\slshape f}} or $0$ if no points are moved by \mbox{\texttt{\mdseries\slshape f}}. 
\item[{For a partial permutation collection}] \texttt{LargestMovedPoint} returns the largest point moved by at least one of the partial permutations in
the collection \mbox{\texttt{\mdseries\slshape C}} or $0$ if no points are moved by any element of \mbox{\texttt{\mdseries\slshape C}}. 
\end{description}
 
\begin{Verbatim}[commandchars=!@|,fontsize=\small,frame=single,label=Example]
  !gapprompt@gap>| !gapinput@f:=PartialPermNC( |
  !gapprompt@>| !gapinput@[ 1, 2, 3, 5, 6, 8, 10, 11, 14, 17, 20, 21, 22, 24, 25, 26, 32 ], |
  !gapprompt@>| !gapinput@[ 30, 20, 29, 24, 9, 14, 26, 5, 25, 15, 11, 6, 35, 2, 10, 19, 23 ] );;|
  !gapprompt@gap>| !gapinput@LargestMovedPoint(f);|
  32
  !gapprompt@gap>| !gapinput@f:=PartialPermNC([1..10]);|
  <identity on [ 1 .. 10 ]>
  !gapprompt@gap>| !gapinput@LargestMovedPoint(f);|
  0
\end{Verbatim}
 }

 

\subsection{\textcolor{Chapter }{SmallestMovedPoint (for a partial perm)}}
\logpage{[ 2, 5, 9 ]}\nobreak
\hyperdef{L}{X84A49C977E1E29AA}{}
{\noindent\textcolor{FuncColor}{$\triangleright$\ \ \texttt{SmallestMovedPoint({\mdseries\slshape f})\index{SmallestMovedPoint@\texttt{SmallestMovedPoint}!for a partial perm}
\label{SmallestMovedPoint:for a partial perm}
}\hfill{\scriptsize (operation)}}\\
\noindent\textcolor{FuncColor}{$\triangleright$\ \ \texttt{SmallestMovedPoint({\mdseries\slshape C})\index{SmallestMovedPoint@\texttt{SmallestMovedPoint}!for a partial perm coll}
\label{SmallestMovedPoint:for a partial perm coll}
}\hfill{\scriptsize (operation)}}\\
\textbf{\indent Returns:\ }
A non-negative integer.



 A positive integer \texttt{i} is moved by a partial permutation \mbox{\texttt{\mdseries\slshape f}} if \texttt{i\texttt{\symbol{94}}}\mbox{\texttt{\mdseries\slshape f}}\texttt{{\textless}{\textgreater}i}. 

 
\begin{description}
\item[{For a partial permutation}]  \texttt{SmallestMovedPoint} returns the smallest point moved by the partial permutation \mbox{\texttt{\mdseries\slshape f}} or $0$ if no points are moved by \mbox{\texttt{\mdseries\slshape f}}. 
\item[{For a partial permutation collection}] \texttt{SmallestMovedPoint} returns the smallest point moved by at least one of the partial permutations
in the collection \mbox{\texttt{\mdseries\slshape C}} or $0$ if no points are moved by any element of \mbox{\texttt{\mdseries\slshape C}}. 
\end{description}
 
\begin{Verbatim}[commandchars=!@|,fontsize=\small,frame=single,label=Example]
  !gapprompt@gap>| !gapinput@f:=PartialPermNC( |
  !gapprompt@>| !gapinput@[ 1, 2, 3, 5, 6, 8, 10, 11, 14, 17, 20, 21, 22, 24, 25, 26, 32 ], |
  !gapprompt@>| !gapinput@[ 30, 20, 29, 24, 9, 14, 26, 5, 25, 15, 11, 6, 35, 2, 10, 19, 23 ] );;|
  !gapprompt@gap>| !gapinput@SmallestMovedPoint(f);|
  1
  !gapprompt@gap>| !gapinput@f:=PartialPermNC([1..10]);|
  <identity on [ 1 .. 10 ]>
  !gapprompt@gap>| !gapinput@SmallestMovedPoint(f);|
  0
\end{Verbatim}
 }

 

\subsection{\textcolor{Chapter }{DenseRangeList}}
\logpage{[ 2, 5, 10 ]}\nobreak
\hyperdef{L}{X7BE1E841813E61A5}{}
{\noindent\textcolor{FuncColor}{$\triangleright$\ \ \texttt{DenseRangeList({\mdseries\slshape f})\index{DenseRangeList@\texttt{DenseRangeList}}
\label{DenseRangeList}
}\hfill{\scriptsize (function)}}\\
\textbf{\indent Returns:\ }
A list of positive integers.



 The \emph{dense range list} of a partial permutation \mbox{\texttt{\mdseries\slshape f}} is a list of positive integers \texttt{list} such that \texttt{list[i]=i\texttt{\symbol{94}}}\mbox{\texttt{\mdseries\slshape f}} where \texttt{i\texttt{\symbol{94}}}\mbox{\texttt{\mdseries\slshape f}}\texttt{=0} if \mbox{\texttt{\mdseries\slshape f}} is undefined on \texttt{i}. 
\begin{Verbatim}[commandchars=!@|,fontsize=\small,frame=single,label=Example]
  !gapprompt@gap>| !gapinput@f:=PartialPermNC( [ 1, 2, 3, 4, 8, 9, 10 ], [ 5, 8, 9, 7, 2, 6, 10 ] );;|
  !gapprompt@gap>| !gapinput@DenseRangeList(f);|
  [ 5, 8, 9, 7, 0, 0, 0, 2, 6, 10 ]
\end{Verbatim}
 }

 

\subsection{\textcolor{Chapter }{RankOfPartialPerm}}
\logpage{[ 2, 5, 11 ]}\nobreak
\hyperdef{L}{X7C1ABD8A80E95B39}{}
{\noindent\textcolor{FuncColor}{$\triangleright$\ \ \texttt{RankOfPartialPerm({\mdseries\slshape f})\index{RankOfPartialPerm@\texttt{RankOfPartialPerm}}
\label{RankOfPartialPerm}
}\hfill{\scriptsize (function)}}\\
\noindent\textcolor{FuncColor}{$\triangleright$\ \ \texttt{Rank({\mdseries\slshape f})\index{Rank@\texttt{Rank}!for a partial perm}
\label{Rank:for a partial perm}
}\hfill{\scriptsize (function)}}\\
\textbf{\indent Returns:\ }
A positive integer.



 The \emph{rank} of a partial permutation \mbox{\texttt{\mdseries\slshape f}} is the number of points in its domain (or equivalently its range). 
\begin{Verbatim}[commandchars=!@|,fontsize=\small,frame=single,label=Example]
  !gapprompt@gap>| !gapinput@f:=PartialPermNC( [ 1, 2, 3, 4, 5, 6 ], [ 10, 3, 9, 1, 5, 8 ] );;|
  !gapprompt@gap>| !gapinput@RankOfPartialPerm(f);|
  6
\end{Verbatim}
 }

 

\subsection{\textcolor{Chapter }{DegreeOfPartialPerm}}
\logpage{[ 2, 5, 12 ]}\nobreak
\hyperdef{L}{X8612A4DC864E7959}{}
{\noindent\textcolor{FuncColor}{$\triangleright$\ \ \texttt{DegreeOfPartialPerm({\mdseries\slshape f})\index{DegreeOfPartialPerm@\texttt{DegreeOfPartialPerm}}
\label{DegreeOfPartialPerm}
}\hfill{\scriptsize (function)}}\\
\noindent\textcolor{FuncColor}{$\triangleright$\ \ \texttt{Degree({\mdseries\slshape f})\index{Degree@\texttt{Degree}!for a partial perm}
\label{Degree:for a partial perm}
}\hfill{\scriptsize (function)}}\\
\noindent\textcolor{FuncColor}{$\triangleright$\ \ \texttt{Degree({\mdseries\slshape C})\index{Degree@\texttt{Degree}!for a partial perm coll}
\label{Degree:for a partial perm coll}
}\hfill{\scriptsize (function)}}\\
\textbf{\indent Returns:\ }
A positive integer.



 The \emph{degree} of a partial permutation \mbox{\texttt{\mdseries\slshape f}} is the largest positive integer in the union of its domain and range. 

 The degree of a partial permutation collection \mbox{\texttt{\mdseries\slshape C}} is the largest degree of any partial permutation in \mbox{\texttt{\mdseries\slshape C}}, i.e. the largest point in the domain or range of any element in \mbox{\texttt{\mdseries\slshape C}}. 
\begin{Verbatim}[commandchars=!@|,fontsize=\small,frame=single,label=Example]
  !gapprompt@gap>| !gapinput@f:=PartialPermNC( [ 1, 2, 3, 4, 6, 10 ], [ 1, 8, 2, 3, 4, 9 ] );;|
  !gapprompt@gap>| !gapinput@DegreeOfPartialPerm(f);|
  10
\end{Verbatim}
 }

 

\subsection{\textcolor{Chapter }{IndexPeriodOfPartialPerm}}
\logpage{[ 2, 5, 13 ]}\nobreak
\hyperdef{L}{X873A9F717DA75CBC}{}
{\noindent\textcolor{FuncColor}{$\triangleright$\ \ \texttt{IndexPeriodOfPartialPerm({\mdseries\slshape f})\index{IndexPeriodOfPartialPerm@\texttt{IndexPeriodOfPartialPerm}}
\label{IndexPeriodOfPartialPerm}
}\hfill{\scriptsize (function)}}\\
\textbf{\indent Returns:\ }
A pair of positive integers.



 Returns the least positive integers \texttt{m, r} such that \texttt{\mbox{\texttt{\mdseries\slshape f}}\texttt{\symbol{94}}(m+r)=\mbox{\texttt{\mdseries\slshape f}}\texttt{\symbol{94}}m}, which are known as the \emph{index} and \emph{period} of the partial permutation \mbox{\texttt{\mdseries\slshape f}}. 
\begin{Verbatim}[commandchars=!@|,fontsize=\small,frame=single,label=Example]
  !gapprompt@gap>| !gapinput@f:=PartialPermNC(|
  !gapprompt@>| !gapinput@[ 1, 2, 3, 4, 5, 6, 7, 8, 10, 11, 12, 16, 17, 18, 19 ], |
  !gapprompt@>| !gapinput@[ 3, 12, 14, 4, 11, 18, 17, 2, 9, 5, 15, 8, 20, 10, 19 ]);;|
  !gapprompt@gap>| !gapinput@IndexPeriodOfPartialPerm(f);                                    |
  [ 5, 2 ]
  !gapprompt@gap>| !gapinput@f^5=f^7;|
  true
\end{Verbatim}
 }

 

\subsection{\textcolor{Chapter }{SmallestIdempotentPower (for a partial perm)}}
\logpage{[ 2, 5, 14 ]}\nobreak
\hyperdef{L}{X7C04AA377F080722}{}
{\noindent\textcolor{FuncColor}{$\triangleright$\ \ \texttt{SmallestIdempotentPower({\mdseries\slshape f})\index{SmallestIdempotentPower@\texttt{SmallestIdempotentPower}!for a partial perm}
\label{SmallestIdempotentPower:for a partial perm}
}\hfill{\scriptsize (operation)}}\\
\textbf{\indent Returns:\ }
A positive integer.



 This function returns the least positive integer \texttt{n} such that the partial permutation \texttt{\mbox{\texttt{\mdseries\slshape f}}\texttt{\symbol{94}}n} is an idempotent. See also \texttt{SmallestIdempotentPower} (\ref{SmallestIdempotentPower:for a transformation}). 
\begin{Verbatim}[commandchars=!@|,fontsize=\small,frame=single,label=Example]
  !gapprompt@gap>| !gapinput@f:=PartialPermNC(|
  !gapprompt@>| !gapinput@[ 1, 2, 3, 4, 5, 6, 7, 8, 9, 10, 11, 12, 15, 16, 17, 23, 25, 29 ],|
  !gapprompt@>| !gapinput@[ 30, 8, 28, 21, 23, 15, 10, 14, 1, 18, 16, 5, 7, 26, 6, 9, 11, 19 ] );;|
  !gapprompt@gap>| !gapinput@SmallestIdempotentPower(f);|
  6
  !gapprompt@gap>| !gapinput@f^6;|
  <empty mapping>
\end{Verbatim}
 }

 }

 
\section{\textcolor{Chapter }{Operators for transformations}}\label{OperatorsT}
\logpage{[ 2, 6, 0 ]}
\hyperdef{L}{X812CEC008609A8A2}{}
{
  
\begin{description}
\item[{\texttt{\mbox{\texttt{\mdseries\slshape i}} \texttt{\symbol{94}} \mbox{\texttt{\mdseries\slshape f}}}}]  \index{^@\texttt{\texttt{\symbol{94}}} (for a pos int and a transformation)} returns the image of the positive integer \mbox{\texttt{\mdseries\slshape i}} under the transformation \mbox{\texttt{\mdseries\slshape f}} if \mbox{\texttt{\mdseries\slshape i}} is less than the degree of \mbox{\texttt{\mdseries\slshape f}}. 
\item[{\texttt{\mbox{\texttt{\mdseries\slshape f}} \texttt{\symbol{94}} \mbox{\texttt{\mdseries\slshape g}}}}] \index{^@\texttt{\texttt{\symbol{94}}} (for a transformation and a permutation)} returns \texttt{\mbox{\texttt{\mdseries\slshape g}}\texttt{\symbol{94}}-1*\mbox{\texttt{\mdseries\slshape f}}*\mbox{\texttt{\mdseries\slshape g}}} when \mbox{\texttt{\mdseries\slshape f}} is a transformation and \mbox{\texttt{\mdseries\slshape g}} is a permutation \texttt{\texttt{\symbol{92}}\texttt{\symbol{94}}} (\textbf{Reference: \texttt{\symbol{94}}}). 
\item[{\texttt{\mbox{\texttt{\mdseries\slshape f}} * \mbox{\texttt{\mdseries\slshape g}}}}]  \index{*@\texttt{*} (for transformations)} returns the composition of the transformations \mbox{\texttt{\mdseries\slshape f}} and \mbox{\texttt{\mdseries\slshape g}}. \textsf{Citrus} contains more efficient methods than the \textsf{GAP} library for \texttt{\texttt{\symbol{92}}*} (\textbf{Reference: *}) when \mbox{\texttt{\mdseries\slshape f}} and \mbox{\texttt{\mdseries\slshape g}} are transformations or permutations (and at least one of \mbox{\texttt{\mdseries\slshape f}} and \mbox{\texttt{\mdseries\slshape g}} is a transformation). 
\item[{\texttt{\mbox{\texttt{\mdseries\slshape f}} {\textless} \mbox{\texttt{\mdseries\slshape g}}}}]  \index{<@\texttt{{\textless}} (for transformations)} returns \texttt{true} if \texttt{ImageListOfTransformation(\mbox{\texttt{\mdseries\slshape f}})} is lexicographically less than \texttt{ImageListOfTransformation(\mbox{\texttt{\mdseries\slshape g}})} and \texttt{false} if it is not. See \texttt{ImageListOfTransformation} (\textbf{Reference: ImageListOfTransformation}). 
\item[{\texttt{\mbox{\texttt{\mdseries\slshape f}} = \mbox{\texttt{\mdseries\slshape g}}}}]  \index{=@\texttt{=} (for transformations)} returns \texttt{true} if the transformation \mbox{\texttt{\mdseries\slshape f}} equals the transformation \mbox{\texttt{\mdseries\slshape g}} and returns \texttt{false} if it does not. 
\end{description}
 }

 
\section{\textcolor{Chapter }{Operators for partial permutations}}\label{OperatorsPP}
\logpage{[ 2, 7, 0 ]}
\hyperdef{L}{X7FA29926872924D7}{}
{
  
\begin{description}
\item[{\texttt{\mbox{\texttt{\mdseries\slshape f}} \texttt{\symbol{94}} \mbox{\texttt{\mdseries\slshape -1}}}}]  \index{^@\texttt{\texttt{\symbol{94}}} (for a partial perm and negative int)} returns the inverse of the partial permutation \mbox{\texttt{\mdseries\slshape f}}. 
\item[{\texttt{\mbox{\texttt{\mdseries\slshape i}} \texttt{\symbol{94}} \mbox{\texttt{\mdseries\slshape f}}}}]  \index{^@\texttt{\texttt{\symbol{94}}} (for a pos int and a partial perm)} returns the image of the positive integer \mbox{\texttt{\mdseries\slshape i}} under the partial permutation \mbox{\texttt{\mdseries\slshape f}} if \mbox{\texttt{\mdseries\slshape f}} is defined on \mbox{\texttt{\mdseries\slshape i}} and \texttt{fail} if \mbox{\texttt{\mdseries\slshape f}} is undefined on \mbox{\texttt{\mdseries\slshape i}}. 
\item[{\texttt{\mbox{\texttt{\mdseries\slshape f}} * \mbox{\texttt{\mdseries\slshape g}}}}]  \index{*@\texttt{*} (for partial permutations)} returns the composition of the partial permutations \mbox{\texttt{\mdseries\slshape f}} and \mbox{\texttt{\mdseries\slshape g}}. Unlike transformations, but similar to permutations, in \textsf{GAP} it is possible to multiply any two partial permutations. It is also possible
to multiply partial permutations and permutations acting on integers not
larger than 65535; see \texttt{PartialPerm} (\ref{PartialPerm:for a domain and range}) for some further comments about composing partial permutations and
permutations. 
\item[{\texttt{\mbox{\texttt{\mdseries\slshape f}} {\textless} \mbox{\texttt{\mdseries\slshape g}}}}]  \index{<@\texttt{{\textless}} (for partial permutations)} returns \texttt{true} if \texttt{Concatenation(DomainOfPartialPerm(\mbox{\texttt{\mdseries\slshape f}}), RangeOfPartialPerm(\mbox{\texttt{\mdseries\slshape f}}))} is short-lex less than \texttt{Concatenation(DomainOfPartialPerm(\mbox{\texttt{\mdseries\slshape g}}), RangeOfPartialPerm(\mbox{\texttt{\mdseries\slshape g}}))} and \texttt{false} if it is not. Note that this is not the natural partial order on elements of
an inverse semigroup; see \texttt{NaturalLeqPartialPerm} (\ref{NaturalLeqPartialPerm}). 
\item[{\texttt{\mbox{\texttt{\mdseries\slshape f}} = \mbox{\texttt{\mdseries\slshape g}}}}]  \index{=@\texttt{=} (for partial permutations)} returns \texttt{true} if the partial permutation \mbox{\texttt{\mdseries\slshape f}} equals the partial permutation \mbox{\texttt{\mdseries\slshape g}} and returns \texttt{false} if it is not. 
\item[{\texttt{\mbox{\texttt{\mdseries\slshape f}} / \mbox{\texttt{\mdseries\slshape g}}}}]  \index{/@\texttt{/} (for partial permutations)} returns \texttt{\mbox{\texttt{\mdseries\slshape f}} * \mbox{\texttt{\mdseries\slshape g\texttt{\symbol{94}}-1}}} where \mbox{\texttt{\mdseries\slshape f}} and \mbox{\texttt{\mdseries\slshape g}} are partial permutations; \texttt{\mbox{\texttt{\mdseries\slshape f}} / \mbox{\texttt{\mdseries\slshape g}}} has better performance than \texttt{\mbox{\texttt{\mdseries\slshape f}} * \mbox{\texttt{\mdseries\slshape g\texttt{\symbol{94}}-1}}}. 
\end{description}
 

\subsection{\textcolor{Chapter }{NaturalLeqPartialPerm}}
\logpage{[ 2, 7, 1 ]}\nobreak
\hyperdef{L}{X87B1ED93785257C1}{}
{\noindent\textcolor{FuncColor}{$\triangleright$\ \ \texttt{NaturalLeqPartialPerm({\mdseries\slshape f, g})\index{NaturalLeqPartialPerm@\texttt{NaturalLeqPartialPerm}}
\label{NaturalLeqPartialPerm}
}\hfill{\scriptsize (function)}}\\
\textbf{\indent Returns:\ }
\texttt{true} or \texttt{false}.



 The \emph{natural partial order} $\leq$ on an inverse semigroup $S$ is defined by $s\leq$$t$ if there exists an idempotent $e$ in $S$ such that $s=et$. Hence if \mbox{\texttt{\mdseries\slshape f}} and \mbox{\texttt{\mdseries\slshape g}} are partial permutations, then \mbox{\texttt{\mdseries\slshape f}}$\leq$\mbox{\texttt{\mdseries\slshape g}} if and only if \mbox{\texttt{\mdseries\slshape f}} is a restriction of \mbox{\texttt{\mdseries\slshape g}}; see \texttt{RestrictedPartialPerm} (\ref{RestrictedPartialPerm}). \texttt{NaturalLeqPartialPerm} returns \texttt{true} if \mbox{\texttt{\mdseries\slshape f}} is a restriction of \mbox{\texttt{\mdseries\slshape g}} and \texttt{false} if it is not. 
\begin{Verbatim}[commandchars=!@|,fontsize=\small,frame=single,label=Example]
  !gapprompt@gap>| !gapinput@f:=PartialPermNC( [ 1, 2, 3, 4, 5, 6, 7, 8, 10, 11, 12, 16, 17, 18, 19 ], |
  !gapprompt@>| !gapinput@[ 3, 12, 14, 4, 11, 18, 17, 2, 9, 5, 15, 8, 20, 10, 19 ] );;|
  !gapprompt@gap>| !gapinput@set:=[ 1, 2, 3, 9, 13, 20 ];;|
  !gapprompt@gap>| !gapinput@g:=RestrictedPartialPerm(f, set);|
  [ 1 .. 3 ] -> [ 3, 12, 14 ]
  !gapprompt@gap>| !gapinput@NaturalLeqPartialPerm(g,f);|
  true
  !gapprompt@gap>| !gapinput@NaturalLeqPartialPerm(f,g);|
  false
  !gapprompt@gap>| !gapinput@g:=PartialPermNC( [ 1, 2, 3, 4, 5, 8, 10 ], [ 7, 1, 4, 3, 2, 6, 5 ] );;|
  !gapprompt@gap>| !gapinput@NaturalLeqPartialPerm(f, g);|
  false
  !gapprompt@gap>| !gapinput@NaturalLeqPartialPerm(g, f);|
  false
\end{Verbatim}
 }

 }

 
\section{\textcolor{Chapter }{Changing representations}}\label{Changing representation}
\logpage{[ 2, 8, 0 ]}
\hyperdef{L}{X83608C407CC8836D}{}
{
  It is possible to change the representation of certain transformations and
partial permutations using the functions described in this section. 

\subsection{\textcolor{Chapter }{AsPermOfKerImg}}
\logpage{[ 2, 8, 1 ]}\nobreak
\hyperdef{L}{X78AA076A7F1C2E90}{}
{\noindent\textcolor{FuncColor}{$\triangleright$\ \ \texttt{AsPermOfKerImg({\mdseries\slshape f})\index{AsPermOfKerImg@\texttt{AsPermOfKerImg}}
\label{AsPermOfKerImg}
}\hfill{\scriptsize (function)}}\\
\textbf{\indent Returns:\ }
A permutation.



 This function returns a permutation \texttt{p} such that 
\begin{Verbatim}[commandchars=!@|,fontsize=\small,frame=single,label=Example]
  OnTuples(CanonicalTransSameKernel(f), p)=ImageListOfTransformation(f);
\end{Verbatim}
 See also \texttt{CanonicalTransSameKernel} (\ref{CanonicalTransSameKernel}) and \texttt{ImageListOfTransformation} (\textbf{Reference: ImageListOfTransformation}). 
\begin{Verbatim}[commandchars=!@|,fontsize=\small,frame=single,label=Example]
  !gapprompt@gap>| !gapinput@f:=Transformation( [ 2, 1, 6, 1, 7, 6, 2, 8, 4, 7 ] );;|
  !gapprompt@gap>| !gapinput@CanonicalTransSameKernel(f); ImageListOfTransformation(f);|
  [ 1, 2, 3, 2, 4, 3, 1, 5, 6, 4 ]
  [ 2, 1, 6, 1, 7, 6, 2, 8, 4, 7 ]
  !gapprompt@gap>| !gapinput@AsPermOfKerImg(f);|
  (1,2)(3,6,4,7)(5,8)
\end{Verbatim}
 }

 

\subsection{\textcolor{Chapter }{AsPermutation}}
\logpage{[ 2, 8, 2 ]}\nobreak
\hyperdef{L}{X8353AB8987E35DF3}{}
{\noindent\textcolor{FuncColor}{$\triangleright$\ \ \texttt{AsPermutation({\mdseries\slshape f[, set]})\index{AsPermutation@\texttt{AsPermutation}}
\label{AsPermutation}
}\hfill{\scriptsize (operation)}}\\
\textbf{\indent Returns:\ }
A permutation.



 If the transformation or partial permutation \mbox{\texttt{\mdseries\slshape f}} is a permutation of the set of positive integers \mbox{\texttt{\mdseries\slshape set}}, then \texttt{AsPermutation} returns this permutation; see \texttt{Permutation} (\textbf{Reference: Permutation}).

 If the optional argument \mbox{\texttt{\mdseries\slshape set}} is not specified, then the image set of \mbox{\texttt{\mdseries\slshape f}} is used by default for transformations and the range of \mbox{\texttt{\mdseries\slshape f}} is used by default for partial permutations; see \texttt{ImageSetOfTransformation} (\textbf{Reference: ImageSetOfTransformation}) and \texttt{RangeSetOfPartialPerm} (\ref{RangeSetOfPartialPerm}). 
\begin{Verbatim}[commandchars=!@|,fontsize=\small,frame=single,label=Example]
  !gapprompt@gap>| !gapinput@f:=Transformation( [ 5, 8, 3, 5, 8, 6, 2, 2, 7, 8 ] );;|
  !gapprompt@gap>| !gapinput@AsPermutation(f);|
  fail
  !gapprompt@gap>| !gapinput@f:=Transformation( [ 8, 2, 10, 2, 4, 4, 7, 6, 9, 10 ] );;|
  !gapprompt@gap>| !gapinput@AsPermutation(f);|
  fail
  !gapprompt@gap>| !gapinput@f:=Transformation( [ 1, 3, 6, 6, 2, 10, 2, 3, 10, 5 ] );;|
  !gapprompt@gap>| !gapinput@AsPermutation(f);|
  (2,3,6,10,5)
  !gapprompt@gap>| !gapinput@f:=Transformation( [ 5, 2, 8, 4, 1, 8, 10, 3, 5, 7 ] );;|
  !gapprompt@gap>| !gapinput@AsPermutation(f);       |
  (1,5)(3,8)(7,10)
  !gapprompt@gap>| !gapinput@AsPermutation(f, [1,5]);|
  (1,5)
  !gapprompt@gap>| !gapinput@AsPermutation(f, [3,8,7,10]);|
  (3,8)(7,10)
  !gapprompt@gap>| !gapinput@f:=PartialPerm([ 1, 2, 3, 4, 5, 6, 7, 8, 10, 11, 12, 16, 17, 18, 19 ],|
  !gapprompt@>| !gapinput@[ 3, 12, 14, 4, 11, 18, 17, 2, 9, 5, 15, 8, 20, 10, 19 ]);;|
  !gapprompt@gap>| !gapinput@AsPermutation(f);|
  fail
  !gapprompt@gap>| !gapinput@AsPermutation(f, [4,5,11]);|
  (5,11)
  !gapprompt@gap>| !gapinput@f:=RandomPartialPerm(20);;|
  !gapprompt@gap>| !gapinput@AsPermutation(f^-1*f);|
  ()
\end{Verbatim}
 }

 
\subsection{\textcolor{Chapter }{AsPartialPerm}}\logpage{[ 2, 8, 3 ]}
\hyperdef{L}{X7B4CDC0579C6887F}{}
{
\noindent\textcolor{FuncColor}{$\triangleright$\ \ \texttt{AsPartialPerm({\mdseries\slshape f[, set]})\index{AsPartialPerm@\texttt{AsPartialPerm}}
\label{AsPartialPerm}
}\hfill{\scriptsize (function)}}\\
\noindent\textcolor{FuncColor}{$\triangleright$\ \ \texttt{AsPartialPermNC({\mdseries\slshape f[, set]})\index{AsPartialPermNC@\texttt{AsPartialPermNC}}
\label{AsPartialPermNC}
}\hfill{\scriptsize (function)}}\\
\textbf{\indent Returns:\ }
A partial permutation.



 
\begin{description}
\item[{For permutations:}] \texttt{AsPartialPerm} and \texttt{AsPartialPermNC} return the partial permutation that equals \mbox{\texttt{\mdseries\slshape f}} on the set of positive integers \mbox{\texttt{\mdseries\slshape set}} and is undefined elsewhere. If the optional second argument \mbox{\texttt{\mdseries\slshape set}} is not specfied, then \texttt{MovedPoints(f)} is used; see \texttt{MovedPoints} (\textbf{Reference: MovedPoints (for a permutation)}).

 If the optional second argument \mbox{\texttt{\mdseries\slshape set}} is specified, then \texttt{AsPartialPerm} first checks that \mbox{\texttt{\mdseries\slshape set}} is a set of positive integers and then calls \texttt{AsPartialPermNC}. If the optional second argument is not specified, then \texttt{AsPartialPerm} simply calls \texttt{AsPartialPermNC} and no further checks are performed. 

 Note that as explained in \texttt{PartialPerm} (\ref{PartialPerm:for a domain and range}) \emph{a permutation is never a partial permutation} in \textsf{GAP}, please keep this in mind when using \texttt{AsPartialPerm}. 
\item[{For transformations:}]  If \mbox{\texttt{\mdseries\slshape f}} is a transformation such that \texttt{Degree(\mbox{\texttt{\mdseries\slshape f}})\texttt{\symbol{94}}\mbox{\texttt{\mdseries\slshape f}}=Degree(\mbox{\texttt{\mdseries\slshape f}})} and \mbox{\texttt{\mdseries\slshape f}} is injective except on those \texttt{i} such that \texttt{i\texttt{\symbol{94}}\mbox{\texttt{\mdseries\slshape f}}=Degree(f)}, then \texttt{AsPartialPerm} returns the corresponding partial permutation.

 \texttt{AsPartialPerm} first verifies that \mbox{\texttt{\mdseries\slshape f}} has the required form, whereas \texttt{AsPartialPermNC} does not. There is currently no method for \texttt{AsPartialPerm} or \texttt{AsPartialPermNC} with two arguments. 
\end{description}
 
\begin{Verbatim}[commandchars=!@|,fontsize=\small,frame=single,label=Example]
  !gapprompt@gap>| !gapinput@f:=(2,8,19,9,14,10,20,17,4,13,12,3,5,7,18,16);;|
  !gapprompt@gap>| !gapinput@AsPartialPerm(f);|
  [ 2, 3, 4, 5, 7, 8, 9, 10, 12, 13, 14, 16, 17, 18, 19, 20 ] -> 
  [ 8, 5, 13, 7, 18, 19, 14, 20, 3, 12, 10, 2, 4, 16, 9, 17 ]
  !gapprompt@gap>| !gapinput@AsPartialPerm(f, [1,2,3]);|
  [ 1 .. 3 ] -> [ 1, 8, 5 ]
  !gapprompt@gap>| !gapinput@f:=Transformation( [ 8, 3, 5, 9, 6, 2, 9, 7, 9 ] );;|
  !gapprompt@gap>| !gapinput@AsPartialPerm(f);|
  [ 1, 2, 3, 5, 6, 8 ] -> [ 8, 3, 5, 6, 2, 7 ]
  !gapprompt@gap>| !gapinput@AsPartialPermNC(f);|
  [ 1, 2, 3, 5, 6, 8 ] -> [ 8, 3, 5, 6, 2, 7 ]
  !gapprompt@gap>| !gapinput@f:=Transformation( [ 2, 10, 2, 4, 4, 7, 6, 9, 10, 1 ] );;|
  !gapprompt@gap>| !gapinput@AsPartialPerm(f);|
  fail
\end{Verbatim}
 }

 

\subsection{\textcolor{Chapter }{AsTransformation}}
\logpage{[ 2, 8, 4 ]}\nobreak
\hyperdef{L}{X7C5360B2799943F3}{}
{\noindent\textcolor{FuncColor}{$\triangleright$\ \ \texttt{AsTransformation({\mdseries\slshape f[, n]})\index{AsTransformation@\texttt{AsTransformation}}
\label{AsTransformation}
}\hfill{\scriptsize (operation)}}\\
\textbf{\indent Returns:\ }
A transformation.



 If \mbox{\texttt{\mdseries\slshape f}} is a partial permutation and \texttt{m} is the largest point in the union of the domain and range of \mbox{\texttt{\mdseries\slshape f}}, then \texttt{AsTransformation} returns the transformation \texttt{g} such that \texttt{i\texttt{\symbol{94}}g=i\texttt{\symbol{94}}f} for all \texttt{i} in the domain of \mbox{\texttt{\mdseries\slshape f}} and \texttt{i\texttt{\symbol{94}}g=m+1} for all \texttt{i} in $\{1,\ldots, m+1\}$ that is not in the domain of \mbox{\texttt{\mdseries\slshape f}}. 

 If the optional second argument \mbox{\texttt{\mdseries\slshape n}} is a positive integer greater than the largest point in the union of the
domain and range of \mbox{\texttt{\mdseries\slshape f}}, then the transformation obtained is defined by replacing \texttt{m} in the definition of \texttt{g} in previous paragraph by \texttt{n}. 

 It is also possible to use \texttt{AsTransformation} to convert permutations and binary relations into transformations; see \texttt{AsTransformation} (\textbf{Reference: AsTransformation}). 
\begin{Verbatim}[commandchars=!@|,fontsize=\small,frame=single,label=Example]
  !gapprompt@gap>| !gapinput@f:=PartialPerm([ 1, 2, 3, 4, 5, 8, 10 ], [ 3, 1, 4, 2, 5, 6, 7 ]);;|
  !gapprompt@gap>| !gapinput@AsTransformation(f);|
  Transformation( [ 3, 1, 4, 2, 5, 11, 11, 6, 11, 7, 11 ] )
  !gapprompt@gap>| !gapinput@AsTransformation(f, 12);|
  Transformation( [ 3, 1, 4, 2, 5, 12, 12, 6, 12, 7, 12, 12 ] )
  !gapprompt@gap>| !gapinput@AsTransformation(f, 14);|
  Transformation( [ 3, 1, 4, 2, 5, 14, 14, 6, 14, 7, 14, 14, 14, 14 ] )
\end{Verbatim}
 }

 }

 
\section{\textcolor{Chapter }{Actions}}\logpage{[ 2, 9, 0 ]}
\hyperdef{L}{X833C5DA683E4EA15}{}
{
 The following actions of transformations and partial permutations are used by \textsf{Citrus} in the computation of Green's relations and to test if an arbitrary
transformation semigroup has a particular property; see Chapter \ref{green} and  (\textbf{Reference: Basic Actions}). 

 

\subsection{\textcolor{Chapter }{OnKernelsAntiAction}}
\logpage{[ 2, 9, 1 ]}\nobreak
\hyperdef{L}{X849A43DE7AF3C639}{}
{\noindent\textcolor{FuncColor}{$\triangleright$\ \ \texttt{OnKernelsAntiAction({\mdseries\slshape ker, f})\index{OnKernelsAntiAction@\texttt{OnKernelsAntiAction}}
\label{OnKernelsAntiAction}
}\hfill{\scriptsize (function)}}\\
\textbf{\indent Returns:\ }
 A list of positive integers. 



 The argument \mbox{\texttt{\mdseries\slshape ker}} should equal \texttt{CanonicalTransSameKernel(g);} for some transformation \texttt{g} with degree equal to the degree of the transformation \mbox{\texttt{\mdseries\slshape f}}. \texttt{OnKernelsAntiAction} returns \texttt{CanonicalTransSameKernel(f*g)}. See also \texttt{CanonicalTransSameKernel} (\ref{CanonicalTransSameKernel}). 
\begin{Verbatim}[commandchars=!@|,fontsize=\small,frame=single,label=Example]
  !gapprompt@gap>| !gapinput@f:=Transformation( [ 3, 6, 9, 2, 4, 2, 2, 2, 8, 2 ] );;|
  !gapprompt@gap>| !gapinput@g:=Transformation( [ 7, 1, 4, 3, 2, 7, 7, 6, 6, 5 ] );;|
  !gapprompt@gap>| !gapinput@ker:=CanonicalTransSameKernel(f);|
  [ 1, 2, 3, 4, 5, 4, 4, 4, 6, 4 ]
  !gapprompt@gap>| !gapinput@OnKernelsAntiAction(ker, g);|
  [ 1, 2, 1, 3, 4, 1, 1, 1, 1, 5 ]
  !gapprompt@gap>| !gapinput@CanonicalTransSameKernel(g*f);|
  [ 1, 2, 1, 3, 4, 1, 1, 1, 1, 5 ]
\end{Verbatim}
 }

 

\subsection{\textcolor{Chapter }{OnIntegerSetsWithPartialPerm}}
\logpage{[ 2, 9, 2 ]}\nobreak
\hyperdef{L}{X7BDF1956855D75CC}{}
{\noindent\textcolor{FuncColor}{$\triangleright$\ \ \texttt{OnIntegerSetsWithPartialPerm({\mdseries\slshape set, f})\index{OnIntegerSetsWithPartialPerm@\texttt{OnIntegerSetsWithPartialPerm}}
\label{OnIntegerSetsWithPartialPerm}
}\hfill{\scriptsize (operation)}}\\
\textbf{\indent Returns:\ }
A set of positive integers.



 \texttt{OnIntegerSetsWithPartialPerm} is a special method for \texttt{OnSets} (\textbf{Reference: OnSets}) for a set of positive integers \mbox{\texttt{\mdseries\slshape set}} and a partial permutation \mbox{\texttt{\mdseries\slshape f}}. 
\begin{Verbatim}[commandchars=!@|,fontsize=\small,frame=single,label=Example]
  !gapprompt@gap>| !gapinput@f:=PartialPermNC([ 1, 2, 4, 5, 6, 8, 9, 10, 11, 15, 16, 17, 18 ],|
  !gapprompt@>| !gapinput@[ 13, 20, 2, 14, 18, 7, 3, 19, 9, 11, 5, 16, 8 ]);;|
  !gapprompt@gap>| !gapinput@OnSets([1,2,3], f);|
  [ 13, 20, fail ]
  !gapprompt@gap>| !gapinput@OnIntegerSetsWithPartialPerm([1,2,3], f);|
  [ 13, 20 ]
\end{Verbatim}
 }

 

\subsection{\textcolor{Chapter }{TransformationActionNC}}
\logpage{[ 2, 9, 3 ]}\nobreak
\hyperdef{L}{X814B3E6E7D3F1036}{}
{\noindent\textcolor{FuncColor}{$\triangleright$\ \ \texttt{TransformationActionNC({\mdseries\slshape obj, list, func})\index{TransformationActionNC@\texttt{TransformationActionNC}}
\label{TransformationActionNC}
}\hfill{\scriptsize (operation)}}\\
\textbf{\indent Returns:\ }
A transformation or collection of transformations.



 If \mbox{\texttt{\mdseries\slshape obj}} is a \textsf{GAP} object that acts on the list \mbox{\texttt{\mdseries\slshape list}} via the function \mbox{\texttt{\mdseries\slshape func}}, then \texttt{TransformationActionNC} returns this action as a transformation; see also \texttt{Transformation} (\textbf{Reference: Transformation}). Mathematically, the argument \mbox{\texttt{\mdseries\slshape obj}} should be an element of a semigroup so that the action of \mbox{\texttt{\mdseries\slshape obj}} on \mbox{\texttt{\mdseries\slshape list}} via \mbox{\texttt{\mdseries\slshape func}} is a \textsc{semigroup action}. However, it is not technically necessary for \mbox{\texttt{\mdseries\slshape obj}} to be an element of a semigroup in \textsf{GAP}. 

 If \mbox{\texttt{\mdseries\slshape obj}} is a semigroup, then \texttt{TransformationActionNC} returns the list obtained from applying \texttt{TransformationActionNC} to every generator of \mbox{\texttt{\mdseries\slshape obj}}. 
\begin{Verbatim}[commandchars=!@|,fontsize=\small,frame=single,label=Example]
  !gapprompt@gap>| !gapinput@mat:=OneMutable(GeneratorsOfGroup(GL(3,3))[1]);|
  [ [ Z(3)^0, 0*Z(3), 0*Z(3) ], [ 0*Z(3), Z(3)^0, 0*Z(3) ],
    [ 0*Z(3), 0*Z(3), Z(3)^0 ] ]
  !gapprompt@gap>| !gapinput@mat[3][3]:=Z(3)*0;|
  0*Z(3)
  !gapprompt@gap>| !gapinput@F:=BaseDomain(mat);|
  GF(3)
  !gapprompt@gap>| !gapinput@TransformationActionNC(mat, Elements(F^3), OnRight);|
  Transformation( [ 1, 1, 1, 4, 4, 4, 7, 7, 7, 10, 10, 10, 13, 13, 13, 16, 16,
    16, 19, 19, 19, 22, 22, 22, 25, 25, 25 ] )
\end{Verbatim}
 }

 }

 
\section{\textcolor{Chapter }{Orbits}}\logpage{[ 2, 10, 0 ]}
\hyperdef{L}{X81E0FF0587C54543}{}
{
 The following functions supplement the \href{ http://www-groups.mcs.st-and.ac.uk/~neunhoef/Computer/Software/Gap/orb.html } {Orb} package by providing methods for computations related to the strongly
connected components of an orbit of a semigroup. 

 Please note that if any of the functions in this section is applied to an open
orbit, then the orbit is enumerated before anything else. It is not possible
to calculate the strongly connected components of an orbit of a semigroup
acting on a set until the entire orbit has been found. 

\subsection{\textcolor{Chapter }{OrbSCC}}
\logpage{[ 2, 10, 1 ]}\nobreak
\hyperdef{L}{X8178A420792E6AAC}{}
{\noindent\textcolor{FuncColor}{$\triangleright$\ \ \texttt{OrbSCC({\mdseries\slshape o})\index{OrbSCC@\texttt{OrbSCC}}
\label{OrbSCC}
}\hfill{\scriptsize (function)}}\\
\textbf{\indent Returns:\ }
The strongly connected components of an orbit.



 If \mbox{\texttt{\mdseries\slshape o}} is an orbit created by the \textsf{Orb} package with the option \texttt{orbitgraph=true}, then \texttt{OrbSCC} returns a set of sets of positions in \mbox{\texttt{\mdseries\slshape o}} corresponding to its strongly connected components. 

 See also \texttt{OrbSCCLookup} (\ref{OrbSCCLookup}) and \texttt{OrbSCCTruthTable} (\ref{OrbSCCTruthTable}). 
\begin{Verbatim}[commandchars=!@|,fontsize=\small,frame=single,label=Example]
  !gapprompt@gap>| !gapinput@S:=FullTransformationSemigroup(4);;|
  !gapprompt@gap>| !gapinput@o:=ImagesOfTransSemigroup(S);;|
  !gapprompt@gap>| !gapinput@OrbSCC(o);|
  [ [ 1 ], [ 2, 3, 4, 5 ], [ 6, 7, 8, 9, 10, 11 ], [ 12, 13, 14, 15 ] ]
\end{Verbatim}
 }

 

\subsection{\textcolor{Chapter }{OrbSCCLookup}}
\logpage{[ 2, 10, 2 ]}\nobreak
\hyperdef{L}{X814337A47B773F50}{}
{\noindent\textcolor{FuncColor}{$\triangleright$\ \ \texttt{OrbSCCLookup({\mdseries\slshape o})\index{OrbSCCLookup@\texttt{OrbSCCLookup}}
\label{OrbSCCLookup}
}\hfill{\scriptsize (function)}}\\
\textbf{\indent Returns:\ }
A lookup table for the strongly connected components of an orbit. 



 If \mbox{\texttt{\mdseries\slshape o}} is an orbit created by the \textsf{Orb} package with the option \texttt{orbitgraph=true}, then \texttt{OrbSCCLookup} returns a lookup table for its strongly connected components. More precisely, \texttt{OrbSCCLookup(o)[i]} equals the index of the strongly connected component containing \texttt{o[i]}. 

 See also \texttt{OrbSCC} (\ref{OrbSCC}) and \texttt{OrbSCCTruthTable} (\ref{OrbSCCTruthTable}). 
\begin{Verbatim}[commandchars=!@|,fontsize=\small,frame=single,label=Example]
  !gapprompt@gap>| !gapinput@S:=FullTransformationSemigroup(4);;|
  !gapprompt@gap>| !gapinput@o:=ImagesOfTransSemigroup(S);;|
  !gapprompt@gap>| !gapinput@OrbSCC(o);|
  [ [ 1 ], [ 2, 3, 4, 5 ], [ 6, 7, 8, 9, 10, 11 ], [ 12, 13, 14, 15 ] ]
  !gapprompt@gap>| !gapinput@OrbSCCLookup(o);|
  [ 1, 2, 2, 2, 2, 3, 3, 3, 3, 3, 3, 4, 4, 4, 4 ]
  !gapprompt@gap>| !gapinput@OrbSCCLookup(o)[1]; OrbSCCLookup(o)[4]; OrbSCCLookup(o)[7]; |
  1
  2
  3
\end{Verbatim}
 }

 

\subsection{\textcolor{Chapter }{OrbSCCTruthTable}}
\logpage{[ 2, 10, 3 ]}\nobreak
\hyperdef{L}{X78AFD003840823BD}{}
{\noindent\textcolor{FuncColor}{$\triangleright$\ \ \texttt{OrbSCCTruthTable({\mdseries\slshape o})\index{OrbSCCTruthTable@\texttt{OrbSCCTruthTable}}
\label{OrbSCCTruthTable}
}\hfill{\scriptsize (function)}}\\
\textbf{\indent Returns:\ }
Truth tables for strongly connected components of an orbit. 



 If \mbox{\texttt{\mdseries\slshape o}} is an orbit created by the \textsf{Orb} package with the option \texttt{orbitgraph=true}, then \texttt{OrbSCCTruthTable} returns a list of boolean lists such that \texttt{OrbSCCTruthTable(o)[i][j]} is \texttt{true} if \texttt{j} belongs to \texttt{OrbSCC(o)[i]}.

 See also \texttt{OrbSCC} (\ref{OrbSCC}) and \texttt{OrbSCCLookup} (\ref{OrbSCCLookup}). 
\begin{Verbatim}[commandchars=!@|,fontsize=\small,frame=single,label=Example]
  !gapprompt@gap>| !gapinput@S:=FullTransformationSemigroup(4);;|
  !gapprompt@gap>| !gapinput@o:=ImagesOfTransSemigroup(S);;|
  !gapprompt@gap>| !gapinput@OrbSCC(o);|
  [ [ 1 ], [ 2, 3, 4, 5 ], [ 6, 7, 8, 9, 10, 11 ], [ 12, 13, 14, 15 ] ]
  !gapprompt@gap>| !gapinput@OrbSCCTruthTable(o);|
  [ [ true, false, false, false, false, false, false, false, false, false, 
        false, false, false, false, false ], 
    [ false, true, true, true, true, false, false, false, false, false, false, 
        false, false, false, false ], 
    [ false, false, false, false, false, true, true, true, true, true, true, 
        false, false, false, false ], 
    [ false, false, false, false, false, false, false, false, false, false, 
        false, true, true, true, true ] ]
\end{Verbatim}
 }

 

\subsection{\textcolor{Chapter }{ReverseSchreierTreeOfSCC}}
\logpage{[ 2, 10, 4 ]}\nobreak
\hyperdef{L}{X7D9A29B47D743213}{}
{\noindent\textcolor{FuncColor}{$\triangleright$\ \ \texttt{ReverseSchreierTreeOfSCC({\mdseries\slshape o, i})\index{ReverseSchreierTreeOfSCC@\texttt{ReverseSchreierTreeOfSCC}}
\label{ReverseSchreierTreeOfSCC}
}\hfill{\scriptsize (function)}}\\
\textbf{\indent Returns:\ }
The reverse Schreier tree corresponding to the \mbox{\texttt{\mdseries\slshape i}}th strongly connected component of an orbit. 



 If \mbox{\texttt{\mdseries\slshape o}} is an orbit created by the \textsf{Orb} package with the option \texttt{orbitgraph=true} and action \texttt{act}, and \mbox{\texttt{\mdseries\slshape i}} is a positive integer, then \texttt{ReverseSchreierTreeOfSCC(\mbox{\texttt{\mdseries\slshape o}}, \mbox{\texttt{\mdseries\slshape i}})} returns a pair \texttt{[ gen, pos ]} of lists with \texttt{Length(o)} entries such that 
\begin{Verbatim}[commandchars=@|A,fontsize=\small,frame=single,label=Example]
  act(o[j], o!.gens[gen[j]])=o[pos[j]].
\end{Verbatim}
 The pair \texttt{[ gen, pos ]} corresponds to a tree with root \texttt{OrbSCC(o)[i][1]} and a path from every element of \texttt{OrbSCC(o)[i]} to the root. 

 See also \texttt{OrbSCC} (\ref{OrbSCC}), \texttt{TraceSchreierTreeOfSCCBack} (\ref{TraceSchreierTreeOfSCCBack}), \texttt{SchreierTreeOfSCC} (\ref{SchreierTreeOfSCC}), and \texttt{TraceSchreierTreeOfSCCForward} (\ref{TraceSchreierTreeOfSCCForward}). 
\begin{Verbatim}[commandchars=!@|,fontsize=\small,frame=single,label=Example]
  !gapprompt@gap>| !gapinput@S:=Semigroup(Transformation( [ 2, 2, 1, 4, 4 ] ), |
  !gapprompt@>| !gapinput@Transformation( [ 3, 3, 3, 4, 5 ] ),|
  !gapprompt@>| !gapinput@Transformation( [ 5, 1, 4, 5, 5 ] ) );;|
  !gapprompt@gap>| !gapinput@o:=Orb(S, [1..4], OnSets, rec(orbitgraph:=true, schreier:=true));;|
  !gapprompt@gap>| !gapinput@OrbSCC(o);|
  [ [ 1 ], [ 2 ], [ 3, 5, 6, 7, 11 ], [ 4 ], [ 8 ], [ 9 ], [ 10, 12 ] ]
  !gapprompt@gap>| !gapinput@ReverseSchreierTreeOfSCC(o, 3);|
  [ [ fail, fail, fail, fail, 2, 1, 2, fail, fail, fail, 1, fail ], 
    [ fail, fail, fail, fail, 3, 5, 3, fail, fail, fail, 7, fail ] ]
  !gapprompt@gap>| !gapinput@ReverseSchreierTreeOfSCC(o, 7);|
  [ [ fail, fail, fail, fail, fail, fail, fail, fail, fail, fail, fail, 3 ], 
    [ fail, fail, fail, fail, fail, fail, fail, fail, fail, fail, fail, 10 ] ]
  !gapprompt@gap>| !gapinput@OnSets(o[11], Generators(S)[1]);|
  [ 1, 4 ]
  !gapprompt@gap>| !gapinput@Position(o, last);|
  7
\end{Verbatim}
 }

 

\subsection{\textcolor{Chapter }{SchreierTreeOfSCC}}
\logpage{[ 2, 10, 5 ]}\nobreak
\hyperdef{L}{X8071C7148255D0DB}{}
{\noindent\textcolor{FuncColor}{$\triangleright$\ \ \texttt{SchreierTreeOfSCC({\mdseries\slshape o, i})\index{SchreierTreeOfSCC@\texttt{SchreierTreeOfSCC}}
\label{SchreierTreeOfSCC}
}\hfill{\scriptsize (function)}}\\
\textbf{\indent Returns:\ }
The Schreier tree corresponding to the \mbox{\texttt{\mdseries\slshape i}}th strongly connected component of an orbit. 



 If \mbox{\texttt{\mdseries\slshape o}} is an orbit created by the \textsf{Orb} package with the option \texttt{orbitgraph=true} and action \texttt{act}, and \mbox{\texttt{\mdseries\slshape i}} is a positive integer, then \texttt{SchreierTreeOfSCC(\mbox{\texttt{\mdseries\slshape o}}, \mbox{\texttt{\mdseries\slshape i}})} returns a pair \texttt{[ gen, pos ]} of lists with \texttt{Length(o)} entries such that 
\begin{Verbatim}[commandchars=@|A,fontsize=\small,frame=single,label=Example]
  act(o[pos[j]], o!.gens[gen[j]])=o[j].
\end{Verbatim}
 The pair \texttt{[ gen, pos ]} corresponds to a tree with root \texttt{OrbSCC(o)[i][1]} and a path from the root to every element of \texttt{OrbSCC(o)[i]}. 

 See also \texttt{OrbSCC} (\ref{OrbSCC}), \texttt{TraceSchreierTreeOfSCCBack} (\ref{TraceSchreierTreeOfSCCBack}), \texttt{ReverseSchreierTreeOfSCC} (\ref{ReverseSchreierTreeOfSCC}), and \texttt{TraceSchreierTreeOfSCCForward} (\ref{TraceSchreierTreeOfSCCForward}). 
\begin{Verbatim}[commandchars=!@|,fontsize=\small,frame=single,label=Example]
  !gapprompt@gap>| !gapinput@S:=Semigroup(Transformation( [ 2, 2, 1, 4, 4 ] ), |
  !gapprompt@>| !gapinput@Transformation( [ 3, 3, 3, 4, 5 ] ),|
  !gapprompt@>| !gapinput@Transformation( [ 5, 1, 4, 5, 5 ] ) );;|
  !gapprompt@gap>| !gapinput@o:=Orb(S, [1..4], OnSets, rec(orbitgraph:=true, schreier:=true));;|
  !gapprompt@gap>| !gapinput@OrbSCC(o);|
  [ [ 1 ], [ 2 ], [ 3, 5, 6, 7, 11 ], [ 4 ], [ 8 ], [ 9 ], [ 10, 12 ] ]
  !gapprompt@gap>| !gapinput@SchreierTreeOfSCC(o, 3);|
  [ [ fail, fail, fail, fail, 1, 3, 1, fail, fail, fail, 2, fail ], 
    [ fail, fail, fail, fail, 7, 5, 3, fail, fail, fail, 6, fail ] ]
  !gapprompt@gap>| !gapinput@SchreierTreeOfSCC(o, 7);|
  [ [ fail, fail, fail, fail, fail, fail, fail, fail, fail, fail, fail, 1 ], 
    [ fail, fail, fail, fail, fail, fail, fail, fail, fail, fail, fail, 10 ] ]
  !gapprompt@gap>| !gapinput@OnSets(o[6], Generators(S)[2]);|
  [ 3, 5 ]
  !gapprompt@gap>| !gapinput@Position(o, last);|
  11
\end{Verbatim}
 }

 

\subsection{\textcolor{Chapter }{TraceSchreierTreeOfSCCBack}}
\logpage{[ 2, 10, 6 ]}\nobreak
\hyperdef{L}{X7853DC817C3102A4}{}
{\noindent\textcolor{FuncColor}{$\triangleright$\ \ \texttt{TraceSchreierTreeOfSCCBack({\mdseries\slshape orb, m, nr})\index{TraceSchreierTreeOfSCCBack@\texttt{TraceSchreierTreeOfSCCBack}}
\label{TraceSchreierTreeOfSCCBack}
}\hfill{\scriptsize (function)}}\\
\textbf{\indent Returns:\ }
A word in the generators.



 \mbox{\texttt{\mdseries\slshape orb}} must be an orbit object with a Schreier tree and orbit graph, that is, the
options \texttt{schreier} and \texttt{orbitgraph} must have been set to \texttt{true} during the creation of the orbit, \mbox{\texttt{\mdseries\slshape m}} must be the number of a strongly connected component of \mbox{\texttt{\mdseries\slshape orb}}, and \texttt{nr} must be the number of a point in the \mbox{\texttt{\mdseries\slshape m}}th strongly connect component of \mbox{\texttt{\mdseries\slshape orb}}. This operation traces the result of \texttt{ReverseSchreierTreeOfSCC} (\ref{ReverseSchreierTreeOfSCC}) and with arguments \mbox{\texttt{\mdseries\slshape orb}} and \mbox{\texttt{\mdseries\slshape m}} and returns a word in the generators that maps the point with number \mbox{\texttt{\mdseries\slshape nr}} to the first point in the \mbox{\texttt{\mdseries\slshape m}}th strongly connected component of \mbox{\texttt{\mdseries\slshape orb}}. Here, a word is a list of integers, where positive integers are numbers of
generators. See also \texttt{OrbSCC} (\ref{OrbSCC}),\texttt{EvaluateWord} (\ref{EvaluateWord}), \texttt{ReverseSchreierTreeOfSCC} (\ref{ReverseSchreierTreeOfSCC}), \texttt{SchreierTreeOfSCC} (\ref{SchreierTreeOfSCC}), and \texttt{TraceSchreierTreeOfSCCForward} (\ref{TraceSchreierTreeOfSCCForward}). 
\begin{Verbatim}[commandchars=!@|,fontsize=\small,frame=single,label=Example]
  !gapprompt@gap>| !gapinput@S:=Semigroup(Transformation( [ 1, 3, 4, 1 ] ), |
  !gapprompt@>| !gapinput@Transformation( [ 2, 4, 1, 2 ] ),|
  !gapprompt@>| !gapinput@Transformation( [ 3, 1, 1, 3 ] ), |
  !gapprompt@>| !gapinput@Transformation( [ 3, 3, 4, 1 ] ) );;|
  !gapprompt@gap>| !gapinput@o:=Orb(S, [1..4], OnSets, rec(orbitgraph:=true, schreier:=true));;|
  !gapprompt@gap>| !gapinput@OrbSCC(o);|
  [ [ 1 ], [ 2 ], [ 3 ], [ 4, 5, 6, 7, 8 ], [ 9, 10, 11, 12 ] ]
  !gapprompt@gap>| !gapinput@ReverseSchreierTreeOfSCC(o, 4);               |
  [ [ fail, fail, fail, fail, 4, 1, 1, 3, fail, fail, fail, fail ], 
    [ fail, fail, fail, fail, 4, 4, 4, 4, fail, fail, fail, fail ] ]
  !gapprompt@gap>| !gapinput@TraceSchreierTreeOfSCCBack(o, 4, 7);|
  [ 1 ]
  !gapprompt@gap>| !gapinput@TraceSchreierTreeOfSCCBack(o, 4, 8);|
  [ 3 ]
\end{Verbatim}
 }

 

\subsection{\textcolor{Chapter }{TraceSchreierTreeOfSCCForward}}
\logpage{[ 2, 10, 7 ]}\nobreak
\hyperdef{L}{X7D2E200A7B2D5946}{}
{\noindent\textcolor{FuncColor}{$\triangleright$\ \ \texttt{TraceSchreierTreeOfSCCForward({\mdseries\slshape orb, m, nr})\index{TraceSchreierTreeOfSCCForward@\texttt{TraceSchreierTreeOfSCCForward}}
\label{TraceSchreierTreeOfSCCForward}
}\hfill{\scriptsize (function)}}\\
\textbf{\indent Returns:\ }
A word in the generators.



 \mbox{\texttt{\mdseries\slshape orb}} must be an orbit object with a Schreier tree and orbit graph, that is, the
options \texttt{schreier} and \texttt{orbitgraph} must have been set to \texttt{true} during the creation of the orbit, \mbox{\texttt{\mdseries\slshape m}} must be the number of a strongly connected component of \mbox{\texttt{\mdseries\slshape orb}}, and \texttt{nr} must be the number of a point in the \mbox{\texttt{\mdseries\slshape m}}th strongly connect component of \mbox{\texttt{\mdseries\slshape orb}}. This operation traces the result of \texttt{SchreierTreeOfSCC} (\ref{SchreierTreeOfSCC}) and with arguments \mbox{\texttt{\mdseries\slshape orb}} and \mbox{\texttt{\mdseries\slshape m}} and returns a word in the generators that maps the first point in the \mbox{\texttt{\mdseries\slshape m}}th strongly connected component of \mbox{\texttt{\mdseries\slshape orb}} to the point with number \mbox{\texttt{\mdseries\slshape nr}}. Here, a word is a list of integers, where positive integers are numbers of
generators. See also \texttt{OrbSCC} (\ref{OrbSCC}), \texttt{EvaluateWord} (\ref{EvaluateWord}), \texttt{ReverseSchreierTreeOfSCC} (\ref{ReverseSchreierTreeOfSCC}), \texttt{SchreierTreeOfSCC} (\ref{SchreierTreeOfSCC}), and \texttt{TraceSchreierTreeOfSCCBack} (\ref{TraceSchreierTreeOfSCCBack}). 
\begin{Verbatim}[commandchars=!@|,fontsize=\small,frame=single,label=Example]
  !gapprompt@gap>| !gapinput@S:=Semigroup(Transformation( [ 1, 3, 4, 1 ] ), |
  !gapprompt@>| !gapinput@Transformation( [ 2, 4, 1, 2 ] ),|
  !gapprompt@>| !gapinput@Transformation( [ 3, 1, 1, 3 ] ), |
  !gapprompt@>| !gapinput@Transformation( [ 3, 3, 4, 1 ] ) );;|
  !gapprompt@gap>| !gapinput@o:=Orb(S, [1..4], OnSets, rec(orbitgraph:=true, schreier:=true));;|
  !gapprompt@gap>| !gapinput@OrbSCC(o);|
  [ [ 1 ], [ 2 ], [ 3 ], [ 4, 5, 6, 7, 8 ], [ 9, 10, 11, 12 ] ]
  !gapprompt@gap>| !gapinput@SchreierTreeOfSCC(o, 4);|
  [ [ fail, fail, fail, fail, 1, 2, 2, 4, fail, fail, fail, fail ], 
    [ fail, fail, fail, fail, 4, 4, 6, 4, fail, fail, fail, fail ] ]
  !gapprompt@gap>| !gapinput@TraceSchreierTreeOfSCCForward(o, 4, 8);|
  [ 4 ]
  !gapprompt@gap>| !gapinput@TraceSchreierTreeOfSCCForward(o, 4, 7);|
  [ 2, 2 ]
\end{Verbatim}
 }

 }

 }

 
\chapter{\textcolor{Chapter }{Creating semigroups and monoids}}\label{create}
\logpage{[ 3, 0, 0 ]}
\hyperdef{L}{X79A4C070831D989D}{}
{
 In this chapter we describe the various ways that semigroups and monoids can
be created in \textsf{Citrus}, the options that are available at the time of creation, and describe some
standard examples availabl in \textsf{Citrus}. 

 Any transformation semigroup created before \textsf{Citrus} has been loaded must be recreated after \textsf{Citrus} is loaded so that the options record (described in Section \ref{opts}) is defined. Almost all of the functions and methods provided by \textsf{Citrus}, including those methods for existing \textsf{GAP} library functions, will return an error when applied to a transformation
semigroup created before \textsf{Citrus} is loaded. 
\section{\textcolor{Chapter }{Semigroups defined by a generating set}}\logpage{[ 3, 1, 0 ]}
\hyperdef{L}{X8208D90078983892}{}
{
 In this section we give details of how to create semigroups and monoids from a
set of generators. Much of what is described here is syntactic sugar. 

\subsection{\textcolor{Chapter }{InverseMonoid}}
\logpage{[ 3, 1, 1 ]}\nobreak
\hyperdef{L}{X80D9B9A98736051B}{}
{\noindent\textcolor{FuncColor}{$\triangleright$\ \ \texttt{InverseMonoid({\mdseries\slshape obj1, obj2, ...[, opts]})\index{InverseMonoid@\texttt{InverseMonoid}}
\label{InverseMonoid}
}\hfill{\scriptsize (function)}}\\
\noindent\textcolor{FuncColor}{$\triangleright$\ \ \texttt{InverseSemigroup({\mdseries\slshape obj1, obj2, ...[, opts]})\index{InverseSemigroup@\texttt{InverseSemigroup}}
\label{InverseSemigroup}
}\hfill{\scriptsize (function)}}\\
\textbf{\indent Returns:\ }
An inverse semigroup or monoid.



 If \mbox{\texttt{\mdseries\slshape obj1}}, \mbox{\texttt{\mdseries\slshape obj2}}, ... are (any combination) of partial permutations, partial permutation
semigroup, or lists of partial permutations, then \texttt{InverseMonoid} or \texttt{InverseSemigroup} returns the inverse monoid or semigroup generated by the union of \mbox{\texttt{\mdseries\slshape obj1}}, \mbox{\texttt{\mdseries\slshape obj2}}, ... which equals the semigroup or monoid generated by the union of \mbox{\texttt{\mdseries\slshape obj1}}, \mbox{\texttt{\mdseries\slshape obj2}}, ... and their inverses.

 If present, the optional final argument \mbox{\texttt{\mdseries\slshape opts}} should be a record containing the values of the options for the inverse
semigroup being created, as described in Section \ref{opts}.

 As an example of how the syntax provided by \textsf{Citrus} can be convenient: \texttt{U:=Semigroup(S, f, Idempotents(T));}, in the example below, returns the same value as: \texttt{U:=Semigroup(Concatenation(Generators(S), [f], Idempotents(T)));}. 
\begin{Verbatim}[commandchars=!@|,fontsize=\small,frame=single,label=Example]
  !gapprompt@gap>| !gapinput@S:=InverseSemigroup(|
  !gapprompt@>| !gapinput@PartialPermNC( [ 1, 2, 3, 4, 6, 8, 9, 10, 11, 13, 14, 16, 17, 18, 20 ], |
  !gapprompt@>| !gapinput@[ 2, 14, 5, 8, 11, 12, 16, 17, 18, 9, 13, 15, 20, 6, 4 ] ),|
  !gapprompt@>| !gapinput@PartialPermNC( [ 1, 2, 3, 4, 5, 6, 7, 8, 9, 12, 13, 14, 15, 17 ], |
  !gapprompt@>| !gapinput@[ 2, 14, 8, 19, 5, 1, 3, 16, 6, 9, 10, 17, 12, 20 ] ));;|
  !gapprompt@gap>| !gapinput@f:=PartialPermNC( |
  !gapprompt@>| !gapinput@[ 1, 2, 3, 4, 5, 6, 7, 8, 9, 10, 11, 12, 13, 14, 15, 16, 17, |
  !gapprompt@>| !gapinput@ 18, 20, 22, 23, 25, 26, 30, 31, 32, 33, 34, 39, 40, 42, 43, 45 ], |
  !gapprompt@>| !gapinput@[ 4, 47, 34, 19, 32, 22, 12, 15, 16, 45, 49, 8, 24, 40, 17, |
  !gapprompt@>| !gapinput@ 46, 14, 6, 44, 2, 48, 41, 10, 31, 18, 50, 23, 5, 37, 11, 38, 30, 21 ] );;|
  !gapprompt@gap>| !gapinput@S:=InverseSemigroup(S, f, Idempotents(SymmetricInverseSemigp(10)));|
  <inverse semigroup with 1027 generators>
  !gapprompt@gap>| !gapinput@S:=InverseSemigroup(S, f, Idempotents(SymmetricInverseSemigp(10)),|
  !gapprompt@>| !gapinput@rec(small:=true));|
  <inverse semigroup with 13 generators>
  !gapprompt@gap>| !gapinput@Size(S);|
  17147970
  !gapprompt@gap>| !gapinput@S:=InverseMonoid(Generators(S));|
  <inverse monoid with 13 generators>
  !gapprompt@gap>| !gapinput@Size(S);|
  17147971
\end{Verbatim}
 }

 

\subsection{\textcolor{Chapter }{InverseMonoidByGenerators}}
\logpage{[ 3, 1, 2 ]}\nobreak
\hyperdef{L}{X79A15C7C83BBA60B}{}
{\noindent\textcolor{FuncColor}{$\triangleright$\ \ \texttt{InverseMonoidByGenerators({\mdseries\slshape coll[, opts]})\index{InverseMonoidByGenerators@\texttt{InverseMonoidByGenerators}}
\label{InverseMonoidByGenerators}
}\hfill{\scriptsize (function)}}\\
\noindent\textcolor{FuncColor}{$\triangleright$\ \ \texttt{InverseSemigroupByGenerators({\mdseries\slshape coll[, opts]})\index{InverseSemigroupByGenerators@\texttt{InverseSemigroupByGenerators}}
\label{InverseSemigroupByGenerators}
}\hfill{\scriptsize (function)}}\\
\textbf{\indent Returns:\ }
An inverse monoid or semigroup.



 If \mbox{\texttt{\mdseries\slshape coll}} is a partial permutation collection, then \texttt{InverseMonoidByGenerators} and \texttt{InverseSemigroupByGenerators} return the inverse monoid and semigroup generated by \mbox{\texttt{\mdseries\slshape coll}}, respectively. 

 If present, the optional second argument \mbox{\texttt{\mdseries\slshape opts}} should be a record containing the values of the options for the semigroup
being created, as described in Section \ref{opts}.

 }

 

\subsection{\textcolor{Chapter }{Monoid}}
\logpage{[ 3, 1, 3 ]}\nobreak
\hyperdef{L}{X7F95328B7C7E49EA}{}
{\noindent\textcolor{FuncColor}{$\triangleright$\ \ \texttt{Monoid({\mdseries\slshape obj1, obj2, ...[, opts]})\index{Monoid@\texttt{Monoid}}
\label{Monoid}
}\hfill{\scriptsize (function)}}\\
\noindent\textcolor{FuncColor}{$\triangleright$\ \ \texttt{Semigroup({\mdseries\slshape obj1, obj2, ...[, opts]})\index{Semigroup@\texttt{Semigroup}}
\label{Semigroup}
}\hfill{\scriptsize (function)}}\\
\textbf{\indent Returns:\ }
A monoid or semigroup.



 If \mbox{\texttt{\mdseries\slshape obj1}}, \mbox{\texttt{\mdseries\slshape obj2}}, ... are (any combination) of transformations, transformation semigroups, or
lists of transformations, then \texttt{Monoid} or \texttt{Semigroup} returns the monoid or semigroup generated by the union of \mbox{\texttt{\mdseries\slshape obj1}}, \mbox{\texttt{\mdseries\slshape obj2}}, .... 

 If present, the optional final argument \mbox{\texttt{\mdseries\slshape opts}} should be a record containing the values of the options for the semigroup
being created, as described in Section \ref{opts}.

 When applied to arguments other than transformations or transformation
collections these functions behave precisely as described in the reference
manual. Please consult \texttt{Monoid} (\textbf{Reference: Monoid}) and \texttt{Semigroup} (\textbf{Reference: Semigroup}) for further details.

 As an example of how the syntax provided by \textsf{Citrus} can be convenient: \texttt{U:=Semigroup(S, f, Idempotents(T));}, in the example below, returns the same value as: \texttt{U:=Semigroup(Concatenation(Generators(S), [f], Idempotents(T)));}. 
\begin{Verbatim}[commandchars=!@|,fontsize=\small,frame=single,label=Example]
  !gapprompt@gap>| !gapinput@S:=Semigroup(Transformation( [ 1, 3, 4, 1, 3, 5 ] ), |
  !gapprompt@>| !gapinput@Transformation( [ 5, 1, 6, 1, 6, 3 ] ) );;|
  !gapprompt@gap>| !gapinput@f:=Transformation( [ 2, 4, 6, 1, 6, 5 ] );;|
  !gapprompt@gap>| !gapinput@T:=Monoid(Transformation( [ 4, 1, 2, 6, 2, 1 ] ),|
  !gapprompt@>| !gapinput@Transformation( [ 5, 2, 5, 3, 5, 3 ] ) );;|
  !gapprompt@gap>| !gapinput@U:=Semigroup(S, f, Idempotents(T));|
  <semigroup with 14 generators>
  !gapprompt@gap>| !gapinput@Size(U);|
  2182
  !gapprompt@gap>| !gapinput@NrRClasses(U);|
  53
\end{Verbatim}
 }

 

\subsection{\textcolor{Chapter }{MonoidByGenerators}}
\logpage{[ 3, 1, 4 ]}\nobreak
\hyperdef{L}{X85129EE387CC4D28}{}
{\noindent\textcolor{FuncColor}{$\triangleright$\ \ \texttt{MonoidByGenerators({\mdseries\slshape coll[, opts]})\index{MonoidByGenerators@\texttt{MonoidByGenerators}}
\label{MonoidByGenerators}
}\hfill{\scriptsize (method)}}\\
\noindent\textcolor{FuncColor}{$\triangleright$\ \ \texttt{SemigroupByGenerators({\mdseries\slshape coll[, opts]})\index{SemigroupByGenerators@\texttt{SemigroupByGenerators}}
\label{SemigroupByGenerators}
}\hfill{\scriptsize (method)}}\\
\textbf{\indent Returns:\ }
A monoid or semigroup.



 If \mbox{\texttt{\mdseries\slshape coll}} is a transformation collection, then \texttt{MonoidByGenerators} and \texttt{SemigroupByGenerators} return the monoid and semigroup generated by \mbox{\texttt{\mdseries\slshape coll}}, respectively. 

 If present, the optional second argument \mbox{\texttt{\mdseries\slshape opts}} should be a record containing the values of the options for the semigroup
being created, as described in Section \ref{opts}.

 When applied to arguments other than transformations or transformation
collections these functions behave precisely as described in the reference
manual. Please consult \texttt{MonoidByGenerators} (\textbf{Reference: MonoidByGenerators}) and \texttt{SemigroupByGenerators} (\textbf{Reference: SemigroupByGenerators}) for further details. }

 

\subsection{\textcolor{Chapter }{RandomInverseMonoid}}
\logpage{[ 3, 1, 5 ]}\nobreak
\hyperdef{L}{X7B341D6C7CECFB55}{}
{\noindent\textcolor{FuncColor}{$\triangleright$\ \ \texttt{RandomInverseMonoid({\mdseries\slshape m, n})\index{RandomInverseMonoid@\texttt{RandomInverseMonoid}}
\label{RandomInverseMonoid}
}\hfill{\scriptsize (function)}}\\
\noindent\textcolor{FuncColor}{$\triangleright$\ \ \texttt{RandomInverseSemigroup({\mdseries\slshape m, n})\index{RandomInverseSemigroup@\texttt{RandomInverseSemigroup}}
\label{RandomInverseSemigroup}
}\hfill{\scriptsize (function)}}\\
\textbf{\indent Returns:\ }
An inverse monoid or semigroup.



 Returns a random inverse monoid or semigroup of partial permutations with
degree at most \mbox{\texttt{\mdseries\slshape n}} with \mbox{\texttt{\mdseries\slshape m}} generators. 
\begin{Verbatim}[commandchars=!@|,fontsize=\small,frame=single,label=Example]
  !gapprompt@gap>| !gapinput@S:=RandomInverseSemigroup(10,10);                                |
  <inverse semigroup with 10 generators>
  !gapprompt@gap>| !gapinput@S:=RandomInverseMonoid(10,10);   |
  <inverse monoid with 10 generators>
\end{Verbatim}
 }

 

\subsection{\textcolor{Chapter }{RandomTransformationMonoid}}
\logpage{[ 3, 1, 6 ]}\nobreak
\hyperdef{L}{X79834BC080B011B4}{}
{\noindent\textcolor{FuncColor}{$\triangleright$\ \ \texttt{RandomTransformationMonoid({\mdseries\slshape m, n})\index{RandomTransformationMonoid@\texttt{RandomTransformationMonoid}}
\label{RandomTransformationMonoid}
}\hfill{\scriptsize (function)}}\\
\noindent\textcolor{FuncColor}{$\triangleright$\ \ \texttt{RandomTransformationSemigroup({\mdseries\slshape m, n})\index{RandomTransformationSemigroup@\texttt{RandomTransformationSemigroup}}
\label{RandomTransformationSemigroup}
}\hfill{\scriptsize (function)}}\\
\textbf{\indent Returns:\ }
A transformation semigroup.



 Returns a random transformation monoid or semigroup of degree \mbox{\texttt{\mdseries\slshape n}} with \mbox{\texttt{\mdseries\slshape m}} generators. 
\begin{Verbatim}[commandchars=!@|,fontsize=\small,frame=single,label=Example]
  !gapprompt@gap>| !gapinput@S:=RandomTransformationMonoid(5,5);|
  <monoid with 5 generators>
  !gapprompt@gap>| !gapinput@S:=RandomTransformationSemigroup(5,5);|
  <semigroup with 5 generators>
\end{Verbatim}
 }

 }

 
\section{\textcolor{Chapter }{New semigroups from old}}\logpage{[ 3, 2, 0 ]}
\hyperdef{L}{X7A5CFD4F8607CBF7}{}
{
 

\subsection{\textcolor{Chapter }{ClosureInverseSemigroup}}
\logpage{[ 3, 2, 1 ]}\nobreak
\hyperdef{L}{X78A488637BBEF7AD}{}
{\noindent\textcolor{FuncColor}{$\triangleright$\ \ \texttt{ClosureInverseSemigroup({\mdseries\slshape S, coll[, opts]})\index{ClosureInverseSemigroup@\texttt{ClosureInverseSemigroup}}
\label{ClosureInverseSemigroup}
}\hfill{\scriptsize (function)}}\\
\textbf{\indent Returns:\ }
An inverse semigroup or monoid.



 This function returns the inverse semigroup or monoid generated by the inverse
semigroup of partial permutations \mbox{\texttt{\mdseries\slshape S}} and the partial permutation collection or partial permutation \mbox{\texttt{\mdseries\slshape coll}} after first removing duplicates and partial permutations in \mbox{\texttt{\mdseries\slshape coll}} that are already in \mbox{\texttt{\mdseries\slshape S}}. In most cases, the new semigroups knows at least as much information about
its structure as was already known about that of \mbox{\texttt{\mdseries\slshape S}}. 

 If present, the optional third argument \mbox{\texttt{\mdseries\slshape opts}} should be a record contiaing the values of the options for the inverse
semigroup being created; these options are described in Section \ref{opts}. 

 Unlike \texttt{ClosureSemigroup} (\ref{ClosureSemigroup}), \texttt{ClosureInverseSemigroup} is always at least as efficient as (and often much more than) simply creating
an inverse semigroup from scratch using \texttt{InverseSemigroup} (\ref{InverseSemigroup}). 
\begin{Verbatim}[commandchars=!@|,fontsize=\small,frame=single,label=Example]
  !gapprompt@gap>| !gapinput@S:=InverseMonoid(|
  !gapprompt@>| !gapinput@PartialPermNC( [ 1, 2, 3, 5, 6, 7, 8 ], [ 5, 9, 10, 6, 3, 8, 4 ] ),|
  !gapprompt@>| !gapinput@PartialPermNC( [ 1, 2, 4, 7, 8, 9 ], [ 10, 7, 8, 5, 9, 1 ] ) );;|
  !gapprompt@gap>| !gapinput@f:=PartialPermNC(|
  !gapprompt@>| !gapinput@[ 1, 2, 3, 4, 5, 7, 8, 10, 11, 13, 18, 19, 20 ],|
  !gapprompt@>| !gapinput@[ 5, 1, 7, 3, 10, 2, 12, 14, 11, 16, 6, 9, 15 ]);;|
  !gapprompt@gap>| !gapinput@S:=ClosureInverseSemigroup(S, f);|
  <inverse semigroup with 3 generators>
  !gapprompt@gap>| !gapinput@Size(S);|
  9726
  !gapprompt@gap>| !gapinput@T:=Idempotents(SymmetricInverseSemigp(10));;|
  !gapprompt@gap>| !gapinput@S:=ClosureInverseSemigroup(S, T);|
  <inverse semigroup with 858 generators>
  !gapprompt@gap>| !gapinput@S:=InverseSemigroup(SmallGeneratingSet(S));|
  <inverse semigroup with 14 generators>
\end{Verbatim}
 }

 

\subsection{\textcolor{Chapter }{ClosureSemigroup}}
\logpage{[ 3, 2, 2 ]}\nobreak
\hyperdef{L}{X7BE36790862AE26F}{}
{\noindent\textcolor{FuncColor}{$\triangleright$\ \ \texttt{ClosureSemigroup({\mdseries\slshape S, coll[, opts]})\index{ClosureSemigroup@\texttt{ClosureSemigroup}}
\label{ClosureSemigroup}
}\hfill{\scriptsize (function)}}\\
\textbf{\indent Returns:\ }
A transformation semigroup or monoid.



 This function returns the semigroup or monoid generated by the transformation
semigroup \mbox{\texttt{\mdseries\slshape S}} and the transformation collection or transformation \mbox{\texttt{\mdseries\slshape coll}} after removing duplicates and transformations in \mbox{\texttt{\mdseries\slshape coll}} that are already in \mbox{\texttt{\mdseries\slshape S}}. In some cases, the new semigroup knows at least as much information about
its structure as was already known about that of \mbox{\texttt{\mdseries\slshape S}}. 

 If present, the optional third argument \mbox{\texttt{\mdseries\slshape opts}} should be a record containing the values of the options for the semigroup
being created as described in Section \ref{opts}. For technical reasons if the option \texttt{schreier} is \texttt{false}, then \texttt{ClosureSemigroup(S, coll)} does exactly the same as \texttt{Semigroup} (\ref{Semigroup}) or \texttt{Monoid} (\ref{Monoid}), in that the new semigroup does not know any information about its structure
even if the old semigroup had been completely determined. 

 More specifically, if the option \texttt{schreier} is \texttt{false}, then the new semigroup knows the Green's class of any representative of a
Green's classes of \mbox{\texttt{\mdseries\slshape S}} that was known at the time that \texttt{ClosureSemigroup} was called. Consequently, if, for example, the number of $\mathcal{R}$-classes of \mbox{\texttt{\mdseries\slshape S}} is greater than the number of elements required to find the $\mathcal{R}$-classes of the new semigroup, then \texttt{ClosureSemigroup} might be less efficient than creating the new semigroup using the command \texttt{Semigroup} (\ref{Semigroup}) or \texttt{Monoid} (\ref{Monoid}). It is unlikely that you will be able to determine which of \texttt{ClosureSemigroup} and \texttt{Semigroup} (\ref{Semigroup}) will be better before calling these functions, except in the following case.
If the rank of the transformations in \mbox{\texttt{\mdseries\slshape coll}} are lower than the ranks of the Green's classes containing the majority of the
known elements of \mbox{\texttt{\mdseries\slshape S}}, then \texttt{ClosureSemigroup} should be superior to \texttt{Semigroup} (\ref{Semigroup}). 
\begin{Verbatim}[commandchars=!@|,fontsize=\small,frame=single,label=Example]
  !gapprompt@gap>| !gapinput@gens:=[ Transformation( [ 2, 6, 7, 2, 6, 1, 1, 5 ] ), |
  !gapprompt@>| !gapinput@ Transformation( [ 3, 8, 1, 4, 5, 6, 7, 1 ] ), |
  !gapprompt@>| !gapinput@ Transformation( [ 4, 3, 2, 7, 7, 6, 6, 5 ] ), |
  !gapprompt@>| !gapinput@ Transformation( [ 7, 1, 7, 4, 2, 5, 6, 3 ] ) ];;|
  !gapprompt@gap>| !gapinput@S:=Monoid(gens[1], rec(schreier:=false));;|
  !gapprompt@gap>| !gapinput@for i in [2..4] do S:=ClosureSemigroup(S, gens[i]); od;|
  !gapprompt@gap>| !gapinput@S;|
  <monoid with 4 generators>
  !gapprompt@gap>| !gapinput@Size(S);|
  233606
\end{Verbatim}
 }

 

\subsection{\textcolor{Chapter }{SubsemigroupByProperty (for a semigroup and
    function)}}
\logpage{[ 3, 2, 3 ]}\nobreak
\hyperdef{L}{X7E5B4C5A82F9E0E0}{}
{\noindent\textcolor{FuncColor}{$\triangleright$\ \ \texttt{SubsemigroupByProperty({\mdseries\slshape S, func})\index{SubsemigroupByProperty@\texttt{SubsemigroupByProperty}!for a semigroup and
    function}
\label{SubsemigroupByProperty:for a semigroup and
    function}
}\hfill{\scriptsize (operation)}}\\
\noindent\textcolor{FuncColor}{$\triangleright$\ \ \texttt{SubsemigroupByProperty({\mdseries\slshape S, func, limit})\index{SubsemigroupByProperty@\texttt{SubsemigroupByProperty}!for a
    semigroup, function, and limit on the size of the subsemigroup}
\label{SubsemigroupByProperty:for a
    semigroup, function, and limit on the size of the subsemigroup}
}\hfill{\scriptsize (operation)}}\\
\textbf{\indent Returns:\ }
A semigroup.



 \texttt{SubsemigroupByProperty} creates a subsemigroup of the semigroup \mbox{\texttt{\mdseries\slshape S}} of transformations or partial permutations consisting of those elements
fulfilling \mbox{\texttt{\mdseries\slshape func}} (which should be a function returning \texttt{true} or \texttt{false}). No test is done to check if the property actually defines a subsemigroup. 

 If the optional third argument \mbox{\texttt{\mdseries\slshape limit}} is present and a positive integer, then once the subsemigroup has at least \mbox{\texttt{\mdseries\slshape limit}} elements the computation stops. }

 }

 
\section{\textcolor{Chapter }{Options when creating semigroups}}\label{opts}
\logpage{[ 3, 3, 0 ]}
\hyperdef{L}{X799EBA2F819D8867}{}
{
 When using any of \texttt{InverseSemigroup} (\ref{InverseSemigroup}), \texttt{InverseMonoid} (\ref{InverseMonoid}), \texttt{Semigroup} (\ref{Semigroup}), \texttt{Monoid} (\ref{Monoid}), \texttt{SemigroupByGenerators} (\ref{SemigroupByGenerators}), \texttt{MonoidByGenerators} (\ref{MonoidByGenerators}), \texttt{ClosureInverseSemigroup} (\ref{ClosureInverseSemigroup}) or \texttt{ClosureSemigroup} (\ref{ClosureSemigroup}) a record can be given as an optional final argument. The components of this
record specify the values of certain options for the semigroup being created.
A list of these options and their default values is given below. 

 Assume that \mbox{\texttt{\mdseries\slshape S}} is the semigroup created by one of the functions given above and that \mbox{\texttt{\mdseries\slshape S}} is generated by the list of transformations \mbox{\texttt{\mdseries\slshape gens}}. 
\begin{description}
\item[{\texttt{hashlen}}]  this component should be a positive integer, which roughly specifies the
lengths of the hash tables used internally by \textsf{Citrus}. \textsf{Citrus} uses hash tables in several fundamental methods. The lengths of these tables
are a compromise between performance and memory usage; larger tables provide
better performance for large computations but use more memory. Note that it is
unlikely that you will need to specify this option unless you find that \textsf{GAP} runs out of memory unexpectedly or that the performance of \textsf{Citrus} is poorer than expected. If you find that \textsf{GAP} runs out of memory unexpectedly, or you plan to do a large number of
computations with relatively small semigroups (say with tens of thousands of
elements), then you might consider setting \texttt{hashlen} to be less than the default value of \texttt{25013} for each of these semigroups. If you find that the performance of \textsf{Citrus} is unexpectedly poor, or you plan to do a computation with a very large
semigroup (say, more than 10 million elements), then you might consider
setting \texttt{hashlen} to be greater than the default value of \texttt{25013}. 

 You might find it useful to set the info level of the info class \texttt{InfoOrb} to 2 or higher since this will indicate when hash tables used by \textsf{Citrus} are being grown; see \texttt{SetInfoLevel} (\textbf{Reference: SetInfoLevel}). 
\item[{\texttt{schreier}}]  if this component is set to \texttt{true}, then \textsf{Citrus} will keep track of various pieces of information so that it is possible to
factorize elements of \mbox{\texttt{\mdseries\slshape S}} using \texttt{Factorization} (\ref{Factorization}). This will cause a slight decrease in performance and an increase in memory
usage. If this component is set to \texttt{false}, then it will not be possible to factorize the elements of \mbox{\texttt{\mdseries\slshape S}} using \texttt{Factorization} (\ref{Factorization}). The default value for this component is \texttt{true}. 
\item[{\texttt{small}}] if this component is set to \texttt{true}, then \textsf{Citrus} will compute a small subset of \mbox{\texttt{\mdseries\slshape gens}} that generates \mbox{\texttt{\mdseries\slshape S}} at the time that \mbox{\texttt{\mdseries\slshape S}} is created. This will increase the amount of time required to create \mbox{\texttt{\mdseries\slshape S}} substantially, but may decrease the amount of time required for subsequent
calculations with \mbox{\texttt{\mdseries\slshape S}}. If this component is set to \texttt{false}, then \textsf{Citrus} will return the semigroup generated by \mbox{\texttt{\mdseries\slshape gens}} without modifying \mbox{\texttt{\mdseries\slshape gens}}. The default value for this component is \texttt{false}. 
\end{description}
 The default values of the options described above are stored in a global
variable named \texttt{CitrusOptionsRec} (\ref{CitrusOptionsRec}). If you want to change the default values of these options for a single \textsf{GAP} session, then you can simply redefine the value in \textsf{GAP}. For example, to change the option \texttt{small} from the default value of \mbox{\texttt{\mdseries\slshape false}} use: 
\begin{Verbatim}[commandchars=!@|,fontsize=\small,frame=single,label=Example]
  !gapprompt@gap>| !gapinput@CitrusOptionsRec.small:=true;|
  true
\end{Verbatim}
 If you want to change the default values of the options stored in \texttt{CitrusOptionsRec} (\ref{CitrusOptionsRec}) for all subsequent \textsf{GAP} sessions, then you can edit these values in the file \texttt{citrus/gap/options.g}. 

 
\begin{Verbatim}[commandchars=!@|,fontsize=\small,frame=single,label=Example]
  !gapprompt@gap>| !gapinput@S:=Semigroup(Transformation( [ 1, 2, 3, 3 ] ), rec(schreier:=true,|
  !gapprompt@>| !gapinput@hashlen:=100003, small:=false));|
  <semigroup with 1 generator>
\end{Verbatim}
 

\subsection{\textcolor{Chapter }{CitrusOptionsRec}}
\logpage{[ 3, 3, 1 ]}\nobreak
\hyperdef{L}{X86BA5FA4816AAF44}{}
{\noindent\textcolor{FuncColor}{$\triangleright$\ \ \texttt{CitrusOptionsRec\index{CitrusOptionsRec@\texttt{CitrusOptionsRec}}
\label{CitrusOptionsRec}
}\hfill{\scriptsize (global variable)}}\\


 This global variable is a record whose components contain the default values
of certain options for transformation semigroups created after \textsf{Citrus} has been loaded. A description of these options is given above in Section \ref{opts}. 

 The value of \texttt{CitrusOptionsRec} is defined in the file \texttt{citrus/gap/options.g} as: 
\begin{Verbatim}[commandchars=!@|,fontsize=\small,frame=single,label=Example]
  rec( schreier:=true, small:=false,
           hashlen:=rec(S:=251, M:=6257, L:=25013));
\end{Verbatim}
 }

 }

 
\section{\textcolor{Chapter }{Standard examples}}\label{Examples}
\logpage{[ 3, 4, 0 ]}
\hyperdef{L}{X7C76D1DC7DAF03D3}{}
{
 In this section we describe functions for creating several standard examples
of semigroups of transformations and partial permutations. 

\subsection{\textcolor{Chapter }{FullMatrixSemigroup}}
\logpage{[ 3, 4, 1 ]}\nobreak
\hyperdef{L}{X78F9812D79A457EF}{}
{\noindent\textcolor{FuncColor}{$\triangleright$\ \ \texttt{FullMatrixSemigroup({\mdseries\slshape d, q})\index{FullMatrixSemigroup@\texttt{FullMatrixSemigroup}}
\label{FullMatrixSemigroup}
}\hfill{\scriptsize (operation)}}\\
\noindent\textcolor{FuncColor}{$\triangleright$\ \ \texttt{GeneralLinearSemigroup({\mdseries\slshape d, q})\index{GeneralLinearSemigroup@\texttt{GeneralLinearSemigroup}}
\label{GeneralLinearSemigroup}
}\hfill{\scriptsize (operation)}}\\
\textbf{\indent Returns:\ }
A matrix semigroup.



 \texttt{FullMatrixSemigroup} and \texttt{GeneralLinearSemigroup} are synonyms for each other. They both return the full matrix semigroup, or if
you prefer the general linear semigroup, of \mbox{\texttt{\mdseries\slshape d}} by \mbox{\texttt{\mdseries\slshape d}} matrices with entries over the field with \mbox{\texttt{\mdseries\slshape q}} elements. This semigroup has \texttt{q\texttt{\symbol{94}}(d\texttt{\symbol{94}}2)} elements. 

 \textsc{Please note:} there are currently no special methods for computing with matrix semigroups in \textsf{Citrus} and so it might be advisable to use \texttt{IsomorphismTransformationSemigroup} (\ref{IsomorphismTransformationSemigroup}). 
\begin{Verbatim}[commandchars=!@|,fontsize=\small,frame=single,label=Example]
  !gapprompt@gap>| !gapinput@S:=FullMatrixSemigroup(3,4);|
  <full matrix semigroup 3x3 over GF(2^2)>
  !gapprompt@gap>| !gapinput@T:=Range(IsomorphismTransformationSemigroup(S));;|
  !gapprompt@gap>| !gapinput@Size(T);|
  262144
\end{Verbatim}
 }

 

\subsection{\textcolor{Chapter }{IsFullMatrixSemigroup}}
\logpage{[ 3, 4, 2 ]}\nobreak
\hyperdef{L}{X85D49CF2826D3AA4}{}
{\noindent\textcolor{FuncColor}{$\triangleright$\ \ \texttt{IsFullMatrixSemigroup({\mdseries\slshape S})\index{IsFullMatrixSemigroup@\texttt{IsFullMatrixSemigroup}}
\label{IsFullMatrixSemigroup}
}\hfill{\scriptsize (property)}}\\
\noindent\textcolor{FuncColor}{$\triangleright$\ \ \texttt{IsGeneralLinearSemigroup({\mdseries\slshape S})\index{IsGeneralLinearSemigroup@\texttt{IsGeneralLinearSemigroup}}
\label{IsGeneralLinearSemigroup}
}\hfill{\scriptsize (property)}}\\


 \texttt{IsFullMatrixSemigroup} and \texttt{IsGeneralLinearSemigroup} return \texttt{true} if the semigroup \texttt{S} was created using either of the commands \texttt{FullMatrixSemigroup} (\ref{FullMatrixSemigroup}) or \texttt{GeneralLinearSemigroup} (\ref{GeneralLinearSemigroup}) and \texttt{false} otherwise. 
\begin{Verbatim}[commandchars=!@|,fontsize=\small,frame=single,label=Example]
  !gapprompt@gap>| !gapinput@S:=RandomTransformationSemigroup(4,4);;|
  !gapprompt@gap>| !gapinput@IsFullMatrixSemigroup(S);|
  false
  !gapprompt@gap>| !gapinput@S:=GeneralLinearSemigroup(3,3);|
  <full matrix semigroup 3x3 over GF(3)>
  !gapprompt@gap>| !gapinput@IsFullMatrixSemigroup(S);|
  true
\end{Verbatim}
 }

 

\subsection{\textcolor{Chapter }{MunnSemigroup}}
\logpage{[ 3, 4, 3 ]}\nobreak
\hyperdef{L}{X78FBE6DD7BCA30C1}{}
{\noindent\textcolor{FuncColor}{$\triangleright$\ \ \texttt{MunnSemigroup({\mdseries\slshape S})\index{MunnSemigroup@\texttt{MunnSemigroup}}
\label{MunnSemigroup}
}\hfill{\scriptsize (operation)}}\\
\textbf{\indent Returns:\ }
The Munn semigroup of a semilattice.



 If \mbox{\texttt{\mdseries\slshape S}} is a semilattice, then \texttt{MunnSemigroup(\mbox{\texttt{\mdseries\slshape S}});} returns the inverse semigroup of partial permutations of isomorphisms of
principal ideals of \mbox{\texttt{\mdseries\slshape S}}; called the \emph{Munn semigroup} of \mbox{\texttt{\mdseries\slshape S}}.

 This function was written jointly by J. D. Mitchell, Yann Peresse (St
Andrews), Yanhui Wang (York). 

 \textsc{Please note:} the \href{http://www.maths.qmul.ac.uk/~leonard/grape/} {Grape} package version 4.5 or higher should be fully installed for this function to
work. 
\begin{Verbatim}[commandchars=!@|,fontsize=\small,frame=single,label=Example]
  !gapprompt@gap>| !gapinput@S:=InverseSemigroup(|
  !gapprompt@>| !gapinput@PartialPermNC( [ 1, 2, 3, 4, 5, 6, 7, 10 ], [ 4, 6, 7, 3, 8, 2, 9, 5 ] ),|
  !gapprompt@>| !gapinput@PartialPermNC( [ 1, 2, 7, 9 ], [ 5, 6, 4, 3 ] ) );|
  <inverse semigroup with 2 generators>
  !gapprompt@gap>| !gapinput@T:=InverseSemigroup(Idempotents(S), rec(small:=true));;|
  !gapprompt@gap>| !gapinput@M:=MunnSemigroup(T);;|
  !gapprompt@gap>| !gapinput@NrIdempotents(M);|
  60
  !gapprompt@gap>| !gapinput@NrIdempotents(S);|
  60
\end{Verbatim}
 }

 
\subsection{\textcolor{Chapter }{Monoids of order preserving functions}}\logpage{[ 3, 4, 4 ]}
\hyperdef{L}{X87B855227B9870BD}{}
{
\noindent\textcolor{FuncColor}{$\triangleright$\ \ \texttt{OrderEndomorphisms({\mdseries\slshape n})\index{OrderEndomorphisms@\texttt{OrderEndomorphisms}!monoid of order preserving transformations}
\label{OrderEndomorphisms:monoid of order preserving transformations}
}\hfill{\scriptsize (operation)}}\\
\noindent\textcolor{FuncColor}{$\triangleright$\ \ \texttt{POI({\mdseries\slshape n})\index{POI@\texttt{POI}!monoid of order preserving partial perms}
\label{POI:monoid of order preserving partial perms}
}\hfill{\scriptsize (operation)}}\\
\noindent\textcolor{FuncColor}{$\triangleright$\ \ \texttt{POPI({\mdseries\slshape n})\index{POPI@\texttt{POPI}!monoid of orientation preserving partial
      perms}
\label{POPI:monoid of orientation preserving partial
      perms}
}\hfill{\scriptsize (operation)}}\\
\textbf{\indent Returns:\ }
A semigroup of transformations or partial permutations related to a linear
order. 



 
\begin{description}
\item[{\texttt{OrderEndomorphisms(\mbox{\texttt{\mdseries\slshape n}})}}]  \texttt{OrderEndomorphisms(\mbox{\texttt{\mdseries\slshape n}})} returns the monoid of transformations that preserve the usual order on $\{1,2,\ldots, n\}$ where \mbox{\texttt{\mdseries\slshape n}} is a positive integer.  \texttt{OrderEndomorphisms(\mbox{\texttt{\mdseries\slshape n}})} is generated by the $\mbox{\texttt{\mdseries\slshape n+1}}$ transformations: 
\[ \left( \begin{array}{ccccccccc} 1&2&3&\cdots&n-1& n\\ 1&1&2&\cdots&n-2&n-1
\end{array}\right), \qquad \left( \begin{array}{ccccccccc} 1&2&\cdots&i-1& i&
i+1&i+2&\cdots &n\\ 1&2&\cdots&i-1& i+1&i+1&i+2&\cdots &n\\ \end{array}\right) \]
 where $i=0,\ldots,n-1$ and has ${2n-1\choose n-1}$ elements.  
\item[{\texttt{POI(\mbox{\texttt{\mdseries\slshape n}})}}]  \texttt{POI(\mbox{\texttt{\mdseries\slshape n}})} returns the inverse monoid of partial permutations that preserve the usual
order on $\{1,2,\ldots, n\}$ where \mbox{\texttt{\mdseries\slshape n}} is a positive integer.  \texttt{POI(\mbox{\texttt{\mdseries\slshape n}})} is generated by the $\mbox{\texttt{\mdseries\slshape n}}$ partial permutations: 
\[ \left( \begin{array}{ccccc} 1&2&3&\cdots&n\\ -&1&2&\cdots&n-1
\end{array}\right), \qquad \left( \begin{array}{ccccccccc} 1&2&\cdots&i-1& i&
i+1&i+2&\cdots &n\\ 1&2&\cdots&i-1& i+1&-&i+2&\cdots&n\\ \end{array}\right) \]
 where $i=1, \ldots, n-1$ and has ${2n\choose n}$ elements.  
\item[{\texttt{POPI(\mbox{\texttt{\mdseries\slshape n}})}}]  \texttt{POPI(\mbox{\texttt{\mdseries\slshape n}})} returns the inverse monoid of partial permutation that preserve the
orientation of $\{1,2,\ldots, n\}$ where $n$ is a positive integer.  \texttt{POPI(\mbox{\texttt{\mdseries\slshape n}})} is generated by the partial permutations: 
\[ \left( \begin{array}{ccccc} 1&2&\cdots&n-1&n\\ 2&3&\cdots&n&1
\end{array}\right),\qquad \left( \begin{array}{cccccc} 1&2&\cdots&n-2&n-1&n\\
1&2&\cdots&n-2&n&- \end{array}\right). \]
 and has $1+\frac{n}{2}{2n\choose n}$ elements.  
\end{description}
 
\begin{Verbatim}[commandchars=!@|,fontsize=\small,frame=single,label=Example]
  !gapprompt@gap>| !gapinput@S:=POPI(10);                                            |
  <inverse monoid with 2 generators>
  !gapprompt@gap>| !gapinput@Size(S);|
  923781
  !gapprompt@gap>| !gapinput@1+5*Binomial(20, 10);|
  923781
  !gapprompt@gap>| !gapinput@S:=POI(10);|
  <inverse monoid with 10 generators>
  !gapprompt@gap>| !gapinput@Size(S);|
  184756
  !gapprompt@gap>| !gapinput@Binomial(20,10);|
  184756
  !gapprompt@gap>| !gapinput@IsSubsemigroup(POPI(10), POI(10));|
  true
  !gapprompt@gap>| !gapinput@S:=OrderEndomorphisms(5);|
  <monoid with 5 generators>
  !gapprompt@gap>| !gapinput@IsIdempotentGenerated(S);|
  true
  !gapprompt@gap>| !gapinput@IsRegularSemigroup(S);|
  true
  !gapprompt@gap>| !gapinput@Size(S)=Binomial(2*5-1, 5-1);|
  true
\end{Verbatim}
 }

 

\subsection{\textcolor{Chapter }{SingularSemigroup}}
\logpage{[ 3, 4, 5 ]}\nobreak
\hyperdef{L}{X79B1A1127B3B784A}{}
{\noindent\textcolor{FuncColor}{$\triangleright$\ \ \texttt{SingularSemigroup({\mdseries\slshape n})\index{SingularSemigroup@\texttt{SingularSemigroup}}
\label{SingularSemigroup}
}\hfill{\scriptsize (operation)}}\\
\textbf{\indent Returns:\ }
The semigroup of non-invertible transformations.



 If \mbox{\texttt{\mdseries\slshape n}} is a positive integer, then \texttt{SingularSemigroup(\mbox{\texttt{\mdseries\slshape n}})} returns the semigroup of non-invertible transformations, which is generated by
the \mbox{\texttt{\mdseries\slshape n(n-1)}} idempotents of degree \mbox{\texttt{\mdseries\slshape n}} and rank $n-1$ and has $n^n-n!$ elements. 
\begin{Verbatim}[commandchars=!@|,fontsize=\small,frame=single,label=Example]
  !gapprompt@gap>| !gapinput@S:=SingularSemigroup(5);|
  <semigroup with 20 generators>
  !gapprompt@gap>| !gapinput@Size(S);|
  3005
\end{Verbatim}
 }

 

\subsection{\textcolor{Chapter }{SymmetricInverseSemigp}}
\logpage{[ 3, 4, 6 ]}\nobreak
\hyperdef{L}{X7A1412CB79039397}{}
{\noindent\textcolor{FuncColor}{$\triangleright$\ \ \texttt{SymmetricInverseSemigp({\mdseries\slshape n})\index{SymmetricInverseSemigp@\texttt{SymmetricInverseSemigp}}
\label{SymmetricInverseSemigp}
}\hfill{\scriptsize (operation)}}\\
\textbf{\indent Returns:\ }
The symmetric inverse semigroup.



 If \mbox{\texttt{\mdseries\slshape n}} is a positive integer, then \texttt{SymmetricInverseSemigp(\mbox{\texttt{\mdseries\slshape n}});} returns the symmetric inverse semigroup consisting of all partial permutations
on the set $\{1,\ldots, n\}$.  The symmetric inverse semigroup on $\{1,\ldots, n\}$ is generated by the partial permutations: 
\[\left( \begin{array}{ccccc} 1&2&\cdots&n-1&n\\ 2&3&\cdots&n&1
\end{array}\right), \quad \left(\begin{array}{cccccc} 1&2&3&\cdots&n-1&n\\
2&1&3&\cdots&n-1&n \end{array}\right), \quad \left(\begin{array}{ccccc}
1&2&\cdots&n-1&n\\ 1&2&\cdots&n-1&- \end{array}\right) \]
 and has $\sum_{r=0}^n{n\choose r}^2\cdot r!$ elements.  
\begin{Verbatim}[commandchars=!@|,fontsize=\small,frame=single,label=Example]
  !gapprompt@gap>| !gapinput@S:=SymmetricInverseSemigp(15);|
  <inverse semigroup with 3 generators>
  !gapprompt@gap>| !gapinput@Size(S);|
  306827170866106
\end{Verbatim}
 }

 }

 
\section{\textcolor{Chapter }{The examples directory}}\label{Catalogues}
\logpage{[ 3, 5, 0 ]}
\hyperdef{L}{X7F59D5A17CE475CC}{}
{
 The \texttt{examples} folder of the \textsf{Citrus} package directory contains catalogues of some naturally occurring semigroups
of transformations and partial permutations. These files can be read into \textsf{GAP} using \texttt{ReadCitrus} (\ref{ReadCitrus}) and similar files can be created using \texttt{WriteCitrus} (\ref{WriteCitrus}).

 Further examples can be downloaded from \vspace{\baselineskip}

 \href{http://tinyurl.com/jdmitchell/examples.html} {\texttt{http://tinyurl.com/jdmitchell/examples.html}}\vspace{\baselineskip}

 \noindent A summary of the available files, a desciption of their contents, and how they
were created is given below.

 
\begin{description}
\item[{Endomorphisms of graphs}]  the files \texttt{eul\mbox{\texttt{\mdseries\slshape n}}c.citrus.gz} with $n=3,...,10$; \texttt{graph\mbox{\texttt{\mdseries\slshape n}}c.citrus.gz} with $n=3,...,8$; and \texttt{selfcomp.citrus.gz} contain small generating sets for the endomorphism monoids of all connected
Eulerian graphs, all connected graphs, and all self complimentary graphs with $n$ vertices, respectively. These files were created using the catalogues of such
graphs available at:

 \href{http://cs.anu.edu.au/~bdm/data/graphs.html} {\texttt{http://cs.anu.edu.au/\texttt{\symbol{126}}bdm/data/graphs.html}}

 a C program written by Max Neunhoeffer which produces a relatively large list
of endomorphisms containing a generating set for the endomorphism monoid, \texttt{SmallGeneratingSet} (\ref{SmallGeneratingSet}) and then \texttt{IrredundantGeneratingSubset} (\ref{IrredundantGeneratingSubset}) in \textsf{Citrus}. The monoid generated by the transformations output by \texttt{ReadCitrus("eul7c.citrus.gz", i);}, say, is the monoid of endomorphisms of the \texttt{i}th graph in the file:

 \href{http://cs.anu.edu.au/~bdm/data/eul7c.g6} {\texttt{http://cs.anu.edu.au/\texttt{\symbol{126}}bdm/data/eul7c.g6}} 
\item[{Munn semigroups}]  the file \texttt{munn.citrus.gz} contains generators for all the Munn semigroups of semilattices with 2 to 8
elements. The semilattices were obtained from the \href{http://www-history.mcs.st-and.ac.uk/~jamesm/smallsemi/index.html} {Smallsemi} package using the command: 
\begin{Verbatim}[commandchars=!@|,fontsize=\small,frame=single,label=Example]
  AllSmallSemigroups([2..8], IsSemilatticeAsSemigroup, true);
\end{Verbatim}
 and the generators for the Munn semigroups were calculated using \texttt{MunnSemigroup} (\ref{MunnSemigroup}). More information is available at:

 \href{http://tinyurl.com/jdmitchell/examples.html} {\texttt{http://tinyurl.com/jdmitchell/examples.html}} 
\item[{Syntactic semigroups}]  the files \texttt{syntactic.citrus.gz} contain generators for the syntactic semigroups of word acceptors of certain
triangle groups, provided by Markus Pfeiffer (St Andrews). A \emph{triangle group} is a group defined by a presentation of the form 
\[ \langle x, y | x^p, y^q, (xy)^r\rangle \]
 for some positive integers $p, q, r$. The file contains generators of the syntactic semigroups of word acceptors
of triangle groups where $p$ ranges from $1$ to $94$, $q=3$, and $r=2$; $p=101$, $q$ ranges from $3$ to $99$ and $r=2$; $p=101$, $q=72$, and $r$ ranges from $7$ to $71$; and some further randomly chosen values of $p,q,r$. 
\item[{Endomorphisms of groups}]  the files \texttt{nonabelian{\textunderscore}groups{\textunderscore}\mbox{\texttt{\mdseries\slshape n}}.citrus.gz} with $n=6,....,64$ contains small generating sets for the endomorphism monoids of all non-abelian
groups with \mbox{\texttt{\mdseries\slshape n}} elements. These files were created using the Small Groups Library in \textsf{GAP} and the \textsf{Sonata} function \texttt{Endomorphisms}. 
\end{description}
 }

 }

 
\chapter{\textcolor{Chapter }{Determining the structure of a semigroup}}\label{green}
\logpage{[ 4, 0, 0 ]}
\hyperdef{L}{X86BAA72482D1C658}{}
{
 In this chapter we describe the functions in \textsf{Citrus} for determining the structure of a semigroup of transformations or partial
permutations, in particular for computing Green's classes and related
properties of semigroups. 
\section{\textcolor{Chapter }{Expressing semigroup elements as words in generators}}\logpage{[ 4, 1, 0 ]}
\hyperdef{L}{X81CEB3717E021643}{}
{
 In some cases it is possible to express the elements of a transformation
semigroup as a word in the generators. This section describes how to
accomplish this in \textsf{Citrus}. 

 Note that at present it is not possible to factorize elements in a semigroup
of partial permutations. 

\subsection{\textcolor{Chapter }{EvaluateWord}}
\logpage{[ 4, 1, 1 ]}\nobreak
\hyperdef{L}{X799D2F3C866B9AED}{}
{\noindent\textcolor{FuncColor}{$\triangleright$\ \ \texttt{EvaluateWord({\mdseries\slshape gens, w})\index{EvaluateWord@\texttt{EvaluateWord}}
\label{EvaluateWord}
}\hfill{\scriptsize (operation)}}\\
\textbf{\indent Returns:\ }
A transformation.



 The argument \mbox{\texttt{\mdseries\slshape gens}} should be a list of transformations and the argument \mbox{\texttt{\mdseries\slshape w}} should be a list of positive integers less than or equal to the length of \mbox{\texttt{\mdseries\slshape gens}}. This operation evaluates the word \mbox{\texttt{\mdseries\slshape w}} in the generators \mbox{\texttt{\mdseries\slshape gens}}. More precisely, \texttt{EvaluateWord} returns the equivalent of: 
\begin{Verbatim}[commandchars=!@|,fontsize=\small,frame=single,label=Example]
  Product(List(w, i-> gens[i]));
\end{Verbatim}
 see also \texttt{Factorization} (\ref{Factorization}). 
\begin{Verbatim}[commandchars=!@|,fontsize=\small,frame=single,label=Example]
  !gapprompt@gap>| !gapinput@gens:=[ Transformation( [ 2, 4, 4, 6, 8, 8, 6, 6 ] ), |
  !gapprompt@>| !gapinput@Transformation( [ 2, 7, 4, 1, 4, 6, 5, 2 ] ), |
  !gapprompt@>| !gapinput@Transformation( [ 3, 6, 2, 4, 2, 2, 2, 8 ] ), |
  !gapprompt@>| !gapinput@Transformation( [ 4, 3, 6, 4, 2, 1, 2, 6 ] ), |
  !gapprompt@>| !gapinput@Transformation( [ 4, 5, 1, 3, 8, 5, 8, 2 ] ) ];;|
  !gapprompt@gap>| !gapinput@S:=Semigroup(gens);;|
  !gapprompt@gap>| !gapinput@f:=Transformation( [ 1, 4, 6, 1, 7, 2, 7, 6 ] );;|
  !gapprompt@gap>| !gapinput@Factorization(S, f);|
  [ 4, 2 ]
  !gapprompt@gap>| !gapinput@EvaluateWord(gens, last);|
  Transformation( [ 1, 4, 6, 1, 7, 2, 7, 6 ] )
\end{Verbatim}
 }

 

\subsection{\textcolor{Chapter }{Factorization}}
\logpage{[ 4, 1, 2 ]}\nobreak
\hyperdef{L}{X8357294D7B164106}{}
{\noindent\textcolor{FuncColor}{$\triangleright$\ \ \texttt{Factorization({\mdseries\slshape S, f})\index{Factorization@\texttt{Factorization}}
\label{Factorization}
}\hfill{\scriptsize (method)}}\\
\textbf{\indent Returns:\ }
A word in the generators.



 If \mbox{\texttt{\mdseries\slshape S}} is a transformation semigroup and \mbox{\texttt{\mdseries\slshape f}} is a transformation belonging to \mbox{\texttt{\mdseries\slshape S}}, then \texttt{Factorization} returns a word in the generators of \mbox{\texttt{\mdseries\slshape S}} that is equal to \mbox{\texttt{\mdseries\slshape f}}. Here, a word is a list of positive integers where \texttt{i} corresponds to \texttt{GeneratorsOfSemigroups(S)[i]}. More specifically, 
\begin{Verbatim}[commandchars=!@|,fontsize=\small,frame=single,label=Example]
  EvaluateWord(GeneratorsOfSemigroup(S), Factorization(S, f))=f;
\end{Verbatim}
 

 Note that \texttt{Factorization} does not return a word of minimum length. An empty list is evaluated to the
identity transformation with degree equal to the degree of \mbox{\texttt{\mdseries\slshape S}} regardless of whether this transformation is an element of \mbox{\texttt{\mdseries\slshape S}} or not.

 See also \texttt{EvaluateWord} (\ref{EvaluateWord}) and \texttt{GeneratorsOfSemigroup} (\textbf{Reference: GeneratorsOfSemigroup}). 
\begin{Verbatim}[commandchars=!@|,fontsize=\small,frame=single,label=Example]
  !gapprompt@gap>| !gapinput@gens:=[ Transformation( [ 2, 2, 9, 7, 4, 9, 5, 5, 4, 8 ] ), |
  !gapprompt@>| !gapinput@Transformation( [ 4, 10, 5, 6, 4, 1, 2, 7, 1, 2 ] ) ];;|
  !gapprompt@gap>| !gapinput@S:=Semigroup(gens);;|
  !gapprompt@gap>| !gapinput@f:=Transformation( [ 1, 10, 2, 10, 1, 2, 7, 10, 2, 7 ] );;|
  !gapprompt@gap>| !gapinput@Factorization(S, f);|
  [ 2, 2, 1, 2 ]
  !gapprompt@gap>| !gapinput@EvaluateWord(gens, last);|
  Transformation( [ 1, 10, 2, 10, 1, 2, 7, 10, 2, 7 ] )
\end{Verbatim}
 }

 }

 
\section{\textcolor{Chapter }{Creating Green's classes}}\logpage{[ 4, 2, 0 ]}
\hyperdef{L}{X7D14B6A080BC189E}{}
{
 
\subsection{\textcolor{Chapter }{XClassOfYClass}}\logpage{[ 4, 2, 1 ]}
\hyperdef{L}{X87558FEF805D24E1}{}
{
\noindent\textcolor{FuncColor}{$\triangleright$\ \ \texttt{DClassOfHClass({\mdseries\slshape class})\index{DClassOfHClass@\texttt{DClassOfHClass}}
\label{DClassOfHClass}
}\hfill{\scriptsize (method)}}\\
\noindent\textcolor{FuncColor}{$\triangleright$\ \ \texttt{DClassOfLClass({\mdseries\slshape class})\index{DClassOfLClass@\texttt{DClassOfLClass}}
\label{DClassOfLClass}
}\hfill{\scriptsize (method)}}\\
\noindent\textcolor{FuncColor}{$\triangleright$\ \ \texttt{DClassOfRClass({\mdseries\slshape class})\index{DClassOfRClass@\texttt{DClassOfRClass}}
\label{DClassOfRClass}
}\hfill{\scriptsize (method)}}\\
\noindent\textcolor{FuncColor}{$\triangleright$\ \ \texttt{LClassOfHClass({\mdseries\slshape class})\index{LClassOfHClass@\texttt{LClassOfHClass}}
\label{LClassOfHClass}
}\hfill{\scriptsize (method)}}\\
\noindent\textcolor{FuncColor}{$\triangleright$\ \ \texttt{RClassOfHClass({\mdseries\slshape class})\index{RClassOfHClass@\texttt{RClassOfHClass}}
\label{RClassOfHClass}
}\hfill{\scriptsize (method)}}\\
\textbf{\indent Returns:\ }
A Green's class.



 \texttt{XClassOfYClass} returns the \texttt{X}-class containing the \texttt{Y}-class \mbox{\texttt{\mdseries\slshape class}} where \texttt{X} and \texttt{Y} should be replaced by an appropriate choice of \texttt{D, H, L,} and \texttt{R}.

 Note that if it is not known to \textsf{GAP} whether or not the representative of \mbox{\texttt{\mdseries\slshape class}} is an element of the semigroup containing \mbox{\texttt{\mdseries\slshape class}}, then no attempt is made to check this.

 The same result can be produced using: 
\begin{Verbatim}[commandchars=!@|,fontsize=\small,frame=single,label=Example]
  First(GreensXClasses(S), x-> Representative(x) in class);
\end{Verbatim}
 but this might be substantially slower. Note that \texttt{XClassOfYClass} is also likely to be faster than 
\begin{Verbatim}[commandchars=!@|,fontsize=\small,frame=single,label=Example]
  GreensXClassOfElement(S, Representative(class));
\end{Verbatim}
 

 \texttt{DClass} can also be used as a synonym for \texttt{DClassOfHClass}, \texttt{DClassOfLClass}, and \texttt{DClassOfRClass}; \texttt{LClass} as a synonym for \texttt{LClassOfHClass}; and \texttt{RClass} as a synonym for \texttt{RClassOfHClass}. See also \texttt{GreensDClassOfElement} (\textbf{Reference: GreensDClassOfElement}) and \texttt{GreensDClassOfElementNC} (\ref{GreensDClassOfElementNC}). 
\begin{Verbatim}[commandchars=!@|,fontsize=\small,frame=single,label=Example]
  !gapprompt@gap>| !gapinput@S:=Semigroup(Transformation( [ 1, 3, 2 ] ), |
  !gapprompt@>| !gapinput@Transformation( [ 2, 1, 3 ] ), Transformation( [ 3, 2, 1 ] ), |
  !gapprompt@>| !gapinput@Transformation( [ 1, 3, 1 ] ) );;|
  !gapprompt@gap>| !gapinput@R:=GreensRClassOfElement(S, Transformation( [ 3, 2, 1 ] ));|
  {Transformation( [ 1, 2, 3 ] )}
  !gapprompt@gap>| !gapinput@DClassOfRClass(R);|
  {Transformation( [ 1, 2, 3 ] )}
  !gapprompt@gap>| !gapinput@IsGreensDClass(DClassOfRClass(R));|
  true
  !gapprompt@gap>| !gapinput@S:=InverseSemigroup(PartialPermNC([ 1, 2, 3, 6, 8, 10 ],|
  !gapprompt@>| !gapinput@[ 2, 6, 7, 9, 1, 5 ]), PartialPermNC([ 1, 2, 3, 4, 6, 7, 8, 10 ],|
  !gapprompt@>| !gapinput@[ 3, 8, 1, 9, 4, 10, 5, 6 ]));|
  <inverse semigroup with 2 generators>
  !gapprompt@gap>| !gapinput@f:=Generators(S)[1];|
  [ 1, 2, 3, 6, 8, 10 ] -> [ 2, 6, 7, 9, 1, 5 ]
  !gapprompt@gap>| !gapinput@h:=HClass(S, f);|
  {[ 1, 2, 3, 6, 8, 10 ] -> [ 2, 6, 7, 9, 1, 5 ]}
  !gapprompt@gap>| !gapinput@r:=RClassOfHClass(h);|
  {<identity on [ 1, 2, 3, 6, 8, 10 ]>}
  !gapprompt@gap>| !gapinput@l:=LClass(h);|
  {[ 1, 2, 3, 6, 8, 10 ] -> [ 2, 6, 7, 9, 1, 5 ]}
  !gapprompt@gap>| !gapinput@DClass(r)=DClass(l);|
  true
  !gapprompt@gap>| !gapinput@DClass(h)=DClass(l);|
  true
\end{Verbatim}
 }

 
\subsection{\textcolor{Chapter }{GreensXClassOfElement}}\logpage{[ 4, 2, 2 ]}
\hyperdef{L}{X81B7AD4C7C552867}{}
{
\noindent\textcolor{FuncColor}{$\triangleright$\ \ \texttt{GreensDClassOfElement({\mdseries\slshape X, f})\index{GreensDClassOfElement@\texttt{GreensDClassOfElement}}
\label{GreensDClassOfElement}
}\hfill{\scriptsize (operation)}}\\
\noindent\textcolor{FuncColor}{$\triangleright$\ \ \texttt{DClass({\mdseries\slshape X, f})\index{DClass@\texttt{DClass}}
\label{DClass}
}\hfill{\scriptsize (function)}}\\
\noindent\textcolor{FuncColor}{$\triangleright$\ \ \texttt{GreensHClassOfElement({\mdseries\slshape X, f})\index{GreensHClassOfElement@\texttt{GreensHClassOfElement}}
\label{GreensHClassOfElement}
}\hfill{\scriptsize (operation)}}\\
\noindent\textcolor{FuncColor}{$\triangleright$\ \ \texttt{HClass({\mdseries\slshape X, f})\index{HClass@\texttt{HClass}}
\label{HClass}
}\hfill{\scriptsize (function)}}\\
\noindent\textcolor{FuncColor}{$\triangleright$\ \ \texttt{GreensLClassOfElement({\mdseries\slshape X, f})\index{GreensLClassOfElement@\texttt{GreensLClassOfElement}}
\label{GreensLClassOfElement}
}\hfill{\scriptsize (operation)}}\\
\noindent\textcolor{FuncColor}{$\triangleright$\ \ \texttt{LClass({\mdseries\slshape X, f})\index{LClass@\texttt{LClass}}
\label{LClass}
}\hfill{\scriptsize (function)}}\\
\noindent\textcolor{FuncColor}{$\triangleright$\ \ \texttt{GreensRClassOfElement({\mdseries\slshape X, f})\index{GreensRClassOfElement@\texttt{GreensRClassOfElement}}
\label{GreensRClassOfElement}
}\hfill{\scriptsize (operation)}}\\
\noindent\textcolor{FuncColor}{$\triangleright$\ \ \texttt{RClass({\mdseries\slshape X, f})\index{RClass@\texttt{RClass}}
\label{RClass}
}\hfill{\scriptsize (function)}}\\
\textbf{\indent Returns:\ }
A Green's class.



 These functions produce essentially the same output as the \textsf{GAP} library functions with the same names; see \texttt{GreensDClassOfElement} (\textbf{Reference: GreensDClassOfElement}). The main difference is that these functions can be applied to a wider class
of objects: 
\begin{description}
\item[{\texttt{GreensDClassOfElement} and \texttt{DClass}}]  \mbox{\texttt{\mdseries\slshape X}} must be a semigroup of transformations or partial permutations. 
\item[{\texttt{GreensHClassOfElement} and \texttt{HClass}}]  \mbox{\texttt{\mdseries\slshape X}} can be a semigroup of transformations or partial permutations, $\mathcal{R}$-class, $\mathcal{L}$-class, or $\mathcal{D}$-class. 
\item[{\texttt{GreensLClassOfElement} and \texttt{LClass}}]  \mbox{\texttt{\mdseries\slshape X}} can be a semigroup of transformations or partial permutations, or $\mathcal{D}$-class. 
\item[{\texttt{GreensRClassOfElement} and \texttt{RClass}}]  \mbox{\texttt{\mdseries\slshape X}} can be a semigroup of transformations or partial permutations, or $\mathcal{D}$-class. 
\end{description}
 Note that \texttt{GreensXClassOfElement} and \texttt{XClass} are synonyms and have identical output. The shorter command is provided for
the sake of convenience.

 }

 
\subsection{\textcolor{Chapter }{GreensXClassOfElementNC}}\logpage{[ 4, 2, 3 ]}
\hyperdef{L}{X7B44317786571F8B}{}
{
\noindent\textcolor{FuncColor}{$\triangleright$\ \ \texttt{GreensDClassOfElementNC({\mdseries\slshape X, f})\index{GreensDClassOfElementNC@\texttt{GreensDClassOfElementNC}}
\label{GreensDClassOfElementNC}
}\hfill{\scriptsize (operation)}}\\
\noindent\textcolor{FuncColor}{$\triangleright$\ \ \texttt{DClassNC({\mdseries\slshape X, f})\index{DClassNC@\texttt{DClassNC}}
\label{DClassNC}
}\hfill{\scriptsize (function)}}\\
\noindent\textcolor{FuncColor}{$\triangleright$\ \ \texttt{GreensHClassOfElementNC({\mdseries\slshape X, f})\index{GreensHClassOfElementNC@\texttt{GreensHClassOfElementNC}}
\label{GreensHClassOfElementNC}
}\hfill{\scriptsize (operation)}}\\
\noindent\textcolor{FuncColor}{$\triangleright$\ \ \texttt{HClassNC({\mdseries\slshape X, f})\index{HClassNC@\texttt{HClassNC}}
\label{HClassNC}
}\hfill{\scriptsize (function)}}\\
\noindent\textcolor{FuncColor}{$\triangleright$\ \ \texttt{GreensLClassOfElementNC({\mdseries\slshape X, f})\index{GreensLClassOfElementNC@\texttt{GreensLClassOfElementNC}}
\label{GreensLClassOfElementNC}
}\hfill{\scriptsize (operation)}}\\
\noindent\textcolor{FuncColor}{$\triangleright$\ \ \texttt{LClassNC({\mdseries\slshape X, f})\index{LClassNC@\texttt{LClassNC}}
\label{LClassNC}
}\hfill{\scriptsize (function)}}\\
\noindent\textcolor{FuncColor}{$\triangleright$\ \ \texttt{GreensRClassOfElementNC({\mdseries\slshape X, f})\index{GreensRClassOfElementNC@\texttt{GreensRClassOfElementNC}}
\label{GreensRClassOfElementNC}
}\hfill{\scriptsize (operation)}}\\
\noindent\textcolor{FuncColor}{$\triangleright$\ \ \texttt{RClassNC({\mdseries\slshape X, f})\index{RClassNC@\texttt{RClassNC}}
\label{RClassNC}
}\hfill{\scriptsize (function)}}\\
\textbf{\indent Returns:\ }
A Green's class.



 These functions are essentially the same as \texttt{GreensDClassOfElement} (\ref{GreensDClassOfElement}) except that no effort is made to verify if \mbox{\texttt{\mdseries\slshape f}} is an element of \mbox{\texttt{\mdseries\slshape X}}. More precisely, \texttt{GreensXClassOfElementNC} and \texttt{XClassNC} first check if \mbox{\texttt{\mdseries\slshape f}} has already been shown to be an element of \mbox{\texttt{\mdseries\slshape X}}. If it is not known to \textsf{GAP} if \mbox{\texttt{\mdseries\slshape f}} is an element of \mbox{\texttt{\mdseries\slshape X}}, then no further attempt to verify this is made. 

 Note that \texttt{GreensXClassOfElementNC} and \texttt{XClassNC} are synonyms and have identical output. The shorter command is provided for
the sake of convenience. 

 It can be quicker to compute the class of an element using \texttt{GreensRClassOfElementNC}, say, than using \texttt{GreensRClassOfElement} if it is known \emph{a priori} that \mbox{\texttt{\mdseries\slshape f}} is an element of \mbox{\texttt{\mdseries\slshape X}}. On the other hand, if \mbox{\texttt{\mdseries\slshape f}} is not an element of \mbox{\texttt{\mdseries\slshape X}}, then the results of this computation are unpredictable.

 For example, if 
\begin{Verbatim}[commandchars=!@|,fontsize=\small,frame=single,label=Example]
  f:=Transformation( [ 15, 18, 20, 20, 20, 20, 20, 20, 20, 20, 20, 20, 20, 20, 
        20, 20, 20, 20, 20, 20 ] );
\end{Verbatim}
 in the semigroup \mbox{\texttt{\mdseries\slshape X}} of order-preserving mappings on 20 points, then 
\begin{Verbatim}[commandchars=!@|,fontsize=\small,frame=single,label=Example]
  GreensRClassOfElementNC(X, f);;
\end{Verbatim}
 returns an answer relatively quickly, whereas \texttt{GreensRClassOfElement} can take a signficant amount of time to return a value.

 See also \texttt{GreensRClassOfElement} (\textbf{Reference: GreensRClassOfElement}) and \texttt{RClassOfHClass} (\ref{RClassOfHClass}). 
\begin{Verbatim}[commandchars=!@|,fontsize=\small,frame=single,label=Example]
  !gapprompt@gap>| !gapinput@S:=RandomTransformationSemigroup(2,1000);;|
  !gapprompt@gap>| !gapinput@f:=[ 1, 1, 2, 2, 2, 1, 1, 1, 1, 1, 2, 2, 2, 2, 1, 1, 2, 2, 1 ];;|
  !gapprompt@gap>| !gapinput@f:=EvaluateWord(Generators(S), f);;                            |
  !gapprompt@gap>| !gapinput@R:=GreensRClassOfElementNC(S, f);;|
  !gapprompt@gap>| !gapinput@Size(R);|
  1
  !gapprompt@gap>| !gapinput@L:=GreensLClassOfElementNC(S, f);;|
  !gapprompt@gap>| !gapinput@Size(L);|
  1 
  !gapprompt@gap>| !gapinput@f:=PartialPermNC([ 1, 2, 3, 4, 7, 8, 9, 10 ],|
  !gapprompt@>| !gapinput@[ 2, 3, 4, 5, 6, 8, 10, 11 ]);;|
  !gapprompt@gap>| !gapinput@L:=LClass(POI(13), f);|
  {[ 1, 2, 3, 4, 7, 8, 9, 10 ] -> [ 2, 3, 4, 5, 6, 8, 10, 11 ]}
  !gapprompt@gap>| !gapinput@Size(L);|
  1287
\end{Verbatim}
 }

 

\subsection{\textcolor{Chapter }{GroupHClass}}
\logpage{[ 4, 2, 4 ]}\nobreak
\hyperdef{L}{X8723756387DD4C0F}{}
{\noindent\textcolor{FuncColor}{$\triangleright$\ \ \texttt{GroupHClass({\mdseries\slshape class})\index{GroupHClass@\texttt{GroupHClass}}
\label{GroupHClass}
}\hfill{\scriptsize (attribute)}}\\
\textbf{\indent Returns:\ }
A group $\mathcal{H}$-class of the $\mathcal{D}$-class \mbox{\texttt{\mdseries\slshape class}} if it is regular and \texttt{fail} if it is not. 



 \texttt{GroupHClass} is a synonym for \texttt{GroupHClassOfGreensDClass} (\textbf{Reference: GroupHClassOfGreensDClass}). 

 See also \texttt{IsGroupHClass} (\textbf{Reference: IsGroupHClass}), \texttt{IsRegularDClass} (\textbf{Reference: IsRegularDClass}), \texttt{IsRegularDClass} (\ref{IsRegularDClass}), and \texttt{IsRegularSemigroup} (\ref{IsRegularSemigroup}). 
\begin{Verbatim}[commandchars=!@|,fontsize=\small,frame=single,label=Example]
  !gapprompt@gap>| !gapinput@S:=Semigroup( Transformation( [ 2, 6, 7, 2, 6, 1, 1, 5 ] ), |
  !gapprompt@>| !gapinput@Transformation( [ 3, 8, 1, 4, 5, 6, 7, 1 ] ) );;|
  !gapprompt@gap>| !gapinput@IsRegularSemigroup(S);|
  false
  !gapprompt@gap>| !gapinput@iter:=IteratorOfDClasses(S);;|
  !gapprompt@gap>| !gapinput@repeat D:=NextIterator(iter); until IsRegularDClass(D);   |
  !gapprompt@gap>| !gapinput@D;|
  {Transformation( [ 6, 1, 1, 6, 1, 2, 2, 6 ] )}
  !gapprompt@gap>| !gapinput@NrIdempotents(D);|
  12
  !gapprompt@gap>| !gapinput@NrRClasses(D);|
  8
  !gapprompt@gap>| !gapinput@NrLClasses(D);|
  4
  !gapprompt@gap>| !gapinput@GroupHClass(D);|
  {Transformation( [ 6, 1, 1, 6, 1, 2, 2, 6 ] )}
  !gapprompt@gap>| !gapinput@GroupHClassOfGreensDClass(D);|
  {Transformation( [ 6, 1, 1, 6, 1, 2, 2, 6 ] )}
  !gapprompt@gap>| !gapinput@StructureDescription(GroupHClass(D));|
  "S3"
  !gapprompt@gap>| !gapinput@repeat D:=NextIterator(iter); until not IsRegularDClass(D);|
  !gapprompt@gap>| !gapinput@D;|
  {Transformation( [ 7, 5, 2, 2, 6, 1, 1, 2 ] )}
  !gapprompt@gap>| !gapinput@IsRegularDClass(D);|
  false
  !gapprompt@gap>| !gapinput@GroupHClass(D);|
  fail
  !gapprompt@gap>| !gapinput@s:=InverseSemigroup( [ PartialPermNC( [ 1, 2, 3, 5 ], [ 2, 1, 6, 3 ] ),|
  !gapprompt@>| !gapinput@PartialPermNC( [ 1, 2, 3, 6 ], [ 3, 5, 2, 6 ] ) ]);;|
  !gapprompt@gap>| !gapinput@f:=PartialPermNC([ 1 .. 3 ], [ 6, 3, 1 ]);;|
  !gapprompt@gap>| !gapinput@First(DClasses(s), x-> not IsTrivial(GroupHClass(x)));|
  {<identity on [ 1, 2 ]>}
  !gapprompt@gap>| !gapinput@StructureDescription(GroupHClass(last));|
  "C2"
\end{Verbatim}
 }

 }

 
\section{\textcolor{Chapter }{Iterators and enumerators of classes and representatives }}\logpage{[ 4, 3, 0 ]}
\hyperdef{L}{X819CCBD67FD27115}{}
{
 
\subsection{\textcolor{Chapter }{GreensXClasses}}\label{GreensXClasses}
\logpage{[ 4, 3, 1 ]}
\hyperdef{L}{X7D51218A80234DE5}{}
{
\noindent\textcolor{FuncColor}{$\triangleright$\ \ \texttt{GreensDClasses({\mdseries\slshape obj})\index{GreensDClasses@\texttt{GreensDClasses}}
\label{GreensDClasses}
}\hfill{\scriptsize (method)}}\\
\noindent\textcolor{FuncColor}{$\triangleright$\ \ \texttt{DClasses({\mdseries\slshape obj})\index{DClasses@\texttt{DClasses}}
\label{DClasses}
}\hfill{\scriptsize (method)}}\\
\noindent\textcolor{FuncColor}{$\triangleright$\ \ \texttt{GreensHClasses({\mdseries\slshape obj})\index{GreensHClasses@\texttt{GreensHClasses}}
\label{GreensHClasses}
}\hfill{\scriptsize (method)}}\\
\noindent\textcolor{FuncColor}{$\triangleright$\ \ \texttt{HClasses({\mdseries\slshape obj})\index{HClasses@\texttt{HClasses}}
\label{HClasses}
}\hfill{\scriptsize (method)}}\\
\noindent\textcolor{FuncColor}{$\triangleright$\ \ \texttt{GreensJClasses({\mdseries\slshape obj})\index{GreensJClasses@\texttt{GreensJClasses}}
\label{GreensJClasses}
}\hfill{\scriptsize (method)}}\\
\noindent\textcolor{FuncColor}{$\triangleright$\ \ \texttt{JClasses({\mdseries\slshape obj})\index{JClasses@\texttt{JClasses}}
\label{JClasses}
}\hfill{\scriptsize (method)}}\\
\noindent\textcolor{FuncColor}{$\triangleright$\ \ \texttt{GreensLClasses({\mdseries\slshape obj})\index{GreensLClasses@\texttt{GreensLClasses}}
\label{GreensLClasses}
}\hfill{\scriptsize (method)}}\\
\noindent\textcolor{FuncColor}{$\triangleright$\ \ \texttt{LClasses({\mdseries\slshape obj})\index{LClasses@\texttt{LClasses}}
\label{LClasses}
}\hfill{\scriptsize (method)}}\\
\noindent\textcolor{FuncColor}{$\triangleright$\ \ \texttt{GreensRClasses({\mdseries\slshape obj})\index{GreensRClasses@\texttt{GreensRClasses}}
\label{GreensRClasses}
}\hfill{\scriptsize (method)}}\\
\noindent\textcolor{FuncColor}{$\triangleright$\ \ \texttt{RClasses({\mdseries\slshape obj})\index{RClasses@\texttt{RClasses}}
\label{RClasses}
}\hfill{\scriptsize (method)}}\\
\textbf{\indent Returns:\ }
A list of Green's classes. 



 These functions produce essentially the same output as the \textsf{GAP} library functions with the same names; see \texttt{GreensDClasses} (\textbf{Reference: GreensDClasses}). The main difference is that these functions can be applied to a wider class
of objects: 
\begin{description}
\item[{\texttt{GreensDClasses} and \texttt{DClasses}}]  \mbox{\texttt{\mdseries\slshape X}} should be a semigroup of transformations or partial permutations. 
\item[{\texttt{GreensHClasses} and \texttt{HClasses}}]  \mbox{\texttt{\mdseries\slshape X}} can be a semigroup of transformations or partial permutations, $\mathcal{R}$-class, $\mathcal{L}$-class, or $\mathcal{D}$-class. 
\item[{\texttt{GreensLClasses} and \texttt{LClasses}}]  \mbox{\texttt{\mdseries\slshape X}} can be a semigroup of transformations or partial permutations, or $\mathcal{D}$-class. 
\item[{\texttt{GreensRClasses} and \texttt{RClasses}}]  \mbox{\texttt{\mdseries\slshape X}} can be a semigroup of transformations or partial permutations, or $\mathcal{D}$-class. 
\end{description}
 Note that \texttt{GreensXClasses} and \texttt{XClasses} are synonyms and have identical output. The shorter command is provided for
the sake of convenience.

 See also \texttt{DClassReps} (\ref{DClassReps}), \texttt{IteratorOfDClassReps} (\ref{IteratorOfDClassReps}), \texttt{IteratorOfDClasses} (\ref{IteratorOfDClasses}), and \texttt{NrDClasses} (\ref{NrDClasses}). 
\begin{Verbatim}[commandchars=!@|,fontsize=\small,frame=single,label=Example]
  !gapprompt@gap>| !gapinput@S:=Semigroup(Transformation( [ 3, 4, 4, 4 ] ), |
  !gapprompt@>| !gapinput@Transformation( [ 4, 3, 1, 2 ] ));;|
  !gapprompt@gap>| !gapinput@GreensDClasses(S);|
  [ {Transformation( [ 1, 2, 3, 4 ] )}, {Transformation( [ 3, 4, 4, 4 ] )}, 
    {Transformation( [ 4, 4, 4, 4 ] )} ]
  !gapprompt@gap>| !gapinput@GreensRClasses(S);|
  [ {Transformation( [ 1, 2, 3, 4 ] )}, {Transformation( [ 3, 4, 4, 4 ] )}, 
    {Transformation( [ 4, 4, 4, 4 ] )}, {Transformation( [ 4, 4, 3, 4 ] )}, 
    {Transformation( [ 4, 3, 4, 4 ] )}, {Transformation( [ 4, 4, 4, 3 ] )} ]
  !gapprompt@gap>| !gapinput@D:=GreensDClasses(S)[2];|
  {Transformation( [ 3, 4, 4, 4 ] )}
  !gapprompt@gap>| !gapinput@GreensLClasses(D);|
  [ {Transformation( [ 3, 4, 4, 4 ] )}, {Transformation( [ 1, 2, 2, 2 ] )} ]
  !gapprompt@gap>| !gapinput@GreensRClasses(D);|
  [ {Transformation( [ 3, 4, 4, 4 ] )}, {Transformation( [ 4, 4, 3, 4 ] )}, 
    {Transformation( [ 4, 3, 4, 4 ] )}, {Transformation( [ 4, 4, 4, 3 ] )} ]
  !gapprompt@gap>| !gapinput@R:=GreensRClasses(D)[1];|
  {Transformation( [ 3, 4, 4, 4 ] )}
  !gapprompt@gap>| !gapinput@GreensHClasses(R);|
  [ {Transformation( [ 3, 4, 4, 4 ] )}, {Transformation( [ 1, 2, 2, 2 ] )} ]
  !gapprompt@gap>| !gapinput@s:=InverseSemigroup( PartialPermNC( [ 1, 2, 3 ], [ 2, 4, 1 ] ),|
  !gapprompt@>| !gapinput@PartialPermNC( [ 1, 3, 4 ], [ 3, 4, 1 ] ) );;|
  !gapprompt@gap>| !gapinput@GreensDClasses(s);|
  [ {<identity on [ 1, 2, 4 ]>}, {<identity on [ 1, 3, 4 ]>},
    {<identity on [ 2, 4 ]>}, {<identity on [ 4 ]>}, {<empty mapping>} ]
  !gapprompt@gap>| !gapinput@GreensLClasses(s);|
  [ {<identity on [ 1, 2, 4 ]>}, {[ 1, 2, 4 ] -> [ 3, 1, 2 ]},
    {<identity on [ 1, 3, 4 ]>}, {<identity on [ 2, 4 ]>},
    {[ 2, 4 ] -> [ 3, 1 ]}, {[ 2, 4 ] -> [ 1, 2 ]}, {[ 2, 4 ] -> [ 3, 2 ]},
    {[ 2, 4 ] -> [ 4, 3 ]}, {[ 2, 4 ] -> [ 1, 4 ]}, {<identity on [ 4 ]>},
    {[ 4 ] -> [ 1 ]}, {[ 4 ] -> [ 3 ]}, {[ 4 ] -> [ 2 ]}, {<empty mapping>} ]
  !gapprompt@gap>| !gapinput@D:=GreensDClasses(s)[3];|
  {<identity on [ 2, 4 ]>}
  !gapprompt@gap>| !gapinput@GreensLClasses(D);|
  [ {<identity on [ 2, 4 ]>}, {[ 2, 4 ] -> [ 3, 1 ]}, {[ 2, 4 ] -> [ 1, 2 ]},
    {[ 2, 4 ] -> [ 3, 2 ]}, {[ 2, 4 ] -> [ 4, 3 ]}, {[ 2, 4 ] -> [ 1, 4 ]} ]
  !gapprompt@gap>| !gapinput@GreensRClasses(D);|
  [ {<identity on [ 2, 4 ]>}, {[ 1, 3 ] -> [ 4, 2 ]}, {[ 1, 2 ] -> [ 2, 4 ]},
    {[ 2, 3 ] -> [ 4, 2 ]}, {[ 3, 4 ] -> [ 4, 2 ]}, {[ 1, 4 ] -> [ 2, 4 ]} ]
\end{Verbatim}
 }

 
\subsection{\textcolor{Chapter }{IteratorOfXClassReps}}\logpage{[ 4, 3, 2 ]}
\hyperdef{L}{X8566F84A7F6D4193}{}
{
\noindent\textcolor{FuncColor}{$\triangleright$\ \ \texttt{IteratorOfDClassReps({\mdseries\slshape S})\index{IteratorOfDClassReps@\texttt{IteratorOfDClassReps}}
\label{IteratorOfDClassReps}
}\hfill{\scriptsize (function)}}\\
\noindent\textcolor{FuncColor}{$\triangleright$\ \ \texttt{IteratorOfHClassReps({\mdseries\slshape S})\index{IteratorOfHClassReps@\texttt{IteratorOfHClassReps}}
\label{IteratorOfHClassReps}
}\hfill{\scriptsize (function)}}\\
\noindent\textcolor{FuncColor}{$\triangleright$\ \ \texttt{IteratorOfLClassReps({\mdseries\slshape S})\index{IteratorOfLClassReps@\texttt{IteratorOfLClassReps}}
\label{IteratorOfLClassReps}
}\hfill{\scriptsize (function)}}\\
\noindent\textcolor{FuncColor}{$\triangleright$\ \ \texttt{IteratorOfRClassReps({\mdseries\slshape S})\index{IteratorOfRClassReps@\texttt{IteratorOfRClassReps}}
\label{IteratorOfRClassReps}
}\hfill{\scriptsize (function)}}\\
\textbf{\indent Returns:\ }
 An iterator. 



 Returns an iterator of the representatives of the Green's classes contained in
the semigroup of transformations or partial permutations \mbox{\texttt{\mdseries\slshape S}}. See  (\textbf{Reference: Iterators}) for more information on iterators.

 See also \texttt{GreensRClasses} (\textbf{Reference: GreensRClasses}), \texttt{GreensRClasses} (\ref{GreensRClasses}), and \texttt{IteratorOfRClasses} (\ref{IteratorOfRClasses}).

 In the display of an \texttt{IteratorOfRClassReps}, for a transformation semigroup, the total number of elements in the orbit of
the semigroup acting on its $\mathcal{R}$-class representatives by multiplication on the left is given (every $\mathcal{R}$-class representative occurs in this orbit), the number of elements of the
semigroup that have so far been computed is given, and the number of distinct $\mathcal{R}$-classes that have been found is given. Note that these numbers depend on the
state of the semigroup and not of the iterator. 
\begin{Verbatim}[commandchars=!@|,fontsize=\small,frame=single,label=Example]
  !gapprompt@gap>| !gapinput@gens:=[ Transformation( [ 3, 2, 1, 5, 4 ] ), |
  !gapprompt@>| !gapinput@Transformation( [ 5, 4, 3, 2, 1 ] ), |
  !gapprompt@>| !gapinput@Transformation( [ 5, 4, 3, 2, 1 ] ), Transformation( [ 5, 5, 4, 5, 1 ] ), |
  !gapprompt@>| !gapinput@Transformation( [ 4, 5, 4, 3, 3 ] ) ];;|
  !gapprompt@gap>| !gapinput@S:=Semigroup(gens);;|
  !gapprompt@gap>| !gapinput@iter:=IteratorOfRClassReps(S);|
  <iterator of R-class reps, 6 candidates, 0 elements, 0 R-classes>
  !gapprompt@gap>| !gapinput@NextIterator(iter);|
  Transformation( [ 1, 2, 3, 4, 5 ] )
  !gapprompt@gap>| !gapinput@NextIterator(iter);|
  Transformation( [ 5, 5, 4, 5, 1 ] )
  !gapprompt@gap>| !gapinput@iter;|
  <iterator of R-class reps, 10 candidates, 20 elements, 2 R-classes>
  !gapprompt@gap>| !gapinput@file:=Concatenation(CitrusDir(), "/examples/inverse.citrus.gz");;|
  !gapprompt@gap>| !gapinput@s:=InverseSemigroup(ReadCitrus(file, 8));|
  <inverse semigroup with 2 generators>
  !gapprompt@gap>| !gapinput@NrMovedPoints(s);|
  983
  !gapprompt@gap>| !gapinput@iter:=IteratorOfLClassReps(s);|
  <iterator of L-class reps>
  !gapprompt@gap>| !gapinput@NextIterator(iter);|
  <partial perm on 634 pts>
\end{Verbatim}
 }

 
\subsection{\textcolor{Chapter }{IteratorOfXClasses}}\logpage{[ 4, 3, 3 ]}
\hyperdef{L}{X867D7B8982915960}{}
{
\noindent\textcolor{FuncColor}{$\triangleright$\ \ \texttt{IteratorOfDClasses({\mdseries\slshape S})\index{IteratorOfDClasses@\texttt{IteratorOfDClasses}}
\label{IteratorOfDClasses}
}\hfill{\scriptsize (function)}}\\
\noindent\textcolor{FuncColor}{$\triangleright$\ \ \texttt{IteratorOfHClasses({\mdseries\slshape S})\index{IteratorOfHClasses@\texttt{IteratorOfHClasses}}
\label{IteratorOfHClasses}
}\hfill{\scriptsize (function)}}\\
\noindent\textcolor{FuncColor}{$\triangleright$\ \ \texttt{IteratorOfLClasses({\mdseries\slshape S})\index{IteratorOfLClasses@\texttt{IteratorOfLClasses}}
\label{IteratorOfLClasses}
}\hfill{\scriptsize (function)}}\\
\noindent\textcolor{FuncColor}{$\triangleright$\ \ \texttt{IteratorOfRClasses({\mdseries\slshape S})\index{IteratorOfRClasses@\texttt{IteratorOfRClasses}}
\label{IteratorOfRClasses}
}\hfill{\scriptsize (function)}}\\
\textbf{\indent Returns:\ }
 An iterator. 



 Returns an iterator of the Green's classes in the semigroup of transformations
or partial permutations \mbox{\texttt{\mdseries\slshape S}}. See  (\textbf{Reference: Iterators}) for more information on iterators.

 This function is useful if you are, for example, looking for an $\mathcal{R}$-class of a semigroup with a particular property but do not necessarily want
to compute all of the $\mathcal{R}$-classes.

 See also \texttt{GreensRClasses} (\ref{GreensRClasses}), \texttt{GreensRClasses} (\textbf{Reference: GreensRClasses}), \texttt{NrRClasses} (\ref{NrRClasses}), and \texttt{IteratorOfRClassReps} (\ref{IteratorOfRClassReps}).

 The transformation semigroup in the example below has 25147892 elements but it
only takes a fraction of a second to find a non-trivial $\mathcal{R}$-class. The inverse semigroup of partial permutations in the example below has
size 158122047816 but it only takes a fraction of a second to find an $\mathcal{R}$-class with more than 1000 elements. 
\begin{Verbatim}[commandchars=!@|,fontsize=\small,frame=single,label=Example]
  !gapprompt@gap>| !gapinput@gens:=[ Transformation( [ 2, 4, 1, 5, 4, 4, 7, 3, 8, 1 ] ),|
  !gapprompt@>| !gapinput@  Transformation( [ 3, 2, 8, 8, 4, 4, 8, 6, 5, 7 ] ),|
  !gapprompt@>| !gapinput@  Transformation( [ 4, 10, 6, 6, 1, 2, 4, 10, 9, 7 ] ),|
  !gapprompt@>| !gapinput@  Transformation( [ 6, 2, 2, 4, 9, 9, 5, 10, 1, 8 ] ),|
  !gapprompt@>| !gapinput@  Transformation( [ 6, 4, 1, 6, 6, 8, 9, 6, 2, 2 ] ),|
  !gapprompt@>| !gapinput@  Transformation( [ 6, 8, 1, 10, 6, 4, 9, 1, 9, 4 ] ),|
  !gapprompt@>| !gapinput@  Transformation( [ 8, 6, 2, 3, 3, 4, 8, 6, 2, 9 ] ),|
  !gapprompt@>| !gapinput@  Transformation( [ 9, 1, 2, 8, 1, 5, 9, 9, 9, 5 ] ),|
  !gapprompt@>| !gapinput@  Transformation( [ 9, 3, 1, 5, 10, 3, 4, 6, 10, 2 ] ),|
  !gapprompt@>| !gapinput@  Transformation( [ 10, 7, 3, 7, 1, 9, 8, 8, 4, 10 ] ) ];;|
  !gapprompt@gap>| !gapinput@S:=Semigroup(gens);;|
  !gapprompt@gap>| !gapinput@iter:=IteratorOfRClasses(S);|
  <iterator of R-classes>
  !gapprompt@gap>| !gapinput@for R in iter do|
  !gapprompt@>| !gapinput@if Size(R)>1 then break; fi;|
  !gapprompt@>| !gapinput@od;|
  !gapprompt@gap>| !gapinput@R;|
  {Transformation( [ 6, 4, 1, 6, 6, 8, 9, 6, 2, 2 ] )}
  !gapprompt@gap>| !gapinput@Size(R);|
  21600
  !gapprompt@gap>| !gapinput@S:=InverseSemigroup(|
  !gapprompt@>| !gapinput@[ PartialPermNC( [ 1, 2, 3, 4, 5, 6, 7, 10, 11, 19, 20 ], |
  !gapprompt@>| !gapinput@[ 19, 4, 11, 15, 3, 20, 1, 14, 8, 13, 17 ] ),|
  !gapprompt@>| !gapinput@ PartialPermNC( [ 1, 2, 3, 4, 6, 7, 8, 14, 15, 16, 17 ], |
  !gapprompt@>| !gapinput@[ 15, 14, 20, 19, 4, 5, 1, 13, 11, 10, 3 ] ),|
  !gapprompt@>| !gapinput@ PartialPermNC( [ 1, 2, 4, 6, 7, 8, 9, 10, 14, 15, 18 ], |
  !gapprompt@>| !gapinput@[ 7, 2, 17, 10, 1, 19, 9, 3, 11, 16, 18 ] ),|
  !gapprompt@>| !gapinput@ PartialPermNC( [ 1, 2, 3, 4, 5, 7, 8, 9, 11, 12, 13, 16 ], |
  !gapprompt@>| !gapinput@[ 8, 3, 18, 1, 4, 13, 12, 7, 19, 20, 2, 11 ] ),|
  !gapprompt@>| !gapinput@ PartialPermNC( [ 1, 2, 3, 4, 5, 6, 7, 9, 11, 15, 16, 17, 20 ], |
  !gapprompt@>| !gapinput@[ 7, 17, 13, 4, 6, 9, 18, 10, 11, 19, 5, 2, 8 ] ),|
  !gapprompt@>| !gapinput@ PartialPermNC( [ 1, 3, 4, 5, 6, 7, 8, 9, 10, 11, 12, 15, 18 ], |
  !gapprompt@>| !gapinput@[ 10, 20, 11, 7, 13, 8, 4, 9, 2, 18, 17, 6, 15 ] ),|
  !gapprompt@>| !gapinput@ PartialPermNC( [ 1, 2, 3, 4, 5, 6, 7, 8, 9, 11, 13, 14, 17, 18 ], |
  !gapprompt@>| !gapinput@[ 10, 20, 18, 1, 14, 16, 9, 5, 15, 4, 8, 12, 19, 11 ] ),|
  !gapprompt@>| !gapinput@ PartialPermNC( [ 1, 2, 3, 4, 5, 6, 7, 9, 10, 11, 12, 15, 16, 19, 20 ], |
  !gapprompt@>| !gapinput@[ 13, 6, 1, 2, 11, 7, 16, 18, 9, 10, 4, 14, 15, 5, 17 ] ),|
  !gapprompt@>| !gapinput@ PartialPermNC( [ 1, 2, 3, 4, 6, 7, 8, 9, 10, 11, 12, 14, 15, 16, 20 ], |
  !gapprompt@>| !gapinput@[ 5, 3, 12, 9, 20, 15, 8, 16, 13, 1, 17, 11, 14, 10, 2 ] ),|
  !gapprompt@>| !gapinput@ PartialPermNC( [ 1, 2, 3, 4, 6, 7, 8, 9, 10, 11, 13, 17, 18, 19, 20 ], |
  !gapprompt@>| !gapinput@[ 8, 3, 9, 20, 2, 12, 14, 15, 4, 18, 13, 1, 17, 19, 5 ] ) ]);;|
  !gapprompt@gap>| !gapinput@iter:=IteratorOfRClasses(S);|
  <iterator of R-classes>
  !gapprompt@gap>| !gapinput@repeat r:=NextIterator(iter); until Size(r)>1000;|
  !gapprompt@gap>| !gapinput@r;|
  {<identity on [ 8, 11, 13, 15, 17, 19 ]>}
  !gapprompt@gap>| !gapinput@Size(r);|
  10020240
\end{Verbatim}
 }

 
\subsection{\textcolor{Chapter }{XClassReps}}\logpage{[ 4, 3, 4 ]}
\hyperdef{L}{X865387A87FAAC395}{}
{
\noindent\textcolor{FuncColor}{$\triangleright$\ \ \texttt{DClassReps({\mdseries\slshape obj})\index{DClassReps@\texttt{DClassReps}}
\label{DClassReps}
}\hfill{\scriptsize (attribute)}}\\
\noindent\textcolor{FuncColor}{$\triangleright$\ \ \texttt{HClassReps({\mdseries\slshape obj})\index{HClassReps@\texttt{HClassReps}}
\label{HClassReps}
}\hfill{\scriptsize (attribute)}}\\
\noindent\textcolor{FuncColor}{$\triangleright$\ \ \texttt{LClassReps({\mdseries\slshape obj})\index{LClassReps@\texttt{LClassReps}}
\label{LClassReps}
}\hfill{\scriptsize (attribute)}}\\
\noindent\textcolor{FuncColor}{$\triangleright$\ \ \texttt{RClassReps({\mdseries\slshape obj})\index{RClassReps@\texttt{RClassReps}}
\label{RClassReps}
}\hfill{\scriptsize (attribute)}}\\
\textbf{\indent Returns:\ }
A list of transformations.



 \texttt{XClassReps} returns a list of the representatives of the Green's classes of \mbox{\texttt{\mdseries\slshape obj}}, which can be a semigroup of transformations or partial permutations, $\mathcal{D}$-, $\mathcal{L}$-, or $\mathcal{R}$-class (where appropriate).

 The same output can be obtained by calling, for example: 
\begin{Verbatim}[commandchars=!@|,fontsize=\small,frame=single,label=Example]
  List(GreensXClasses(obj), Representative);
\end{Verbatim}
 Note that if the Green's classes themselves are not required, then \texttt{XClassReps} will return an answer more quickly than the above, since the Green's class
objects are not created.

 See also \texttt{GreensDClasses} (\ref{GreensDClasses}), \texttt{IteratorOfDClassReps} (\ref{IteratorOfDClassReps}), \texttt{IteratorOfDClasses} (\ref{IteratorOfDClasses}), and \texttt{NrDClasses} (\ref{NrDClasses}). 
\begin{Verbatim}[commandchars=!@|,fontsize=\small,frame=single,label=Example]
  !gapprompt@gap>| !gapinput@S:=Semigroup(Transformation( [ 3, 4, 4, 4 ] ),|
  !gapprompt@>| !gapinput@Transformation( [ 4, 3, 1, 2 ] ));;|
  !gapprompt@gap>| !gapinput@DClassReps(S);|
  [ Transformation( [ 1, 2, 3, 4 ] ), Transformation( [ 3, 4, 4, 4 ] ), 
    Transformation( [ 4, 4, 4, 4 ] ) ]
  !gapprompt@gap>| !gapinput@LClassReps(S);|
  [ Transformation( [ 1, 2, 3, 4 ] ), Transformation( [ 3, 4, 4, 4 ] ), 
    Transformation( [ 1, 2, 2, 2 ] ), Transformation( [ 4, 4, 4, 4 ] ), 
    Transformation( [ 2, 2, 2, 2 ] ), Transformation( [ 3, 3, 3, 3 ] ), 
    Transformation( [ 1, 1, 1, 1 ] ) ]
  !gapprompt@gap>| !gapinput@D:=GreensDClasses(S)[2];|
  {Transformation( [ 3, 4, 4, 4 ] )}
  !gapprompt@gap>| !gapinput@LClassReps(D);|
  [ Transformation( [ 3, 4, 4, 4 ] ), Transformation( [ 1, 2, 2, 2 ] ) ]
  !gapprompt@gap>| !gapinput@RClassReps(D);|
  [ Transformation( [ 3, 4, 4, 4 ] ), Transformation( [ 4, 4, 3, 4 ] ), 
    Transformation( [ 4, 3, 4, 4 ] ), Transformation( [ 4, 4, 4, 3 ] ) ]
  !gapprompt@gap>| !gapinput@R:=GreensRClasses(D)[1];;|
  !gapprompt@gap>| !gapinput@HClassReps(R);|
  [ Transformation( [ 3, 4, 4, 4 ] ), Transformation( [ 1, 2, 2, 2 ] ) ]
  !gapprompt@gap>| !gapinput@S:=SymmetricInverseSemigp(6);;|
  !gapprompt@gap>| !gapinput@e:=InverseSemigroup(Idempotents(S));;|
  !gapprompt@gap>| !gapinput@M:=MunnSemigroup(e);;|
  !gapprompt@gap>| !gapinput@DClassReps(M);|
  [ <partial perm on 64 pts>, <identity on [ 51 ]>, <identity on [ 27, 51 ]>, 
    <identity on [ 15, 27, 50, 51 ]>, 
    <identity on [ 8, 15, 26, 27, 49, 50, 51, 64 ]>, 
    <identity on [ 4, 8, 14, 15, 25, 26, 27, 48, 49, 50, 51, 60, 61, 62, 63, 64 
       ]>, <partial perm on 32 pts> ]
  !gapprompt@gap>| !gapinput@L:=LClassNC(M, PartialPermNC( [ 51, 63 ] , [ 51, 47 ] ));;|
  !gapprompt@gap>| !gapinput@HClassReps(L);|
  [ [ 27, 51 ] -> [ 47, 51 ], <identity on [ 47, 51 ]>, 
    [ 50, 51 ] -> [ 47, 51 ], [ 51, 59 ] -> [ 51, 47 ], 
    [ 51, 63 ] -> [ 51, 47 ], [ 51, 64 ] -> [ 51, 47 ] ]
\end{Verbatim}
 }

 }

 
\section{\textcolor{Chapter }{Attributes and properties directly related to Green's classes}}\logpage{[ 4, 4, 0 ]}
\hyperdef{L}{X798C0DF184E51D7F}{}
{
 
\subsection{\textcolor{Chapter }{Less than for Green's classes}}\logpage{[ 4, 4, 1 ]}
\hyperdef{L}{X85F30ACF86C3A733}{}
{
\noindent\textcolor{FuncColor}{$\triangleright$\ \ \texttt{\texttt{\symbol{92}}{\textless}({\mdseries\slshape left-expr, right-expr})\index{<@\texttt{\texttt{\symbol{92}}{\textless}}!for Green's classes}
\label{<:for Green's classes}
}\hfill{\scriptsize (method)}}\\
\textbf{\indent Returns:\ }
\texttt{true} or \texttt{false}.



 The Green's class \mbox{\texttt{\mdseries\slshape left-expr}} is less than or equal to \mbox{\texttt{\mdseries\slshape right-expr}} if they belong to the same semigroup and the representative of \mbox{\texttt{\mdseries\slshape left-expr}} is less than the representative of \mbox{\texttt{\mdseries\slshape right-expr}} under \texttt{{\textless}}; see also \texttt{Representative} (\textbf{Reference: Representative}), and Sections \ref{OperatorsT} and \ref{OperatorsPP}.

 Please note that this is not the usual order on the Green's classes of a
semigroup as defined in  (\textbf{Reference: Green's Relations}). See also \texttt{IsGreensLessThanOrEqual} (\textbf{Reference: IsGreensLessThanOrEqual}). 
\begin{Verbatim}[commandchars=!@|,fontsize=\small,frame=single,label=Example]
  !gapprompt@gap>| !gapinput@S:=FullTransformationSemigroup(4);;|
  !gapprompt@gap>| !gapinput@A:=GreensRClassOfElement(S, Transformation( [ 2, 1, 3, 1 ] ));|
  {Transformation( [ 2, 1, 3, 1 ] )}
  !gapprompt@gap>| !gapinput@B:=GreensRClassOfElement(S, Transformation( [ 1, 2, 3, 4 ] ));|
  {Transformation( [ 1, 2, 3, 4 ] )}
  !gapprompt@gap>| !gapinput@A<B;|
  false
  !gapprompt@gap>| !gapinput@B<A;|
  true
  !gapprompt@gap>| !gapinput@IsGreensLessThanOrEqual(A,B);|
  true
  !gapprompt@gap>| !gapinput@IsGreensLessThanOrEqual(B,A);|
  false
  !gapprompt@gap>| !gapinput@s:=SymmetricInverseSemigp(4);;|
  !gapprompt@gap>| !gapinput@A:=GreensJClassOfElement(s, PartialPermNC([ 1 .. 3 ], [ 1, 3, 4 ]) );|
  {<identity on [ 1 .. 3 ]>}
  !gapprompt@gap>| !gapinput@B:=GreensJClassOfElement(s, PartialPermNC([ 1, 2 ], [ 3, 1 ]) );|
  {<identity on [ 1, 2 ]>}
  !gapprompt@gap>| !gapinput@A<B;|
  false
  !gapprompt@gap>| !gapinput@B<A;|
  true
  !gapprompt@gap>| !gapinput@IsGreensLessThanOrEqual(A, B);|
  false
  !gapprompt@gap>| !gapinput@IsGreensLessThanOrEqual(B, A);|
  true
\end{Verbatim}
 }

 

\subsection{\textcolor{Chapter }{InjectionPrincipalFactor}}
\logpage{[ 4, 4, 2 ]}\nobreak
\hyperdef{L}{X7EBB4F1981CC2AE9}{}
{\noindent\textcolor{FuncColor}{$\triangleright$\ \ \texttt{InjectionPrincipalFactor({\mdseries\slshape D})\index{InjectionPrincipalFactor@\texttt{InjectionPrincipalFactor}}
\label{InjectionPrincipalFactor}
}\hfill{\scriptsize (attribute)}}\\
\noindent\textcolor{FuncColor}{$\triangleright$\ \ \texttt{IsomorphismReesMatrixSemigroup({\mdseries\slshape D})\index{IsomorphismReesMatrixSemigroup@\texttt{IsomorphismReesMatrixSemigroup}}
\label{IsomorphismReesMatrixSemigroup}
}\hfill{\scriptsize (attribute)}}\\
\textbf{\indent Returns:\ }
A injective mapping.



 The argument \mbox{\texttt{\mdseries\slshape D}} should be a $\mathcal{D}$-class of a semigroup of transformations or partial permutations. If \mbox{\texttt{\mdseries\slshape D}} is a subsemigroup of \texttt{S}, then the \emph{principal factor} of \mbox{\texttt{\mdseries\slshape D}} is just \mbox{\texttt{\mdseries\slshape D}} itself. If \mbox{\texttt{\mdseries\slshape D}} is not a subsemigroup of \texttt{S}, then the principal factor of \mbox{\texttt{\mdseries\slshape D}} is the semigroup with elements \mbox{\texttt{\mdseries\slshape D}} and a new element \texttt{0} with multiplication of $x,y\in D$ defined by:  
\[ xy=\left\{\begin{array}{ll} x*y\ (\textrm{in }S)&\textrm{if }x*y\in D\\
0&\textrm{if }xy\not\in D. \end{array}\right. \]
   \texttt{InjectionPrincipalFactor} returns an injective function from the $\mathcal{D}$-class \mbox{\texttt{\mdseries\slshape D}} to a Rees matrix semigroup, which contains the principal factor of \mbox{\texttt{\mdseries\slshape D}} as a subsemigroup. 

 If \mbox{\texttt{\mdseries\slshape D}} is a subsemigroup of its parent semigroup, then the function returned by \texttt{InjectionPrincipalFactor} or \texttt{IsomorphismReesMatrixSemigroup} is an isomorphism from \mbox{\texttt{\mdseries\slshape D}} to a Rees matrix semigroup; see \texttt{ReesMatrixSemigroup} (\textbf{Reference: ReesMatrixSemigroup}).

 If \mbox{\texttt{\mdseries\slshape D}} is not a semigroup, then the function returned by \texttt{InjectionPrincipalFactor} is an injective function from \mbox{\texttt{\mdseries\slshape D}} to a Rees 0-matrix semigroup isomorphic to the principal factor of \mbox{\texttt{\mdseries\slshape D}}; see \texttt{ReesZeroMatrixSemigroup} (\textbf{Reference: ReesZeroMatrixSemigroup}). In this case, \texttt{IsomorphismReesMatrixSemigroup} returns an error.

 See also \texttt{PrincipalFactor} (\ref{PrincipalFactor}). 
\begin{Verbatim}[commandchars=!@|,fontsize=\small,frame=single,label=Example]
  !gapprompt@gap>| !gapinput@S:=InverseSemigroup(|
  !gapprompt@>| !gapinput@PartialPermNC( [ 1, 2, 3, 6, 8, 10 ], [ 2, 6, 7, 9, 1, 5 ] ),|
  !gapprompt@>| !gapinput@PartialPermNC( [ 1, 2, 3, 4, 6, 7, 8, 10 ], |
  !gapprompt@>| !gapinput@[ 3, 8, 1, 9, 4, 10, 5, 6 ] ) );;|
  !gapprompt@gap>| !gapinput@f:=PartialPermNC([ 1, 2, 5, 6, 7, 9 ], [ 1, 2, 5, 6, 7, 9 ]);;|
  !gapprompt@gap>| !gapinput@d:=GreensDClassOfElement(S, f);|
  {<identity on [ 1, 2, 5, 6, 7, 9 ]>}
  !gapprompt@gap>| !gapinput@InjectionPrincipalFactor(d);|
  MappingByFunction( {<identity on [ 1, 2, 5, 6, 7, 9 
   ]>}, Rees Zero Matrix Semigroup over <zero group with 
  2 generators>, function( f ) ... end, function( x ) ... end )
  !gapprompt@gap>| !gapinput@rms:=Range(last);|
  Rees Zero Matrix Semigroup over <zero group with 2 generators>
  !gapprompt@gap>| !gapinput@SandwichMatrixOfReesZeroMatrixSemigroup(rms);|
  [ [ (), 0, 0 ], [ 0, (), 0 ], [ 0, 0, () ] ]
  !gapprompt@gap>| !gapinput@Size(rms);|
  10
  !gapprompt@gap>| !gapinput@Size(d);|
  9
\end{Verbatim}
 }

 

\subsection{\textcolor{Chapter }{PrincipalFactor}}
\logpage{[ 4, 4, 3 ]}\nobreak
\hyperdef{L}{X86C6D777847AAEC7}{}
{\noindent\textcolor{FuncColor}{$\triangleright$\ \ \texttt{PrincipalFactor({\mdseries\slshape D})\index{PrincipalFactor@\texttt{PrincipalFactor}}
\label{PrincipalFactor}
}\hfill{\scriptsize (attribute)}}\\
\textbf{\indent Returns:\ }
A Rees matrix semigroup.



 \texttt{PrincipalFactor(\mbox{\texttt{\mdseries\slshape D}})} is just shorthand for \texttt{Range(InjectionPrincipalFactor(\mbox{\texttt{\mdseries\slshape D}}))}; see \texttt{InjectionPrincipalFactor} (\ref{InjectionPrincipalFactor}) for more details. }

 

\subsection{\textcolor{Chapter }{IsGreensClassOfTransSemigp}}
\logpage{[ 4, 4, 4 ]}\nobreak
\hyperdef{L}{X87B94FE27E06F5D8}{}
{\noindent\textcolor{FuncColor}{$\triangleright$\ \ \texttt{IsGreensClassOfTransSemigp({\mdseries\slshape obj})\index{IsGreensClassOfTransSemigp@\texttt{IsGreensClassOfTransSemigp}}
\label{IsGreensClassOfTransSemigp}
}\hfill{\scriptsize (property)}}\\
\textbf{\indent Returns:\ }
\texttt{true} or \texttt{false}.



 \texttt{IsGreensClassOfTransSemigp} return \texttt{true} if the object \mbox{\texttt{\mdseries\slshape obj}} is a Green's class of a transformation semigroup and \texttt{false} if it is not. 

 This property is required so that a Green's class knows that it belongs to a
transformation semigroup, so that the methods defined in the \textsf{Citrus} are used in preference to those in the library. 
\begin{Verbatim}[commandchars=!@|,fontsize=\small,frame=single,label=Example]
  !gapprompt@gap>| !gapinput@S:=Semigroup(Transformation( [ 2, 1, 4, 5, 6, 3 ] ), |
  !gapprompt@>| !gapinput@Transformation( [ 2, 3, 1, 5, 4, 1 ] ));;|
  !gapprompt@gap>| !gapinput@L:=GreensLClassOfElement(S, GeneratorsOfSemigroup(S)[1]);|
  {Transformation( [ 2, 1, 4, 5, 6, 3 ] )}
  !gapprompt@gap>| !gapinput@IsGreensClassOfTransSemigp(L);|
  true
\end{Verbatim}
 }

 

\subsection{\textcolor{Chapter }{IsGreensClassOfPartialPermSemigp}}
\logpage{[ 4, 4, 5 ]}\nobreak
\hyperdef{L}{X87A50B4E80D1B870}{}
{\noindent\textcolor{FuncColor}{$\triangleright$\ \ \texttt{IsGreensClassOfPartialPermSemigp({\mdseries\slshape obj})\index{IsGreensClassOfPartialPermSemigp@\texttt{IsGreensClassOfPartialPermSemigp}}
\label{IsGreensClassOfPartialPermSemigp}
}\hfill{\scriptsize (property)}}\\
\textbf{\indent Returns:\ }
\texttt{true} or \texttt{false}.



 \texttt{IsGreensClassOfPartialPermSemigp} return \texttt{true} if the object \mbox{\texttt{\mdseries\slshape obj}} is a Green's class of a semigroup of partial permutations and \texttt{false} if it is not. 

 This property is required so that a Green's class knows that it belongs to a
semigroup of partial permutations, so that the methods defined in the \textsf{Citrus} are used in preference to those in the library. 
\begin{Verbatim}[commandchars=!@|,fontsize=\small,frame=single,label=Example]
  !gapprompt@gap>| !gapinput@S:=InverseSemigroup( PartialPermNC( [ 1, 3 ], [ 4, 3 ] ),|
  !gapprompt@>| !gapinput@PartialPermNC( [ 1, 2, 3 ], [ 4, 1, 2 ] ) );;|
  !gapprompt@gap>| !gapinput@D:=GreensDClassOfElement(S, PartialPermNC( [ 1, 3 ], [ 4, 3 ] ) );        |
  {<identity on [ 1, 3 ]>}
  !gapprompt@gap>| !gapinput@IsGreensClassOfPartPermSemigroup(D);|
  true
\end{Verbatim}
 }

 
\subsection{\textcolor{Chapter }{IsRegularXClass}}\logpage{[ 4, 4, 6 ]}
\hyperdef{L}{X78192CBC7F74B419}{}
{
\noindent\textcolor{FuncColor}{$\triangleright$\ \ \texttt{IsRegularDClass({\mdseries\slshape class})\index{IsRegularDClass@\texttt{IsRegularDClass}}
\label{IsRegularDClass}
}\hfill{\scriptsize (property)}}\\
\noindent\textcolor{FuncColor}{$\triangleright$\ \ \texttt{IsRegularLClass({\mdseries\slshape class})\index{IsRegularLClass@\texttt{IsRegularLClass}}
\label{IsRegularLClass}
}\hfill{\scriptsize (property)}}\\
\noindent\textcolor{FuncColor}{$\triangleright$\ \ \texttt{IsRegularRClass({\mdseries\slshape class})\index{IsRegularRClass@\texttt{IsRegularRClass}}
\label{IsRegularRClass}
}\hfill{\scriptsize (property)}}\\
\textbf{\indent Returns:\ }
 \texttt{true} or \texttt{false}. 



 This function returns \texttt{true} if \mbox{\texttt{\mdseries\slshape class}} is a regular Green's class and \texttt{false} if it is not. See also \texttt{IsRegularDClass} (\textbf{Reference: IsRegularDClass}), \texttt{IsGroupHClass} (\textbf{Reference: IsGroupHClass}), \texttt{GroupHClassOfGreensDClass} (\textbf{Reference: GroupHClassOfGreensDClass}), \texttt{GroupHClass} (\ref{GroupHClass}), \texttt{NrIdempotents} (\ref{NrIdempotents}), \texttt{Idempotents} (\ref{Idempotents}), and \texttt{IsRegularTransformation} (\ref{IsRegularTransformation}). 

 The function \texttt{IsRegularDClass} produces the same output as the \textsf{GAP} library functions with the same name; see \texttt{IsRegularDClass} (\textbf{Reference: IsRegularDClass}). 
\begin{Verbatim}[commandchars=!@|,fontsize=\small,frame=single,label=Example]
  !gapprompt@gap>| !gapinput@S:=Monoid(Transformation( [ 10, 8, 7, 4, 1, 4, 10, 10, 7, 2 ] ),|
  !gapprompt@>| !gapinput@Transformation( [ 5, 2, 5, 5, 9, 10, 8, 3, 8, 10 ] ));;|
  !gapprompt@gap>| !gapinput@f:=Transformation( [ 1, 1, 10, 8, 8, 8, 1, 1, 10, 8 ] );;|
  !gapprompt@gap>| !gapinput@R:=RClass(S, f);;|
  !gapprompt@gap>| !gapinput@IsRegularRClass(R);|
  true
  !gapprompt@gap>| !gapinput@S:=Monoid(Transformation([2,3,4,5,1,8,7,6,2,7]), |
  !gapprompt@>| !gapinput@Transformation( [ 3, 8, 7, 4, 1, 4, 3, 3, 7, 2 ] ));;|
  !gapprompt@gap>| !gapinput@f:=Transformation( [ 3, 8, 7, 4, 1, 4, 3, 3, 7, 2 ] );;|
  !gapprompt@gap>| !gapinput@R:=RClass(S, f);;|
  !gapprompt@gap>| !gapinput@IsRegularRClass(R);|
  false
  !gapprompt@gap>| !gapinput@NrIdempotents(R);|
  0
  !gapprompt@gap>| !gapinput@S:=Semigroup(Transformation( [ 2, 1, 3, 1 ] ), |
  !gapprompt@>| !gapinput@Transformation( [ 3, 1, 2, 1 ] ), Transformation( [ 4, 2, 3, 3 ] ));;|
  !gapprompt@gap>| !gapinput@f:=Transformation( [ 4, 2, 3, 3 ] );;|
  !gapprompt@gap>| !gapinput@L:=GreensLClassOfElement(S, f);;|
  !gapprompt@gap>| !gapinput@IsRegularLClass(L);|
  false
  !gapprompt@gap>| !gapinput@R:=GreensRClassOfElement(S, f);;|
  !gapprompt@gap>| !gapinput@IsRegularRClass(R);|
  false
  !gapprompt@gap>| !gapinput@g:=Transformation( [ 4, 4, 4, 4 ] );;|
  !gapprompt@gap>| !gapinput@IsRegularTransformation(S, g);|
  true
  !gapprompt@gap>| !gapinput@IsRegularLClass(LClass(S, g));|
  true
  !gapprompt@gap>| !gapinput@IsRegularLClass(RClass(S, g));|
  false
  !gapprompt@gap>| !gapinput@IsRegularRClass(RClass(S, g));|
  true
  !gapprompt@gap>| !gapinput@IsRegularDClass(DClass(S, g));|
  true
  !gapprompt@gap>| !gapinput@DClass(S, g)=RClass(S, g);|
  true
\end{Verbatim}
 }

 

\subsection{\textcolor{Chapter }{NrRegularDClasses}}
\logpage{[ 4, 4, 7 ]}\nobreak
\hyperdef{L}{X7AA3F0A77D0043FB}{}
{\noindent\textcolor{FuncColor}{$\triangleright$\ \ \texttt{NrRegularDClasses({\mdseries\slshape S})\index{NrRegularDClasses@\texttt{NrRegularDClasses}}
\label{NrRegularDClasses}
}\hfill{\scriptsize (attribute)}}\\
\textbf{\indent Returns:\ }
 A positive integer. 



 \texttt{NrRegularDClasses} returns the number of regular $\mathcal{D}$-classes of the semigroup of transformations or partial permutations \mbox{\texttt{\mdseries\slshape S}}. See also \texttt{IsRegularDClass} (\ref{IsRegularDClass}) and \texttt{IsRegularDClass} (\textbf{Reference: IsRegularDClass}). 
\begin{Verbatim}[commandchars=!@|,fontsize=\small,frame=single,label=Example]
  !gapprompt@gap>| !gapinput@S:=Semigroup( [ Transformation( [ 1, 3, 4, 1, 3, 5 ] ), |
  !gapprompt@>| !gapinput@Transformation( [ 5, 1, 6, 1, 6, 3 ] ) ]);;|
  !gapprompt@gap>| !gapinput@NrRegularDClasses(S); |
  3 
  !gapprompt@gap>| !gapinput@NrDClasses(S); |
  7
\end{Verbatim}
 }

 
\subsection{\textcolor{Chapter }{NrXClasses}}\logpage{[ 4, 4, 8 ]}
\hyperdef{L}{X7E45FD9F7BADDFBD}{}
{
\noindent\textcolor{FuncColor}{$\triangleright$\ \ \texttt{NrDClasses({\mdseries\slshape obj})\index{NrDClasses@\texttt{NrDClasses}}
\label{NrDClasses}
}\hfill{\scriptsize (attribute)}}\\
\noindent\textcolor{FuncColor}{$\triangleright$\ \ \texttt{NrHClasses({\mdseries\slshape obj})\index{NrHClasses@\texttt{NrHClasses}}
\label{NrHClasses}
}\hfill{\scriptsize (attribute)}}\\
\noindent\textcolor{FuncColor}{$\triangleright$\ \ \texttt{NrLClasses({\mdseries\slshape obj})\index{NrLClasses@\texttt{NrLClasses}}
\label{NrLClasses}
}\hfill{\scriptsize (attribute)}}\\
\noindent\textcolor{FuncColor}{$\triangleright$\ \ \texttt{NrRClasses({\mdseries\slshape obj})\index{NrRClasses@\texttt{NrRClasses}}
\label{NrRClasses}
}\hfill{\scriptsize (attribute)}}\\
\textbf{\indent Returns:\ }
 A positive integer. 



 \texttt{NrXClasses} returns the number of Green's classes in \mbox{\texttt{\mdseries\slshape obj}} where \mbox{\texttt{\mdseries\slshape obj}} can be a semigroup of transformations or partial permutations, $\mathcal{D}$-, $\mathcal{L}$-, or $\mathcal{R}$-class of a semigroup of transformations or partial permutations (where
appropriate). If the actual Green's classes are not required, then it is more
efficient to use 
\begin{Verbatim}[commandchars=!@|,fontsize=\small,frame=single,label=Example]
  NrHClasses(obj)
\end{Verbatim}
 than 
\begin{Verbatim}[commandchars=!@|,fontsize=\small,frame=single,label=Example]
  Length(HClasses(obj))
\end{Verbatim}
 since the Green's classes themselves are not created when \texttt{NrXClasses} is called. 

 See also \texttt{GreensRClasses} (\ref{GreensRClasses}), \texttt{GreensRClasses} (\textbf{Reference: GreensRClasses}), \texttt{IteratorOfRClasses} (\ref{IteratorOfRClasses}), and \texttt{IteratorOfRClassReps} (\ref{IteratorOfRClassReps}). 
\begin{Verbatim}[commandchars=!@|,fontsize=\small,frame=single,label=Example]
  !gapprompt@gap>| !gapinput@gens:=[ Transformation( [ 1, 2, 5, 4, 3, 8, 7, 6 ] ),|
  !gapprompt@>| !gapinput@  Transformation( [ 1, 6, 3, 4, 7, 2, 5, 8 ] ),|
  !gapprompt@>| !gapinput@  Transformation( [ 2, 1, 6, 7, 8, 3, 4, 5 ] ),|
  !gapprompt@>| !gapinput@  Transformation( [ 3, 2, 3, 6, 1, 6, 1, 2 ] ),|
  !gapprompt@>| !gapinput@  Transformation( [ 5, 2, 3, 6, 3, 4, 7, 4 ] ) ];;|
  !gapprompt@gap>| !gapinput@S:=Semigroup(gens);;|
  !gapprompt@gap>| !gapinput@f:=Transformation( [ 2, 5, 4, 7, 4, 3, 6, 3 ] );;|
  !gapprompt@gap>| !gapinput@R:=RClass(S, f);|
  {Transformation( [ 5, 2, 3, 6, 3, 4, 7, 4 ] )}
  !gapprompt@gap>| !gapinput@NrHClasses(R);|
  12
  !gapprompt@gap>| !gapinput@D:=DClass(R);|
  {Transformation( [ 5, 2, 3, 6, 3, 4, 7, 4 ] )}
  !gapprompt@gap>| !gapinput@NrHClasses(D);|
  72
  !gapprompt@gap>| !gapinput@L:=LClass(S, f);|
  {Transformation( [ 5, 2, 3, 6, 3, 4, 7, 4 ] )}
  !gapprompt@gap>| !gapinput@NrHClasses(L);|
  6 
  !gapprompt@gap>| !gapinput@NrHClasses(S);|
  1555
  !gapprompt@gap>| !gapinput@gens:=[ Transformation( [ 4, 6, 5, 2, 1, 3 ] ),|
  !gapprompt@>| !gapinput@  Transformation( [ 6, 3, 2, 5, 4, 1 ] ),|
  !gapprompt@>| !gapinput@  Transformation( [ 1, 2, 4, 3, 5, 6 ] ),|
  !gapprompt@>| !gapinput@  Transformation( [ 3, 5, 6, 1, 2, 3 ] ),|
  !gapprompt@>| !gapinput@  Transformation( [ 5, 3, 6, 6, 6, 2 ] ),|
  !gapprompt@>| !gapinput@  Transformation( [ 2, 3, 2, 6, 4, 6 ] ),|
  !gapprompt@>| !gapinput@  Transformation( [ 2, 1, 2, 2, 2, 4 ] ),|
  !gapprompt@>| !gapinput@  Transformation( [ 4, 4, 1, 2, 1, 2 ] ) ];;|
  !gapprompt@gap>| !gapinput@S:=Semigroup(gens);;|
  !gapprompt@gap>| !gapinput@NrRClasses(S);|
  150
  !gapprompt@gap>| !gapinput@Size(S);|
  6342
  !gapprompt@gap>| !gapinput@f:=Transformation( [ 1, 3, 3, 1, 3, 5 ] );;|
  !gapprompt@gap>| !gapinput@D:=DClass(S, f);|
  {Transformation( [ 2, 1, 1, 2, 1, 4 ] )}
  !gapprompt@gap>| !gapinput@NrRClasses(D);|
  87
  !gapprompt@gap>| !gapinput@s:=SymmetricInverseSemigp(10);;|
  !gapprompt@gap>| !gapinput@NrDClasses(s); NrRClasses(s); NrHClasses(s); NrLClasses(s);|
  11
  1024
  184756
  1024
  !gapprompt@gap>| !gapinput@s:=POPI(10);;|
  !gapprompt@gap>| !gapinput@NrDClasses(s);|
  11
  !gapprompt@gap>| !gapinput@NrRClasses(s);|
  1024
\end{Verbatim}
 }

 

\subsection{\textcolor{Chapter }{PartialOrderOfDClasses}}
\logpage{[ 4, 4, 9 ]}\nobreak
\hyperdef{L}{X83F1C306846DF26B}{}
{\noindent\textcolor{FuncColor}{$\triangleright$\ \ \texttt{PartialOrderOfDClasses({\mdseries\slshape S})\index{PartialOrderOfDClasses@\texttt{PartialOrderOfDClasses}}
\label{PartialOrderOfDClasses}
}\hfill{\scriptsize (attribute)}}\\
\textbf{\indent Returns:\ }
The partial order of the $\mathcal{D}$-classes of \mbox{\texttt{\mdseries\slshape S}}. 



 Returns a list \texttt{l} where \texttt{j} is in \texttt{l[i]} if and only if \texttt{GreensDClasses(S)[j]} is immediately less than \texttt{GreensDClasses(S)[i]} in the partial order of $\mathcal{D}$- classes of \mbox{\texttt{\mdseries\slshape S}}. The transitive closure of the relation $\{(j,i): j\in l[i]\}$ is the partial order of $\mathcal{D}$-classes of \mbox{\texttt{\mdseries\slshape S}}. 

 The partial order on the $\mathcal{D}$-classes is defined by $x\leq y$ if and only if $S^1xS^1$ is a subset of $S^1yS^1$. 

 See also \texttt{GreensDClasses} (\ref{GreensDClasses}), \texttt{GreensDClasses} (\textbf{Reference: GreensDClasses}), \texttt{IsGreensLessThanOrEqual} (\textbf{Reference: IsGreensLessThanOrEqual}), and \texttt{\texttt{\symbol{92}}{\textless}} (\ref{<:for Green's classes}). 
\begin{Verbatim}[commandchars=!@|,fontsize=\small,frame=single,label=Example]
  !gapprompt@gap>| !gapinput@S:=Semigroup( Transformation( [ 2, 4, 1, 2 ] ), |
  !gapprompt@>| !gapinput@Transformation( [ 3, 3, 4, 1 ] ) );;|
  !gapprompt@gap>| !gapinput@PartialOrderOfDClasses(S);|
  [ [ 3 ], [ 2, 3 ], [ 3, 4 ], [ 4 ] ]
  !gapprompt@gap>| !gapinput@IsGreensLessThanOrEqual(GreensDClasses(S)[1], GreensDClasses(S)[2]);|
  false
  !gapprompt@gap>| !gapinput@IsGreensLessThanOrEqual(GreensDClasses(S)[2], GreensDClasses(S)[1]);|
  false
  !gapprompt@gap>| !gapinput@IsGreensLessThanOrEqual(GreensDClasses(S)[3], GreensDClasses(S)[1]);|
  true
  !gapprompt@gap>| !gapinput@S:=InverseSemigroup( PartialPermNC( [ 1, 2, 3 ], [ 1, 3, 4 ] ),|
  !gapprompt@>| !gapinput@PartialPermNC( [ 1, 3, 5 ], [ 5, 1, 3 ] ) );;|
  !gapprompt@gap>| !gapinput@Size(S);|
  58
  !gapprompt@gap>| !gapinput@PartialOrderOfDClasses(S);              |
  [ [ 1, 3 ], [ 2, 3 ], [ 3, 4 ], [ 4, 5 ], [ 5 ] ]
  !gapprompt@gap>| !gapinput@IsGreensLessThanOrEqual(GreensDClasses(S)[1], GreensDClasses(S)[2]);|
  false
  !gapprompt@gap>| !gapinput@IsGreensLessThanOrEqual(GreensDClasses(S)[5], GreensDClasses(S)[2]);|
  true
  !gapprompt@gap>| !gapinput@IsGreensLessThanOrEqual(GreensDClasses(S)[3], GreensDClasses(S)[4]);|
  false
  !gapprompt@gap>| !gapinput@IsGreensLessThanOrEqual(GreensDClasses(S)[4], GreensDClasses(S)[3]);|
  true
\end{Verbatim}
 }

 

\subsection{\textcolor{Chapter }{SchutzenbergerGroup}}
\logpage{[ 4, 4, 10 ]}\nobreak
\hyperdef{L}{X84F1321E8217D2A8}{}
{\noindent\textcolor{FuncColor}{$\triangleright$\ \ \texttt{SchutzenbergerGroup({\mdseries\slshape class})\index{SchutzenbergerGroup@\texttt{SchutzenbergerGroup}}
\label{SchutzenbergerGroup}
}\hfill{\scriptsize (attribute)}}\\
\textbf{\indent Returns:\ }
 A permutation group. 



 \texttt{SchutzenbergerGroup} returns the generalized Schutzenberger group (defined below) of the $\mathcal{R}$-, $\mathcal{D}$-, $\mathcal{L}$-, or $\mathcal{H}$-class \mbox{\texttt{\mdseries\slshape class}} of a semigroup of transformations or partial permutations. 

 If \texttt{f} is an element of a semigroup of transformations or partial permutations and \texttt{im(f)} denotes the image of \texttt{f}, then the \emph{generalized Schutzenberger group} of \texttt{im(f)} is the permutation group  
\[ \{\:g|_{\textrm{im}(f)}\::\:\textrm{im}(f*g)=\textrm{im}(f)\:\}. \]
  

 The generalized Schutzenberger group of the kernel \texttt{ker(f)} of a transformation \texttt{f} or the domain \texttt{dom(f)} of a partial permutation \texttt{f} is defined analogously. 

 The generalized Schutzenberger group of a Green's class is then defined as
follows. 
\begin{description}
\item[{$\mathcal{R}$-class}] The generalized Schutzenberger group of the image or range of the
representative of the $\mathcal{R}$-class. 
\item[{$\mathcal{L}$-class}] The generalized Schutzenberger group of the kernel or domain of the
representative of the $\mathcal{L}$-class. 
\item[{$\mathcal{H}$-class}] The intersection of the generalized Schutzenberger groups of the $\mathcal{R}$- and $\mathcal{L}$-class containing the $\mathcal{H}$-class. 
\item[{$\mathcal{D}$-class}] The intersection of the generalized Schutzenberger groups of the $\mathcal{R}$- and $\mathcal{L}$-class containing the representative of the $\mathcal{D}$-class. 
\end{description}
 
\begin{Verbatim}[commandchars=!@|,fontsize=\small,frame=single,label=Example]
  !gapprompt@gap>| !gapinput@S:=Semigroup( Transformation( [ 4, 4, 3, 5, 3 ] ), |
  !gapprompt@>| !gapinput@Transformation( [ 5, 1, 1, 4, 1 ] ), |
  !gapprompt@>| !gapinput@Transformation( [ 5, 5, 4, 4, 5 ] ) );;|
  !gapprompt@gap>| !gapinput@f:=Transformation( [ 5, 5, 4, 4, 5 ] );;|
  !gapprompt@gap>| !gapinput@SchutzenbergerGroup(RClass(S, f));|
  Group([ (4,5) ])
  !gapprompt@gap>| !gapinput@S:=InverseSemigroup(|
  !gapprompt@>| !gapinput@[ PartialPermNC([ 1, 2, 3, 7 ], [ 9, 2, 4, 8 ]),|
  !gapprompt@>| !gapinput@PartialPermNC([ 1, 2, 6, 7, 8, 9, 10 ], [ 6, 8, 4, 5, 9, 1, 3 ]),|
  !gapprompt@>| !gapinput@PartialPermNC([ 1, 2, 3, 5, 6, 7, 8, 9 ], [ 7, 4, 1, 6, 9, 5, 2, 3 ]) ] );;|
  !gapprompt@gap>| !gapinput@List(DClasses(S), SchutzenbergerGroup);|
  [ Group(()), Group(()), Group(()), Group(()), Group([ (1,9,8), (8,9) ]), 
    Group([ (4,9) ]), Group(()), Group(()), Group(()), Group(()), Group(()), 
    Group(()), Group(()), Group(()), Group(()), Group(()), 
    Group([ (2,5)(3,7) ]), Group([ (1,7,5,6,9,3) ]), Group(()), Group(()), 
    Group(()), Group(()), Group(()) ]
\end{Verbatim}
 }

 }

 
\section{\textcolor{Chapter }{Further attributes of semigroups}}\logpage{[ 4, 5, 0 ]}
\hyperdef{L}{X83802FC67CEB6C14}{}
{
 In this section we describe the attributes  (\textbf{Reference: Attributes}) of arbitrary semigroups of transformations or partial permutations that can be
found using \textsf{Citrus}. 

 In addition to those functions described in this section, the operation \texttt{Degree} (\ref{Degree:for a transformation coll}) is also relevant for transformation semigroups; and the following operations
are also relevant to semigroups of partial permutations: \texttt{MovedPoints} (\ref{MovedPoints:for a partial perm coll}), \texttt{NrMovedPoints} (\ref{NrMovedPoints:for a partial perm coll}), \texttt{LargestMovedPoint} (\ref{LargestMovedPoint:for a partial perm coll}), \texttt{SmallestMovedPoint} (\ref{SmallestMovedPoint:for a partial perm coll}), \texttt{Degree} (\ref{Degree:for a partial perm coll}).

 

\subsection{\textcolor{Chapter }{DomainOfPartialPermSemigroup}}
\logpage{[ 4, 5, 1 ]}\nobreak
\hyperdef{L}{X828A475A835EE0AE}{}
{\noindent\textcolor{FuncColor}{$\triangleright$\ \ \texttt{DomainOfPartialPermSemigroup({\mdseries\slshape S})\index{DomainOfPartialPermSemigroup@\texttt{DomainOfPartialPermSemigroup}}
\label{DomainOfPartialPermSemigroup}
}\hfill{\scriptsize (operation)}}\\
\noindent\textcolor{FuncColor}{$\triangleright$\ \ \texttt{Points({\mdseries\slshape S})\index{Points@\texttt{Points}!for a semigroup of partial perms}
\label{Points:for a semigroup of partial perms}
}\hfill{\scriptsize (operation)}}\\
\noindent\textcolor{FuncColor}{$\triangleright$\ \ \texttt{Points({\mdseries\slshape C})\index{Points@\texttt{Points}!for a partial perm coll}
\label{Points:for a partial perm coll}
}\hfill{\scriptsize (operation)}}\\
\textbf{\indent Returns:\ }
a set of positive integers.



 \texttt{Points} returns the points acted on by the semigroup of partial permutations \mbox{\texttt{\mdseries\slshape S}} or the partial permutation collection \mbox{\texttt{\mdseries\slshape C}}, respectively. See also \texttt{MovedPoints} (\ref{MovedPoints:for a partial perm}). 

 \texttt{DomainOfPartialPermSemigroup} is a synonym of \texttt{Points}. 
\begin{Verbatim}[commandchars=!@|,fontsize=\small,frame=single,label=Example]
  !gapprompt@gap>| !gapinput@S:=InverseSemigroup( PartialPermNC( [ 1, 2, 3 ], [ 1, 3, 2 ] ),|
  !gapprompt@>| !gapinput@PartialPermNC( [ 1, 2, 3 ], [ 1, 2, 20 ] ) );;|
  !gapprompt@gap>| !gapinput@MovedPoints(S);|
  [ 2, 3, 20 ]
  !gapprompt@gap>| !gapinput@Points(S);|
  [ 1, 2, 3, 20 ]
\end{Verbatim}
 }

 

\subsection{\textcolor{Chapter }{GeneratorsOfInverseSemigroup}}
\logpage{[ 4, 5, 2 ]}\nobreak
\hyperdef{L}{X87C373597F787250}{}
{\noindent\textcolor{FuncColor}{$\triangleright$\ \ \texttt{GeneratorsOfInverseSemigroup({\mdseries\slshape S})\index{GeneratorsOfInverseSemigroup@\texttt{GeneratorsOfInverseSemigroup}}
\label{GeneratorsOfInverseSemigroup}
}\hfill{\scriptsize (attribute)}}\\
\textbf{\indent Returns:\ }
a list of partial permutations.



 If \mbox{\texttt{\mdseries\slshape S}} is an inverse semigroup of partial permutations, then \texttt{GeneratorsOfInverseSemigroup} returns the list of partial permutations used to define \mbox{\texttt{\mdseries\slshape S}}, that is, an inverse semigroup generating set for \mbox{\texttt{\mdseries\slshape S}}. Note that \mbox{\texttt{\mdseries\slshape S}} is then the semigroup generated by the partial permutations in \texttt{GeneratorsOfInverseSemigroup(\mbox{\texttt{\mdseries\slshape S}})} and their inverses. 

 If \mbox{\texttt{\mdseries\slshape S}} is an inverse monoid, then \texttt{GeneratorsOfInverseSemigroup} returns the list of partial permutations used to define \mbox{\texttt{\mdseries\slshape S}} (as described above) and the identity of \mbox{\texttt{\mdseries\slshape S}}; see \texttt{One} (\ref{One:for a partial perm}). 

 See also \texttt{InverseSemigroup} (\ref{InverseSemigroup}) and \texttt{Generators} (\ref{Generators}). 
\begin{Verbatim}[commandchars=!@|,fontsize=\small,frame=single,label=Example]
  !gapprompt@gap>| !gapinput@S:=InverseMonoid(|
  !gapprompt@>| !gapinput@ PartialPermNC( [ 1, 2 ], [ 1, 4 ] ),|
  !gapprompt@>| !gapinput@ PartialPermNC( [ 1, 2, 4 ], [ 3, 4, 1 ] ) );;|
  !gapprompt@gap>| !gapinput@GeneratorsOfSemigroup(S);|
  [ <identity on [ 1 .. 4 ]>, [ 1, 2 ] -> [ 1, 4 ], [ 1, 2, 4 ] -> [ 3, 4, 1 ], 
    [ 1, 4 ] -> [ 1, 2 ], [ 1, 3, 4 ] -> [ 4, 1, 2 ] ]
  !gapprompt@gap>| !gapinput@GeneratorsOfInverseSemigroup(S);|
  [ <identity on [ 1 .. 4 ]>, [ 1, 2 ] -> [ 1, 4 ], [ 1, 2, 4 ] -> [ 3, 4, 1 ] ]
  !gapprompt@gap>| !gapinput@GeneratorsOfMonoid(S);|
  [ [ 1, 2 ] -> [ 1, 4 ], [ 1, 2, 4 ] -> [ 3, 4, 1 ], [ 1, 4 ] -> [ 1, 2 ], 
    [ 1, 3, 4 ] -> [ 4, 1, 2 ] ]
\end{Verbatim}
 }

 

\subsection{\textcolor{Chapter }{GeneratorsOfInverseMonoid}}
\logpage{[ 4, 5, 3 ]}\nobreak
\hyperdef{L}{X7A3B262C85B6D475}{}
{\noindent\textcolor{FuncColor}{$\triangleright$\ \ \texttt{GeneratorsOfInverseMonoid({\mdseries\slshape S})\index{GeneratorsOfInverseMonoid@\texttt{GeneratorsOfInverseMonoid}}
\label{GeneratorsOfInverseMonoid}
}\hfill{\scriptsize (attribute)}}\\
\textbf{\indent Returns:\ }
a list of partial permutations.



 If \mbox{\texttt{\mdseries\slshape S}} is an inverse monoid of partial permutations, then \texttt{GeneratorsOfInverseMonoid} returns the list of partial permutations used to define \mbox{\texttt{\mdseries\slshape S}}, that is, an inverse monoid generating set for \mbox{\texttt{\mdseries\slshape S}}. Note that \mbox{\texttt{\mdseries\slshape S}} is then the semigroup generated by the partial permutations in \texttt{GeneratorsOfInverseSemigroup(\mbox{\texttt{\mdseries\slshape S}})}, their inverses, and the identity of \mbox{\texttt{\mdseries\slshape S}}; see \texttt{One} (\ref{One:for a partial perm}). See also \texttt{InverseMonoid} (\ref{InverseMonoid}) and \texttt{Generators} (\ref{Generators}). 
\begin{Verbatim}[commandchars=!@|,fontsize=\small,frame=single,label=Example]
  !gapprompt@gap>| !gapinput@S:=InverseMonoid(|
  !gapprompt@>| !gapinput@ PartialPermNC( [ 1, 2 ], [ 1, 4 ] ),|
  !gapprompt@>| !gapinput@ PartialPermNC( [ 1, 2, 4 ], [ 3, 4, 1 ] ) );;|
  !gapprompt@gap>| !gapinput@GeneratorsOfInverseMonoid(S);|
  [ [ 1, 2 ] -> [ 1, 4 ], [ 1, 2, 4 ] -> [ 3, 4, 1 ] ]
\end{Verbatim}
 }

 

\subsection{\textcolor{Chapter }{Generators}}
\logpage{[ 4, 5, 4 ]}\nobreak
\hyperdef{L}{X7BD5B55C802805B4}{}
{\noindent\textcolor{FuncColor}{$\triangleright$\ \ \texttt{Generators({\mdseries\slshape S})\index{Generators@\texttt{Generators}}
\label{Generators}
}\hfill{\scriptsize (function)}}\\
\textbf{\indent Returns:\ }
A list of transformations or partial permutations.



 \texttt{Generators} returns the generating set used to define the semigroup \mbox{\texttt{\mdseries\slshape S}} of transformations or partial permutations. The generators of a monoid or
inverse semigroup \mbox{\texttt{\mdseries\slshape S}} can be defined in several ways, for example, including or excluding the
identity element, including or not the inverses of the generators. \texttt{Generators(\mbox{\texttt{\mdseries\slshape S}})} uses the definition that returns the least number of generators. Nothing new
is computed when \texttt{Generators} is called and this function should not be confused with \texttt{SmallGeneratingSet} (\ref{SmallGeneratingSet}). 
\begin{description}
\item[{Transformation semigroup}] \texttt{Generators(\mbox{\texttt{\mdseries\slshape S}})} is a synonym for \texttt{GeneratorsOfSemigroup} (\textbf{Reference: GeneratorsOfSemigroup}). 
\item[{Transformation monoid}] \texttt{Generators(\mbox{\texttt{\mdseries\slshape S}})} is a synonym for \texttt{GeneratorsOfMonoid} (\textbf{Reference: GeneratorsOfMonoid}). 
\item[{Inverse semigroup}] \texttt{Generators(\mbox{\texttt{\mdseries\slshape S}})} is a synonym for \texttt{GeneratorsOfInverseSemigroup} (\ref{GeneratorsOfInverseSemigroup}). 
\item[{Inverse monoid}] \texttt{Generators(\mbox{\texttt{\mdseries\slshape S}})} is a synonym for \texttt{GeneratorsOfInverseMonoid} (\ref{GeneratorsOfInverseMonoid}). 
\end{description}
 
\begin{Verbatim}[commandchars=!@|,fontsize=\small,frame=single,label=Example]
  !gapprompt@gap>| !gapinput@M:=Monoid(Transformation( [ 1, 4, 6, 2, 5, 3, 7, 8, 9, 9 ] ),|
  !gapprompt@>| !gapinput@Transformation( [ 6, 3, 2, 7, 5, 1, 8, 8, 9, 9 ] ) );;|
  !gapprompt@gap>| !gapinput@GeneratorsOfSemigroup(M);|
  [ Transformation( [ 1, 2, 3, 4, 5, 6, 7, 8, 9, 10 ] ), 
    Transformation( [ 1, 4, 6, 2, 5, 3, 7, 8, 9, 9 ] ), 
    Transformation( [ 6, 3, 2, 7, 5, 1, 8, 8, 9, 9 ] ) ]
  !gapprompt@gap>| !gapinput@GeneratorsOfMonoid(M);|
  [ Transformation( [ 1, 4, 6, 2, 5, 3, 7, 8, 9, 9 ] ), 
    Transformation( [ 6, 3, 2, 7, 5, 1, 8, 8, 9, 9 ] ) ]
  !gapprompt@gap>| !gapinput@Generators(M);|
  [ Transformation( [ 1, 4, 6, 2, 5, 3, 7, 8, 9, 9 ] ), 
    Transformation( [ 6, 3, 2, 7, 5, 1, 8, 8, 9, 9 ] ) ]
  !gapprompt@gap>| !gapinput@S:=Semigroup(Generators(M));;|
  !gapprompt@gap>| !gapinput@Generators(S);|
  [ Transformation( [ 1, 4, 6, 2, 5, 3, 7, 8, 9, 9 ] ), 
    Transformation( [ 6, 3, 2, 7, 5, 1, 8, 8, 9, 9 ] ) ]
  !gapprompt@gap>| !gapinput@GeneratorsOfSemigroup(S);|
  [ Transformation( [ 1, 4, 6, 2, 5, 3, 7, 8, 9, 9 ] ), 
    Transformation( [ 6, 3, 2, 7, 5, 1, 8, 8, 9, 9 ] ) ]
\end{Verbatim}
 }

 

\subsection{\textcolor{Chapter }{GroupOfUnits}}
\logpage{[ 4, 5, 5 ]}\nobreak
\hyperdef{L}{X811AEDD88280C277}{}
{\noindent\textcolor{FuncColor}{$\triangleright$\ \ \texttt{GroupOfUnits({\mdseries\slshape S})\index{GroupOfUnits@\texttt{GroupOfUnits}}
\label{GroupOfUnits}
}\hfill{\scriptsize (attribute)}}\\
\textbf{\indent Returns:\ }
The group of units of a semigroup.



 \texttt{GroupOfUnits} returns the group of units of the semigroup of transformations or partial
permutations \mbox{\texttt{\mdseries\slshape S}} as a subsemigroup of \mbox{\texttt{\mdseries\slshape S}} if it exists and returns \texttt{fail} if it does not. Use \texttt{IsomorphismPermGroup} (\ref{IsomorphismPermGroup}) if you require a permutation representation of the group of units.

 If a semigroup \mbox{\texttt{\mdseries\slshape S}} has an identity \texttt{e}, then the \emph{group of units} of \mbox{\texttt{\mdseries\slshape S}} is the set of those \texttt{s} in \mbox{\texttt{\mdseries\slshape S}} such that there exists \texttt{t} in \mbox{\texttt{\mdseries\slshape S}} where \texttt{s*t=t*s=e}. Equivalently, the group of units is the $\mathcal{H}$-class of the identity of \mbox{\texttt{\mdseries\slshape S}}.

 See also \texttt{GreensHClassOfElement} (\textbf{Reference: GreensHClassOfElement}), \texttt{IsMonoidAsSemigroup} (\ref{IsMonoidAsSemigroup}), and \texttt{MultiplicativeNeutralElement} (\textbf{Reference: MultiplicativeNeutralElement}). 
\begin{Verbatim}[commandchars=!@|,fontsize=\small,frame=single,label=Example]
  !gapprompt@gap>| !gapinput@S:=Semigroup(Transformation( [ 1, 2, 5, 4, 3, 8, 7, 6 ] ),|
  !gapprompt@>| !gapinput@  Transformation( [ 1, 6, 3, 4, 7, 2, 5, 8 ] ),|
  !gapprompt@>| !gapinput@  Transformation( [ 2, 1, 6, 7, 8, 3, 4, 5 ] ),|
  !gapprompt@>| !gapinput@  Transformation( [ 3, 2, 3, 6, 1, 6, 1, 2 ] ),|
  !gapprompt@>| !gapinput@  Transformation( [ 5, 2, 3, 6, 3, 4, 7, 4 ] ) );;|
  !gapprompt@gap>| !gapinput@Size(S);|
  5304
  !gapprompt@gap>| !gapinput@StructureDescription(GroupOfUnits(S));|
  "C2 x S4"
  !gapprompt@gap>| !gapinput@S:=InverseSemigroup( PartialPermNC( [ 1, 2, 3, 4, 5, 6, 7, 8, 9, 10 ], |
  !gapprompt@>| !gapinput@[ 2, 4, 5, 3, 6, 7, 10, 9, 8, 1 ] ),|
  !gapprompt@>| !gapinput@PartialPermNC( [ 1, 2, 3, 4, 5, 6, 7, 8, 10 ], |
  !gapprompt@>| !gapinput@[ 8, 2, 3, 1, 4, 5, 10, 6, 9 ] ) );;|
  !gapprompt@gap>| !gapinput@StructureDescription(GroupOfUnits(S));|
  "C8"
  !gapprompt@gap>| !gapinput@S:=InverseSemigroup( PartialPermNC( [ 1, 3, 4 ], [ 4, 3, 5 ] ),|
  !gapprompt@>| !gapinput@PartialPermNC( [ 1, 2, 3, 5 ], [ 3, 1, 5, 2 ] ) );;|
  !gapprompt@gap>| !gapinput@GroupOfUnits(S);|
  fail
\end{Verbatim}
 }

 

\subsection{\textcolor{Chapter }{Idempotents}}
\logpage{[ 4, 5, 6 ]}\nobreak
\hyperdef{L}{X7C651C9C78398FFF}{}
{\noindent\textcolor{FuncColor}{$\triangleright$\ \ \texttt{Idempotents({\mdseries\slshape obj[, n]})\index{Idempotents@\texttt{Idempotents}}
\label{Idempotents}
}\hfill{\scriptsize (attribute)}}\\
\textbf{\indent Returns:\ }
A list of idempotents.



 \mbox{\texttt{\mdseries\slshape obj}} should be a semigroup of transformations or partial permutations, $\mathcal{D}$-class, $\mathcal{H}$-class, $\mathcal{L}$-class, or $\mathcal{R}$-class.

 If the optional second argument \mbox{\texttt{\mdseries\slshape n}} is present and \mbox{\texttt{\mdseries\slshape obj}} is a semigroup of transformations or partial permutations, then a list of the
idempotents in \mbox{\texttt{\mdseries\slshape obj}} of rank \mbox{\texttt{\mdseries\slshape n}} is returned. If you are only interested in the idempotents of a given rank,
then the second version of the function will probably be faster. However, if
the optional second argument is present, then nothing is stored in \mbox{\texttt{\mdseries\slshape obj}} and so every time the function is called the computation must be repeated.

 This functions produce essentially the same output as the \textsf{GAP} library function with the same name; see \texttt{Idempotents} (\textbf{Reference: Idempotents}). The main difference is that this function can be applied to a wider class of
objects as described above.

 See also \texttt{IsRegularDClass} (\textbf{Reference: IsRegularDClass}), \texttt{IsRegularDClass} (\ref{IsRegularDClass}) \texttt{IsGroupHClass} (\textbf{Reference: IsGroupHClass}), \texttt{NrIdempotents} (\ref{NrIdempotents}), and \texttt{GroupHClass} (\ref{GroupHClass}). 
\begin{Verbatim}[commandchars=!@|,fontsize=\small,frame=single,label=Example]
  !gapprompt@gap>| !gapinput@S:=Semigroup([ Transformation( [ 2, 3, 4, 1 ] ), |
  !gapprompt@>| !gapinput@Transformation( [ 3, 3, 1, 1 ] ) ]);;|
  !gapprompt@gap>| !gapinput@Idempotents(S, 1);|
  [  ]
  !gapprompt@gap>| !gapinput@Idempotents(S, 2);|
  [ Transformation( [ 1, 1, 3, 3 ] ), Transformation( [ 2, 2, 4, 4 ] ), 
    Transformation( [ 1, 3, 3, 1 ] ), Transformation( [ 4, 2, 2, 4 ] ) ]
  !gapprompt@gap>| !gapinput@Idempotents(S);|
  [ Transformation( [ 1, 1, 3, 3 ] ), Transformation( [ 2, 2, 4, 4 ] ), 
    Transformation( [ 1, 3, 3, 1 ] ), Transformation( [ 4, 2, 2, 4 ] ), 
    Transformation( [ 1, 2, 3, 4 ] ) ]
  !gapprompt@gap>| !gapinput@f:=Transformation( [ 2, 2, 4, 4 ] );;|
  !gapprompt@gap>| !gapinput@R:=GreensRClassOfElement(S, f);|
  {Transformation( [ 3, 3, 1, 1 ] )}
  !gapprompt@gap>| !gapinput@Idempotents(R);|
  [ Transformation( [ 1, 1, 3, 3 ] ), Transformation( [ 2, 2, 4, 4 ] ) ]
  !gapprompt@gap>| !gapinput@f:=Transformation( [ 4, 2, 2, 4 ] );;|
  !gapprompt@gap>| !gapinput@L:=GreensLClassOfElement(S, f);;|
  !gapprompt@gap>| !gapinput@Idempotents(L);|
  [ Transformation( [ 4, 2, 2, 4 ] ), Transformation( [ 2, 2, 4, 4 ] ) ]
  !gapprompt@gap>| !gapinput@D:=DClassOfLClass(L);|
  {Transformation( [ 1, 3, 3, 1 ] )}
  !gapprompt@gap>| !gapinput@Idempotents(D);|
  [ Transformation( [ 1, 3, 3, 1 ] ), Transformation( [ 1, 1, 3, 3 ] ), 
    Transformation( [ 4, 2, 2, 4 ] ), Transformation( [ 2, 2, 4, 4 ] ) ]
  !gapprompt@gap>| !gapinput@L:=GreensLClassOfElement(S, Transformation( [ 3, 1, 1, 3 ] ));;|
  !gapprompt@gap>| !gapinput@Idempotents(L);|
  [ Transformation( [ 1, 3, 3, 1 ] ), Transformation( [ 1, 1, 3, 3 ] ) ]
  !gapprompt@gap>| !gapinput@H:=GroupHClass(D);|
  {Transformation( [ 3, 1, 1, 3 ] )}
  !gapprompt@gap>| !gapinput@Idempotents(H);|
  [ Transformation( [ 1, 3, 3, 1 ] ) ]
  !gapprompt@gap>| !gapinput@s:=InverseSemigroup(|
  !gapprompt@>| !gapinput@[ PartialPermNC( [ 1, 2, 3, 4, 5, 7 ], [ 10, 6, 3, 4, 9, 1 ] ),|
  !gapprompt@>| !gapinput@PartialPermNC( [ 1, 2, 3, 4, 5, 6, 7, 8 ], [ 6, 10, 7, 4, 8, 2, 9, 1 ] ) ]);;|
  !gapprompt@gap>| !gapinput@Idempotents(s, 1);|
  [ <identity on [ 4 ]> ]
  !gapprompt@gap>| !gapinput@Idempotents(s, 0);|
  [  ]
\end{Verbatim}
 }

 

\subsection{\textcolor{Chapter }{NrIdempotents}}
\logpage{[ 4, 5, 7 ]}\nobreak
\hyperdef{L}{X7CFC4DB387452320}{}
{\noindent\textcolor{FuncColor}{$\triangleright$\ \ \texttt{NrIdempotents({\mdseries\slshape obj})\index{NrIdempotents@\texttt{NrIdempotents}}
\label{NrIdempotents}
}\hfill{\scriptsize (attribute)}}\\
\textbf{\indent Returns:\ }
 A positive integer. 



 This function returns the number of idempotents in \mbox{\texttt{\mdseries\slshape obj}} where \mbox{\texttt{\mdseries\slshape obj}} can be a semigroup of transformations or partial permutations, $\mathcal{D}$-, $\mathcal{L}$-, $\mathcal{H}$-, or $\mathcal{R}$-class of a semigroup of transformations or partial permutations. If the
actual idempotents are not required, then it is more efficient to use \texttt{NrIdempotents(obj)} than \texttt{Length(Idempotents(obj))} since the idempotents themselves are not created when \texttt{NrIdempotents} is called.

 See also \texttt{Idempotents} (\textbf{Reference: Idempotents}) and \texttt{Idempotents} (\ref{Idempotents}), \texttt{IsRegularDClass} (\textbf{Reference: IsRegularDClass}), \texttt{IsRegularDClass} (\ref{IsRegularDClass}) \texttt{IsGroupHClass} (\textbf{Reference: IsGroupHClass}), and \texttt{GroupHClass} (\ref{GroupHClass}). 
\begin{Verbatim}[commandchars=!@|,fontsize=\small,frame=single,label=Example]
  !gapprompt@gap>| !gapinput@S:=Semigroup([ Transformation( [ 2, 3, 4, 1 ] ), |
  !gapprompt@>| !gapinput@Transformation( [ 3, 3, 1, 1 ] ) ]);;|
  !gapprompt@gap>| !gapinput@NrIdempotents(S);   |
  5
  !gapprompt@gap>| !gapinput@f:=Transformation( [ 2, 2, 4, 4 ] );;|
  !gapprompt@gap>| !gapinput@R:=GreensRClassOfElement(S, f);;|
  !gapprompt@gap>| !gapinput@NrIdempotents(R);|
  2
  !gapprompt@gap>| !gapinput@f:=Transformation( [ 4, 2, 2, 4 ] );;|
  !gapprompt@gap>| !gapinput@L:=GreensLClassOfElement(S, f);;|
  !gapprompt@gap>| !gapinput@NrIdempotents(L);|
  2
  !gapprompt@gap>| !gapinput@D:=DClassOfLClass(L);;|
  !gapprompt@gap>| !gapinput@NrIdempotents(D);|
  4
  !gapprompt@gap>| !gapinput@L:=GreensLClassOfElement(S, Transformation( [ 3, 1, 1, 3 ] ));;|
  !gapprompt@gap>| !gapinput@NrIdempotents(L);|
  2
  !gapprompt@gap>| !gapinput@H:=GroupHClass(D);;|
  !gapprompt@gap>| !gapinput@NrIdempotents(H);|
  1
  !gapprompt@gap>| !gapinput@s:=InverseSemigroup(|
  !gapprompt@>| !gapinput@[ PartialPermNC( [ 1, 2, 3, 5, 7, 9, 10 ], [ 6, 7, 2, 9, 1, 5, 3 ] ),|
  !gapprompt@>| !gapinput@PartialPermNC( [ 1, 2, 3, 5, 6, 7, 9, 10 ], [ 8, 1, 9, 4, 10, 5, 6, 7 ] ) ]);;|
  !gapprompt@gap>| !gapinput@NrIdempotents(s);|
  236
  !gapprompt@gap>| !gapinput@f:=PartialPermNC([ 2, 3, 7, 9, 10 ], [ 7, 2, 1, 5, 3 ]);;|
  !gapprompt@gap>| !gapinput@d:=DClassNC(s, f);;|
  !gapprompt@gap>| !gapinput@NrIdempotents(d);|
  13
\end{Verbatim}
 }

 

\subsection{\textcolor{Chapter }{IdempotentGeneratedSubsemigp}}
\logpage{[ 4, 5, 8 ]}\nobreak
\hyperdef{L}{X7C37F8577EC53529}{}
{\noindent\textcolor{FuncColor}{$\triangleright$\ \ \texttt{IdempotentGeneratedSubsemigp({\mdseries\slshape S})\index{IdempotentGeneratedSubsemigp@\texttt{IdempotentGeneratedSubsemigp}}
\label{IdempotentGeneratedSubsemigp}
}\hfill{\scriptsize (operation)}}\\
\textbf{\indent Returns:\ }
A semigroup. 



 \texttt{IdempotentGeneratedSubsemigp} returns the subsemigroup of the semigroup of transformations or partial
permutations \mbox{\texttt{\mdseries\slshape S}} generated by the idempotents of \mbox{\texttt{\mdseries\slshape S}}.

 See also \texttt{Idempotents} (\ref{Idempotents}) and \texttt{SmallGeneratingSet} (\ref{SmallGeneratingSet}). 
\begin{Verbatim}[commandchars=!@|,fontsize=\small,frame=single,label=Example]
  !gapprompt@gap>| !gapinput@file:=Concatenation(CitrusDir(), "/examples/graph8c.citrus.gz");;|
  !gapprompt@gap>| !gapinput@S:=Semigroup(ReadCitrus(file, 13));;|
  !gapprompt@gap>| !gapinput@IdempotentGeneratedSubsemigp(S);|
  <monoid with 20 generators>
  !gapprompt@gap>| !gapinput@S:=SymmetricInverseSemigp(5);|
  <inverse semigroup with 3 generators>
  !gapprompt@gap>| !gapinput@IdempotentGeneratedSubsemigp(S);|
  <inverse semigroup with 6 generators>
\end{Verbatim}
 }

 

\subsection{\textcolor{Chapter }{IrredundantGeneratingSubset}}
\logpage{[ 4, 5, 9 ]}\nobreak
\hyperdef{L}{X7F88DA9487720D2B}{}
{\noindent\textcolor{FuncColor}{$\triangleright$\ \ \texttt{IrredundantGeneratingSubset({\mdseries\slshape coll})\index{IrredundantGeneratingSubset@\texttt{IrredundantGeneratingSubset}}
\label{IrredundantGeneratingSubset}
}\hfill{\scriptsize (operation)}}\\
\textbf{\indent Returns:\ }
 A list of transformations or partial permutations. 



 If \mbox{\texttt{\mdseries\slshape coll}} is a transformation or partial permutation collection, then this function
returns a subset \texttt{U} of \mbox{\texttt{\mdseries\slshape coll}} or \texttt{Generators(\mbox{\texttt{\mdseries\slshape coll}})} such that no element of \texttt{U} is generated by the other elements of \texttt{U}. 

 See also \texttt{Generators} (\ref{Generators}), \texttt{IsTransformationCollection} (\textbf{Reference: IsTransformationCollection}), and \texttt{SmallGeneratingSet} (\ref{SmallGeneratingSet}). 
\begin{Verbatim}[commandchars=!@|,fontsize=\small,frame=single,label=Example]
  !gapprompt@gap>| !gapinput@S:=Semigroup( Transformation( [ 5, 1, 4, 6, 2, 3 ] ),|
  !gapprompt@>| !gapinput@Transformation( [ 1, 2, 3, 4, 5, 6 ] ),|
  !gapprompt@>| !gapinput@Transformation( [ 4, 6, 3, 4, 2, 5 ] ),|
  !gapprompt@>| !gapinput@Transformation( [ 5, 4, 6, 3, 1, 3 ] ),|
  !gapprompt@>| !gapinput@Transformation( [ 2, 2, 6, 5, 4, 3 ] ),|
  !gapprompt@>| !gapinput@Transformation( [ 3, 5, 5, 1, 2, 4 ] ),|
  !gapprompt@>| !gapinput@Transformation( [ 6, 5, 1, 3, 3, 4 ] ),|
  !gapprompt@>| !gapinput@Transformation( [ 1, 3, 4, 3, 2, 1 ] ) );;|
  !gapprompt@gap>| !gapinput@IrredundantGeneratingSubset(S);|
  [ Transformation( [ 1, 3, 4, 3, 2, 1 ] ), 
    Transformation( [ 2, 2, 6, 5, 4, 3 ] ), 
    Transformation( [ 3, 5, 5, 1, 2, 4 ] ), 
    Transformation( [ 5, 1, 4, 6, 2, 3 ] ), 
    Transformation( [ 5, 4, 6, 3, 1, 3 ] ), 
    Transformation( [ 6, 5, 1, 3, 3, 4 ] ) ]
  !gapprompt@gap>| !gapinput@S:=RandomInverseMonoid(1000,10);|
  <inverse monoid with 1000 generators>
  !gapprompt@gap>| !gapinput@SmallGeneratingSet(S);|
  [ [ 1 .. 10 ] -> [ 6, 5, 1, 9, 8, 3, 10, 4, 7, 2 ], 
    [ 1 .. 10 ] -> [ 1, 4, 6, 2, 8, 5, 7, 10, 3, 9 ], 
    [ 1, 2, 3, 4, 6, 7, 8, 9 ] -> [ 7, 5, 10, 1, 8, 4, 9, 6 ]
    [ 1 .. 9 ] -> [ 4, 3, 5, 7, 10, 9, 1, 6, 8 ] ]
  !gapprompt@gap>| !gapinput@IrredundantGeneratingSubset(last);|
  [ [ 1 .. 9 ] -> [ 4, 3, 5, 7, 10, 9, 1, 6, 8 ], 
    [ 1 .. 10 ] -> [ 1, 4, 6, 2, 8, 5, 7, 10, 3, 9 ], 
    [ 1 .. 10 ] -> [ 6, 5, 1, 9, 8, 3, 10, 4, 7, 2 ] ]
\end{Verbatim}
 }

 

\subsection{\textcolor{Chapter }{MinimalIdeal}}
\logpage{[ 4, 5, 10 ]}\nobreak
\hyperdef{L}{X7BC68589879C3BE9}{}
{\noindent\textcolor{FuncColor}{$\triangleright$\ \ \texttt{MinimalIdeal({\mdseries\slshape S})\index{MinimalIdeal@\texttt{MinimalIdeal}}
\label{MinimalIdeal}
}\hfill{\scriptsize (attribute)}}\\
\textbf{\indent Returns:\ }
 The minimal ideal of a semigroup. 



 The minimal ideal of a semigroup is the least ideal with respect to
containment. 

 Currently, \texttt{MinimalIdeal} returns a semigroup with as many generators as elements. There are plans to
improve this in future versions of \textsf{Citrus}. 

 Note that \texttt{MinimalIdeal} is significantly faster than finding the $\mathcal{D}$-class with minimum rank representative (which is also the minimal ideal). See
also \texttt{PartialOrderOfDClasses} (\ref{PartialOrderOfDClasses}) and \texttt{IsGreensLessThanOrEqual} (\textbf{Reference: IsGreensLessThanOrEqual}). 
\begin{Verbatim}[commandchars=!@|,fontsize=\small,frame=single,label=Example]
  !gapprompt@gap>| !gapinput@S:=Semigroup( Transformation( [ 3, 4, 1, 3, 6, 3, 4, 6, 10, 1 ] ), |
  !gapprompt@>| !gapinput@Transformation( [ 8, 2, 3, 8, 4, 1, 3, 4, 9, 7 ] ));;|
  !gapprompt@gap>| !gapinput@MinimalIdeal(S);|
  <semigroup with 5 generators>
  !gapprompt@gap>| !gapinput@Elements(MinimalIdeal(S));|
  [ Transformation( [ 1, 1, 1, 1, 1, 1, 1, 1, 1, 1 ] ), 
    Transformation( [ 3, 3, 3, 3, 3, 3, 3, 3, 3, 3 ] ), 
    Transformation( [ 4, 4, 4, 4, 4, 4, 4, 4, 4, 4 ] ), 
    Transformation( [ 6, 6, 6, 6, 6, 6, 6, 6, 6, 6 ] ), 
    Transformation( [ 8, 8, 8, 8, 8, 8, 8, 8, 8, 8 ] ) ]
  !gapprompt@gap>| !gapinput@f:=Transformation( [ 8, 8, 8, 8, 8, 8, 8, 8, 8, 8 ] );;|
  !gapprompt@gap>| !gapinput@D:=DClass(S, f);|
  {Transformation( [ 3, 3, 3, 3, 3, 3, 3, 3, 3, 3 ] )}
  !gapprompt@gap>| !gapinput@ForAll(GreensDClasses(S), x-> IsGreensLessThanOrEqual(D, x));|
  true
  !gapprompt@gap>| !gapinput@S:=POI(10);                  |
  <inverse monoid with 10 generators>
  !gapprompt@gap>| !gapinput@MinimalIdeal(S);|
  <inverse semigroup with 1 generator>
\end{Verbatim}
 }

 

\subsection{\textcolor{Chapter }{MultiplicativeZero}}
\logpage{[ 4, 5, 11 ]}\nobreak
\hyperdef{L}{X7B39F93C8136D642}{}
{\noindent\textcolor{FuncColor}{$\triangleright$\ \ \texttt{MultiplicativeZero({\mdseries\slshape S})\index{MultiplicativeZero@\texttt{MultiplicativeZero}}
\label{MultiplicativeZero}
}\hfill{\scriptsize (attribute)}}\\
\textbf{\indent Returns:\ }
 The zero element of a semigroup. 



 \texttt{MultiplicativeZero} returns the zero element of the semigroup \mbox{\texttt{\mdseries\slshape S}} of transformations or partial permutations if it has one and \texttt{fail} if it does not. See also \texttt{MultiplicativeZero} (\textbf{Reference: MultiplicativeZero}). 
\begin{Verbatim}[commandchars=!@|,fontsize=\small,frame=single,label=Example]
  !gapprompt@gap>| !gapinput@S:=Semigroup( Transformation( [ 1, 4, 2, 6, 6, 5, 2 ] ), |
  !gapprompt@>| !gapinput@Transformation( [ 1, 6, 3, 6, 2, 1, 6 ] ));;|
  !gapprompt@gap>| !gapinput@MultiplicativeZero(S);|
  Transformation( [ 1, 1, 1, 1, 1, 1, 1 ] )
  !gapprompt@gap>| !gapinput@S:=Semigroup(Transformation( [ 2, 8, 3, 7, 1, 5, 2, 6 ] ), |
  !gapprompt@>| !gapinput@Transformation( [ 3, 5, 7, 2, 5, 6, 3, 8 ] ), |
  !gapprompt@>| !gapinput@Transformation( [ 6, 7, 4, 1, 4, 1, 6, 2 ] ), |
  !gapprompt@>| !gapinput@Transformation( [ 8, 8, 5, 1, 7, 5, 2, 8 ] ));;|
  !gapprompt@gap>| !gapinput@MultiplicativeZero(S);|
  fail
  !gapprompt@gap>| !gapinput@S:=InverseSemigroup( PartialPermNC( [ 1, 3, 4 ], [ 5, 3, 1 ] ),|
  !gapprompt@>| !gapinput@PartialPermNC( [ 1, 2, 3, 4 ], [ 4, 3, 1, 2 ] ),|
  !gapprompt@>| !gapinput@PartialPermNC( [ 1, 3, 4, 5 ], [ 2, 4, 5, 3 ] ) );;|
  !gapprompt@gap>| !gapinput@MultiplicativeZero(S);|
  <empty mapping>
\end{Verbatim}
 }

 

\subsection{\textcolor{Chapter }{NaturalPartialOrder}}
\logpage{[ 4, 5, 12 ]}\nobreak
\hyperdef{L}{X7EA51F087CF7621F}{}
{\noindent\textcolor{FuncColor}{$\triangleright$\ \ \texttt{NaturalPartialOrder({\mdseries\slshape S})\index{NaturalPartialOrder@\texttt{NaturalPartialOrder}}
\label{NaturalPartialOrder}
}\hfill{\scriptsize (attribute)}}\\
\textbf{\indent Returns:\ }
The natural partial order on an inverse semigroup.



 The \emph{natural partial order} $\leq$ on an inverse semigroup $S$ is defined by $s\leq$$t$ if there exists an idempotent $e$ in $S$ such that $s=et$. Hence if \mbox{\texttt{\mdseries\slshape f}} and \mbox{\texttt{\mdseries\slshape g}} are partial permutations, then \mbox{\texttt{\mdseries\slshape f}}$\leq$\mbox{\texttt{\mdseries\slshape g}} if and only if \mbox{\texttt{\mdseries\slshape f}} is a restriction of \mbox{\texttt{\mdseries\slshape g}}; see \texttt{RestrictedPartialPerm} (\ref{RestrictedPartialPerm}).

 \texttt{NaturalPartialOrder} returns the natural partial order on the inverse semigroup of partial
permutations \mbox{\texttt{\mdseries\slshape S}} as a list of sets of positive integers where entry \texttt{i} in \texttt{NaturalPartialOrder(\mbox{\texttt{\mdseries\slshape S}})} is the set of positions in \texttt{Elements(\mbox{\texttt{\mdseries\slshape S}})} of elements less than \texttt{Elements(\mbox{\texttt{\mdseries\slshape S}})[i]}. See also \texttt{NaturalLeqPartialPerm} (\ref{NaturalLeqPartialPerm}).

 
\begin{Verbatim}[commandchars=!@|,fontsize=\small,frame=single,label=Example]
  !gapprompt@gap>| !gapinput@S:=InverseSemigroup([ PartialPermNC( [ 1, 3 ], [ 1, 3 ] ),|
  !gapprompt@>| !gapinput@PartialPermNC( [ 1, 2 ], [ 3, 2 ] ) ] );|
  <inverse semigroup with 2 generators>
  !gapprompt@gap>| !gapinput@Size(S);|
  11
  !gapprompt@gap>| !gapinput@NaturalPartialOrder(S);|
  [ [  ], [ 1 ], [ 1 ], [ 1 ], [ 1 ], [ 1 ], [ 1, 2, 4, 5 ], [ 1, 3, 4, 5 ], 
    [ 1, 2, 6 ], [ 1, 4, 5 ], [ 1, 4, 6 ] ]
  !gapprompt@gap>| !gapinput@NaturalLeqPartialPerm(Elements(S)[4], Elements(S)[10]);|
  true
  !gapprompt@gap>| !gapinput@NaturalLeqPartialPerm(Elements(S)[4], Elements(S)[9]); |
  false
\end{Verbatim}
 }

 

\subsection{\textcolor{Chapter }{NrElementsOfRank}}
\logpage{[ 4, 5, 13 ]}\nobreak
\hyperdef{L}{X876B05E37A450BCE}{}
{\noindent\textcolor{FuncColor}{$\triangleright$\ \ \texttt{NrElementsOfRank({\mdseries\slshape S, n})\index{NrElementsOfRank@\texttt{NrElementsOfRank}}
\label{NrElementsOfRank}
}\hfill{\scriptsize (attribute)}}\\
\textbf{\indent Returns:\ }
the number of elements of a given rank. 



\texttt{NrElementsOfRank} returns the number of elements of the semigroup \mbox{\texttt{\mdseries\slshape S}} of partial permutations or transformations with rank \mbox{\texttt{\mdseries\slshape n}}; see \texttt{Rank} (\ref{Rank:for a transformation}), \texttt{RankOfPartialPerm} (\ref{RankOfPartialPerm}) and \texttt{RankOfTransformation} (\textbf{Reference: RankOfTransformation}). 
\begin{Verbatim}[commandchars=!@|,fontsize=\small,frame=single,label=Example]
  !gapprompt@gap>| !gapinput@S:=Semigroup( Transformation( [ 1, 3, 4, 1, 3 ] ), |
  !gapprompt@>| !gapinput@Transformation( [ 2, 4, 1, 5, 5 ] ), |
  !gapprompt@>| !gapinput@Transformation( [ 2, 5, 3, 5, 3 ] ), |
  !gapprompt@>| !gapinput@Transformation( [ 4, 1, 2, 2, 1 ] ), |
  !gapprompt@>| !gapinput@ Transformation( [ 5, 5, 1, 1, 3 ] ) );;|
  !gapprompt@gap>| !gapinput@NrElementsOfRank(S, 10);|
  0
  !gapprompt@gap>| !gapinput@Size(S);|
  602
  !gapprompt@gap>| !gapinput@List([1..5], x-> NrElementsOfRank(S, x));|
  [ 5, 260, 336, 1, 0 ]
  !gapprompt@gap>| !gapinput@Sum(last);|
  602
  !gapprompt@gap>| !gapinput@T:=FullTransformationSemigroup(5);;|
  !gapprompt@gap>| !gapinput@List([1..5], x-> NrElementsOfRank(T, x));|
  [ 5, 300, 1500, 1200, 120 ]
  !gapprompt@gap>| !gapinput@Sum(last);|
  3125
  !gapprompt@gap>| !gapinput@S:=SymmetricInverseSemigp(5);;|
  !gapprompt@gap>| !gapinput@NrElementsOfRank(S, 4);|
  600
  !gapprompt@gap>| !gapinput@Binomial(5,4)^2*Factorial(4);|
  600
\end{Verbatim}
 }

 

\subsection{\textcolor{Chapter }{PrimitiveIdempotents}}
\logpage{[ 4, 5, 14 ]}\nobreak
\hyperdef{L}{X80C0C6C37C4A2ABD}{}
{\noindent\textcolor{FuncColor}{$\triangleright$\ \ \texttt{PrimitiveIdempotents({\mdseries\slshape S})\index{PrimitiveIdempotents@\texttt{PrimitiveIdempotents}}
\label{PrimitiveIdempotents}
}\hfill{\scriptsize (attribute)}}\\
\textbf{\indent Returns:\ }
A list of idempotent partial permutations.



 An idempotent in an inverse semigroup \mbox{\texttt{\mdseries\slshape S}} is \emph{primitive} if it is non-zero and minimal with respect to the \texttt{NaturalPartialOrder} (\ref{NaturalPartialOrder}) on \mbox{\texttt{\mdseries\slshape S}}. \texttt{PrimitiveIdempotents} returns the list of primitive idempotents in the inverse semigroup of partial
permutations \mbox{\texttt{\mdseries\slshape S}}. 
\begin{Verbatim}[commandchars=!@|,fontsize=\small,frame=single,label=Example]
  !gapprompt@gap>| !gapinput@S:= InverseMonoid(|
  !gapprompt@>| !gapinput@PartialPermNC( [ 1 ], [ 4 ] ),|
  !gapprompt@>| !gapinput@PartialPermNC( [ 1, 2, 3 ], [ 2, 1, 3 ] ),|
  !gapprompt@>| !gapinput@PartialPermNC( [ 1, 2, 3 ], [ 3, 1, 2 ] ) );;|
  !gapprompt@gap>| !gapinput@MultiplicativeZero(S);|
  <empty mapping>
  !gapprompt@gap>| !gapinput@PrimitiveIdempotents(S);|
  [ <identity on [ 4 ]>, <identity on [ 1 ]>, <identity on [ 2 ]>, 
    <identity on [ 3 ]> ]
\end{Verbatim}
 }

 

\subsection{\textcolor{Chapter }{Random (for a semigroup)}}
\logpage{[ 4, 5, 15 ]}\nobreak
\hyperdef{L}{X7BB7FDFE7AFFD672}{}
{\noindent\textcolor{FuncColor}{$\triangleright$\ \ \texttt{Random({\mdseries\slshape S})\index{Random@\texttt{Random}!for a semigroup}
\label{Random:for a semigroup}
}\hfill{\scriptsize (method)}}\\
\textbf{\indent Returns:\ }
A transformation or a partial permutation.



 This function returns a random element of the semigroup of transformations or
partial permutations \mbox{\texttt{\mdseries\slshape S}}. If the $\mathcal{R}$-class structure of \mbox{\texttt{\mdseries\slshape S}} has not been calculated, then a short product (at most \texttt{2*Length(GeneratorsOfSemigroup(\mbox{\texttt{\mdseries\slshape S}}))}) of generators is returned. If the $\mathcal{R}$-class structure of \mbox{\texttt{\mdseries\slshape S}} is known, then a random element of a randomly chosen $\mathcal{R}$-class is returned. }

 

\subsection{\textcolor{Chapter }{SmallGeneratingSet}}
\logpage{[ 4, 5, 16 ]}\nobreak
\hyperdef{L}{X814DBABC878D5232}{}
{\noindent\textcolor{FuncColor}{$\triangleright$\ \ \texttt{SmallGeneratingSet({\mdseries\slshape S})\index{SmallGeneratingSet@\texttt{SmallGeneratingSet}}
\label{SmallGeneratingSet}
}\hfill{\scriptsize (attribute)}}\\
\textbf{\indent Returns:\ }
A small generating set for a semigroup.



 \texttt{SmallGeneratingSet} returns a relatively small generating subset of the set of generators of the
semigroup \mbox{\texttt{\mdseries\slshape S}} of transformations or partial permutations; see \texttt{Generators} (\ref{Generators}). If the number of generators for \mbox{\texttt{\mdseries\slshape S}} is already relatively small, then this function will often return the original
generating set.

 As neither irredundancy, nor minimal length are proven, \texttt{SmallGeneratingSet} usually returns an answer much faster than \texttt{IrredundantGeneratingSubset} (\ref{IrredundantGeneratingSubset}). It can be used whenever a small generating set is desired which does not
necessarily needs to be optimal. \texttt{SmallGeneratingSet} works particularly well for inverse semigroups of partial permutations.

 Note that \texttt{SmallGeneratingSet} may return different results in different \textsf{GAP} sessions. 
\begin{Verbatim}[commandchars=!@|,fontsize=\small,frame=single,label=Example]
  !gapprompt@gap>| !gapinput@S:=Semigroup( Transformation( [ 1, 2, 3, 2, 4 ] ), |
  !gapprompt@>| !gapinput@Transformation( [ 1, 5, 4, 3, 2 ] ),|
  !gapprompt@>| !gapinput@Transformation( [ 2, 1, 4, 2, 2 ] ), Transformation( [ 2, 4, 4, 2, 1 ] ),|
  !gapprompt@>| !gapinput@Transformation( [ 3, 1, 4, 3, 2 ] ), Transformation( [ 3, 2, 3, 4, 1 ] ),|
  !gapprompt@>| !gapinput@Transformation( [ 4, 4, 3, 3, 5 ] ), Transformation( [ 5, 1, 5, 5, 3 ] ),|
  !gapprompt@>| !gapinput@Transformation( [ 5, 4, 3, 5, 2 ] ), Transformation( [ 5, 5, 4, 5, 5 ] ) );;|
  !gapprompt@gap>| !gapinput@SmallGeneratingSet(S);                  |
  [ Transformation( [ 1, 5, 4, 3, 2 ] ), Transformation( [ 3, 2, 3, 4, 1 ] ), 
    Transformation( [ 5, 4, 3, 5, 2 ] ), Transformation( [ 1, 2, 3, 2, 4 ] ), 
    Transformation( [ 4, 4, 3, 3, 5 ] ) ]
  !gapprompt@gap>| !gapinput@S:=RandomInverseMonoid(10000,10);;|
  !gapprompt@gap>| !gapinput@SmallGeneratingSet(S);|
  [ [ 1 .. 10 ] -> [ 3, 2, 4, 5, 6, 1, 7, 10, 9, 8 ], 
    [ 1 .. 10 ] -> [ 5, 10, 8, 9, 3, 2, 4, 7, 6, 1 ], 
    [ 1, 3, 4, 5, 6, 7, 8, 9, 10 ] -> [ 1, 6, 4, 8, 2, 10, 7, 3, 9 ] ]
\end{Verbatim}
 }

 
\subsection{\textcolor{Chapter }{Graded images and kernels}}\logpage{[ 4, 5, 17 ]}
\hyperdef{L}{X7C4F9F007D55EF8B}{}
{
\noindent\textcolor{FuncColor}{$\triangleright$\ \ \texttt{GradedImagesOfTransSemigroup({\mdseries\slshape S})\index{GradedImagesOfTransSemigroup@\texttt{GradedImagesOfTransSemigroup}}
\label{GradedImagesOfTransSemigroup}
}\hfill{\scriptsize (attribute)}}\\
\noindent\textcolor{FuncColor}{$\triangleright$\ \ \texttt{GradedKernelsOfTransSemigroup({\mdseries\slshape S})\index{GradedKernelsOfTransSemigroup@\texttt{GradedKernelsOfTransSemigroup}}
\label{GradedKernelsOfTransSemigroup}
}\hfill{\scriptsize (attribute)}}\\
\textbf{\indent Returns:\ }
A list of images or kernels.



 \texttt{GradedImagesOfTransSemigroup} returns a list where the \texttt{i}th entry is a list of all the images of transformations in the transformation
semigroup \mbox{\texttt{\mdseries\slshape S}} with size \texttt{i}. 

 \texttt{GradedKernelsOfTransSemigroup} returns a list where the \texttt{i}th entry is a list of the values of \texttt{CanonicalTransSameKernel} (\ref{CanonicalTransSameKernel}) for all transformations in \mbox{\texttt{\mdseries\slshape S}} with rank \texttt{i}. 

 See also \texttt{ImagesOfTransSemigroup} (\ref{ImagesOfTransSemigroup}) and \texttt{KernelsOfTransSemigroup} (\ref{KernelsOfTransSemigroup}). 
\begin{Verbatim}[commandchars=!@|,fontsize=\small,frame=single,label=Example]
  !gapprompt@gap>| !gapinput@S:=Semigroup(Transformation( [ 1, 5, 1, 1, 1 ] ), |
  !gapprompt@>| !gapinput@Transformation( [ 4, 4, 5, 2, 2 ] ));;|
  !gapprompt@gap>| !gapinput@GradedImagesOfTransSemigroup(S);|
  [ [ [ 1 ], [ 4 ], [ 2 ], [ 5 ] ], [ [ 1, 5 ], [ 2, 4 ] ], [ [ 2, 4, 5 ] ], 
    [  ], [  ] ]
  !gapprompt@gap>| !gapinput@GradedKernelsOfTransSemigroup(S);|
  [ [ [ 1, 1, 1, 1, 1 ] ], [ [ 1, 2, 1, 1, 1 ], [ 1, 1, 1, 2, 2 ] ], 
    [ [ 1, 1, 2, 3, 3 ] ], [  ], [  ] ]
\end{Verbatim}
 }

 
\subsection{\textcolor{Chapter }{Images and kernels}}\logpage{[ 4, 5, 18 ]}
\hyperdef{L}{X7BC9579885CFC361}{}
{
\noindent\textcolor{FuncColor}{$\triangleright$\ \ \texttt{ImagesOfTransSemigroup({\mdseries\slshape S[, n]})\index{ImagesOfTransSemigroup@\texttt{ImagesOfTransSemigroup}}
\label{ImagesOfTransSemigroup}
}\hfill{\scriptsize (attribute)}}\\
\noindent\textcolor{FuncColor}{$\triangleright$\ \ \texttt{KernelsOfTransSemigroup({\mdseries\slshape S[, n]})\index{KernelsOfTransSemigroup@\texttt{KernelsOfTransSemigroup}}
\label{KernelsOfTransSemigroup}
}\hfill{\scriptsize (attribute)}}\\
\textbf{\indent Returns:\ }
An orbit.



 The argument \mbox{\texttt{\mdseries\slshape S}} should be a transformation semigroup and the optional second argument \mbox{\texttt{\mdseries\slshape n}} should be a positive integer not greater than the degree of \mbox{\texttt{\mdseries\slshape S}}. \texttt{ImagesOfTransSemigroup} returns the \texttt{Orb} (\textbf{orb: orb}) object: 
\begin{Verbatim}[commandchars=!@|,fontsize=\small,frame=single,label=Example]
  Orb(S, [1..Degree(S)], OnSets);
\end{Verbatim}
 which contains the image sets of all the transformations belonging to \mbox{\texttt{\mdseries\slshape S}}. 

 \texttt{KernelsOfTransSemigroup} returns the \texttt{Orb} (\textbf{orb: orb}) object: 
\begin{Verbatim}[commandchars=!@|,fontsize=\small,frame=single,label=Example]
  Orb(S, [1,..Degree(S)], OnKernelsAntiAction);
\end{Verbatim}
 

 If the optional second argument \mbox{\texttt{\mdseries\slshape n}} (a positive integer) is present, then only the images or kernels of size at
least \mbox{\texttt{\mdseries\slshape n}} are found.

 Note that the image/kernel \texttt{[1..Degree(S)]} always occurs in both orbits even if there is no element of \mbox{\texttt{\mdseries\slshape S}} with image/kernel equal to \texttt{[1..Degree(S)]}.

 See also \texttt{GradedImagesOfTransSemigroup} (\ref{GradedImagesOfTransSemigroup}), \texttt{GradedKernelsOfTransSemigroup} (\ref{GradedKernelsOfTransSemigroup}), \texttt{OnKernelsAntiAction} (\ref{OnKernelsAntiAction}), and \texttt{CanonicalTransSameKernel} (\ref{CanonicalTransSameKernel}) 
\begin{Verbatim}[commandchars=!@|,fontsize=\small,frame=single,label=Example]
  !gapprompt@gap>| !gapinput@ S:=Semigroup( Transformation( [ 6, 4, 4, 4, 6, 1 ] ), |
  !gapprompt@>| !gapinput@Transformation( [ 6, 5, 1, 6, 2, 2 ] ) );;|
  !gapprompt@gap>| !gapinput@o:=ImagesOfTransSemigroup(S, 6); Enumerate(o); AsList(o);|
  <open orbit, 1 points with Schreier tree with grading>
  <closed orbit, 1 points with Schreier tree with grading>
  [ [ 1 .. 6 ] ]
  !gapprompt@gap>| !gapinput@o:=ImagesOfTransSemigroup(S, 5); Enumerate(o); AsList(o);|
  <open orbit, 1 points with Schreier tree with grading>
  <closed orbit, 1 points with Schreier tree with grading>
  [ [ 1 .. 6 ] ]
  !gapprompt@gap>| !gapinput@o:=ImagesOfTransSemigroup(S, 4); Enumerate(o); AsList(o);|
  <open orbit, 1 points with Schreier tree with grading>
  <closed orbit, 2 points with Schreier tree with grading>
  [ [ 1 .. 6 ], [ 1, 2, 5, 6 ] ]
  !gapprompt@gap>| !gapinput@o:=ImagesOfTransSemigroup(S, 3); Enumerate(o); AsList(o);|
  <open orbit, 1 points with Schreier tree with grading>
  <closed orbit, 4 points with Schreier tree with grading>
  [ [ 1 .. 6 ], [ 1, 4, 6 ], [ 1, 2, 5, 6 ], [ 2, 5, 6 ] ]
  !gapprompt@gap>| !gapinput@o:=ImagesOfTransSemigroup(S, 2); Enumerate(o); AsList(o);|
  <open orbit, 1 points with Schreier tree with grading>
  <closed orbit, 8 points with Schreier tree with grading>
  [ [ 1 .. 6 ], [ 1, 4, 6 ], [ 1, 2, 5, 6 ], [ 2, 6 ], [ 2, 5, 6 ], [ 1, 4 ],
    [ 2, 5 ], [ 4, 6 ] ]
  !gapprompt@gap>| !gapinput@o:=ImagesOfTransSemigroup(S, 1); Enumerate(o); AsList(o);|
  <open orbit, 1 points with Schreier tree with grading>
  <closed orbit, 13 points with Schreier tree with grading>
  [ [ 1 .. 6 ], [ 1, 4, 6 ], [ 1, 2, 5, 6 ], [ 2, 6 ], [ 2, 5, 6 ], [ 1, 4 ],
    [ 2, 5 ], [ 4, 6 ], [ 6 ], [ 1 ], [ 2 ], [ 4 ], [ 5 ] ]
  !gapprompt@gap>| !gapinput@o:=ImagesOfTransSemigroup(S); Enumerate(o); AsList(o);|
  <open orbit, 1 points with Schreier tree>
  <closed orbit, 13 points with Schreier tree>
  [ [ 1 .. 6 ], [ 1, 4, 6 ], [ 1, 2, 5, 6 ], [ 2, 6 ], [ 2, 5, 6 ], [ 1, 4 ],
    [ 2, 5 ], [ 4, 6 ], [ 6 ], [ 1 ], [ 2 ], [ 4 ], [ 5 ] ]
\end{Verbatim}
 }

 }

 
\section{\textcolor{Chapter }{Further properties of semigroups}}\logpage{[ 4, 6, 0 ]}
\hyperdef{L}{X7CC47DE17B361189}{}
{
 In this section we describe several properties  (\textbf{Reference: Properties}) of an arbitrary semigroup of transformations or partial permutations that can
be determined using \textsf{Citrus}. 

\subsection{\textcolor{Chapter }{IsBand}}
\logpage{[ 4, 6, 1 ]}\nobreak
\hyperdef{L}{X7C8DB14587D1B55A}{}
{\noindent\textcolor{FuncColor}{$\triangleright$\ \ \texttt{IsBand({\mdseries\slshape S})\index{IsBand@\texttt{IsBand}}
\label{IsBand}
}\hfill{\scriptsize (property)}}\\
\textbf{\indent Returns:\ }
\texttt{true} or \texttt{false}. 



 \texttt{IsBand} returns \texttt{true} if every element of the semigroup of transformations or partial permutations \mbox{\texttt{\mdseries\slshape S}} is an idempotent and \texttt{false} if it is not. An inverse semigroup is band if and only if it is a semilattice;
see \texttt{IsSemilatticeAsSemigroup} (\ref{IsSemilatticeAsSemigroup}). 
\begin{Verbatim}[commandchars=!@|,fontsize=\small,frame=single,label=Example]
  !gapprompt@gap>| !gapinput@gens:=[ Transformation( [ 1, 1, 1, 4, 4, 4, 7, 7, 7, 1 ] ), |
  !gapprompt@>| !gapinput@Transformation( [ 2, 2, 2, 5, 5, 5, 8, 8, 8, 2 ] ), |
  !gapprompt@>| !gapinput@Transformation( [ 3, 3, 3, 6, 6, 6, 9, 9, 9, 3 ] ), |
  !gapprompt@>| !gapinput@Transformation( [ 1, 1, 1, 4, 4, 4, 7, 7, 7, 4 ] ), |
  !gapprompt@>| !gapinput@Transformation( [ 1, 1, 1, 4, 4, 4, 7, 7, 7, 7 ] ) ];;|
  !gapprompt@gap>| !gapinput@S:=Semigroup(gens);;|
  !gapprompt@gap>| !gapinput@IsBand(S);|
  true
  !gapprompt@gap>| !gapinput@S:=InverseSemigroup(|
  !gapprompt@>| !gapinput@PartialPermNC( [ 1, 2, 3, 4, 8, 9 ], [ 5, 8, 7, 6, 9, 1 ] ),|
  !gapprompt@>| !gapinput@PartialPermNC( [ 1, 3, 4, 7, 8, 9, 10 ], [ 2, 3, 8, 7, 10, 6, 1 ] ) );;|
  !gapprompt@gap>| !gapinput@IsBand(S);|
  false
  !gapprompt@gap>| !gapinput@IsBand(IdempotentGeneratedSubsemigp(S));|
  true
\end{Verbatim}
 }

 

\subsection{\textcolor{Chapter }{IsBlockGroup}}
\logpage{[ 4, 6, 2 ]}\nobreak
\hyperdef{L}{X79659C467C8A7EBD}{}
{\noindent\textcolor{FuncColor}{$\triangleright$\ \ \texttt{IsBlockGroup({\mdseries\slshape S})\index{IsBlockGroup@\texttt{IsBlockGroup}}
\label{IsBlockGroup}
}\hfill{\scriptsize (property)}}\\
\noindent\textcolor{FuncColor}{$\triangleright$\ \ \texttt{IsSemigroupWithCommutingIdempotents({\mdseries\slshape S})\index{IsSemigroupWithCommutingIdempotents@\texttt{IsSemigroupWithCommutingIdempotents}}
\label{IsSemigroupWithCommutingIdempotents}
}\hfill{\scriptsize (property)}}\\
\textbf{\indent Returns:\ }
\texttt{true} or \texttt{false}. 



 \texttt{IsBlockGroup} and \texttt{IsSemigroupWithCommutingIdempotents} return \texttt{true} if the semigroup \mbox{\texttt{\mdseries\slshape S}} of transformations or partial permutations is a block group and \texttt{false} if it is not.

 A semigroup \mbox{\texttt{\mdseries\slshape S}} is a \emph{block group} if every $\mathcal{L}$-class and every $\mathcal{R}$-class of \mbox{\texttt{\mdseries\slshape S}} contains at most one idempotent. Every semigroup of partial permutations is a
block group. 
\begin{Verbatim}[commandchars=!@|,fontsize=\small,frame=single,label=Example]
  !gapprompt@gap>| !gapinput@S:=Semigroup(Transformation( [ 5, 6, 7, 3, 1, 4, 2, 8 ] ),|
  !gapprompt@>| !gapinput@  Transformation( [ 3, 6, 8, 5, 7, 4, 2, 8 ] ));;|
  !gapprompt@gap>| !gapinput@IsBlockGroup(S);|
  true
  !gapprompt@gap>| !gapinput@S:=Semigroup(Transformation( [ 2, 1, 10, 4, 5, 9, 7, 4, 8, 4 ] ),|
  !gapprompt@>| !gapinput@Transformation( [ 10, 7, 5, 6, 1, 3, 9, 7, 10, 2 ] ));;|
  !gapprompt@gap>| !gapinput@IsBlockGroup(S);|
  false
\end{Verbatim}
 }

 

\subsection{\textcolor{Chapter }{IsBrandtSemigroup}}
\logpage{[ 4, 6, 3 ]}\nobreak
\hyperdef{L}{X7EFDBA687DCDA6FA}{}
{\noindent\textcolor{FuncColor}{$\triangleright$\ \ \texttt{IsBrandtSemigroup({\mdseries\slshape S})\index{IsBrandtSemigroup@\texttt{IsBrandtSemigroup}}
\label{IsBrandtSemigroup}
}\hfill{\scriptsize (property)}}\\
\textbf{\indent Returns:\ }
\texttt{true} or \texttt{false}. 



 \texttt{IsBrandtSemigroup} return \texttt{true} if the semigroup of transformations or partial permutations is a 0-simple
inverse semigroup, and \texttt{false} if it is not. See also \texttt{IsZeroSimpleSemigroup} (\ref{IsZeroSimpleSemigroup}) and \texttt{IsInverseSemigroup} (\ref{IsInverseSemigroup}). 
\begin{Verbatim}[commandchars=!@|,fontsize=\small,frame=single,label=Example]
  !gapprompt@gap>| !gapinput@S:=Semigroup(Transformation( [ 2, 8, 8, 8, 8, 8, 8, 8 ] ),|
  !gapprompt@>| !gapinput@Transformation( [ 5, 8, 8, 8, 8, 8, 8, 8 ] ),|
  !gapprompt@>| !gapinput@Transformation( [ 8, 3, 8, 8, 8, 8, 8, 8 ] ),|
  !gapprompt@>| !gapinput@Transformation( [ 8, 6, 8, 8, 8, 8, 8, 8 ] ),|
  !gapprompt@>| !gapinput@Transformation( [ 8, 8, 1, 8, 8, 8, 8, 8 ] ),|
  !gapprompt@>| !gapinput@Transformation( [ 8, 8, 8, 1, 8, 8, 8, 8 ] ),|
  !gapprompt@>| !gapinput@Transformation( [ 8, 8, 8, 8, 4, 8, 8, 8 ] ),|
  !gapprompt@>| !gapinput@Transformation( [ 8, 8, 8, 8, 8, 7, 8, 8 ] ),|
  !gapprompt@>| !gapinput@Transformation( [ 8, 8, 8, 8, 8, 8, 2, 8 ] ));;|
  !gapprompt@gap>| !gapinput@IsBrandtSemigroup(S);|
  true
  !gapprompt@gap>| !gapinput@T:=Range(IsomorphismPartialPermSemigroup(S));;|
  !gapprompt@gap>| !gapinput@IsBrandtSemigroup(T);|
  true
\end{Verbatim}
 }

 

\subsection{\textcolor{Chapter }{IsCliffordSemigroup}}
\logpage{[ 4, 6, 4 ]}\nobreak
\hyperdef{L}{X81DE11987BB81017}{}
{\noindent\textcolor{FuncColor}{$\triangleright$\ \ \texttt{IsCliffordSemigroup({\mdseries\slshape S})\index{IsCliffordSemigroup@\texttt{IsCliffordSemigroup}}
\label{IsCliffordSemigroup}
}\hfill{\scriptsize (property)}}\\
\textbf{\indent Returns:\ }
\texttt{true} or \texttt{false}. 



 \texttt{IsCliffordSemigroup} returns \texttt{true} if the semigroup of transformations or partial permutations \mbox{\texttt{\mdseries\slshape S}} is regular and its idempotents are central, and \texttt{false} if it is not. 
\begin{Verbatim}[commandchars=!@|,fontsize=\small,frame=single,label=Example]
  !gapprompt@gap>| !gapinput@S:=Semigroup( Transformation( [ 1, 2, 4, 5, 6, 3, 7, 8 ] ), |
  !gapprompt@>| !gapinput@Transformation( [ 3, 3, 4, 5, 6, 2, 7, 8 ] ), |
  !gapprompt@>| !gapinput@Transformation( [ 1, 2, 5, 3, 6, 8, 4, 4 ] ) );;|
  !gapprompt@gap>| !gapinput@IsCliffordSemigroup(S);|
  true
  !gapprompt@gap>| !gapinput@T:=Range(IsomorphismPartialPermSemigroup(S));;|
  !gapprompt@gap>| !gapinput@IsCliffordSemigroup(S);|
  true
\end{Verbatim}
 }

 

\subsection{\textcolor{Chapter }{IsCommutativeSemigroup}}
\logpage{[ 4, 6, 5 ]}\nobreak
\hyperdef{L}{X843EFDA4807FDC31}{}
{\noindent\textcolor{FuncColor}{$\triangleright$\ \ \texttt{IsCommutativeSemigroup({\mdseries\slshape S})\index{IsCommutativeSemigroup@\texttt{IsCommutativeSemigroup}}
\label{IsCommutativeSemigroup}
}\hfill{\scriptsize (property)}}\\
\textbf{\indent Returns:\ }
\texttt{true} or \texttt{false}. 



 \texttt{IsCommutativeSemigroup} returns \texttt{true} if the semigroup of transformations or partial permutations \mbox{\texttt{\mdseries\slshape S}} is commutative and \texttt{false} if it is not. The function \texttt{IsCommutative} (\textbf{Reference: IsCommutative}) can also be used to test if a semigroup is commutative. 

 A semigroup \mbox{\texttt{\mdseries\slshape S}} is \emph{commutative} if \texttt{x*y=y*x} for all \texttt{x,y} in \mbox{\texttt{\mdseries\slshape S}}. 
\begin{Verbatim}[commandchars=!@|,fontsize=\small,frame=single,label=Example]
  !gapprompt@gap>| !gapinput@gens:=[ Transformation( [ 2, 4, 5, 3, 7, 8, 6, 9, 1 ] ), |
  !gapprompt@>| !gapinput@ Transformation( [ 3, 5, 6, 7, 8, 1, 9, 2, 4 ] ) ];;|
  !gapprompt@gap>| !gapinput@S:=Semigroup(gens);;|
  !gapprompt@gap>| !gapinput@IsCommutativeSemigroup(S);|
  true
  !gapprompt@gap>| !gapinput@IsCommutative(S);|
  true
  !gapprompt@gap>| !gapinput@S:=InverseSemigroup(|
  !gapprompt@>| !gapinput@ PartialPermNC( [ 1, 2, 3, 4, 5, 6 ], [ 2, 5, 1, 3, 9, 6 ] ),|
  !gapprompt@>| !gapinput@ PartialPermNC( [ 1, 2, 3, 4, 6, 8 ], [ 8, 5, 7, 6, 2, 1 ] ) );;|
  !gapprompt@gap>| !gapinput@IsCommutativeSemigroup(S);|
  false
\end{Verbatim}
 }

 

\subsection{\textcolor{Chapter }{IsCompletelyRegularSemigroup}}
\logpage{[ 4, 6, 6 ]}\nobreak
\hyperdef{L}{X7AFA23AF819FBF3D}{}
{\noindent\textcolor{FuncColor}{$\triangleright$\ \ \texttt{IsCompletelyRegularSemigroup({\mdseries\slshape S})\index{IsCompletelyRegularSemigroup@\texttt{IsCompletelyRegularSemigroup}}
\label{IsCompletelyRegularSemigroup}
}\hfill{\scriptsize (property)}}\\
\textbf{\indent Returns:\ }
\texttt{true} or \texttt{false}. 



 \texttt{IsCompletelyRegularSemigroup} returns \texttt{true} if every element of the semigroup of transformations or partial permutations \mbox{\texttt{\mdseries\slshape S}} is contained in a subgroup of \mbox{\texttt{\mdseries\slshape S}}.

 An inverse semigroup is completely regular if and only if it is a Clifford
semigroup; see \texttt{IsCliffordSemigroup} (\ref{IsCliffordSemigroup}). 
\begin{Verbatim}[commandchars=!@|,fontsize=\small,frame=single,label=Example]
  !gapprompt@gap>| !gapinput@gens:=[ Transformation( [ 1, 2, 4, 3, 6, 5, 4 ] ), |
  !gapprompt@>| !gapinput@ Transformation( [ 1, 2, 5, 6, 3, 4, 5 ] ), |
  !gapprompt@>| !gapinput@ Transformation( [ 2, 1, 2, 2, 2, 2, 2 ] ) ];;|
  !gapprompt@gap>| !gapinput@S:=Semigroup(gens);;|
  !gapprompt@gap>| !gapinput@IsCompletelyRegularSemigroup(S);|
  true
  !gapprompt@gap>| !gapinput@IsInverseSemigroup(S);|
  true
  !gapprompt@gap>| !gapinput@T:=Range(IsomorphismPartialPermSemigroup(S));;|
  !gapprompt@gap>| !gapinput@IsCompletelyRegularSemigroup(T);|
  true
  !gapprompt@gap>| !gapinput@IsCliffordSemigroup(T);         |
  true
\end{Verbatim}
 }

 

\subsection{\textcolor{Chapter }{IsFactorisableSemigroup}}
\logpage{[ 4, 6, 7 ]}\nobreak
\hyperdef{L}{X862158348720781D}{}
{\noindent\textcolor{FuncColor}{$\triangleright$\ \ \texttt{IsFactorisableSemigroup({\mdseries\slshape S})\index{IsFactorisableSemigroup@\texttt{IsFactorisableSemigroup}}
\label{IsFactorisableSemigroup}
}\hfill{\scriptsize (property)}}\\
\textbf{\indent Returns:\ }
\texttt{true} or \texttt{false}.



 An inverse monoid is \emph{factorisable} if every element is the product of an element of the group of units and an
idempotent; see also \texttt{GroupOfUnits} (\ref{GroupOfUnits}) and \texttt{Idempotents} (\ref{Idempotents}). Hence an inverse semigroup of partial permutations is factorisable if and
only if each of its generators is the restriction of some element in the group
of units. 
\begin{Verbatim}[commandchars=!@|,fontsize=\small,frame=single,label=Example]
  !gapprompt@gap>| !gapinput@S:=InverseSemigroup( PartialPermNC( [ 1, 2, 4 ], [ 3, 1, 4 ] ),|
  !gapprompt@>| !gapinput@PartialPermNC( [ 1, 2, 3, 5 ], [ 4, 1, 5, 2 ] ) );;|
  !gapprompt@gap>| !gapinput@IsFactorisableSemigroup(S);|
  false
  !gapprompt@gap>| !gapinput@IsFactorisableSemigroup(SymmetricInverseSemigp(5)); |
  true
\end{Verbatim}
 }

 

\subsection{\textcolor{Chapter }{IsGroupAsSemigroup}}
\logpage{[ 4, 6, 8 ]}\nobreak
\hyperdef{L}{X852F29E8795FA489}{}
{\noindent\textcolor{FuncColor}{$\triangleright$\ \ \texttt{IsGroupAsSemigroup({\mdseries\slshape S})\index{IsGroupAsSemigroup@\texttt{IsGroupAsSemigroup}}
\label{IsGroupAsSemigroup}
}\hfill{\scriptsize (property)}}\\
\textbf{\indent Returns:\ }
\texttt{true} or \texttt{false}.



 If the semigroup of transformations or partial permutations \mbox{\texttt{\mdseries\slshape S}} is actually a group, then \texttt{IsGroupAsSemigroup} returns \texttt{true}. If it is not a group, then \texttt{false} is returned. 
\begin{Verbatim}[commandchars=!@|,fontsize=\small,frame=single,label=Example]
  !gapprompt@gap>| !gapinput@gens:=[ Transformation( [ 2, 4, 5, 3, 7, 8, 6, 9, 1 ] ), |
  !gapprompt@>| !gapinput@ Transformation( [ 3, 5, 6, 7, 8, 1, 9, 2, 4 ] ) ];;|
  !gapprompt@gap>| !gapinput@S:=Semigroup(gens);;|
  !gapprompt@gap>| !gapinput@IsGroupAsSemigroup(S);|
  true
  !gapprompt@gap>| !gapinput@G:=SymmetricGroup(5);;|
  !gapprompt@gap>| !gapinput@S:=Range(IsomorphismPartialPermSemigroup(G));|
  <inverse semigroup with 2 generators>
  !gapprompt@gap>| !gapinput@IsGroupAsSemigroup(S);|
  true
\end{Verbatim}
 }

 
\subsection{\textcolor{Chapter }{IsIdempotentGenerated}}\logpage{[ 4, 6, 9 ]}
\hyperdef{L}{X835484C481CF3DDD}{}
{
\noindent\textcolor{FuncColor}{$\triangleright$\ \ \texttt{IsIdempotentGenerated({\mdseries\slshape S})\index{IsIdempotentGenerated@\texttt{IsIdempotentGenerated}}
\label{IsIdempotentGenerated}
}\hfill{\scriptsize (property)}}\\
\noindent\textcolor{FuncColor}{$\triangleright$\ \ \texttt{IsSemiBand({\mdseries\slshape S})\index{IsSemiBand@\texttt{IsSemiBand}}
\label{IsSemiBand}
}\hfill{\scriptsize (property)}}\\
\textbf{\indent Returns:\ }
\texttt{true} or \texttt{false}. 



 \texttt{IsIdempotentGenerated} and \texttt{IsSemiBand} return \texttt{true} if the semigroup of transformations or partial permutations \mbox{\texttt{\mdseries\slshape S}} is generated by its idempotents and \texttt{false} if it is not. See also \texttt{Idempotents} (\ref{Idempotents}) and \texttt{IdempotentGeneratedSubsemigp} (\ref{IdempotentGeneratedSubsemigp}). 

 An inverse semigroup is idempotent-generated if and only if it is a
semilattice; see \texttt{IsSemilatticeAsSemigroup} (\ref{IsSemilatticeAsSemigroup}).

 Semiband and idempotent-generated are synonymous in this context. 
\begin{Verbatim}[commandchars=!@|,fontsize=\small,frame=single,label=Example]
  !gapprompt@gap>| !gapinput@S:=SingularSemigroup(4);|
  <semigroup with 12 generators>
  !gapprompt@gap>| !gapinput@IsIdempotentGenerated(S);|
  true
\end{Verbatim}
 }

 

\subsection{\textcolor{Chapter }{IsInverseSemigroup}}
\logpage{[ 4, 6, 10 ]}\nobreak
\hyperdef{L}{X83F1529479D56665}{}
{\noindent\textcolor{FuncColor}{$\triangleright$\ \ \texttt{IsInverseSemigroup({\mdseries\slshape S})\index{IsInverseSemigroup@\texttt{IsInverseSemigroup}}
\label{IsInverseSemigroup}
}\hfill{\scriptsize (property)}}\\
\noindent\textcolor{FuncColor}{$\triangleright$\ \ \texttt{IsInverseMonoid({\mdseries\slshape S})\index{IsInverseMonoid@\texttt{IsInverseMonoid}}
\label{IsInverseMonoid}
}\hfill{\scriptsize (property)}}\\
\textbf{\indent Returns:\ }
\texttt{true} or \texttt{false}.



 If \mbox{\texttt{\mdseries\slshape S}} is a semigroup of transformations or partial permutations, then \texttt{IsInverseSemigroup} returns \texttt{true} if \mbox{\texttt{\mdseries\slshape S}} is an inverse semigroup and \texttt{false} if it is not. If \mbox{\texttt{\mdseries\slshape S}} is monoid of transformations or partial permutations, then \texttt{IsInverseMonoid} returns \texttt{true} if \mbox{\texttt{\mdseries\slshape S}} is an inverse monoid and \texttt{false} if it is not.

 A semigroup is an \emph{inverse semigroup} if every element \texttt{x} has a unique semigroup inverse, that is, a unique element \texttt{y} such that \texttt{x*y*x=x} and \texttt{y*x*y=y}. 
\begin{Verbatim}[commandchars=!@|,fontsize=\small,frame=single,label=Example]
  !gapprompt@gap>| !gapinput@gens:=[Transformation([1,2,4,5,6,3,7,8]),|
  !gapprompt@>| !gapinput@Transformation([3,3,4,5,6,2,7,8]),|
  !gapprompt@>| !gapinput@Transformation([1,2,5,3,6,8,4,4])];;|
  !gapprompt@gap>| !gapinput@S:=Semigroup(gens);;|
  !gapprompt@gap>| !gapinput@IsInverseSemigroup(S);|
  true
\end{Verbatim}
 }

 

\subsection{\textcolor{Chapter }{IsLeftSimple}}
\logpage{[ 4, 6, 11 ]}\nobreak
\hyperdef{L}{X8206D2B0809952EF}{}
{\noindent\textcolor{FuncColor}{$\triangleright$\ \ \texttt{IsLeftSimple({\mdseries\slshape S})\index{IsLeftSimple@\texttt{IsLeftSimple}}
\label{IsLeftSimple}
}\hfill{\scriptsize (property)}}\\
\noindent\textcolor{FuncColor}{$\triangleright$\ \ \texttt{IsRightSimple({\mdseries\slshape S})\index{IsRightSimple@\texttt{IsRightSimple}}
\label{IsRightSimple}
}\hfill{\scriptsize (property)}}\\
\textbf{\indent Returns:\ }
\texttt{true} or \texttt{false}. 



 \texttt{IsLeftSimple} and \texttt{IsRightSimple} returns \texttt{true} if the semigroup of transformations or partial permutations \mbox{\texttt{\mdseries\slshape S}} has only one $\mathcal{L}$-class or one $\mathcal{R}$-class, respectively, and returns \texttt{false} if it has more than one. 

 An inverse semigroup is left simple if and only if it is right simple if and
only if it is a group; see \texttt{IsGroupAsSemigroup} (\ref{IsGroupAsSemigroup}). 
\begin{Verbatim}[commandchars=!@|,fontsize=\small,frame=single,label=Example]
  !gapprompt@gap>| !gapinput@S:=Semigroup( Transformation( [ 6, 7, 9, 6, 8, 9, 8, 7, 6 ] ), |
  !gapprompt@>| !gapinput@ Transformation( [ 6, 8, 9, 6, 8, 8, 7, 9, 6 ] ), |
  !gapprompt@>| !gapinput@ Transformation( [ 6, 8, 9, 7, 8, 8, 7, 9, 6 ] ), |
  !gapprompt@>| !gapinput@ Transformation( [ 6, 9, 8, 6, 7, 9, 7, 8, 6 ] ), |
  !gapprompt@>| !gapinput@ Transformation( [ 6, 9, 9, 6, 8, 8, 7, 9, 6 ] ), |
  !gapprompt@>| !gapinput@ Transformation( [ 6, 9, 9, 7, 8, 8, 6, 9, 7 ] ), |
  !gapprompt@>| !gapinput@ Transformation( [ 7, 8, 8, 7, 9, 9, 7, 8, 6 ] ), |
  !gapprompt@>| !gapinput@ Transformation( [ 7, 9, 9, 7, 6, 9, 6, 8, 7 ] ), |
  !gapprompt@>| !gapinput@ Transformation( [ 8, 7, 6, 9, 8, 6, 8, 7, 9 ] ), |
  !gapprompt@>| !gapinput@ Transformation( [ 9, 6, 6, 7, 8, 8, 7, 6, 9 ] ), |
  !gapprompt@>| !gapinput@ Transformation( [ 9, 6, 6, 7, 9, 6, 9, 8, 7 ] ), |
  !gapprompt@>| !gapinput@ Transformation( [ 9, 6, 7, 9, 6, 6, 9, 7, 8 ] ), |
  !gapprompt@>| !gapinput@ Transformation( [ 9, 6, 8, 7, 9, 6, 9, 8, 7 ] ), |
  !gapprompt@>| !gapinput@ Transformation( [ 9, 7, 6, 8, 7, 7, 9, 6, 8 ] ), |
  !gapprompt@>| !gapinput@ Transformation( [ 9, 7, 7, 8, 9, 6, 9, 7, 8 ] ), |
  !gapprompt@>| !gapinput@ Transformation( [ 9, 8, 8, 9, 6, 7, 6, 8, 9 ] ) );;|
  !gapprompt@gap>| !gapinput@IsRightSimple(S);|
  false
  !gapprompt@gap>| !gapinput@IsLeftSimple(S);|
  true
  !gapprompt@gap>| !gapinput@IsGroupAsSemigroup(S);|
  false
  !gapprompt@gap>| !gapinput@NrRClasses(S);|
  16
\end{Verbatim}
 }

 

\subsection{\textcolor{Chapter }{IsLeftZeroSemigroup}}
\logpage{[ 4, 6, 12 ]}\nobreak
\hyperdef{L}{X7E9261367C8C52C0}{}
{\noindent\textcolor{FuncColor}{$\triangleright$\ \ \texttt{IsLeftZeroSemigroup({\mdseries\slshape S})\index{IsLeftZeroSemigroup@\texttt{IsLeftZeroSemigroup}}
\label{IsLeftZeroSemigroup}
}\hfill{\scriptsize (property)}}\\
\textbf{\indent Returns:\ }
\texttt{true} or \texttt{false}. 



 \texttt{IsLeftZeroSemigroup} returns \texttt{true} if the semigroup \mbox{\texttt{\mdseries\slshape S}} of transformations or partial permutations is a left zero semigroup and \texttt{false} if it is not. 

 A semigroup is a \emph{left zero semigroup} if \texttt{x*y=x} for all \texttt{x,y}. An inverse semigroup is a left zero semigroup if and only if it is trivial. 
\begin{Verbatim}[commandchars=!@|,fontsize=\small,frame=single,label=Example]
  !gapprompt@gap>| !gapinput@gens:=[ Transformation( [ 2, 1, 4, 3, 5 ] ), |
  !gapprompt@>| !gapinput@ Transformation( [ 3, 2, 3, 1, 1 ] ) ];;|
  !gapprompt@gap>| !gapinput@S:=Semigroup(gens);;|
  !gapprompt@gap>| !gapinput@IsRightZeroSemigroup(S);|
  false
  !gapprompt@gap>| !gapinput@gens:=[Transformation( [ 1, 2, 3, 3, 1 ] ), |
  !gapprompt@>| !gapinput@Transformation( [ 1, 2, 3, 3, 3 ] ) ];;|
  !gapprompt@gap>| !gapinput@S:=Semigroup(gens);;|
  !gapprompt@gap>| !gapinput@IsLeftZeroSemigroup(S);|
  true
\end{Verbatim}
 }

 

\subsection{\textcolor{Chapter }{IsMonogenicSemigroup}}
\logpage{[ 4, 6, 13 ]}\nobreak
\hyperdef{L}{X79D46BAB7E327AD1}{}
{\noindent\textcolor{FuncColor}{$\triangleright$\ \ \texttt{IsMonogenicSemigroup({\mdseries\slshape S})\index{IsMonogenicSemigroup@\texttt{IsMonogenicSemigroup}}
\label{IsMonogenicSemigroup}
}\hfill{\scriptsize (property)}}\\
\textbf{\indent Returns:\ }
\texttt{true} or \texttt{false}. 



 \texttt{IsMonogenicSemigroup} returns \texttt{true} if the semigroup \mbox{\texttt{\mdseries\slshape S}} of transformations or partial permutations is monogenic and it returns \texttt{false} if it is not. 

 A semigroup is \emph{monogenic} if it is generated by a single element. See also \texttt{IsMonogenicInverseSemigroup} (\ref{IsMonogenicInverseSemigroup}) and \texttt{IndexPeriodOfTransformation} (\ref{IndexPeriodOfTransformation}). 
\begin{Verbatim}[commandchars=!@|,fontsize=\small,frame=single,label=Example]
  !gapprompt@gap>| !gapinput@S:=Semigroup(|
  !gapprompt@>| !gapinput@Transformation( [ 2, 2, 2, 11, 10, 8, 10, 11, 2, 11, 10, 2, 11, 11, 10 ] ),|
  !gapprompt@>| !gapinput@Transformation( [ 2, 2, 2, 8, 11, 15, 11, 10, 2, 10, 11, 2, 10, 4, 7 ] ), |
  !gapprompt@>| !gapinput@Transformation( [ 2, 2, 2, 11, 10, 8, 10, 11, 2, 11, 10, 2, 11, 11, 10 ] ),|
  !gapprompt@>| !gapinput@Transformation( [ 2, 2, 12, 7, 8, 14, 8, 11, 2, 11, 10, 2, 11, 15, 4 ] ));;|
  !gapprompt@gap>| !gapinput@IsMonogenicSemigroup(S);|
  true
\end{Verbatim}
 }

 

\subsection{\textcolor{Chapter }{IsMonogenicInverseSemigroup}}
\logpage{[ 4, 6, 14 ]}\nobreak
\hyperdef{L}{X7D2641AD830DEC1C}{}
{\noindent\textcolor{FuncColor}{$\triangleright$\ \ \texttt{IsMonogenicInverseSemigroup({\mdseries\slshape S})\index{IsMonogenicInverseSemigroup@\texttt{IsMonogenicInverseSemigroup}}
\label{IsMonogenicInverseSemigroup}
}\hfill{\scriptsize (property)}}\\
\textbf{\indent Returns:\ }
\texttt{true} or \texttt{false}. 



 \texttt{IsMonogenicInverseSemigroup} returns \texttt{true} if the semigroup \mbox{\texttt{\mdseries\slshape S}} of transformations or partial permutations is an inverse monogenic semigroup
and it returns \texttt{false} if it is not. 

 A inverse semigroup is \emph{monogenic} if it is generated as an inverse semigroup by a single element. See also \texttt{IsMonogenicSemigroup} (\ref{IsMonogenicSemigroup}) and \texttt{IndexPeriodOfTransformation} (\ref{IndexPeriodOfTransformation}). 
\begin{Verbatim}[commandchars=!@|,fontsize=\small,frame=single,label=Example]
  !gapprompt@gap>| !gapinput@f:=PartialPermNC( [ 1, 2, 3, 6, 8, 10 ], [ 2, 6, 7, 9, 1, 5 ] );;|
  !gapprompt@gap>| !gapinput@S:=InverseSemigroup(f, f^2, f^3);;|
  !gapprompt@gap>| !gapinput@IsMonogenicSemigroup(S);|
  false
  !gapprompt@gap>| !gapinput@IsMonogenicInverseSemigroup(S);|
  true
\end{Verbatim}
 }

 

\subsection{\textcolor{Chapter }{IsMonoidAsSemigroup}}
\logpage{[ 4, 6, 15 ]}\nobreak
\hyperdef{L}{X7E4DEECD7CD9886D}{}
{\noindent\textcolor{FuncColor}{$\triangleright$\ \ \texttt{IsMonoidAsSemigroup({\mdseries\slshape S})\index{IsMonoidAsSemigroup@\texttt{IsMonoidAsSemigroup}}
\label{IsMonoidAsSemigroup}
}\hfill{\scriptsize (property)}}\\
\textbf{\indent Returns:\ }
\texttt{true} or \texttt{false}. 



 \texttt{IsMonoidAsSemigroup} returns \texttt{true} if the semigroup of transformations or partial permutations \mbox{\texttt{\mdseries\slshape S}} is a monoid and \texttt{false} if it is not. It is possible that \mbox{\texttt{\mdseries\slshape S}} is a monoid but does not satisfy \texttt{IsMonoid} (\textbf{Reference: IsMonoid}) and so \mbox{\texttt{\mdseries\slshape S}} does not possess the attributes of a monoid (such as, \texttt{GeneratorsOfMonoid} (\textbf{Reference: GeneratorsOfMonoid})).

 A semigroup of transformations satisfies \texttt{IsMonoidAsSemigroup} if and only if it satisfies \texttt{IsTransformationMonoid} (\textbf{Reference: IsTransformationMonoid}). A semigroup of partial permutations satisfies \texttt{IsMonoidAsSemigroup} if and only if it satisfies \texttt{IsPartialPermMonoid} (\ref{IsPartialPermMonoid}). 

 See also \texttt{One} (\textbf{Reference: One}), \texttt{IsInverseMonoid} (\ref{IsInverseMonoid}) and \texttt{IsomorphismTransformationMonoid} (\ref{IsomorphismTransformationMonoid}). 
\begin{Verbatim}[commandchars=!@|,fontsize=\small,frame=single,label=Example]
  !gapprompt@gap>| !gapinput@S:=Semigroup( Transformation( [ 1, 4, 6, 2, 5, 3, 7, 8, 9, 9 ] ),|
  !gapprompt@>| !gapinput@Transformation( [ 6, 3, 2, 7, 5, 1, 8, 8, 9, 9 ] ) );;|
  !gapprompt@gap>| !gapinput@IsMonoidAsSemigroup(S);|
  true
  !gapprompt@gap>| !gapinput@MultiplicativeNeutralElement(S);|
  Transformation( [ 1, 2, 3, 4, 5, 6, 7, 8, 9, 9 ] )
  !gapprompt@gap>| !gapinput@S:=Monoid(Transformation( [ 8, 2, 8, 9, 10, 6, 2, 8, 7, 8 ] ),|
  !gapprompt@>| !gapinput@Transformation( [ 9, 2, 6, 3, 6, 4, 5, 5, 3, 2 ] ));;|
  !gapprompt@gap>| !gapinput@IsMonoidAsSemigroup(S);|
  true
\end{Verbatim}
 }

 

\subsection{\textcolor{Chapter }{IsPartialPermSemigroup}}
\logpage{[ 4, 6, 16 ]}\nobreak
\hyperdef{L}{X7D161674800B50E0}{}
{\noindent\textcolor{FuncColor}{$\triangleright$\ \ \texttt{IsPartialPermSemigroup({\mdseries\slshape S})\index{IsPartialPermSemigroup@\texttt{IsPartialPermSemigroup}}
\label{IsPartialPermSemigroup}
}\hfill{\scriptsize (property)}}\\
\noindent\textcolor{FuncColor}{$\triangleright$\ \ \texttt{IsPartialPermMonoid({\mdseries\slshape S})\index{IsPartialPermMonoid@\texttt{IsPartialPermMonoid}}
\label{IsPartialPermMonoid}
}\hfill{\scriptsize (property)}}\\
\textbf{\indent Returns:\ }
\texttt{true} or \texttt{false}.



 \texttt{IsPartialPermSemigroup} returns \texttt{true} if \mbox{\texttt{\mdseries\slshape S}} is a semigroup of partial permutations and \texttt{false} if it is not. \texttt{IsPartialPermMonoid} returns \texttt{true} if \mbox{\texttt{\mdseries\slshape S}} is a monoid of partial permutations and \texttt{false} if it is not. Note that a semigroup \mbox{\texttt{\mdseries\slshape S}} satisfies \texttt{IsPartialPermMonoid} if and only if it contains an identity but \mbox{\texttt{\mdseries\slshape S}} may or may not satisfy \texttt{IsMonoid} (\textbf{Reference: IsMonoid}); see \texttt{IsMonoidAsSemigroup} (\ref{IsMonoidAsSemigroup}). }

 

\subsection{\textcolor{Chapter }{IsOrthodoxSemigroup}}
\logpage{[ 4, 6, 17 ]}\nobreak
\hyperdef{L}{X7935C476808C8773}{}
{\noindent\textcolor{FuncColor}{$\triangleright$\ \ \texttt{IsOrthodoxSemigroup({\mdseries\slshape S})\index{IsOrthodoxSemigroup@\texttt{IsOrthodoxSemigroup}}
\label{IsOrthodoxSemigroup}
}\hfill{\scriptsize (property)}}\\
\textbf{\indent Returns:\ }
\texttt{true} or \texttt{false}. 



 \texttt{IsOrthodoxSemigroup} returns \texttt{true} if the semigroup \mbox{\texttt{\mdseries\slshape S}} of transformations or partial permutations is orthodox and \texttt{false} if it is not.

 A semigroup is \emph{orthodox} if it is regular and its idempotent elements form a subsemigroup. Every
inverse semigroup is also an orthodox semigroup. 

 See also \texttt{IsRegularSemigroup} (\ref{IsRegularSemigroup}) and \texttt{IsRegularSemigroup} (\textbf{Reference: IsRegularSemigroup}). 
\begin{Verbatim}[commandchars=!@|,fontsize=\small,frame=single,label=Example]
  !gapprompt@gap>| !gapinput@gens:=[ Transformation( [ 1, 1, 1, 4, 5, 4 ] ), |
  !gapprompt@>| !gapinput@ Transformation( [ 1, 2, 3, 1, 1, 2 ] ), |
  !gapprompt@>| !gapinput@ Transformation( [ 1, 2, 3, 1, 1, 3 ] ), |
  !gapprompt@>| !gapinput@ Transformation( [ 5, 5, 5, 5, 5, 5 ] ) ];;|
  !gapprompt@gap>| !gapinput@S:=Semigroup(gens);;|
  !gapprompt@gap>| !gapinput@IsOrthodoxSemigroup(S);|
  true
\end{Verbatim}
 }

 

\subsection{\textcolor{Chapter }{IsRectangularBand}}
\logpage{[ 4, 6, 18 ]}\nobreak
\hyperdef{L}{X7E9B674D781B072C}{}
{\noindent\textcolor{FuncColor}{$\triangleright$\ \ \texttt{IsRectangularBand({\mdseries\slshape S})\index{IsRectangularBand@\texttt{IsRectangularBand}}
\label{IsRectangularBand}
}\hfill{\scriptsize (property)}}\\
\textbf{\indent Returns:\ }
\texttt{true} or \texttt{false}. 



 \texttt{IsRectangularBand} returns \texttt{true} if the semigroup \mbox{\texttt{\mdseries\slshape S}} of transformations or partial permutations is a rectangular band and \texttt{false} if it is not.

 A semigroup \mbox{\texttt{\mdseries\slshape S}} is a \emph{rectangular band} if for all \texttt{x,y,z} in \mbox{\texttt{\mdseries\slshape S}} we have that \texttt{x\texttt{\symbol{94}}2=x} and \texttt{xyz=xz}. An inverse semigroup is a rectangular band if and only if it is a group. 
\begin{Verbatim}[commandchars=!@|,fontsize=\small,frame=single,label=Example]
  !gapprompt@gap>| !gapinput@gens:=[ Transformation( [ 1, 1, 1, 4, 4, 4, 7, 7, 7, 1 ] ), |
  !gapprompt@>| !gapinput@Transformation( [ 2, 2, 2, 5, 5, 5, 8, 8, 8, 2 ] ), |
  !gapprompt@>| !gapinput@Transformation( [ 3, 3, 3, 6, 6, 6, 9, 9, 9, 3 ] ), |
  !gapprompt@>| !gapinput@Transformation( [ 1, 1, 1, 4, 4, 4, 7, 7, 7, 4 ] ), |
  !gapprompt@>| !gapinput@Transformation( [ 1, 1, 1, 4, 4, 4, 7, 7, 7, 7 ] ) ];;|
  !gapprompt@gap>| !gapinput@S:=Semigroup(gens);;|
  !gapprompt@gap>| !gapinput@IsRectangularBand(S);|
  true
\end{Verbatim}
 }

 

\subsection{\textcolor{Chapter }{IsRegularSemigroup}}
\logpage{[ 4, 6, 19 ]}\nobreak
\hyperdef{L}{X7C4663827C5ACEF1}{}
{\noindent\textcolor{FuncColor}{$\triangleright$\ \ \texttt{IsRegularSemigroup({\mdseries\slshape S})\index{IsRegularSemigroup@\texttt{IsRegularSemigroup}}
\label{IsRegularSemigroup}
}\hfill{\scriptsize (property)}}\\
\textbf{\indent Returns:\ }
\texttt{true} or \texttt{false}. 



 \texttt{IsRegularSemigroup} returns \texttt{true} if the semigroup of transformations or partial permutations \mbox{\texttt{\mdseries\slshape S}} is regular and \texttt{false} if it is not. 

 A semigroup \texttt{S} is \emph{regular} if for all \texttt{x} in \texttt{S} there exists \texttt{y} in \texttt{S} such that \texttt{x*y*x=x}. Every inverse semigroup is regular, and a semigroup of partial permutations
is regular if and only if it is an inverse semigroup.

 See also \texttt{IsRegularDClass} (\textbf{Reference: IsRegularDClass}), \texttt{IsRegularDClass} (\ref{IsRegularDClass}), and \texttt{IsRegularTransformation} (\ref{IsRegularTransformation}). 
\begin{Verbatim}[commandchars=!@|,fontsize=\small,frame=single,label=Example]
  !gapprompt@gap>| !gapinput@IsRegularSemigroup(FullTransformationSemigroup(5));|
  true
\end{Verbatim}
 }

 

\subsection{\textcolor{Chapter }{IsRightZeroSemigroup}}
\logpage{[ 4, 6, 20 ]}\nobreak
\hyperdef{L}{X7CB099958658F979}{}
{\noindent\textcolor{FuncColor}{$\triangleright$\ \ \texttt{IsRightZeroSemigroup({\mdseries\slshape S})\index{IsRightZeroSemigroup@\texttt{IsRightZeroSemigroup}}
\label{IsRightZeroSemigroup}
}\hfill{\scriptsize (property)}}\\
\textbf{\indent Returns:\ }
\texttt{true} or \texttt{false}. 



 \texttt{IsRightZeroSemigroup} returns \texttt{true} if the semigroup of transformations or partial permutations \mbox{\texttt{\mdseries\slshape S}} is a right zero semigroup and \texttt{false} if it is not.

 A semigroup \texttt{S} is a \emph{right zero semigroup} if \texttt{x*y=y} for all \texttt{x,y} in \texttt{S}. An inverse semigroup is a right zero semigroup if and only if it is trivial. 
\begin{Verbatim}[commandchars=!@|,fontsize=\small,frame=single,label=Example]
  !gapprompt@gap>| !gapinput@gens:=[ Transformation( [ 2, 1, 4, 3, 5 ] ), |
  !gapprompt@>| !gapinput@ Transformation( [ 3, 2, 3, 1, 1 ] ) ];;|
  !gapprompt@gap>| !gapinput@S:=Semigroup(gens);;|
  !gapprompt@gap>| !gapinput@IsRightZeroSemigroup(S);|
  false
  !gapprompt@gap>| !gapinput@gens:=[Transformation( [ 1, 2, 3, 3, 1 ] ), |
  !gapprompt@>| !gapinput@ Transformation( [ 1, 2, 4, 4, 1 ] )];;|
  !gapprompt@gap>| !gapinput@S:=Semigroup(gens);;|
  !gapprompt@gap>| !gapinput@IsRightZeroSemigroup(S);|
  true
\end{Verbatim}
 }

 
\subsection{\textcolor{Chapter }{IsXTrivial}}\logpage{[ 4, 6, 21 ]}
\hyperdef{L}{X8752642C7F7E512B}{}
{
\noindent\textcolor{FuncColor}{$\triangleright$\ \ \texttt{IsRTrivial({\mdseries\slshape S})\index{IsRTrivial@\texttt{IsRTrivial}}
\label{IsRTrivial}
}\hfill{\scriptsize (property)}}\\
\noindent\textcolor{FuncColor}{$\triangleright$\ \ \texttt{IsLTrivial({\mdseries\slshape S})\index{IsLTrivial@\texttt{IsLTrivial}}
\label{IsLTrivial}
}\hfill{\scriptsize (property)}}\\
\noindent\textcolor{FuncColor}{$\triangleright$\ \ \texttt{IsHTrivial({\mdseries\slshape S})\index{IsHTrivial@\texttt{IsHTrivial}}
\label{IsHTrivial}
}\hfill{\scriptsize (property)}}\\
\noindent\textcolor{FuncColor}{$\triangleright$\ \ \texttt{IsDTrivial({\mdseries\slshape S})\index{IsDTrivial@\texttt{IsDTrivial}}
\label{IsDTrivial}
}\hfill{\scriptsize (property)}}\\
\noindent\textcolor{FuncColor}{$\triangleright$\ \ \texttt{IsAperiodicSemigroup({\mdseries\slshape S})\index{IsAperiodicSemigroup@\texttt{IsAperiodicSemigroup}}
\label{IsAperiodicSemigroup}
}\hfill{\scriptsize (property)}}\\
\noindent\textcolor{FuncColor}{$\triangleright$\ \ \texttt{IsCombinatorialSemigroup({\mdseries\slshape S})\index{IsCombinatorialSemigroup@\texttt{IsCombinatorialSemigroup}}
\label{IsCombinatorialSemigroup}
}\hfill{\scriptsize (property)}}\\
\textbf{\indent Returns:\ }
\texttt{true} or \texttt{false}. 



 \texttt{IsXTrivial} returns \texttt{true} if Green's $\mathcal{R}$-relation, $\mathcal{L}$-relation, $\mathcal{H}$-relation, $\mathcal{D}$-relation, respectively, on the semigroup of transformations or partial
permutations \mbox{\texttt{\mdseries\slshape S}} is trivial and \texttt{false} if it is not. These properties can also be applied to a Green's class instead
of a semigroup where applicable. 

 For inverse semigroups, the properties of being $\mathcal{R}$-trivial, $\mathcal{L}$-trivial, $\mathcal{D}$-trivial, and a semilattice are equivalent; see \texttt{IsSemilatticeAsSemigroup} (\ref{IsSemilatticeAsSemigroup}). 

 A semigroup is \emph{aperiodic} if its contains no non-trivial subgroups (equivalently, all of its group $\mathcal{H}$-classes are trivial). A finite semigroup is aperiodic if and only if it is $\mathcal{H}$-trivial. 

 \emph{Combinatorial} is a synonym for aperiodic in this context. 
\begin{Verbatim}[commandchars=!@|,fontsize=\small,frame=single,label=Example]
  !gapprompt@gap>| !gapinput@S:=Semigroup( Transformation( [ 1, 5, 1, 3, 7, 10, 6, 2, 7, 10 ] ), |
  !gapprompt@>| !gapinput@ Transformation( [ 4, 4, 5, 6, 7, 7, 7, 4, 3, 10 ] ) );;|
  !gapprompt@gap>| !gapinput@IsHTrivial(S);|
  true
  !gapprompt@gap>| !gapinput@Size(S);|
  108
  !gapprompt@gap>| !gapinput@IsRTrivial(S);|
  false
  !gapprompt@gap>| !gapinput@IsLTrivial(S);|
  false
\end{Verbatim}
 }

 

\subsection{\textcolor{Chapter }{IsSemilatticeAsSemigroup}}
\logpage{[ 4, 6, 22 ]}\nobreak
\hyperdef{L}{X7BF9F1BE87F0636D}{}
{\noindent\textcolor{FuncColor}{$\triangleright$\ \ \texttt{IsSemilatticeAsSemigroup({\mdseries\slshape S})\index{IsSemilatticeAsSemigroup@\texttt{IsSemilatticeAsSemigroup}}
\label{IsSemilatticeAsSemigroup}
}\hfill{\scriptsize (property)}}\\
\textbf{\indent Returns:\ }
\texttt{true} or \texttt{false}. 



 \texttt{IsSemilatticeAsSemigroup} returns \texttt{true} if the semigroup \mbox{\texttt{\mdseries\slshape S}} is a semilattice and \texttt{false} if it is not. 

 A semigroup is a \emph{semilattice} if it is commutative and every element is an idempotent. The idempotents of an
inverse semigroup form a semilattice. 
\begin{Verbatim}[commandchars=!@|,fontsize=\small,frame=single,label=Example]
  !gapprompt@gap>| !gapinput@S:=Semigroup(Transformation( [ 2, 5, 1, 7, 3, 7, 7 ] ), |
  !gapprompt@>| !gapinput@Transformation( [ 3, 6, 5, 7, 2, 1, 7 ] ) );;                    |
  !gapprompt@gap>| !gapinput@Size(S);|
  631
  !gapprompt@gap>| !gapinput@IsInverseSemigroup(S);|
  true
  !gapprompt@gap>| !gapinput@A:=Semigroup(Idempotents(S)); |
  <semigroup with 32 generators>
  !gapprompt@gap>| !gapinput@IsSemilatticeAsSemigroup(A);|
  true
\end{Verbatim}
 }

 
\subsection{\textcolor{Chapter }{IsSimpleSemigroup}}\logpage{[ 4, 6, 23 ]}
\hyperdef{L}{X836F4692839F4874}{}
{
\noindent\textcolor{FuncColor}{$\triangleright$\ \ \texttt{IsSimpleSemigroup({\mdseries\slshape S})\index{IsSimpleSemigroup@\texttt{IsSimpleSemigroup}}
\label{IsSimpleSemigroup}
}\hfill{\scriptsize (property)}}\\
\noindent\textcolor{FuncColor}{$\triangleright$\ \ \texttt{IsCompletelySimpleSemigroup({\mdseries\slshape S})\index{IsCompletelySimpleSemigroup@\texttt{IsCompletelySimpleSemigroup}}
\label{IsCompletelySimpleSemigroup}
}\hfill{\scriptsize (property)}}\\
\textbf{\indent Returns:\ }
\texttt{true} or \texttt{false}. 



 \texttt{IsSimpleSemigroup} returns \texttt{true} if the semigroup of transformations or partial permutations \mbox{\texttt{\mdseries\slshape S}} is simple and \texttt{false} if it is not.

 A semigroup is \emph{simple} if it has no proper 2-sided ideals. A semigroup is \emph{completely simple} if it is simple and possesses minimal left and right ideals. A finite
semigroup is simple if and only if it is completely simple. An inverse
semigroup is simple if and only if it is a group. 
\begin{Verbatim}[commandchars=!@|,fontsize=\small,frame=single,label=Example]
  !gapprompt@gap>| !gapinput@gens:=[ Transformation( [ 2, 2, 4, 4, 6, 6, 8, 8, 10, 10, 12, 12, 2 ] ), |
  !gapprompt@>| !gapinput@ Transformation( [ 1, 1, 3, 3, 5, 5, 7, 7, 9, 9, 11, 11, 3 ] ), |
  !gapprompt@>| !gapinput@ Transformation( [ 1, 7, 3, 9, 5, 11, 7, 1, 9, 3, 11, 5, 5 ] ), |
  !gapprompt@>| !gapinput@ Transformation( [ 7, 7, 9, 9, 11, 11, 1, 1, 3, 3, 5, 5, 7 ] ) ];;|
  !gapprompt@gap>| !gapinput@S:=Semigroup(gens);;|
  !gapprompt@gap>| !gapinput@IsSimpleSemigroup(S);|
  true
  !gapprompt@gap>| !gapinput@IsCompletelySimpleSemigroup(S);|
  true
\end{Verbatim}
 }

 

\subsection{\textcolor{Chapter }{IsSynchronizingSemigroup}}
\logpage{[ 4, 6, 24 ]}\nobreak
\hyperdef{L}{X84A1B84180811785}{}
{\noindent\textcolor{FuncColor}{$\triangleright$\ \ \texttt{IsSynchronizingSemigroup({\mdseries\slshape S})\index{IsSynchronizingSemigroup@\texttt{IsSynchronizingSemigroup}}
\label{IsSynchronizingSemigroup}
}\hfill{\scriptsize (property)}}\\
\textbf{\indent Returns:\ }
\texttt{true} or \texttt{false}. 



 \texttt{IsSynchronizingSemigroup} returns \texttt{true} if the semigroup of transformations \mbox{\texttt{\mdseries\slshape S}} contains a constant transformation. See also \texttt{ConstantTransformation} (\ref{ConstantTransformation}). 
\begin{Verbatim}[commandchars=!@|,fontsize=\small,frame=single,label=Example]
  !gapprompt@gap>| !gapinput@S:=Semigroup( Transformation( [ 1, 1, 8, 7, 6, 6, 4, 1, 8, 9 ] ), |
  !gapprompt@>| !gapinput@ Transformation( [ 5, 8, 7, 6, 10, 8, 7, 6, 9, 7 ] ) );;|
  !gapprompt@gap>| !gapinput@IsSynchronizingSemigroup(S);|
  true
  !gapprompt@gap>| !gapinput@S:=Semigroup( Transformation( [ 3, 8, 1, 1, 9, 9, 8, 7, 9, 6 ] ), |
  !gapprompt@>| !gapinput@ Transformation( [ 7, 6, 8, 7, 5, 6, 8, 7, 8, 9 ] ) );;|
  !gapprompt@gap>| !gapinput@IsSynchronizingSemigroup(S);|
  false
  !gapprompt@gap>| !gapinput@Representative(MinimalIdeal(S));|
  Transformation( [ 7, 7, 8, 7, 7, 7, 8, 7, 8, 7 ] )
\end{Verbatim}
 }

 

\subsection{\textcolor{Chapter }{IsZeroGroup}}
\logpage{[ 4, 6, 25 ]}\nobreak
\hyperdef{L}{X85F7E5CD86F0643B}{}
{\noindent\textcolor{FuncColor}{$\triangleright$\ \ \texttt{IsZeroGroup({\mdseries\slshape S})\index{IsZeroGroup@\texttt{IsZeroGroup}}
\label{IsZeroGroup}
}\hfill{\scriptsize (property)}}\\
\textbf{\indent Returns:\ }
\texttt{true} or \texttt{false}. 



 \texttt{IsZeroGroup} returns \texttt{true} if the semigroup of transformations or partial permutations \mbox{\texttt{\mdseries\slshape S}} is a zero group and \texttt{false} if it is not.

 A semigroup \texttt{S} is a \emph{zero group} if there exists an element \texttt{z} in \texttt{S} such that \texttt{S} without \texttt{z} is a group and \texttt{x*z=z*x=z} for all \texttt{x} in \texttt{S}. Every zero group is an inverse semigroup. 
\begin{Verbatim}[commandchars=!@|,fontsize=\small,frame=single,label=Example]
  !gapprompt@gap>| !gapinput@S:=Semigroup(Transformation( [ 2, 2, 3, 4, 6, 8, 5, 5, 9 ] ),|
  !gapprompt@>| !gapinput@Transformation( [ 3, 3, 8, 2, 5, 6, 4, 4, 9 ] ),|
  !gapprompt@>| !gapinput@ConstantTransformation(9, 9));;|
  !gapprompt@gap>| !gapinput@IsZeroGroup(S);|
  true
  !gapprompt@gap>| !gapinput@T:=Range(IsomorphismPartialPermSemigroup(S));;|
  !gapprompt@gap>| !gapinput@IsZeroGroup(T);|
  true
\end{Verbatim}
 }

 

\subsection{\textcolor{Chapter }{IsZeroRectangularBand}}
\logpage{[ 4, 6, 26 ]}\nobreak
\hyperdef{L}{X7C6787D07B95B450}{}
{\noindent\textcolor{FuncColor}{$\triangleright$\ \ \texttt{IsZeroRectangularBand({\mdseries\slshape S})\index{IsZeroRectangularBand@\texttt{IsZeroRectangularBand}}
\label{IsZeroRectangularBand}
}\hfill{\scriptsize (property)}}\\
\textbf{\indent Returns:\ }
\texttt{true} or \texttt{false}. 



 \texttt{IsZeroRectangularBand} returns \texttt{true} if the semigroup of transformations or partial permutations \mbox{\texttt{\mdseries\slshape S}} is a zero rectangular band and \texttt{false} if it is not.

 A semigroup is a \emph{zero rectangular band} if it is zero simple and $\mathcal{H}$-trivial; see also \texttt{IsZeroSimpleSemigroup} (\ref{IsZeroSimpleSemigroup}) and \texttt{IsHTrivial} (\ref{IsHTrivial}). An inverse semigroup is a zero rectangular band if and only if it is a zero
group; see \texttt{IsZeroGroup} (\ref{IsZeroGroup}). 
\begin{Verbatim}[commandchars=!@|,fontsize=\small,frame=single,label=Example]
  !gapprompt@gap>| !gapinput@S:=Semigroup( |
  !gapprompt@>| !gapinput@ Transformation( [ 1, 3, 7, 9, 1, 12, 13, 1, 15, 9, 1, 18, 1, 1, 13, 1, 1, |
  !gapprompt@>| !gapinput@     21, 1, 1, 1, 1, 1, 25, 26, 1 ] ),|
  !gapprompt@>| !gapinput@Transformation( [ 1, 5, 1, 5, 11, 1, 1, 14, 1, 16, 17, 1, 1, 19, 1, 11, 1,|
  !gapprompt@>| !gapinput@     1, 1, 23, 1, 16, 19, 1, 1, 1 ] ),|
  !gapprompt@>| !gapinput@Transformation( [ 1, 4, 8, 1, 10, 1, 8, 1, 1, 1, 10, 1, 8, 10, 1, 1, 20, 1,|
  !gapprompt@>| !gapinput@     22, 1, 8, 1, 1, 1, 1, 1 ] ),|
  !gapprompt@>| !gapinput@Transformation( [ 1, 6, 6, 1, 1, 1, 6, 1, 1, 1, 1, 1, 6, 1, 6, 1, 1, 6, 1,|
  !gapprompt@>| !gapinput@     1, 24, 1, 1, 1, 1, 6 ] ) );;|
  !gapprompt@gap>| !gapinput@IsZeroRectangularBand(Semigroup(Elements(GreensDClasses(S)[9]))); |
  true
  !gapprompt@gap>| !gapinput@IsZeroRectangularBand(Semigroup(Elements(GreensDClasses(S)[3])));|
  true
  !gapprompt@gap>| !gapinput@IsZeroRectangularBand(Semigroup(Elements(GreensDClasses(S)[1])));|
  false
\end{Verbatim}
 }

 

\subsection{\textcolor{Chapter }{IsZeroSemigroup}}
\logpage{[ 4, 6, 27 ]}\nobreak
\hyperdef{L}{X81A1882181B75CC9}{}
{\noindent\textcolor{FuncColor}{$\triangleright$\ \ \texttt{IsZeroSemigroup({\mdseries\slshape S})\index{IsZeroSemigroup@\texttt{IsZeroSemigroup}}
\label{IsZeroSemigroup}
}\hfill{\scriptsize (property)}}\\
\textbf{\indent Returns:\ }
\texttt{true} or \texttt{false}. 



 \texttt{IsZeroSemigroup} returns \texttt{true} if the semigroup of transformations or partial permutations \mbox{\texttt{\mdseries\slshape S}} is a zero semigroup and \texttt{false} if it is not.

 A semigroup \texttt{S} is a \emph{zero semigroup} if there exists an element \texttt{z} in \texttt{S} such that \texttt{x*y=z} for all \texttt{x,y} in \texttt{S}. An inverse semigroup is a zero semigroup if and only if it is trivial. 
\begin{Verbatim}[commandchars=!@|,fontsize=\small,frame=single,label=Example]
  !gapprompt@gap>| !gapinput@S:=Semigroup( Transformation( [ 4, 7, 6, 3, 1, 5, 3, 6, 5, 9 ] ), |
  !gapprompt@>| !gapinput@Transformation( [ 5, 3, 5, 1, 9, 3, 8, 7, 4, 3 ] ) );;|
  !gapprompt@gap>| !gapinput@IsZeroSemigroup(S);|
  false
  !gapprompt@gap>| !gapinput@S:=Semigroup( Transformation( [ 7, 8, 8, 8, 5, 8, 8, 8 ] ), |
  !gapprompt@>| !gapinput@ Transformation( [ 8, 8, 8, 8, 5, 7, 8, 8 ] ), |
  !gapprompt@>| !gapinput@ Transformation( [ 8, 7, 8, 8, 5, 8, 8, 8 ] ), |
  !gapprompt@>| !gapinput@ Transformation( [ 8, 8, 8, 7, 5, 8, 8, 8 ] ), |
  !gapprompt@>| !gapinput@ Transformation( [ 8, 8, 7, 8, 5, 8, 8, 8 ] ) );;|
  !gapprompt@gap>| !gapinput@IsZeroSemigroup(S);|
  true
  !gapprompt@gap>| !gapinput@MultiplicativeZero(S);|
  Transformation( [ 8, 8, 8, 8, 5, 8, 8, 8 ] )
\end{Verbatim}
 }

 

\subsection{\textcolor{Chapter }{IsZeroSimpleSemigroup}}
\logpage{[ 4, 6, 28 ]}\nobreak
\hyperdef{L}{X8193A60F839C064E}{}
{\noindent\textcolor{FuncColor}{$\triangleright$\ \ \texttt{IsZeroSimpleSemigroup({\mdseries\slshape S})\index{IsZeroSimpleSemigroup@\texttt{IsZeroSimpleSemigroup}}
\label{IsZeroSimpleSemigroup}
}\hfill{\scriptsize (property)}}\\
\textbf{\indent Returns:\ }
\texttt{true} or \texttt{false}. 



 \texttt{IsZeroSimpleSemigroup} returns \texttt{true} if the semigroup of transformations or partial permutations \mbox{\texttt{\mdseries\slshape S}} is a zero simple semigroup and \texttt{false} if it is not.

 A semigroup is a \emph{zero simple semigroup} if it has no two-sided ideals other than itself and the set containing the
zero element; see also \texttt{MultiplicativeZero} (\ref{MultiplicativeZero}). An inverse semigroup is zero simple if and only if it is a Brandt semigroup;
see \texttt{IsBrandtSemigroup} (\ref{IsBrandtSemigroup}). 
\begin{Verbatim}[commandchars=!@|,fontsize=\small,frame=single,label=Example]
  !gapprompt@gap>| !gapinput@S:=Semigroup( |
  !gapprompt@>| !gapinput@ Transformation( [ 1, 17, 17, 17, 17, 17, 17, 17, 17, 17, 5, 17, |
  !gapprompt@>| !gapinput@ 17, 17, 17, 17, 17 ] ), |
  !gapprompt@>| !gapinput@ Transformation( [ 1, 17, 17, 17, 11, 17, 17, 17, 17, 17, 17, 17, |
  !gapprompt@>| !gapinput@ 17, 17, 17, 17, 17 ] ), |
  !gapprompt@>| !gapinput@ Transformation( [ 1, 17, 17, 17, 17, 17, 17, 17, 17, 17, 4, 17, |
  !gapprompt@>| !gapinput@ 17, 17, 17, 17, 17 ] ), |
  !gapprompt@>| !gapinput@ Transformation( [ 1, 17, 17, 5, 17, 17, 17, 17, 17, 17, 17, 17, |
  !gapprompt@>| !gapinput@ 17, 17, 17, 17, 17 ] ));;|
  !gapprompt@gap>| !gapinput@IsZeroSimpleSemigroup(S);|
  true
  !gapprompt@gap>| !gapinput@S:=Semigroup(|
  !gapprompt@>| !gapinput@Transformation( [ 2, 3, 4, 5, 1, 8, 7, 6, 2, 7 ] ),|
  !gapprompt@>| !gapinput@Transformation([ 2, 3, 4, 5, 6, 8, 7, 1, 2, 2 ] ));;|
  !gapprompt@gap>| !gapinput@IsZeroSimpleSemigroup(S);|
  false
\end{Verbatim}
 }

 }

 
\section{\textcolor{Chapter }{Changing the representation of a semigroup}}\logpage{[ 4, 7, 0 ]}
\hyperdef{L}{X82CCC1A781650878}{}
{
 

\subsection{\textcolor{Chapter }{AntiIsomorphismTransformationSemigroup}}
\logpage{[ 4, 7, 1 ]}\nobreak
\hyperdef{L}{X820ECE00846E480F}{}
{\noindent\textcolor{FuncColor}{$\triangleright$\ \ \texttt{AntiIsomorphismTransformationSemigroup({\mdseries\slshape S})\index{AntiIsomorphismTransformationSemigroup@\texttt{Anti}\-\texttt{Isomorphism}\-\texttt{Transformation}\-\texttt{Semigroup}}
\label{AntiIsomorphismTransformationSemigroup}
}\hfill{\scriptsize (operation)}}\\
\textbf{\indent Returns:\ }
An anti-isomorphism. 



 If \mbox{\texttt{\mdseries\slshape S}} is a semigroup, then \texttt{AntiIsomorphismTransformationSemigroup} returns an anti-isomorphism from \mbox{\texttt{\mdseries\slshape S}} to a transformation semigroup. At present, the degree of the resulting
transformation semigroup equals the size of \mbox{\texttt{\mdseries\slshape S}} plus $1$, and, consequently, this function is of limited use. 

 See also \texttt{IsomorphismTransformationSemigroup} (\textbf{Reference: IsomorphismTransformationSemigroup}). 
\begin{Verbatim}[commandchars=!@|,fontsize=\small,frame=single,label=Example]
  !gapprompt@gap>| !gapinput@file:=Concatenation(CitrusDir(), "/examples/selfcomp.citrus.gz");;|
  !gapprompt@gap>| !gapinput@S:=Semigroup(ReadCitrus(file, 34));;|
  !gapprompt@gap>| !gapinput@Size(S);|
  1016
  !gapprompt@gap>| !gapinput@AntiIsomorphismTransformationSemigroup(S); |
  MappingByFunction( <semigroup with 7 generators>, <semigroup with 
  7 generators>, function( a ) ... end )
\end{Verbatim}
 }

 

\subsection{\textcolor{Chapter }{IsomorphismPartialPermMonoid}}
\logpage{[ 4, 7, 2 ]}\nobreak
\hyperdef{L}{X857F68DF7EDA3BE4}{}
{\noindent\textcolor{FuncColor}{$\triangleright$\ \ \texttt{IsomorphismPartialPermMonoid({\mdseries\slshape S})\index{IsomorphismPartialPermMonoid@\texttt{IsomorphismPartialPermMonoid}}
\label{IsomorphismPartialPermMonoid}
}\hfill{\scriptsize (operation)}}\\
\noindent\textcolor{FuncColor}{$\triangleright$\ \ \texttt{IsomorphismPartialPermSemigroup({\mdseries\slshape S})\index{IsomorphismPartialPermSemigroup@\texttt{IsomorphismPartialPermSemigroup}}
\label{IsomorphismPartialPermSemigroup}
}\hfill{\scriptsize (operation)}}\\
\textbf{\indent Returns:\ }
An isomorphism.



 \texttt{IsomorphismPartialPermSemigroup(\mbox{\texttt{\mdseries\slshape S}})} returns an isomorphism from the inverse semigroup or group \mbox{\texttt{\mdseries\slshape S}} to an inverse semigroup of partial permutations.

 \texttt{IsomorphismPartialPermMonoid(\mbox{\texttt{\mdseries\slshape S}})} returns an isomorphism from the inverse monoid or group \mbox{\texttt{\mdseries\slshape S}} to an inverse monoid of partial permutations.

 We only describe \texttt{IsomorphismPartialPermMonoid}, the corresponding statements for \texttt{IsomorphismPartialPermSemigroup} also hold. 
\begin{description}
\item[{Partial permutation semigroups}]  If \mbox{\texttt{\mdseries\slshape S}} is a partial permutation semigroup that does not satisfy \texttt{IsMonoid} (\textbf{Reference: IsMonoid}) but does satisfy \texttt{IsMonoidAsSemigroup} (\ref{IsMonoidAsSemigroup}), then \texttt{IsomorphismPartialPermMonoid(\mbox{\texttt{\mdseries\slshape S}})} returns an isomorphism from \mbox{\texttt{\mdseries\slshape S}} to an inverse monoid of partial permutations; see \texttt{InverseMonoid} (\ref{InverseMonoid}). 
\item[{Permutation groups}]  If \mbox{\texttt{\mdseries\slshape S}} is a permutation group, then \texttt{IsomorphismPartialPermMonoid} returns an isomorphism from \mbox{\texttt{\mdseries\slshape S}} to an inverse monoid of partial permutations on the set \texttt{MovedPoints(\mbox{\texttt{\mdseries\slshape S}})} obtained using \texttt{AsPartialPerm} (\ref{AsPartialPerm}). The inverse of this isomorphism is obtained using \texttt{AsPermutation} (\ref{AsPermutation}); see \texttt{MovedPoints} (\textbf{Reference: MovedPoints (for a permutation)}). 
\item[{Transformation semigroups}]  If \mbox{\texttt{\mdseries\slshape S}} is a transformation semigroup satisfying \texttt{IsInverseMonoid} (\ref{IsInverseMonoid}), then \texttt{IsomorphismPartialPermMonoid} returns an isomorphism from \mbox{\texttt{\mdseries\slshape S}} to an inverse monoid of partial permutations on a subset of \texttt{[1..DegreeOfTransformationSemigroup(\mbox{\texttt{\mdseries\slshape S}})]}. 
\end{description}
 
\begin{Verbatim}[commandchars=!@|,fontsize=\small,frame=single,label=Example]
  !gapprompt@gap>| !gapinput@s:=InverseSemigroup( |
  !gapprompt@>| !gapinput@PartialPermNC( [ 1, 2, 3, 4, 5 ], [ 4, 2, 3, 1, 5 ] ),|
  !gapprompt@>| !gapinput@PartialPermNC( [ 1, 2, 4, 5 ], [ 3, 1, 4, 2 ] ) );;|
  !gapprompt@gap>| !gapinput@IsMonoid(s); |
  false
  !gapprompt@gap>| !gapinput@IsMonoidAsSemigroup(s);|
  true
  !gapprompt@gap>| !gapinput@iso:=IsomorphismPartialPermMonoid(s);|
  MappingByFunction( <inverse semigroup with 
  2 generators>, <inverse monoid with 
  2 generators>, function( x ) ... end, function( x ) ... end )
  !gapprompt@gap>| !gapinput@Size(s);|
  508
  !gapprompt@gap>| !gapinput@Size(Range(iso));|
  508
  !gapprompt@gap>| !gapinput@g:=Group((1,2)(3,8)(4,6)(5,7), (1,3,4,7)(2,5,6,8), (1,4)(2,6)(3,7)(5,8));;|
  !gapprompt@gap>| !gapinput@IsomorphismPartialPermSemigroup(g);|
  MappingByFunction( Group([ (1,2)(3,8)(4,6)(5,7), (1,3,4,7)(2,5,6,8),
    (1,4)(2,6)(3,7)(5,8) ]), <inverse semigroup with
  3 generators>, function( p ) ... end, function( f ) ... end )
  !gapprompt@gap>| !gapinput@s:=Semigroup(Transformation( [ 2, 5, 1, 7, 3, 7, 7 ] ), |
  !gapprompt@>| !gapinput@Transformation( [ 3, 6, 5, 7, 2, 1, 7 ] ) );;|
  !gapprompt@gap>| !gapinput@iso:=IsomorphismPartialPermMonoid(s);|
  MappingByFunction( <inverse monoid with 2 generators>, <inverse monoid with 
  2 generators>, function( f ) ... end, function( x ) ... end )
  !gapprompt@gap>| !gapinput@MovedPoints(Range(iso));|
  [ 1, 2, 3, 5, 6 ]
\end{Verbatim}
 }

 

\subsection{\textcolor{Chapter }{IsomorphismPermGroup}}
\logpage{[ 4, 7, 3 ]}\nobreak
\hyperdef{L}{X80B7B1C783AA1567}{}
{\noindent\textcolor{FuncColor}{$\triangleright$\ \ \texttt{IsomorphismPermGroup({\mdseries\slshape S})\index{IsomorphismPermGroup@\texttt{IsomorphismPermGroup}}
\label{IsomorphismPermGroup}
}\hfill{\scriptsize (operation)}}\\
\textbf{\indent Returns:\ }
An isomorphism. 



 If the semigroup of transformations or partial permutations \mbox{\texttt{\mdseries\slshape S}} satisfies \texttt{IsGroupAsSemigroup} (\ref{IsGroupAsSemigroup}), then \texttt{IsomorphismPermGroup} returns an isomorphism to a permutation group.

 If \mbox{\texttt{\mdseries\slshape S}} does not satisfy \texttt{IsGroupAsSemigroup} (\ref{IsGroupAsSemigroup}), then an error is given. 
\begin{Verbatim}[commandchars=!@|,fontsize=\small,frame=single,label=Example]
  !gapprompt@gap>| !gapinput@S:=Semigroup( Transformation( [ 2, 2, 3, 4, 6, 8, 5, 5 ] ),|
  !gapprompt@>| !gapinput@Transformation( [ 3, 3, 8, 2, 5, 6, 4, 4 ] ) );;|
  !gapprompt@gap>| !gapinput@IsGroupAsSemigroup(S);|
  true
  !gapprompt@gap>| !gapinput@IsomorphismPermGroup(S); |
  MappingByFunction( <semigroup with 2 generators>, Group([ (5,6,8), (2,3,8,4) 
   ]), <Operation "AsPermutation">, function( x ) ... end )
  !gapprompt@gap>| !gapinput@StructureDescription(Range(IsomorphismPermGroup(S)));|
  "S6"
  !gapprompt@gap>| !gapinput@s:=Range(IsomorphismPartialPermSemigroup(SymmetricGroup(4)));|
  <inverse semigroup with 2 generators>
  !gapprompt@gap>| !gapinput@IsomorphismPermGroup(s);|
  MappingByFunction( <inverse semigroup with 2 generators>, Group(
  [ (1,2,3,4), (1,2) ]), <Operation "AsPermutation">, function( x ) ... end )
\end{Verbatim}
 }

 

\subsection{\textcolor{Chapter }{IsomorphismTransformationMonoid}}
\logpage{[ 4, 7, 4 ]}\nobreak
\hyperdef{L}{X84AF7B907E82F2D1}{}
{\noindent\textcolor{FuncColor}{$\triangleright$\ \ \texttt{IsomorphismTransformationMonoid({\mdseries\slshape S})\index{IsomorphismTransformationMonoid@\texttt{IsomorphismTransformationMonoid}}
\label{IsomorphismTransformationMonoid}
}\hfill{\scriptsize (operation)}}\\
\noindent\textcolor{FuncColor}{$\triangleright$\ \ \texttt{IsomorphismTransformationSemigroup({\mdseries\slshape S})\index{IsomorphismTransformationSemigroup@\texttt{IsomorphismTransformationSemigroup}}
\label{IsomorphismTransformationSemigroup}
}\hfill{\scriptsize (operation)}}\\
\textbf{\indent Returns:\ }
An isomorphism. 



 \texttt{IsomorphismTransformationSemigroup(\mbox{\texttt{\mdseries\slshape S}})} returns an isomorphism from the semigroup \mbox{\texttt{\mdseries\slshape S}} to a semigroup of transformations.

 \texttt{IsomorphismTransformationMonoid(\mbox{\texttt{\mdseries\slshape S}})} returns an isomorphism from the monoid \mbox{\texttt{\mdseries\slshape S}} to a monoid of transformations. 

 We only describe \texttt{IsomorphismTransformationMonoid}, the corresponding statements for \texttt{IsomorphismTransformationSemigroup} also hold. 
\begin{description}
\item[{Partial permutation semigroups}]  If \mbox{\texttt{\mdseries\slshape S}} is a partial permutation monoid, then \texttt{IsomorphismTransformationMonoid(\mbox{\texttt{\mdseries\slshape S}})} returns an isomorphism from \mbox{\texttt{\mdseries\slshape S}} to a monoid of partial permutations on \texttt{[1..LargestMovedPoint(\mbox{\texttt{\mdseries\slshape S}})+1]} obtained using \texttt{AsTransformation} (\ref{AsTransformation}). The inverse of this isomorphism is obtained using \texttt{AsPartialPerm} (\ref{AsPartialPerm}); see \texttt{LargestMovedPoint} (\ref{LargestMovedPoint:for a partial perm}), \texttt{InverseMonoid} (\ref{InverseMonoid}) and \texttt{Monoid} (\textbf{Reference: Monoid}). 
\item[{Permutation groups}]  If \mbox{\texttt{\mdseries\slshape S}} is a permutation group, then \texttt{IsomorphismTransformationMonoid} returns an isomorphism from \mbox{\texttt{\mdseries\slshape S}} to a transformation monoid acting on the set \texttt{[1..NrMovedPoints(\mbox{\texttt{\mdseries\slshape S}})]} obtained using \texttt{AsTransformation} (\ref{AsTransformation}); see \texttt{NrMovedPoints} (\textbf{Reference: NrMovedPoints (for a permutation)}). 
\item[{Transformation semigroups}]  If \mbox{\texttt{\mdseries\slshape obj}} is a transformation semigroup satisfying \texttt{IsMonoidAsSemigroup} (\ref{IsMonoidAsSemigroup}), then this function returns an isomorphism from \mbox{\texttt{\mdseries\slshape obj}} to a transformation monoid. 

 
\end{description}
 
\begin{Verbatim}[commandchars=!@|,fontsize=\small,frame=single,label=Example]
  !gapprompt@gap>| !gapinput@S:=Semigroup( Transformation( [ 1, 4, 6, 2, 5, 3, 7, 8, 9, 9 ] ),|
  !gapprompt@>| !gapinput@Transformation( [ 6, 3, 2, 7, 5, 1, 8, 8, 9, 9 ] ) );;|
  !gapprompt@gap>| !gapinput@IsTransformationMonoid(S);|
  false
  !gapprompt@gap>| !gapinput@IsMonoidAsSemigroup(S);|
  true
  !gapprompt@gap>| !gapinput@M:=Range(IsomorphismTransformationMonoid(S));|
  <monoid with 2 generators>
  !gapprompt@gap>| !gapinput@IsTransformationMonoid(M);|
  true
  !gapprompt@gap>| !gapinput@s:=InverseMonoid(|
  !gapprompt@>| !gapinput@ PartialPermNC( [ 1, 2, 3 ], [ 4, 2, 3 ] ),|
  !gapprompt@>| !gapinput@ PartialPermNC( [ 1, 2, 4 ], [ 1, 3, 2 ] ),|
  !gapprompt@>| !gapinput@ PartialPermNC( [ 1, 2, 4 ], [ 4, 1, 2 ] ) );;|
  !gapprompt@gap>| !gapinput@t:=Range(IsomorphismTransformationMonoid(s));|
  <monoid with 5 generators>
  !gapprompt@gap>| !gapinput@Size(s); Size(t);|
  117
  117
\end{Verbatim}
 }

 }

 }

 \def\bibname{References\logpage{[ "Bib", 0, 0 ]}
\hyperdef{L}{X7A6F98FD85F02BFE}{}
}

\bibliographystyle{alpha}
\bibliography{citrus}

\addcontentsline{toc}{chapter}{References}

\def\indexname{Index\logpage{[ "Ind", 0, 0 ]}
\hyperdef{L}{X83A0356F839C696F}{}
}

\cleardoublepage
\phantomsection
\addcontentsline{toc}{chapter}{Index}


\printindex

\newpage
\immediate\write\pagenrlog{["End"], \arabic{page}];}
\immediate\closeout\pagenrlog
\end{document}
