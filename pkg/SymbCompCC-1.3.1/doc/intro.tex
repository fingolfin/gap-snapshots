%%%%%%%%%%%%%%%%%%%%%%%%%%%%%%%%%%%%%%%%%%%%%%%%%%%%%%%%%%%%%%%%%%%%%%%%%
%%
%W  intro.tex              GAP documentation      Dörte Feichtenschlager
%%

%%%%%%%%%%%%%%%%%%%%%%%%%%%%%%%%%%%%%%%%%%%%%%%%%%%%%%%%%%%%%%%%%%%%%%%%%
\Chapter{Introduction}

%%%%%%%%%%%%%%%%%%%%%%%%%%%%%%%%%%%%%%%%%%%%%%%%%%%%%%%%%%%%%%%%%%%%%%%%%
\Section{Overview}

The coclass of a finite $p$-group of order $p^n$ and nilpotency class $c$ is 
defined as $n-c$. This invariant of finite $p$-groups has been introduced by
Leedham-Green and Newman in \cite{LGN80} and it became of major importance
in $p$-group theory. 

A first tool in the classification of all $p$-groups of coclass $r$ is the
coclass graph $G(p,r)$. Its vertices are the isomorphism types of finite 
$p$-groups of coclass $r$. Two vertices $G$ and $H$ are joined by an edge if
$G$ is isomorphic to the quotient $H/\gamma(H)$ where $\gamma(H)$ is the last
non-trivial term of the lower series of $H$. 

Du Sautoy \cite{dS00} and Eick and Leedham-Green \cite{ELG08} proved that
$G(p,r)$ contains certain periodic patterns. Eick and Leedham-Green 
\cite{ELG08} define infinite coclass sequences of finite $p$-groups of 
coclass $r$ which underpin this periodic pattern. In $G(2,r)$ and $G(3,1)$
almost all groups are contained in an infinite coclass sequence. 

Eick and Leedham-Green \cite{ELG08} also proved that the infinitely many
$p$-groups in an infinite coclass sequence can be defined by a single 
parametrised presentation.

The first aim of this package is the definition of polycyclic parametrised
presentations; these are parametrised presentations as defined by Eick
and Leedham-Green \cite{ELG08} and additionally they have various features
of polycyclic presentations. Each such presentation defines all the 
infinitely many finite $p$-groups in an infinite coclass sequence.

We then provide some algorithms to compute with polycyclic parametrised
presentations. In particular, we introduce a generalisation of the
collection algorithm for polycyclic parametrised presentations. Based
on this, we describe algorithms to compute polycyclic parametrised
presentations for Schur extensions, for the Schur multiplicator and
for some low-dimensional cohomology groups. We refer to \cite{EF11}
for details on the underlying algorithms and further references.

Finally, we exhibit a database of polycyclic parametrised presentations
for the infinite coclass families of the finite $2$-groups of coclass at
most $2$ and the finite $3$-groups of coclass $1$.

%%%%%%%%%%%%%%%%%%%%%%%%%%%%%%%%%%%%%%%%%%%%%%%%%%%%%%%%%%%%%%%%%%%%%%%%%
\Section{Background on (polycyclic) parametrised presentations}

In this section we describe the polycyclic parametrised presentations 
(pp-presentations) for infinite coclass sequences.

Let $(G_x | x\in \N)$, where $\N$ denotes the natural numbers, be an 
infinite coclass sequence; $x$ is the parameter of this infinite coclass 
sequence. Then every group $G_x$ is an extension of a finite $p$-group $P$ 
of order $p^n$ by an abelian $p$-group $T_x$ of rank $d$. Furthermore, every  
$G_x$ has a polycyclic presentation (short pp-presentation) on generators 
$g_1, \ldots, g_n, t_1, \ldots, t_d$ with relations of the form
%display{nontext}
$$
\eqalign{
&g_i^{p} = g_{i+1}^{a_{i,i,i+1}} \cdots g_n^{a_{i,i,n}}t_1^{\alpha_{i,i,1}(x)} \cdots t_d^{\alpha_{i,i,d}(x)}, \cr
&g_i^{g_j} = g_{j+1}^{a_{i,j,j+1}} \cdots g_n^{a_{i,j,n}}t_1^{\alpha_{i,j,1}(x)} \cdots t_d^{\alpha_{i,j,d}(x)}, \cr
&t_k^{g_i} = t_1^{b_{k,i,1}(x)} \cdots t_d^{b_{k,i,d}(x)}, \cr
&t_k^{t_l} = t_k, \cr
&t_k^{p^{x+e}} = 1,
}
$$
%display{text}
%g_i^p = g_{i+1}^{a_{i,i,i+1}} ... g_n^{a_{i,i,n}}t_1^{\alpha_{i,i,1}(x)}... t_d^{\alpha_{i,i,d}(x)}, 
%g_i^{g_j} = g_{j+1}^{a_{i,j,j+1}} ... g_n^{a_{i,j,n}}t_1^{\alpha_{i,j,1}(x)}... t_d^{\alpha_{i,j,d}(x)}, 
%t_k^{g_i} = t_1^{b_{k,i,1}(x)}... t_d^{b_{k,i,d}(x)}, 
%t_k^{t_l} = t_k,
%t_k^{p^{x+e}} = 1,
%enddisplay
where $1\le j \< i\le n$ and $1 \le k \< l\le d$; certain $a_{i,j,m}\in \{0,
\ldots, p-1\}$, a non-negative integer $e$, $\alpha_{k,l,m}(x)$ of the form 
$c_{k,l,m}+p^xd_{k,l,m}$ and $b_{k,l,m}$ with $b_{k,l,m},c_{k,l,m},d_{k,l,m}$ 
certain $p$-adic integers. The $p$-adic exponents arising in the relations
can be reduced modulo the relative orders of the involved elements and thus 
can be reduced to integers for every specific $x$. 

We call such a pp-presentation $integral$ if
all the $p$-adic numbers $b_{k,l,m}, c_{k,l,m}, d_{k,l,m}$ are integers.
Our algorithms introduced in this package compute with integral 
pp-presentations only.

We call such an pp-presentation $consistent$ if for every $x \in \N$
the presentation is consistent as a polycyclic presentation; where we 
possibly reduce the exponents in the presentation modulo the relative
orders of the generators. 

%%%%%%%%%%%%%%%%%%%%%%%%%%%%%%%%%%%%%%%%%%%%%%%%%%%%%%%%%%%%%%%%%%%%%%%%%
\Section{Computation of Schur multiplicators}

In this section we recall briefly the method of \cite{EF11} to determine
the Schur multiplicators of almost all groups $G_x$ in an infinite coclass 
sequence.

Suppose we are given a consistent integral pp-presentation $F/R_x$ for the
groups $G_x$ in an infinite coclass sequence, where $F$ is a free group and
$R_x$ is generated by parametrised relations as above. Note that the 
exponents in these relations depend on $x$, while the number of generators
and the number of relations does not depend on the parameter.

Using this presentation we can define a parametrised presentation for the
Schur extensions $G_x^{*} = F/[F,R_x]$, corresponding to the parametrised
presentation $F/R_x$. The next step is to find the isomorphism types of
$Y_x = R_x/[F,R_x]$ since $M(G_x) \cong (F^\prime \cap R_x)/[F,R_x]$ are the
torsion subgroups of $Y_x$ as all $G_x$ are finite <p>-groups.

Then $Y_x = R_x/[F,R_x]$ are generated by certain so-called
consistency relations. Using this we can compute the isomorphism types of
$Y_x$ and thus the isomorphism types of $M(G_x)$ for almost all $G_x$ in the
chosen infinite coclass sequence.

%%%%%%%%%%%%%%%%%%%%%%%%%%%%%%%%%%%%%%%%%%%%%%%%%%%%%%%%%%%%%%%%%%%%%%%%%
\Section{Computation of low-dimensional cohomology}

From the parametrised presentation $F/R_x$ we can see that the Abelian
invariants are the same for all groups $G_x$ in an infinite coclass sequence,
and we can compute them. Using this and the computation of the Schur
multiplicators one obtains $H^n(G_x,\Z)$ and $H^n(G_x,GF(p))$ for $0 \le n
\le 2$, where the $G_x$ act trivially on $\Z$ and $GF(p)$, respectively.

%%%%%%%%%%%%%%%%%%%%%%%%%%%%%%%%%%%%%%%%%%%%%%%%%%%%%%%%%%%%%%%%%%%%%%%%%
\Section{Example}

In this section we present the well-known example of quaternion groups 
$Q_{2^{x+3}}$. It is well known that they have a parametrised presentation of 
the following form:

%display{nontext}
$$
\eqalign{
\{ g_1,g_2,t_1 | &\, g_1^{2} = t_1^{2^x}, \, g_2^{g_1} = g_2t_1^{-1+2^{x+1}},\cr
&\, g_2^{2} = t_1, \, t_1^{g_1} = t_1^{-1+2^{x+1}},\cr
&\, t_1^{2^{x+1}} = 1 \}.
}
$$
%display{text}
% { g_1,g_2,t_1|g_1^2 = t_1^{2^x}, g_2^{g_1} = g_2t_1^{-1+2^{x+1}},
%                       g_2^{2} = t_1, t_1^{g_1} = t_1^{-1+2^{x+1}}, 
%                       t_1^{2^{x+1}} = 1 }.
%enddisplay

Using this we can define the Schur extensions $Q_{2^{x+3}}^{*}$

%display{nontext}
$$
\eqalign{
\{ g_1,g_2,t_1,c_1, c_2, c_3 | &g_1^{2} = t_1^{2^x}c_3,
 \, g_2^{g_1} = g_2t_1^{-1+2^{x+1}}c_2^{1-2^{x+1}},\cr
 &\, g_2^{2} = t_1c_1,\, t_1^{g_1} = t_1^{-1+2^{x+1}}c_2^{2-2^{x+1}},\cr
 &\, t_1^{2^{x+1}} = c_2^{2^{x+1}}, \cr
 &\, c_1,c_2,c_3 central \}.
}
$$
%display{text}
% { g_1,g_2,t_1,c_1,c_2,c_3|g_1^{2} = t_1^{2^x}c_3,
%                           g_2^{g_1} = g_2t_1^{-1+2^{x+1}} c_2^{1-2^{x+1}},
%                           g_2^{2} = t_1c_1, 
%                           t_1^{g_1} = t_1^{-1+2^{x+1}} c_2^{2-2^{x+1}},
%                           t_1^{2^{x+1}} = c_2^{2^{x+1}},
%                           c_1,c_2,c_3 central }.
%enddisplay

This yields $M(Q_{2^{x+3}}) = 1$. 
