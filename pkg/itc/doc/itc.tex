%%%%%%%%%%%%%%%%%%%%%%%%%%%%%%%%%%%%%%%%%%%%%%%%%%%%%%%%%%%%%%%%%%%%%%%%%
%%
%W  itc.tex                GAP documentation            Joachim Neub"user
%%
%H  $Id: itc.tex,v 1.16 2002/07/16 07:22:17 gap Exp $
%%
%Y  Copyright (C) 1999, Lehrstuhl D fuer Mathematik, RWTH Aachen, Germany
%%
\def\ITC{{\sf ITC}}
\def\XGAP{{\sf XGAP}}


%%%%%%%%%%%%%%%%%%%%%%%%%%%%%%%%%%%%%%%%%%%%%%%%%%%%%%%%%%%%%%%%%%%%%%%%%
\Chapter{What is ITC?}

{\ITC} stands for *Interactive Todd Coxeter*, it is a program that
allows you to execute interactively single steps in an enumeration of
the cosets of a subgroup of a finitely presented group using the
graphics surface {\XGAP} of {\GAP} and thus to see in various windows
exactly what is happening. In this chapter we will

\beginlist

\item{--}
  discuss the background situation that led to the concept of this
  program,

\item{--}
  very briefly point to literature about coset enumeration,

\item{--}
  talk about envisaged use of {\ITC}, and

\item{--}
  explain some technical points.

\endlist


%%%%%%%%%%%%%%%%%%%%%%%%%%%%%%%%%%%%%%%%%%%%%%%%%%%%%%%%%%%%%%%%%%%%%%%%%
\Section{Background}

Coset Enumeration, in the sequel abbreviated as CE, also known as the
Todd-Coxeter procedure, is not only the oldest but also one of the most
often used tools in Computational Group Theory and also probably the
most often implemented one.

An easy introduction to CE and some of its variants is given in
\cite{Neu82}. In his book \cite{Sims94} Charles Sims puts CE into a
much more general context based on the theory of automata. Both texts
provide a fairly comprehensive guide to the extensive literature on
CE. So we will assume in the sequel that the user of this manual
understands the basic ideas of CE. We will use as far as possible the
terminology of \cite{Neu82}.

The main {\GAP} library does contain routines for executing coset
enumerations (see Section "ref:Coset Tables and Coset Enumeration" in
the {\GAP} reference manual). These routines are partly written in the
{\GAP} language but for efficiency also use {\GAP} kernel functions
written in C. Some of these programs are used by {\ITC}.

The most efficient implementation of CE, allowing many ``strategies'',
at present is the package ACE by George Havas and Colin Ramsey
(see~\cite{Ram99}) which is also available as a {\GAP4} package.

It should be understood that {\ITC} is no competitor for ACE. ACE can
successfully deal with coset enumerations with millions of cosets. On
the other hand handling coset tables graphically on the screen, even
with using scrolling, in practice limits the use of {\ITC} to
tables with a few thousand rows. For that reason also the speed of
the single step in the enumeration does not matter as much and {\ITC}
therefore is not tuned for speed in the way ACE is.


%%%%%%%%%%%%%%%%%%%%%%%%%%%%%%%%%%%%%%%%%%%%%%%%%%%%%%%%%%%%%%%%%%%%%%%%%
\Section{Teaching CE}

In texts on CE usually examples are demonstrated, and whenever one
teaches the method in class, one will have to work through some
examples. In printing as well as on the blackboard this suffers from
a certain difficulty: In CE a number of tables, the ``coset table'', the
``subgroup tables'', and the ``relation tables'' are being changed all the
time, one has to gain information from (the closing of rows of)
subgroup and relation tables and has to transfer this information into
all relevant places of all tables. Sometimes, when so-called
``coincidences'' occur, one even has to erase and modify already
existing entries in tables. Describing such steps in some detail as
well as the ever changing tables soon occupy many pages in printing.
Doing them in front of a class one can pretty soon get confused,
forgetting to make entries or changes etc.

So it is a natural demand to have a program that allows one to perform
such steps in windows on the screen of a computer where such changes
can be cleanly made, are transferred to all relevant places in all
tables automatically, and are marked by colours so that one can really
``play'' through nontrivial examples. This is what {\ITC} provides. So
in a first place it can be considered as a teaching tool.


%%%%%%%%%%%%%%%%%%%%%%%%%%%%%%%%%%%%%%%%%%%%%%%%%%%%%%%%%%%%%%%%%%%%%%%%%
\Section{Experimenting with CE Strategies}

However we hope that {\ITC} can also be used for experiments that
ultimately may lead to a better understanding of existing methods or
even the design of new ones.

As explained in many papers about CE, the sequence in which steps are
made has great influence on the efficiency of the methods. To be more
exact: CE is basically a method of trial and error in which one
defines cosets of a subgroup of a finitely presented group by
constructing coset representatives as words in the generators of the
group. But only the termination
of the method (the closing of the tables) in the end guarantees that
one has not made a mistake by defining more than one word representing
the same coset. The discovery of such a mistake and its removal is
what is called ``handling a coincidence'' and it is the number of
occurrences of such mistakes (the number of coincidences encountered)
that may vary immensely depending on the chosen sequence of definition
of pretended representatives.

Therefore many different ``strategies'' for CE have been discussed in
the literature, and programs such as ACE allow one to use a wide
variety of such strategies. However in spite of very detailed case
studies, in particular by George Havas, that lead to certain ``rules of
thumb'', no strategy is uniformly the best one, a strategy that is good
for one presentation may turn out to be much less good for another one
(see for instance \cite{CDHW73}, \cite{Hav91}, \cite{HR99a},
and \cite{HR99b}). The
possibility to try one's own strategy stepwise interactively therefore
can be used to get some more insight into CE which to some extent is
an art rather than a science.

Already in the building of {\ITC} a nice side product was obtained
from such experimentation: It is important to understand that in many
cases coincidences cannot be completely avoided for a CE to terminate.
However when many coincidences have occurred, one is led to ask if
there is not a shorter sequence of definitions leading to the closure
of the tables. Starting from a given CE with coincidences involved,
one may try to ``prune'' the definition sequence from definitions that
in retrospect can be recognized as being redundant. Such a procedure
has for instance been employed many years ago ``by hand'' by John Leech
in a search for a short definition sequence for the Fibonacci Group
$F(2,7)$ that we will take as one of our examples. A very nice
discussion of this idea and its history is given in Margaret Edeson's
Master's Thesis (see~\cite{Ede89}). Using {\ITC} such ``pruning'' was
first done interactively. This led to an algorithm that is described
in the section ``The Short-cut Method'' (see Section~"The Short-cut
Method"). It is now part of {\ITC} and can be called by `short-cut'
(see~"short-cut"). In fact with the above mentioned old challenge
problem of $F(2,7)$, it beats the formerly best known definition
sequence leading to closure of the tables.

Since {\ITC} will store sequences of definitions we intend in the
future to investigate the possibility to improve runs of a ``Modified
TC'' (MTC) by using such ``pruned'' definition sequences.


%%%%%%%%%%%%%%%%%%%%%%%%%%%%%%%%%%%%%%%%%%%%%%%%%%%%%%%%%%%%%%%%%%%%%%%%%
\Section{How does it work?}

{\ITC} is written using the graphics facilities provided by the
{\GAP4} package {\XGAP}. We refer to the {\XGAP} manual for a
general description of these.

The {\ITC} Coset Table as well as the Relation Tables and Subgroup
Tables can be depicted in windows, and definitions be made in these by
mouse click. Moreover tables with the sequence of definitions, of
pending coincidences, and of gaps of length 1 in Subgroup or
Relation Tables can be inspected in further windows and be used for
deciding about the next step. A number of menus allow certain
actions to be called.

In addition to the Coset Table Window which plays a special role,
three different kinds of windows are used in {\ITC}:

\beginitems

*Pull-down menus* &
  are used with the ``top buttons'' of the Coset Table (see~"the Top
  Buttons"). Clicking one of these will pop up
  a menu from which you can select actions of {\ITC}. You do this by
  holding the left mouse button while moving the pointer to the
  respective entry in the menu and releasing the mouse button there.
  Then the respective action of the program will be executed while
  the menu vanishes.

*Query windows* &
  (sometimes also called transient windows) are used when the user
  has to specify additonal (in most cases numerical) data for an
  action to be performed. E.g. a user wanting to fill rows has to
  specify in such a query window which rows shall be filled. Note
  that no further action can be taken while such a query window is
  open. All other possibilities to operate {\ITC} are disabled
  until the user has made the needed decision and closed the window.
  Such query windows are opened by clicking the following menu
  entries in the menus opened via the top buttons of the Coset Table
  Window: In the menu of `Settings': `change default table size'
  (see~"change default table size"), `extend table size' (see~"extend
  table size"); in the menu of `File': `read definitions from file'
  (see~"read definitions from file"), `write definitions to file'
  (see~"write definitions to file"), and `write standardized table to
  file' (see~"write standardized table to file"). Also such query
  windows are opened by the following ``bottom buttons'' of the Coset
  Table Window: `scroll by' (see~"scroll by"), `scroll to'
  (see~"scroll to"), `Felsch' (see~"Felsch"), `fill gaps' (see~"fill
  gaps"), `fill rows' (see~"fill rows"), `back to' (see~"back to"),
  `HLT' (see~"HLT"), `short-cut' (see~"short-cut"), and `mark cosets'
  (see~"mark cosets").

All other windows &
  are ordinary *display windows* that can stay open while operations
  are caused by using other buttons or windows. Each of these display
  windows contains a top button `Sheet' that via a pull-down menu
  allows one either to write the contents of the display window to a
  PostScript file or to close this display window. See the paragraph
  *Button* `Sheet' in Section "The Top Buttons".
  A display window is opened by clicking the menu entry
  `show current settings' (see~"show current settings") in the menu
  opened via the top button `Settings'. Further display windows are
  opened by clicking the following bottom buttons in the Coset Table
  Window: `show rels' (see~"show rels"), `show subgrp' (see~"show
  subgrp"), `show defs' (see~"show defs"), `show gaps' (see~"show
  gaps"), and `show coincs' (see~"show coincs").
  Further display windows can be opened by clicking with the right
  mouse button: coset numbers or dots in the Coset Table (see~"The
  Coset Table"), entries in the Table of Definitions (see~"The Table
  of Definitions"), entries in the List of Length 1 Gaps (see~"The
  List of Length 1 Gaps"), and entries in the List of Pending
  Coincidences (see~"The List of Pending Coincidences"). Also the
  Subgroup Tables (see~"The Subgroup Tables") and the Relation Tables
  (see~"The Relation Tables") are display windows that can be opened
  by clicking the respective entries in the List of Subgroup
  Generators (see~"The List of Subgroup Generators") or the List of
  Relators (see~"The List of Relators"), respectively. The windows
  of Relation Tables and of Subgroup Tables differ from the other
  display windows in a technical way: Like the Coset Table Window,
  they stay open while the contents of the displayed tables changes.
  Moreover, the Relation Tables are scrolled parallel to the Coset
  Table (see~"The Relation Tables").

\enditems

Depending on the window system and window manager you use, placing a
new window on your screen might be done automatically or might require
you to use the mouse to choose a position for the window and pressing
the left mouse button to place the window.

Making a window active is system and window manager dependent, in most
cases you have either to move the pointer inside the window or you
have to click the title bar of the window.

We recommend to switch on title bars for all windows in your window
manager setup. This seems to avoid a small bug in at least one window
manager we know of and makes it easier to move your windows
around. For the `twm' window manager for example you can achieve this
with the option `DecorateTransients'.

If you do not want to do this and experience problems with small
dialog boxes that do not accept text entered via the keyboard, try to
start up XGAP with the command line option `-W'. This invokes some
code within XGAP which works around a bug in some window managers.

Please note that because of some 16 bit limitations in the underlying
X toolkit the present implementation of {\XGAP} may not be able to
handle very large virtual windows properly. For instance, sheets
containing more than 1630 coset definitions may be corrupt.

In the following chapters we will describe the possibilities to work
with windows in a systematic way and by way of examples of the use of
{\ITC}.


%%%%%%%%%%%%%%%%%%%%%%%%%%%%%%%%%%%%%%%%%%%%%%%%%%%%%%%%%%%%%%%%%%%%%%%%%
\Section{Acknowledgements}

We thank Max Neunh{\accent127 o}ffer for providing the needed
environment in his package {\XGAP} and Thomas Breuer for frequent
advice. A first version of {\ITC} was implemented, still under
{\GAP}~3, by Ludger Hippe as part of his ``Staatsexamensarbeit'' in
1997.


%%%%%%%%%%%%%%%%%%%%%%%%%%%%%%%%%%%%%%%%%%%%%%%%%%%%%%%%%%%%%%%%%%%%%%%%%
%%
%E

