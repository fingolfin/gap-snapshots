%%%%%%%%%%%%%%%%%%%%%%%%%%%%%%%%%%%%%%%%%%%%%%%%%%%%%%%%%%%%%%%%%%%%%%%%%
%%
%W  intro.tex             GrpConst documentation             Bettina Eick
%%

%%%%%%%%%%%%%%%%%%%%%%%%%%%%%%%%%%%%%%%%%%%%%%%%%%%%%%%%%%%%%%%%%%%%%%%%%
\Chapter{Introduction to GrpConst}

\index{GrpConst}

This package contains three methods to construct the groups of a 
certain type up to isomorphism. 


*The Frattini Extension Method*:

This method can be used to determine all soluble groups of a given 
order. The practicability of the method depends clearly on the 
chosen order. Furthermore, the efficiency of the method might be
increased by restricting the construction to groups with certain 
properties. This is easily possible for a number of properties;
for example, it is useful and straightforward to compute non-nilpotent
groups only. 
(See Chapter "The Frattini Extension Method".)


*The Cyclic Split Extension Method*:

The cyclic split extension method can be used to list all groups of 
order $p^n \cdot q$ for different primes $p$ and $q$ which have a 
normal Sylow subgroup. These groups are also soluble, and hence might
also be obtained by the Frattini extension method. However, the cyclic 
split extension method is more effective on this case. Note that this
method relies on a list of groups of order $p^n$. Moreover, the 
efficiency of this method depends on an effective method to compute 
automorphism groups of $p$-groups. (See Chapter "The Cyclic Split Extension 
Method".)


*The Upwards Extension Method*:

The upwards extension method is the most general of the three methods.
It can be used to construct all groups of a given order.  However, it 
is the least efficient of the three methods and hence should only be
used to construct non-soluble groups. Note that this method needs
a list of perfect groups of order dividing the given one. 
(See Chapter "The Upwards Extension Method".)


Furthermore, the package contains a wrap up function which combines
the three methods to a general algorithm to construct the groups of
a given order. (See Chapter "Construction of All Groups".) Finally,
there is an info class `InfoGrpCon' available for the functions of
this share package with possible levels 1 up to 4.





