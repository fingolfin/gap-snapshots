\newcommand{\QPAIntroPartNumber}{4}
\documentclass[usenames,dvipsnames]{beamer}

%\usetheme{ntnu}
\usetheme{Warsaw}

\usepackage{times}
\usepackage[T1]{fontenc}
\usepackage[all]{xy}
\usepackage{textpos}
\usepackage{tikz}

\newcommand{\defn}[1]{\textit{#1}}
\newcommand{\Q}{\mathbb{Q}}
\newcommand{\equivalence}{\simeq}
\newcommand{\VV}[2]{\begin{pmatrix} #1 & #2 \end{pmatrix}}
\newcommand{\vv}[2]{\left( \begin{smallmatrix} #1 & #2 \end{smallmatrix} \right)}
\newcommand{\into}{\hookrightarrow}
\newcommand{\iso}{\cong}
\newcommand{\dsum}{\oplus}
\DeclareMathOperator{\fmod}{mod}
\DeclareMathOperator{\Rep}{Rep}
\DeclareMathOperator{\Hom}{Hom}
\DeclareMathOperator{\coker}{coker}
\DeclareMathOperator{\im}{im}
\DeclareMathOperator{\rad}{rad}
\DeclareMathOperator{\soc}{soc}
\DeclareMathOperator{\Top}{top}

\title[Introduction to QPA, part \QPAIntroPartNumber]
      {Introduction to QPA}
\subtitle{Part \QPAIntroPartNumber}

\author{\O{}ystein~Skarts\ae{}terhagen \and \O{}yvind~Solberg}
\institute{
Department of Mathematical Sciences\\
Norwegian University of Science and Technology}

\date{Third GAP Days}

% If you have a file called "university-logo-filename.xxx", where xxx
% is a graphic format that can be processed by latex or pdflatex,
% resp., then you can add a logo as follows:

% \pgfdeclareimage[height=0.5cm]{university-logo}{university-logo-filename}
% \logo{\pgfuseimage{university-logo}}



% Delete this, if you do not want the table of contents to pop up at
% the beginning of each subsection:
% \AtBeginSubsection[]
% {
%   \begin{frame}<beamer>{Outline}
%     \tableofcontents[currentsection,currentsubsection]
%   \end{frame}
% }

\renewcommand{\mod}{\operatorname{mod}\nolimits}
\newcommand{\umod}{\operatorname{\underline{mod}}\nolimits}
\newcommand{\omod}{\operatorname{\overline{mod}}\nolimits}
\newcommand{\gr}{{\operatorname{gr}\nolimits}}
\newcommand{\add}{\operatorname{add}\nolimits}
\newcommand{\obj}{\operatorname{obj}\nolimits}
\renewcommand{\rad}{\operatorname{rad}\nolimits}
\renewcommand{\soc}{\operatorname{soc}\nolimits}
\newcommand{\rk}{\operatorname{rank}\nolimits}
\newcommand{\kar}{\operatorname{char}\nolimits}
\newcommand{\Hom}{\operatorname{Hom}\nolimits}
\newcommand{\End}{\operatorname{End}\nolimits}
\newcommand{\uHom}{\operatorname{\underline{Hom}}\nolimits}
\newcommand{\oHom}{\operatorname{\overline{Hom}}\nolimits}
\renewcommand{\Im}{\operatorname{Im}\nolimits}
\newcommand{\Ker}{\operatorname{Ker}\nolimits}
\newcommand{\Coker}{\operatorname{Coker}\nolimits}
\newcommand{\rrad}{\mathfrak{r}}
\newcommand{\Ann}{\operatorname{Ann}\nolimits}
\newcommand{\Soc}{\operatorname{Soc}\nolimits}
\newcommand{\Top}{\operatorname{Top}\nolimits}
\newcommand{\Tr}{\operatorname{Tr}\nolimits}
\newcommand{\Ext}{\operatorname{Ext}\nolimits}
\newcommand{\cExt}{\operatorname{\widehat{Ext}}\nolimits}
\newcommand{\op}{{\operatorname{op}\nolimits}}
\newcommand{\env}{{\operatorname{env}\nolimits}}
\newcommand{\Ab}{{\operatorname{Ab}\nolimits}}
\newcommand{\CM}{{\operatorname{CM}\nolimits}}
\newcommand{\domdim}{{\operatorname{domdim}\nolimits}}
\newcommand{\gldim}{{\operatorname{gldim}\nolimits}}
\newcommand{\resdim}{{\operatorname{resdim}\nolimits}}
\newcommand{\id}{{\operatorname{id}\nolimits}}
\newcommand{\pd}{{\operatorname{pd}\nolimits}}
\newcommand{\comp}{\operatorname{\scriptstyle\circ}}
\newcommand{\m}{\mathfrak{m}}
\newcommand{\frakp}{\mathfrak{p}}
\newcommand{\fraka}{\mathfrak{a}}
\newcommand{\frakb}{\mathfrak{b}}
\newcommand{\frakc}{\mathfrak{c}}
\newcommand{\frako}{\mathfrak{o}}
\newcommand{\frakt}{\mathfrak{t}}
\newcommand{\G}{\Gamma}
\renewcommand{\L}{\Lambda}
\newcommand{\Z}{{\mathbb Z}}
\newcommand{\B}{{\mathcal B}}
\newcommand{\C}{{\mathcal C}}
\newcommand{\D}{{\mathcal D}}
\newcommand{\E}{{\mathcal E}}
\newcommand{\I}{{\mathcal I}}
\newcommand{\N}{{\mathcal N}}
\newcommand{\calO}{{\mathcal O}}
\newcommand{\X}{{\mathcal X}}
\newcommand{\Y}{{\mathcal Y}}
\renewcommand{\P}{{\mathcal P}}
\newcommand{\M}{{\mathcal M}}
\newcommand{\extto}{\xrightarrow}
\newcommand{\MaxSpec}{\operatorname{MaxSpec}\nolimits}
\newcommand{\HH}{\operatorname{HH}\nolimits}
\newcommand{\arrowtilde}{\widetilde{\rule{7mm}{0mm}}}
\newcommand{\Ind}{\operatorname{Ind}\nolimits}

\begin{document}

\begin{frame}
  \titlepage
\end{frame}

\begin{frame}{Outline}
  \tableofcontents
\end{frame}

\section{Examples of computations}

\begin{frame}[fragile]{$\Omega$-periodic modules}
$\Lambda$ -- finite dimensional algebra\medskip

\textbf{Recall:} A module $M$ is \emph{$\Omega$-periodic} if
$\Omega^n_\Lambda(M)\simeq M$ for some positive integer $n$.\medskip 

In some situations the period can indicate the degree of generators in
the Hochschild cohomology ring, $\oplus_{i\geqslant
  0}\Ext^i_{\Lambda^\env}(\Lambda, \Lambda)$. 
\end{frame}

\begin{frame}[fragile]{Periodic algebras}

$\Lambda$ -- finite dimensional $k$-algebra\medskip

$\Lambda^\env = \Lambda^\op\otimes_k \Lambda$ -- enveloping
algebra\medskip

\textbf{Recall:} $\Lambda$ is a \emph{periodic algebra} if $\Lambda$ is a
$\Omega$-periodic module, that is $\Omega^n_{\Lambda^\env}(\Lambda)
\simeq \Lambda$ as $\Lambda^\env$-modules.\medskip

\textbf{Facts:}
\begin{itemize}
\item $\Lambda$ is a selfinjective algebra.
\item All modules are $\Omega$-periodic.
\item The Hochschild cohomology module nilpotent elements is
  isomorphic to $k[x]$, where the degree of $x$ is the period.
\end{itemize}
\end{frame}

\begin{frame}[fragile]{Finding quivers}

\textbf{Recall:} $\Lambda = kQ/I$ with $I$ admissible is a basic algebra, that is, 
\[\Lambda = \oplus_{i=1}^t P_i\]
with $P_i$ indecomposable, then $P_i\not\simeq P_j$ for $i\neq j$. 

\textbf{Facts:} 
\begin{itemize}
\item $\Lambda$ -- finite dimensional algebra.
\item $\rad \Lambda = \langle \textrm{arrows}\rangle/I$.
\item $\Lambda/\rad\Lambda \simeq \frac{kQ/I}{\langle
    \textrm{arrows}\rangle/I} \simeq \textrm{linear span of vertices}$.
\item \begin{align}
    \rad\Lambda/\rad^2\Lambda & \simeq \frac{\langle
    \textrm{arrows}\rangle}{I}/\frac{\langle
    \textrm{arrows}\rangle^2}{I} \notag\\
    & \simeq \frac{\langle
    \textrm{arrows}\rangle}{\langle \textrm{arrows}\rangle^2}\notag\\
   & \simeq \textrm{linear span of arrows}\notag
\end{align}
\end{itemize}
\end{frame}

\begin{frame}[fragile]{Finding quivers}
$\Lambda$ -- finite dimensional algebra with $\Lambda/\rad\Lambda
\simeq k^n$ for some $n$.\medskip 

Algorithm:
\begin{enumerate}[\rm(1)]
\item Lift a complete set of orthogonal idempotents from
  $\Lambda/\rad\Lambda$ to a complete set of orthogonal idempotents in
  $\Lambda$, say $\{e_i\}_{i=1}^n$ -- the vertices. 
\item Compute $e_i\rad\Lambda/\rad^2\Lambda e_j$, find a basis and
  lift back to $e_i\rad\Lambda e_j$ -- the arrows from vertex $i$ to
  vertex $j$.   
\item Construct a quiver $Q$ from this and a homomorphism
  $\varphi\colon kQ \to \Lambda$. 
\item Find the kernel of $\varphi$. 
\end{enumerate}
\end{frame}

\begin{frame}[fragile]{Trivial extensions}
$\Lambda$ -- finite dimensional algebra\medskip

$T(\Lambda) = \Lambda\oplus D(\Lambda)$ -- \emph{trivial extension},

\[ (\lambda, f)\cdot (\lambda',f') = (\lambda\lambda', \lambda f' +
f\lambda)\]

\begin{itemize}
\item $\rad T(\Lambda) = \rad\Lambda \oplus D(\Lambda)$. 
\item $\rad^2 T(\Lambda) = \rad^2\Lambda \oplus D(\Lambda)\rad\Lambda
  + \rad\Lambda D(\Lambda)$. 
\item $\frac{\rad T(\Lambda)}{\rad^2 T(\Lambda)} \simeq
\frac{\rad\Lambda}{\rad^2\Lambda}\oplus \frac{D(\Lambda)}{D(\Lambda)\rad\Lambda
  + \rad\Lambda D(\Lambda)}$
\item $T(\Lambda)$ is a symmetric algebra, $\Gamma\simeq D(\Gamma)$ as
bimodules.
\end{itemize} 
\end{frame}

\section{Tilting modules}

\begin{frame}[fragile]{AR-theory}
Recall that a short exact sequence 
\[0\to A\xrightarrow{f} B\xrightarrow{g} C\to 0\]
is \emph{almost split exact} if it is not split exact and 
\begin{enumerate}[\rm(i)]
\item for any not splittable epimorphism $t\colon X\to C$ there is a
  homomorphism $t'\colon X\to B$ such that $gt' = t$, 
\item for any not splittable monomorphism $s\colon A\to Y$ there is a
  homomorphism $s'\colon A\to Y$ such that $s'f = s$.
\end{enumerate}
\end{frame}

\begin{frame}[fragile]{AR-theory}
Facts:
\begin{itemize}
\item $C$ and $A$ are indecomposable modules. 
\item $A\simeq D\Tr C$ and $C\simeq \Tr D(A)$.  
\item For any indecomposable non-projective module $C$ and
for any indecomposable non-injective module $A$, there is an almost
split sequence ending in $C$ and starting in $A$.  
\item An almost split sequence is a generator of the socle of
  $\Ext^1_\Lambda(C,D\Tr(C))$ as an $\End_\Lambda(C)$-module. 
\end{itemize}
\end{frame}

\begin{frame}[fragile]{APR-tilting}
$\Lambda=kQ$ -- hereditary, $Q$ no oriented cycle and connected.\medskip 

$S$ simple projetive module (and not injective)

$\Lambda$ is a classical tilting module $T$: 
\begin{itemize}
\item $\pd_\Lambda T \geqslant 1$, 
\item $\Ext^1_\Lambda(T,T) = (0)$, 
\item the number of indecomposable non-isomorphic summands in $T$ is
  equal to the number of isomorphism class of simple modules of
  $\Lambda$. 
\end{itemize}

$\Lambda = P \oplus S \longrightarrow T = P\oplus \Tr D(S)$ -- APR-tilting







\end{frame}
\end{document}
