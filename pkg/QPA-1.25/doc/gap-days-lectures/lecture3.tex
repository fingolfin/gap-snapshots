\newcommand{\QPAIntroPartNumber}{3}
\documentclass[usenames,dvipsnames]{beamer}

%\usetheme{ntnu}
\usetheme{Warsaw}

\usepackage{times}
\usepackage[T1]{fontenc}
\usepackage[all]{xy}
\usepackage{textpos}
\usepackage{tikz}

\newcommand{\defn}[1]{\textit{#1}}
\newcommand{\Q}{\mathbb{Q}}
\newcommand{\equivalence}{\simeq}
\newcommand{\VV}[2]{\begin{pmatrix} #1 & #2 \end{pmatrix}}
\newcommand{\vv}[2]{\left( \begin{smallmatrix} #1 & #2 \end{smallmatrix} \right)}
\newcommand{\into}{\hookrightarrow}
\newcommand{\iso}{\cong}
\newcommand{\dsum}{\oplus}
\DeclareMathOperator{\fmod}{mod}
\DeclareMathOperator{\Rep}{Rep}
\DeclareMathOperator{\Hom}{Hom}
\DeclareMathOperator{\coker}{coker}
\DeclareMathOperator{\im}{im}
\DeclareMathOperator{\rad}{rad}
\DeclareMathOperator{\soc}{soc}
\DeclareMathOperator{\Top}{top}

\title[Introduction to QPA, part \QPAIntroPartNumber]
      {Introduction to QPA}
\subtitle{Part \QPAIntroPartNumber}

\author{\O{}ystein~Skarts\ae{}terhagen \and \O{}yvind~Solberg}
\institute{
Department of Mathematical Sciences\\
Norwegian University of Science and Technology}

\date{Third GAP Days}

% If you have a file called "university-logo-filename.xxx", where xxx
% is a graphic format that can be processed by latex or pdflatex,
% resp., then you can add a logo as follows:

% \pgfdeclareimage[height=0.5cm]{university-logo}{university-logo-filename}
% \logo{\pgfuseimage{university-logo}}



% Delete this, if you do not want the table of contents to pop up at
% the beginning of each subsection:
% \AtBeginSubsection[]
% {
%   \begin{frame}<beamer>{Outline}
%     \tableofcontents[currentsection,currentsubsection]
%   \end{frame}
% }

\renewcommand{\mod}{\operatorname{mod}\nolimits}
\newcommand{\umod}{\operatorname{\underline{mod}}\nolimits}
\newcommand{\omod}{\operatorname{\overline{mod}}\nolimits}
\newcommand{\gr}{{\operatorname{gr}\nolimits}}
\newcommand{\add}{\operatorname{add}\nolimits}
\newcommand{\obj}{\operatorname{obj}\nolimits}
\newcommand{\rk}{\operatorname{rank}\nolimits}
\newcommand{\kar}{\operatorname{char}\nolimits}
\newcommand{\End}{\operatorname{End}\nolimits}
\newcommand{\uHom}{\operatorname{\underline{Hom}}\nolimits}
\newcommand{\oHom}{\operatorname{\overline{Hom}}\nolimits}
\renewcommand{\Im}{\operatorname{Im}\nolimits}
\newcommand{\Ker}{\operatorname{Ker}\nolimits}
\newcommand{\Coker}{\operatorname{Coker}\nolimits}
\newcommand{\rrad}{\mathfrak{r}}
\newcommand{\Ann}{\operatorname{Ann}\nolimits}
\newcommand{\Soc}{\operatorname{Soc}\nolimits}
\newcommand{\Tr}{\operatorname{Tr}\nolimits}
\newcommand{\Ext}{\operatorname{Ext}\nolimits}
\newcommand{\cExt}{\operatorname{\widehat{Ext}}\nolimits}
\newcommand{\op}{{\operatorname{op}\nolimits}}
\newcommand{\Ab}{{\operatorname{Ab}\nolimits}}
\newcommand{\CM}{{\operatorname{CM}\nolimits}}
\newcommand{\domdim}{{\operatorname{domdim}\nolimits}}
\newcommand{\gldim}{{\operatorname{gldim}\nolimits}}
\newcommand{\resdim}{{\operatorname{resdim}\nolimits}}
\newcommand{\id}{{\operatorname{id}\nolimits}}
\newcommand{\pd}{{\operatorname{pd}\nolimits}}
\newcommand{\comp}{\operatorname{\scriptstyle\circ}}
\newcommand{\m}{\mathfrak{m}}
\newcommand{\frakp}{\mathfrak{p}}
\newcommand{\fraka}{\mathfrak{a}}
\newcommand{\frakb}{\mathfrak{b}}
\newcommand{\frakc}{\mathfrak{c}}
\newcommand{\frako}{\mathfrak{o}}
\newcommand{\frakt}{\mathfrak{t}}
\newcommand{\G}{\Gamma}
\renewcommand{\L}{\Lambda}
\newcommand{\Z}{{\mathbb Z}}
\newcommand{\B}{{\mathcal B}}
\newcommand{\C}{{\mathcal C}}
\newcommand{\D}{{\mathcal D}}
\newcommand{\E}{{\mathcal E}}
\newcommand{\I}{{\mathcal I}}
\newcommand{\N}{{\mathcal N}}
\newcommand{\calO}{{\mathcal O}}
\newcommand{\X}{{\mathcal X}}
\newcommand{\Y}{{\mathcal Y}}
\renewcommand{\P}{{\mathcal P}}
\newcommand{\M}{{\mathcal M}}
\newcommand{\extto}{\xrightarrow}
\newcommand{\MaxSpec}{\operatorname{MaxSpec}\nolimits}
\newcommand{\HH}{\operatorname{HH}\nolimits}
\newcommand{\arrowtilde}{\widetilde{\rule{7mm}{0mm}}}
\newcommand{\Ind}{\operatorname{Ind}\nolimits}

\begin{document}

\begin{frame}
  \titlepage
\end{frame}

\begin{frame}{Outline}
  \tableofcontents
\end{frame}

\section{Homological algebra}

\begin{frame}{Original aim}
What we are not going to talk about:
\begin{itemize}
\item One original aim for QPA: \parbox[t]{5cm}{Do projective resolutions
    using Gr\"obner basis.}
\pause
\item Based on a non-minimal projective resolution - Green-S-Zacharia.
\item The resolution is available in QPA, \\
          \texttt{ProjectiveResolutionOfPathAlgebraModule(N,3);}
\pause
\item But nothing is developed around it. 
\end{itemize}
\end{frame}

\begin{frame}[fragile]{Projective cover}

Applies to: Only finitely generated modules in QPA.\medskip

Algorithm:

\begin{enumerate}[\rm(1)] 
\item Find minimal set of generators
\item Find maps from indecomposable projective covering the generators
\item Sum up the maps
\end{enumerate}
\pause 
\begin{verbatim}
gap> Q:= Quiver(3,[[1,2,"a"],[1,2,"b"],[2,2,"c"],
     [2,3,"d"],[3,1,"e"]]);;
gap> KQ:= PathAlgebra(Rationals, Q);;
gap> AssignGeneratorVariables(KQ);;
gap> A:= KQ/[d*e,c^2,a*c*d-b*d,e*a];;
gap> S := SimpleModules(A)[1];;
gap> f := ProjectiveCover(S);
<<[ 1, 4, 3 ]> ---> <[ 1, 0, 0 ]>>
\end{verbatim}

\end{frame}

\begin{frame}[fragile]{Injective envelope}

Injective envelope not implemented directly.\medskip 

Must use DIY:

\begin{verbatim}
gap> S := SimpleModules(A)[1];;
gap> Sop := DualOfModule(S);
<[ 1, 0, 0 ]>
gap> fop := ProjectiveCover(Sop);
<<[ 1, 0, 1 ]> ---> <[ 1, 0, 0 ]>>
gap> f := DualOfModuleHomomorphism(fop);
<<[ 1, 0, 0 ]> ---> <[ 1, 0, 1 ]>>
\end{verbatim}
\end{frame}

\begin{frame}[fragile]{Syzygies}

Finding syzygies, that is, the kernels in a projective resolution, is
based on finding kernels of \texttt{ProjectiveCover}s and this uses
linear algebra.\medskip  

The command:  \texttt{NthSyzygy} -- two arguments, one module $M$ and a
positive integer $n$.  
\begin{verbatim}
gap> NthSyzygy(S,3);
Computing syzygy number: 1 ...
Dimension vector for syzygy: [ 0, 4, 3 ]
Top of the 1th syzygy: [ 0, 2, 0 ]
Computing syzygy number: 2 ...
Dimension vector for syzygy: [ 0, 0, 1 ]
Top of the 2th syzygy: [ 0, 0, 1 ]
\end{verbatim}
\end{frame}

\begin{frame}[fragile]{Syzygies}

\begin{verbatim}
Computing syzygy number: 3 ...
Dimension vector for syzygy: [ 1, 2, 1 ]
Top of the 3th syzygy: [ 1, 0, 0 ]
<[ 1, 2, 1 ]>
gap> NthSyzygy(S,20);
Computing syzygy number: 1 ...
Top of the 1th syzygy: [ 0, 2, 0 ]
.....
Top of the 3th syzygy: [ 1, 0, 0 ]
Computing syzygy number: 4 ...
Dimension vector for syzygy: [ 0, 2, 2 ]
Top of the 4th syzygy: [ 0, 1, 0 ]
The module has projective dimension 4.
<[ 0, 2, 2 ]>
\end{verbatim}
\end{frame}

\begin{frame}[fragile]{Injective and projective dimension}

\begin{itemize}
\item Again based on using \texttt{ProjectiveCover}.
\item Injective and projective dimension can be infinite!
\item Can only check if they are less or equal to a number in general.
\item In some situations, we know that they are finite.
\item In general, no algorithm exists.
\item In some cases (work in progress): Can prove infinite projective
dimension.  Using modules over $kQ$ when $A = kQ/I$.  
\end{itemize} 

The command \texttt{InjDimensionOfModule} and
\texttt{ProjDimensionOfModule} takes two arguments, one module and one
non-negative integer:
\end{frame}

\begin{frame}[fragile]{Injective and projective dimension}

\begin{verbatim}
gap> ProjDimensionOfModule(S,3);
false
gap> ProjDimensionOfModule(S,4);
4
gap> ProjDimension(S);
4
gap> InjDimensionOfModule(S,1); 
false
gap> InjDimensionOfModule(S,2);
false
gap> InjDimensionOfModule(S,3);
3
\end{verbatim}
TODO: Set ProjDimension and InjDimension when computing NthSyzygy. 
\end{frame}

\begin{frame}[fragile]{Global dimension}
\begin{itemize}
\item Enough to check the projective dimension of the simple modules.
\item Global dimension can be infinite!
\item Can only check if it is less or equal to a number in general.
\item $kQ$ = hereditary $\longrightarrow$ global dimension set in QPA
\item $kQ/I$ and $Q$ no oriented cycle $\longrightarrow$ \parbox[t]{4cm}{global dimension finite, not
implemented.}
\item $kQ/I$ selfinjective $\longrightarrow$ \parbox[t]{6cm}{infinite global dimension, not implemented.}
\end{itemize}
\end{frame}

\begin{frame}[fragile]{Global dimension}
\begin{verbatim}
gap> A := NakayamaAlgebra(Rationals,[3,2,1]);;
gap> GlobalDimension(A);
1
gap> A := NakayamaAlgebra(Rationals, [2]);;
gap> GlobalDimensionOfAlgebra(A,3);
infinity
gap> A := NakayamaAlgebra(Rationals,
            [3,3,3,3,3,3,3,2,1]);;
gap> GlobalDimensionOfAlgebra(A,7);
5
\end{verbatim}
Known bound for monomial algebras, not implemented. 
\end{frame}

\begin{frame}[fragile]{Pullback and pushout}
Given
\[\vcenter{\xymatrix{A\ar[r]^f\ar[d]_g & B\\ C & }}\]
can construct pushout
\[\xymatrix{A\ar[r]^f\ar[d]_g & B\ar[d]^{g'}\\ C\ar[r]^{f'} & E}\]
\end{frame}

\begin{frame}[fragile]{Pullback and pushout}
\begin{verbatim}
gap> A := NakayamaAlgebra(Rationals, [3,2,1]);;                       
gap> B := IndecProjectiveModules(A)[1];
<[ 1, 1, 1 ]>
gap> f := RadicalOfModuleInclusion(B);
<<[ 0, 1, 1 ]> ---> <[ 1, 1, 1 ]>>
gap> g := TopOfModuleProjection(Source(f));
<<[ 0, 1, 1 ]> ---> <[ 0, 1, 0 ]>>
gap> po := PushOut(f,g);
[ <<[ 0, 1, 0 ]> ---> <[ 1, 1, 0 ]>>,
  <<[ 1, 1, 1 ]> ---> <[ 1, 1, 0 ]>> ]
gap> Range(g) = Source(po[1]);
true
gap> Range(f) = Source(po[2]);
true
\end{verbatim}
\end{frame}

\begin{frame}[fragile]{Pullback and pushout}
Given
\[\vcenter{\xymatrix{ & C\ar[d]^g\\  A\ar[r]^f & B,}}\]
we can construct the pullback 
\[\xymatrix{E\ar[r]^{f'}\ar[d]_{g'} & C\ar[d]^g\\ A\ar[r]^f & B.}\]
\end{frame}

\begin{frame}[fragile]{Pullback and pushout}

\begin{verbatim}
gap> g := CoKernelProjection(
              SocleOfModuleInclusion(C));
<<[ 1, 1, 1 ]> ---> <[ 1, 1, 0 ]>>
gap> f := RadicalOfModuleInclusion(Range(g));
<<[ 0, 1, 0 ]> ---> <[ 1, 1, 0 ]>>
gap> pb := PullBack(f,g);
[ <<[ 0, 1, 1 ]> ---> <[ 1, 1, 1 ]>>,
  <<[ 0, 1, 1 ]> ---> <[ 0, 1, 0 ]>> ]
\end{verbatim}
\end{frame}

\begin{frame}{Extensions}
Given a projective resolution of a module $M$ 
\[\cdots\to P_2\xrightarrow{d_2} P_1\xrightarrow{d_1}
P_0\xrightarrow{d_0} M\to 0,\]
the extension group $\Ext^1_\Lambda(M,N)$ is the homology 
of 
\[\Hom_\Lambda(P_0,N)\xrightarrow{d_1^*} \Hom_\Lambda(P_1,N)
\xrightarrow{d_2^*} \Hom_\Lambda(P_2,N)
\]
in the middle term.  The kernel of $d_2^*$ can be identified with
$\Hom_\Lambda(\Omega^1_\Lambda(M),N)$, so that 
\[\Ext^1_\Lambda(M,N)\simeq \Hom_\Lambda(\Omega^1_\Lambda(M),N)/\{
\Omega^1_\Lambda(M)\to P_0\xrightarrow{\forall f} N\}.\]  
\end{frame}

\begin{frame}{Extensions}
The function \texttt{ExtOverAlgebra} takes two arguments, two modules
$M$ and $N$, and returns a list of three elements: 
\begin{enumerate}[\rm(1)]
\item the inclusion $\Omega^1_\Lambda(M)\to P_0$, 
\item a basis $\mathcal{B}$ over the ground field $k$ of
  $\Ext^1_\Lambda(M,N)$ as homomorhisms from $\Omega^1_\Lambda(M)\to
  N$, and 
\item a function $\varphi\colon \Hom_\Lambda(\Omega^1_\Lambda(M),N)
  \to k^{|\mathcal{B}|}$, which computes the coordinates of any
  element in $\Hom_\Lambda(\Omega^1_\Lambda(M),N)$  as an element in 
$\Ext^1_\Lambda(M,N)$. 
\end{enumerate}
\end{frame}

\begin{frame}[fragile]{Extensions}
\begin{verbatim}
gap> Q:= Quiver(3,[[1,2,"a"],[1,2,"b"],[2,2,"c"],
                   [2,3,"d"],[3,1,"e"]]);;
gap> KQ:= PathAlgebra(Rationals, Q);;
gap> AssignGeneratorVariables(KQ);;
#I  Assigned the global variables [ v1, v2, v3, 
             a, b, c, d, e ]
gap> A:= KQ/[d*e,c^2,a*c*d-b*d,e*a];;
gap> S := SimpleModules(A)[1];;
gap> M := Kernel(ProjectiveCover(S));;
\end{verbatim}
\end{frame}

\begin{frame}[fragile]{Extensions}
\begin{verbatim}
gap> ext := ExtOverAlgebra(S,M);
[ <<[ 0, 4, 3 ]> ---> <[ 1, 4, 3 ]>>, 
  [ <<[ 0, 4, 3 ]> ---> <[ 0, 4, 3 ]>>,
    <<[ 0, 4, 3 ]> ---> <[ 0, 4, 3 ]>>, 
    <<[ 0, 4, 3 ]> ---> <[ 0, 4, 3 ]>>,
    <<[ 0, 4, 3 ]> ---> <[ 0, 4, 3 ]>>, 
    <<[ 0, 4, 3 ]> ---> <[ 0, 4, 3 ]>> ],
  function( map ) ... end ]
gap> U := Source(ext[1]);;
\end{verbatim}
\end{frame}

\begin{frame}[fragile]{Extensions}
\begin{verbatim}
gap> homUM := HomOverAlgebra(U,M);
[ <<[ 0, 4, 3 ]> ---> <[ 0, 4, 3 ]>>,
  <<[ 0, 4, 3 ]> ---> <[ 0, 4, 3 ]>>, 
  <<[ 0, 4, 3 ]> ---> <[ 0, 4, 3 ]>>,
  <<[ 0, 4, 3 ]> ---> <[ 0, 4, 3 ]>>, 
  <<[ 0, 4, 3 ]> ---> <[ 0, 4, 3 ]>> ]
gap> ext[3](homUM[4]);
[ -1, 0, 0, 1, 0 ]
\end{verbatim}
\end{frame}

\begin{frame}[fragile]{Extensions}
Elements in $\Ext^1_\Lambda(M,N) \longleftrightarrow \{f\colon
\Omega^1_\Lambda(M)\to N\}$

Representation by short exact sequences:

\[\xymatrix{
0\ar[r] & \Omega^1_\Lambda(M)\ar[r]\ar[d]^f & P_0\ar[r]\ar[d] &
M\ar[r]\ar@{=}[d] & 0\\
0\ar[r] & N \ar[r] & E\ar[r] & M\ar[r] & 0
}\]

\begin{verbatim}
gap> pushout := PushOut(ext[1], ext[2][1]);
[ <<[ 0, 4, 3 ]> ---> <[ 1, 4, 3 ]>>,
  <<[ 1, 4, 3 ]> ---> <[ 1, 4, 3 ]>> ]
gap> fprime := pushout[1];     
<<[ 0, 4, 3 ]> ---> <[ 1, 4, 3 ]>>
gap> gprime := CoKernelProjection(fprime); 
<<[ 1, 4, 3 ]> ---> <[ 1, 0, 0 ]>>
\end{verbatim}
\end{frame}

\begin{frame}[fragile]{Yoneda algebras}
Interesting examples
\begin{itemize}
\item Group cohomology ring, 
\[H^*(G,k) = \oplus_{i\geqslant 0} \Ext^i_{kG}(k,k).\]
\item Hochschild cohomology ring, 
\[\HH^*(\Lambda) = \oplus_{i\geqslant 0} \Ext^i_{\L^\op\otimes_k\Lambda}(\Lambda,\Lambda).\] 
\item Hopf algebra $H$, cohomology ring, $\oplus_{i\geqslant 0}
  \Ext^i_H(k,k)$. 
\item $\Lambda$ Koszul algebra, Koszul dual = $\oplus_{i\geqslant 0}
  \Ext^i_\Lambda(\Lambda_0,\Lambda_0)$. 
\item In general, $\oplus_{i\geqslant 0} \Ext^i_\Lambda(M,M)$. 
\end{itemize}
\end{frame}

\begin{frame}[fragile]{Yoneda algebras}
  The function \texttt{ExtAlgebraGenerators}, which takes two
  arguments, one module and one non-negative integer.
\begin{verbatim}
gap> ExtAlgebraGenerators(M,10);
[ [ 1, 1, 1, 2, 1, 1, 1, 1, 1, 1, 1 ], 
  [ 1, 1, 0, 1, 0, 0, 0, 0, 0, 0, 0 ], 
[ [ <<[ 0, 1, 0 ]> ---> <[ 0, 1, 0 ]>> ], 
[ <<[ 0, 1, 2 ]> ---> <[ 0, 1, 0 ]>> ], 
[  ], 
[ <<[ 0, 3, 3 ]> ---> <[ 0, 1, 0 ]>> ], 
[  ], [  ], [  ], [  ], [  ], [  ], [  ] ] ]
\end{verbatim}
\end{frame}

\begin{frame}[fragile]{Yoneda algebras}
The output from this function is a list of three elements, where the
\begin{itemize}
\item first element is the dimensions of $\Ext^i_\Lambda(M,M)$ for
$i=0,1,\ldots,n$, 
\item the second element is the number of minimal generators in the
  degrees $[0,\ldots,n]$, 
\item the third element is the generators in these degrees.
\end{itemize} 
\end{frame}

\begin{frame}[fragile]{Yoneda algebras}
\begin{verbatim}
gap> Q := Quiver(1,[[1,1,"a"],[1,1,"b"]]);;
gap> KQ := PathAlgebra(Rationals, Q);;
gap> AssignGeneratorVariables(KQ);;
gap> A := KQ/[a*a,b*b,a*b + 2*b*a];;
gap> ExtAlgebraGenerators(biA, 5);
[ [ 2, 2, 1, 0, 0, 0 ], [ 2, 2, 0, 0, 0, 0 ], 
  [ 
      [ <<[ 4 ]> ---> <[ 4 ]>>, 
          <<[ 4 ]> ---> <[ 4 ]>> ], 
      [ <<[ 12 ]> ---> <[ 4 ]>>, 
          <<[ 12 ]> ---> <[ 4 ]>> ], [  ], [  ], 
      [  ], [  ] ] ]
\end{verbatim}
\end{frame}

\section{AR-theory}

\begin{frame}[fragile]{AR-theory}
Recall that a short exact sequence 
\[0\to A\xrightarrow{f} B\xrightarrow{g} C\to 0\]
is \emph{almost split exact} if it is not split exact and 
\begin{enumerate}[\rm(i)]
\item for any not splittable epimorphism $t\colon X\to C$ there is a
  homomorphism $t'\colon X\to B$ such that $gt' = t$, 
\item for any not splittable monomorphism $s\colon A\to Y$ there is a
  homomorphism $s'\colon A\to Y$ such that $s'f = s$.
\end{enumerate}
\end{frame}

\begin{frame}[fragile]{AR-theory}
Facts:
\begin{itemize}
\item $C$ and $A$ are indecomposable modules. 
\item $A\simeq D\Tr C$ and $C\simeq \Tr D(A)$.  
\item For any indecomposable non-projective module $C$ and
for any indecomposable non-injective module $A$, there is an almost
split sequence ending in $C$ and starting in $A$.  
\item An almost split sequence is a generator of the socle of
  $\Ext^1_\Lambda(C,D\Tr(C))$ as an $\End_\Lambda(C)$-module. 
\end{itemize}
\end{frame}

\begin{frame}[fragile]{AR-theory}
\begin{enumerate}[\rm(1)]
\item Choose a non-zero element in $\Ext^1_\Lambda(C,D\Tr(C))$ (take a
  basis vector).
\item Check if it is annihilated by all elements in the radical of
  $\End_\Lambda(C)$. 
\item If not annihilated by the radical of $\End_\Lambda(C)$, multiply
  with an element in the radical of $\End_\Lambda(C)$ which is not
  annihilating it.  Go to (2).  If it is annihilated by the radical of
  $\End_\Lambda(C)$, it is in the socle of $\Ext^1_\Lambda(C,D\Tr(C))$
  and therefore gives the almost split sequence.  Jump to (4).
\item Done.
\end{enumerate}
In special cases, other algorithms exist, but this is the only
implemented in QPA. 
\end{frame}

\begin{frame}[fragile]{AR-theory}
The correspondence between the end terms are given by $D\Tr$ and $\Tr
D$. 
\begin{verbatim}
gap> A := NakayamaAlgebra(GF(3), [3,2,1]);;      
gap> S := SimpleModules(A);;
gap> DTr(S[1]);
<[ 0, 1, 0 ]>
gap> DTr(S[1],2);
Computing step 1...
Computing step 2...
<[ 0, 0, 1 ]>
\end{verbatim}
\end{frame}

\begin{frame}[fragile]{AR-theory}
\begin{verbatim}
gap> DTr(S[1],4);
Computing step 1...
Computing step 2...
Computing step 3...
Computing step 4...
<[ 0, 0, 0 ]>
gap> TrD(DTr(S[1])) = S[1];     
true
\end{verbatim}
\end{frame}

\begin{frame}[fragile]{AR-theory}
The function \texttt{AlmostSplitSequence} computes the almost split
sequence ending in the argument, assuming the argument is an
indecomposable module.\medskip

Given an almost split sequence: 
\[0\to A\xrightarrow{(f_i)} \oplus_{i=1}^t B_i\xrightarrow{(g_i)} C\to
0\]
\begin{itemize}
\item $f_i$ and $g_i$ are \emph{irreducible homomorphisms}. 
\item An irreducible homomorphism is either a monomorphism or an
  epimorphism.
\item All irreducible homomorphisms starting in an indecomposable
  injective module $I$ occur in $I\to I/\soc I$.
\item All irreducible homomorphisms ending in an indecomposable
  projective $P$ module  occur in $\rad P\to P$.
\item The valuation we ignore here. 
\end{itemize}
\end{frame}

\begin{frame}[fragile]{AR-theory}
The AR-quiver:
\begin{itemize}
\item Each vertex correspond to an indecomposable module. 
\item Each arrow correspond to an irreducible homomorphism. 
\item Each arrow has an valuation $(a,b)$ of pairs of positive
  integers.  
\end{itemize}
\end{frame}

\begin{frame}[fragile]{AR-theory}
\begin{verbatim}
gap> IsInjectiveModule(S[1]);
true
gap> ass1 := AlmostSplitSequence(S[1]);
[ <<[ 0, 1, 0 ]> ---> <[ 1, 1, 0 ]>>, 
  <<[ 1, 1, 0 ]> ---> <[ 1, 0, 0 ]>> ]
gap> I2 := Range(ass1[1]);
<[ 1, 1, 0 ]>
gap> IsIndecomposableModule(I2);
true
gap> IsInjectiveModule(I2);
true
\end{verbatim}
\end{frame}

\begin{frame}[fragile]{AR-theory}
\begin{verbatim}
gap> ass2 := AlmostSplitSequence(I2);
[ <<[ 0, 1, 1 ]> ---> <[ 1, 2, 1 ]>>, 
  <<[ 1, 2, 1 ]> ---> <[ 1, 1, 0 ]>> ]
gap> U := Range(ass2[1]);
<[ 1, 2, 1 ]>
gap> IsIndecomposableModule(U);
false
gap> decompU := DecomposeModule(U);
[ <[ 0, 1, 0 ]>, <[ 1, 1, 1 ]> ]
gap> I3 := decompU[2];
<[ 1, 1, 1 ]>
gap> IsInjectiveModule(I3);
true
\end{verbatim}
\end{frame}

\begin{frame}[fragile]{AR-theory}
\begin{verbatim}
gap> P2 := Source(ass2[1]);
<[ 0, 1, 1 ]>
gap> IsProjectiveModule(P2);
true
gap> RadicalOfModule(I3) = P2;
true
gap> IsProjectiveModule(I3);
true
gap> ass3 := AlmostSplitSequence(S[2]);
[ <<[ 0, 0, 1 ]> ---> <[ 0, 1, 1 ]>>, 
  <<[ 0, 1, 1 ]> ---> <[ 0, 1, 0 ]>> ]
gap> Range(ass3[1]) = P2;
true
gap> IsProjectiveModule(S[3]);
true
\end{verbatim}
\end{frame}

\begin{frame}[fragile]{AR-theory}
From the above calculations we get that the AR-quiver is 
\[\xymatrix@C=5pt{
& & I3\ar[dr] & &  \\
& P2\ar[ur]\ar[dr]\ar@{..}[rr] && I2\ar[dr] & \\
S[3]\ar[ur]\ar@{..}[rr] && S[2]\ar[ur]\ar@{..}[rr] && S[1] 
}\]
where a dotted line means that modules in this mesh forms an almost
split sequence. 
\end{frame}

\begin{frame}[fragile]{AR-theory}
\begin{itemize}
\item $\Lambda$ an indecomposable finite dimensional algebra. 
\item A component $\mathcal{C}$ of the AR-quiver is a collection of
  vertices/indecomposable modules closed under irreducible
  homomorphisms.
\item If there is a component $\mathcal{C}$ where the length of the
  indecomposable modules in $\mathcal{C}$ is bounded (in particular
  when it is finite), then $\Lambda$ is of finite representation type
  (only finitely many isomorphism classes of indecomposable modules)
  and $\mathcal{C}$ consists of all isomorphism classes of
  indecomposable modules.  
\item Hence we can see from the above calculations that $\Lambda$ is
  of finite representation type.
\end{itemize}
\end{frame}

\begin{frame}[fragile]{AR-theory}
\begin{verbatim}
gap> IsFiniteTypeAlgebra(A);
A_3
Finite type!
Quiver is a (union of) Dynkin quiver(s).
true
\end{verbatim}
\end{frame}
\end{document}
