Let $G$ be a finite group and $N\subseteq G$. The set $R\subseteq G$
with $|R|=k$ is called a ``relative difference set of order
$k-\lambda$ relative to the forbidden set $N$'' if the following
properties hold:

\beginlist%ordered{(a)}
\item{(a)} The multiset $\{ a.b^{-1}\colon a,b\in R\}$ contains
  every nontrivial ($\neq 1$) element of $G-N$ exactly $\lambda$
  times.  
\item{(b)} $\{ a.b^{-1}\colon a,b\in R\}$ does not contain
  any non-trivial element of $N$.
\endlist

Relative difference sets with $N=1$ are called (ordinary) difference
sets. As a special case, difference sets with $N=1$ and $\lambda=1$
correspond to projective planes of order $k-1$.  Here the blocks are
the translates of $R$ and the points are the elements of $G$.

In group ring notation a relative difference set satisfies
$$
RR^{-1}=k+\lambda(G-N).
$$

The set $D\subseteq G$ is called *partial relative difference set*
with forbidden set $N$, if
$$
    DD^{-1}=\kappa+\sum_{g\in G-N}v_gg   
$$ 

holds for some $1\leq\kappa\leq k$ and $0\leq v_g \leq \lambda$ for
all $g\in G-N$.  If $D$ is a relative difference set then ,obviously,
$D$ is also a partial relative difference set.

Two relative difference sets $D,D'\subseteq G$ are called *strongly
equivalent* if they have the same forbidden set $N\subseteq G$ and if
there is $g\in G$ and an automorphism $\alpha$ of $G$ such that
$D'g^{-1}=D^\alpha$. The same term is applied to partial relative
difference sets.

Let $D\subseteq G$ be a difference set, then the incidence structure
with points $G$ and blocks $\{Dg\;|\;g\in G\}$ is called the
*development* of $D$. In short:  ${\rm dev} D$. Obviously, $G$ acts on
${\rm dev}D$ by multiplication from the right.

If $D$ is a difference set, then $D^{-1}$ is also a difference set.
And ${\rm dev} D^{-1}$ is the dual of ${\rm dev} D$. So a group
admitting an operation some structure defined by a difference set does
also admit an operation on the dual structure. We may therefore change
the notion of equivalence and take $\phi$ to be an automorphism or an
anti-automorphism. Forbidden sets are closed under inversion, so this
gives a ``weak'' sort of strong equivalence.



%%%%%%%%%%%%%%%%%%%%%%%%%%%%%%%%%%%%%%%%%%%%%%%%%%%%%%%%%%%%%%%%%%%%%%%%%%%%%
%% 
%E ENDE 
%%
