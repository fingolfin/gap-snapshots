%%%%%%%%%%%%%%%%%%%%%%%%%%%%%%%%%%%%%%%%%%%%%%%%%%%%%%%%%%%%%%%%%%%%%%%%%
%%
%W  strategies.tex      ACE documentation - strategies   Alexander Hulpke
%W                                                      Joachim Neub"user
%W                                                            Greg Gamble
%%
%Y  Copyright (C) 2000  Centre for Discrete Mathematics and Computing
%Y                      Department of Information Tech. & Electrical Eng.
%Y                      University of Queensland, Australia.
%%

%%%%%%%%%%%%%%%%%%%%%%%%%%%%%%%%%%%%%%%%%%%%%%%%%%%%%%%%%%%%%%%%%%%%%%%%%
\Chapter{Strategy Options for ACE}

It can be difficult to select appropriate options when presented  with
a new enumeration. The problem is  compounded  by  the  fact  that  no
generally applicable rules exist to  predict,  given  a  presentation,
which option settings are ``good''. To  help  overcome  this  problem,
{\ACE} contains various commands which select  particular  enumeration
strategies. One or other of these strategies may work and, if not, the
results may indicate how  the  options  can  be  varied  to  obtain  a
successful enumeration.

If no strategy option is passed to {\ACE}, the `default'  strategy  is
assumed, which starts out presuming that the enumeration will be easy,
and if it turns out not to  be  so,  {\ACE}  switches  to  a  strategy
designed for more difficult enumerations.  The  other  straightforward
options for beginning users are `easy' and `hard'. Thus,  `easy'  will
quickly succeed or fail (in  the  context  of  the  given  resources);
`default' may succeed quickly, or if not will try the `hard' strategy;
and `hard' will run more slowly, from the beginning trying to succeed.

Strategy options are merely options that set a number of  the  options
seen in Chapter~"Options for ACE", all at once;  they  are  parsed  in
*exactly* the same way as other options; order *is* important.  It  is
usual to specify one strategy option and possibly  follow  it  with  a
number of options defined in Chapter~"Options for ACE", some of  which
may over-ride those options set by the strategy option.  Please  refer
to the introductory sections  of  Chapter~"Options  for  ACE",  paying
particular attention to Sections "Warnings regarding  Options",  "What
happens if no ACE Strategy Option or if  no  ACE  Option  is  passed",
and~"Interpretation of ACE  Options",  which  give  various  warnings,
hints and information on the interpretation of options.

There are eight strategy options. Each is passed without a value  (see
Section~"Interpretation of  ACE  Options")  except  for  `sims'  which
expects one of the integer values: 1, 3, 5, 7, or 9; and, `felsch' can
accept a value of 0 or 1, where 0  has  the  same  effect  as  passing
`felsch' with  no  value.  Thus  the  eight  strategy  options  define
thirteen standard strategies; these are listed  in  the  table  below,
along with all but three of the options (of Chapter~"Options for ACE")
that they set. Additionally, each strategy sets `path = 0',  `psize  =
256', and `dsize = 1000'. Recall `mend', `look' and  `com'  abbreviate
`mendelsohn'  (see~"option  mendelsohn"),   `lookahead'   (see~"option
lookahead") and `compaction' (see~"option compaction"), respectively.

% \begin{table}
% \hrule
% \caption{The Predefined Strategies}
% \label{tab:pred}
% \smallskip
% \renewcommand{\arraystretch}{0.875}
% \begin{tabular*}{\textwidth}{@{\extracolsep{\fill}}lrrrrrrrrrrrrr} 
% \hline\hline
%           & \multicolumn{13}{c}{parameter} \\ 
% \cline{2-14}
% strategy & path & row & mend & no & look & com & ct   & rt    & fill &
% pmode & psize & dmode & dsize \\ 
% \hline
% default    & 0   & 1   & 0    & -1 & 0    & 10  & 0    & 0     & 0  & 3    & 256  & 4    & 1000 \\
% easy   & 0   & 1   & 0    & 0  & 0    & 100 & 0    & 1000  & 1  & 0    & 256  & 0    & 1000 \\
% felsch:0& 0   & 0   & 0    & 0  & 0    & 10  & 1000 & 0     & 1  & 0    & 256  & 4    & 1000 \\
% felsch:1  & 0   & 0   & 0    & -1 & 0    & 10  & 1000 & 0     & 0  & 3    & 256  & 4    & 1000 \\
% hard   & 0   & 1   & 0    & -1 & 0    & 10  & 1000 & 1     & 0  & 3    & 256  & 4    & 1000 \\
% hlt    & 0   & 1   & 0    & 0  & 1    & 10  & 0    & 1000  & 1  & 0    & 256  & 0    & 1000 \\
% purec & 0   & 0   & 0    & 0  & 0    & 100 & 1000 & 0     & 1  & 0    & 256  & 4    & 1000 \\
% purer & 0   & 0   & 0    & 0  & 0    & 100 & 0    & 1000  & 1  & 0    & 256  & 0    & 1000 \\
% sims:1 & 0   & 1   & 0    & 0  & 0    & 10  & 0    & 1000  & 1  & 0    & 256  & 0    & 1000 \\
% sims:3 & 0   & 1   & 0    & 0  & 0    & 10  & 0    & -1000 & 1  & 0    & 256  & 4    & 1000 \\
% sims:5 & 0   & 1   & 1    & 0  & 0    & 10  & 0    & 1000  & 1  & 0    & 256  & 0    & 1000 \\
% sims:7 & 0   & 1   & 1    & 0  & 0    & 10  & 0    & -1000 & 1  & 0    & 256  & 4    & 1000 \\
% sims:9 & 0   & 0   & 0    & 0  & 0    & 10  & 1000 & 0     & 1  & 0    & 256  & 4    & 1000 \\
% \hline\hline
% \end{tabular*}
% \end{table}

\begintt
                               option
            ---------------------------------------------------------
strategy    row  mend  no  look  com    ct     rt  fill  pmode  dmode
---------------------------------------------------------------------
default       1     0  -1     0   10     0      0     0      3      4
easy          1     0   0     0  100     0   1000     1      0      0
felsch := 0   0     0   0     0   10  1000      0     1      0      4
felsch := 1   0     0  -1     0   10  1000      0     0      3      4
hard          1     0  -1     0   10  1000      1     0      3      4
hlt           1     0   0     1   10     0   1000     1      0      0
purec         0     0   0     0  100  1000      0     1      0      4
purer         0     0   0     0  100     0   1000     1      0      0
sims := 1     1     0   0     0   10     0   1000     1      0      0
sims := 3     1     0   0     0   10     0  -1000     1      0      4
sims := 5     1     1   0     0   10     0   1000     1      0      0
sims := 7     1     1   0     0   10     0  -1000     1      0      4
sims := 9     0     0   0     0   10  1000      0     1      0      4
---------------------------------------------------------------------
\endtt

Note that we explicitly (re)set all of the listed  enumerator  options
in all of the predefined strategies, even though some of them have  no
effect. For example,  the  `fill'  value  is  irrelevant  in  HLT-type
enumeration (see Section~"Enumeration Style"). The idea behind this is
that,  if  you  later  change  some  options  individually,  then  the
enumeration retains the ``flavour'' of the last selected  predefined
strategy.

Note also that other options which may effect an enumeration are  left
untouched by setting one of the predefined  strategies;  for  example,
the values of `max' (see~"option max") and `asis' (see~"option asis").
These options have an effect which  is  independent  of  the  selected
strategy.

Note that, apart from the `felsch := 0' and `sims  :=  9'  strategies,
all of the strategies are distinct, although some are very similar.

%%%%%%%%%%%%%%%%%%%%%%%%%%%%%%%%%%%%%%%%%%%%%%%%%%%%%%%%%%%%%%%%%%%%%%
\Section{The Strategies in Detail}

\atindex{C style}{@C style}\atindex{Cr style}{@Cr style}
\atindex{CR style}{@CR style}\atindex{R style}{@R style}
\atindex{R\* style}{@R\* style}\atindex{Rc style}{@Rc style}
\atindex{R/C style}{@R/C style}
\atindex{Defaulted  R/C  style}{@Defaulted  R/C  style}
\atindex{R/C (defaulted) style}{@R/C (defaulted) style}
Please note that the strategies  are  based  on  various  *enumeration
styles*: *C style*, *Cr style*, *CR style*, *R  style*,  *R\*  style*,
*Rc style*, *R/C style* and *defaulted R/C style*, all  of  which  are
described in detail in Section~"Enumeration Style".

\beginitems

\>`default'{option default}@{option `default'}&
Selects the default strategy. (Shortest abbreviation: `def'.)

This  strategy  is  based  on   the   *defaulted   R/C   style*   (see
Section~"Enumeration Style"). The idea here is that we assume that the
enumeration is ``easy'' and start out in *R style*. If  it  turns  out
not to be easy, then we regard it  as  ``hard'',  and  switch  to  *CR
style*, after performing a `lookahead' (see~"option lookahead") on the
entire table.

\>`easy'{option easy}@{option `easy'}&
Selects an ``easy'' R style strategy.

If this strategy is selected, we follow a HLT-type enumeration  style,
i.e. *R style* (see Section~"Enumeration Style"), but turn `lookahead'
(see~"option lookahead") and  `compaction'  (see~"option  compaction")
off. For small and/or easy enumerations, this strategy is likely to be
the fastest.

\>`felsch'{option felsch}@{option `felsch'}
\>`felsch:=<val>'{option felsch}@{option `felsch'}&
Selects a Felsch strategy; <val> should be 0 or 1. 
(Shortest abbreviation: `fel'.)

Here a *C style*  (see  Section~"Enumeration  Style")  enumeration  is
selected. Assigning `felsch' 0 or  no  value  selects  a  pure  Felsch
strategy, and a value of 1 selects a Felsch strategy with all relators
in  the  subgroup,  i.e.~`no'${}=-1$  (see~"option  no"),  and   turns
gap-`fill'ing (see "option fill") on.

\>`hard'{option hard}@{option `hard'}&  
Selects a mixed *R style* and *C style* strategy.

In many ``hard'' enumerations, a mixture of *R style*  and  *C  style*
(see Section~"Enumeration  Style")  definitions,  all  tested  in  all
essentially different positions, is appropriate. This  option  selects
such a mixed strategy. The idea here is that most of the work is  done
C  style  (with  the  relators  in  the   subgroup,   i.e.~`no'${}=-1$
(see~"option no"), and with gap-`fill'ing (see "option fill") on), but
that every $1000$ C  style  definitions  a  further  coset  number  is
applied to all relators.

*Guru  Notes:*
The $1000/1$ mix is not necessarily optimal, and some  experimentation
may be needed  to  find  an  acceptable  balance  (see,  for  example,
\cite{HR01}). Note also that, the  longer  the  total  length  of  the
presentation, the more work needs to be done  for  each  coset  number
application to the relators; one coset number application  can  result
in more than $1000$ definitions, reversing the balance between R style
and C style definitions.

\>`hlt'{option hlt}@{option `hlt'}&
Selects  {\ACE}'s   standard   HLT   strategy.   

Unlike  Sims'  \cite{Sim94}  default  HLT  strategy,  `hlt'  sets  the
`lookahead'  option  (see~"option  lookahead").  However,  the  option
sequence ```hlt, lookahead := 0''' easily achieves Sims'  default  HLT
strategy  (recall,  the  ordering  of  options   is   respected;   see
Section~"Honouring of the order in which ACE Options are passed").

This is an *R style* (see Section~"Enumeration Style") strategy.

\>`purec'{option purec}@{option `purec'}&
Sets the strategy to basic *C style* (see Section~"Enumeration  Style").

In this strategy there is no `compaction'  (see~"option  compaction"),
no gap-`fill'ing (see "option fill")  and  no  relators  in  subgroup,
i.e.~`no'${}=0$ (see~"option no").

\>`purer'{option purer}@{option `purer'}&
Sets the strategy  to basic *R style* (see Section~"Enumeration  Style").

In this strategy there is no `mendelsohn' (see  "option  mendelsohn"),
no `compaction' (see~"option compaction"), no `lookahead' (see~"option
lookahead") and no `row'-filling (see "option row").

\>`sims:=<val>'{option sims}@{option `sims'}&
Sets a Sims strategy; <val> should be one of 1, 3, 5, 7 or 9.

In his book~\cite{Sim94}, Sims discusses (and lists  in  Table  5.5.1)
ten  standard  enumeration  strategies.  The  Sims'   strategies   are
effectively `hlt' (see~"option hlt") without `lookahead'  (see~"option
lookahead"), with or without  `mendelsohn'  (see~"option  mendelsohn")
set, in R (`rt' positive, `ct := 0') or R\* style (`rt' negative,  `ct
:= 0'); and `felsch' (see~"option felsch"); all either with or without
(`lenlex')   table   standardisation   (see    Section~"Coset    Table
Standardisation  Schemes"  and~"ACEStandardCosetNumbering"  or~"option
standard") as the enumeration  proceeds.  {\ACE}  does  not  implement
table standardisation during an enumeration, and so only provides  the
odd-numbered strategies of Sims  ({\ACE}'s  numbering  coincides  with
that of Sims).

With care, it is often possible  to  duplicate  the  statistics  given
in~\cite{Sim94}  for  his  odd-numbered  strategies  and  it  is  also
possible  (using  the  interactive  facilities)  to  approximate   his
even-numbered strategies. Examples and a more detailed  exposition  of
the Sims strategies are given in Section~"Emulating Sims".

\enditems

%%%%%%%%%%%%%%%%%%%%%%%%%%%%%%%%%%%%%%%%%%%%%%%%%%%%%%%%%%%%%%%%%%%%%%%%%
%%
%E
