% This file was created automatically from iso.msk.
% DO NOT EDIT!
\Chapter{Block Designs and Projective Planes}

This section contains functions to help studying projective planes.
There is also a function converting relative difference sets to block
designs. Those desings can be studied with the \package{DESIGN} \cite{DESIGN}
package by L. Soicher.

Projective planes are always assumed to consist of positive integers
(as points) and sets of integers (as blocks). The incidence relation
is assumed to be the element relation. The blocks of a projective
plane must be *sets*.

\>ProjectivePlane( <blocks> ) O

Given a list of lists <blocks> which represents the blocks of a 
projective plane, a block design is generated. If the <blocks> is not 
a set of sets of the integers `[1..v]' for some $v$, the points are 
sorted and enumerated and the blocks are changed accordingly.
But the original names are known to the returned BlockDesign.

The block design generated this way will contain two extra entries,
<jblock> and <isProjectivePlane>. The matrix <.jblock> contains the 
number of the block containing the points $i$ and $j$ at the $(i,j)$th 
position. And <isProjectivePlane> will be `true'.
If <blocks> do not form the lines of a projective plane, an error is 
issued.


\>PointJoiningLinesProjectivePlane( <plane> ) O

Returns a matrix which has as $ij$th entry the point wich is contained
in the blocks with numbers $i$ and $j$. This matrix is also stored in
<plane>. Some operations are faster if <plane> contains this matrix.
If <plane> is not a projective plane, an error is issued.



\beginexample
gap> b:=[ [ 1, 3 ], [ 1, 6 ], [ 2, 4 ], [ 2, 7 ], 
>       [ 3, 5 ], [ 4, 6 ], [ 5, 7 ] ];;
gap> plane:=ProjectivePlane(b);
rec( isBlockDesign := true, v := 7, 
     blocks := [ [ 1, 3 ], [ 1, 6 ], [ 2, 4 ], [ 2, 7 ], 
                 [ 3, 5 ], [ 4, 6 ], [ 5, 7 ] ], 
     jblock := [ [ 0, 0, 1, 0, 0, 2, 0 ], [ 0, 0, 0, 3, 0, 0, 4 ], 
     [ 1, 0, 0, 0, 5, 0, 0 ], [ 0, 3, 0, 0, 0, 6, 0 ], 
     [ 0, 0, 5, 0, 0, 0, 7 ], [ 2, 0, 0, 6, 0, 0, 0 ], 
     [ 0, 4, 0, 0, 7, 0, 0 ] ], 
     isProjectivePlane := true )
gap> PointJoiningLinesProjectivePlane(plane);
[ [ 0, 1, 0, 0, 3, 0, 0 ], [ 1, 0, 0, 0, 0, 6, 0 ], [ 0, 0, 0, 2, 0, 4, 0 ], 
  [ 0, 0, 2, 0, 0, 0, 7 ], [ 3, 0, 0, 0, 0, 0, 5 ], [ 0, 6, 4, 0, 0, 0, 0 ], 
  [ 0, 0, 0, 7, 5, 0, 0 ] ]
gap> RecNames(plane);
[ "isBlockDesign", "v", "blocks", "jblock", "isProjectivePlane", "jpoint" ]
\endexample

\>DevelopmentOfRDS( <diffset>, <Gdata> ) O

This calculates the development of a (partial relative) difference set
<diffset> in the group given by <Gdata>. 
That is, the associated block design. 

<diffset> can be given as a list of group 
elements or a list of integers (positions in the set of group elements).
<Gdata> can either be the record returned by 
"PermutationRepForDiffsetCalculations" or a group or a set of group elements.

In either case, the returned object is a `BlockDesign' in the sense of 
L. Soichers DESIGN package.



\beginexample
gap> G:=CyclicGroup(21);;  Gdata:=PermutationRepForDiffsetCalculations(G);; 
gap> AllDiffsets([2],[1..21],4,[],Gdata,1);                                  
[ [ 2, 5, 16, 17 ], [ 2, 6, 10, 18 ] ]
gap> d1:=DevelopmentOfRDS(Set(G){[2,5,16,17]},Set(G)); 
rec( isBlockDesign := true, v := 21, 
  blocks := [ [ 1, 2, 5, 16, 17 ], [ 1, 3, 14, 15, 21 ], [ 1, 4, 8, 10, 13 ], 
      [ 1, 6, 7, 9, 20 ], [ 1, 11, 12, 18, 19 ], [ 2, 3, 9, 10, 12 ], 
      [ 2, 4, 7, 15, 19 ], [ 2, 6, 8, 11, 21 ], [ 2, 13, 14, 18, 20 ], 
      [ 3, 4, 6, 17, 18 ], [ 3, 5, 8, 19, 20 ], [ 3, 7, 11, 13, 16 ], 
      [ 4, 5, 9, 11, 14 ], [ 4, 12, 16, 20, 21 ], [ 5, 6, 12, 13, 15 ], 
      [ 5, 7, 10, 18, 21 ], [ 6, 10, 14, 16, 19 ], [ 7, 8, 12, 14, 17 ], 
      [ 8, 9, 15, 16, 18 ], [ 9, 13, 17, 19, 21 ], [ 10, 11, 15, 17, 20 ] ], 
  autSubgroup := <permutation group with 21 generators>, 
  pointNames := [ <identity> of ..., f1, f2, f1^2, f1*f2, f2^2, f1^2*f2, 
      f1*f2^2, f2^3, f1^2*f2^2, f1*f2^3, f2^4, f1^2*f2^3, f1*f2^4, f2^5, 
      f1^2*f2^4, f1*f2^5, f2^6, f1^2*f2^5, f1*f2^6, f1^2*f2^6 ], 
  blockSizes := [ 5 ], blockNumbers := [ 21 ], isSimple := true, 
  isBinary := true )
gap> d2:=DevelopmentOfRDS([2,5,16,17],Gdata);;
gap> d1=d2
true
gap> d1=DevelopmentOfRDS(Set(G){[2,5,16,17]},G);  
true
gap> d1=DevelopmentOfRDS([2,5,16,17],G);        
true
\endexample

Note that equality for block designs means equality of records. So
`DevelopmentOfRDS' generates exactly the same record in each of the
above examples. The output is in fact independent of the chosen data type of the input (as long as it is valid). In particular, the design always knows its `pointNames'.

\>ProjectiveClosureOfPointSet( <points>[, <maxsize>], <plane> ) O

Let <plane> be a projective plane. Let <points> be a set of non-collinear
points (integers) of this plane. Then
`ProjectiveClosureOfPointSet' returns a record with the entries <.closure> 
and <.embedding>.

Here <.closure> is the projective closure of <points>  (the smallest 
projectively closed subset of <plane> containing the points <points>). 
It is not checked, whether this is a projective plane. As the BlockDesign
<.closure> has points `[1..w]' and <plane> has poins `[1..v]' with 
$w\leq v$, we need an embedding of <.closure> into <plane>. This embedding
is the permutation <.embedding>. It is a permutation on `[1..v]' which
takes the points of <.closure> to a set of points in <plane> containing
<points> and preserving incidence. Note that nothing is known about the
behaviour of <.embedding> on any point outside `[1..w]' and 
`[1..w]^<.embedding>'.

If $<maxsize>$ is given and $<maxsize> \neq 0$, calculations are stopped
if the closure is known to
have at least <maxsize> points and the plane <plane> is returned as 
<.closure> with the trivial permutation as embedding.




Let's find a Baer subplane in the desarguesian plane of order $4$:
\beginexample
gap> G:=CyclicGroup(21);; Gdata:=PermutationRepForDiffsetCalculations(G);;
gap> AllDiffsets([2],[1..21],4,[],Gdata,1);                                  
[ [ 2, 5, 16, 17 ], [ 2, 6, 10, 18 ] ]
gap> plane:=DevelopmentOfRDS([2,5,16,17],Gdata);;
gap> ProjectiveClosureOfPointSet([1,3,4],plane);  
rec( closure := rec( isBlockDesign := true, v := 3, 
     blocks := [ [ 1, 2 ], [ 1, 3 ], [ 2, 3 ] ] 
     pointNames := [ <identity> of ..., f2, f1^2 ]), 
     embedding := (2,3,4) )
gap> IsProjectivePlane(last.closure);
false
gap> baer:=ProjectiveClosureOfPointSet([1,3,4,5],plane);;
gap> baer.closure.blocks;
[ [ 1, 2, 6 ], [ 1, 3, 5 ], [ 1, 4, 7 ], [ 2, 3, 7 ], 
  [ 2, 4, 5 ], [ 3, 4, 6 ], [ 5, 6, 7 ] ]
gap> IsProjectivePlane(baer.closure);
true
gap> Set(baer.closure.blocks,b->OnSets(b,baer.embedding));
[ [ 1, 3, 14 ], [ 1, 4, 8 ], [ 1, 5, 17 ], [ 3, 4, 17 ], 
  [ 3, 5, 8 ], [ 4, 5, 14 ], [ 8, 14, 17 ] ]
\endexample

%%%%%%%%%%%%%%%%%%%%
\Section{Isomorphisms and Collineations} 

Isomorphisms of projective planes are mappings which take points to
points and blocks to blocks and respect incidence. A *collineation* of
a projective plane $P$ is an isomorphism from $P$ to $P$.

As projective planes are assumed to live on the integers, isomorphisms
of projective planes are represented by permutations. To test if a
permutation on points is actually an isomorphism of projective planes,
the following methods can be used.

\>IsIsomorphismOfProjectivePlanes( <perm>, <plane1>, <plane2> ) O

Let <plane1>, <plane2> be two projective planes.
`IsIsomorphismOfProjectivePlanes' test if the permutation
<perm> on points defines an isomorphism of the projective planes 
 <plane1> and <plane2>.


\>IsCollineationOfProjectivePlane( <perm>, <plane> ) O

Let <plane> be a  projective plane and <perm> a permutation
on the points of this plane. `IsCollineationOfProjectivePlane(<perm>,<plane>)' returns 
`true', if <perm> induces a collineation of  <plane>.

This is just another form to call `IsIsomorphismOfProjectivePlanes(<perm>,<plane>,<plane>)'


\>IsomorphismProjPlanesByGenerators( <gens1>, <plane1>, <gens2>, <plane2> ) O
\>IsomorphismProjPlanesByGeneratorsNC( <gens1>, <plane1>, <gens2>, <plane2> ) O

Let <gens1> be a list of points generating the projective plane  
<plane1> and <gens2> a list of generating points for <plane2>. Then a 
permutation is returned representing a mapping from the points of <plane1> 
to those of <plane2> and taking the list <gens1> to the list <gens2>.
If there is no such mapping which defines an isomorphism of projective
planes, `fail' is returned.

`IsomorphismProjPlanesByGeneratorsNC' does *not* check whether <gens1> 
and <gens2> really generate the planes <plane1> and <plane2>. 



Look at the above example again:
\beginexample
gap> P:=ProjectivePlane( [ [ 1, 2, 6 ], [ 1, 3, 5 ], [ 1, 4, 7 ], 
>       [ 2, 3, 7 ], [ 2, 4, 5 ], [ 3, 4, 6 ], [ 5, 6, 7 ] ]);;
gap> pi:=IsomorphismProjPlanesByGenerators([1,2,3,4],P,[2,4,6,7],P);
(1,2,4,7,3,6,5)
gap> IsIsomorphismOfProjectivePlanes(pi,P,P);
true
gap> IsCollineationOfProjectivePlane(pi,P);                 
true
gap> IsomorphismProjPlanesByGenerators([1,2,3,4],P,[1,2,3,5],P);
fail
gap> ProjectiveClosureOfPointSet([1,2,3,5],P).closure.v;
4
\endexample


%%%%%%%%%%%%%%%%%%%%%%%%%%%
\Section{Central Collineations}

Let $\phi$ be a collineation of a projective plane which fixes one
point block-wise (the so-called *centre*) and one block point-wise
(the so-called *axis*). If the centre is contained in the axis, $\phi$
is called *elation*. Otherwise, $\phi$ is called *homology*. The group
of elations with given axis is called *translation group* of the plane
(relative to the chosen axis). A projective plane with transitive
translation group is called *translation plane*. Here transitivity is
on the points outside the axis.

\>ElationByPair( <centre>, <axis>, <pair>, <plane> ) O

Let <centre> be a point and  <axis> a block of a projective plane  
<plane> .
<pair>  must be a pair of points outside <axis> and lie on a block 
containing <center>. Then there is a unique collineation fixing <axis> 
pointwise  and <centre> blockwise (an elation) and taking  <point[1]> 
to <point[2]>. 

If one of the conditions is not met, an error is issued. 
This method is faster, if <plane.jpoint> is known (see 
"RDS:PointJoiningLinesProjectivePlane")


\>AllElationsCentAx( <centre>, <axis>, <plane>[, "generators"] ) O

Let <centre> be a point and  <axis> a block of the projective plane  
<plane>.
`AllElationsCentAx' returns the group of all elations with centre
<centre> and axis <axis> as a group of permutations on the points of 
<plane>.

If ``generators'' is set, only a list of generators of the translation
group is returned.
This method is faster, if <plane.jpoint> is known (see 
"RDS:PointJoiningLinesProjectivePlane")


\>AllElationsAx( <axis>, <plane>[, "generators"] ) O

Let <axis> be a block of a projective plane <plane>.
`AllElationsAx' returns the group of all elations with axis 
<axis>.

If ``generators'' is set, only a set of generators for the group of elations
is returned.
This method is faster, if <plane.jpoint> is known (see 
"RDS:PointJoiningLinesProjectivePlane")


\beginexample
gap> P:=ProjectivePlane( [ [ 1, 2, 6 ], [ 1, 3, 5 ], [ 1, 4, 7 ], 
>       [ 2, 3, 7 ], [ 2, 4, 5 ], [ 3, 4, 6 ], [ 5, 6, 7 ] ]);;
gap> pi:=ElationByPair(1,[1,2,6],[3,5],P);
(3,5)(4,7)
gap> AllElationsCentAx(1,[1,2,6],P);
Group([ (3,5)(4,7) ])
gap> AllElationsAx([1,2,6],P); 
Group([ (3,5)(4,7), (3,7)(4,5) ])
gap> AllElationsAx([1,2,6],P);
Group([ (3,5)(4,7), (3,7)(4,5) ])
gap> Size(last);
4
\endexample

\>IsTranslationPlane( [<infline>, ]<plane> ) O

Returns `true' if the plane <plane> has a block $b$ such that the
group of elations with axis $b$ is transitive outside $b$.

If <infline> is given, only the group of elations with axis 
<infline> is considered.
This is faster than 
calculating the full translation group if the projective plane <plane> 
is not a translation plane. If <plane> is a translation plane, the full
translation group is calculated.

This method is faster, if <plane.jpoint> is known (see 
"RDS:PointJoiningLinesProjectivePlane")



\beginexample
gap> AllElationsAx(P.blocks[1],P);
Group([ (3,5)(4,7), (3,7)(4,5) ])
gap> Size(last);
4
gap> IsTranslationPlane(P);
true
\endexample

\>HomologyByPair( <centre>, <axis>, <pair>, <plane> ) O

`HomologyByPair' returns the homology defined by the pair
<pair> fixing <centre> blockwise and <axis> pointwise. 
The returned permutation fixes <axis> pointwise and <centre> linewise and
takes <pair[1]> to <pair[2]>.


\>GroupOfHomologies( <centre>, <axis>, <plane> ) O

returns the group of homologies with centre <centre> and axis 
<axis> of the plane <plane>.


\beginexample
gap> HomologyByPair(3,[1,2,6],[4,5],P);
Error, The centre must be fixed blockwise called from
 # ...
gap> GroupOfHomologies(3,[1,2,6],P);   
Group(())
\endexample

%%%%%%%%%%%%%%%%%%%%
\Section{Collineations on Baer Subplanes}

Let $P$ be a projective plane of order $n^2$. A subplane $B$ of order
$n$ of $P$ is called *Baer subplane*. Baer suplanes are exactly the
maximal subplanes of $P$.

\>InducedCollineation( <baerplane>, <baercoll>, <point>, <image>, <planedata>, <embedding> ) O

If a projective plane contains a Baer subplane, collineations of the 
subplane may be lifted to the full plane. If such an extension to the 
full plane exists, it is uniquely determined by the image of one point 
outside the Baer plane.

Here <baercoll> is a collineation (a permutation of the points)
of the projective plane <baerplane>. 
The permutation <embedding> is a permutation on the points of the full pane
which converts the enumeration of <baerplane> to that of the full plane. 
This means that the image of the points of <baerplane> under <embedding> 
is a subset of the points of <plane>. Namely the one representing the Baer 
plane  in the enumeration used for the whole plane.
<point> and <image> are points outside the Baer plane.

The data for <baerplane> and <embedding> can be calculated using 
"ProjectiveClosureOfPointSet".

`InducedCollineation' returns a collineation of the full plane (as a  
permutation on the points of <plane>) which takes <point> to <image> and 
acts on the Baer plane as <baercoll> does. If no such collineation 
exists, `fail' is returned.

This method needs <plane.jpoint>. If it is unknown, it is calculated (see 
"RDS:PointJoiningLinesProjectivePlane")



Let's go back to an earlier example and find a planar collineation:
\beginexample
gap> G:=CyclicGroup(21);; Gdata:=PermutationRepForDiffsetCalculations(G);; 
gap> AllDiffsets([2],[1..21],4,[],Gdata,1); 
[ [ 2, 5, 16, 17 ], [ 2, 6, 10, 18 ] ]
gap> plane:=DevelopmentOfRDS([2,5,16,17],Gdata);;
gap> baer:=ProjectiveClosureOfPointSet([1,3,4,5],plane);; 
gap> pi:=InducedCollineation(baer.closure,(),21,15,plane,baer.embedding);
(2,16)(6,18)(7,12)(9,11)(10,13)(15,21)(19,20)
gap> 21^pi;
15
gap> ForAll(OnSets([1..7],baer.embedding),i->i^pi=i);
true
\endexample

%%%%%%%%%%%%%%%%%%%%
\Section{Invariants for Projective Planes}

The functions `NrFanoPlanesAtPoints', `PRank@RDS', `FingerprintAntiFlag'
and `FingerprintProjPlane' calculate invariants for finite projective
planes. For more details see \cite{RoederDiss} and
\cite{MoorhouseGraphs}. The values of some of these invariants are
available from the homepages of \cite{Moorhouse} and \cite{Royle} for
many planes.

\>NrFanoPlanesAtPoints( <points>, <plane> ) O

For a projective plane <plane>, `NrFanoPlanesAtPoints(<points>,<plane>)' 
calculates the so-called Fano invariant. That is, for each point 
in <points>, the number of subplanes of order 2 (so-called Fano planes)
containing this point is calculated.
The method returns a list of pairs of the form $[<point>,<number>]$
where <number> is the number of Fano sub-planes in <point>.

This method is faster, if <plane.jpoint> is known (see 
"RDS:PointJoiningLinesProjectivePlane"). Indeed, if <plane.jpoint> is 
not known, this method is very slow.


\beginexample
gap> G:=CyclicGroup(4^2+5);
<pc group of size 21 with 2 generators>
gap> diffset:=OneDiffset(G);        
[ f1, f1*f2, f1^2*f2^4, f1*f2^5 ]
gap> P:=DevelopmentOfRDS(diffset,G);;
gap> NrFanoPlanesAtPoints([3],P);
[ [ 3, 240 ] ]
\endexample
\>IncidenceMatrix( <plane> ) O

returns a matrix <I>, where the columns are numbered by the blocks and
the rows are numbered by points. And <I[i][j]=1> if and only if 
<points[i]> is incident (contained in) <blocks[j]> (an 0 else). 


\>PRank( <plane>, <p> ) O

Let $I$ be the incidence matrix of the projective plane <plane> and <p> a 
prime power.
The rank of $I.I^t$ as a matrix over
$GF(p)$ is called  <p>-rank of the projective plane. Here $I^t$ denotes
the transposed matrix. Note that this is a method within the RDS workspace,
so it has to be called as PRank@RDS


\beginexample
gap> G:=CyclicGroup(2^2+3);
<pc group of size 7 with 1 generators>
gap> P:=DevelopmentOfRDS(OneDiffset(G),G);;
gap> IncidenceMatrix(P);
[ [ 1, 1, 1, 0, 0, 0, 0 ], [ 1, 0, 0, 1, 1, 0, 0 ], [ 0, 1, 0, 1, 0, 1, 0 ], 
  [ 1, 0, 0, 0, 0, 1, 1 ], [ 0, 0, 1, 1, 0, 0, 1 ], [ 0, 0, 1, 0, 1, 1, 0 ], 
  [ 0, 1, 0, 0, 1, 0, 1 ] ]
gap> PRank@RDS(P,3);
6
gap> PRank@RDS(P,2);
4
\endexample
\>FingerprintProjPlane( <plane> ) O

For each anti-flag $(p,l)$ of a projective plane <plane> of order $n$, 
define an arbitrary but fixed enumeration of the lines through $p$ and 
the points on $l$. Say $l_1,\dots,l_{n+1}$ and $p_1,\dots,p_{n+1}$
The incidence relation defines a canonical bijection between the $l_i$ and
the $p_i$ and hence a permutation on the indices $1,\dots,n+1$. 
Let $\sigma_{(p,l)}$ be this permutation.

Denote the points and lines of the plane by $q_1,\dots q_{n^2+n+1}$ 
and $e_1,\dots,e_{n^2+n+1}$. 
Define the sign matrix as $A_{ij}=sgn(\sigma_{(q_i,e_j)})$ if $(q_i,e_j)$ 
is an anti-flag and $=0$ if it is a flag.
Then the fingerprint is defnied as the multiset of the entries of $|AA^t|$.


\>FingerprintAntiFlag( <point>, <linenr>, <plane> ) O

Let $m_1,\dots,m_{n+1}$ be the lines containing <point> and 
$E_1,\dots,E_{n+1}$ the points on the line given by <linenr> such that
$E_i$ is incident with $m_i$. Now label the points of $m_i$ as 
$<point>=P_{i,1},\dots,P_{i,n+1}=E_i$ and the lines of $E_i$ as
$<line>=l_1,\dots,l_{i,n+1}=m_i$. 
For $i\not = j$, each $P_{j,k}$ lies on exactly one line 
$l_{i,k\sigma_{i,j}}$ containing $E_i$ for some permutation $\sigma_{i,j}$

Define a matrix $A$, where $A_{i,j}$ is the sign of $\sigma_{i,j}$ if 
$i\neq j$ and $A_{i,i}=0$ for all $i$.
The partial fingerprint is the multiset of entries of $|AA^t|$ where $A^t$ 
denotes the transposed matrix of $A$.



Look at the above example again:
\beginexample
gap> NrFanoPlanesAtPoints([1,2,3],plane);
[ [ 1, 240 ], [ 2, 240 ], [ 3, 240 ] ]
gap> Set(NrFanoPlanesAtPoints([1..plane.v],plane),i->i[2])=[240];
true
gap> PRank@RDS(plane,2);
10
gap> PRank@RDS(plane,3);
21
gap> PRank@RDS(plane,5);
20
gap> FingerprintProjPlane(plane);
[ [ 12, 420 ], [ 16, 21 ] ]
gap> FingerprintAntiFlag(1,6,plane);
[ [ 3, 20 ], [ 4, 5 ] ]
\endexample

%%%%%%%%%%%%%%%%%%%%%%%%%%%%%%%%%%%%%%%%%%%%%%%%%%%%%%%%%%%%%%%%%%%%%%%%%%%%%
%%
%E
%%

