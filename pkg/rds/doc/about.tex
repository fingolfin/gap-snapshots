% This file was created automatically from about.msk.
% DO NOT EDIT!
\Chapter{About this package}

The {\RDS} package is meant to help with complete searches for
relative difference sets in non-abelian groups. Of course, it also
works for abelian groups, but no special features are implemented for
this case. In particular, there is no support for multipliers.

{\RDS} has no undocumented functions. While this is generally regarded
as a feature, it leads to a quite long manual and a lot of
documentation not needed for everyday work. To make reading easier,
all but the basic chapters contain a small introductory paragraph
pointing out which functions may be interesting for the user and which
are merely helper functions called by other functions.

The structure of this manual is a follows: First, there is a chapter
about brute force methods which are easy to use but are not suitable
for very difficult calculations.

Then, chapter "RDS:A basic example" shows the use of the more advanced
methods in \package{RDS} and explains the basic idea of a complete
search for difference sets with this package. After reading this
chapter, you should be able to use \package{RDS} even for large
examples. 

The following chapters "RDS:General concepts" and "RDS:Invariants for
Difference Sets" contain the documentation of the functions used in a
search for difference sets. They explain the concepts and low level
functions which provide a lot of control over the searching process. If
you are searching for difference sets in several groups of the same
order, you may find this helpful.

The next chapter shows an example of calculating a relative
difference set using low level functions.

Chapter "RDS:Ordered Signatures" introduces another invariant for
difference sets. The functions for calculating this invariant do only
work effectively in a few cases, so this part of \package{RDS} is a
little bit experimental. However, the invariant is very powerful, so
this chapter is kept.

In "RDS:Block Designs and Projective Planes", the methods for
generating a BlockDesign in the sense of \package{DESIGN} \cite{DESIGN} from a
difference set are described. A few functions for analyzing projective
planes are given as well.

The final chapter describes a few functions which are not related to
difference sets and may be useful in other situations.

%%%%%%%%%%%%%%%%%%%%%%
\Section{Acknowledgements}

I would like to thank U.~Dempwolff for supervising the thesis out of
which \package{RDS} grew, and L.~Soicher for many suggestions which
greatly improved the usability of this package.


%%%%%%%%%%%%%%%%%%%%%%
\Section{Installation}

\package{RDS} depends on Leonard Soicher's \package{DESIGN} \cite{DESIGN} package
which, in turn, depends on \package{GRAPE} \cite{GRAPE}. You need to install these
packages before you can run \package{RDS}.

\beginlist%ordered{1}
\item{1.} Download the package archive rds$ ver$ .$ ext$
   where $ver$ is some version number and $ext$ is an extension like tar.bz2,
   tar.gz, -win.zip or zoo.


\item{2.} Copy the archive to the directory where the other packages live.
   This is either the directory `pkg' in the GAP root path or a local directory in your home 
   directory (on most unix-like systems, this will probably be {`\~{}/gap/pkg/'}).

\item{3.} change directory to your package directory and unpack the
archive by using the right one of the following commands:
%
  \itemitem{} %unordered{}
                    tar -xjf rds<ver>.tar.bz2
  \itemitem{}tar -xzf rds<ver>.tar.gz
  \itemitem{}zoo -extract rds<ver>.zoo 
  \itemitem{}unzip rds<ver>-win.zip 

(replace $ver$ with the version number)

\item{4.}%ordered{}
  start GAP. If you have unpacked the archive to 'gap/pkg' in your
  home directory, you might have to use ''gap -l '$homedir$/gap;' ''
  where $homedir$ is the path of your home directory (use 'pwd' to
  find out what it is, if you don't know it)

\item{5.} Type `LoadPackage("rds");' to load \package{RDS}

\endlist
%
For a test, see the examples in chapters "RDS:AllDiffsets and
OneDiffset" and "RDS:A basic example".

%%%%%%%%%%%%%%%%%%%%%%
\Section{Verbosity}

There are two info classes that control the about of additional
information \package{RDS} prints:

\>`InfoRDS' V

Some methods of the RDS package print additional information if `InfoRDS'
is set to a level of 1 or higher. At level 0, no information is output. 
The default value is 1.


\>`DebugRDS' V

Some methods of the RDS package print additional information if `DebugRDS'
is set to a level of 1 or higher. At level 0, no information is output. 
The default level is 0. Expect a lot of output at level 2.





%%%%%%%%%%%%%%%%%%%%%%
\Section{Definitions and Objects}

This section lists the definition of ordinary and relative difference
sets as well as the concept of partial difference sets and their
development.  This will be repeated in "RDS:Introduction" where a
notion of equivalence is introduced and the implementation in
\package{RDS} is discussed.

%\input rdsshort
Let $G$ be a finite group and $N\subseteq G$. The set $R\subseteq G$
with $|R|=k$ is called a ``relative difference set of order
$k-\lambda$ relative to the forbidden set $N$'' if the following
properties hold:

\beginlist%ordered{(a)}
\item{(a)} The multiset $\{ a.b^{-1}\colon a,b\in R\}$ contains
  every nontrivial ($\neq 1$) element of $G-N$ exactly $\lambda$
  times.  
\item{(b)} $\{ a.b^{-1}\colon a,b\in R\}$ does not contain
  any non-trivial element of $N$.
\endlist

Let $D\subseteq G$ be a difference set, then the incidence structure
with points $G$ and blocks $\{Dg\;|\;g\in G\}$ is called the
*development* of $D$. In short:  ${\rm dev} D$. Obviously, $G$ acts on
${\rm dev}D$ by multiplication from the right.

Relative difference sets with $N=1$ are called (ordinary) difference
sets. The development of a difference set with $N=1$ and $\lambda=1$
is projective plane of order $k-1$.

In group ring notation a relative difference set satisfies
$$
RR^{-1}=k+\lambda(G-N).
$$

The set $D\subseteq G$ is called *partial relative difference set*
with forbidden set $N$, if
$$
    DD^{-1}=\kappa+\sum_{g\in G-N}v_gg   
$$ 

holds for some $1\leq\kappa\leq k$ and $0\leq v_g \leq \lambda$ for
all $g\in G-N$.  If $D$ is a relative difference set then ,obviously,
$D$ is also a partial relative difference set.


*IMPORTANT NOTE*

\package{RDS} implicitly assumes that the *every* partial difference
set contains the identity element (see the notion of equivalence in
"RDS:Introduction" for the mathematical reason). However, the identity
*must not* be contained in the lists representing partial relative
difference sets.

So in \package{RDS}, the difference set `[ (), (1,2,3,4,5,6,7),
(1,4,7,3,6,2,5) ]' is represented by the list `[ (1,2,3,4,5,6,7),
(1,4,7,3,6,2,5) ]'. And no set of three non-trivial permutations will
be accepted as an ordinary difference set of `Group((1,2,3,4,5,6,7))'.

For this reason the lists returned by functions like "AllDiffsets" do
only contain non-trivial elements and look too short.



%%%%%%%%%%%%%%%%%%%%%%%%%%%%%%%%%%%%%%%%%%%%%%%%%%%%%%%%%%%%%%%%%%%%%%%%%%%%%
%% 
%E ENDE 
%%


