\documentclass[12pt]{article}
\usepackage{amsmath,amssymb,palatino,verbatim,a4wide,amsthm,euler}
\usepackage[matrix,arrow]{xy}
\newtheorem{definition}{Definition}
\newtheorem{claim}[definition]{Claim}
\newtheorem{lemma}[definition]{Lemma}
\newtheorem{theorem}[definition]{Theorem}
\newtheorem{observation}[definition]{Observation}
\newtheorem{trouble}[definition]{Troubling Observation}
\renewcommand{\qedsymbol}{\rule{0.5em}{0.5em}}
\title{Cohomology Products in \textsf{CRIME}}
\author{Marcus Bishop}
\newcounter{savedenumii}
\begin{document}
\maketitle
The purpose of this document is to explain
the implementation of cohomology products
in the \textsf{CRIME} package for \textsf{GAP}.
In this document, the composition $g\circ f$ of two functions
$f$ and $g$ is the function obtained by applying $f$ first 
and then $g$. The symbol $\circlearrowright$ is used
in diagrams to indicate that a polygon either commutes
or anticommutes. 

Let $G$ be a finite $p$-group for some prime $p$
and let $k=\mathbb{F}_p$. We write $k$ for the
trivial $kG$-module. We assume that we can
calculate a $kG$-projective resolution $P_\ast$
of $k$. In other words, for $n$ as large as we need,
we can compute the integers
$\left\{b_m:0\le m\le n\right\}$ the
maps $\left\{\partial_m:1\le m\le n\right\}$
and the map $\epsilon$ such that
\begin{equation}\label{proj}
\xymatrix{P_n\ar[r]^{\partial_n}
&P_{n-1}\ar[r]&\dots\ar[r]
&P_1\ar[r]^{\partial_1}
&P_0\ar[r]^{\epsilon}&k}\end{equation}
is exact where $P_m=\left(kG\right)^{\oplus b_m}$.
Later, we will assume also that $P_\ast$ is minimal,
but this assumption isn't needed at this point.

\section{Cohomology Products}\label{cp}
The following construction is taken from \cite{carlson}.
We begin with two cocycles $f:P_i\to k$ and $g:P_j\to k$.
This means that $f\circ\partial_{i+1}=g\circ\partial_{j+1}=0$.
We want to compute the cup product $fg:P_{i+j}\to k$.

We first convert $f$ into an chain map,
resulting in the following commutative diagram.
\begin{equation}\label{ff}\xymatrix{
P_m\ar[r]^{\partial_m}\ar[d]^{f_m}
&P_{m-1}\ar[d]^{f_{m-1}}\ar[r]^{\partial_{m-1}}
&P_{m-2}\ar[d]^{f_{m-2}}\ar[r]
&\dots\ar[r]
&P_{i+2}\ar[r]^{\partial_{i+2}}\ar[d]^{f_{i+2}}
&P_{i+1}\ar[r]^{\partial_{i+1}}\ar[d]^{f_{i+1}}
&P_i\ar[d]^{f_i}\ar[dr]^f
\\
P_{m-i}\ar[r]_{\partial_{m-i}}
&P_{m-i-1}\ar[r]\ar[r]_{\partial_{m-i-1}}
&P_{m-i-2}\ar[r]
&\dots\ar[r]
&P_2\ar[r]_{\partial_2}
&P_1\ar[r]_{\partial_1}
&P_0\ar[r]_\epsilon&k\ar[r]&0
}\end{equation}
This is done as follows.
\begin{enumerate}
\item Define $f_i$ such that $\epsilon\circ f_i=f$.
This is possible by projectivity of $P_i$.
\item Define $f_{i+1}$ such that $\partial_1\circ f_{i+1}
=f_i\circ\partial_{i+1}$.
This is possible by projectivity of $P_{i+1}$ since
\[\mathrm{im}\left(f_i\circ\partial_{i+1}\right)
\le\ker\left(\epsilon\right)
=\mathrm{im}\left(\partial_1\right)\]
as $\epsilon\circ\left(f_i\circ\partial_{i+1}\right)
=f\circ\partial_{i+1}=0$ since $f$ is a cocycle.
\item Define $f_{i+2}$ such that $\partial_2\circ f_{i+2}
=f_{i+1}\circ\partial_{i+2}$.
This is possible by projectivity of $P_{i+2}$ since
\[\mathrm{im}\left(f_{i+1}\circ\partial_{i+2}\right)
\le\ker\left(\partial_1\right)
=\mathrm{im}\left(\partial_2\right)\]
as $\partial_1\circ\left(f_{i+1}\circ\partial_{i+2}\right)
=f_i\circ\partial_{i+1}\circ\partial_{i+2}=0$.
\item Define $f_m$ for $m>i+2$ by recursion 
such that $\partial_{m-i}\circ f_m
=f_{m-1}\circ\partial_m$.
This is possible by projectivity of $P_m$ since
\[\mathrm{im}\left(f_{m-1}\circ\partial_m\right)
\le\ker\left(\partial_{m-i-1}\right)
=\mathrm{im}\left(\partial_{m-i}\right)\]
as $\partial_{m-i-1}\circ\left(f_{m-1}\circ\partial_m\right)
=f_{m-2}\circ\partial_{m-1}\circ\partial_m=0$.
\end{enumerate}
Then the product $fg$ is calculated as $g\circ f_{i+j}$.
The process above is used to compute the multiplication
table used by the \verb!CohomologyRing! command
and is used to find generators by the \verb!CohomologyGenerators!
command.

\section{The Yoneda Cocomplex}
My understanding of the purpose of the Yoneda Cocomplex
is the following. The definition of the Massey product below
requires a cocomplex having an associative product.
The product defined above,
however, is defined only for $f$ and $g$ {\em cocycles}
in $\mathrm{Hom}\left(P_\ast,k\right)$.
The Yoneda cocomplex $Y$, on the other hand, has the same
cohomology as $\mathrm{Hom}\left(P_\ast,k\right)$,
but has an associative product defined 
for all cochains, namely composition.
Moreover, we will show that via the isomorphism 
$\Phi:H^\ast\left(G,k\right)\to
H^\ast\left(Y\right)$, composition in $Y$ 
agrees with the product defined in Section \ref{cp}
up to the factor $\left(-1\right)^{\deg f\deg g}$.
In other words, \[\Phi\left(fg\right)
=\left(-1\right)^{\deg f\deg g}\Phi\left(g\right)
\circ\Phi\left(f\right).\]

The following construction comes from \cite{borge}.
\begin{definition}
For $i\ge 0$ define
\[Y^i=\prod_{m\ge i}\mathrm{Hom}_{kG}\left(P_m,P_{m-i}\right).\]
Then an element $f\in Y^i$ is a collection of $kG$-homomorphisms
$\left\{f_m:P_m\to P_{m-i}:m\ge i\right\}$
as in the following diagram.

\begin{equation}\label{f}\xymatrix{
P_n\ar[r]^{\partial_n}\ar[d]^{f_n}
&P_{n-1}\ar[r]\ar[d]^{f_{n-1}}
&\dots\ar[r]
&P_m\ar[d]^{f_m}\ar[r]^{\partial_m}
&P_{m-1}\ar[d]\ar[d]^{f_{m-1}}\ar[r]^{\partial_{m-1}}
&P_{m-2}\ar[d]^{f_{m-2}}\ar[r]
&\dots\ar[r]
&P_{i+1}\ar[r]^{\partial_{i+1}}\ar[d]^{f_{i+1}}
&P_i\ar[d]^{f_i}
\\
P_{n-i}\ar[r]_{\partial_{n-i}}
&P_{n-i-1}\ar[r]
&\dots\ar[r]\ar[r]
&P_{m-i}\ar[r]_{\partial_{m-i}}
&P_{m-i-1}\ar[r]_{\partial_{m-i-1}}
&P_{m-i-2}\ar[r]
&\dots\ar[r]
&P_1\ar[r]_{\partial_1}
&P_0
}\end{equation}
\end{definition}
Diagram (\ref{f}) is not required to commute.

\begin{definition}
$\displaystyle{Y=\bigoplus_{i\ge 0}Y^i}$
is called the {\em Yoneda cocomplex} of $P_\ast$.
We write $\deg\left(f\right)=~i$ for $f\in Y^i$.
Let $f=\left\{f_m:m\ge i\right\}\in Y^i$ and define 
\begin{align*}
\partial:Y^i&\to Y^{i+1}\\
f&\mapsto\left\{f_{m-1}\circ\partial_m
-\left(-1\right)^i\partial_{m-i}\circ f_m
:m\ge 1\right\}.
\end{align*}
\end{definition}

We observe that cocycles in $Y$ are those
elements $f$ for which (\ref{f}) commutes if $\deg f$ is even
and anticommutes if $\deg f$ is odd.

\begin{lemma}
$Y$ with differentiation
$\partial$ is a cocomplex, that is, $\partial^2=0$. 
\end{lemma}

\begin{proof}
Let $f\in Y^i$.  We will show that $\partial^2 f=0$ at the point $P_m$
in (\ref{f}) for $m\ge i+2=\deg\left(\partial^2 f\right)$. 
Follow along in the picture.
\begin{eqnarray*}
\left(\partial\left(\partial f\right)\right)_m
&=&\left(\partial f\right)_{m-1}\circ\partial_m
-\left(-1\right)^{i+1}
\partial_{m-i-1}\circ
\left(\partial f\right)_m\\
&=&\left(f_{m-2}\circ\partial_{m-1}
-\left(-1\right)^i\partial_{m-i-1}\circ f_{m-1} \right)
\circ\partial_m\\
&&-\left(-1\right)^{i+1}
\partial_{m-i-1}\circ
\left(f_{m-1}\circ\partial_m
-\left(-1\right)^i\partial_{m-i}\circ f_m\right)\\
&=&f_{m-2}\circ\partial_{m-1}\circ\partial_m
-\partial_{m-i-1}\circ\partial_{m-i}\circ f_m\\
&=&0
\end{eqnarray*}
\end{proof}

\begin{theorem}
The cohomology groups of $Y$ are $H^\ast\left(G,k\right)$.
\end{theorem}
\begin{proof}
We will define a group isomorphism
$\Phi:H^i\left(G,k\right)\to H^i\left(Y\right)$.
\begin{enumerate}
\item Let $f:P_i\to k$ be a cocycle
in $\mathrm{Hom}_{kG}^i\left(P_\ast,k\right)$,
so $f\circ\partial_{i+1}=0$.
Define $\Phi\left(f\right)=
\left\{f_m:m\ge i\right\}\in Y^i$ as follows.
The element $\Phi\left(f\right)$, together with $f$, is pictured
in the following diagram.

\begin{equation}\label{pff}\xymatrix{
P_m\ar[r]^{\partial_m}\ar[d]^{f_m}
&P_{m-1}\ar[d]^{f_{m-1}}\ar[r]^{\partial_{m-1}}
&P_{m-2}\ar[d]^{f_{m-2}}\ar[r]
&\dots\ar[r]
&P_{i+2}\ar[r]^{\partial_{i+2}}\ar[d]^{f_{i+2}}
&P_{i+1}\ar[r]^{\partial_{i+1}}\ar[d]^{f_{i+1}}
&P_i\ar[d]^{f_i}\ar[dr]^f
\\
P_{m-i}\ar[r]_{\partial_{m-i}}
&P_{m-i-1}\ar[r]\ar[r]_{\partial_{m-i-1}}
&P_{m-i-2}\ar[r]
&\dots\ar[r]
&P_2\ar[r]_{\partial_2}
&P_1\ar[r]_{\partial_1}
&P_0\ar[r]_\epsilon&k\ar[r]&0
}\end{equation}

\begin{enumerate}
\item\label{first} Define $f_i$ such that $\epsilon\circ f_i=f$.
This is possible by projectivity of $P_i$.
\item\label{firstsecond} Define $f_{i+1}$ such that $\partial_1\circ f_{i+1}
=\left(-1\right)^i f_i\circ\partial_{i+1}$.
This is possible by projectivity of $P_{i+1}$ since
\[\mathrm{im}\left(\left(-1\right)^i f_i\circ\partial_{i+1}\right)
\le\ker\left(\epsilon\right)
=\mathrm{im}\left(\partial_1\right)\]
as $\epsilon\circ\left(\left(-1\right)^i f_i\circ\partial_{i+1}\right)
=\left(-1\right)^i f\circ\partial_{i+1}=0$.
\item Define $f_{i+2}$ such that $\partial_2\circ f_{i+2}
=\left(-1\right)^i f_{i+1}\circ\partial_{i+2}$.
This is possible by projectivity of $P_{i+2}$ since
\[\mathrm{im}\left(\left(-1\right)^i f_{i+1}\circ\partial_{i+2}\right)
\le\ker\left(\partial_1\right)
=\mathrm{im}\left(\partial_2\right)\]
as $\partial_1\circ\left(\left(-1\right)^i
f_{i+1}\circ\partial_{i+2}\right)
=f_i\circ\partial_{i+1}\circ\partial_{i+2}=0$.
\item\label{last} Define $f_m$ for $m>i+2$ by recursion
such that $\partial_{m-i}\circ f_m
=\left(-1\right)^i f_{m-1}\circ\partial_m$.
This is possible by projectivity of $P_m$ since
\[\mathrm{im}\left(\left(-1\right)^i f_{m-1}\circ\partial_m\right)
\le\ker\left(\partial_{m-i-1}\right)
=\mathrm{im}\left(\partial_{m-i}\right)\]
as $\partial_{m-i-1}\circ\left(\left(-1\right)^i
f_{m-1}\circ\partial_m\right)
=f_{m-2}\circ\partial_{m-1}\circ\partial_m=0$.
\setcounter{savedenumii}{\value{enumii}}
\end{enumerate}
This completes the definition of $\Phi$.
The maps $\left\{f_m:m\ge i\right\}$
defined in Steps \ref{firstsecond}-\ref{last} above satisfy
\[\partial_{m-i}\circ f_m=\left(-1\right)^i f_{m-1}\partial_m.\]
In other words, $\left(\partial \Phi\left(f\right)\right)_{m+1}=0$ for all $m\ge i+1$
so that $\partial\Phi\left(f\right)=0$.
Thus, $\Phi\left(f\right)$ is a cocycle by construction.

\item\label{otherf}
We claim than any other choice of maps
$\left\{f'_m:m\ge i\right\}$ 
satisfying the conditions in \ref{first}-\ref{last} above
will be equivalent to
$\left\{f_m:m\ge i\right\}$ in $H^i\left(Y\right)$.
More precisely, if $f$ and $f'$ both satisfy
conditions \ref{first}-\ref{last}, 
then will define a map $\theta\in Y^{i-1}$
such that $\partial\theta=f-f'$.
Write $g_m=f_m-f'_m$ for $m\ge i$.
\begin{equation}
\begin{xy}{\xymatrix{
P_m\ar[r]^{\partial_m}\ar[d]_{g_m}
&P_{m-1}\ar[r]^{\partial_{m-1}}\ar[d]_{g_{m-1}}
\ar[dl]_{\theta_{m-1}}
&P_{m-2}\ar[r]^{\partial_{m-2}}\ar[d]_{g_{m-2}}
\ar[dl]_{\theta_{m-2}}
&P_{m-3}\ar[r]\ar[d]_{g_{m-3}}
\ar[dl]_{\theta_{m-3}}
&\dots\ar[r]
&P_{i+2}\ar[r]^{\partial_{i+2}}\ar[d]_{g_{i+2}}
&P_{i+1}\ar[r]^{\partial_{i+1}}\ar[d]_{g_{i+1}}
\ar[dl]_{\theta_{i+1}}
&P_i\ar[d]_{g_i}\ar[dl]_{\theta_i}\ar[r]^{\partial_{i}}
&P_{i-1}\ar[dl]_{\theta_{i-1}}
\\
P_{m-i}\ar[r]_{\partial_{m-i}}
&P_{m-i-1}\ar[r]_{\partial_{m-i-1}}
&P_{m-i-2}\ar[r]\ar[r]_{\partial_{m-i-2}}
&P_{m-i-3}\ar[r]
&\dots\ar[r]
&P_2\ar[r]_{\partial_2}
&P_1\ar[r]_{\partial_1}
&P_0\ar[r]_\epsilon
&k
}};
"2,8";"1,7"**{};
?(.35)*{\circlearrowright}
\end{xy}
\end{equation}
\begin{enumerate}
\setcounter{enumii}{\value{savedenumii}}
\item\label{secondfirst} Take $\theta_{i-1}=0$.
\item\label{secondsecond}
Since $\epsilon\circ f_i=\epsilon\circ f'_i=f$, we have
$\mathrm{im}\left(g_i\right)\le
\ker\left(\epsilon\right)=\mathrm{im}\left(\partial_1\right)$.
Define $\theta_i$ such that $\partial_1\circ\theta_i=\left(-1\right)^i g_i$.
This is possible by projectivity of $P_i$.
We rewrite the condition on $\theta_i$ for future reference as follows.
\begin{equation}\label{t1}
\left(\partial\theta\right)_i=
0-\left(-1\right)^{i-1}\partial_1\circ\theta_i=g_i
\end{equation}
\item By (\ref{secondsecond}), we have
\[\partial_1\circ\theta_i\circ\partial_{i+1}
=\left(-1\right)^i g_i\circ\partial_{i+1}
=\partial_1\circ g_{i+1}
\] so that 
\[\mathrm{im}\left(g_{i+1}-\theta_i\circ\partial_{i+1}\right)
\le\ker\left(\partial_1\right)=\mathrm{im}\left(\partial_2\right).\]
Define $\theta_{i+1}$
such that \[\partial_2\circ\theta_{i+1}
=\left(-1\right)^i \left(g_{i+1}-\theta_i\circ\partial_{i+1}\right),\]
and again, for future reference, we rewrite this as follows.
\begin{equation}\label{t2}
\left(\partial\theta\right)_{i+1}=\theta_i\circ\partial_{i+1}
-\left(-1\right)^{i-1}\partial_2\circ\theta_{i+1}=g_{i+1}
\end{equation}
\item Assume by recursion that we have computed $\theta_{m-2}$
and $\theta_{m-3}$ such that \[\partial_{m-i-1}\circ\theta_{m-2}
=\left(-1\right)^i\left(g_{m-2}-\theta_{m-3}\circ\partial_{m-2}\right).\]
Then $\partial_{m-i-1}\circ\theta_{m-2}\circ\partial_{m-1}
=\left(-1\right)^i g_{m-2}\circ\partial_{m-1}
=\partial_{m-i-1}\circ g_{m-1}$ so that
\[\mathrm{im}\left(g_{m-1}-\theta_{m-2}\circ\partial_{m-1}\right)
\le\ker\left(\partial_{m-i-1}\right)=\mathrm{im}\left(\partial_{m-i}\right).\]
Define $\theta_{m-1}$
such that \[\partial_{m-i}\circ\theta_{m-1}
=\left(-1\right)^i\left(g_{m-1}-\theta_{m-2}\circ\partial_{m-1}\right),\]
and again, for future reference, we rewrite this as follows.
\begin{equation}\label{tm}
\left(\partial\theta\right)_{m-1}=\theta_{m-2}\circ\partial_{m-1}
-\left(-1\right)^{i-1}\partial_{m-i}\circ\theta_{m-1}=g_{m-1}
\end{equation}
\setcounter{savedenumii}{\value{enumii}}
\end{enumerate}
This completes the definition of $\theta$. Then $\theta$ satisfies
$\partial\theta=f-f'$ by (\ref{t1}), (\ref{t2}), and (\ref{tm}).

\item\label{dg}
Suppose now that $f=\partial g$ for some cochain $g:P_{i-1}\to k$.
Write $\Phi\left(g\circ\partial_i\right)=\left\{
g_m:m\ge i\right\}$. We will construct $\theta$ such
that $\Phi\left(\partial g\right)=
\partial\theta$ for some $\theta\in Y^{i-1}$ as in the
following diagram.
\[\begin{xy}{\xymatrix@+5pt{
P_{m+1}\ar[r]^{\partial_{m+1}}\ar[d]_{g_{m+1}}
&P_m\ar[r]^{\partial_m}\ar[d]_{g_m}\ar[dl]_{\theta_m}
&P_{m-1}\ar[r]^{\partial_{m-1}}
\ar[d]_{g_{m-1}}\ar[dl]_{\theta_{m-1}}
&P_{m-2}\ar[r]\ar[d]_{g_{m-2}}\ar[dl]_{\theta_{m-2}}
&\dots\ar[r]
&P_{i+1}\ar[r]^{\partial_{i+1}}\ar[d]_{g_{i+1}}
&P_i\ar[d]_{g_i}\ar[r]^{\partial_i}\ar[dl]_{\theta_i}
&P_{i-1}\ar[d]_g\ar[dl]_{\theta_{i-1}}
\\
P_{m-i+1}\ar[r]_{\partial_{m-i+1}}
&P_{m-i}\ar[r]_{\partial_{m-i}}
&P_{m-i-1}\ar[r]_{\partial_{m-i-1}}
&P_{m-i-2}\ar[r]
&\dots\ar[r]
&P_1\ar[r]_{\partial_1}
&P_0\ar[r]_{\epsilon}
&k}};
"2,8";"1,7"**{};
?(.35)*{\circlearrowright}
\end{xy}\]
\begin{enumerate}
\setcounter{enumii}{\value{savedenumii}}
\item Define $\theta_{i-1}$ such that $\epsilon\circ\theta_{i-1}=g$.
This is possible by projectivity of $P_{i-1}$.
\item Since $\epsilon\circ\theta_{i-1}\circ\partial_i
=g\circ\partial_i=\epsilon\circ g_i$, we have that
\[\mathrm{im}\left(g_i-\theta_{i-1}\circ\partial_i\right)
\le\ker\left(\epsilon\right) =\mathrm{im}\left(\partial_1\right).\]
Thus, by projectivity of $P_i$, we have $\theta_i$ such that
\[\partial_1\circ\theta_i
=\left(-1\right)^i\left(g_i-\theta_{i-1}\circ\partial_i\right).\]
Then
\[\left(\partial\theta\right)_i=
\theta_{i-1}\circ\partial_i-\left(-1\right)^{i-1}\partial_1\circ\theta_i=g_i.\]
\item Assume by recursion that we have computed the maps
$\theta_{m-1}$ and $\theta_{m-2}$ such that
\[\theta_{m-2}\circ\partial_{m-1}
-\left(-1\right)^{i-1}\partial_{m-i}\circ\theta_{m-1}
= g_{m-1}.\]
Then
\[\partial_{m-i}\circ g_m=\left(-1\right)^i
g_{m-1}\circ\partial_m
=\partial_{m-i}\circ\theta_{m-1}\circ\partial_m\]
so that
\[\mathrm{im}\left(g_m-\theta_{m-1}\circ\partial_m\right)
\le\ker\left(\partial_{m-i}\right)=\mathrm{im}
\left(\partial_{m-i+1}\right).\] 
Define $\theta_m$ such that
\[\partial_{m-i-1}\circ\theta_m=
\left(-1\right)^i\left(g_m-\theta_{m-1}\circ\partial_m\right).\]
Then
\[\left(\partial\theta\right)_m=
\theta_{m-1}\circ\partial_m-\left(-1\right)^{i-1}
\partial_{m-i+1}\circ\theta_m=g_m.\]
\end{enumerate}
This completes the definition of $\theta$.
Then $g=\partial\theta$ by construction.

\item\label{hom}
We will now show that $\Phi$ is a $k$-module homomorphism.
Let $f,g:P_i\to k$ be cocycles and let $\alpha,\beta\in k$.
Write $h=\alpha f+\beta g$. We want to show that 
$\Phi\left(h\right)=\alpha\Phi\left(f\right)+\beta\Phi\left(g\right)$.
But $\epsilon\circ h_0=\epsilon\circ\left(\alpha f_0+\beta g_0\right)
=\alpha f+\beta g$, so that we are in the situation
of Step \ref{otherf} above. Thus, 
$\Phi\left(h\right)$ and 
$\alpha\Phi\left(f\right)+\beta\Phi\left(g\right)$
are equivalent elements of $Y$.

\item By Steps \ref{dg} and \ref{hom}, we have that
if $f$ and $f'$ are equivalent in $H^\ast\left(G,k\right)$,
then $\Phi\left(f\right)$ and $\Phi\left(f'\right)$
are equivalent in $H^\ast\left(Y\right)$.
This together with \ref{otherf} shows that $\Phi$ is a 
well-defined $k$-module homomorphism.

\item Finally, $\Phi$ is a bijection, having inverse given by
\[\left\{f_m:m\ge i\right\}\mapsto\epsilon\circ f_i.\]
\end{enumerate}
\end{proof}

\section{Products in $Y$}
In this section we define a product 
$Y^i\otimes Y^j\to Y^{i+j}$ on $Y$.
If $f\in Y^i$ and $g\in Y^j$ then let $fg$ be 
element of $Y^{i+j}$ whose component maps are
the compositions of the component
maps of $f$ with those of $g$ such that legitimate
compositions are obtained, as in the following diagram.
\begin{equation}\label{comp}\xymatrix{
P_n\ar[r]^{\partial_n}\ar[d]^{f_n}
&P_{n-1}\ar[d]^{f_{n-1}}\ar[r]
&\dots\ar[r]
&P_m\ar[r]^{\partial_m}\ar[d]^{f_m}
&P_{m-1}\ar[r]\ar[d]^{f_{m-1}}
&\dots\ar[r]
&P_{i+j+1}\ar[r]^{\partial_{i+j+1}}\ar[d]^{f_{i+j+1}}
&P_{i+j}\ar[d]^{f_{i+j}}
\\
P_{n-i}\ar[r]^{\partial_{n-i}}\ar[d]^{g_{n-i}}
&P_{n-i-1}\ar[r]\ar[d]^{g_{n-i-1}}
&\dots\ar[r]
&P_{m-i}\ar[r]^{\partial_{m-i}}\ar[d]^{g_{m-i}}
&P_{m-i-1}\ar[r]\ar[d]^{g_{m-i-1}}
&\dots\ar[r]
&P_{j+1}\ar[r]^{\partial_{j+1}}\ar[d]^{g_{j+1}}
&P_j\ar[d]^{g_j}
\\
P_{n-i-j}\ar[r]^{\partial_{n-i-j}}
&P_{n-i-j-1}\ar[r]
&\dots\ar[r]
&P_{m-i-j}\ar[r]^{\partial_{m-i-j}}
&P_{m-i-j-1}\ar[r]
&\dots\ar[r]
&P_1\ar[r]^{\partial_1}
&P_0
}\end{equation}
Observe that we have thrown away the maps 
$\left\{f_m:i\le m\le i+j-1\right\}$.
I suppose that the natural symbol 
for the object in (\ref{comp}) 
would be $g\circ f$, to emphasize
the fact that we're talking about
the component-wise composition of two elements of $Y$
and {\em not} a cohomology product.

\begin{observation}\label{differential}
$\displaystyle{\partial\left(g\circ f\right)
=g\circ\partial f
+\left(-1\right)^{\deg f}\partial g\circ f}$.
\end{observation}
\begin{proof} Write $i=\deg\left(f\right)$
and $j=\deg\left(g\right)$ as in (\ref{comp}). 
We will show the claim at the point $P_m$ in (\ref{comp})
for $m\ge i+j+1=\deg\left(\partial\left(g\circ f\right)\right)$.
Follow along in the picture.
\begin{eqnarray*}
\left(g\circ\partial f+\left(-1\right)^i 
\partial g\circ f\right)_m
&=&
g_{m-i-1}\circ
\left(f_{m-1}\circ\partial_m
-\left(-1\right)^i\partial_{m-i}\circ f_m
\right)\\
&&+\left(-1\right)^i\left(
g_{m-i-1}\circ\partial_{m-i}
-\left(-1\right)^j
\partial_{m-i-j}\circ g_{m-i}\right)
\circ f_m\\
&=&g_{m-i-1}\circ f_{m-1}\circ\partial_m
-\left(-1\right)^{i+j}
\partial_{m-i-j}\circ g_{m-i}\circ f_m\\
&=&\left(\partial\left(g\circ f\right)
\right)_m
\end{eqnarray*}
\end{proof}
\begin{claim}
Composition in $Y$ induces via $\Phi$ an associative binary operation 
\[H^i\left(G,k\right)
\otimes H^j\left(G,k\right)
\to H^{i+j}\left(G,k\right)\]
making $H^\ast\left(G,k\right)$ into a ring with 1.
\end{claim}

\section{Relationships among products on $H^\ast\left(G,k\right)$}
Let $f\in H^i\left(G,k\right)$ 
and $g\in H^j\left(G,k\right)$.
Consider the following products on $H^\ast\left(G,k\right)$.
\begin{enumerate}
\item The {\em cup product}\label{cup} $fg$ defined in Section \ref{cp}
\item\label{yp} The product induced from composition in $Y$
\[\left(f,g\right)\stackrel{\Phi}{\mapsto}
\big(\Phi\left(f\right),\Phi\left(g\right)\big)
\stackrel{\circ}{\mapsto}
\Phi\left(g\right)\circ\Phi\left(f\right)
\stackrel{\Phi^{-1}}{\mapsto}
\epsilon\circ\big(\Phi\left(g\right)\circ\Phi\left(f\right)\big)_{i+j} \]
\item The {\em Massey 2-fold product} $\left\langle f,g\right\rangle$,
defined more generally in Section \ref{mp} below,
\[\left(f,g\right)\stackrel{\Phi}{\mapsto}
\big(\Phi\left(f\right),\Phi\left(g\right)\big)
\stackrel{\left\langle\cdot\right\rangle}{\mapsto}
\left(-1\right)^i \Phi\left(g\right)\circ\Phi\left(f\right)
\stackrel{\Phi^{-1}}{\mapsto}
\left(-1\right)^i \epsilon\circ
\big(\Phi\left(g\right)\circ\Phi\left(f\right)\big)_{i+j} \]
\end{enumerate}
The cup product is calculated as $g\circ f_{i+j}$, where $f_{i+j}$
is as in (\ref{ff}), whereas product \ref{yp} 
is calculated as $g\circ f_{i+j}$, where $f_{i+j}$ is as
in (\ref{pff}). Comparing (\ref{ff}) and (\ref{pff}), we see
that the two $f_i$'s are the same, the $f_{i+1}$'s differ by
$\left(-1\right)^i$, the $f_{i+2}$'s differ by $\left(-1\right)^{2i}$,
and in general, the $f_{i+m}$'s differ
by $\left(-1\right)^{im}$. Thus, products \ref{cup} and \ref{yp}
differ by $\left(-1\right)^{ij}$, that is,
\begin{align*}
&\Phi^{-1}\big(\Phi\left(g\right)\circ\Phi\left(f\right)\big)=
\left(-1\right)^{ij} fg\\
\intertext{so that}
&\Phi\left(fg\right)=\left(-1\right)^{ij}\Phi\left(g\right)\circ
\Phi\left(f\right)
\intertext{and therefore}
&\Phi\left(fg\right)=\left(-1\right)^{i\left(j+1\right)}
\left\langle f,g\right\rangle.
\end{align*}
We observe that product \ref{cup} is associative
(see \cite{carlson}), and that product \ref{yp}
is also associative, consisting of composition of functions.
The Massey product, however, is not associative
in general.

\section{Massey Products}\label{mp}
The idea of the Massey product is to extend the cohomology
product to an $n$-fold product for $n\ge 2$.
The following definition is adapted from \cite{kraines}.

\begin{definition}\label{massey}
For $k\ge 2$ let 
$f^{\left(1\right)},
f^{\left(2\right)},\dots,f^{\left(k\right)}$ be cocycles in $Y$.
The {\em Massey $k$-fold product} 
$\left\langle f^{\left(1\right)},
f^{\left(2\right)},\dots,f^{\left(k\right)}\right\rangle$
is defined provided that for each pair $\left(i,j\right)$
with $1\le i<j\le k$ other than $\left(1,k\right)$
the lower-degree product 
$\left\langle f^{\left(i\right)},
f^{\left(i+1\right)},\dots,f^{\left(j\right)}\right\rangle$
is defined and vanishes as an element of $H^\ast\left(Y\right)$.
That is, if for each qualifying $\left(i,j\right)$
there exists $u^{i,j}\in Y$ such that 
$\partial u^{i,j}
=\left\langle f^{\left(i\right)},
f^{\left(i+1\right)},\dots,f^{\left(j\right)}\right\rangle$.
In this situation, the value of
$\left\langle f^{\left(1\right)},
f^{\left(2\right)},\dots,f^{\left(k\right)}\right\rangle$
is defined to be
\[\sum_{t=1}^{k-1}
u^{t+1,k}\circ
\overline{u^{1,t}}
\]
where the symbols $u^{1,1}$ and $u^{k,k}$ are taken
to be $f^{\left(1\right)}$ and 
$f^{\left(k\right)}$ respectively
and $\overline{u}=\left(-1\right)^{\deg\left(u\right)}u$.
\end{definition}

Observe that in the case $k=2$ the condition
on $\left(i,j\right)$ is vacuously satisfied,
so that $\left\langle f,g\right\rangle=g\circ\overline{f}$.
Traditionally, one organizes the information
in Definition \ref{massey} 
into the array like
\[\begin{array}{cccc}
f^{\left(1\right)}&u^{1,2}&u^{1,3}\\
&f^{\left(2\right)}&u^{2,3}&u^{2,4}\\
&&f^{\left(3\right)}&u^{3,4}\\
&&&f^{\left(4\right)}\end{array}\]
and traces the top row with one hand
while tracing the rightmost column with
the other hand as $t$ runs from 1 to $k-1$.
In the example shown above we have
\[\left\langle f^{\left(1\right)},f^{\left(2\right)},
f^{\left(3\right)},f^{\left(4\right)}\right\rangle
=u^{2,4}\circ \overline{f^{\left(1\right)}}
+ u^{3,4}\circ \overline{u^{1,2}} 
+ f^{\left(4\right)}\circ \overline{u^{1,3}}.\]

\begin{lemma}\label{cocycle}
$\left\langle f^{\left(1\right)},
f^{\left(2\right)},\dots,f^{\left(k\right)}\right\rangle$
is a cocycle in $Y$.
\end{lemma}
The reason for the sign appearing 
in Definition \ref{massey} becomes apparent
in the following proof.
\begin{proof}
We begin by making a general observation about $Y$.
Suppose $f\in Y^i$ and that $g=\partial\theta$
for some $\theta\in Y^{j-1}$ as in the following diagram.
\[\xymatrix@+15pt{
&P_{i+j+m+1}\ar[r]^{\partial_{i+j+m+1}}\ar[d]^{f_{i+j+m+1}}
&P_{i+j+m}\ar[d]^{f_{i+j+m}}
\\
&P_{j+m+1}\ar[r]^{\partial_{j+m+1}}
\ar[d]^{g_{j+m+1}}\ar[dl]^{\theta_{j+m+1}}
&P_{j+m}\ar[d]^{g_{j+m}}\ar[dl]^{\theta_{j+m}}
\\
P_{m+2}\ar[r]_{\partial_{m+2}}
&P_{m+1}\ar[r]_{\partial_{m+1}}
&P_m
}\]
Then by Observation \ref{differential} we have
\begin{eqnarray*}
\left(g\circ f\right)_{i+j+m+1}
&=&g_{j+m+1}\circ f_{i+j+m+1}\\
&=&\theta_{j+m}\circ\partial_{j+m+1}\circ f_{i+j+m+1}
-\left(-1\right)^{j-1}
\partial_{m+2}\circ\theta_{j+m+1}\circ f_{i+j+m+1}\\
&=&\theta_{j+m}\circ\partial_{j+m+1}\circ f_{i+j+m+1}
-\left(-1\right)^{j-1}
\partial_{m+2}\circ\theta_{j+m+1}\circ f_{i+j+m+1}\\
&&
-\left(-1\right)^i\theta_{j+m}\circ f_{i+j+m}\circ\partial_{i+j+m+1}
+\left(-1\right)^i\theta_{j+m}\circ f_{i+j+m}\circ\partial_{i+j+m+1}\\
&=&-\left(-1\right)^i
\left(\theta\circ\left(\partial f\right)\right)_{i+j+m+1} 
+\left(-1\right)^i \partial\left(\theta\circ f\right)_{i+j+m+1}
\end{eqnarray*}
so that as elements of $H^\ast\left(Y\right)$ we have
\begin{equation}
\partial\theta\circ f
=-\left(-1\right)^i\theta\circ\partial f.\end{equation}
Now we compute the derivative of
$\left\langle f^{\left(1\right)},
f^{\left(2\right)},\dots,f^{\left(k\right)}\right\rangle$.
\begin{eqnarray*}
\partial\left(\sum_{t=1}^{k-1}
\left(-1\right)^{\left(\deg u^{1,t}\right)}
u^{t+1,k}\circ u^{1,t}\right)
&=&\sum_{t=1}^{k-1}\left(
\left(-1\right)^{\left(\deg u^{1,t}\right)}
u^{t+1,k}\circ
\partial u^{1,t}
+\partial u^{t+1,k}\circ
u^{1,t}
\right)\\
&=&\sum_{t=1}^{k-1}\left(
-\partial u^{t+1,k}\circ
u^{1,t}
+\partial u^{t+1,k}\circ
u^{1,t}
\right)\\
&=&0
\end{eqnarray*}
\end{proof}

\begin{observation}\label{degree}
The  condition $\partial u^{i,j}
=\left\langle f^{\left(i\right)},
f^{\left(i+1\right)},\dots,f^{\left(j\right)}\right\rangle$ forces
\begin{align*}
\deg\left(u^{i,j}\right)
&=\sum_{t=i}^j \deg\left(f^{\left(t\right)}\right) +i-j\\
\text{and}\qquad\deg\left\langle f^{\left(i\right)},
f^{\left(i+1\right)},\dots,f^{\left(j\right)}\right\rangle
&=\sum_{t=i}^j \deg\left(f^{\left(t\right)}\right) +i-j+1.
\end{align*}
\end{observation}

\begin{trouble}
$\left\langle f^{\left(1\right)},
f^{\left(2\right)},\dots,f^{\left(k\right)}\right\rangle$
is not uniquely defined, unless for each $\left(i,j\right)$
the condition
$\partial u^{i,j}=\left\langle f^{\left(i\right)},
f^{\left(i+1\right)},\dots,f^{\left(j\right)}\right\rangle$
is satisfied by exactly one cochain $u^{i,j}$.
\end{trouble}

Suppose that we are given cocycles
$f^{\left(1\right)}, f^{\left(2\right)},\dots,f^{\left(k\right)}$
and we want to compute the 
map $u^{i,j}$ for some $\left(i,j\right)$
with $1\le i<j\le k$ other than $\left(1,k\right)$.
Assume that recursively,
we have computed all of the maps in the following array.
\[\begin{array}{ccccc}
f^{\left(i\right)}&u^{i,i+1}
&\dots&u^{i,j-1}&\\
&f^{\left(i+1\right)}&&u^{i+1,j-1}
&u^{i+1,j}\\
&&&&\vdots\\
&&&f^{\left(j-1\right)}&u^{j-1,j}\\
&&&&f^{\left(j\right)}
\end{array}\]
The map $u^{i,j}$ will be such that
\begin{equation}\label{ij}
\partial u^{i,j}=
\left\langle f^{\left(i\right)}, f^{\left(i+1\right)},\dots, 
f^{\left(j\right)}\right\rangle
=\sum_{t=i}^{j-1}
u^{t+1,j}\circ
\overline{u^{i,t}}
\end{equation}
where $u^{i,i}=f^{\left(i\right)}$
and $u^{j,j}=f^{\left(j\right)}$.
Write $g$ for the map on the right-hand side
of (\ref{ij}).
Write \[d=\deg\left(g\right)
=\sum_{t=i}^j\deg\left(f^{\left(t\right)}\right)+i-j+1.\]
The relevant maps are all pictured below.

\begin{multline*}\label{masseypic}
\begin{xy}\xymatrix@+13pt{
P_m\ar[r]^{\partial_m}\ar[d]_{g_m}
&P_{m-1}\ar[r]^{\partial_{m-1}}\ar[d]_{g_{m-1}}\ar[dl]_{u^{i,j}_{m-1}}
&P_{m-2}\ar[r]^{\partial_{m-2}}\ar[d]_{g_{m-2}}\ar[dl]_{u^{i,j}_{m-2}}
&P_{m-3}\ar[r]\ar[d]_{g_{m-3}}\ar[dl]_{u^{i,j}_{m-3}}
&\dots\\
P_{m-d}\ar[r]_{\partial_{m-d}}
&P_{m-d-1}\ar[r]_{\partial_{m-d-1}}
&P_{m-d-2}\ar[r]_{\partial_{m-d-2}}
&P_{m-d-3}\ar[r]
&\dots
}
\end{xy}
\\
\begin{xy}{\xymatrix@+13pt{
\dots\ar[r]
&P_{d+2}\ar[r]^{\partial_{d+2}}\ar[d]_{g_{d+2}}
&P_{d+1}\ar[r]^{\partial_{d+1}}\ar[d]_{g_{d+1}}\ar[dl]_{u^{i,j}_{d+1}}
&P_d\ar[d]_{g_d}\ar[r]^{\partial_d}\ar[dl]_{u^{i,j}_d}
&P_{d-1}\ar[dl]_{u^{i,j}_{d-1}}\\
\dots\ar[r]
&P_2\ar[r]_{\partial_2}
&P_1\ar[r]_{\partial_1}
&P_0\ar[r]_\epsilon&k
}};
"2,4";"1,3"**{};
?(.35)*{\circlearrowright}
\end{xy}
\end{multline*}

We assume now that $P_\ast$ is {\em minimal}. In other words, we assume
that $\partial_m\left(P_m\right)\le\mathrm{Rad}\left(P_{m-1}\right)$
for all $m\ge 1$.
This implies that $\partial f=0$ for {\em any} cochain $f$,
that is, we have $\partial_{i+1}\circ f=0$ for any
$kG$-homomorphism $f:P_i\to k$.

The map $u^{i,j}\in Y^{d-1}$ is constructed as follows.
\begin{enumerate}
\item\label{zero} We take $u^{i,j}_{d-1}=0$. 
\item The assumption that
$\left\langle f^{\left(i\right)},
f^{\left(i+1\right)},\dots,f^{\left(j\right)}\right\rangle=g$ 
vanishes as an element of $H^d\left(Y\right)$
tells us that $\epsilon\circ g_d$ vanishes
as an element of $H^d\left(G,k\right)$.
But since $P_\ast$ is {\em minimal}, this means
that $\epsilon\circ g_d$ is actually the zero map.
Then by projectivity of $P_d$, there exists
$u^{i,j}_d$ such that $\partial_1\circ u^{i,j}_d=\left(-1\right)^d g_d$.
Observe that this  means
\[\left(\partial u^{i,j}\right)_d
=0-\left(-1\right)^{d-1}\partial_1\circ u^{i,j}_d=g_d.\]
\item The map $g$ is a cocycle by Lemma \ref{cocycle}. 
This means that the {\em rectangles} in (\ref{masseypic})
either commute or anticommute, depending on whether $d$
is even or odd. Thus,
\[\partial_1\circ
\left(g_{d+1}-u_d^{i,j}\circ\partial_{d+1}\right)
=\partial_1\circ g_{d+1}
-\left(-1\right)^d g_d\circ\partial_{d+1}=0\]
so that
\[\mathrm{im}\left(g_{d+1}-
u_d^{i,j}\circ\partial_{d+1}\right)\le\ker\left(\partial_1\right)
=\mathrm{im}\left(\partial_2\right).\]
Thus, there exists $u^{i,j}_{d+1}$ such that
\[\partial_2\circ u^{i,j}_{d+1}=
\left(-1\right)^d\left(g_{d+1}- u_d^{i,j}\circ\partial_{d+1}\right).\]
Observe that this means
\[\left(\partial u^{i,j}\right)_{d+1}
=u^{i,j}_d\circ\partial_{d+1}-\left(-1\right)^{d-1}
\partial_2\circ u^{i,j}_{d+1}=g_{d+1}.\]
\item Assume by recursion that we have constructed that maps $u^{i,j}_{m-2}$ and $u_{m-3}^{i,j}$ such that
\[\partial_{m-d-1}\circ u^{i,j}_{m-2}=
\left(-1\right)^d\left(g_{m-2}
-u_{m-3}^{i,j}\circ\partial_{m-2}\right).\] Thus
\[\partial_{m-d-1}\circ
\left(g_{m-1}-u_{m-2}^{i,j}\circ\partial_{m-1}\right)
=\partial_{m-d-1}\circ g_{m-1}
-\left(-1\right)^d g_{m-2}\circ\partial_{m-1}=0\]
so that
\[\mathrm{im}\left(g_{m-1}-
u_{m-2}^{i,j}\circ\partial_{m-1}\right)\le\ker\left(\partial_{m-d-1}\right)
=\mathrm{im}\left(\partial_{m-d}\right).\]
Thus, there exists $u^{i,j}_{m-1}$ such that
\[\partial_{m-d}\circ u^{i,j}_{m-1}=
\left(-1\right)^d\left(g_{m-1}- u_{m-2}^{i,j}\circ\partial_{m-1}\right).\]
Observe that this means
\[\left(\partial u^{i,j}\right)_{m-1}=
u^{i,j}_{m-2}\circ\partial_{m-1}-\left(-1\right)^{d-1}
\partial_{m-d}\circ u^{i,j}_{m-1}=g_{m-1}.\]
\end{enumerate}
This completes the construction of $u^{i,j}$.
By construction, we have $\partial\left(u^{i,j}\right)=g$.

Finally, observe that in the last step in the calculation of
$\left\langle f^{\left(1\right)}, f^{\left(2\right)},\dots, 
f^{\left(k\right)}\right\rangle$,
which is actually
the {\em first} step, as this is a recursive process,
it is only necessary to calculate $u^{1,k-1}$,
but none of the maps $u^{1,m}$ for $2\le m\le k-2$,
and none of the maps $u^{m,k}$ for $2\le m\le k-1$.
In effect, the sum
\[\sum_{t=1}^{k-1}
u^{t+1,k}\circ
\overline{u^{1,t}}
=\sum_{t=1}^{k-2}
u^{t+1,k}\circ
\overline{u^{1,t}}
+f^{\left(k\right)}\circ\overline{u^{1,k-1}}
\] 
appearing in Definition \ref{massey} 
is calculated as
\[ \sum_{t=1}^{k-2}
\stackrel{!}{\boxed{u^{t+1,k}
_{\deg u^{t+1,k}}}}
\circ
\overline{u^{1,t}_{\deg u^{t+1,k}+\deg u^{1,t}}}
+f^{\left(k\right)}_{\deg f^{\left(k\right)}}
\circ\overline{u^{1,k-1}_{\deg f^{\left(k\right)}+\deg u^{1,k-1}}},
\] 
But $u^{t+1,k}_{\deg u^{t+1,k}}=0$
by construction (see Step \ref{zero} above),
so the sum reduces to a single term.
This is not the case with the intermediate maps $u^{i,j}$
with $j-i\le k-2$.

\bibliographystyle{plain}
\bibliography{crimebib.xml}
\end{document}

