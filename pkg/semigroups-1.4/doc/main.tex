% generated by GAPDoc2LaTeX from XML source (Frank Luebeck)
\documentclass[a4paper,11pt]{report}

\usepackage{a4wide}
\sloppy
\pagestyle{myheadings}
\usepackage{amssymb}
\usepackage[latin1]{inputenc}
\usepackage{makeidx}
\makeindex
\usepackage{color}
\definecolor{FireBrick}{rgb}{0.5812,0.0074,0.0083}
\definecolor{RoyalBlue}{rgb}{0.0236,0.0894,0.6179}
\definecolor{RoyalGreen}{rgb}{0.0236,0.6179,0.0894}
\definecolor{RoyalRed}{rgb}{0.6179,0.0236,0.0894}
\definecolor{LightBlue}{rgb}{0.8544,0.9511,1.0000}
\definecolor{Black}{rgb}{0.0,0.0,0.0}

\definecolor{linkColor}{rgb}{0.0,0.0,0.554}
\definecolor{citeColor}{rgb}{0.0,0.0,0.554}
\definecolor{fileColor}{rgb}{0.0,0.0,0.554}
\definecolor{urlColor}{rgb}{0.0,0.0,0.554}
\definecolor{promptColor}{rgb}{0.0,0.0,0.589}
\definecolor{brkpromptColor}{rgb}{0.589,0.0,0.0}
\definecolor{gapinputColor}{rgb}{0.589,0.0,0.0}
\definecolor{gapoutputColor}{rgb}{0.0,0.0,0.0}

%%  for a long time these were red and blue by default,
%%  now black, but keep variables to overwrite
\definecolor{FuncColor}{rgb}{0.0,0.0,0.0}
%% strange name because of pdflatex bug:
\definecolor{Chapter }{rgb}{0.0,0.0,0.0}
\definecolor{DarkOlive}{rgb}{0.1047,0.2412,0.0064}


\usepackage{fancyvrb}

\usepackage{mathptmx,helvet}
\usepackage[T1]{fontenc}
\usepackage{textcomp}


\usepackage[
            pdftex=true,
            bookmarks=true,        
            a4paper=true,
            pdftitle={Written with GAPDoc},
            pdfcreator={LaTeX with hyperref package / GAPDoc},
            colorlinks=true,
            backref=page,
            breaklinks=true,
            linkcolor=linkColor,
            citecolor=citeColor,
            filecolor=fileColor,
            urlcolor=urlColor,
            pdfpagemode={UseNone}, 
           ]{hyperref}

\newcommand{\maintitlesize}{\fontsize{50}{55}\selectfont}

% write page numbers to a .pnr log file for online help
\newwrite\pagenrlog
\immediate\openout\pagenrlog =\jobname.pnr
\immediate\write\pagenrlog{PAGENRS := [}
\newcommand{\logpage}[1]{\protect\write\pagenrlog{#1, \thepage,}}
%% were never documented, give conflicts with some additional packages

\newcommand{\GAP}{\textsf{GAP}}

%% nicer description environments, allows long labels
\usepackage{enumitem}
\setdescription{style=nextline}

%% depth of toc
\setcounter{tocdepth}{1}





%% command for ColorPrompt style examples
\newcommand{\gapprompt}[1]{\color{promptColor}{\bfseries #1}}
\newcommand{\gapbrkprompt}[1]{\color{brkpromptColor}{\bfseries #1}}
\newcommand{\gapinput}[1]{\color{gapinputColor}{#1}}


\begin{document}

\logpage{[ 0, 0, 0 ]}
\begin{titlepage}
\mbox{}\vfill

\begin{center}{\maintitlesize \textbf{\textsf{Semigroups}\mbox{}}}\\
\vfill

\hypersetup{pdftitle=\textsf{Semigroups}}
\markright{\scriptsize \mbox{}\hfill \textsf{Semigroups} \hfill\mbox{}}
{\Huge \textbf{Method for semigroups\mbox{}}}\\
\vfill

{\Huge Version 1.4\mbox{}}\\[1cm]
\mbox{}\\[2cm]
{\Large \textbf{J. D. Mitchell   \mbox{}}}\\
\hypersetup{pdfauthor=J. D. Mitchell   }
\end{center}\vfill

\mbox{}\\
{\mbox{}\\
\small \noindent \textbf{J. D. Mitchell   }  Email: \href{mailto://jdm3@st-and.ac.uk} {\texttt{jdm3@st-and.ac.uk}}\\
  Homepage: \href{http://tinyurl.com/jdmitchell} {\texttt{http://tinyurl.com/jdmitchell}}}\\
\end{titlepage}

\newpage\setcounter{page}{2}
{\small 
\section*{Abstract}
\logpage{[ 0, 0, 2 ]}
 The \textsf{Semigroups} package is a \textsf{GAP} package containing methods for semigroups, principally semigroups of
transformations, partial permutations or subsemigroups of regular Rees
0-matrix semigroups. \textsf{Semigroups} contains more efficient methods than those available in the \textsf{GAP} library (and in many cases more efficient than any other software) for
creating semigroups, calculating their Green's classes, size, elements, group
of units, minimal ideal, small generating sets, testing membership, finding
the inverses of a regular element, factorizing elements over the generators,
and many more. It is also possible to test if a semigroup satisfies a
particular property, such as if it is regular, simple, inverse, completely
regular, and a variety of further properties. 

 There are also functions to define and manipulate free inverse semigroups and
their elements. \mbox{}}\\[1cm]
{\small 
\section*{Copyright}
\logpage{[ 0, 0, 1 ]}
{\copyright} 2011-13 by J. D. Mitchell.

 \textsf{Semigroups} is free software; you can redistribute it and/or modify it under the terms of
the \href{ http://www.fsf.org/licenses/gpl.html} {GNU General Public License} as published by the Free Software Foundation; either version 2 of the License,
or (at your option) any later version. \mbox{}}\\[1cm]
{\small 
\section*{Acknowledgements}
\logpage{[ 0, 0, 3 ]}
 I would like to thank P. von Bunau, A. Distler, A. Egri-Nagy, S. Linton, C.
Nehaniv, J. Neubueser, M. R. Quick, E. F. Robertson, and N. Ruskuc for their
help and suggestions. Special thanks go to J. Araujo for his mathematical
suggestions and to M. Neunhoeffer for his invaluable help in improving the
efficiency of the package. 

 I would also like to acknowledge the support of the Centre of Algebra at the
University of Lisbon, and of EPSRC grant number GR/S/56085/01. \mbox{}}\\[1cm]
\newpage

\def\contentsname{Contents\logpage{[ 0, 0, 4 ]}}

\tableofcontents
\newpage

 
\chapter{\textcolor{Chapter }{The \textsf{Semigroups} package}}\label{semigroups}
\logpage{[ 1, 0, 0 ]}
\hyperdef{L}{X7C913C76836AC46D}{}
{
  \index{Semigroups@\textsf{Semigroups} package overview} 
\section{\textcolor{Chapter }{An important note}}\logpage{[ 1, 1, 0 ]}
\hyperdef{L}{X816ACF99846DE485}{}
{
 The operations, methods, properties, and functions described in this manual
only apply to semigroups of transformations, partial permutations, and
subsemigroups of regular Rees 0-matrix semigroups over groups. For the sake of
brevity, we have opted to say \textsc{semigroup} rather than \textsc{semigroup of transformations, partial permutations, and subsemigroups of
regular Rees 0-matrix semigroups over groups}. }

 
\section{\textcolor{Chapter }{Introduction}}\logpage{[ 1, 2, 0 ]}
\hyperdef{L}{X7DFB63A97E67C0A1}{}
{
 This is the manual for the \textsf{Semigroups} package version 1.4. \textsf{Semigroups} 1.4 is an updated and expanded version of the \href{ http://schmidt.nuigalway.ie/monoid/index.html} {Monoid package for GAP 3} by Goetz Pfeiffer, Steve A. Linton, Edmund F. Robertson, and Nik Ruskuc and
the \href{ http://www-history.mcs.st-and.ac.uk/~jamesm/monoid/index.html} {Monoid package for GAP 4} by J. D. Mitchell.

 Some of the theory behind the algorithms in \textsf{Semigroups} is described in \cite{pfeiffer1} and described in \cite{pfeiffer2}. Another reference is \cite{lallement}. A reference for free inverse semigroups is Section 5.10 in \cite{howie}. The functions for: finding a smaller degree partial permutation
representation of an inverse semigroup are based on \cite{Schein1992aa}; and finding the maximal subsemigroups of a Rees matrix or 0-matrix semigroup
are based on Remark 1 in \cite{Graham1968aa}. 

 \textsf{Semigroups} 1.4 retains all the functionality of the original \textsf{Monoid} for \textsf{GAP} 3; and those functions from \textsf{Monoid} 3.1.4 not involved in the computation of automorphism groups of semigroups.

 \noindent \textsf{Semigroups} 1.4 contains more efficient methods than those available in the \textsf{GAP} library (and in many cases more efficient than any other software) for
creating semigroups, calculating their Green's classes, size, elements, group
of units, minimal ideal, and testing membership, finding the inverses of a
regular element, and factorizing elements over the generators, and many more;
see Chapters \ref{create} and \ref{green}. There are also methods for testing if a semigroup satisfies a particular
property, such as if it is regular, simple, inverse, completely regular, and a
variety of further properties; see Chapter \ref{green}. Several standard examples of semigroups are provided see Section \ref{Examples}. \textsf{Semigroups} also provides abbreviated names for some of the commonly used \textsf{GAP} library functions related to semigroups, and functions to read and write
collections of transformations or partial permutations to a file; see \texttt{ReadGenerators} (\ref{ReadGenerators}) and \texttt{WriteGenerators} (\ref{WriteGenerators}). 

 There are also functions in \textsf{Semigroups} to define and manipulate free inverse semigroups and their elements; this part
of the package was written by Julius Jonu{\v s}as.

 \noindent The \textsf{Semigroups} package is written \textsf{GAP} code and requires the \href{ http://www-groups.mcs.st-and.ac.uk/~neunhoef/Computer/Software/Gap/orb.html } {Orb} and \href{ http://www-groups.mcs.st-and.ac.uk/~neunhoef/Computer/Software/Gap/io.html } {IO}packages. The \href{ http://www-groups.mcs.st-and.ac.uk/~neunhoef/Computer/Software/Gap/orb.html } {Orb} package is used to efficiently compute components of actions, which underpin
many of the features of \textsf{Semigroups}. The \href{ http://www-groups.mcs.st-and.ac.uk/~neunhoef/Computer/Software/Gap/io.html } {IO} package is used to read and write transformations and partial permutations to
a file. The \href{http://www.maths.qmul.ac.uk/~leonard/grape/} {Grape} package is used in the functions \texttt{MunnSemigroup} (\ref{MunnSemigroup}) and \texttt{MaximalSubsemigroups} (\ref{MaximalSubsemigroups}) but nowhere else in \textsf{Semigroups}. If \href{http://www.maths.qmul.ac.uk/~leonard/grape/} {Grape} is not available, then \textsf{Semigroups} can be used as normal with the exception that \texttt{MunnSemigroup} (\ref{MunnSemigroup}) and \texttt{MaximalSubsemigroups} (\ref{MaximalSubsemigroups}) will not work. 

 \textsf{Semigroups} is still under development, and so some features may not function as expected.
At the present time, I do not know of any errors or serious issues with \textsf{Semigroups}. If you find a bug or an issue with the package, then please let me know by
emailing me the details at: \href{mailto://jdm3@st-and.ac.uk} {\texttt{jdm3@st-and.ac.uk}}.

 For more details about semigroups in \textsf{GAP} or Green's relations in particular, see  (\textbf{Reference: Semigroups}) or  (\textbf{Reference: Green's Relations}). }

 
\section{\textcolor{Chapter }{Installing the \textsf{Semigroups} package}}\label{install}
\logpage{[ 1, 3, 0 ]}
\hyperdef{L}{X7BA677207FAC28B3}{}
{
  In this section we give a brief description of how to start using \textsf{Semigroups}. If you have any problems getting \textsf{Semigroups} working, then you could try emailing me at \href{mailto://jdm3@st-and.ac.uk} {\texttt{jdm3@st-and.ac.uk}}. 

 It is assumed that you have a working copy of \textsf{GAP} with version number 4.7.1 or higher. The most up-to-date version of \textsf{GAP} and instructions on how to install it can be obtained from the main \textsf{GAP} webpage \vspace{\baselineskip}

\noindent\vspace{\baselineskip} \href{http://www.gap-system.org} {\texttt{http://www.gap-system.org}}

 \noindent The following is a summary of the steps that should lead to a successful
installation of \textsf{Semigroups}: 
\begin{itemize}
\item  ensure that the \href{ http://www-groups.mcs.st-and.ac.uk/~neunhoef/Computer/Software/Gap/io.html } {IO} package version 4.1 or higher is available. For more details go to:\vspace{\baselineskip}

\noindent \href{http://www-groups.mcs.st-and.ac.uk/~neunhoef/Computer/Software/Gap/ io.html } {\texttt{http://www-groups.mcs.st-and.ac.uk/\texttt{\symbol{126}}neunhoef/Computer/Software/Gap/ io.html }} 
\item  ensure that the \href{ http://www-groups.mcs.st-and.ac.uk/~neunhoef/Computer/Software/Gap/orb.html } {Orb} package version 4.6 or higher is available. For more details go to: \vspace{\baselineskip}

\noindent \href{http://www-groups.mcs.st-and.ac.uk/~neunhoef/Computer/Software/Gap/orb.html } {\texttt{http://www-groups.mcs.st-and.ac.uk/\texttt{\symbol{126}}neunhoef/Computer/Software/Gap/orb.html }}

 \noindent Note that \href{ http://www-groups.mcs.st-and.ac.uk/~neunhoef/Computer/Software/Gap/orb.html } {Orb} and \textsf{Semigroups} both perform better if \href{ http://www-groups.mcs.st-and.ac.uk/~neunhoef/Computer/Software/Gap/orb.html } {Orb} is compiled. 
\item  if you want to be able to calculate the Munn semigroup of a semilattice or the
maximal subsemigroups of a Rees 0-matrix semigroup, then make sure that the \href{http://www.maths.qmul.ac.uk/~leonard/grape/} {Grape} package version 4.5 or higher is available. For more details go to:\vspace{\baselineskip}

\noindent \href{http://www.maths.qmul.ac.uk/~leonard/grape/} {\texttt{http://www.maths.qmul.ac.uk/\texttt{\symbol{126}}leonard/grape/}}

 If \href{http://www.maths.qmul.ac.uk/~leonard/grape/} {Grape} is not fully installed, then \textsf{Semigroups} can be used as normal with the exception that \texttt{MunnSemigroup} (\ref{MunnSemigroup}) will not work. The method for \texttt{MaximalSubsemigroups} (\ref{MaximalSubsemigroups}) will work provided \href{http://www.maths.qmul.ac.uk/~leonard/grape/} {Grape} can be loaded, i.e. it is not necessary for \href{http://www.maths.qmul.ac.uk/~leonard/grape/} {Grape} to be compiled. 
\item  download the package archive \texttt{semigroups-1.4.tar.gz} from \vspace{\baselineskip}

\noindent\vspace{\baselineskip} \href{http://www-groups.mcs.st-andrews.ac.uk/~jamesm/semigroups.php} {\texttt{http://www-groups.mcs.st-andrews.ac.uk/\texttt{\symbol{126}}jamesm/semigroups.php}} 
\item  unzip and untar the file, this should create a directory called \texttt{semigroups-1.4}.
\item  locate the \texttt{pkg} directory of your \textsf{GAP} directory, which contains the directories \texttt{lib}, \texttt{doc} and so on. Move the directory \texttt{semigroups-1.4} into the \texttt{pkg} directory. 
\item  start \textsf{GAP} in the usual way.
\item  type \texttt{LoadPackage("semigroups");}
\item  compile the documentation by using \texttt{SemigroupsMakeDoc} (\ref{SemigroupsMakeDoc}) 
\end{itemize}
  Presuming that the above steps can be completed successfully you will be
running the \textsf{Semigroups} package!

 If you want to check that the package is working correctly, you should run
some of the tests described in Section \ref{testing}.

 }

  
\section{\textcolor{Chapter }{Compiling the documentation}}\label{doc}
\logpage{[ 1, 4, 0 ]}
\hyperdef{L}{X7E61798C7D949C4E}{}
{
 To compile the documentation use \texttt{SemigroupsMakeDoc} (\ref{SemigroupsMakeDoc}). If you want to use the help system, it is essential that you compile the
documentation. 

\subsection{\textcolor{Chapter }{SemigroupsMakeDoc}}
\logpage{[ 1, 4, 1 ]}\nobreak
\hyperdef{L}{X838F4B0D780C2A3F}{}
{\noindent\textcolor{FuncColor}{$\triangleright$\ \ \texttt{SemigroupsMakeDoc({\mdseries\slshape })\index{SemigroupsMakeDoc@\texttt{SemigroupsMakeDoc}}
\label{SemigroupsMakeDoc}
}\hfill{\scriptsize (function)}}\\
\textbf{\indent Returns:\ }
Nothing.



 This function should be called with no argument to compile the \textsf{Semigroups} documentation. }

 }

 
\section{\textcolor{Chapter }{Testing the installation}}\label{testing}
\logpage{[ 1, 5, 0 ]}
\hyperdef{L}{X7AE7A7077F513655}{}
{
 In this section we describe how to test that \textsf{Semigroups} is working as intended. To test that \textsf{Semigroups} is installed correctly use \texttt{SemigroupsTestInstall} (\ref{SemigroupsTestInstall}) or for more extensive tests use \texttt{SemigroupsTestAll} (\ref{SemigroupsTestAll}). Please note that it will take a few seconds for \texttt{SemigroupsTestInstall} (\ref{SemigroupsTestInstall}) to finish and it may take several minutes for \texttt{SemigroupsTestAll} (\ref{SemigroupsTestAll}) to finish.

 Note that after calling \texttt{SemigroupsTestAll} (\ref{SemigroupsTestAll}), \texttt{SemigroupsTestAll} (\ref{SemigroupsTestAll}), or \texttt{SemigroupsTestManualExamples} (\ref{SemigroupsTestManualExamples}), the message \texttt{gzip: stdout: Broken pipe} might be displayed (several times). While this is unfortunate, it is not an
error and should simply be ignored. We hope to resolve this issue in the
future. If something goes wrong, then please review the instructions in
Section \ref{install} and ensure that \textsf{Semigroups} has been properly installed. If you continue having problems, please email me
at \href{mailto://jdm3@st-and.ac.uk} {\texttt{jdm3@st-and.ac.uk}}. 

\subsection{\textcolor{Chapter }{SemigroupsTestAll}}
\logpage{[ 1, 5, 1 ]}\nobreak
\hyperdef{L}{X8544F4BD79F0BF3C}{}
{\noindent\textcolor{FuncColor}{$\triangleright$\ \ \texttt{SemigroupsTestAll({\mdseries\slshape })\index{SemigroupsTestAll@\texttt{SemigroupsTestAll}}
\label{SemigroupsTestAll}
}\hfill{\scriptsize (function)}}\\
\textbf{\indent Returns:\ }
Nothing.



 This function should be called with no argument to comprehensively test that \textsf{Semigroups} is working correctly. These tests should take no more than a few minutes to
complete. To quickly test that \textsf{Semigroups} is installed correctly use \texttt{SemigroupsTestInstall} (\ref{SemigroupsTestInstall}). }

 

\subsection{\textcolor{Chapter }{SemigroupsTestInstall}}
\logpage{[ 1, 5, 2 ]}\nobreak
\hyperdef{L}{X80F85B577A3DFCF9}{}
{\noindent\textcolor{FuncColor}{$\triangleright$\ \ \texttt{SemigroupsTestInstall({\mdseries\slshape })\index{SemigroupsTestInstall@\texttt{SemigroupsTestInstall}}
\label{SemigroupsTestInstall}
}\hfill{\scriptsize (function)}}\\
\textbf{\indent Returns:\ }
Nothing.



 This function should be called with no argument to test your installation of \textsf{Semigroups} is working correctly. These tests should take no more than a fraction of a
second to complete. To more comprehensively test that \textsf{Semigroups} is installed correctly use \texttt{SemigroupsTestAll} (\ref{SemigroupsTestAll}). }

 

\subsection{\textcolor{Chapter }{SemigroupsTestManualExamples}}
\logpage{[ 1, 5, 3 ]}\nobreak
\hyperdef{L}{X7E6C196A78E8665C}{}
{\noindent\textcolor{FuncColor}{$\triangleright$\ \ \texttt{SemigroupsTestManualExamples({\mdseries\slshape })\index{SemigroupsTestManualExamples@\texttt{SemigroupsTestManualExamples}}
\label{SemigroupsTestManualExamples}
}\hfill{\scriptsize (function)}}\\
\textbf{\indent Returns:\ }
Nothing.



 This function should be called with no argument to test the examples in the \textsf{Semigroups} manual. These tests should take no more than a few minutes to complete. To
more comprehensively test that \textsf{Semigroups} is installed correctly use \texttt{SemigroupsTestAll} (\ref{SemigroupsTestAll}). See also \texttt{SemigroupsTestInstall} (\ref{SemigroupsTestInstall}). }

 }

 
\section{\textcolor{Chapter }{More information during a computation}}\logpage{[ 1, 6, 0 ]}
\hyperdef{L}{X798CBC46800AB80F}{}
{
 

\subsection{\textcolor{Chapter }{InfoSemigroups}}
\logpage{[ 1, 6, 1 ]}\nobreak
\hyperdef{L}{X85CD4E6C82BECAF3}{}
{\noindent\textcolor{FuncColor}{$\triangleright$\ \ \texttt{InfoSemigroups\index{InfoSemigroups@\texttt{InfoSemigroups}}
\label{InfoSemigroups}
}\hfill{\scriptsize (info class)}}\\


 \texttt{InfoSemigroups} is the info class of the \textsf{Semigroups} package. The info level is initially set to 0 and no info messages are
displayed. We recommend that you set the level to 1 so that basic info
messages are displayed. To increase the amount of information displayed during
a computation increase the info level to 2 or 3. To stop all info messages
from being displayed, set the info level to 0. See also  (\textbf{Reference: Info Functions}) and \texttt{SetInfoLevel} (\textbf{Reference: SetInfoLevel}). }

 }

 
\section{\textcolor{Chapter }{Reading and writing transformations and partial permutations to a file}}\logpage{[ 1, 7, 0 ]}
\hyperdef{L}{X875277447E3DF377}{}
{
 The functions \texttt{ReadGenerators} (\ref{ReadGenerators}) and \texttt{WriteGenerators} (\ref{WriteGenerators}) can be used to read or write transformations or partial permutations to a
file. 

\subsection{\textcolor{Chapter }{SemigroupsDir}}
\logpage{[ 1, 7, 1 ]}\nobreak
\hyperdef{L}{X7BCF050D7C9CB451}{}
{\noindent\textcolor{FuncColor}{$\triangleright$\ \ \texttt{SemigroupsDir({\mdseries\slshape })\index{SemigroupsDir@\texttt{SemigroupsDir}}
\label{SemigroupsDir}
}\hfill{\scriptsize (function)}}\\
\textbf{\indent Returns:\ }
A string.



 This function returns the absolute path to the \textsf{Semigroups} package directory as a string. The same result can be obtained typing: 
\begin{Verbatim}[commandchars=@|A,fontsize=\small,frame=single,label=Example]
  PackageInfo("semigroups")[1]!.InstallationPath;
\end{Verbatim}
 at the \textsf{GAP} prompt. }

 

\subsection{\textcolor{Chapter }{ReadGenerators}}
\logpage{[ 1, 7, 2 ]}\nobreak
\hyperdef{L}{X8728096E8427EDE8}{}
{\noindent\textcolor{FuncColor}{$\triangleright$\ \ \texttt{ReadGenerators({\mdseries\slshape filename[, nr]})\index{ReadGenerators@\texttt{ReadGenerators}}
\label{ReadGenerators}
}\hfill{\scriptsize (function)}}\\
\textbf{\indent Returns:\ }
A list of lists of semigroup elements.



 If \mbox{\texttt{\mdseries\slshape filename}} is the name of a file created using \texttt{WriteGenerators} (\ref{WriteGenerators}), then \texttt{ReadGenerators} returns the contents of this file as a list of lists of transformations, or
partial permutations. 

 If the optional second argument \mbox{\texttt{\mdseries\slshape nr}} is present, then \texttt{ReadGenerators} returns the elements stored in the \mbox{\texttt{\mdseries\slshape nr}}th line of \mbox{\texttt{\mdseries\slshape filename}}. 

 Note that if the file \mbox{\texttt{\mdseries\slshape filename}} is gzipped, then the message \texttt{gzip: stdout: Broken pipe} might be displayed when \texttt{ReadGenerators} is used. While this is unfortunate, it is not an error that affects the output
of \texttt{ReadGenerators}. We hope to resolve this issue in the future. }

 

\subsection{\textcolor{Chapter }{WriteGenerators}}
\logpage{[ 1, 7, 3 ]}\nobreak
\hyperdef{L}{X78041E8F87EFDE62}{}
{\noindent\textcolor{FuncColor}{$\triangleright$\ \ \texttt{WriteGenerators({\mdseries\slshape filename, list[, append]})\index{WriteGenerators@\texttt{WriteGenerators}}
\label{WriteGenerators}
}\hfill{\scriptsize (function)}}\\
\textbf{\indent Returns:\ }
\texttt{true} or \texttt{fail}.



 This function provides a method for writing transformations and partial
permutations to a file, that uses a relatively small amount of disk space. The
resulting file can be further compressed using \texttt{gzip} or \texttt{xz}.

 The argument \mbox{\texttt{\mdseries\slshape list}} should be a list of elements, a semigroup, or a list of lists of elements, or
semigroups; or an \textsf{IO} file object. The types of elements and semigroups supported are:
transformations and partial permutations.

 The argument \mbox{\texttt{\mdseries\slshape filename}} should be a string containing the name of a file where the entries in \mbox{\texttt{\mdseries\slshape list}} will be written. 

 If the optional third argument \mbox{\texttt{\mdseries\slshape append}} is given and equals \texttt{"w"}, then the previous content of the file is deleted. If the optional third
argument is \texttt{"a"} or is not present, then \texttt{list} is appended to the file. This function returns \texttt{true} if everything went well or \texttt{fail} if something went wrong.

 \texttt{WriteGenerators} appends a line to the file \mbox{\texttt{\mdseries\slshape filename}} for every entry in \mbox{\texttt{\mdseries\slshape list}}. If any element of \mbox{\texttt{\mdseries\slshape list}} is a semigroup, then the generators of that semigroup are written to \mbox{\texttt{\mdseries\slshape filename}}. 

 The first character of the appended line indicates which type of element is
contained in that line, the second character \texttt{m} is the number of characters in the degree of the elements to be written, the
next \texttt{m} characters are the degree \texttt{n} of the elements to be written, and the internal representation of the elements
themselves are written in blocks of \texttt{m*n} in the remainder of the line. For example, the transformations: 
\begin{Verbatim}[commandchars=!@|,fontsize=\small,frame=single,label=Example]
  [ Transformation( [ 2, 6, 7, 2, 6, 9, 9, 1, 1, 5 ] ), 
    Transformation( [ 3, 8, 1, 9, 9, 4, 10, 5, 10, 6 ] )]
\end{Verbatim}
 are written as: 
\begin{Verbatim}[commandchars=!@|,fontsize=\small,frame=single,label=Example]
  t210 2 2 6 7 2 6 9 9 1 1 5 3 8 1 9 9 410 510 6
\end{Verbatim}
 The file \mbox{\texttt{\mdseries\slshape filename}} can be read using \texttt{ReadGenerators} (\ref{ReadGenerators}). 
\begin{Verbatim}[commandchars=!@|,fontsize=\small,frame=single,label=Example]
  !gapprompt@gap>| !gapinput@file:=Concatenation(SemigroupsDir(), "/examples/graph7c.semigroups.gz");;|
  !gapprompt@gap>| !gapinput@ReadGenerators(file, 453);|
  [ Transformation( [ 1, 2, 2 ] ), IdentityTransformation, 
    Transformation( [ 1, 2, 3, 4, 5, 7, 7 ] ), 
    Transformation( [ 1, 3, 2, 4, 7, 6, 7 ] ), 
    Transformation( [ 4, 2, 1, 1, 6, 5 ] ), 
    Transformation( [ 4, 3, 2, 1, 6, 7, 7 ] ), 
    Transformation( [ 4, 4, 5, 7, 6, 1, 1 ] ), 
    Transformation( [ 7, 6, 6, 1, 2, 4, 4 ] ), 
    Transformation( [ 7, 7, 5, 4, 3, 1, 1 ] ) ]
\end{Verbatim}
 }

 }

 
\section{\textcolor{Chapter }{Use $\mathcal{R}$-classes not $\mathcal{L}$-classes}}\logpage{[ 1, 8, 0 ]}
\hyperdef{L}{X861CF23F862D7F43}{}
{
 It is harder for \textsf{Semigroups} to compute Green's $\mathcal{L}$- and $\mathcal{H}$-classes of a transformation semigroup and the methods used to compute with
Green's $\mathcal{R}$- and $\mathcal{D}$-classes are the most efficient in \textsf{Semigroups}. Thus, if you are computing with a transformation semigroup, wherever
possible it is advisable to use the commands relating to Green's $\mathcal{R}$- or $\mathcal{D}$-classes rather than those relating to Green's $\mathcal{L}$- or $\mathcal{H}$-classes. No such difficulties are present when computing with semigroups of
partial permutations, or subsemigroups of a regular Rees 0-matrix semigroup
over a group.

 The methods in \textsf{Semigroups} allow the computation of individual Green's classes without the need to
compute all the elements of the underlying semigroup; see \texttt{GreensRClassOfElementNC} (\ref{GreensRClassOfElementNC}). It is also possible to compute all the $\mathcal{R}$-classes, the number of elements and test membership in a semigroup without
computing all the elements; see, for example, \texttt{GreensRClasses} (\ref{GreensRClasses}), \texttt{RClassReps} (\ref{RClassReps}), \texttt{IteratorOfRClassReps} (\ref{IteratorOfRClassReps}), \texttt{IteratorOfRClasses} (\ref{IteratorOfRClasses}), or \texttt{NrRClasses} (\ref{NrRClasses}). This may be useful if you want to study a very large semigroup where
computing all the elements of the semigroup is not feasible. }

 
\section{\textcolor{Chapter }{New methods for existing functions}}\logpage{[ 1, 9, 0 ]}
\hyperdef{L}{X7F3F95577FB2396F}{}
{
 \textsf{Semigroups} contains functions synonymous to some of those define in the \textsf{GAP} library but, for the sake of convenience, they have abbreviated names; further
details can be found at the appropriate points in the later chapters of the
manual. 

 \textsf{Semigroups} contains different methods for some \textsf{GAP} library functions, and so you might notice that \textsf{GAP} behaves differently when \textsf{Semigroups} is loaded. For more details about semigroups in \textsf{GAP} or Green's relations in particular, see  (\textbf{Reference: Semigroups}) or  (\textbf{Reference: Green's Relations}). }

 }

 
\chapter{\textcolor{Chapter }{Creating semigroups and monoids}}\label{create}
\logpage{[ 2, 0, 0 ]}
\hyperdef{L}{X79A4C070831D989D}{}
{
 In this chapter we describe the various ways that semigroups and monoids can
be created in \textsf{Semigroups}, the options that are available at the time of creation, and describe some
standard examples available in \textsf{Semigroups}. 

 Any semigroup created before \textsf{Semigroups} has been loaded must be recreated after \textsf{Semigroups} is loaded so that the options record (described in Section \ref{opts}) is defined. Almost all of the functions and methods provided by \textsf{Semigroups}, including those methods for existing \textsf{GAP} library functions, will return an error when applied to a semigroup created
before \textsf{Semigroups} is loaded. 
\section{\textcolor{Chapter }{Random semigroups}}\logpage{[ 2, 1, 0 ]}
\hyperdef{L}{X7C3F130B8362D55A}{}
{
 

\subsection{\textcolor{Chapter }{RandomInverseMonoid}}
\logpage{[ 2, 1, 1 ]}\nobreak
\hyperdef{L}{X7B341D6C7CECFB55}{}
{\noindent\textcolor{FuncColor}{$\triangleright$\ \ \texttt{RandomInverseMonoid({\mdseries\slshape m, n})\index{RandomInverseMonoid@\texttt{RandomInverseMonoid}}
\label{RandomInverseMonoid}
}\hfill{\scriptsize (operation)}}\\
\noindent\textcolor{FuncColor}{$\triangleright$\ \ \texttt{RandomInverseSemigroup({\mdseries\slshape m, n})\index{RandomInverseSemigroup@\texttt{RandomInverseSemigroup}}
\label{RandomInverseSemigroup}
}\hfill{\scriptsize (operation)}}\\
\textbf{\indent Returns:\ }
An inverse monoid or semigroup.



 Returns a random inverse monoid or semigroup of partial permutations with
degree at most \mbox{\texttt{\mdseries\slshape n}} with \mbox{\texttt{\mdseries\slshape m}} generators. 
\begin{Verbatim}[commandchars=!@|,fontsize=\small,frame=single,label=Example]
  !gapprompt@gap>| !gapinput@S:=RandomInverseSemigroup(10,10);                                |
  <inverse partial perm semigroup on 10 pts with 10 generators>
  !gapprompt@gap>| !gapinput@S:=RandomInverseMonoid(10,10);   |
  <inverse partial perm monoid on 10 pts with 10 generators>
\end{Verbatim}
 }

 

\subsection{\textcolor{Chapter }{RandomTransformationMonoid}}
\logpage{[ 2, 1, 2 ]}\nobreak
\hyperdef{L}{X79834BC080B011B4}{}
{\noindent\textcolor{FuncColor}{$\triangleright$\ \ \texttt{RandomTransformationMonoid({\mdseries\slshape m, n})\index{RandomTransformationMonoid@\texttt{RandomTransformationMonoid}}
\label{RandomTransformationMonoid}
}\hfill{\scriptsize (operation)}}\\
\noindent\textcolor{FuncColor}{$\triangleright$\ \ \texttt{RandomTransformationSemigroup({\mdseries\slshape m, n})\index{RandomTransformationSemigroup@\texttt{RandomTransformationSemigroup}}
\label{RandomTransformationSemigroup}
}\hfill{\scriptsize (operation)}}\\
\textbf{\indent Returns:\ }
A transformation semigroup or monoid.



 Returns a random transformation monoid or semigroup of at most degree \mbox{\texttt{\mdseries\slshape n}} with \mbox{\texttt{\mdseries\slshape m}} generators. 
\begin{Verbatim}[commandchars=!@|,fontsize=\small,frame=single,label=Example]
  !gapprompt@gap>| !gapinput@S:=RandomTransformationMonoid(5,5);|
  <transformation monoid on 5 pts with 5 generators>
  !gapprompt@gap>| !gapinput@S:=RandomTransformationSemigroup(5,5);|
  <transformation semigroup on 5 pts with 5 generators>
\end{Verbatim}
 }

 

\subsection{\textcolor{Chapter }{RandomPartialPermMonoid}}
\logpage{[ 2, 1, 3 ]}\nobreak
\hyperdef{L}{X843D7E2B7D951523}{}
{\noindent\textcolor{FuncColor}{$\triangleright$\ \ \texttt{RandomPartialPermMonoid({\mdseries\slshape m, n})\index{RandomPartialPermMonoid@\texttt{RandomPartialPermMonoid}}
\label{RandomPartialPermMonoid}
}\hfill{\scriptsize (function)}}\\
\noindent\textcolor{FuncColor}{$\triangleright$\ \ \texttt{RandomPartialPermSemigroup({\mdseries\slshape m, n})\index{RandomPartialPermSemigroup@\texttt{RandomPartialPermSemigroup}}
\label{RandomPartialPermSemigroup}
}\hfill{\scriptsize (function)}}\\
\textbf{\indent Returns:\ }
A partial perm semigroup or monoid.



 Returns a random partial perm monoid or semigroup of degree at most \mbox{\texttt{\mdseries\slshape n}} with \mbox{\texttt{\mdseries\slshape m}} generators. 
\begin{Verbatim}[commandchars=!@|,fontsize=\small,frame=single,label=Example]
  !gapprompt@gap>| !gapinput@S:=RandomPartialPermSemigroup(5, 5);|
  <partial perm semigroup on 4 pts with 5 generators>
  !gapprompt@gap>| !gapinput@S:=RandomPartialPermMonoid(5, 5);|
  <partial perm monoid on 5 pts with 5 generators>
\end{Verbatim}
 }

 

\subsection{\textcolor{Chapter }{RandomBinaryRelationMonoid}}
\logpage{[ 2, 1, 4 ]}\nobreak
\hyperdef{L}{X84882D0F7F6C12D3}{}
{\noindent\textcolor{FuncColor}{$\triangleright$\ \ \texttt{RandomBinaryRelationMonoid({\mdseries\slshape m, n})\index{RandomBinaryRelationMonoid@\texttt{RandomBinaryRelationMonoid}}
\label{RandomBinaryRelationMonoid}
}\hfill{\scriptsize (function)}}\\
\noindent\textcolor{FuncColor}{$\triangleright$\ \ \texttt{RandomBinaryRelationSemigroup({\mdseries\slshape m, n})\index{RandomBinaryRelationSemigroup@\texttt{RandomBinaryRelationSemigroup}}
\label{RandomBinaryRelationSemigroup}
}\hfill{\scriptsize (function)}}\\
\textbf{\indent Returns:\ }
A semigroup or monoid of binary relations.



 Returns a random monoid or semigroup of binary relations on \mbox{\texttt{\mdseries\slshape n}} points with \mbox{\texttt{\mdseries\slshape m}} generators. 
\begin{Verbatim}[commandchars=!@|,fontsize=\small,frame=single,label=Example]
  !gapprompt@gap>| !gapinput@RandomBinaryRelationSemigroup(5,5);|
  <semigroup with 5 generators>
  !gapprompt@gap>| !gapinput@RandomBinaryRelationMonoid(5,5);   |
  <monoid with 5 generators>
\end{Verbatim}
 }

 }

 
\section{\textcolor{Chapter }{New semigroups from old}}\logpage{[ 2, 2, 0 ]}
\hyperdef{L}{X7A5CFD4F8607CBF7}{}
{
 

\subsection{\textcolor{Chapter }{ClosureInverseSemigroup}}
\logpage{[ 2, 2, 1 ]}\nobreak
\hyperdef{L}{X78A488637BBEF7AD}{}
{\noindent\textcolor{FuncColor}{$\triangleright$\ \ \texttt{ClosureInverseSemigroup({\mdseries\slshape S, coll[, opts]})\index{ClosureInverseSemigroup@\texttt{ClosureInverseSemigroup}}
\label{ClosureInverseSemigroup}
}\hfill{\scriptsize (function)}}\\
\textbf{\indent Returns:\ }
An inverse semigroup or monoid.



 This function returns the inverse semigroup or monoid generated by the inverse
semigroup \mbox{\texttt{\mdseries\slshape S}} and the collection of elements \mbox{\texttt{\mdseries\slshape coll}} after first removing duplicates and elements in \mbox{\texttt{\mdseries\slshape coll}} that are already in \mbox{\texttt{\mdseries\slshape S}}. In most cases, the new semigroup knows at least as much information about
its structure as was already known about that of \mbox{\texttt{\mdseries\slshape S}}. 

 If present, the optional third argument \mbox{\texttt{\mdseries\slshape opts}} should be a record containing the values of the options for the inverse
semigroup being created; these options are described in Section \ref{opts}. 

 
\begin{Verbatim}[commandchars=!@|,fontsize=\small,frame=single,label=Example]
  !gapprompt@gap>| !gapinput@S:=InverseMonoid(|
  !gapprompt@>| !gapinput@PartialPerm( [ 1, 2, 3, 5, 6, 7, 8 ], [ 5, 9, 10, 6, 3, 8, 4 ] ),|
  !gapprompt@>| !gapinput@PartialPerm( [ 1, 2, 4, 7, 8, 9 ], [ 10, 7, 8, 5, 9, 1 ] ) );;|
  !gapprompt@gap>| !gapinput@f:=PartialPerm(|
  !gapprompt@>| !gapinput@[ 1, 2, 3, 4, 5, 7, 8, 10, 11, 13, 18, 19, 20 ],|
  !gapprompt@>| !gapinput@[ 5, 1, 7, 3, 10, 2, 12, 14, 11, 16, 6, 9, 15 ]);;|
  !gapprompt@gap>| !gapinput@S:=ClosureInverseSemigroup(S, f);|
  <inverse partial perm semigroup on 19 pts with 4 generators>
  !gapprompt@gap>| !gapinput@Size(S);|
  9744
  !gapprompt@gap>| !gapinput@T:=Idempotents(SymmetricInverseSemigroup(10));;|
  !gapprompt@gap>| !gapinput@S:=ClosureInverseSemigroup(S, T);|
  <inverse partial perm semigroup on 19 pts with 854 generators>
  !gapprompt@gap>| !gapinput@S:=InverseSemigroup(SmallGeneratingSet(S));|
  <inverse partial perm semigroup on 19 pts with 14 generators>
\end{Verbatim}
 }

 

\subsection{\textcolor{Chapter }{ClosureSemigroup}}
\logpage{[ 2, 2, 2 ]}\nobreak
\hyperdef{L}{X7BE36790862AE26F}{}
{\noindent\textcolor{FuncColor}{$\triangleright$\ \ \texttt{ClosureSemigroup({\mdseries\slshape S, coll[, opts]})\index{ClosureSemigroup@\texttt{ClosureSemigroup}}
\label{ClosureSemigroup}
}\hfill{\scriptsize (function)}}\\
\textbf{\indent Returns:\ }
A semigroup or monoid.



 This function returns the semigroup or monoid generated by the semigroup \mbox{\texttt{\mdseries\slshape S}} and the collection of elements \mbox{\texttt{\mdseries\slshape coll}} after removing duplicates and elements from \mbox{\texttt{\mdseries\slshape coll}} that are already in \mbox{\texttt{\mdseries\slshape S}}. In most cases, the new semigroup knows at least as much information about
its structure as was already known about that of \mbox{\texttt{\mdseries\slshape S}}. 

 If present, the optional third argument \mbox{\texttt{\mdseries\slshape opts}} should be a record containing the values of the options for the semigroup
being created as described in Section \ref{opts}.

 
\begin{Verbatim}[commandchars=!@|,fontsize=\small,frame=single,label=Example]
  !gapprompt@gap>| !gapinput@gens:=[ Transformation( [ 2, 6, 7, 2, 6, 1, 1, 5 ] ), |
  !gapprompt@>| !gapinput@ Transformation( [ 3, 8, 1, 4, 5, 6, 7, 1 ] ), |
  !gapprompt@>| !gapinput@ Transformation( [ 4, 3, 2, 7, 7, 6, 6, 5 ] ), |
  !gapprompt@>| !gapinput@ Transformation( [ 7, 1, 7, 4, 2, 5, 6, 3 ] ) ];;|
  !gapprompt@gap>| !gapinput@S:=Monoid(gens[1]);;|
  !gapprompt@gap>| !gapinput@for i in [2..4] do S:=ClosureSemigroup(S, gens[i]); od;|
  !gapprompt@gap>| !gapinput@S;|
  <transformation monoid on 8 pts with 4 generators>
  !gapprompt@gap>| !gapinput@Size(S);|
  233606
\end{Verbatim}
 }

 

\subsection{\textcolor{Chapter }{SubsemigroupByProperty (for a semigroup and function)}}
\logpage{[ 2, 2, 3 ]}\nobreak
\hyperdef{L}{X7E5B4C5A82F9E0E0}{}
{\noindent\textcolor{FuncColor}{$\triangleright$\ \ \texttt{SubsemigroupByProperty({\mdseries\slshape S, func})\index{SubsemigroupByProperty@\texttt{SubsemigroupByProperty}!for a semigroup and function}
\label{SubsemigroupByProperty:for a semigroup and function}
}\hfill{\scriptsize (operation)}}\\
\noindent\textcolor{FuncColor}{$\triangleright$\ \ \texttt{SubsemigroupByProperty({\mdseries\slshape S, func, limit})\index{SubsemigroupByProperty@\texttt{SubsemigroupByProperty}!for a semigroup, function, and limit on the size of the subsemigroup}
\label{SubsemigroupByProperty:for a semigroup, function, and limit on the size of the subsemigroup}
}\hfill{\scriptsize (operation)}}\\
\textbf{\indent Returns:\ }
A semigroup.



 \texttt{SubsemigroupByProperty} returns the subsemigroup of the semigroup \mbox{\texttt{\mdseries\slshape S}} generated by those elements of \mbox{\texttt{\mdseries\slshape S}} fulfilling \mbox{\texttt{\mdseries\slshape func}} (which should be a function returning \texttt{true} or \texttt{false}).

 If no elements of \mbox{\texttt{\mdseries\slshape S}} fulfil \mbox{\texttt{\mdseries\slshape func}}, then \texttt{fail} is returned.

 If the optional third argument \mbox{\texttt{\mdseries\slshape limit}} is present and a positive integer, then once the subsemigroup has at least \mbox{\texttt{\mdseries\slshape limit}} elements the computation stops. 
\begin{Verbatim}[commandchars=!@|,fontsize=\small,frame=single,label=Example]
  !gapprompt@gap>| !gapinput@func:=function(f) return 1^f<>1 and|
  !gapprompt@>| !gapinput@ForAll([1..DegreeOfTransformation(f)], y-> y=1 or y^f=y); end;|
  function( f ) ... end
  !gapprompt@gap>| !gapinput@T:=SubsemigroupByProperty(FullTransformationSemigroup(3), func);|
  <transformation semigroup of size 2, on 3 pts with 2 generators>
  !gapprompt@gap>| !gapinput@T:=SubsemigroupByProperty(FullTransformationSemigroup(4), func);|
  <transformation semigroup of size 3, on 4 pts with 3 generators>
  !gapprompt@gap>| !gapinput@T:=SubsemigroupByProperty(FullTransformationSemigroup(5), func);|
  <transformation semigroup of size 4, on 5 pts with 4 generators>
\end{Verbatim}
 }

 

\subsection{\textcolor{Chapter }{InverseSubsemigroupByProperty}}
\logpage{[ 2, 2, 4 ]}\nobreak
\hyperdef{L}{X832AEDCC7BA9E5F5}{}
{\noindent\textcolor{FuncColor}{$\triangleright$\ \ \texttt{InverseSubsemigroupByProperty({\mdseries\slshape ??})\index{InverseSubsemigroupByProperty@\texttt{InverseSubsemigroupByProperty}}
\label{InverseSubsemigroupByProperty}
}\hfill{\scriptsize (function)}}\\
\textbf{\indent Returns:\ }
An inverse semigroup.



 \texttt{InverseSubsemigroupByProperty} returns the inverse subsemigroup of the inverse semigroup \mbox{\texttt{\mdseries\slshape S}} generated by those elements of \mbox{\texttt{\mdseries\slshape S}} fulfilling \mbox{\texttt{\mdseries\slshape func}} (which should be a function returning \texttt{true} or \texttt{false}).

 If no elements of \mbox{\texttt{\mdseries\slshape S}} fulfil \mbox{\texttt{\mdseries\slshape func}}, then \texttt{fail} is returned.

 If the optional third argument \mbox{\texttt{\mdseries\slshape limit}} is present and a positive integer, then once the subsemigroup has at least \mbox{\texttt{\mdseries\slshape limit}} elements the computation stops. 
\begin{Verbatim}[commandchars=!@|,fontsize=\small,frame=single,label=Example]
  !gapprompt@gap>| !gapinput@IsIsometry:=function(f)|
  !gapprompt@>| !gapinput@local n, i, j, k, l;|
  !gapprompt@>| !gapinput@ n:=RankOfPartialPerm(f);|
  !gapprompt@>| !gapinput@ for i in [1..n-1] do|
  !gapprompt@>| !gapinput@   k:=DomainOfPartialPerm(f)[i];|
  !gapprompt@>| !gapinput@   for j in [i+1..n] do|
  !gapprompt@>| !gapinput@     l:=DomainOfPartialPerm(f)[j];|
  !gapprompt@>| !gapinput@     if not AbsInt(k^f-l^f)=AbsInt(k-l) then|
  !gapprompt@>| !gapinput@       return false;|
  !gapprompt@>| !gapinput@     fi;|
  !gapprompt@>| !gapinput@   od;|
  !gapprompt@>| !gapinput@ od;|
  !gapprompt@>| !gapinput@ return true;|
  !gapprompt@>| !gapinput@end;;|
  !gapprompt@gap>| !gapinput@S:=InverseSubsemigroupByProperty(SymmetricInverseSemigroup(5),|
  !gapprompt@>| !gapinput@IsIsometry);;|
  !gapprompt@gap>| !gapinput@Size(S);|
  142
\end{Verbatim}
 }

 }

 
\section{\textcolor{Chapter }{Options when creating semigroups}}\label{opts}
\logpage{[ 2, 3, 0 ]}
\hyperdef{L}{X799EBA2F819D8867}{}
{
 When using any of \texttt{InverseSemigroup} (\textbf{Reference: InverseSemigroup}), \texttt{InverseMonoid} (\textbf{Reference: InverseMonoid}), \texttt{Semigroup} (\textbf{Reference: Semigroup}), \texttt{Monoid} (\textbf{Reference: Monoid}), \texttt{SemigroupByGenerators} (\textbf{Reference: SemigroupByGenerators}), \texttt{MonoidByGenerators} (\textbf{Reference: MonoidByGenerators}), \texttt{ClosureInverseSemigroup} (\ref{ClosureInverseSemigroup}) or \texttt{ClosureSemigroup} (\ref{ClosureSemigroup}) a record can be given as an optional final argument. The components of this
record specify the values of certain options for the semigroup being created.
A list of these options and their default values is given below. 

 Assume that \mbox{\texttt{\mdseries\slshape S}} is the semigroup created by one of the functions given above and that \mbox{\texttt{\mdseries\slshape S}} is generated by a collection \mbox{\texttt{\mdseries\slshape gens}} of transformations, partial permutations, or Rees 0-matrix semigroup elements. 
\begin{description}
\item[{\texttt{acting}}]  this component should be \texttt{true} or \texttt{false}. In order for a semigroup to use the methods in \textsf{Semigroups} it must satisfy \texttt{IsActingSemigroup}. By default any semigroup or monoid of transformations, partial permutations,
or Rees 0-matrix elements satisfies \texttt{IsActingSemigroup}. From time to time, it might be preferable to use the exhaustive algorithm in
the \textsf{GAP} library to compute with a semigroup. If this is the case, then the value of
this component can be set \texttt{false} when the semigroup is created. Following this none of the methods in the \textsf{Semigroups} package will be used to compute anything about the semigroup. 
\item[{\texttt{regular}}]  this component should be \texttt{true} or \texttt{false}. If it is known \emph{a priori} that the semigroup \texttt{S} being created is a regular semigroup, then this component can be set to \texttt{true}. In this case, \texttt{S} knows it is a regular semigroup and can take advantage of the methods for
regular semigroups in \textsf{Semigroups}. It is usually much more efficient to compute with a regular semigroup that
to compute with a non-regular semigroup.

 If this option is set to \texttt{true} when the semigroup being defined is \textsc{not} regular, then the results maybe unpredictable. 

 The default value for this option is \texttt{false}. 
\item[{\texttt{hashlen}}]  this component should be a positive integer, which roughly specifies the
lengths of the hash tables used internally by \textsf{Semigroups}. \textsf{Semigroups} uses hash tables in several fundamental methods. The lengths of these tables
are a compromise between performance and memory usage; larger tables provide
better performance for large computations but use more memory. Note that it is
unlikely that you will need to specify this option unless you find that \textsf{GAP} runs out of memory unexpectedly or that the performance of \textsf{Semigroups} is poorer than expected. If you find that \textsf{GAP} runs out of memory unexpectedly, or you plan to do a large number of
computations with relatively small semigroups (say with tens of thousands of
elements), then you might consider setting \texttt{hashlen} to be less than the default value of \texttt{25013} for each of these semigroups. If you find that the performance of \textsf{Semigroups} is unexpectedly poor, or you plan to do a computation with a very large
semigroup (say, more than 10 million elements), then you might consider
setting \texttt{hashlen} to be greater than the default value of \texttt{25013}. 

 You might find it useful to set the info level of the info class \texttt{InfoOrb} to 2 or higher since this will indicate when hash tables used by \textsf{Semigroups} are being grown; see \texttt{SetInfoLevel} (\textbf{Reference: SetInfoLevel}). 
\item[{\texttt{small}}] if this component is set to \texttt{true}, then \textsf{Semigroups} will compute a small subset of \mbox{\texttt{\mdseries\slshape gens}} that generates \mbox{\texttt{\mdseries\slshape S}} at the time that \mbox{\texttt{\mdseries\slshape S}} is created. This will increase the amount of time required to create \mbox{\texttt{\mdseries\slshape S}} substantially, but may decrease the amount of time required for subsequent
calculations with \mbox{\texttt{\mdseries\slshape S}}. If this component is set to \texttt{false}, then \textsf{Semigroups} will return the semigroup generated by \mbox{\texttt{\mdseries\slshape gens}} without modifying \mbox{\texttt{\mdseries\slshape gens}}. The default value for this component is \texttt{false}. 
\end{description}
 The default values of the options described above are stored in a global
variable named \texttt{SemigroupsOptionsRec} (\ref{SemigroupsOptionsRec}). If you want to change the default values of these options for a single \textsf{GAP} session, then you can simply redefine the value in \textsf{GAP}. For example, to change the option \texttt{small} from the default value of \mbox{\texttt{\mdseries\slshape false}} use: 
\begin{Verbatim}[commandchars=!@|,fontsize=\small,frame=single,label=Example]
  !gapprompt@gap>| !gapinput@SemigroupsOptionsRec.small:=true;|
  true
\end{Verbatim}
 If you want to change the default values of the options stored in \texttt{SemigroupsOptionsRec} (\ref{SemigroupsOptionsRec}) for all \textsf{GAP} sessions, then you can edit these values in the file \texttt{semigroups/gap/options.g}. 

 
\begin{Verbatim}[commandchars=!@|,fontsize=\small,frame=single,label=Example]
  !gapprompt@gap>| !gapinput@S:=Semigroup(Transformation( [ 1, 2, 3, 3 ] ), |
  !gapprompt@>| !gapinput@rec(hashlen:=100003, small:=false));|
  <commutative transformation semigroup on 4 pts with 1 generator>
\end{Verbatim}
 

\subsection{\textcolor{Chapter }{SemigroupsOptionsRec}}
\logpage{[ 2, 3, 1 ]}\nobreak
\hyperdef{L}{X7969E83287A8FD5D}{}
{\noindent\textcolor{FuncColor}{$\triangleright$\ \ \texttt{SemigroupsOptionsRec\index{SemigroupsOptionsRec@\texttt{SemigroupsOptionsRec}}
\label{SemigroupsOptionsRec}
}\hfill{\scriptsize (global variable)}}\\


 This global variable is a record whose components contain the default values
of certain options for transformation semigroups created after \textsf{Semigroups} has been loaded. A description of these options is given above in Section \ref{opts}. 

 The value of \texttt{SemigroupsOptionsRec} is defined in the file \texttt{semigroups/gap/options.g} as: 
\begin{Verbatim}[commandchars=!@|,fontsize=\small,frame=single,label=Example]
  rec( acting := true, hashlen := rec( L := 25013, M := 6257, S :=
           251 ), regular := false, small := false )
\end{Verbatim}
 }

 }

 
\section{\textcolor{Chapter }{Changing the representation of a semigroup}}\logpage{[ 2, 4, 0 ]}
\hyperdef{L}{X82CCC1A781650878}{}
{
 

\subsection{\textcolor{Chapter }{IsomorphismPermGroup}}
\logpage{[ 2, 4, 1 ]}\nobreak
\hyperdef{L}{X80B7B1C783AA1567}{}
{\noindent\textcolor{FuncColor}{$\triangleright$\ \ \texttt{IsomorphismPermGroup({\mdseries\slshape S})\index{IsomorphismPermGroup@\texttt{IsomorphismPermGroup}}
\label{IsomorphismPermGroup}
}\hfill{\scriptsize (operation)}}\\
\textbf{\indent Returns:\ }
 An isomorphism. 



 If the semigroup \mbox{\texttt{\mdseries\slshape S}} satisfies \texttt{IsGroupAsSemigroup} (\ref{IsGroupAsSemigroup}), then \texttt{IsomorphismPermGroup} returns an isomorphism to a permutation group.

 If \mbox{\texttt{\mdseries\slshape S}} does not satisfy \texttt{IsGroupAsSemigroup} (\ref{IsGroupAsSemigroup}), then an error is given. 
\begin{Verbatim}[commandchars=!@|,fontsize=\small,frame=single,label=Example]
  !gapprompt@gap>| !gapinput@S:=Semigroup( Transformation( [ 2, 2, 3, 4, 6, 8, 5, 5 ] ),|
  !gapprompt@>| !gapinput@Transformation( [ 3, 3, 8, 2, 5, 6, 4, 4 ] ) );;|
  !gapprompt@gap>| !gapinput@IsGroupAsSemigroup(S);|
  true
  !gapprompt@gap>| !gapinput@IsomorphismPermGroup(S); |
  MappingByFunction( <transformation group on 8 pts with 2 generators>
   , Group([ (5,6,8), (2,3,8,
  4) ]), <Attribute "PermutationOfImage">, function( x ) ... end )
  !gapprompt@gap>| !gapinput@StructureDescription(Range(IsomorphismPermGroup(S)));|
  "S6"
  !gapprompt@gap>| !gapinput@S:=Range(IsomorphismPartialPermSemigroup(SymmetricGroup(4)));|
  <inverse partial perm semigroup on 4 pts with 2 generators>
  !gapprompt@gap>| !gapinput@IsomorphismPermGroup(S);|
  MappingByFunction( <partial perm group on 4 pts with 2 generators>
   , Group([ (1,2,3,4), (1,
  2) ]), <Attribute "AsPermutation">, function( x ) ... end )
\end{Verbatim}
 }

 

\subsection{\textcolor{Chapter }{IsomorphismTransformationMonoid}}
\logpage{[ 2, 4, 2 ]}\nobreak
\hyperdef{L}{X84AF7B907E82F2D1}{}
{\noindent\textcolor{FuncColor}{$\triangleright$\ \ \texttt{IsomorphismTransformationMonoid({\mdseries\slshape S})\index{IsomorphismTransformationMonoid@\texttt{IsomorphismTransformationMonoid}}
\label{IsomorphismTransformationMonoid}
}\hfill{\scriptsize (operation)}}\\
\noindent\textcolor{FuncColor}{$\triangleright$\ \ \texttt{IsomorphismTransformationSemigroup({\mdseries\slshape S})\index{IsomorphismTransformationSemigroup@\texttt{IsomorphismTransformationSemigroup}}
\label{IsomorphismTransformationSemigroup}
}\hfill{\scriptsize (operation)}}\\
\textbf{\indent Returns:\ }
An isomorphism. 



 \texttt{IsomorphismTransformationSemigroup} returns an isomorphism from the semigroup \mbox{\texttt{\mdseries\slshape S}} to a semigroup of transformations.

 \texttt{IsomorphismTransformationMonoid} returns an isomorphism from the monoid \mbox{\texttt{\mdseries\slshape S}} to a monoid of transformations. 

 We only describe \texttt{IsomorphismTransformationMonoid}, the corresponding statements for \texttt{IsomorphismTransformationSemigroup} also hold. 
\begin{description}
\item[{Partial permutation semigroups}]  If \mbox{\texttt{\mdseries\slshape S}} is a partial permutation monoid, then this function returns an isomorphism
from \mbox{\texttt{\mdseries\slshape S}} to a monoid of partial permutations on \texttt{[1..LargestMovedPoint(\mbox{\texttt{\mdseries\slshape S}})+1]} obtained using \texttt{AsTransformation} (\textbf{Reference: AsTransformation}). The inverse of this isomorphism is obtained using \texttt{AsPartialPerm} (\textbf{Reference: AsPartialPerm (for a permutation and a set of
    positive integers)}); see \texttt{LargestMovedPoint} (\textbf{Reference: LargestMovedPoint (for a partial perm)}), \texttt{InverseMonoid} (\textbf{Reference: InverseMonoid}) and \texttt{Monoid} (\textbf{Reference: Monoid}). 
\item[{Permutation groups}]  If \mbox{\texttt{\mdseries\slshape S}} is a permutation group, then this function returns an isomorphism from \mbox{\texttt{\mdseries\slshape S}} to a transformation monoid acting on the set \texttt{[1..NrMovedPoints(\mbox{\texttt{\mdseries\slshape S}})]} obtained using \texttt{AsTransformation} (\textbf{Reference: AsTransformation}); see \texttt{NrMovedPoints} (\textbf{Reference: NrMovedPoints (for a permutation)}). 
\item[{Transformation semigroups}]  If \mbox{\texttt{\mdseries\slshape S}} is a transformation semigroup satisfying \texttt{IsMonoidAsSemigroup} (\ref{IsMonoidAsSemigroup}), then this function returns an isomorphism from \mbox{\texttt{\mdseries\slshape S}} to a transformation monoid. 

 
\end{description}
 
\begin{Verbatim}[commandchars=!@|,fontsize=\small,frame=single,label=Example]
  !gapprompt@gap>| !gapinput@S:=Semigroup( Transformation( [ 1, 4, 6, 2, 5, 3, 7, 8, 9, 9 ] ),|
  !gapprompt@>| !gapinput@Transformation( [ 6, 3, 2, 7, 5, 1, 8, 8, 9, 9 ] ) );;|
  !gapprompt@gap>| !gapinput@IsTransformationMonoid(S);|
  false
  !gapprompt@gap>| !gapinput@IsMonoidAsSemigroup(S);|
  true
  !gapprompt@gap>| !gapinput@M:=Range(IsomorphismTransformationMonoid(S));|
  <transformation monoid on 8 pts with 2 generators>
  !gapprompt@gap>| !gapinput@IsTransformationMonoid(M);|
  true
  !gapprompt@gap>| !gapinput@S:=InverseMonoid(|
  !gapprompt@>| !gapinput@ PartialPerm( [ 1, 2, 3 ], [ 4, 2, 3 ] ),|
  !gapprompt@>| !gapinput@ PartialPerm( [ 1, 2, 4 ], [ 1, 3, 2 ] ),|
  !gapprompt@>| !gapinput@ PartialPerm( [ 1, 2, 4 ], [ 4, 1, 2 ] ) );;|
  !gapprompt@gap>| !gapinput@T:=Range(IsomorphismTransformationMonoid(S));|
  <transformation monoid on 5 pts with 5 generators>
  !gapprompt@gap>| !gapinput@Size(S); Size(T);|
  117
  117
\end{Verbatim}
 }

 }

 
\section{\textcolor{Chapter }{Standard examples}}\label{Examples}
\logpage{[ 2, 5, 0 ]}
\hyperdef{L}{X7C76D1DC7DAF03D3}{}
{
  In this section, we describe the operations in \textsf{Semigroups} that can be used to creating semigroups belonging to several standard classes
of example. 

\subsection{\textcolor{Chapter }{PartialTransformationSemigroup}}
\logpage{[ 2, 5, 1 ]}\nobreak
\hyperdef{L}{X7E1A7FF08064440C}{}
{\noindent\textcolor{FuncColor}{$\triangleright$\ \ \texttt{PartialTransformationSemigroup({\mdseries\slshape n})\index{PartialTransformationSemigroup@\texttt{PartialTransformationSemigroup}}
\label{PartialTransformationSemigroup}
}\hfill{\scriptsize (operation)}}\\
\textbf{\indent Returns:\ }
A transformation monoid.



 If \mbox{\texttt{\mdseries\slshape n}} is a positive integer, then this function returns a semigroup of
transformations on \texttt{\mbox{\texttt{\mdseries\slshape n}}+1} points which is isomorphic to the semigroup consisting of all partial
transformation on \mbox{\texttt{\mdseries\slshape n}} points. This monoid has \texttt{\mbox{\texttt{\mdseries\slshape n}}\texttt{\symbol{94}}\mbox{\texttt{\mdseries\slshape n}}+1} elements.  
\begin{Verbatim}[commandchars=!@|,fontsize=\small,frame=single,label=Example]
  !gapprompt@gap>| !gapinput@PartialTransformationSemigroup(8); |
  <regular transformation monoid on 9 pts with 4 generators>
  !gapprompt@gap>| !gapinput@Size(last);|
  43046721
\end{Verbatim}
 }

 

\subsection{\textcolor{Chapter }{FullMatrixSemigroup}}
\logpage{[ 2, 5, 2 ]}\nobreak
\hyperdef{L}{X78F9812D79A457EF}{}
{\noindent\textcolor{FuncColor}{$\triangleright$\ \ \texttt{FullMatrixSemigroup({\mdseries\slshape d, q})\index{FullMatrixSemigroup@\texttt{FullMatrixSemigroup}}
\label{FullMatrixSemigroup}
}\hfill{\scriptsize (operation)}}\\
\noindent\textcolor{FuncColor}{$\triangleright$\ \ \texttt{GeneralLinearSemigroup({\mdseries\slshape d, q})\index{GeneralLinearSemigroup@\texttt{GeneralLinearSemigroup}}
\label{GeneralLinearSemigroup}
}\hfill{\scriptsize (operation)}}\\
\textbf{\indent Returns:\ }
A matrix semigroup.



 \texttt{FullMatrixSemigroup} and \texttt{GeneralLinearSemigroup} are synonyms for each other. They both return the full matrix semigroup, or if
you prefer the general linear semigroup, of \mbox{\texttt{\mdseries\slshape d}} by \mbox{\texttt{\mdseries\slshape d}} matrices with entries over the field with \mbox{\texttt{\mdseries\slshape q}} elements. This semigroup has \texttt{q\texttt{\symbol{94}}(d\texttt{\symbol{94}}2)} elements. 

 \textsc{Please note:} there are currently no special methods for computing with matrix semigroups in \textsf{Semigroups} and so it might be advisable to use \texttt{IsomorphismTransformationSemigroup} (\ref{IsomorphismTransformationSemigroup}). 
\begin{Verbatim}[commandchars=!@|,fontsize=\small,frame=single,label=Example]
  !gapprompt@gap>| !gapinput@S:=FullMatrixSemigroup(3,4);|
  <full matrix semigroup 3x3 over GF(2^2)>
  !gapprompt@gap>| !gapinput@T:=Range(IsomorphismTransformationSemigroup(S));;|
  !gapprompt@gap>| !gapinput@Size(T);|
  262144
\end{Verbatim}
 }

 

\subsection{\textcolor{Chapter }{IsFullMatrixSemigroup}}
\logpage{[ 2, 5, 3 ]}\nobreak
\hyperdef{L}{X85D49CF2826D3AA4}{}
{\noindent\textcolor{FuncColor}{$\triangleright$\ \ \texttt{IsFullMatrixSemigroup({\mdseries\slshape S})\index{IsFullMatrixSemigroup@\texttt{IsFullMatrixSemigroup}}
\label{IsFullMatrixSemigroup}
}\hfill{\scriptsize (property)}}\\
\noindent\textcolor{FuncColor}{$\triangleright$\ \ \texttt{IsGeneralLinearSemigroup({\mdseries\slshape S})\index{IsGeneralLinearSemigroup@\texttt{IsGeneralLinearSemigroup}}
\label{IsGeneralLinearSemigroup}
}\hfill{\scriptsize (property)}}\\


 \texttt{IsFullMatrixSemigroup} and \texttt{IsGeneralLinearSemigroup} return \texttt{true} if the semigroup \texttt{S} was created using either of the commands \texttt{FullMatrixSemigroup} (\ref{FullMatrixSemigroup}) or \texttt{GeneralLinearSemigroup} (\ref{GeneralLinearSemigroup}) and \texttt{false} otherwise. 
\begin{Verbatim}[commandchars=!@|,fontsize=\small,frame=single,label=Example]
  !gapprompt@gap>| !gapinput@S:=RandomTransformationSemigroup(4,4);;|
  !gapprompt@gap>| !gapinput@IsFullMatrixSemigroup(S);|
  false
  !gapprompt@gap>| !gapinput@S:=GeneralLinearSemigroup(3,3);|
  <full matrix semigroup 3x3 over GF(3)>
  !gapprompt@gap>| !gapinput@IsFullMatrixSemigroup(S);|
  true
\end{Verbatim}
 }

 

\subsection{\textcolor{Chapter }{MunnSemigroup}}
\logpage{[ 2, 5, 4 ]}\nobreak
\hyperdef{L}{X78FBE6DD7BCA30C1}{}
{\noindent\textcolor{FuncColor}{$\triangleright$\ \ \texttt{MunnSemigroup({\mdseries\slshape S})\index{MunnSemigroup@\texttt{MunnSemigroup}}
\label{MunnSemigroup}
}\hfill{\scriptsize (operation)}}\\
\textbf{\indent Returns:\ }
The Munn semigroup of a semilattice.



 If \mbox{\texttt{\mdseries\slshape S}} is a semilattice, then \texttt{MunnSemigroup} returns the inverse semigroup of partial permutations of isomorphisms of
principal ideals of \mbox{\texttt{\mdseries\slshape S}}; called the \emph{Munn semigroup} of \mbox{\texttt{\mdseries\slshape S}}.

 This function was written jointly by J. D. Mitchell, Yann Peresse (St
Andrews), Yanhui Wang (York). 

 \textsc{Please note:} the \href{http://www.maths.qmul.ac.uk/~leonard/grape/} {Grape} package version 4.5 or higher should be fully installed for this function to
work. 
\begin{Verbatim}[commandchars=!@|,fontsize=\small,frame=single,label=Example]
  !gapprompt@gap>| !gapinput@S:=InverseSemigroup(|
  !gapprompt@>| !gapinput@PartialPerm( [ 1, 2, 3, 4, 5, 6, 7, 10 ], [ 4, 6, 7, 3, 8, 2, 9, 5 ] ),|
  !gapprompt@>| !gapinput@PartialPerm( [ 1, 2, 7, 9 ], [ 5, 6, 4, 3 ] ) );|
  <inverse partial perm semigroup on 10 pts with 2 generators>
  !gapprompt@gap>| !gapinput@T:=InverseSemigroup(Idempotents(S), rec(small:=true));;|
  !gapprompt@gap>| !gapinput@M:=MunnSemigroup(T);;|
  !gapprompt@gap>| !gapinput@NrIdempotents(M);|
  60
  !gapprompt@gap>| !gapinput@NrIdempotents(S);|
  60
\end{Verbatim}
 }

 
\subsection{\textcolor{Chapter }{Monoids of order preserving functions}}\logpage{[ 2, 5, 5 ]}
\hyperdef{L}{X87B855227B9870BD}{}
{
\noindent\textcolor{FuncColor}{$\triangleright$\ \ \texttt{OrderEndomorphisms({\mdseries\slshape n})\index{OrderEndomorphisms@\texttt{OrderEndomorphisms}!monoid of order preserving transformations}
\label{OrderEndomorphisms:monoid of order preserving transformations}
}\hfill{\scriptsize (operation)}}\\
\noindent\textcolor{FuncColor}{$\triangleright$\ \ \texttt{POI({\mdseries\slshape n})\index{POI@\texttt{POI}!monoid of order preserving partial perms}
\label{POI:monoid of order preserving partial perms}
}\hfill{\scriptsize (operation)}}\\
\noindent\textcolor{FuncColor}{$\triangleright$\ \ \texttt{POPI({\mdseries\slshape n})\index{POPI@\texttt{POPI}!monoid of orientation preserving partial
      perms}
\label{POPI:monoid of orientation preserving partial
      perms}
}\hfill{\scriptsize (operation)}}\\
\textbf{\indent Returns:\ }
A semigroup of transformations or partial permutations related to a linear
order. 



 
\begin{description}
\item[{\texttt{OrderEndomorphisms(\mbox{\texttt{\mdseries\slshape n}})}}]  \texttt{OrderEndomorphisms(\mbox{\texttt{\mdseries\slshape n}})} returns the monoid of transformations that preserve the usual order on $\{1,2,\ldots, n\}$ where \mbox{\texttt{\mdseries\slshape n}} is a positive integer.  \texttt{OrderEndomorphisms(\mbox{\texttt{\mdseries\slshape n}})} is generated by the $\mbox{\texttt{\mdseries\slshape n+1}}$ transformations: 
\[ \left( \begin{array}{ccccccccc} 1&2&3&\cdots&n-1& n\\ 1&1&2&\cdots&n-2&n-1
\end{array}\right), \qquad \left( \begin{array}{ccccccccc} 1&2&\cdots&i-1& i&
i+1&i+2&\cdots &n\\ 1&2&\cdots&i-1& i+1&i+1&i+2&\cdots &n\\ \end{array}\right) \]
 where $i=0,\ldots,n-1$ and has ${2n-1\choose n-1}$ elements.  
\item[{\texttt{POI(\mbox{\texttt{\mdseries\slshape n}})}}]  \texttt{POI(\mbox{\texttt{\mdseries\slshape n}})} returns the inverse monoid of partial permutations that preserve the usual
order on $\{1,2,\ldots, n\}$ where \mbox{\texttt{\mdseries\slshape n}} is a positive integer.  \texttt{POI(\mbox{\texttt{\mdseries\slshape n}})} is generated by the $\mbox{\texttt{\mdseries\slshape n}}$ partial permutations: 
\[ \left( \begin{array}{ccccc} 1&2&3&\cdots&n\\ -&1&2&\cdots&n-1
\end{array}\right), \qquad \left( \begin{array}{ccccccccc} 1&2&\cdots&i-1& i&
i+1&i+2&\cdots &n\\ 1&2&\cdots&i-1& i+1&-&i+2&\cdots&n\\ \end{array}\right) \]
 where $i=1, \ldots, n-1$ and has ${2n\choose n}$ elements.  
\item[{\texttt{POPI(\mbox{\texttt{\mdseries\slshape n}})}}]  \texttt{POPI(\mbox{\texttt{\mdseries\slshape n}})} returns the inverse monoid of partial permutation that preserve the
orientation of $\{1,2,\ldots, n\}$ where $n$ is a positive integer.  \texttt{POPI(\mbox{\texttt{\mdseries\slshape n}})} is generated by the partial permutations: 
\[ \left( \begin{array}{ccccc} 1&2&\cdots&n-1&n\\ 2&3&\cdots&n&1
\end{array}\right),\qquad \left( \begin{array}{cccccc} 1&2&\cdots&n-2&n-1&n\\
1&2&\cdots&n-2&n&- \end{array}\right). \]
 and has $1+\frac{n}{2}{2n\choose n}$ elements.  
\end{description}
 
\begin{Verbatim}[commandchars=!@|,fontsize=\small,frame=single,label=Example]
  !gapprompt@gap>| !gapinput@S:=POPI(10);                                            |
  <inverse partial perm monoid on 10 pts with 2 generators>
  !gapprompt@gap>| !gapinput@Size(S);|
  923781
  !gapprompt@gap>| !gapinput@1+5*Binomial(20, 10);|
  923781
  !gapprompt@gap>| !gapinput@S:=POI(10);|
  <inverse partial perm monoid on 10 pts with 10 generators>
  !gapprompt@gap>| !gapinput@Size(S);|
  184756
  !gapprompt@gap>| !gapinput@Binomial(20,10);|
  184756
  !gapprompt@gap>| !gapinput@IsSubsemigroup(POPI(10), POI(10));|
  true
  !gapprompt@gap>| !gapinput@S:=OrderEndomorphisms(5);|
  <regular transformation monoid on 5 pts with 5 generators>
  !gapprompt@gap>| !gapinput@IsIdempotentGenerated(S);|
  true
  !gapprompt@gap>| !gapinput@Size(S)=Binomial(2*5-1, 5-1);|
  true
\end{Verbatim}
 }

 

\subsection{\textcolor{Chapter }{SingularTransformationSemigroup}}
\logpage{[ 2, 5, 6 ]}\nobreak
\hyperdef{L}{X7894EE357D103806}{}
{\noindent\textcolor{FuncColor}{$\triangleright$\ \ \texttt{SingularTransformationSemigroup({\mdseries\slshape n})\index{SingularTransformationSemigroup@\texttt{SingularTransformationSemigroup}}
\label{SingularTransformationSemigroup}
}\hfill{\scriptsize (operation)}}\\
\textbf{\indent Returns:\ }
The semigroup of non-invertible transformations.



 If \mbox{\texttt{\mdseries\slshape n}} is a positive integer, then this function returns the semigroup of
non-invertible transformations, which is generated by the \texttt{\mbox{\texttt{\mdseries\slshape n}}(\mbox{\texttt{\mdseries\slshape n}}-1)} idempotents of degree \mbox{\texttt{\mdseries\slshape n}} and rank \texttt{\mbox{\texttt{\mdseries\slshape n}}-1} and has $n^n-n!$ elements. 
\begin{Verbatim}[commandchars=!@|,fontsize=\small,frame=single,label=Example]
  !gapprompt@gap>| !gapinput@S:=SingularTransformationSemigroup(5);|
  <regular transformation semigroup on 5 pts with 20 generators>
  !gapprompt@gap>| !gapinput@Size(S);|
  3005
\end{Verbatim}
 }

 

\subsection{\textcolor{Chapter }{RegularBinaryRelationSemigroup}}
\logpage{[ 2, 5, 7 ]}\nobreak
\hyperdef{L}{X831612CD78F55B3C}{}
{\noindent\textcolor{FuncColor}{$\triangleright$\ \ \texttt{RegularBinaryRelationSemigroup({\mdseries\slshape n})\index{RegularBinaryRelationSemigroup@\texttt{RegularBinaryRelationSemigroup}}
\label{RegularBinaryRelationSemigroup}
}\hfill{\scriptsize (operation)}}\\
\textbf{\indent Returns:\ }
A semigroup of binary relations.



 \texttt{RegularBinaryRelationSemigroup} return the semigroup generated by the regular binary relations on the set $\{1,\ldots, \mbox{\texttt{\mdseries\slshape n}}\}$ for a positive integer \mbox{\texttt{\mdseries\slshape n}}.  \texttt{RegularBinaryRelationSemigroup(\mbox{\texttt{\mdseries\slshape n}})} is generated by the $4$ binary relations: 
\[ \begin{array}{ll} \left(\begin{array}{ccccccccc} 1&2&\cdots&n-1& n\\
2&3&\cdots&n&1 \end{array}\right),& \quad \left(\begin{array}{ccccccccc}
1&2&3&\cdots&n\\ 2&1&3&\cdots&n \end{array}\right),\\
\left(\begin{array}{ccccccccc} 1&2&\cdots&n-1& n\\ 2&2&\cdots&n-1&\{1,n\}
\end{array}\right), &\quad \left(\begin{array}{ccccccccc} 1&2&\cdots&n-1&n\\
2&2&\cdots&n-1&- \end{array}\right). \end{array} \]
  Note that this semigroup has nearly $2^{(n^2)}$ elements. }

 }

 
\section{\textcolor{Chapter }{The examples directory}}\label{Catalogues}
\logpage{[ 2, 6, 0 ]}
\hyperdef{L}{X7F59D5A17CE475CC}{}
{
  The \texttt{examples} folder of the \textsf{Semigroups} package directory contains catalogues of some naturally occurring semigroups
of transformations and partial permutations. These files can be read into \textsf{GAP} using \texttt{ReadGenerators} (\ref{ReadGenerators}) and similar files can be created using \texttt{WriteGenerators} (\ref{WriteGenerators}).

 Further examples can be downloaded from \vspace{\baselineskip}

 \href{http://tinyurl.com/jdmitchell/data.php} {\texttt{http://tinyurl.com/jdmitchell/data.php}}\vspace{\baselineskip}

 \noindent A summary of the available files, a desciption of their contents, and how they
were created is given below.

 
\begin{description}
\item[{Endomorphisms of graphs}]  the files \texttt{eul\mbox{\texttt{\mdseries\slshape n}}c.semigroups.gz} with $n=3,...,10$; \texttt{graph\mbox{\texttt{\mdseries\slshape n}}c.semigroups.gz} with $n=3,...,8$; and \texttt{selfcomp.semigroups.gz} contain small generating sets for the endomorphism monoids of all connected
Eulerian graphs, all connected graphs, and all self complimentary graphs with $n$ vertices, respectively. These files were created using the catalogues of such
graphs available at:

 \href{http://cs.anu.edu.au/~bdm/data/graphs.html} {\texttt{http://cs.anu.edu.au/\texttt{\symbol{126}}bdm/data/graphs.html}}

 a C program written by Max Neunhoeffer which produces a relatively large list
of endomorphisms containing a generating set for the endomorphism monoid, \texttt{SmallGeneratingSet} (\ref{SmallGeneratingSet}) and then \texttt{IrredundantGeneratingSubset} (\ref{IrredundantGeneratingSubset}) in \textsf{Semigroups}. The monoid generated by the transformations output by \texttt{ReadGenerators("eul7c.semigroups.gz", i);}, say, is the monoid of endomorphisms of the \texttt{i}th graph in the file:

 \href{http://cs.anu.edu.au/~bdm/data/eul7c.g6} {\texttt{http://cs.anu.edu.au/\texttt{\symbol{126}}bdm/data/eul7c.g6}} 
\item[{Munn semigroups}]  the file \texttt{munn.semigroups.gz} contains generators for all the Munn semigroups of semilattices with 2 to 8
elements. The semilattices were obtained from the \href{http://www-history.mcs.st-and.ac.uk/~jamesm/smallsemi.php} {Smallsemi} package using the command: 
\begin{Verbatim}[commandchars=!@|,fontsize=\small,frame=single,label=Example]
  AllSmallSemigroups([2..8], IsSemilatticeAsSemigroup, true);
\end{Verbatim}
 and the generators for the Munn semigroups were calculated using \texttt{MunnSemigroup} (\ref{MunnSemigroup}). 
\item[{Syntactic semigroups}]  the files \texttt{syntactic.semigroups.gz} contain generators for the syntactic semigroups of word acceptors of certain
triangle groups, provided by Markus Pfeiffer (St Andrews). A \emph{triangle group} is a group defined by a presentation of the form 
\[ \langle x, y | x^p, y^q, (xy)^r\rangle \]
 for some positive integers $p, q, r$. The file contains generators of the syntactic semigroups of word acceptors
of triangle groups where $p$ ranges from $1$ to $94$, $q=3$, and $r=2$; $p=101$, $q$ ranges from $3$ to $99$ and $r=2$; $p=101$, $q=72$, and $r$ ranges from $7$ to $71$; and some further randomly chosen values of $p,q,r$. 
\item[{Endomorphisms of groups}]  the files \texttt{nonabelian{\textunderscore}groups{\textunderscore}\mbox{\texttt{\mdseries\slshape n}}.semigroups.gz} with $n=6,....,64$ contains small generating sets for the endomorphism monoids of all non-abelian
groups with \mbox{\texttt{\mdseries\slshape n}} elements. These files were created using the Small Groups Library in \textsf{GAP} and the \textsf{Sonata} function \texttt{Endomorphisms}. 
\end{description}
 }

 }

 
\chapter{\textcolor{Chapter }{ Determining the structure of a semigroup }}\label{green}
\logpage{[ 3, 0, 0 ]}
\hyperdef{L}{X86BAA72482D1C658}{}
{
  In this chapter we describe the functions in \textsf{Semigroups} for determining the structure of a semigroup, in particular for computing
Green's classes and related properties of semigroups. 
\section{\textcolor{Chapter }{Expressing semigroup elements as words in generators}}\logpage{[ 3, 1, 0 ]}
\hyperdef{L}{X81CEB3717E021643}{}
{
  It is possible to express an element of a semigroup as a word in the
generators of that semigroup. This section describes how to accomplish this in \textsf{Semigroups}. 

 

\subsection{\textcolor{Chapter }{EvaluateWord}}
\logpage{[ 3, 1, 1 ]}\nobreak
\hyperdef{L}{X799D2F3C866B9AED}{}
{\noindent\textcolor{FuncColor}{$\triangleright$\ \ \texttt{EvaluateWord({\mdseries\slshape gens, w})\index{EvaluateWord@\texttt{EvaluateWord}}
\label{EvaluateWord}
}\hfill{\scriptsize (operation)}}\\
\textbf{\indent Returns:\ }
A semigroup element.



 
\begin{description}
\item[{for elements of a semigroup}]  When \mbox{\texttt{\mdseries\slshape gens}} is a list of elements of a semigroup and \mbox{\texttt{\mdseries\slshape w}} is a list of positive integers less than or equal to the length of \mbox{\texttt{\mdseries\slshape gens}}, this operation returns the product \texttt{gens[w[1]]*gens[w[2]]*...*gens[w[n]]} when the length of \mbox{\texttt{\mdseries\slshape w}} is \texttt{n}. 
\item[{for elements of an inverse semigroup}]  When \mbox{\texttt{\mdseries\slshape gens}} is a list of elements with a semigroup inverse and \mbox{\texttt{\mdseries\slshape w}} is a list of non-zero integers whose absolute value does not exceed the length
of \mbox{\texttt{\mdseries\slshape gens}}, this operation returns the product \texttt{gens[AbsInt(w[1])]\texttt{\symbol{94}}SignInt(w[1])*...*gens[AbsInt(w[n])]\texttt{\symbol{94}}SignInt(w[n])} where \texttt{n} is the length of \mbox{\texttt{\mdseries\slshape w}}. 
\end{description}
 Note that \texttt{EvaluateWord(\mbox{\texttt{\mdseries\slshape gens}}, [])} returns \texttt{One(\mbox{\texttt{\mdseries\slshape gens}})} if \mbox{\texttt{\mdseries\slshape gens}} is belongs to the category \texttt{IsMultiplicativeElementWithOne} (\textbf{Reference: IsMultiplicativeElementWithOne}). 
\begin{Verbatim}[commandchars=!@|,fontsize=\small,frame=single,label=Example]
  !gapprompt@gap>| !gapinput@gens:=[ Transformation( [ 2, 4, 4, 6, 8, 8, 6, 6 ] ), |
  !gapprompt@>| !gapinput@Transformation( [ 2, 7, 4, 1, 4, 6, 5, 2 ] ), |
  !gapprompt@>| !gapinput@Transformation( [ 3, 6, 2, 4, 2, 2, 2, 8 ] ), |
  !gapprompt@>| !gapinput@Transformation( [ 4, 3, 6, 4, 2, 1, 2, 6 ] ), |
  !gapprompt@>| !gapinput@Transformation( [ 4, 5, 1, 3, 8, 5, 8, 2 ] ) ];;|
  !gapprompt@gap>| !gapinput@S:=Semigroup(gens);;|
  !gapprompt@gap>| !gapinput@f:=Transformation( [ 1, 4, 6, 1, 7, 2, 7, 6 ] );;|
  !gapprompt@gap>| !gapinput@Factorization(S, f);|
  [ 4, 2 ]
  !gapprompt@gap>| !gapinput@EvaluateWord(gens, last);|
  Transformation( [ 1, 4, 6, 1, 7, 2, 7, 6 ] )
  !gapprompt@gap>| !gapinput@S:=SymmetricInverseMonoid(10);;|
  !gapprompt@gap>| !gapinput@f:=PartialPerm( [ 1, 2, 3, 6, 8, 10 ], [ 2, 6, 7, 9, 1, 5 ] );|
  [3,7][8,1,2,6,9][10,5]
  !gapprompt@gap>| !gapinput@Factorization(S, f);|
  [ -2, -2, -2, -2, -3, -4, -3, -2, -2, -2, -2, -3, -2, 2, 2, 2, 2, 4, 
    4, 4, 4, 2, 2, 2, 2, 2, 3, 4, -3, -2, -3, -2, -3, -2, 2, 2, 2, 2, 
    2, 3, 4, -3, -2, -3, -2, -3, -2, 2, 2, 2, 2, 2, 3, 4, -3, -2, -3, 
    -2, -3, -2, 2, 2, 2, 2, 2, 3, 4, -3, -2, -3, -2, -3, -2, 2, 2, 2, 
    2, 2, 3, 4, -3, -2, -3, -2, -3, -2, 3, 2, 2, 2, 2, 2, 3, 4, -3, -2, 
    -3, -2, -3, -2, 2, 3, 2, 3, 2, 2, 2, 3, 2, 2, 2, 2, 2, 3, 2, 3, 2 ]
  !gapprompt@gap>| !gapinput@EvaluateWord(GeneratorsOfSemigroup(S), last); |
  [3,7][8,1,2,6,9][10,5]
\end{Verbatim}
 }

 

\subsection{\textcolor{Chapter }{Factorization}}
\logpage{[ 3, 1, 2 ]}\nobreak
\hyperdef{L}{X8357294D7B164106}{}
{\noindent\textcolor{FuncColor}{$\triangleright$\ \ \texttt{Factorization({\mdseries\slshape S, f})\index{Factorization@\texttt{Factorization}}
\label{Factorization}
}\hfill{\scriptsize (method)}}\\
\textbf{\indent Returns:\ }
A word in the generators.



 
\begin{description}
\item[{for semigroups}]  When \mbox{\texttt{\mdseries\slshape S}} is a semigroup and \mbox{\texttt{\mdseries\slshape f}} belongs to \mbox{\texttt{\mdseries\slshape S}}, \texttt{Factorization} return a word in the generators of \mbox{\texttt{\mdseries\slshape S}} that is equal to \mbox{\texttt{\mdseries\slshape f}}. In this case, a word is a list of positive integers where \texttt{i} corresponds to \texttt{GeneratorsOfSemigroups(S)[i]}. More specifically, 
\begin{Verbatim}[commandchars=!@|,fontsize=\small,frame=single,label=Example]
  EvaluateWord(GeneratorsOfSemigroup(S), Factorization(S, f))=f;
\end{Verbatim}
 
\item[{for inverse semigroups}]  When \mbox{\texttt{\mdseries\slshape S}} is a inverse semigroup and \mbox{\texttt{\mdseries\slshape f}} belongs to \mbox{\texttt{\mdseries\slshape S}}, \texttt{Factorization} return a word in the generators of \mbox{\texttt{\mdseries\slshape S}} that is equal to \mbox{\texttt{\mdseries\slshape f}}. In this case, a word is a list of non-zero integers where \texttt{i} corresponds to \texttt{GeneratorsOfSemigroups(S)[i]} and \texttt{-i} corresponds to \texttt{GeneratorsOfSemigroups(S)[i]\texttt{\symbol{94}}-1}. As in the previous case, 
\begin{Verbatim}[commandchars=!@|,fontsize=\small,frame=single,label=Example]
  EvaluateWord(GeneratorsOfSemigroup(S), Factorization(S, f))=f;
\end{Verbatim}
 
\end{description}
 Note that \texttt{Factorization} does not return a word of minimum length. 

 See also \texttt{EvaluateWord} (\ref{EvaluateWord}) and \texttt{GeneratorsOfSemigroup} (\textbf{Reference: GeneratorsOfSemigroup}). 
\begin{Verbatim}[commandchars=!@|,fontsize=\small,frame=single,label=Example]
  !gapprompt@gap>| !gapinput@gens:=[ Transformation( [ 2, 2, 9, 7, 4, 9, 5, 5, 4, 8 ] ), |
  !gapprompt@>| !gapinput@Transformation( [ 4, 10, 5, 6, 4, 1, 2, 7, 1, 2 ] ) ];;|
  !gapprompt@gap>| !gapinput@S:=Semigroup(gens);;|
  !gapprompt@gap>| !gapinput@f:=Transformation( [ 1, 10, 2, 10, 1, 2, 7, 10, 2, 7 ] );;|
  !gapprompt@gap>| !gapinput@Factorization(S, f);|
  [ 2, 2, 1, 2 ]
  !gapprompt@gap>| !gapinput@EvaluateWord(gens, last);|
  Transformation( [ 1, 10, 2, 10, 1, 2, 7, 10, 2, 7 ] )
  !gapprompt@gap>| !gapinput@S:=SymmetricInverseMonoid(8);|
  <symmetric inverse semigroup on 8 pts>
  !gapprompt@gap>| !gapinput@f:=PartialPerm( [ 1, 2, 3, 4, 5, 8 ], [ 7, 1, 4, 3, 2, 6 ] );|
  [5,2,1,7][8,6](3,4)
  !gapprompt@gap>| !gapinput@Factorization(S, f);|
  [ -2, -2, -2, -2, -2, -2, -2, 2, 2, 4, 4, 2, 3, 2, 3, -2, -2, -2, 2, 
    3, 2, 3, -2, -2, -2, 2, 3, 2, 3, -2, -2, -2, 3, 2, 3, 2, 3, -2, -2, 
    -2, 3, 2, 3, 2, 3, -2, -2, -2, 2, 3, 2, 3, -2, -2, -2, 2, 3, 2, 3, 
    -2, -2, -2, 2, 3, 2, 3, -2, -2, -2, 2, 3, 2, 3, -2, -2, -2, 3, 2, 
    3, 2, 3, -2, -2, -2, 2, 3, 2, 3, -2, -2, -2, 2, 3, 2, 3, -2, -2, 
    -2, 2, 3, 2, 3, -2, -2, -2, 2, 3, 2, 2, 3, 2, 2, 2, 2 ]
  !gapprompt@gap>| !gapinput@EvaluateWord(GeneratorsOfSemigroup(S), last);|
  [5,2,1,7][8,6](3,4)
\end{Verbatim}
 }

 }

 
\section{\textcolor{Chapter }{Creating Green's classes}}\logpage{[ 3, 2, 0 ]}
\hyperdef{L}{X7D14B6A080BC189E}{}
{
  
\subsection{\textcolor{Chapter }{XClassOfYClass}}\logpage{[ 3, 2, 1 ]}
\hyperdef{L}{X87558FEF805D24E1}{}
{
\noindent\textcolor{FuncColor}{$\triangleright$\ \ \texttt{DClassOfHClass({\mdseries\slshape class})\index{DClassOfHClass@\texttt{DClassOfHClass}}
\label{DClassOfHClass}
}\hfill{\scriptsize (method)}}\\
\noindent\textcolor{FuncColor}{$\triangleright$\ \ \texttt{DClassOfLClass({\mdseries\slshape class})\index{DClassOfLClass@\texttt{DClassOfLClass}}
\label{DClassOfLClass}
}\hfill{\scriptsize (method)}}\\
\noindent\textcolor{FuncColor}{$\triangleright$\ \ \texttt{DClassOfRClass({\mdseries\slshape class})\index{DClassOfRClass@\texttt{DClassOfRClass}}
\label{DClassOfRClass}
}\hfill{\scriptsize (method)}}\\
\noindent\textcolor{FuncColor}{$\triangleright$\ \ \texttt{LClassOfHClass({\mdseries\slshape class})\index{LClassOfHClass@\texttt{LClassOfHClass}}
\label{LClassOfHClass}
}\hfill{\scriptsize (method)}}\\
\noindent\textcolor{FuncColor}{$\triangleright$\ \ \texttt{RClassOfHClass({\mdseries\slshape class})\index{RClassOfHClass@\texttt{RClassOfHClass}}
\label{RClassOfHClass}
}\hfill{\scriptsize (method)}}\\
\textbf{\indent Returns:\ }
A Green's class.



 \texttt{XClassOfYClass} returns the \texttt{X}-class containing the \texttt{Y}-class \mbox{\texttt{\mdseries\slshape class}} where \texttt{X} and \texttt{Y} should be replaced by an appropriate choice of \texttt{D, H, L,} and \texttt{R}.

 Note that if it is not known to \textsf{GAP} whether or not the representative of \mbox{\texttt{\mdseries\slshape class}} is an element of the semigroup containing \mbox{\texttt{\mdseries\slshape class}}, then no attempt is made to check this.

 The same result can be produced using: 
\begin{Verbatim}[commandchars=!@|,fontsize=\small,frame=single,label=Example]
  First(GreensXClasses(S), x-> Representative(x) in class);
\end{Verbatim}
 but this might be substantially slower. Note that \texttt{XClassOfYClass} is also likely to be faster than 
\begin{Verbatim}[commandchars=!@|,fontsize=\small,frame=single,label=Example]
  GreensXClassOfElement(S, Representative(class));
\end{Verbatim}
 

 \texttt{DClass} can also be used as a synonym for \texttt{DClassOfHClass}, \texttt{DClassOfLClass}, and \texttt{DClassOfRClass}; \texttt{LClass} as a synonym for \texttt{LClassOfHClass}; and \texttt{RClass} as a synonym for \texttt{RClassOfHClass}. See also \texttt{GreensDClassOfElement} (\textbf{Reference: GreensDClassOfElement}) and \texttt{GreensDClassOfElementNC} (\ref{GreensDClassOfElementNC}). 
\begin{Verbatim}[commandchars=!@|,fontsize=\small,frame=single,label=Example]
  !gapprompt@gap>| !gapinput@S:=Semigroup(Transformation( [ 1, 3, 2 ] ), |
  !gapprompt@>| !gapinput@Transformation( [ 2, 1, 3 ] ), Transformation( [ 3, 2, 1 ] ), |
  !gapprompt@>| !gapinput@Transformation( [ 1, 3, 1 ] ) );;|
  !gapprompt@gap>| !gapinput@R:=GreensRClassOfElement(S, Transformation( [ 3, 2, 1 ] ));|
  {Transformation( [ 1, 3, 2 ] )}
  !gapprompt@gap>| !gapinput@DClassOfRClass(R);|
  {Transformation( [ 1, 3, 2 ] )}
  !gapprompt@gap>| !gapinput@IsGreensDClass(DClassOfRClass(R));|
  true
  !gapprompt@gap>| !gapinput@S:=InverseSemigroup(PartialPerm([ 1, 2, 3, 6, 8, 10 ],|
  !gapprompt@>| !gapinput@[ 2, 6, 7, 9, 1, 5 ]), PartialPerm([ 1, 2, 3, 4, 6, 7, 8, 10 ],|
  !gapprompt@>| !gapinput@[ 3, 8, 1, 9, 4, 10, 5, 6 ]));|
  <inverse partial perm semigroup on 10 pts with 2 generators>
  !gapprompt@gap>| !gapinput@f:=Generators(S)[1];|
  [3,7][8,1,2,6,9][10,5]
  !gapprompt@gap>| !gapinput@h:=HClass(S, f);|
  {PartialPerm( [ 1, 2, 3, 6, 8, 10 ], [ 2, 6, 7, 9, 1, 5 ] )}
  !gapprompt@gap>| !gapinput@r:=RClassOfHClass(h);|
  {PartialPerm( [ 1, 2, 3, 6, 8, 10 ], [ 2, 6, 7, 9, 1, 5 ] )}
  !gapprompt@gap>| !gapinput@l:=LClass(h);|
  {PartialPerm( [ 1, 2, 5, 6, 7, 9 ], [ 1, 2, 5, 6, 7, 9 ] )}
  !gapprompt@gap>| !gapinput@DClass(r)=DClass(l);|
  true
  !gapprompt@gap>| !gapinput@DClass(h)=DClass(l);|
  true
\end{Verbatim}
 }

 
\subsection{\textcolor{Chapter }{GreensXClassOfElement}}\logpage{[ 3, 2, 2 ]}
\hyperdef{L}{X81B7AD4C7C552867}{}
{
\noindent\textcolor{FuncColor}{$\triangleright$\ \ \texttt{GreensDClassOfElement({\mdseries\slshape X, f})\index{GreensDClassOfElement@\texttt{GreensDClassOfElement}}
\label{GreensDClassOfElement}
}\hfill{\scriptsize (operation)}}\\
\noindent\textcolor{FuncColor}{$\triangleright$\ \ \texttt{DClass({\mdseries\slshape X, f})\index{DClass@\texttt{DClass}}
\label{DClass}
}\hfill{\scriptsize (function)}}\\
\noindent\textcolor{FuncColor}{$\triangleright$\ \ \texttt{GreensHClassOfElement({\mdseries\slshape X, f})\index{GreensHClassOfElement@\texttt{GreensHClassOfElement}}
\label{GreensHClassOfElement}
}\hfill{\scriptsize (operation)}}\\
\noindent\textcolor{FuncColor}{$\triangleright$\ \ \texttt{HClass({\mdseries\slshape X, f})\index{HClass@\texttt{HClass}}
\label{HClass}
}\hfill{\scriptsize (function)}}\\
\noindent\textcolor{FuncColor}{$\triangleright$\ \ \texttt{GreensLClassOfElement({\mdseries\slshape X, f})\index{GreensLClassOfElement@\texttt{GreensLClassOfElement}}
\label{GreensLClassOfElement}
}\hfill{\scriptsize (operation)}}\\
\noindent\textcolor{FuncColor}{$\triangleright$\ \ \texttt{LClass({\mdseries\slshape X, f})\index{LClass@\texttt{LClass}}
\label{LClass}
}\hfill{\scriptsize (function)}}\\
\noindent\textcolor{FuncColor}{$\triangleright$\ \ \texttt{GreensRClassOfElement({\mdseries\slshape X, f})\index{GreensRClassOfElement@\texttt{GreensRClassOfElement}}
\label{GreensRClassOfElement}
}\hfill{\scriptsize (operation)}}\\
\noindent\textcolor{FuncColor}{$\triangleright$\ \ \texttt{RClass({\mdseries\slshape X, f})\index{RClass@\texttt{RClass}}
\label{RClass}
}\hfill{\scriptsize (function)}}\\
\textbf{\indent Returns:\ }
A Green's class.



 These functions produce essentially the same output as the \textsf{GAP} library functions with the same names; see \texttt{GreensDClassOfElement} (\textbf{Reference: GreensDClassOfElement}). The main difference is that these functions can be applied to a wider class
of objects: 
\begin{description}
\item[{\texttt{GreensDClassOfElement} and \texttt{DClass}}]  \mbox{\texttt{\mdseries\slshape X}} must be a semigroup. 
\item[{\texttt{GreensHClassOfElement} and \texttt{HClass}}]  \mbox{\texttt{\mdseries\slshape X}} can be a semigroup, $\mathcal{R}$-class, $\mathcal{L}$-class, or $\mathcal{D}$-class. 
\item[{\texttt{GreensLClassOfElement} and \texttt{LClass}}]  \mbox{\texttt{\mdseries\slshape X}} can be a semigroup or $\mathcal{D}$-class. 
\item[{\texttt{GreensRClassOfElement} and \texttt{RClass}}]  \mbox{\texttt{\mdseries\slshape X}} can be a semigroup or $\mathcal{D}$-class. 
\end{description}
 Note that \texttt{GreensXClassOfElement} and \texttt{XClass} are synonyms and have identical output. The shorter command is provided for
the sake of convenience.

 }

 
\subsection{\textcolor{Chapter }{GreensXClassOfElementNC}}\logpage{[ 3, 2, 3 ]}
\hyperdef{L}{X7B44317786571F8B}{}
{
\noindent\textcolor{FuncColor}{$\triangleright$\ \ \texttt{GreensDClassOfElementNC({\mdseries\slshape X, f})\index{GreensDClassOfElementNC@\texttt{GreensDClassOfElementNC}}
\label{GreensDClassOfElementNC}
}\hfill{\scriptsize (operation)}}\\
\noindent\textcolor{FuncColor}{$\triangleright$\ \ \texttt{DClassNC({\mdseries\slshape X, f})\index{DClassNC@\texttt{DClassNC}}
\label{DClassNC}
}\hfill{\scriptsize (function)}}\\
\noindent\textcolor{FuncColor}{$\triangleright$\ \ \texttt{GreensHClassOfElementNC({\mdseries\slshape X, f})\index{GreensHClassOfElementNC@\texttt{GreensHClassOfElementNC}}
\label{GreensHClassOfElementNC}
}\hfill{\scriptsize (operation)}}\\
\noindent\textcolor{FuncColor}{$\triangleright$\ \ \texttt{HClassNC({\mdseries\slshape X, f})\index{HClassNC@\texttt{HClassNC}}
\label{HClassNC}
}\hfill{\scriptsize (function)}}\\
\noindent\textcolor{FuncColor}{$\triangleright$\ \ \texttt{GreensLClassOfElementNC({\mdseries\slshape X, f})\index{GreensLClassOfElementNC@\texttt{GreensLClassOfElementNC}}
\label{GreensLClassOfElementNC}
}\hfill{\scriptsize (operation)}}\\
\noindent\textcolor{FuncColor}{$\triangleright$\ \ \texttt{LClassNC({\mdseries\slshape X, f})\index{LClassNC@\texttt{LClassNC}}
\label{LClassNC}
}\hfill{\scriptsize (function)}}\\
\noindent\textcolor{FuncColor}{$\triangleright$\ \ \texttt{GreensRClassOfElementNC({\mdseries\slshape X, f})\index{GreensRClassOfElementNC@\texttt{GreensRClassOfElementNC}}
\label{GreensRClassOfElementNC}
}\hfill{\scriptsize (operation)}}\\
\noindent\textcolor{FuncColor}{$\triangleright$\ \ \texttt{RClassNC({\mdseries\slshape X, f})\index{RClassNC@\texttt{RClassNC}}
\label{RClassNC}
}\hfill{\scriptsize (function)}}\\
\textbf{\indent Returns:\ }
A Green's class.



 These functions are essentially the same as \texttt{GreensDClassOfElement} (\ref{GreensDClassOfElement}) except that no effort is made to verify if \mbox{\texttt{\mdseries\slshape f}} is an element of \mbox{\texttt{\mdseries\slshape X}}. More precisely, \texttt{GreensXClassOfElementNC} and \texttt{XClassNC} first check if \mbox{\texttt{\mdseries\slshape f}} has already been shown to be an element of \mbox{\texttt{\mdseries\slshape X}}. If it is not known to \textsf{GAP} if \mbox{\texttt{\mdseries\slshape f}} is an element of \mbox{\texttt{\mdseries\slshape X}}, then no further attempt to verify this is made. 

 Note that \texttt{GreensXClassOfElementNC} and \texttt{XClassNC} are synonyms and have identical output. The shorter command is provided for
the sake of convenience. 

 It can be quicker to compute the class of an element using \texttt{GreensRClassOfElementNC}, say, than using \texttt{GreensRClassOfElement} if it is known \emph{a priori} that \mbox{\texttt{\mdseries\slshape f}} is an element of \mbox{\texttt{\mdseries\slshape X}}. On the other hand, if \mbox{\texttt{\mdseries\slshape f}} is not an element of \mbox{\texttt{\mdseries\slshape X}}, then the results of this computation are unpredictable.

 For example, if 
\begin{Verbatim}[commandchars=!@|,fontsize=\small,frame=single,label=Example]
  f:=Transformation( [ 15, 18, 20, 20, 20, 20, 20, 20, 20, 20, 20, 20, 20, 20, 
        20, 20, 20, 20, 20, 20 ] );
\end{Verbatim}
 in the semigroup \mbox{\texttt{\mdseries\slshape X}} of order-preserving mappings on 20 points, then 
\begin{Verbatim}[commandchars=!@|,fontsize=\small,frame=single,label=Example]
  GreensRClassOfElementNC(X, f);;
\end{Verbatim}
 returns an answer relatively quickly, whereas \texttt{GreensRClassOfElement} can take a signficant amount of time to return a value.

 See also \texttt{GreensRClassOfElement} (\textbf{Reference: GreensRClassOfElement}) and \texttt{RClassOfHClass} (\ref{RClassOfHClass}). 
\begin{Verbatim}[commandchars=!@|,fontsize=\small,frame=single,label=Example]
  !gapprompt@gap>| !gapinput@S:=RandomTransformationSemigroup(2,1000);;|
  !gapprompt@gap>| !gapinput@f:=[ 1, 1, 2, 2, 2, 1, 1, 1, 1, 1, 2, 2, 2, 2, 1, 1, 2, 2, 1 ];;|
  !gapprompt@gap>| !gapinput@f:=EvaluateWord(Generators(S), f);;                            |
  !gapprompt@gap>| !gapinput@R:=GreensRClassOfElementNC(S, f);;|
  !gapprompt@gap>| !gapinput@Size(R);|
  1
  !gapprompt@gap>| !gapinput@L:=GreensLClassOfElementNC(S, f);;|
  !gapprompt@gap>| !gapinput@Size(L);|
  1
  !gapprompt@gap>| !gapinput@f:=PartialPerm([ 1, 2, 3, 4, 7, 8, 9, 10 ],|
  !gapprompt@>| !gapinput@[ 2, 3, 4, 5, 6, 8, 10, 11 ]);;|
  !gapprompt@gap>| !gapinput@L:=LClass(POI(13), f);|
  {PartialPerm( [ 1, 2, 3, 4, 5, 6, 7, 8 ], [ 2, 3, 4, 5, 6, 8, 10, 11 ]
   )}
  !gapprompt@gap>| !gapinput@Size(L);|
  1287
\end{Verbatim}
 }

 

\subsection{\textcolor{Chapter }{GroupHClass}}
\logpage{[ 3, 2, 4 ]}\nobreak
\hyperdef{L}{X8723756387DD4C0F}{}
{\noindent\textcolor{FuncColor}{$\triangleright$\ \ \texttt{GroupHClass({\mdseries\slshape class})\index{GroupHClass@\texttt{GroupHClass}}
\label{GroupHClass}
}\hfill{\scriptsize (attribute)}}\\
\textbf{\indent Returns:\ }
A group $\mathcal{H}$-class of the $\mathcal{D}$-class \mbox{\texttt{\mdseries\slshape class}} if it is regular and \texttt{fail} if it is not. 



 \texttt{GroupHClass} is a synonym for \texttt{GroupHClassOfGreensDClass} (\textbf{Reference: GroupHClassOfGreensDClass}). 

 See also \texttt{IsGroupHClass} (\textbf{Reference: IsGroupHClass}), \texttt{IsRegularDClass} (\textbf{Reference: IsRegularDClass}), \texttt{IsRegularClass} (\ref{IsRegularClass}), and \texttt{IsRegularSemigroup} (\ref{IsRegularSemigroup}). 
\begin{Verbatim}[commandchars=!@|,fontsize=\small,frame=single,label=Example]
  !gapprompt@gap>| !gapinput@S:=Semigroup( Transformation( [ 2, 6, 7, 2, 6, 1, 1, 5 ] ), |
  !gapprompt@>| !gapinput@Transformation( [ 3, 8, 1, 4, 5, 6, 7, 1 ] ) );;|
  !gapprompt@gap>| !gapinput@IsRegularSemigroup(S);|
  false
  !gapprompt@gap>| !gapinput@iter:=IteratorOfDClasses(S);;|
  !gapprompt@gap>| !gapinput@repeat D:=NextIterator(iter); until IsRegularDClass(D);   |
  !gapprompt@gap>| !gapinput@D;|
  {Transformation( [ 6, 1, 1, 6, 1, 2, 2, 6 ] )}
  !gapprompt@gap>| !gapinput@NrIdempotents(D);|
  12
  !gapprompt@gap>| !gapinput@NrRClasses(D);|
  8
  !gapprompt@gap>| !gapinput@NrLClasses(D);|
  4
  !gapprompt@gap>| !gapinput@GroupHClass(D);|
  {Transformation( [ 1, 2, 2, 1, 2, 6, 6, 1 ] )}
  !gapprompt@gap>| !gapinput@GroupHClassOfGreensDClass(D);|
  {Transformation( [ 1, 2, 2, 1, 2, 6, 6, 1 ] )}
  !gapprompt@gap>| !gapinput@StructureDescription(GroupHClass(D));|
  "S3"
  !gapprompt@gap>| !gapinput@repeat D:=NextIterator(iter); until not IsRegularDClass(D);|
  !gapprompt@gap>| !gapinput@D;|
  {Transformation( [ 7, 5, 2, 2, 6, 1, 1, 2 ] )}
  !gapprompt@gap>| !gapinput@IsRegularDClass(D);|
  false
  !gapprompt@gap>| !gapinput@GroupHClass(D);|
  fail
  !gapprompt@gap>| !gapinput@s:=InverseSemigroup( [ PartialPerm( [ 1, 2, 3, 5 ], [ 2, 1, 6, 3 ] ),|
  !gapprompt@>| !gapinput@PartialPerm( [ 1, 2, 3, 6 ], [ 3, 5, 2, 6 ] ) ]);;|
  !gapprompt@gap>| !gapinput@f:=PartialPerm([ 1 .. 3 ], [ 6, 3, 1 ]);;|
  !gapprompt@gap>| !gapinput@First(DClasses(s), x-> not IsTrivial(GroupHClass(x)));|
  {PartialPerm( [ 1, 2 ], [ 1, 2 ] )}
  !gapprompt@gap>| !gapinput@StructureDescription(GroupHClass(last));|
  "C2"
\end{Verbatim}
 }

 }

 
\section{\textcolor{Chapter }{Iterators and enumerators of classes and representatives  }}\logpage{[ 3, 3, 0 ]}
\hyperdef{L}{X819CCBD67FD27115}{}
{
 
\subsection{\textcolor{Chapter }{GreensXClasses}}\label{GreensXClasses}
\logpage{[ 3, 3, 1 ]}
\hyperdef{L}{X7D51218A80234DE5}{}
{
\noindent\textcolor{FuncColor}{$\triangleright$\ \ \texttt{GreensDClasses({\mdseries\slshape obj})\index{GreensDClasses@\texttt{GreensDClasses}}
\label{GreensDClasses}
}\hfill{\scriptsize (method)}}\\
\noindent\textcolor{FuncColor}{$\triangleright$\ \ \texttt{DClasses({\mdseries\slshape obj})\index{DClasses@\texttt{DClasses}}
\label{DClasses}
}\hfill{\scriptsize (method)}}\\
\noindent\textcolor{FuncColor}{$\triangleright$\ \ \texttt{GreensHClasses({\mdseries\slshape obj})\index{GreensHClasses@\texttt{GreensHClasses}}
\label{GreensHClasses}
}\hfill{\scriptsize (method)}}\\
\noindent\textcolor{FuncColor}{$\triangleright$\ \ \texttt{HClasses({\mdseries\slshape obj})\index{HClasses@\texttt{HClasses}}
\label{HClasses}
}\hfill{\scriptsize (method)}}\\
\noindent\textcolor{FuncColor}{$\triangleright$\ \ \texttt{GreensJClasses({\mdseries\slshape obj})\index{GreensJClasses@\texttt{GreensJClasses}}
\label{GreensJClasses}
}\hfill{\scriptsize (method)}}\\
\noindent\textcolor{FuncColor}{$\triangleright$\ \ \texttt{JClasses({\mdseries\slshape obj})\index{JClasses@\texttt{JClasses}}
\label{JClasses}
}\hfill{\scriptsize (method)}}\\
\noindent\textcolor{FuncColor}{$\triangleright$\ \ \texttt{GreensLClasses({\mdseries\slshape obj})\index{GreensLClasses@\texttt{GreensLClasses}}
\label{GreensLClasses}
}\hfill{\scriptsize (method)}}\\
\noindent\textcolor{FuncColor}{$\triangleright$\ \ \texttt{LClasses({\mdseries\slshape obj})\index{LClasses@\texttt{LClasses}}
\label{LClasses}
}\hfill{\scriptsize (method)}}\\
\noindent\textcolor{FuncColor}{$\triangleright$\ \ \texttt{GreensRClasses({\mdseries\slshape obj})\index{GreensRClasses@\texttt{GreensRClasses}}
\label{GreensRClasses}
}\hfill{\scriptsize (method)}}\\
\noindent\textcolor{FuncColor}{$\triangleright$\ \ \texttt{RClasses({\mdseries\slshape obj})\index{RClasses@\texttt{RClasses}}
\label{RClasses}
}\hfill{\scriptsize (method)}}\\
\textbf{\indent Returns:\ }
A list of Green's classes. 



 These functions produce essentially the same output as the \textsf{GAP} library functions with the same names; see \texttt{GreensDClasses} (\textbf{Reference: GreensDClasses}). The main difference is that these functions can be applied to a wider class
of objects: 
\begin{description}
\item[{\texttt{GreensDClasses} and \texttt{DClasses}}]  \mbox{\texttt{\mdseries\slshape X}} should be a semigroup. 
\item[{\texttt{GreensHClasses} and \texttt{HClasses}}]  \mbox{\texttt{\mdseries\slshape X}} can be a semigroup, $\mathcal{R}$-class, $\mathcal{L}$-class, or $\mathcal{D}$-class. 
\item[{\texttt{GreensLClasses} and \texttt{LClasses}}]  \mbox{\texttt{\mdseries\slshape X}} can be a semigroup or $\mathcal{D}$-class. 
\item[{\texttt{GreensRClasses} and \texttt{RClasses}}]  \mbox{\texttt{\mdseries\slshape X}} can be a semigroup or $\mathcal{D}$-class. 
\end{description}
 Note that \texttt{GreensXClasses} and \texttt{XClasses} are synonyms and have identical output. The shorter command is provided for
the sake of convenience.

 See also \texttt{DClassReps} (\ref{DClassReps}), \texttt{IteratorOfDClassReps} (\ref{IteratorOfDClassReps}), \texttt{IteratorOfDClasses} (\ref{IteratorOfDClasses}), and \texttt{NrDClasses} (\ref{NrDClasses}). 
\begin{Verbatim}[commandchars=!@|,fontsize=\small,frame=single,label=Example]
  !gapprompt@gap>| !gapinput@S:=Semigroup(Transformation( [ 3, 4, 4, 4 ] ), |
  !gapprompt@>| !gapinput@Transformation( [ 4, 3, 1, 2 ] ));;|
  !gapprompt@gap>| !gapinput@GreensDClasses(S);|
  [ {Transformation( [ 3, 4, 4, 4 ] )}, 
    {Transformation( [ 4, 3, 1, 2 ] )}, 
    {Transformation( [ 4, 4, 4, 4 ] )} ]
  !gapprompt@gap>| !gapinput@GreensRClasses(S);|
  [ {Transformation( [ 3, 4, 4, 4 ] )}, 
    {Transformation( [ 4, 3, 1, 2 ] )}, 
    {Transformation( [ 4, 4, 4, 4 ] )}, 
    {Transformation( [ 4, 4, 3, 4 ] )}, 
    {Transformation( [ 4, 3, 4, 4 ] )}, 
    {Transformation( [ 4, 4, 4, 3 ] )} ]
  !gapprompt@gap>| !gapinput@D:=GreensDClasses(S)[1];|
  {Transformation( [ 3, 4, 4, 4 ] )}
  !gapprompt@gap>| !gapinput@GreensLClasses(D);|
  [ {Transformation( [ 3, 4, 4, 4 ] )}, 
    {Transformation( [ 1, 2, 2, 2 ] )} ]
  !gapprompt@gap>| !gapinput@GreensRClasses(D);|
  [ {Transformation( [ 3, 4, 4, 4 ] )}, 
    {Transformation( [ 4, 4, 3, 4 ] )}, 
    {Transformation( [ 4, 3, 4, 4 ] )}, 
    {Transformation( [ 4, 4, 4, 3 ] )} ]
  !gapprompt@gap>| !gapinput@R:=GreensRClasses(D)[1];|
  {Transformation( [ 3, 4, 4, 4 ] )}
  !gapprompt@gap>| !gapinput@GreensHClasses(R);|
  [ {Transformation( [ 3, 4, 4, 4 ] )}, 
    {Transformation( [ 1, 2, 2, 2 ] )} ]
  !gapprompt@gap>| !gapinput@s:=InverseSemigroup( PartialPerm( [ 1, 2, 3 ], [ 2, 4, 1 ] ),|
  !gapprompt@>| !gapinput@PartialPerm( [ 1, 3, 4 ], [ 3, 4, 1 ] ) );;|
  !gapprompt@gap>| !gapinput@GreensDClasses(s);|
  [ {PartialPerm( [ 1, 2, 4 ], [ 1, 2, 4 ] )}, 
    {PartialPerm( [ 1, 3, 4 ], [ 1, 3, 4 ] )}, 
    {PartialPerm( [ 2, 4 ], [ 2, 4 ] )}, {PartialPerm( [ 4 ], [ 4 ] )}, 
    {PartialPerm( [  ], [  ] )} ]
  !gapprompt@gap>| !gapinput@GreensLClasses(s);|
  [ {PartialPerm( [ 1, 2, 4 ], [ 1, 2, 4 ] )}, 
    {PartialPerm( [ 1, 2, 4 ], [ 3, 1, 2 ] )}, 
    {PartialPerm( [ 1, 3, 4 ], [ 1, 3, 4 ] )}, 
    {PartialPerm( [ 2, 4 ], [ 2, 4 ] )}, 
    {PartialPerm( [ 2, 4 ], [ 3, 1 ] )}, 
    {PartialPerm( [ 2, 4 ], [ 1, 2 ] )}, 
    {PartialPerm( [ 2, 4 ], [ 3, 2 ] )}, 
    {PartialPerm( [ 2, 4 ], [ 4, 3 ] )}, 
    {PartialPerm( [ 2, 4 ], [ 1, 4 ] )}, {PartialPerm( [ 4 ], [ 4 ] )}, 
    {PartialPerm( [ 4 ], [ 1 ] )}, {PartialPerm( [ 4 ], [ 3 ] )}, 
    {PartialPerm( [ 4 ], [ 2 ] )}, {PartialPerm( [  ], [  ] )} ]
  !gapprompt@gap>| !gapinput@D:=GreensDClasses(s)[3];|
  {PartialPerm( [ 2, 4 ], [ 2, 4 ] )}
  !gapprompt@gap>| !gapinput@GreensLClasses(D);|
  [ {PartialPerm( [ 2, 4 ], [ 2, 4 ] )}, 
    {PartialPerm( [ 2, 4 ], [ 3, 1 ] )}, 
    {PartialPerm( [ 2, 4 ], [ 1, 2 ] )}, 
    {PartialPerm( [ 2, 4 ], [ 3, 2 ] )}, 
    {PartialPerm( [ 2, 4 ], [ 4, 3 ] )}, 
    {PartialPerm( [ 2, 4 ], [ 1, 4 ] )} ]
  !gapprompt@gap>| !gapinput@GreensRClasses(D);|
  [ {PartialPerm( [ 2, 4 ], [ 2, 4 ] )}, 
    {PartialPerm( [ 1, 3 ], [ 4, 2 ] )}, 
    {PartialPerm( [ 1, 2 ], [ 2, 4 ] )}, 
    {PartialPerm( [ 2, 3 ], [ 4, 2 ] )}, 
    {PartialPerm( [ 3, 4 ], [ 4, 2 ] )}, 
    {PartialPerm( [ 1, 4 ], [ 2, 4 ] )} ]
\end{Verbatim}
 }

 
\subsection{\textcolor{Chapter }{IteratorOfXClassReps}}\logpage{[ 3, 3, 2 ]}
\hyperdef{L}{X8566F84A7F6D4193}{}
{
\noindent\textcolor{FuncColor}{$\triangleright$\ \ \texttt{IteratorOfDClassReps({\mdseries\slshape S})\index{IteratorOfDClassReps@\texttt{IteratorOfDClassReps}}
\label{IteratorOfDClassReps}
}\hfill{\scriptsize (function)}}\\
\noindent\textcolor{FuncColor}{$\triangleright$\ \ \texttt{IteratorOfHClassReps({\mdseries\slshape S})\index{IteratorOfHClassReps@\texttt{IteratorOfHClassReps}}
\label{IteratorOfHClassReps}
}\hfill{\scriptsize (function)}}\\
\noindent\textcolor{FuncColor}{$\triangleright$\ \ \texttt{IteratorOfLClassReps({\mdseries\slshape S})\index{IteratorOfLClassReps@\texttt{IteratorOfLClassReps}}
\label{IteratorOfLClassReps}
}\hfill{\scriptsize (function)}}\\
\noindent\textcolor{FuncColor}{$\triangleright$\ \ \texttt{IteratorOfRClassReps({\mdseries\slshape S})\index{IteratorOfRClassReps@\texttt{IteratorOfRClassReps}}
\label{IteratorOfRClassReps}
}\hfill{\scriptsize (function)}}\\
\textbf{\indent Returns:\ }
 An iterator. 



 Returns an iterator of the representatives of the Green's classes contained in
the semigroup \mbox{\texttt{\mdseries\slshape S}}. See  (\textbf{Reference: Iterators}) for more information on iterators.

 See also \texttt{GreensRClasses} (\textbf{Reference: GreensRClasses}), \texttt{GreensRClasses} (\ref{GreensRClasses}), and \texttt{IteratorOfRClasses} (\ref{IteratorOfRClasses}).

 
\begin{Verbatim}[commandchars=!@|,fontsize=\small,frame=single,label=Example]
  !gapprompt@gap>| !gapinput@gens:=[ Transformation( [ 3, 2, 1, 5, 4 ] ), |
  !gapprompt@>| !gapinput@Transformation( [ 5, 4, 3, 2, 1 ] ), |
  !gapprompt@>| !gapinput@Transformation( [ 5, 4, 3, 2, 1 ] ), Transformation( [ 5, 5, 4, 5, 1 ] ), |
  !gapprompt@>| !gapinput@Transformation( [ 4, 5, 4, 3, 3 ] ) ];;|
  !gapprompt@gap>| !gapinput@S:=Semigroup(gens);;|
  !gapprompt@gap>| !gapinput@iter:=IteratorOfRClassReps(S);|
  <iterator of R-class reps>
  !gapprompt@gap>| !gapinput@NextIterator(iter);|
  Transformation( [ 3, 2, 1, 5, 4 ] )
  !gapprompt@gap>| !gapinput@NextIterator(iter);|
  Transformation( [ 5, 5, 4, 5, 1 ] )
  !gapprompt@gap>| !gapinput@iter;|
  <iterator of R-class reps>
  !gapprompt@gap>| !gapinput@file:=Concatenation(SemigroupsDir(), "/examples/inverse.semigroups.gz");;|
  !gapprompt@gap>| !gapinput@s:=InverseSemigroup(ReadGenerators(file, 8));|
  <inverse partial perm semigroup on 983 pts with 2 generators>
  !gapprompt@gap>| !gapinput@NrMovedPoints(s);|
  983
  !gapprompt@gap>| !gapinput@iter:=IteratorOfLClassReps(s);|
  <iterator of L-class reps>
  !gapprompt@gap>| !gapinput@NextIterator(iter);|
  <partial perm on 634 pts with degree 1000, codegree 1000>
\end{Verbatim}
 }

 
\subsection{\textcolor{Chapter }{IteratorOfXClasses}}\logpage{[ 3, 3, 3 ]}
\hyperdef{L}{X867D7B8982915960}{}
{
\noindent\textcolor{FuncColor}{$\triangleright$\ \ \texttt{IteratorOfDClasses({\mdseries\slshape S})\index{IteratorOfDClasses@\texttt{IteratorOfDClasses}}
\label{IteratorOfDClasses}
}\hfill{\scriptsize (function)}}\\
\noindent\textcolor{FuncColor}{$\triangleright$\ \ \texttt{IteratorOfHClasses({\mdseries\slshape S})\index{IteratorOfHClasses@\texttt{IteratorOfHClasses}}
\label{IteratorOfHClasses}
}\hfill{\scriptsize (function)}}\\
\noindent\textcolor{FuncColor}{$\triangleright$\ \ \texttt{IteratorOfLClasses({\mdseries\slshape S})\index{IteratorOfLClasses@\texttt{IteratorOfLClasses}}
\label{IteratorOfLClasses}
}\hfill{\scriptsize (function)}}\\
\noindent\textcolor{FuncColor}{$\triangleright$\ \ \texttt{IteratorOfRClasses({\mdseries\slshape S})\index{IteratorOfRClasses@\texttt{IteratorOfRClasses}}
\label{IteratorOfRClasses}
}\hfill{\scriptsize (function)}}\\
\textbf{\indent Returns:\ }
 An iterator. 



 Returns an iterator of the Green's classes in the semigroup \mbox{\texttt{\mdseries\slshape S}}. See  (\textbf{Reference: Iterators}) for more information on iterators.

 This function is useful if you are, for example, looking for an $\mathcal{R}$-class of a semigroup with a particular property but do not necessarily want
to compute all of the $\mathcal{R}$-classes.

 See also \texttt{GreensRClasses} (\ref{GreensRClasses}), \texttt{GreensRClasses} (\textbf{Reference: GreensRClasses}), \texttt{NrRClasses} (\ref{NrRClasses}), and \texttt{IteratorOfRClassReps} (\ref{IteratorOfRClassReps}).

 The transformation semigroup in the example below has 25147892 elements but it
only takes a fraction of a second to find a non-trivial $\mathcal{R}$-class. The inverse semigroup of partial permutations in the example below has
size 158122047816 but it only takes a fraction of a second to find an $\mathcal{R}$-class with more than 1000 elements. 
\begin{Verbatim}[commandchars=!@|,fontsize=\small,frame=single,label=Example]
  !gapprompt@gap>| !gapinput@gens:=[ Transformation( [ 2, 4, 1, 5, 4, 4, 7, 3, 8, 1 ] ),|
  !gapprompt@>| !gapinput@  Transformation( [ 3, 2, 8, 8, 4, 4, 8, 6, 5, 7 ] ),|
  !gapprompt@>| !gapinput@  Transformation( [ 4, 10, 6, 6, 1, 2, 4, 10, 9, 7 ] ),|
  !gapprompt@>| !gapinput@  Transformation( [ 6, 2, 2, 4, 9, 9, 5, 10, 1, 8 ] ),|
  !gapprompt@>| !gapinput@  Transformation( [ 6, 4, 1, 6, 6, 8, 9, 6, 2, 2 ] ),|
  !gapprompt@>| !gapinput@  Transformation( [ 6, 8, 1, 10, 6, 4, 9, 1, 9, 4 ] ),|
  !gapprompt@>| !gapinput@  Transformation( [ 8, 6, 2, 3, 3, 4, 8, 6, 2, 9 ] ),|
  !gapprompt@>| !gapinput@  Transformation( [ 9, 1, 2, 8, 1, 5, 9, 9, 9, 5 ] ),|
  !gapprompt@>| !gapinput@  Transformation( [ 9, 3, 1, 5, 10, 3, 4, 6, 10, 2 ] ),|
  !gapprompt@>| !gapinput@  Transformation( [ 10, 7, 3, 7, 1, 9, 8, 8, 4, 10 ] ) ];;|
  !gapprompt@gap>| !gapinput@S:=Semigroup(gens);;|
  !gapprompt@gap>| !gapinput@iter:=IteratorOfRClasses(S);|
  <iterator of R-classes>
  !gapprompt@gap>| !gapinput@for R in iter do|
  !gapprompt@>| !gapinput@if Size(R)>1 then break; fi;|
  !gapprompt@>| !gapinput@od;|
  !gapprompt@gap>| !gapinput@R;|
  {Transformation( [ 6, 4, 1, 6, 6, 8, 9, 6, 2, 2 ] )}
  !gapprompt@gap>| !gapinput@Size(R);|
  21600
  !gapprompt@gap>| !gapinput@S:=InverseSemigroup(|
  !gapprompt@>| !gapinput@[ PartialPerm( [ 1, 2, 3, 4, 5, 6, 7, 10, 11, 19, 20 ], |
  !gapprompt@>| !gapinput@[ 19, 4, 11, 15, 3, 20, 1, 14, 8, 13, 17 ] ),|
  !gapprompt@>| !gapinput@ PartialPerm( [ 1, 2, 3, 4, 6, 7, 8, 14, 15, 16, 17 ], |
  !gapprompt@>| !gapinput@[ 15, 14, 20, 19, 4, 5, 1, 13, 11, 10, 3 ] ),|
  !gapprompt@>| !gapinput@ PartialPerm( [ 1, 2, 4, 6, 7, 8, 9, 10, 14, 15, 18 ], |
  !gapprompt@>| !gapinput@[ 7, 2, 17, 10, 1, 19, 9, 3, 11, 16, 18 ] ),|
  !gapprompt@>| !gapinput@ PartialPerm( [ 1, 2, 3, 4, 5, 7, 8, 9, 11, 12, 13, 16 ], |
  !gapprompt@>| !gapinput@[ 8, 3, 18, 1, 4, 13, 12, 7, 19, 20, 2, 11 ] ),|
  !gapprompt@>| !gapinput@ PartialPerm( [ 1, 2, 3, 4, 5, 6, 7, 9, 11, 15, 16, 17, 20 ], |
  !gapprompt@>| !gapinput@[ 7, 17, 13, 4, 6, 9, 18, 10, 11, 19, 5, 2, 8 ] ),|
  !gapprompt@>| !gapinput@ PartialPerm( [ 1, 3, 4, 5, 6, 7, 8, 9, 10, 11, 12, 15, 18 ], |
  !gapprompt@>| !gapinput@[ 10, 20, 11, 7, 13, 8, 4, 9, 2, 18, 17, 6, 15 ] ),|
  !gapprompt@>| !gapinput@ PartialPerm( [ 1, 2, 3, 4, 5, 6, 7, 8, 9, 11, 13, 14, 17, 18 ], |
  !gapprompt@>| !gapinput@[ 10, 20, 18, 1, 14, 16, 9, 5, 15, 4, 8, 12, 19, 11 ] ),|
  !gapprompt@>| !gapinput@ PartialPerm( [ 1, 2, 3, 4, 5, 6, 7, 9, 10, 11, 12, 15, 16, 19, 20 ], |
  !gapprompt@>| !gapinput@[ 13, 6, 1, 2, 11, 7, 16, 18, 9, 10, 4, 14, 15, 5, 17 ] ),|
  !gapprompt@>| !gapinput@ PartialPerm( [ 1, 2, 3, 4, 6, 7, 8, 9, 10, 11, 12, 14, 15, 16, 20 ], |
  !gapprompt@>| !gapinput@[ 5, 3, 12, 9, 20, 15, 8, 16, 13, 1, 17, 11, 14, 10, 2 ] ),|
  !gapprompt@>| !gapinput@ PartialPerm( [ 1, 2, 3, 4, 6, 7, 8, 9, 10, 11, 13, 17, 18, 19, 20 ], |
  !gapprompt@>| !gapinput@[ 8, 3, 9, 20, 2, 12, 14, 15, 4, 18, 13, 1, 17, 19, 5 ] ) ]);;|
  !gapprompt@gap>| !gapinput@iter:=IteratorOfRClasses(S);|
  <iterator of R-classes>
  !gapprompt@gap>| !gapinput@repeat r:=NextIterator(iter); until Size(r)>1000;|
  !gapprompt@gap>| !gapinput@r;|
  {PartialPerm( [ 8, 11, 13, 15, 17, 19 ], [ 3, 5, 1, 2, 6, 7 ] )}
  !gapprompt@gap>| !gapinput@Size(r);|
  10020240
\end{Verbatim}
 }

 
\subsection{\textcolor{Chapter }{XClassReps}}\logpage{[ 3, 3, 4 ]}
\hyperdef{L}{X865387A87FAAC395}{}
{
\noindent\textcolor{FuncColor}{$\triangleright$\ \ \texttt{DClassReps({\mdseries\slshape obj})\index{DClassReps@\texttt{DClassReps}}
\label{DClassReps}
}\hfill{\scriptsize (attribute)}}\\
\noindent\textcolor{FuncColor}{$\triangleright$\ \ \texttt{HClassReps({\mdseries\slshape obj})\index{HClassReps@\texttt{HClassReps}}
\label{HClassReps}
}\hfill{\scriptsize (attribute)}}\\
\noindent\textcolor{FuncColor}{$\triangleright$\ \ \texttt{LClassReps({\mdseries\slshape obj})\index{LClassReps@\texttt{LClassReps}}
\label{LClassReps}
}\hfill{\scriptsize (attribute)}}\\
\noindent\textcolor{FuncColor}{$\triangleright$\ \ \texttt{RClassReps({\mdseries\slshape obj})\index{RClassReps@\texttt{RClassReps}}
\label{RClassReps}
}\hfill{\scriptsize (attribute)}}\\
\textbf{\indent Returns:\ }
A list of representatives.



 \texttt{XClassReps} returns a list of the representatives of the Green's classes of \mbox{\texttt{\mdseries\slshape obj}}, which can be a semigroup, $\mathcal{D}$-, $\mathcal{L}$-, or $\mathcal{R}$-class where appropriate.

 The same output can be obtained by calling, for example: 
\begin{Verbatim}[commandchars=!@|,fontsize=\small,frame=single,label=Example]
  List(GreensXClasses(obj), Representative);
\end{Verbatim}
 Note that if the Green's classes themselves are not required, then \texttt{XClassReps} will return an answer more quickly than the above, since the Green's class
objects are not created.

 See also \texttt{GreensDClasses} (\ref{GreensDClasses}), \texttt{IteratorOfDClassReps} (\ref{IteratorOfDClassReps}), \texttt{IteratorOfDClasses} (\ref{IteratorOfDClasses}), and \texttt{NrDClasses} (\ref{NrDClasses}). 
\begin{Verbatim}[commandchars=!@|,fontsize=\small,frame=single,label=Example]
  !gapprompt@gap>| !gapinput@S:=Semigroup(Transformation( [ 3, 4, 4, 4 ] ),|
  !gapprompt@>| !gapinput@Transformation( [ 4, 3, 1, 2 ] ));;|
  !gapprompt@gap>| !gapinput@DClassReps(S);|
  [ Transformation( [ 3, 4, 4, 4 ] ), Transformation( [ 4, 3, 1, 2 ] ), 
    Transformation( [ 4, 4, 4, 4 ] ) ]
  !gapprompt@gap>| !gapinput@LClassReps(S);|
  [ Transformation( [ 3, 4, 4, 4 ] ), Transformation( [ 1, 2, 2, 2 ] ), 
    Transformation( [ 4, 3, 1, 2 ] ), Transformation( [ 4, 4, 4, 4 ] ), 
    Transformation( [ 2, 2, 2, 2 ] ), Transformation( [ 3, 3, 3, 3 ] ), 
    Transformation( [ 1, 1, 1, 1 ] ) ]
  !gapprompt@gap>| !gapinput@D:=GreensDClasses(S)[1];|
  {Transformation( [ 3, 4, 4, 4 ] )}
  !gapprompt@gap>| !gapinput@LClassReps(D);|
  [ Transformation( [ 3, 4, 4, 4 ] ), Transformation( [ 1, 2, 2, 2 ] ) ]
  !gapprompt@gap>| !gapinput@RClassReps(D);|
  [ Transformation( [ 3, 4, 4, 4 ] ), Transformation( [ 4, 4, 3, 4 ] ), 
    Transformation( [ 4, 3, 4, 4 ] ), Transformation( [ 4, 4, 4, 3 ] ) ]
  !gapprompt@gap>| !gapinput@R:=GreensRClasses(D)[1];;|
  !gapprompt@gap>| !gapinput@HClassReps(R);|
  [ Transformation( [ 3, 4, 4, 4 ] ), Transformation( [ 1, 2, 2, 2 ] ) ]
  !gapprompt@gap>| !gapinput@S:=SymmetricInverseSemigroup(6);;|
  !gapprompt@gap>| !gapinput@e:=InverseSemigroup(Idempotents(S));;|
  !gapprompt@gap>| !gapinput@M:=MunnSemigroup(e);;|
  !gapprompt@gap>| !gapinput@DClassReps(M);|
  [ <identity partial perm on [ 51 ]>, 
    <identity partial perm on [ 27, 51 ]>, 
    <identity partial perm on [ 15, 27, 50, 51 ]>, 
    <identity partial perm on [ 8, 15, 26, 27, 49, 50, 51, 64 ]>, 
    <identity partial perm on 
      [ 4, 8, 14, 15, 25, 26, 27, 48, 49, 50, 51, 60, 61, 62, 63, 64 ]>,
    <identity partial perm on 
      [ 2, 4, 7, 8, 13, 14, 15, 21, 25, 26, 27, 29, 34, 39, 44, 48, 49, \
  50, 51, 52, 53, 54, 55, 56, 57, 58, 59, 60, 61, 62, 63, 64 ]>, 
    <identity partial perm on 
      [ 1, 2, 3, 4, 5, 6, 7, 8, 9, 10, 11, 12, 13, 14, 15, 16, 17, 18, 1\
  9, 20, 21, 22, 23, 24, 25, 26, 27, 28, 29, 30, 31, 32, 33, 34, 35, 36,\
   37, 38, 39, 40, 41, 42, 43, 44, 45, 46, 47, 48, 49, 50, 51, 52, 53, 5\
  4, 55, 56, 57, 58, 59, 60, 61, 62, 63, 64 ]> ]
  !gapprompt@gap>| !gapinput@L:=LClassNC(M, PartialPerm( [ 51, 63 ] , [ 51, 47 ] ));;|
  !gapprompt@gap>| !gapinput@HClassReps(L);|
  [ <identity partial perm on [ 47, 51 ]>, [27,47](51), [50,47](51), 
    [59,47](51), [63,47](51), [64,47](51) ]
\end{Verbatim}
 }

 }

 
\section{\textcolor{Chapter }{Attributes and properties directly related to Green's classes}}\logpage{[ 3, 4, 0 ]}
\hyperdef{L}{X798C0DF184E51D7F}{}
{
  
\subsection{\textcolor{Chapter }{Less than for Green's classes}}\logpage{[ 3, 4, 1 ]}
\hyperdef{L}{X85F30ACF86C3A733}{}
{
\noindent\textcolor{FuncColor}{$\triangleright$\ \ \texttt{\texttt{\symbol{92}}{\textless}({\mdseries\slshape left-expr, right-expr})\index{<@\texttt{\texttt{\symbol{92}}{\textless}}!for Green's classes}
\label{<:for Green's classes}
}\hfill{\scriptsize (method)}}\\
\textbf{\indent Returns:\ }
\texttt{true} or \texttt{false}.



 The Green's class \mbox{\texttt{\mdseries\slshape left-expr}} is less than or equal to \mbox{\texttt{\mdseries\slshape right-expr}} if they belong to the same semigroup and the representative of \mbox{\texttt{\mdseries\slshape left-expr}} is less than the representative of \mbox{\texttt{\mdseries\slshape right-expr}} under \texttt{{\textless}}; see also \texttt{Representative} (\textbf{Reference: Representative}).

 Please note that this is not the usual order on the Green's classes of a
semigroup as defined in  (\textbf{Reference: Green's Relations}). See also \texttt{IsGreensLessThanOrEqual} (\textbf{Reference: IsGreensLessThanOrEqual}). 
\begin{Verbatim}[commandchars=!@|,fontsize=\small,frame=single,label=Example]
  !gapprompt@gap>| !gapinput@S:=FullTransformationSemigroup(4);;|
  !gapprompt@gap>| !gapinput@A:=GreensRClassOfElement(S, Transformation( [ 2, 1, 3, 1 ] ));|
  {Transformation( [ 2, 1, 3, 1 ] )}
  !gapprompt@gap>| !gapinput@B:=GreensRClassOfElement(S, Transformation( [ 1, 2, 3, 4 ] ));|
  {IdentityTransformation}
  !gapprompt@gap>| !gapinput@A<B;|
  false
  !gapprompt@gap>| !gapinput@B<A;|
  true
  !gapprompt@gap>| !gapinput@IsGreensLessThanOrEqual(A,B);|
  true
  !gapprompt@gap>| !gapinput@IsGreensLessThanOrEqual(B,A);|
  false
  !gapprompt@gap>| !gapinput@s:=SymmetricInverseSemigroup(4);;|
  !gapprompt@gap>| !gapinput@A:=GreensJClassOfElement(s, PartialPerm([ 1 .. 3 ], [ 1, 3, 4 ]) );|
  {PartialPerm( [ 1, 2, 3 ], [ 1, 2, 3 ] )}
  !gapprompt@gap>| !gapinput@B:=GreensJClassOfElement(s, PartialPerm([ 1, 2 ], [ 3, 1 ]) );|
  {PartialPerm( [ 1, 2 ], [ 1, 2 ] )}
  !gapprompt@gap>| !gapinput@A<B;|
  false
  !gapprompt@gap>| !gapinput@B<A;|
  true
  !gapprompt@gap>| !gapinput@IsGreensLessThanOrEqual(A, B);|
  false
  !gapprompt@gap>| !gapinput@IsGreensLessThanOrEqual(B, A);|
  true
\end{Verbatim}
 }

 

\subsection{\textcolor{Chapter }{InjectionPrincipalFactor}}
\logpage{[ 3, 4, 2 ]}\nobreak
\hyperdef{L}{X7EBB4F1981CC2AE9}{}
{\noindent\textcolor{FuncColor}{$\triangleright$\ \ \texttt{InjectionPrincipalFactor({\mdseries\slshape D})\index{InjectionPrincipalFactor@\texttt{InjectionPrincipalFactor}}
\label{InjectionPrincipalFactor}
}\hfill{\scriptsize (attribute)}}\\
\noindent\textcolor{FuncColor}{$\triangleright$\ \ \texttt{IsomorphismReesMatrixSemigroup({\mdseries\slshape D})\index{IsomorphismReesMatrixSemigroup@\texttt{IsomorphismReesMatrixSemigroup}}
\label{IsomorphismReesMatrixSemigroup}
}\hfill{\scriptsize (attribute)}}\\
\textbf{\indent Returns:\ }
A injective mapping.



 If the $\mathcal{D}$-class \mbox{\texttt{\mdseries\slshape D}} is a subsemigroup of \texttt{S}, then the \emph{principal factor} of \mbox{\texttt{\mdseries\slshape D}} is just \mbox{\texttt{\mdseries\slshape D}} itself. If \mbox{\texttt{\mdseries\slshape D}} is not a subsemigroup of \texttt{S}, then the principal factor of \mbox{\texttt{\mdseries\slshape D}} is the semigroup with elements \mbox{\texttt{\mdseries\slshape D}} and a new element \texttt{0} with multiplication of $x,y\in D$ defined by:  
\[ xy=\left\{\begin{array}{ll} x*y\ (\textrm{in }S)&\textrm{if }x*y\in D\\
0&\textrm{if }xy\not\in D. \end{array}\right. \]
   \texttt{InjectionPrincipalFactor} returns an injective function from the $\mathcal{D}$-class \mbox{\texttt{\mdseries\slshape D}} to a Rees matrix semigroup, which contains the principal factor of \mbox{\texttt{\mdseries\slshape D}} as a subsemigroup. 

 If \mbox{\texttt{\mdseries\slshape D}} is a subsemigroup of its parent semigroup, then the function returned by \texttt{InjectionPrincipalFactor} or \texttt{IsomorphismReesMatrixSemigroup} is an isomorphism from \mbox{\texttt{\mdseries\slshape D}} to a Rees matrix semigroup; see \texttt{ReesMatrixSemigroup} (\textbf{Reference: ReesMatrixSemigroup}).

 If \mbox{\texttt{\mdseries\slshape D}} is not a semigroup, then the function returned by \texttt{InjectionPrincipalFactor} is an injective function from \mbox{\texttt{\mdseries\slshape D}} to a Rees 0-matrix semigroup isomorphic to the principal factor of \mbox{\texttt{\mdseries\slshape D}}; see \texttt{ReesZeroMatrixSemigroup} (\textbf{Reference: ReesZeroMatrixSemigroup}). In this case, \texttt{IsomorphismReesMatrixSemigroup} returns an error.

 See also \texttt{PrincipalFactor} (\ref{PrincipalFactor}). 
\begin{Verbatim}[commandchars=!@|,fontsize=\small,frame=single,label=Example]
  !gapprompt@gap>| !gapinput@S:=InverseSemigroup(|
  !gapprompt@>| !gapinput@PartialPerm( [ 1, 2, 3, 6, 8, 10 ], [ 2, 6, 7, 9, 1, 5 ] ),|
  !gapprompt@>| !gapinput@PartialPerm( [ 1, 2, 3, 4, 6, 7, 8, 10 ], |
  !gapprompt@>| !gapinput@[ 3, 8, 1, 9, 4, 10, 5, 6 ] ) );;|
  !gapprompt@gap>| !gapinput@f:=PartialPerm([ 1, 2, 5, 6, 7, 9 ], [ 1, 2, 5, 6, 7, 9 ]);;|
  !gapprompt@gap>| !gapinput@d:=GreensDClassOfElement(S, f);|
  {PartialPerm( [ 1, 2, 5, 6, 7, 9 ], [ 1, 2, 5, 6, 7, 9 ] )}
  !gapprompt@gap>| !gapinput@InjectionPrincipalFactor(d);;|
  !gapprompt@gap>| !gapinput@rms:=Range(last);|
  <Rees 0-matrix semigroup 3x3 over Group(())>
  !gapprompt@gap>| !gapinput@MatrixOfReesZeroMatrixSemigroup(rms);|
  [ [ (), 0, 0 ], [ 0, (), 0 ], [ 0, 0, () ] ]
  !gapprompt@gap>| !gapinput@Size(rms);|
  10
  !gapprompt@gap>| !gapinput@Size(d);|
  9
\end{Verbatim}
 }

 

\subsection{\textcolor{Chapter }{PrincipalFactor}}
\logpage{[ 3, 4, 3 ]}\nobreak
\hyperdef{L}{X86C6D777847AAEC7}{}
{\noindent\textcolor{FuncColor}{$\triangleright$\ \ \texttt{PrincipalFactor({\mdseries\slshape D})\index{PrincipalFactor@\texttt{PrincipalFactor}}
\label{PrincipalFactor}
}\hfill{\scriptsize (attribute)}}\\
\textbf{\indent Returns:\ }
A Rees matrix semigroup.



 \texttt{PrincipalFactor(\mbox{\texttt{\mdseries\slshape D}})} is just shorthand for \texttt{Range(InjectionPrincipalFactor(\mbox{\texttt{\mdseries\slshape D}}))}, where \mbox{\texttt{\mdseries\slshape D}} is a $\mathcal{D}$-class of semigroup; see \texttt{InjectionPrincipalFactor} (\ref{InjectionPrincipalFactor}) for more details. 
\begin{Verbatim}[commandchars=!@|,fontsize=\small,frame=single,label=Example]
  !gapprompt@gap>| !gapinput@S:=Semigroup([ PartialPerm( [ 1, 2, 3 ], [ 1, 3, 4 ] ), |
  !gapprompt@>| !gapinput@ PartialPerm( [ 1, 2, 3 ], [ 2, 5, 3 ] ), |
  !gapprompt@>| !gapinput@ PartialPerm( [ 1, 2, 3, 4 ], [ 2, 4, 1, 5 ] ), |
  !gapprompt@>| !gapinput@ PartialPerm( [ 1, 3, 5 ], [ 5, 1, 3 ] ) ] );;|
  !gapprompt@gap>| !gapinput@PrincipalFactor(MinimalDClass(S));|
  <Rees matrix semigroup 1x1 over Group(())>
  !gapprompt@gap>| !gapinput@MultiplicativeZero(S);|
  <empty partial perm>
\end{Verbatim}
 }

 

\subsection{\textcolor{Chapter }{IsRegularClass}}
\logpage{[ 3, 4, 4 ]}\nobreak
\hyperdef{L}{X813A259E8463B3A9}{}
{\noindent\textcolor{FuncColor}{$\triangleright$\ \ \texttt{IsRegularClass({\mdseries\slshape class})\index{IsRegularClass@\texttt{IsRegularClass}}
\label{IsRegularClass}
}\hfill{\scriptsize (property)}}\\
\textbf{\indent Returns:\ }
 \texttt{true} or \texttt{false}. 



 This function returns \texttt{true} if \mbox{\texttt{\mdseries\slshape class}} is a regular Green's class and \texttt{false} if it is not. See also \texttt{IsRegularDClass} (\textbf{Reference: IsRegularDClass}), \texttt{IsGroupHClass} (\textbf{Reference: IsGroupHClass}), \texttt{GroupHClassOfGreensDClass} (\textbf{Reference: GroupHClassOfGreensDClass}), \texttt{GroupHClass} (\ref{GroupHClass}), \texttt{NrIdempotents} (\ref{NrIdempotents}), \texttt{Idempotents} (\ref{Idempotents}), and \texttt{IsRegularSemigroupElement} (\textbf{Reference: IsRegularSemigroupElement}). 

 The function \texttt{IsRegularDClass} produces the same output as the \textsf{GAP} library functions with the same name; see \texttt{IsRegularDClass} (\textbf{Reference: IsRegularDClass}). 
\begin{Verbatim}[commandchars=!@|,fontsize=\small,frame=single,label=Example]
  !gapprompt@gap>| !gapinput@S:=Monoid(Transformation( [ 10, 8, 7, 4, 1, 4, 10, 10, 7, 2 ] ),|
  !gapprompt@>| !gapinput@Transformation( [ 5, 2, 5, 5, 9, 10, 8, 3, 8, 10 ] ));;|
  !gapprompt@gap>| !gapinput@f:=Transformation( [ 1, 1, 10, 8, 8, 8, 1, 1, 10, 8 ] );;|
  !gapprompt@gap>| !gapinput@R:=RClass(S, f);;|
  !gapprompt@gap>| !gapinput@IsRegularClass(R);|
  true
  !gapprompt@gap>| !gapinput@S:=Monoid(Transformation([2,3,4,5,1,8,7,6,2,7]), |
  !gapprompt@>| !gapinput@Transformation( [ 3, 8, 7, 4, 1, 4, 3, 3, 7, 2 ] ));;|
  !gapprompt@gap>| !gapinput@f:=Transformation( [ 3, 8, 7, 4, 1, 4, 3, 3, 7, 2 ] );;|
  !gapprompt@gap>| !gapinput@R:=RClass(S, f);;|
  !gapprompt@gap>| !gapinput@IsRegularClass(R);|
  false
  !gapprompt@gap>| !gapinput@NrIdempotents(R);|
  0
  !gapprompt@gap>| !gapinput@S:=Semigroup(Transformation( [ 2, 1, 3, 1 ] ), |
  !gapprompt@>| !gapinput@Transformation( [ 3, 1, 2, 1 ] ), Transformation( [ 4, 2, 3, 3 ] ));;|
  !gapprompt@gap>| !gapinput@f:=Transformation( [ 4, 2, 3, 3 ] );;|
  !gapprompt@gap>| !gapinput@L:=GreensLClassOfElement(S, f);;|
  !gapprompt@gap>| !gapinput@IsRegularClass(L);|
  false
  !gapprompt@gap>| !gapinput@R:=GreensRClassOfElement(S, f);;|
  !gapprompt@gap>| !gapinput@IsRegularClass(R);|
  false
  !gapprompt@gap>| !gapinput@g:=Transformation( [ 4, 4, 4, 4 ] );;|
  !gapprompt@gap>| !gapinput@IsRegularSemigroupElement(S, g);|
  true
  !gapprompt@gap>| !gapinput@IsRegularClass(LClass(S, g));|
  true
  !gapprompt@gap>| !gapinput@IsRegularClass(RClass(S, g));|
  true
  !gapprompt@gap>| !gapinput@IsRegularDClass(DClass(S, g));|
  true
  !gapprompt@gap>| !gapinput@DClass(S, g)=RClass(S, g);|
  true
\end{Verbatim}
 }

 

\subsection{\textcolor{Chapter }{NrRegularDClasses}}
\logpage{[ 3, 4, 5 ]}\nobreak
\hyperdef{L}{X7AA3F0A77D0043FB}{}
{\noindent\textcolor{FuncColor}{$\triangleright$\ \ \texttt{NrRegularDClasses({\mdseries\slshape S})\index{NrRegularDClasses@\texttt{NrRegularDClasses}}
\label{NrRegularDClasses}
}\hfill{\scriptsize (attribute)}}\\
\textbf{\indent Returns:\ }
 A positive integer. 



 \texttt{NrRegularDClasses} returns the number of regular $\mathcal{D}$-classes of the semigroup \mbox{\texttt{\mdseries\slshape S}}. See also \texttt{IsRegularClass} (\ref{IsRegularClass}) and \texttt{IsRegularDClass} (\textbf{Reference: IsRegularDClass}). 
\begin{Verbatim}[commandchars=!@|,fontsize=\small,frame=single,label=Example]
  !gapprompt@gap>| !gapinput@S:=Semigroup( [ Transformation( [ 1, 3, 4, 1, 3, 5 ] ), |
  !gapprompt@>| !gapinput@Transformation( [ 5, 1, 6, 1, 6, 3 ] ) ]);;|
  !gapprompt@gap>| !gapinput@NrRegularDClasses(S); |
  3
  !gapprompt@gap>| !gapinput@NrDClasses(S); |
  7
\end{Verbatim}
 }

 
\subsection{\textcolor{Chapter }{NrXClasses}}\logpage{[ 3, 4, 6 ]}
\hyperdef{L}{X7E45FD9F7BADDFBD}{}
{
\noindent\textcolor{FuncColor}{$\triangleright$\ \ \texttt{NrDClasses({\mdseries\slshape obj})\index{NrDClasses@\texttt{NrDClasses}}
\label{NrDClasses}
}\hfill{\scriptsize (attribute)}}\\
\noindent\textcolor{FuncColor}{$\triangleright$\ \ \texttt{NrHClasses({\mdseries\slshape obj})\index{NrHClasses@\texttt{NrHClasses}}
\label{NrHClasses}
}\hfill{\scriptsize (attribute)}}\\
\noindent\textcolor{FuncColor}{$\triangleright$\ \ \texttt{NrLClasses({\mdseries\slshape obj})\index{NrLClasses@\texttt{NrLClasses}}
\label{NrLClasses}
}\hfill{\scriptsize (attribute)}}\\
\noindent\textcolor{FuncColor}{$\triangleright$\ \ \texttt{NrRClasses({\mdseries\slshape obj})\index{NrRClasses@\texttt{NrRClasses}}
\label{NrRClasses}
}\hfill{\scriptsize (attribute)}}\\
\textbf{\indent Returns:\ }
 A positive integer. 



 \texttt{NrXClasses} returns the number of Green's classes in \mbox{\texttt{\mdseries\slshape obj}} where \mbox{\texttt{\mdseries\slshape obj}} can be a semigroup, $\mathcal{D}$-, $\mathcal{L}$-, or $\mathcal{R}$-class where appropriate. If the actual Green's classes are not required, then
it is more efficient to use 
\begin{Verbatim}[commandchars=!@|,fontsize=\small,frame=single,label=Example]
  NrHClasses(obj)
\end{Verbatim}
 than 
\begin{Verbatim}[commandchars=!@|,fontsize=\small,frame=single,label=Example]
  Length(HClasses(obj))
\end{Verbatim}
 since the Green's classes themselves are not created when \texttt{NrXClasses} is called. 

 See also \texttt{GreensRClasses} (\ref{GreensRClasses}), \texttt{GreensRClasses} (\textbf{Reference: GreensRClasses}), \texttt{IteratorOfRClasses} (\ref{IteratorOfRClasses}), and \texttt{IteratorOfRClassReps} (\ref{IteratorOfRClassReps}). 
\begin{Verbatim}[commandchars=!@|,fontsize=\small,frame=single,label=Example]
  !gapprompt@gap>| !gapinput@gens:=[ Transformation( [ 1, 2, 5, 4, 3, 8, 7, 6 ] ),|
  !gapprompt@>| !gapinput@  Transformation( [ 1, 6, 3, 4, 7, 2, 5, 8 ] ),|
  !gapprompt@>| !gapinput@  Transformation( [ 2, 1, 6, 7, 8, 3, 4, 5 ] ),|
  !gapprompt@>| !gapinput@  Transformation( [ 3, 2, 3, 6, 1, 6, 1, 2 ] ),|
  !gapprompt@>| !gapinput@  Transformation( [ 5, 2, 3, 6, 3, 4, 7, 4 ] ) ];;|
  !gapprompt@gap>| !gapinput@S:=Semigroup(gens);;|
  !gapprompt@gap>| !gapinput@f:=Transformation( [ 2, 5, 4, 7, 4, 3, 6, 3 ] );;|
  !gapprompt@gap>| !gapinput@R:=RClass(S, f);|
  {Transformation( [ 5, 2, 3, 6, 3, 4, 7, 4 ] )}
  !gapprompt@gap>| !gapinput@NrHClasses(R);|
  12
  !gapprompt@gap>| !gapinput@D:=DClass(R);|
  {Transformation( [ 5, 2, 3, 6, 3, 4, 7, 4 ] )}
  !gapprompt@gap>| !gapinput@NrHClasses(D);|
  72
  !gapprompt@gap>| !gapinput@L:=LClass(S, f);|
  {Transformation( [ 2, 5, 4, 7, 4, 3, 6, 3 ] )}
  !gapprompt@gap>| !gapinput@NrHClasses(L);|
  6
  !gapprompt@gap>| !gapinput@NrHClasses(S);|
  1555
  !gapprompt@gap>| !gapinput@gens:=[ Transformation( [ 4, 6, 5, 2, 1, 3 ] ),|
  !gapprompt@>| !gapinput@  Transformation( [ 6, 3, 2, 5, 4, 1 ] ),|
  !gapprompt@>| !gapinput@  Transformation( [ 1, 2, 4, 3, 5, 6 ] ),|
  !gapprompt@>| !gapinput@  Transformation( [ 3, 5, 6, 1, 2, 3 ] ),|
  !gapprompt@>| !gapinput@  Transformation( [ 5, 3, 6, 6, 6, 2 ] ),|
  !gapprompt@>| !gapinput@  Transformation( [ 2, 3, 2, 6, 4, 6 ] ),|
  !gapprompt@>| !gapinput@  Transformation( [ 2, 1, 2, 2, 2, 4 ] ),|
  !gapprompt@>| !gapinput@  Transformation( [ 4, 4, 1, 2, 1, 2 ] ) ];;|
  !gapprompt@gap>| !gapinput@S:=Semigroup(gens);;|
  !gapprompt@gap>| !gapinput@NrRClasses(S);|
  150
  !gapprompt@gap>| !gapinput@Size(S);|
  6342
  !gapprompt@gap>| !gapinput@f:=Transformation( [ 1, 3, 3, 1, 3, 5 ] );;|
  !gapprompt@gap>| !gapinput@D:=DClass(S, f);|
  {Transformation( [ 2, 1, 1, 2, 1, 4 ] )}
  !gapprompt@gap>| !gapinput@NrRClasses(D);|
  87
  !gapprompt@gap>| !gapinput@s:=SymmetricInverseSemigroup(10);;|
  !gapprompt@gap>| !gapinput@NrDClasses(s); NrRClasses(s); NrHClasses(s); NrLClasses(s);|
  11
  1024
  184756
  1024
  !gapprompt@gap>| !gapinput@s:=POPI(10);;|
  !gapprompt@gap>| !gapinput@NrDClasses(s);|
  11
  !gapprompt@gap>| !gapinput@NrRClasses(s);|
  1024
\end{Verbatim}
 }

 

\subsection{\textcolor{Chapter }{PartialOrderOfDClasses}}
\logpage{[ 3, 4, 7 ]}\nobreak
\hyperdef{L}{X83F1C306846DF26B}{}
{\noindent\textcolor{FuncColor}{$\triangleright$\ \ \texttt{PartialOrderOfDClasses({\mdseries\slshape S})\index{PartialOrderOfDClasses@\texttt{PartialOrderOfDClasses}}
\label{PartialOrderOfDClasses}
}\hfill{\scriptsize (attribute)}}\\
\textbf{\indent Returns:\ }
The partial order of the $\mathcal{D}$-classes of \mbox{\texttt{\mdseries\slshape S}}. 



 Returns a list \texttt{l} where \texttt{j} is in \texttt{l[i]} if and only if \texttt{GreensDClasses(S)[j]} is immediately less than \texttt{GreensDClasses(S)[i]} in the partial order of $\mathcal{D}$- classes of \mbox{\texttt{\mdseries\slshape S}}. The transitive closure of the relation $\{(j,i): j\in l[i]\}$ is the partial order of $\mathcal{D}$-classes of \mbox{\texttt{\mdseries\slshape S}}. 

 The partial order on the $\mathcal{D}$-classes is defined by $x\leq y$ if and only if $S^1xS^1$ is a subset of $S^1yS^1$. 

 See also \texttt{GreensDClasses} (\ref{GreensDClasses}), \texttt{GreensDClasses} (\textbf{Reference: GreensDClasses}), \texttt{IsGreensLessThanOrEqual} (\textbf{Reference: IsGreensLessThanOrEqual}), and \texttt{\texttt{\symbol{92}}{\textless}} (\ref{<:for Green's classes}). 
\begin{Verbatim}[commandchars=!@|,fontsize=\small,frame=single,label=Example]
  !gapprompt@gap>| !gapinput@S:=Semigroup( Transformation( [ 2, 4, 1, 2 ] ), |
  !gapprompt@>| !gapinput@Transformation( [ 3, 3, 4, 1 ] ) );;|
  !gapprompt@gap>| !gapinput@PartialOrderOfDClasses(S);|
  [ [ 3 ], [ 2, 3 ], [ 3, 4 ], [ 4 ] ]
  !gapprompt@gap>| !gapinput@IsGreensLessThanOrEqual(GreensDClasses(S)[1], GreensDClasses(S)[2]);|
  false
  !gapprompt@gap>| !gapinput@IsGreensLessThanOrEqual(GreensDClasses(S)[2], GreensDClasses(S)[1]);|
  false
  !gapprompt@gap>| !gapinput@IsGreensLessThanOrEqual(GreensDClasses(S)[3], GreensDClasses(S)[1]);|
  true
  !gapprompt@gap>| !gapinput@S:=InverseSemigroup( PartialPerm( [ 1, 2, 3 ], [ 1, 3, 4 ] ),|
  !gapprompt@>| !gapinput@PartialPerm( [ 1, 3, 5 ], [ 5, 1, 3 ] ) );;|
  !gapprompt@gap>| !gapinput@Size(S);|
  58
  !gapprompt@gap>| !gapinput@PartialOrderOfDClasses(S);              |
  [ [ 1, 3 ], [ 2, 3 ], [ 3, 4 ], [ 4, 5 ], [ 5 ] ]
  !gapprompt@gap>| !gapinput@IsGreensLessThanOrEqual(GreensDClasses(S)[1], GreensDClasses(S)[2]);|
  false
  !gapprompt@gap>| !gapinput@IsGreensLessThanOrEqual(GreensDClasses(S)[5], GreensDClasses(S)[2]);|
  true
  !gapprompt@gap>| !gapinput@IsGreensLessThanOrEqual(GreensDClasses(S)[3], GreensDClasses(S)[4]);|
  false
  !gapprompt@gap>| !gapinput@IsGreensLessThanOrEqual(GreensDClasses(S)[4], GreensDClasses(S)[3]);|
  true
\end{Verbatim}
 }

 

\subsection{\textcolor{Chapter }{SchutzenbergerGroup}}
\logpage{[ 3, 4, 8 ]}\nobreak
\hyperdef{L}{X84F1321E8217D2A8}{}
{\noindent\textcolor{FuncColor}{$\triangleright$\ \ \texttt{SchutzenbergerGroup({\mdseries\slshape class})\index{SchutzenbergerGroup@\texttt{SchutzenbergerGroup}}
\label{SchutzenbergerGroup}
}\hfill{\scriptsize (attribute)}}\\
\textbf{\indent Returns:\ }
 A permutation group. 



 \texttt{SchutzenbergerGroup} returns the generalized Schutzenberger group (defined below) of the $\mathcal{R}$-, $\mathcal{D}$-, $\mathcal{L}$-, or $\mathcal{H}$-class \mbox{\texttt{\mdseries\slshape class}}. 

 If \texttt{f} is an element of a semigroup of transformations or partial permutations and \texttt{im(f)} denotes the image of \texttt{f}, then the \emph{generalized Schutzenberger group} of \texttt{im(f)} is the permutation group  
\[ \{\:g|_{\textrm{im}(f)}\::\:\textrm{im}(f*g)=\textrm{im}(f)\:\}. \]
  

 The generalized Schutzenberger group of the kernel \texttt{ker(f)} of a transformation \texttt{f} or the domain \texttt{dom(f)} of a partial permutation \texttt{f} is defined analogously. 

 The generalized Schutzenberger group of a Green's class is then defined as
follows. 
\begin{description}
\item[{$\mathcal{R}$-class}] The generalized Schutzenberger group of the image or range of the
representative of the $\mathcal{R}$-class. 
\item[{$\mathcal{L}$-class}] The generalized Schutzenberger group of the kernel or domain of the
representative of the $\mathcal{L}$-class. 
\item[{$\mathcal{H}$-class}] The intersection of the generalized Schutzenberger groups of the $\mathcal{R}$- and $\mathcal{L}$-class containing the $\mathcal{H}$-class. 
\item[{$\mathcal{D}$-class}] The intersection of the generalized Schutzenberger groups of the $\mathcal{R}$- and $\mathcal{L}$-class containing the representative of the $\mathcal{D}$-class. 
\end{description}
 
\begin{Verbatim}[commandchars=!@|,fontsize=\small,frame=single,label=Example]
  !gapprompt@gap>| !gapinput@S:=Semigroup( Transformation( [ 4, 4, 3, 5, 3 ] ), |
  !gapprompt@>| !gapinput@Transformation( [ 5, 1, 1, 4, 1 ] ), |
  !gapprompt@>| !gapinput@Transformation( [ 5, 5, 4, 4, 5 ] ) );;|
  !gapprompt@gap>| !gapinput@f:=Transformation( [ 5, 5, 4, 4, 5 ] );;|
  !gapprompt@gap>| !gapinput@SchutzenbergerGroup(RClass(S, f));|
  Group([ (4,5) ])
  !gapprompt@gap>| !gapinput@S:=InverseSemigroup(|
  !gapprompt@>| !gapinput@[ PartialPerm([ 1, 2, 3, 7 ], [ 9, 2, 4, 8 ]),|
  !gapprompt@>| !gapinput@PartialPerm([ 1, 2, 6, 7, 8, 9, 10 ], [ 6, 8, 4, 5, 9, 1, 3 ]),|
  !gapprompt@>| !gapinput@PartialPerm([ 1, 2, 3, 5, 6, 7, 8, 9 ], [ 7, 4, 1, 6, 9, 5, 2, 3 ]) ] );;|
  !gapprompt@gap>| !gapinput@List(DClasses(S), SchutzenbergerGroup);|
  [ Group(()), Group(()), Group(()), Group(()), Group([ (1,9,8), (8,
     9) ]), Group([ (4,9) ]), Group(()), Group(()), Group(()), 
    Group(()), Group(()), Group(()), Group(()), Group(()), Group(()), 
    Group(()), Group([ (2,5)(3,7) ]), Group([ (1,7,5,6,9,3) ]), 
    Group(()), Group(()), Group(()), Group(()), Group(()) ]
\end{Verbatim}
 }

 

\subsection{\textcolor{Chapter }{MinimalDClass}}
\logpage{[ 3, 4, 9 ]}\nobreak
\hyperdef{L}{X81E5A04F7DA3A1E1}{}
{\noindent\textcolor{FuncColor}{$\triangleright$\ \ \texttt{MinimalDClass({\mdseries\slshape S})\index{MinimalDClass@\texttt{MinimalDClass}}
\label{MinimalDClass}
}\hfill{\scriptsize (attribute)}}\\
\textbf{\indent Returns:\ }
The minimal $\mathcal{D}$-class of a semigroup.



 The minimal ideal of a semigroup is the least ideal with respect to
containment. \texttt{MinimalDClass} returns the $\mathcal{D}$-class corresponding to the minimal ideal of the semigroup \mbox{\texttt{\mdseries\slshape S}}. Equivalently, \texttt{MinimalDClass} returns the minimal $\mathcal{D}$-class with respect to the partial order of $\mathcal{D}$-classes.

 See also \texttt{PartialOrderOfDClasses} (\ref{PartialOrderOfDClasses}), \texttt{IsGreensLessThanOrEqual} (\textbf{Reference: IsGreensLessThanOrEqual}), and \texttt{MinimalIdeal} (\ref{MinimalIdeal}). 
\begin{Verbatim}[commandchars=!@|,fontsize=\small,frame=single,label=Example]
  !gapprompt@gap>| !gapinput@S:=InverseSemigroup( |
  !gapprompt@>| !gapinput@PartialPerm( [ 1, 2, 3, 5, 7, 8, 9 ], [ 2, 6, 9, 1, 5, 3, 8 ] ), |
  !gapprompt@>| !gapinput@PartialPerm( [ 1, 3, 4, 5, 7, 8, 9 ], [ 9, 4, 10, 5, 6, 7, 1 ] ) );;|
  !gapprompt@gap>| !gapinput@MinimalDClass(S);|
  {PartialPerm( [  ], [  ] )}
\end{Verbatim}
 }

 

\subsection{\textcolor{Chapter }{MaximalDClasses}}
\logpage{[ 3, 4, 10 ]}\nobreak
\hyperdef{L}{X81F030B27ACB141D}{}
{\noindent\textcolor{FuncColor}{$\triangleright$\ \ \texttt{MaximalDClasses({\mdseries\slshape S})\index{MaximalDClasses@\texttt{MaximalDClasses}}
\label{MaximalDClasses}
}\hfill{\scriptsize (attribute)}}\\
\textbf{\indent Returns:\ }
The maximal $\mathcal{D}$-classes of a semigroup.



 \texttt{MaximalDClasses} returns the maximal $\mathcal{D}$-classes with respect to the partial order of $\mathcal{D}$-classes. 

 See also \texttt{PartialOrderOfDClasses} (\ref{PartialOrderOfDClasses}), \texttt{IsGreensLessThanOrEqual} (\textbf{Reference: IsGreensLessThanOrEqual}), and \texttt{MinimalDClass} (\ref{MinimalDClass}). 
\begin{Verbatim}[commandchars=!@|,fontsize=\small,frame=single,label=Example]
  !gapprompt@gap>| !gapinput@MaximalDClasses(FullTransformationMonoid(5));|
  [ {IdentityTransformation} ]
  !gapprompt@gap>| !gapinput@S:=Semigroup( |
  !gapprompt@>| !gapinput@PartialPerm( [ 1, 2, 3, 4, 5, 6, 7 ], [ 3, 8, 1, 4, 5, 6, 7 ] ), |
  !gapprompt@>| !gapinput@PartialPerm( [ 1, 2, 3, 6, 8 ], [ 2, 6, 7, 1, 5 ] ), |
  !gapprompt@>| !gapinput@PartialPerm( [ 1, 2, 3, 4, 6, 8 ], [ 4, 3, 2, 7, 6, 5 ] ), |
  !gapprompt@>| !gapinput@PartialPerm( [ 1, 2, 4, 5, 6, 7, 8 ], [ 7, 1, 4, 2, 5, 6, 3 ] ) );;|
  !gapprompt@gap>| !gapinput@MaximalDClasses(S);|
  [ {PartialPerm( [ 1, 2, 3, 4, 5, 6, 7 ], [ 3, 8, 1, 4, 5, 6, 7 ] )}, 
    {PartialPerm( [ 1, 2, 4, 5, 6, 7, 8 ], [ 7, 1, 4, 2, 5, 6, 3 ] )} ]
\end{Verbatim}
 }

 

\subsection{\textcolor{Chapter }{StructureDescriptionSchutzenbergerGroups}}
\logpage{[ 3, 4, 11 ]}\nobreak
\hyperdef{L}{X81202126806443F9}{}
{\noindent\textcolor{FuncColor}{$\triangleright$\ \ \texttt{StructureDescriptionSchutzenbergerGroups({\mdseries\slshape S})\index{StructureDescriptionSchutzenbergerGroups@\texttt{Structure}\-\texttt{Description}\-\texttt{Schutzenberger}\-\texttt{Groups}}
\label{StructureDescriptionSchutzenbergerGroups}
}\hfill{\scriptsize (attribute)}}\\
\textbf{\indent Returns:\ }
Distinct structure descriptions of the Schutzenberger groups of a semigroup.



 \texttt{StructureDescriptionSchutzenbergerGroups} returns the distinct values of \texttt{StructureDescription} (\textbf{Reference: StructureDescription}) when it is applied to the Schutzenberger groups of the $\mathcal{R}$-classes of the semigroup \mbox{\texttt{\mdseries\slshape S}}. 
\begin{Verbatim}[commandchars=!@|,fontsize=\small,frame=single,label=Example]
  !gapprompt@gap>| !gapinput@S:=Semigroup( PartialPerm( [ 1, 2, 3 ], [ 2, 5, 4 ] ), |
  !gapprompt@>| !gapinput@ PartialPerm( [ 1, 2, 3 ], [ 4, 1, 2 ] ), |
  !gapprompt@>| !gapinput@ PartialPerm( [ 1, 2, 3 ], [ 5, 2, 3 ] ), |
  !gapprompt@>| !gapinput@ PartialPerm( [ 1, 2, 4, 5 ], [ 2, 1, 4, 3 ] ), |
  !gapprompt@>| !gapinput@ PartialPerm( [ 1, 2, 5 ], [ 2, 3, 5 ] ), |
  !gapprompt@>| !gapinput@ PartialPerm( [ 1, 2, 3, 5 ], [ 2, 3, 5, 4 ] ), |
  !gapprompt@>| !gapinput@ PartialPerm( [ 1, 2, 3, 5 ], [ 4, 2, 5, 1 ] ), |
  !gapprompt@>| !gapinput@ PartialPerm( [ 1, 2, 3, 5 ], [ 5, 2, 4, 3 ] ), |
  !gapprompt@>| !gapinput@ PartialPerm( [ 1, 2, 5 ], [ 5, 4, 3 ] ) );;|
  !gapprompt@gap>| !gapinput@StructureDescriptionSchutzenbergerGroups(S);            |
  [ "1", "C2", "S3" ]
\end{Verbatim}
 }

 

\subsection{\textcolor{Chapter }{StructureDescriptionMaximalSubgroups}}
\logpage{[ 3, 4, 12 ]}\nobreak
\hyperdef{L}{X838F43FE79A8C678}{}
{\noindent\textcolor{FuncColor}{$\triangleright$\ \ \texttt{StructureDescriptionMaximalSubgroups({\mdseries\slshape S})\index{StructureDescriptionMaximalSubgroups@\texttt{Structure}\-\texttt{Description}\-\texttt{Maximal}\-\texttt{Subgroups}}
\label{StructureDescriptionMaximalSubgroups}
}\hfill{\scriptsize (attribute)}}\\
\textbf{\indent Returns:\ }
Distinct structure descriptions of the maximal subgroups of a semigroup.



 \texttt{StructureDescriptionMaximalSubgroups} returns the distinct values of \texttt{StructureDescription} (\textbf{Reference: StructureDescription}) when it is applied to the maximal subgroups of the semigroup \mbox{\texttt{\mdseries\slshape S}}. 
\begin{Verbatim}[commandchars=!@|,fontsize=\small,frame=single,label=Example]
  !gapprompt@gap>| !gapinput@S:=Semigroup( PartialPerm( [ 1, 3, 4, 5, 8 ], [ 8, 3, 9, 4, 5 ] ), |
  !gapprompt@>| !gapinput@ PartialPerm( [ 1, 2, 3, 4, 8 ], [ 10, 4, 1, 9, 6 ] ), |
  !gapprompt@>| !gapinput@ PartialPerm( [ 1, 2, 3, 4, 5, 6, 7, 10 ], [ 4, 1, 6, 7, 5, 3, 2, 10 ] ), |
  !gapprompt@>| !gapinput@ PartialPerm( [ 1, 2, 3, 4, 6, 8, 10 ], [ 4, 9, 10, 3, 1, 5, 2 ] ) );;|
  !gapprompt@gap>| !gapinput@StructureDescriptionMaximalSubgroups(S);|
  [ "1", "C2", "C3", "C4" ]
\end{Verbatim}
 }

 

\subsection{\textcolor{Chapter }{MultiplicativeNeutralElement (for an H-class)}}
\logpage{[ 3, 4, 13 ]}\nobreak
\hyperdef{L}{X8459E4067C5773AD}{}
{\noindent\textcolor{FuncColor}{$\triangleright$\ \ \texttt{MultiplicativeNeutralElement({\mdseries\slshape H})\index{MultiplicativeNeutralElement@\texttt{MultiplicativeNeutralElement}!for an H-class}
\label{MultiplicativeNeutralElement:for an H-class}
}\hfill{\scriptsize (method)}}\\
\textbf{\indent Returns:\ }
A semigroup element or \texttt{fail}.



 If the $\mathcal{H}$-class \mbox{\texttt{\mdseries\slshape H}} of a semigroup \texttt{S} is a subgroup of \texttt{S}, then \texttt{MultiplicativeNeutralElement} returns the identity of \mbox{\texttt{\mdseries\slshape H}}. If \mbox{\texttt{\mdseries\slshape H}} is not a subgroup of \texttt{S}, then \texttt{fail} is returned. 
\begin{Verbatim}[commandchars=!@|,fontsize=\small,frame=single,label=Example]
  !gapprompt@gap>| !gapinput@S:=Semigroup( |
  !gapprompt@>| !gapinput@ PartialPerm( [ 1, 2, 3 ], [ 1, 5, 2 ] ), PartialPerm( [ 1, 3 ], [ 2, 4 ] ), |
  !gapprompt@>| !gapinput@ PartialPerm( [ 1, 2, 3 ], [ 4, 1, 5 ] ), |
  !gapprompt@>| !gapinput@ PartialPerm( [ 1, 3, 5 ], [ 1, 3, 4 ] ), |
  !gapprompt@>| !gapinput@ PartialPerm( [ 1, 2, 4, 5 ], [ 1, 2, 3, 5 ] ), |
  !gapprompt@>| !gapinput@ PartialPerm( [ 1, 2, 3, 5 ], [ 1, 3, 2, 5 ] ), |
  !gapprompt@>| !gapinput@ PartialPerm( [ 1, 4, 5 ], [ 5, 4, 3 ] ) );;|
  !gapprompt@gap>| !gapinput@H:=HClass(S, PartialPerm( [ 1, 2 ], [ 1, 2 ] ));;|
  !gapprompt@gap>| !gapinput@MultiplicativeNeutralElement(H);|
  <identity partial perm on [ 1, 2 ]>
  !gapprompt@gap>| !gapinput@H:=HClass(S, PartialPerm( [ 1, 2 ], [ 1, 4 ] ));;|
  !gapprompt@gap>| !gapinput@MultiplicativeNeutralElement(H);|
  fail
\end{Verbatim}
 }

 

\subsection{\textcolor{Chapter }{IsGreensClassNC}}
\logpage{[ 3, 4, 14 ]}\nobreak
\hyperdef{L}{X7E9BD34B8021045A}{}
{\noindent\textcolor{FuncColor}{$\triangleright$\ \ \texttt{IsGreensClassNC({\mdseries\slshape class})\index{IsGreensClassNC@\texttt{IsGreensClassNC}}
\label{IsGreensClassNC}
}\hfill{\scriptsize (property)}}\\
\textbf{\indent Returns:\ }
\texttt{true} or \texttt{false}.



 A Green's class \mbox{\texttt{\mdseries\slshape class}} of a semigroup \texttt{S} satisfies \texttt{IsGreensClassNC} if it was known to \textsf{GAP} that the representative of \mbox{\texttt{\mdseries\slshape class}} was an element of \texttt{S} at the point that \mbox{\texttt{\mdseries\slshape class}} was created. }

 

\subsection{\textcolor{Chapter }{IsTransformationSemigroupGreensClass}}
\logpage{[ 3, 4, 15 ]}\nobreak
\hyperdef{L}{X82EF429D862932BD}{}
{\noindent\textcolor{FuncColor}{$\triangleright$\ \ \texttt{IsTransformationSemigroupGreensClass({\mdseries\slshape class})\index{IsTransformationSemigroupGreensClass@\texttt{IsTransformation}\-\texttt{Semigroup}\-\texttt{Greens}\-\texttt{Class}}
\label{IsTransformationSemigroupGreensClass}
}\hfill{\scriptsize (property)}}\\
\textbf{\indent Returns:\ }
\texttt{true} or \texttt{false}.



 A Green's class \mbox{\texttt{\mdseries\slshape class}} of a semigroup \texttt{S} satisfies the property \texttt{IsTransformationSemigroupGreensClass} if and only if \texttt{S} is a semigroup of transformations. }

 

\subsection{\textcolor{Chapter }{IsPartialPermSemigroupGreensClass}}
\logpage{[ 3, 4, 16 ]}\nobreak
\hyperdef{L}{X785C499F852F0915}{}
{\noindent\textcolor{FuncColor}{$\triangleright$\ \ \texttt{IsPartialPermSemigroupGreensClass({\mdseries\slshape class})\index{IsPartialPermSemigroupGreensClass@\texttt{IsPartialPermSemigroupGreensClass}}
\label{IsPartialPermSemigroupGreensClass}
}\hfill{\scriptsize (property)}}\\
\textbf{\indent Returns:\ }
\texttt{true} or \texttt{false}.



 A Green's class \mbox{\texttt{\mdseries\slshape class}} of a semigroup \texttt{S} satisfies the property \texttt{IsPartialPermSemigroupGreensClass} if and only if \texttt{S} is a semigroup of partial perms. }

 

\subsection{\textcolor{Chapter }{StructureDescription (for an H-class)}}
\logpage{[ 3, 4, 17 ]}\nobreak
\hyperdef{L}{X85B34FFB82C83127}{}
{\noindent\textcolor{FuncColor}{$\triangleright$\ \ \texttt{StructureDescription({\mdseries\slshape class})\index{StructureDescription@\texttt{StructureDescription}!for an H-class}
\label{StructureDescription:for an H-class}
}\hfill{\scriptsize (method)}}\\
\textbf{\indent Returns:\ }
A string or \texttt{fail}.



 \texttt{StructureDescription} returns the value of \texttt{StructureDescription} (\textbf{Reference: StructureDescription}) when it is applied to a group isomorphic to the group $\mathcal{H}$-class \mbox{\texttt{\mdseries\slshape class}}. If \mbox{\texttt{\mdseries\slshape class}} is not a group $\mathcal{H}$-class, then \texttt{fail} is returned. 
\begin{Verbatim}[commandchars=!@|,fontsize=\small,frame=single,label=Example]
  !gapprompt@gap>| !gapinput@S:=Semigroup( |
  !gapprompt@>| !gapinput@PartialPerm( [ 1, 2, 3, 4, 6, 7, 8, 9 ], [ 1, 9, 4, 3, 5, 2, 10, 7 ] ), |
  !gapprompt@>| !gapinput@PartialPerm( [ 1, 2, 4, 7, 8, 9 ], [ 6, 2, 4, 9, 1, 3 ] ) );;|
  !gapprompt@gap>| !gapinput@H:=HClass(S, |
  !gapprompt@>| !gapinput@PartialPerm( [ 1, 2, 3, 4, 7, 9 ], [ 1, 7, 3, 4, 9, 2 ] ));;|
  !gapprompt@gap>| !gapinput@StructureDescription(H);|
  "C6"
\end{Verbatim}
 }

 

\subsection{\textcolor{Chapter }{IsGreensDLeq}}
\logpage{[ 3, 4, 18 ]}\nobreak
\hyperdef{L}{X8693E8B184934EA0}{}
{\noindent\textcolor{FuncColor}{$\triangleright$\ \ \texttt{IsGreensDLeq({\mdseries\slshape S})\index{IsGreensDLeq@\texttt{IsGreensDLeq}}
\label{IsGreensDLeq}
}\hfill{\scriptsize (attribute)}}\\
\textbf{\indent Returns:\ }
A function.



 \texttt{IsGreensDLeq(\mbox{\texttt{\mdseries\slshape S}})} returns a function \texttt{func} such that for any two elements \texttt{x} and \texttt{y} of \mbox{\texttt{\mdseries\slshape S}}, \texttt{func(x, y)} return \texttt{true} if the $\mathcal{D}$-class of \texttt{x} in \mbox{\texttt{\mdseries\slshape S}} is greater than or equal to the $\mathcal{D}$-class of \texttt{y} in \mbox{\texttt{\mdseries\slshape S}} under the usual ordering of Green's $\mathcal{D}$-classes of a semigroup. 
\begin{Verbatim}[commandchars=!@|,fontsize=\small,frame=single,label=Example]
  !gapprompt@gap>| !gapinput@ S:=Semigroup( [ Transformation( [ 1, 3, 4, 1, 3 ] ), |
  !gapprompt@>| !gapinput@ Transformation( [ 2, 4, 1, 5, 5 ] ), |
  !gapprompt@>| !gapinput@ Transformation( [ 2, 5, 3, 5, 3 ] ), |
  !gapprompt@>| !gapinput@ Transformation( [ 5, 5, 1, 1, 3 ] ) ] );;|
  !gapprompt@gap>| !gapinput@reps:=ShallowCopy(DClassReps(S));|
  [ Transformation( [ 1, 3, 4, 1, 3 ] ), 
    Transformation( [ 2, 4, 1, 5, 5 ] ), 
    Transformation( [ 1, 4, 1, 1, 4 ] ), 
    Transformation( [ 1, 1, 1, 1, 1 ] ) ]
  !gapprompt@gap>| !gapinput@Sort(reps, IsGreensDLeq(S));|
  !gapprompt@gap>| !gapinput@reps;|
  [ Transformation( [ 2, 4, 1, 5, 5 ] ), 
    Transformation( [ 1, 3, 4, 1, 3 ] ), 
    Transformation( [ 1, 4, 1, 1, 4 ] ), 
    Transformation( [ 1, 1, 1, 1, 1 ] ) ]
  !gapprompt@gap>| !gapinput@IsGreensLessThanOrEqual(DClass(S, reps[2]), DClass(S, reps[1]));|
  true
\end{Verbatim}
 }

 }

 
\section{\textcolor{Chapter }{Further attributes of semigroups}}\logpage{[ 3, 5, 0 ]}
\hyperdef{L}{X83802FC67CEB6C14}{}
{
  In this section we describe the attributes  (\textbf{Reference: Attributes}) of arbitrary semigroups of transformations or partial permutations that can be
found using \textsf{Semigroups}. 

 

\subsection{\textcolor{Chapter }{Generators}}
\logpage{[ 3, 5, 1 ]}\nobreak
\hyperdef{L}{X7BD5B55C802805B4}{}
{\noindent\textcolor{FuncColor}{$\triangleright$\ \ \texttt{Generators({\mdseries\slshape S})\index{Generators@\texttt{Generators}}
\label{Generators}
}\hfill{\scriptsize (function)}}\\
\textbf{\indent Returns:\ }
A list of transformations or partial permutations.



 \texttt{Generators} returns the generating set used to define the semigroup \mbox{\texttt{\mdseries\slshape S}} of transformations or partial permutations. The generators of a monoid or
inverse semigroup \mbox{\texttt{\mdseries\slshape S}} can be defined in several ways, for example, including or excluding the
identity element, including or not the inverses of the generators. \texttt{Generators(\mbox{\texttt{\mdseries\slshape S}})} uses the definition that returns the least number of generators. Nothing new
is computed when \texttt{Generators} is called and this function should not be confused with \texttt{SmallGeneratingSet} (\ref{SmallGeneratingSet}). 
\begin{description}
\item[{Transformation semigroup}] \texttt{Generators(\mbox{\texttt{\mdseries\slshape S}})} is a synonym for \texttt{GeneratorsOfSemigroup} (\textbf{Reference: GeneratorsOfSemigroup}). 
\item[{Transformation monoid}] \texttt{Generators(\mbox{\texttt{\mdseries\slshape S}})} is a synonym for \texttt{GeneratorsOfMonoid} (\textbf{Reference: GeneratorsOfMonoid}). 
\item[{Inverse semigroup}] \texttt{Generators(\mbox{\texttt{\mdseries\slshape S}})} is a synonym for \texttt{GeneratorsOfInverseSemigroup} (\textbf{Reference: GeneratorsOfInverseSemigroup}). 
\item[{Inverse monoid}] \texttt{Generators(\mbox{\texttt{\mdseries\slshape S}})} is a synonym for \texttt{GeneratorsOfInverseMonoid} (\textbf{Reference: GeneratorsOfInverseMonoid}). 
\end{description}
 
\begin{Verbatim}[commandchars=!@|,fontsize=\small,frame=single,label=Example]
  !gapprompt@gap>| !gapinput@M:=Monoid(Transformation( [ 1, 4, 6, 2, 5, 3, 7, 8, 9, 9 ] ),|
  !gapprompt@>| !gapinput@Transformation( [ 6, 3, 2, 7, 5, 1, 8, 8, 9, 9 ] ) );;|
  !gapprompt@gap>| !gapinput@GeneratorsOfSemigroup(M);|
  [ IdentityTransformation, 
    Transformation( [ 1, 4, 6, 2, 5, 3, 7, 8, 9, 9 ] ), 
    Transformation( [ 6, 3, 2, 7, 5, 1, 8, 8, 9, 9 ] ) ]
  !gapprompt@gap>| !gapinput@GeneratorsOfMonoid(M);|
  [ Transformation( [ 1, 4, 6, 2, 5, 3, 7, 8, 9, 9 ] ), 
    Transformation( [ 6, 3, 2, 7, 5, 1, 8, 8, 9, 9 ] ) ]
  !gapprompt@gap>| !gapinput@Generators(M);|
  [ Transformation( [ 1, 4, 6, 2, 5, 3, 7, 8, 9, 9 ] ), 
    Transformation( [ 6, 3, 2, 7, 5, 1, 8, 8, 9, 9 ] ) ]
  !gapprompt@gap>| !gapinput@S:=Semigroup(Generators(M));;|
  !gapprompt@gap>| !gapinput@Generators(S);|
  [ Transformation( [ 1, 4, 6, 2, 5, 3, 7, 8, 9, 9 ] ), 
    Transformation( [ 6, 3, 2, 7, 5, 1, 8, 8, 9, 9 ] ) ]
  !gapprompt@gap>| !gapinput@GeneratorsOfSemigroup(S);|
  [ Transformation( [ 1, 4, 6, 2, 5, 3, 7, 8, 9, 9 ] ), 
    Transformation( [ 6, 3, 2, 7, 5, 1, 8, 8, 9, 9 ] ) ]
\end{Verbatim}
 }

 

\subsection{\textcolor{Chapter }{GroupOfUnits}}
\logpage{[ 3, 5, 2 ]}\nobreak
\hyperdef{L}{X811AEDD88280C277}{}
{\noindent\textcolor{FuncColor}{$\triangleright$\ \ \texttt{GroupOfUnits({\mdseries\slshape S})\index{GroupOfUnits@\texttt{GroupOfUnits}}
\label{GroupOfUnits}
}\hfill{\scriptsize (attribute)}}\\
\textbf{\indent Returns:\ }
The group of units of a semigroup.



 \texttt{GroupOfUnits} returns the group of units of the semigroup \mbox{\texttt{\mdseries\slshape S}} as a subsemigroup of \mbox{\texttt{\mdseries\slshape S}} if it exists and returns \texttt{fail} if it does not. Use \texttt{IsomorphismPermGroup} (\ref{IsomorphismPermGroup}) if you require a permutation representation of the group of units.

 If a semigroup \mbox{\texttt{\mdseries\slshape S}} has an identity \texttt{e}, then the \emph{group of units} of \mbox{\texttt{\mdseries\slshape S}} is the set of those \texttt{s} in \mbox{\texttt{\mdseries\slshape S}} such that there exists \texttt{t} in \mbox{\texttt{\mdseries\slshape S}} where \texttt{s*t=t*s=e}. Equivalently, the group of units is the $\mathcal{H}$-class of the identity of \mbox{\texttt{\mdseries\slshape S}}.

 See also \texttt{GreensHClassOfElement} (\textbf{Reference: GreensHClassOfElement}), \texttt{IsMonoidAsSemigroup} (\ref{IsMonoidAsSemigroup}), and \texttt{MultiplicativeNeutralElement} (\textbf{Reference: MultiplicativeNeutralElement}). 
\begin{Verbatim}[commandchars=!@|,fontsize=\small,frame=single,label=Example]
  !gapprompt@gap>| !gapinput@S:=Semigroup(Transformation( [ 1, 2, 5, 4, 3, 8, 7, 6 ] ),|
  !gapprompt@>| !gapinput@  Transformation( [ 1, 6, 3, 4, 7, 2, 5, 8 ] ),|
  !gapprompt@>| !gapinput@  Transformation( [ 2, 1, 6, 7, 8, 3, 4, 5 ] ),|
  !gapprompt@>| !gapinput@  Transformation( [ 3, 2, 3, 6, 1, 6, 1, 2 ] ),|
  !gapprompt@>| !gapinput@  Transformation( [ 5, 2, 3, 6, 3, 4, 7, 4 ] ) );;|
  !gapprompt@gap>| !gapinput@Size(S);|
  5304
  !gapprompt@gap>| !gapinput@StructureDescription(GroupOfUnits(S));|
  "C2 x S4"
  !gapprompt@gap>| !gapinput@S:=InverseSemigroup( PartialPerm( [ 1, 2, 3, 4, 5, 6, 7, 8, 9, 10 ], |
  !gapprompt@>| !gapinput@[ 2, 4, 5, 3, 6, 7, 10, 9, 8, 1 ] ),|
  !gapprompt@>| !gapinput@PartialPerm( [ 1, 2, 3, 4, 5, 6, 7, 8, 10 ], |
  !gapprompt@>| !gapinput@[ 8, 2, 3, 1, 4, 5, 10, 6, 9 ] ) );;|
  !gapprompt@gap>| !gapinput@StructureDescription(GroupOfUnits(S));|
  "C8"
  !gapprompt@gap>| !gapinput@S:=InverseSemigroup( PartialPerm( [ 1, 3, 4 ], [ 4, 3, 5 ] ),|
  !gapprompt@>| !gapinput@PartialPerm( [ 1, 2, 3, 5 ], [ 3, 1, 5, 2 ] ) );;|
  !gapprompt@gap>| !gapinput@GroupOfUnits(S);|
  fail
\end{Verbatim}
 }

 

\subsection{\textcolor{Chapter }{Idempotents}}
\logpage{[ 3, 5, 3 ]}\nobreak
\hyperdef{L}{X7C651C9C78398FFF}{}
{\noindent\textcolor{FuncColor}{$\triangleright$\ \ \texttt{Idempotents({\mdseries\slshape obj[, n]})\index{Idempotents@\texttt{Idempotents}}
\label{Idempotents}
}\hfill{\scriptsize (attribute)}}\\
\textbf{\indent Returns:\ }
A list of idempotents.



 The argument \mbox{\texttt{\mdseries\slshape obj}} should be a semigroup, $\mathcal{D}$-class, $\mathcal{H}$-class, $\mathcal{L}$-class, or $\mathcal{R}$-class.

 If the optional second argument \mbox{\texttt{\mdseries\slshape n}} is present and \mbox{\texttt{\mdseries\slshape obj}} is a semigroup, then a list of the idempotents in \mbox{\texttt{\mdseries\slshape obj}} of rank \mbox{\texttt{\mdseries\slshape n}} is returned. If you are only interested in the idempotents of a given rank,
then the second version of the function will probably be faster. However, if
the optional second argument is present, then nothing is stored in \mbox{\texttt{\mdseries\slshape obj}} and so every time the function is called the computation must be repeated.

 This functions produce essentially the same output as the \textsf{GAP} library function with the same name; see \texttt{Idempotents} (\textbf{Reference: Idempotents}). The main difference is that this function can be applied to a wider class of
objects as described above.

 See also \texttt{IsRegularDClass} (\textbf{Reference: IsRegularDClass}), \texttt{IsRegularClass} (\ref{IsRegularClass}) \texttt{IsGroupHClass} (\textbf{Reference: IsGroupHClass}), \texttt{NrIdempotents} (\ref{NrIdempotents}), and \texttt{GroupHClass} (\ref{GroupHClass}). 
\begin{Verbatim}[commandchars=!@|,fontsize=\small,frame=single,label=Example]
  !gapprompt@gap>| !gapinput@S:=Semigroup([ Transformation( [ 2, 3, 4, 1 ] ), |
  !gapprompt@>| !gapinput@Transformation( [ 3, 3, 1, 1 ] ) ]);;|
  !gapprompt@gap>| !gapinput@Idempotents(S, 1);|
  [  ]
  !gapprompt@gap>| !gapinput@Idempotents(S, 2);|
  [ Transformation( [ 1, 1, 3, 3 ] ), Transformation( [ 1, 3, 3, 1 ] ), 
    Transformation( [ 2, 2, 4, 4 ] ), Transformation( [ 4, 2, 2, 4 ] ) ]
  !gapprompt@gap>| !gapinput@Idempotents(S);|
  [ IdentityTransformation, Transformation( [ 1, 1, 3, 3 ] ), 
    Transformation( [ 1, 3, 3, 1 ] ), Transformation( [ 2, 2, 4, 4 ] ), 
    Transformation( [ 4, 2, 2, 4 ] ) ]
  !gapprompt@gap>| !gapinput@f:=Transformation( [ 2, 2, 4, 4 ] );;|
  !gapprompt@gap>| !gapinput@R:=GreensRClassOfElement(S, f);|
  {Transformation( [ 3, 3, 1, 1 ] )}
  !gapprompt@gap>| !gapinput@Idempotents(R);|
  [ Transformation( [ 1, 1, 3, 3 ] ), Transformation( [ 2, 2, 4, 4 ] ) ]
  !gapprompt@gap>| !gapinput@f:=Transformation( [ 4, 2, 2, 4 ] );;|
  !gapprompt@gap>| !gapinput@L:=GreensLClassOfElement(S, f);;|
  !gapprompt@gap>| !gapinput@Idempotents(L);|
  [ Transformation( [ 2, 2, 4, 4 ] ), Transformation( [ 4, 2, 2, 4 ] ) ]
  !gapprompt@gap>| !gapinput@D:=DClassOfLClass(L);|
  {Transformation( [ 1, 1, 3, 3 ] )}
  !gapprompt@gap>| !gapinput@Idempotents(D);|
  [ Transformation( [ 1, 1, 3, 3 ] ), Transformation( [ 1, 3, 3, 1 ] ), 
    Transformation( [ 2, 2, 4, 4 ] ), Transformation( [ 4, 2, 2, 4 ] ) ]
  !gapprompt@gap>| !gapinput@L:=GreensLClassOfElement(S, Transformation( [ 3, 1, 1, 3 ] ));;|
  !gapprompt@gap>| !gapinput@Idempotents(L);|
  [ Transformation( [ 1, 1, 3, 3 ] ), Transformation( [ 1, 3, 3, 1 ] ) ]
  !gapprompt@gap>| !gapinput@H:=GroupHClass(D);|
  {Transformation( [ 1, 1, 3, 3 ] )}
  !gapprompt@gap>| !gapinput@Idempotents(H);|
  [ Transformation( [ 1, 1, 3, 3 ] ) ]
  !gapprompt@gap>| !gapinput@s:=InverseSemigroup(|
  !gapprompt@>| !gapinput@[ PartialPerm( [ 1, 2, 3, 4, 5, 7 ], [ 10, 6, 3, 4, 9, 1 ] ),|
  !gapprompt@>| !gapinput@PartialPerm( [ 1, 2, 3, 4, 5, 6, 7, 8 ], [ 6, 10, 7, 4, 8, 2, 9, 1 ] ) ]);;|
  !gapprompt@gap>| !gapinput@Idempotents(s, 1);|
  [ <identity partial perm on [ 4 ]> ]
  !gapprompt@gap>| !gapinput@Idempotents(s, 0);|
  [  ]
\end{Verbatim}
 }

 

\subsection{\textcolor{Chapter }{NrIdempotents}}
\logpage{[ 3, 5, 4 ]}\nobreak
\hyperdef{L}{X7CFC4DB387452320}{}
{\noindent\textcolor{FuncColor}{$\triangleright$\ \ \texttt{NrIdempotents({\mdseries\slshape obj})\index{NrIdempotents@\texttt{NrIdempotents}}
\label{NrIdempotents}
}\hfill{\scriptsize (attribute)}}\\
\textbf{\indent Returns:\ }
 A positive integer. 



 This function returns the number of idempotents in \mbox{\texttt{\mdseries\slshape obj}} where \mbox{\texttt{\mdseries\slshape obj}} can be a semigroup, $\mathcal{D}$-, $\mathcal{L}$-, $\mathcal{H}$-, or $\mathcal{R}$-class. If the actual idempotents are not required, then it is more efficient
to use \texttt{NrIdempotents(obj)} than \texttt{Length(Idempotents(obj))} since the idempotents themselves are not created when \texttt{NrIdempotents} is called.

 See also \texttt{Idempotents} (\textbf{Reference: Idempotents}) and \texttt{Idempotents} (\ref{Idempotents}), \texttt{IsRegularDClass} (\textbf{Reference: IsRegularDClass}), \texttt{IsRegularClass} (\ref{IsRegularClass}) \texttt{IsGroupHClass} (\textbf{Reference: IsGroupHClass}), and \texttt{GroupHClass} (\ref{GroupHClass}). 
\begin{Verbatim}[commandchars=!@|,fontsize=\small,frame=single,label=Example]
  !gapprompt@gap>| !gapinput@S:=Semigroup([ Transformation( [ 2, 3, 4, 1 ] ), |
  !gapprompt@>| !gapinput@Transformation( [ 3, 3, 1, 1 ] ) ]);;|
  !gapprompt@gap>| !gapinput@NrIdempotents(S);   |
  5
  !gapprompt@gap>| !gapinput@f:=Transformation( [ 2, 2, 4, 4 ] );;|
  !gapprompt@gap>| !gapinput@R:=GreensRClassOfElement(S, f);;|
  !gapprompt@gap>| !gapinput@NrIdempotents(R);|
  2
  !gapprompt@gap>| !gapinput@f:=Transformation( [ 4, 2, 2, 4 ] );;|
  !gapprompt@gap>| !gapinput@L:=GreensLClassOfElement(S, f);;|
  !gapprompt@gap>| !gapinput@NrIdempotents(L);|
  2
  !gapprompt@gap>| !gapinput@D:=DClassOfLClass(L);;|
  !gapprompt@gap>| !gapinput@NrIdempotents(D);|
  4
  !gapprompt@gap>| !gapinput@L:=GreensLClassOfElement(S, Transformation( [ 3, 1, 1, 3 ] ));;|
  !gapprompt@gap>| !gapinput@NrIdempotents(L);|
  2
  !gapprompt@gap>| !gapinput@H:=GroupHClass(D);;|
  !gapprompt@gap>| !gapinput@NrIdempotents(H);|
  1
  !gapprompt@gap>| !gapinput@s:=InverseSemigroup(|
  !gapprompt@>| !gapinput@[ PartialPerm( [ 1, 2, 3, 5, 7, 9, 10 ], [ 6, 7, 2, 9, 1, 5, 3 ] ),|
  !gapprompt@>| !gapinput@PartialPerm( [ 1, 2, 3, 5, 6, 7, 9, 10 ], [ 8, 1, 9, 4, 10, 5, 6, 7 ] ) ]);;|
  !gapprompt@gap>| !gapinput@NrIdempotents(s);|
  236
  !gapprompt@gap>| !gapinput@f:=PartialPerm([ 2, 3, 7, 9, 10 ], [ 7, 2, 1, 5, 3 ]);;|
  !gapprompt@gap>| !gapinput@d:=DClassNC(s, f);;|
  !gapprompt@gap>| !gapinput@NrIdempotents(d);|
  13
\end{Verbatim}
 }

 

\subsection{\textcolor{Chapter }{IdempotentGeneratedSubsemigroup}}
\logpage{[ 3, 5, 5 ]}\nobreak
\hyperdef{L}{X83970D028143B79B}{}
{\noindent\textcolor{FuncColor}{$\triangleright$\ \ \texttt{IdempotentGeneratedSubsemigroup({\mdseries\slshape S})\index{IdempotentGeneratedSubsemigroup@\texttt{IdempotentGeneratedSubsemigroup}}
\label{IdempotentGeneratedSubsemigroup}
}\hfill{\scriptsize (operation)}}\\
\textbf{\indent Returns:\ }
A semigroup. 



 \texttt{IdempotentGeneratedSubsemigroup} returns the subsemigroup of the semigroup \mbox{\texttt{\mdseries\slshape S}} generated by the idempotents of \mbox{\texttt{\mdseries\slshape S}}.

 See also \texttt{Idempotents} (\ref{Idempotents}) and \texttt{SmallGeneratingSet} (\ref{SmallGeneratingSet}). 
\begin{Verbatim}[commandchars=!@|,fontsize=\small,frame=single,label=Example]
  !gapprompt@gap>| !gapinput@file:=Concatenation(SemigroupsDir(), "/examples/graph8c.semigroups.gz");;|
  !gapprompt@gap>| !gapinput@S:=Semigroup(ReadGenerators(file, 13));;|
  !gapprompt@gap>| !gapinput@IdempotentGeneratedSubsemigroup(S);|
  <transformation monoid on 8 pts with 18 generators>
  !gapprompt@gap>| !gapinput@S:=SymmetricInverseSemigroup(5);|
  <symmetric inverse semigroup on 5 pts>
  !gapprompt@gap>| !gapinput@IdempotentGeneratedSubsemigroup(S);|
  <inverse partial perm semigroup on 5 pts with 6 generators>
\end{Verbatim}
 }

 

\subsection{\textcolor{Chapter }{IrredundantGeneratingSubset}}
\logpage{[ 3, 5, 6 ]}\nobreak
\hyperdef{L}{X7F88DA9487720D2B}{}
{\noindent\textcolor{FuncColor}{$\triangleright$\ \ \texttt{IrredundantGeneratingSubset({\mdseries\slshape coll})\index{IrredundantGeneratingSubset@\texttt{IrredundantGeneratingSubset}}
\label{IrredundantGeneratingSubset}
}\hfill{\scriptsize (operation)}}\\
\textbf{\indent Returns:\ }
 A list of irredundant generators. 



 If \mbox{\texttt{\mdseries\slshape coll}} is a collection of elements of a semigroup, then this function returns a
subset \texttt{U} of \mbox{\texttt{\mdseries\slshape coll}} or \texttt{Generators(\mbox{\texttt{\mdseries\slshape coll}})} such that no element of \texttt{U} is generated by the other elements of \texttt{U}.

 See also \texttt{Generators} (\ref{Generators}), \texttt{IsTransformationCollection} (\textbf{Reference: IsTransformationCollection}), and \texttt{SmallGeneratingSet} (\ref{SmallGeneratingSet}). 
\begin{Verbatim}[commandchars=!@|,fontsize=\small,frame=single,label=Example]
  !gapprompt@gap>| !gapinput@S:=Semigroup( Transformation( [ 5, 1, 4, 6, 2, 3 ] ),|
  !gapprompt@>| !gapinput@Transformation( [ 1, 2, 3, 4, 5, 6 ] ),|
  !gapprompt@>| !gapinput@Transformation( [ 4, 6, 3, 4, 2, 5 ] ),|
  !gapprompt@>| !gapinput@Transformation( [ 5, 4, 6, 3, 1, 3 ] ),|
  !gapprompt@>| !gapinput@Transformation( [ 2, 2, 6, 5, 4, 3 ] ),|
  !gapprompt@>| !gapinput@Transformation( [ 3, 5, 5, 1, 2, 4 ] ),|
  !gapprompt@>| !gapinput@Transformation( [ 6, 5, 1, 3, 3, 4 ] ),|
  !gapprompt@>| !gapinput@Transformation( [ 1, 3, 4, 3, 2, 1 ] ) );;|
  !gapprompt@gap>| !gapinput@IrredundantGeneratingSubset(S);|
  [ Transformation( [ 1, 3, 4, 3, 2, 1 ] ), 
    Transformation( [ 2, 2, 6, 5, 4, 3 ] ), 
    Transformation( [ 3, 5, 5, 1, 2, 4 ] ), 
    Transformation( [ 5, 1, 4, 6, 2, 3 ] ), 
    Transformation( [ 5, 4, 6, 3, 1, 3 ] ), 
    Transformation( [ 6, 5, 1, 3, 3, 4 ] ) ]
  !gapprompt@gap>| !gapinput@S:=RandomInverseMonoid(1000,10);|
  <inverse monoid with 1000 generators>
  !gapprompt@gap>| !gapinput@SmallGeneratingSet(S);|
  [ [ 1 .. 10 ] -> [ 6, 5, 1, 9, 8, 3, 10, 4, 7, 2 ], 
    [ 1 .. 10 ] -> [ 1, 4, 6, 2, 8, 5, 7, 10, 3, 9 ], 
    [ 1, 2, 3, 4, 6, 7, 8, 9 ] -> [ 7, 5, 10, 1, 8, 4, 9, 6 ]
    [ 1 .. 9 ] -> [ 4, 3, 5, 7, 10, 9, 1, 6, 8 ] ]
  !gapprompt@gap>| !gapinput@IrredundantGeneratingSubset(last);|
  [ [ 1 .. 9 ] -> [ 4, 3, 5, 7, 10, 9, 1, 6, 8 ], 
    [ 1 .. 10 ] -> [ 1, 4, 6, 2, 8, 5, 7, 10, 3, 9 ], 
    [ 1 .. 10 ] -> [ 6, 5, 1, 9, 8, 3, 10, 4, 7, 2 ] ]
\end{Verbatim}
 }

 

\subsection{\textcolor{Chapter }{MinimalIdeal}}
\logpage{[ 3, 5, 7 ]}\nobreak
\hyperdef{L}{X7BC68589879C3BE9}{}
{\noindent\textcolor{FuncColor}{$\triangleright$\ \ \texttt{MinimalIdeal({\mdseries\slshape S})\index{MinimalIdeal@\texttt{MinimalIdeal}}
\label{MinimalIdeal}
}\hfill{\scriptsize (attribute)}}\\
\textbf{\indent Returns:\ }
 The minimal ideal of a semigroup. 



 The minimal ideal of a semigroup is the least ideal with respect to
containment. 

 Currently, \texttt{MinimalIdeal} returns a semigroup with as many generators as elements. There are plans to
improve this in future versions of \textsf{Semigroups}. 

 Note that \texttt{MinimalIdeal} is significantly faster than finding the $\mathcal{D}$-class with minimum rank representative (which is also the minimal ideal). See
also \texttt{PartialOrderOfDClasses} (\ref{PartialOrderOfDClasses}), \texttt{IsGreensLessThanOrEqual} (\textbf{Reference: IsGreensLessThanOrEqual}), and \texttt{MinimalDClass} (\ref{MinimalDClass}). 
\begin{Verbatim}[commandchars=!@|,fontsize=\small,frame=single,label=Example]
  !gapprompt@gap>| !gapinput@S:=Semigroup( Transformation( [ 3, 4, 1, 3, 6, 3, 4, 6, 10, 1 ] ), |
  !gapprompt@>| !gapinput@Transformation( [ 8, 2, 3, 8, 4, 1, 3, 4, 9, 7 ] ));;|
  !gapprompt@gap>| !gapinput@MinimalIdeal(S);|
  <simple transformation semigroup on 10 pts with 5 generators>
  !gapprompt@gap>| !gapinput@Elements(MinimalIdeal(S));|
  [ Transformation( [ 1, 1, 1, 1, 1, 1, 1, 1, 1, 1 ] ), 
    Transformation( [ 3, 3, 3, 3, 3, 3, 3, 3, 3, 3 ] ), 
    Transformation( [ 4, 4, 4, 4, 4, 4, 4, 4, 4, 4 ] ), 
    Transformation( [ 6, 6, 6, 6, 6, 6, 6, 6, 6, 6 ] ), 
    Transformation( [ 8, 8, 8, 8, 8, 8, 8, 8, 8, 8 ] ) ]
  !gapprompt@gap>| !gapinput@f:=Transformation( [ 8, 8, 8, 8, 8, 8, 8, 8, 8, 8 ] );;|
  !gapprompt@gap>| !gapinput@D:=DClass(S, f);|
  {Transformation( [ 3, 3, 3, 3, 3, 3, 3, 3, 3, 3 ] )}
  !gapprompt@gap>| !gapinput@ForAll(GreensDClasses(S), x-> IsGreensLessThanOrEqual(D, x));|
  true
  !gapprompt@gap>| !gapinput@MinimalIdeal(POI(10));|
  <partial perm group on 0 pts with 1 generator>
\end{Verbatim}
 }

 

\subsection{\textcolor{Chapter }{MultiplicativeZero}}
\logpage{[ 3, 5, 8 ]}\nobreak
\hyperdef{L}{X7B39F93C8136D642}{}
{\noindent\textcolor{FuncColor}{$\triangleright$\ \ \texttt{MultiplicativeZero({\mdseries\slshape S})\index{MultiplicativeZero@\texttt{MultiplicativeZero}}
\label{MultiplicativeZero}
}\hfill{\scriptsize (attribute)}}\\
\textbf{\indent Returns:\ }
 The zero element of a semigroup. 



 \texttt{MultiplicativeZero} returns the zero element of the semigroup \mbox{\texttt{\mdseries\slshape S}} if it has one and \texttt{fail} if it does not. See also \texttt{MultiplicativeZero} (\textbf{Reference: MultiplicativeZero}). 
\begin{Verbatim}[commandchars=!@|,fontsize=\small,frame=single,label=Example]
  !gapprompt@gap>| !gapinput@S:=Semigroup( Transformation( [ 1, 4, 2, 6, 6, 5, 2 ] ), |
  !gapprompt@>| !gapinput@Transformation( [ 1, 6, 3, 6, 2, 1, 6 ] ));;|
  !gapprompt@gap>| !gapinput@MultiplicativeZero(S);|
  Transformation( [ 1, 1, 1, 1, 1, 1, 1 ] )
  !gapprompt@gap>| !gapinput@S:=Semigroup(Transformation( [ 2, 8, 3, 7, 1, 5, 2, 6 ] ), |
  !gapprompt@>| !gapinput@Transformation( [ 3, 5, 7, 2, 5, 6, 3, 8 ] ), |
  !gapprompt@>| !gapinput@Transformation( [ 6, 7, 4, 1, 4, 1, 6, 2 ] ), |
  !gapprompt@>| !gapinput@Transformation( [ 8, 8, 5, 1, 7, 5, 2, 8 ] ));;|
  !gapprompt@gap>| !gapinput@MultiplicativeZero(S);|
  fail
  !gapprompt@gap>| !gapinput@S:=InverseSemigroup( PartialPerm( [ 1, 3, 4 ], [ 5, 3, 1 ] ),|
  !gapprompt@>| !gapinput@PartialPerm( [ 1, 2, 3, 4 ], [ 4, 3, 1, 2 ] ),|
  !gapprompt@>| !gapinput@PartialPerm( [ 1, 3, 4, 5 ], [ 2, 4, 5, 3 ] ) );;|
  !gapprompt@gap>| !gapinput@MultiplicativeZero(S);|
  <empty partial perm>
\end{Verbatim}
 }

 

\subsection{\textcolor{Chapter }{Random (for a semigroup)}}
\logpage{[ 3, 5, 9 ]}\nobreak
\hyperdef{L}{X7BB7FDFE7AFFD672}{}
{\noindent\textcolor{FuncColor}{$\triangleright$\ \ \texttt{Random({\mdseries\slshape S})\index{Random@\texttt{Random}!for a semigroup}
\label{Random:for a semigroup}
}\hfill{\scriptsize (method)}}\\
\textbf{\indent Returns:\ }
A random element.



 This function returns a random element of the semigroup \mbox{\texttt{\mdseries\slshape S}}. If the $\mathcal{R}$-class structure of \mbox{\texttt{\mdseries\slshape S}} has not been calculated, then a short product (at most \texttt{2*Length(GeneratorsOfSemigroup(\mbox{\texttt{\mdseries\slshape S}}))}) of generators is returned. If the $\mathcal{R}$-class structure of \mbox{\texttt{\mdseries\slshape S}} is known, then a random element of a randomly chosen $\mathcal{R}$-class is returned. }

 

\subsection{\textcolor{Chapter }{SmallGeneratingSet}}
\logpage{[ 3, 5, 10 ]}\nobreak
\hyperdef{L}{X814DBABC878D5232}{}
{\noindent\textcolor{FuncColor}{$\triangleright$\ \ \texttt{SmallGeneratingSet({\mdseries\slshape S})\index{SmallGeneratingSet@\texttt{SmallGeneratingSet}}
\label{SmallGeneratingSet}
}\hfill{\scriptsize (attribute)}}\\
\textbf{\indent Returns:\ }
A small generating set for a semigroup.



 \texttt{SmallGeneratingSet} returns a relatively small generating subset of the set of generators of the
semigroup \mbox{\texttt{\mdseries\slshape S}}; see \texttt{Generators} (\ref{Generators}). If the number of generators for \mbox{\texttt{\mdseries\slshape S}} is already relatively small, then this function will often return the original
generating set.

 As neither irredundancy, nor minimal length are proven, \texttt{SmallGeneratingSet} usually returns an answer much more quickly than \texttt{IrredundantGeneratingSubset} (\ref{IrredundantGeneratingSubset}). It can be used whenever a small generating set is desired which does not
necessarily needs to be minimal. \texttt{SmallGeneratingSet} works particularly well for inverse semigroups of partial permutations.

 Note that \texttt{SmallGeneratingSet} may return different results in different \textsf{GAP} sessions. 
\begin{Verbatim}[commandchars=!@|,fontsize=\small,frame=single,label=Example]
  !gapprompt@gap>| !gapinput@S:=Semigroup( Transformation( [ 1, 2, 3, 2, 4 ] ), |
  !gapprompt@>| !gapinput@Transformation( [ 1, 5, 4, 3, 2 ] ),|
  !gapprompt@>| !gapinput@Transformation( [ 2, 1, 4, 2, 2 ] ), |
  !gapprompt@>| !gapinput@Transformation( [ 2, 4, 4, 2, 1 ] ),|
  !gapprompt@>| !gapinput@Transformation( [ 3, 1, 4, 3, 2 ] ), |
  !gapprompt@>| !gapinput@Transformation( [ 3, 2, 3, 4, 1 ] ),|
  !gapprompt@>| !gapinput@Transformation( [ 4, 4, 3, 3, 5 ] ), |
  !gapprompt@>| !gapinput@Transformation( [ 5, 1, 5, 5, 3 ] ),|
  !gapprompt@>| !gapinput@Transformation( [ 5, 4, 3, 5, 2 ] ), |
  !gapprompt@>| !gapinput@Transformation( [ 5, 5, 4, 5, 5 ] ) );;|
  !gapprompt@gap>| !gapinput@SmallGeneratingSet(S);                  |
  [ Transformation( [ 1, 5, 4, 3, 2 ] ), Transformation( [ 3, 2, 3, 4, 1 ] ), 
    Transformation( [ 5, 4, 3, 5, 2 ] ), Transformation( [ 1, 2, 3, 2, 4 ] ), 
    Transformation( [ 4, 4, 3, 3, 5 ] ) ]
  !gapprompt@gap>| !gapinput@S:=RandomInverseMonoid(10000,10);;|
  !gapprompt@gap>| !gapinput@SmallGeneratingSet(S);|
  [ [ 1 .. 10 ] -> [ 3, 2, 4, 5, 6, 1, 7, 10, 9, 8 ], 
    [ 1 .. 10 ] -> [ 5, 10, 8, 9, 3, 2, 4, 7, 6, 1 ], 
    [ 1, 3, 4, 5, 6, 7, 8, 9, 10 ] -> [ 1, 6, 4, 8, 2, 10, 7, 3, 9 ] ]
  !gapprompt@gap>| !gapinput@M:=MathieuGroup(24);;|
  !gapprompt@gap>| !gapinput@mat:=List([1..1000], x-> Random(G));;|
  !gapprompt@gap>| !gapinput@Append(mat, [1..1000]*0);|
  !gapprompt@gap>| !gapinput@mat:=List([1..138], x-> List([1..57], x-> Random(mat)));;|
  !gapprompt@gap>| !gapinput@R:=ReesZeroMatrixSemigroup(G, mat);;|
  !gapprompt@gap>| !gapinput@U:=Semigroup(List([1..200], x-> Random(R)));|
  <subsemigroup of 57x138 Rees 0-matrix semigroup with 100 generators>
  !gapprompt@gap>| !gapinput@Length(SmallGeneratingSet(U));|
  84
\end{Verbatim}
 }

 

\subsection{\textcolor{Chapter }{MaximalSubsemigroups}}
\logpage{[ 3, 5, 11 ]}\nobreak
\hyperdef{L}{X80292361867AF3FE}{}
{\noindent\textcolor{FuncColor}{$\triangleright$\ \ \texttt{MaximalSubsemigroups({\mdseries\slshape S})\index{MaximalSubsemigroups@\texttt{MaximalSubsemigroups}}
\label{MaximalSubsemigroups}
}\hfill{\scriptsize (attribute)}}\\
\textbf{\indent Returns:\ }
The maximal subsemigroups of \mbox{\texttt{\mdseries\slshape S}}.



 If \mbox{\texttt{\mdseries\slshape S}} is a Rees matrix semigroup or regular Rees 0-matrix semigroup over a group,
then \texttt{MaximalSubsemigroups} returns a list of the maximal subsemigroups of \mbox{\texttt{\mdseries\slshape S}}. 

 The method for this function is based on Remark 1 of \cite{Graham1968aa}. 
\begin{Verbatim}[commandchars=!@|,fontsize=\small,frame=single,label=Example]
  !gapprompt@gap>| !gapinput@S:=FullTransformationSemigroup(4);                              |
  <full transformation semigroup on 4 pts>
  !gapprompt@gap>| !gapinput@D:=DClass(S, Transformation([2,2]));|
  {Transformation( [ 2, 2, 3, 1 ] )}
  !gapprompt@gap>| !gapinput@R:=PrincipalFactor(D);|
  <Rees 0-matrix semigroup 6x4 over Group([ (2,3,4), (2,4) ])>
  !gapprompt@gap>| !gapinput@MaximalSubsemigroups(R);                                       |
  [ <Rees 0-matrix semigroup 6x3 over Group([ (2,3,4), (2,4) ])>, 
    <Rees 0-matrix semigroup 6x3 over Group([ (2,3,4), (2,4) ])>, 
    <Rees 0-matrix semigroup 6x3 over Group([ (2,3,4), (2,4) ])>, 
    <Rees 0-matrix semigroup 6x3 over Group([ (2,3,4), (2,4) ])>, 
    <Rees 0-matrix semigroup 5x4 over Group([ (2,3,4), (2,4) ])>, 
    <Rees 0-matrix semigroup 5x4 over Group([ (2,3,4), (2,4) ])>, 
    <Rees 0-matrix semigroup 5x4 over Group([ (2,3,4), (2,4) ])>, 
    <Rees 0-matrix semigroup 5x4 over Group([ (2,3,4), (2,4) ])>, 
    <Rees 0-matrix semigroup 5x4 over Group([ (2,3,4), (2,4) ])>, 
    <Rees 0-matrix semigroup 5x4 over Group([ (2,3,4), (2,4) ])>, 
    <subsemigroup of 6x4 Rees 0-matrix semigroup with 22 generators>, 
    <subsemigroup of 6x4 Rees 0-matrix semigroup with 22 generators>, 
    <subsemigroup of 6x4 Rees 0-matrix semigroup with 20 generators>, 
    <subsemigroup of 6x4 Rees 0-matrix semigroup with 22 generators>, 
    <subsemigroup of 6x4 Rees 0-matrix semigroup with 20 generators>, 
    <subsemigroup of 6x4 Rees 0-matrix semigroup with 22 generators>, 
    <subsemigroup of 6x4 Rees 0-matrix semigroup with 20 generators>, 
    <subsemigroup of 6x4 Rees 0-matrix semigroup with 20 generators>, 
    <subsemigroup of 6x4 Rees 0-matrix semigroup with 20 generators>, 
    <subsemigroup of 6x4 Rees 0-matrix semigroup with 20 generators> ]
\end{Verbatim}
 }

 

\subsection{\textcolor{Chapter }{ComponentRepsOfTransformationSemigroup}}
\logpage{[ 3, 5, 12 ]}\nobreak
\hyperdef{L}{X8065DBC48722B085}{}
{\noindent\textcolor{FuncColor}{$\triangleright$\ \ \texttt{ComponentRepsOfTransformationSemigroup({\mdseries\slshape S})\index{ComponentRepsOfTransformationSemigroup@\texttt{Component}\-\texttt{Reps}\-\texttt{Of}\-\texttt{Transformation}\-\texttt{Semigroup}}
\label{ComponentRepsOfTransformationSemigroup}
}\hfill{\scriptsize (attribute)}}\\
\textbf{\indent Returns:\ }
The representatives of components of \mbox{\texttt{\mdseries\slshape S}}.



 This function returns the representatives of the components of the action of
the transformation semigroup \mbox{\texttt{\mdseries\slshape S}} on the set of positive integers on which it acts. The representatives are the
least set of points such that every point can be reached from some
representative under the action of \mbox{\texttt{\mdseries\slshape S}}. 
\begin{Verbatim}[commandchars=!@|,fontsize=\small,frame=single,label=Example]
  !gapprompt@gap>| !gapinput@S:=Semigroup( |
  !gapprompt@>| !gapinput@Transformation( [ 11, 11, 9, 6, 4, 1, 4, 1, 6, 7, 12, 5 ] ), |
  !gapprompt@>| !gapinput@Transformation( [ 12, 10, 7, 10, 4, 1, 12, 9, 11, 9, 1, 12 ] ) );;|
  !gapprompt@gap>| !gapinput@ComponentRepsOfTransformationSemigroup(S);|
  [ 2, 3, 8 ]
\end{Verbatim}
 }

 }

 
\section{\textcolor{Chapter }{Further properties of semigroups}}\logpage{[ 3, 6, 0 ]}
\hyperdef{L}{X7CC47DE17B361189}{}
{
  In this section we describe several properties  (\textbf{Reference: Properties}) of an arbitrary semigroup of transformations or partial permutations that can
be determined using \textsf{Semigroups}. 

\subsection{\textcolor{Chapter }{IsBand}}
\logpage{[ 3, 6, 1 ]}\nobreak
\hyperdef{L}{X7C8DB14587D1B55A}{}
{\noindent\textcolor{FuncColor}{$\triangleright$\ \ \texttt{IsBand({\mdseries\slshape S})\index{IsBand@\texttt{IsBand}}
\label{IsBand}
}\hfill{\scriptsize (property)}}\\
\textbf{\indent Returns:\ }
\texttt{true} or \texttt{false}. 



 \texttt{IsBand} returns \texttt{true} if every element of the semigroup \mbox{\texttt{\mdseries\slshape S}} is an idempotent and \texttt{false} if it is not. An inverse semigroup is band if and only if it is a semilattice;
see \texttt{IsSemilatticeAsSemigroup} (\ref{IsSemilatticeAsSemigroup}). 
\begin{Verbatim}[commandchars=!@|,fontsize=\small,frame=single,label=Example]
  !gapprompt@gap>| !gapinput@gens:=[ Transformation( [ 1, 1, 1, 4, 4, 4, 7, 7, 7, 1 ] ), |
  !gapprompt@>| !gapinput@Transformation( [ 2, 2, 2, 5, 5, 5, 8, 8, 8, 2 ] ), |
  !gapprompt@>| !gapinput@Transformation( [ 3, 3, 3, 6, 6, 6, 9, 9, 9, 3 ] ), |
  !gapprompt@>| !gapinput@Transformation( [ 1, 1, 1, 4, 4, 4, 7, 7, 7, 4 ] ), |
  !gapprompt@>| !gapinput@Transformation( [ 1, 1, 1, 4, 4, 4, 7, 7, 7, 7 ] ) ];;|
  !gapprompt@gap>| !gapinput@S:=Semigroup(gens);;|
  !gapprompt@gap>| !gapinput@IsBand(S);|
  true
  !gapprompt@gap>| !gapinput@S:=InverseSemigroup(|
  !gapprompt@>| !gapinput@PartialPerm( [ 1, 2, 3, 4, 8, 9 ], [ 5, 8, 7, 6, 9, 1 ] ),|
  !gapprompt@>| !gapinput@PartialPerm( [ 1, 3, 4, 7, 8, 9, 10 ], [ 2, 3, 8, 7, 10, 6, 1 ] ) );;|
  !gapprompt@gap>| !gapinput@IsBand(S);|
  false
  !gapprompt@gap>| !gapinput@IsBand(IdempotentGeneratedSubsemigroup(S));|
  true
\end{Verbatim}
 }

 

\subsection{\textcolor{Chapter }{IsBlockGroup}}
\logpage{[ 3, 6, 2 ]}\nobreak
\hyperdef{L}{X79659C467C8A7EBD}{}
{\noindent\textcolor{FuncColor}{$\triangleright$\ \ \texttt{IsBlockGroup({\mdseries\slshape S})\index{IsBlockGroup@\texttt{IsBlockGroup}}
\label{IsBlockGroup}
}\hfill{\scriptsize (property)}}\\
\noindent\textcolor{FuncColor}{$\triangleright$\ \ \texttt{IsSemigroupWithCommutingIdempotents({\mdseries\slshape S})\index{IsSemigroupWithCommutingIdempotents@\texttt{IsSemigroupWithCommutingIdempotents}}
\label{IsSemigroupWithCommutingIdempotents}
}\hfill{\scriptsize (property)}}\\
\textbf{\indent Returns:\ }
\texttt{true} or \texttt{false}. 



 \texttt{IsBlockGroup} and \texttt{IsSemigroupWithCommutingIdempotents} return \texttt{true} if the semigroup \mbox{\texttt{\mdseries\slshape S}} is a block group and \texttt{false} if it is not.

 A semigroup \mbox{\texttt{\mdseries\slshape S}} is a \emph{block group} if every $\mathcal{L}$-class and every $\mathcal{R}$-class of \mbox{\texttt{\mdseries\slshape S}} contains at most one idempotent. Every semigroup of partial permutations is a
block group. 
\begin{Verbatim}[commandchars=!@|,fontsize=\small,frame=single,label=Example]
  !gapprompt@gap>| !gapinput@S:=Semigroup(Transformation( [ 5, 6, 7, 3, 1, 4, 2, 8 ] ),|
  !gapprompt@>| !gapinput@  Transformation( [ 3, 6, 8, 5, 7, 4, 2, 8 ] ));;|
  !gapprompt@gap>| !gapinput@IsBlockGroup(S);|
  true
  !gapprompt@gap>| !gapinput@S:=Semigroup(Transformation( [ 2, 1, 10, 4, 5, 9, 7, 4, 8, 4 ] ),|
  !gapprompt@>| !gapinput@Transformation( [ 10, 7, 5, 6, 1, 3, 9, 7, 10, 2 ] ));;|
  !gapprompt@gap>| !gapinput@IsBlockGroup(S);|
  false
\end{Verbatim}
 }

 

\subsection{\textcolor{Chapter }{IsCommutativeSemigroup}}
\logpage{[ 3, 6, 3 ]}\nobreak
\hyperdef{L}{X843EFDA4807FDC31}{}
{\noindent\textcolor{FuncColor}{$\triangleright$\ \ \texttt{IsCommutativeSemigroup({\mdseries\slshape S})\index{IsCommutativeSemigroup@\texttt{IsCommutativeSemigroup}}
\label{IsCommutativeSemigroup}
}\hfill{\scriptsize (property)}}\\
\textbf{\indent Returns:\ }
\texttt{true} or \texttt{false}. 



 \texttt{IsCommutativeSemigroup} returns \texttt{true} if the semigroup \mbox{\texttt{\mdseries\slshape S}} is commutative and \texttt{false} if it is not. The function \texttt{IsCommutative} (\textbf{Reference: IsCommutative}) can also be used to test if a semigroup is commutative. 

 A semigroup \mbox{\texttt{\mdseries\slshape S}} is \emph{commutative} if \texttt{x*y=y*x} for all \texttt{x,y} in \mbox{\texttt{\mdseries\slshape S}}. 
\begin{Verbatim}[commandchars=!@|,fontsize=\small,frame=single,label=Example]
  !gapprompt@gap>| !gapinput@gens:=[ Transformation( [ 2, 4, 5, 3, 7, 8, 6, 9, 1 ] ), |
  !gapprompt@>| !gapinput@ Transformation( [ 3, 5, 6, 7, 8, 1, 9, 2, 4 ] ) ];;|
  !gapprompt@gap>| !gapinput@S:=Semigroup(gens);;|
  !gapprompt@gap>| !gapinput@IsCommutativeSemigroup(S);|
  true
  !gapprompt@gap>| !gapinput@IsCommutative(S);|
  true
  !gapprompt@gap>| !gapinput@S:=InverseSemigroup(|
  !gapprompt@>| !gapinput@ PartialPerm( [ 1, 2, 3, 4, 5, 6 ], [ 2, 5, 1, 3, 9, 6 ] ),|
  !gapprompt@>| !gapinput@ PartialPerm( [ 1, 2, 3, 4, 6, 8 ], [ 8, 5, 7, 6, 2, 1 ] ) );;|
  !gapprompt@gap>| !gapinput@IsCommutativeSemigroup(S);|
  false
\end{Verbatim}
 }

 

\subsection{\textcolor{Chapter }{IsCompletelyRegularSemigroup}}
\logpage{[ 3, 6, 4 ]}\nobreak
\hyperdef{L}{X7AFA23AF819FBF3D}{}
{\noindent\textcolor{FuncColor}{$\triangleright$\ \ \texttt{IsCompletelyRegularSemigroup({\mdseries\slshape S})\index{IsCompletelyRegularSemigroup@\texttt{IsCompletelyRegularSemigroup}}
\label{IsCompletelyRegularSemigroup}
}\hfill{\scriptsize (property)}}\\
\textbf{\indent Returns:\ }
\texttt{true} or \texttt{false}. 



 \texttt{IsCompletelyRegularSemigroup} returns \texttt{true} if every element of the semigroup \mbox{\texttt{\mdseries\slshape S}} is contained in a subgroup of \mbox{\texttt{\mdseries\slshape S}}.

 An inverse semigroup is completely regular if and only if it is a Clifford
semigroup; see \texttt{IsCliffordSemigroup} (\ref{IsCliffordSemigroup}). 
\begin{Verbatim}[commandchars=!@|,fontsize=\small,frame=single,label=Example]
  !gapprompt@gap>| !gapinput@gens:=[ Transformation( [ 1, 2, 4, 3, 6, 5, 4 ] ), |
  !gapprompt@>| !gapinput@ Transformation( [ 1, 2, 5, 6, 3, 4, 5 ] ), |
  !gapprompt@>| !gapinput@ Transformation( [ 2, 1, 2, 2, 2, 2, 2 ] ) ];;|
  !gapprompt@gap>| !gapinput@S:=Semigroup(gens);;|
  !gapprompt@gap>| !gapinput@IsCompletelyRegularSemigroup(S);|
  true
  !gapprompt@gap>| !gapinput@IsInverseSemigroup(S);|
  true
  !gapprompt@gap>| !gapinput@T:=Range(IsomorphismPartialPermSemigroup(S));;|
  !gapprompt@gap>| !gapinput@IsCompletelyRegularSemigroup(T);|
  true
  !gapprompt@gap>| !gapinput@IsCliffordSemigroup(T);         |
  true
\end{Verbatim}
 }

 

\subsection{\textcolor{Chapter }{IsGroupAsSemigroup}}
\logpage{[ 3, 6, 5 ]}\nobreak
\hyperdef{L}{X852F29E8795FA489}{}
{\noindent\textcolor{FuncColor}{$\triangleright$\ \ \texttt{IsGroupAsSemigroup({\mdseries\slshape S})\index{IsGroupAsSemigroup@\texttt{IsGroupAsSemigroup}}
\label{IsGroupAsSemigroup}
}\hfill{\scriptsize (property)}}\\
\textbf{\indent Returns:\ }
\texttt{true} or \texttt{false}.



 If the semigroup \mbox{\texttt{\mdseries\slshape S}} is actually a group, then \texttt{IsGroupAsSemigroup} returns \texttt{true}. If it is not a group, then \texttt{false} is returned. 
\begin{Verbatim}[commandchars=!@|,fontsize=\small,frame=single,label=Example]
  !gapprompt@gap>| !gapinput@gens:=[ Transformation( [ 2, 4, 5, 3, 7, 8, 6, 9, 1 ] ), |
  !gapprompt@>| !gapinput@ Transformation( [ 3, 5, 6, 7, 8, 1, 9, 2, 4 ] ) ];;|
  !gapprompt@gap>| !gapinput@S:=Semigroup(gens);;|
  !gapprompt@gap>| !gapinput@IsGroupAsSemigroup(S);|
  true
  !gapprompt@gap>| !gapinput@G:=SymmetricGroup(5);;|
  !gapprompt@gap>| !gapinput@S:=Range(IsomorphismPartialPermSemigroup(G));|
  <inverse partial perm semigroup on 5 pts with 2 generators>
  !gapprompt@gap>| !gapinput@IsGroupAsSemigroup(S);|
  true
\end{Verbatim}
 }

 
\subsection{\textcolor{Chapter }{IsIdempotentGenerated}}\logpage{[ 3, 6, 6 ]}
\hyperdef{L}{X835484C481CF3DDD}{}
{
\noindent\textcolor{FuncColor}{$\triangleright$\ \ \texttt{IsIdempotentGenerated({\mdseries\slshape S})\index{IsIdempotentGenerated@\texttt{IsIdempotentGenerated}}
\label{IsIdempotentGenerated}
}\hfill{\scriptsize (property)}}\\
\noindent\textcolor{FuncColor}{$\triangleright$\ \ \texttt{IsSemiBand({\mdseries\slshape S})\index{IsSemiBand@\texttt{IsSemiBand}}
\label{IsSemiBand}
}\hfill{\scriptsize (property)}}\\
\textbf{\indent Returns:\ }
\texttt{true} or \texttt{false}. 



 \texttt{IsIdempotentGenerated} and \texttt{IsSemiBand} return \texttt{true} if the semigroup \mbox{\texttt{\mdseries\slshape S}} is generated by its idempotents and \texttt{false} if it is not. See also \texttt{Idempotents} (\ref{Idempotents}) and \texttt{IdempotentGeneratedSubsemigroup} (\ref{IdempotentGeneratedSubsemigroup}). 

 An inverse semigroup is idempotent-generated if and only if it is a
semilattice; see \texttt{IsSemilatticeAsSemigroup} (\ref{IsSemilatticeAsSemigroup}).

 Semiband and idempotent-generated are synonymous in this context. 
\begin{Verbatim}[commandchars=!@|,fontsize=\small,frame=single,label=Example]
  !gapprompt@gap>| !gapinput@S:=SingularTransformationSemigroup(4);|
  <regular transformation semigroup on 4 pts with 12 generators>
  !gapprompt@gap>| !gapinput@IsIdempotentGenerated(S);|
  true
\end{Verbatim}
 }

 

\subsection{\textcolor{Chapter }{IsLeftSimple}}
\logpage{[ 3, 6, 7 ]}\nobreak
\hyperdef{L}{X8206D2B0809952EF}{}
{\noindent\textcolor{FuncColor}{$\triangleright$\ \ \texttt{IsLeftSimple({\mdseries\slshape S})\index{IsLeftSimple@\texttt{IsLeftSimple}}
\label{IsLeftSimple}
}\hfill{\scriptsize (property)}}\\
\noindent\textcolor{FuncColor}{$\triangleright$\ \ \texttt{IsRightSimple({\mdseries\slshape S})\index{IsRightSimple@\texttt{IsRightSimple}}
\label{IsRightSimple}
}\hfill{\scriptsize (property)}}\\
\textbf{\indent Returns:\ }
\texttt{true} or \texttt{false}. 



 \texttt{IsLeftSimple} and \texttt{IsRightSimple} returns \texttt{true} if the semigroup \mbox{\texttt{\mdseries\slshape S}} has only one $\mathcal{L}$-class or one $\mathcal{R}$-class, respectively, and returns \texttt{false} if it has more than one. 

 An inverse semigroup is left simple if and only if it is right simple if and
only if it is a group; see \texttt{IsGroupAsSemigroup} (\ref{IsGroupAsSemigroup}). 
\begin{Verbatim}[commandchars=!@|,fontsize=\small,frame=single,label=Example]
  !gapprompt@gap>| !gapinput@S:=Semigroup( Transformation( [ 6, 7, 9, 6, 8, 9, 8, 7, 6 ] ), |
  !gapprompt@>| !gapinput@ Transformation( [ 6, 8, 9, 6, 8, 8, 7, 9, 6 ] ), |
  !gapprompt@>| !gapinput@ Transformation( [ 6, 8, 9, 7, 8, 8, 7, 9, 6 ] ), |
  !gapprompt@>| !gapinput@ Transformation( [ 6, 9, 8, 6, 7, 9, 7, 8, 6 ] ), |
  !gapprompt@>| !gapinput@ Transformation( [ 6, 9, 9, 6, 8, 8, 7, 9, 6 ] ), |
  !gapprompt@>| !gapinput@ Transformation( [ 6, 9, 9, 7, 8, 8, 6, 9, 7 ] ), |
  !gapprompt@>| !gapinput@ Transformation( [ 7, 8, 8, 7, 9, 9, 7, 8, 6 ] ), |
  !gapprompt@>| !gapinput@ Transformation( [ 7, 9, 9, 7, 6, 9, 6, 8, 7 ] ), |
  !gapprompt@>| !gapinput@ Transformation( [ 8, 7, 6, 9, 8, 6, 8, 7, 9 ] ), |
  !gapprompt@>| !gapinput@ Transformation( [ 9, 6, 6, 7, 8, 8, 7, 6, 9 ] ), |
  !gapprompt@>| !gapinput@ Transformation( [ 9, 6, 6, 7, 9, 6, 9, 8, 7 ] ), |
  !gapprompt@>| !gapinput@ Transformation( [ 9, 6, 7, 9, 6, 6, 9, 7, 8 ] ), |
  !gapprompt@>| !gapinput@ Transformation( [ 9, 6, 8, 7, 9, 6, 9, 8, 7 ] ), |
  !gapprompt@>| !gapinput@ Transformation( [ 9, 7, 6, 8, 7, 7, 9, 6, 8 ] ), |
  !gapprompt@>| !gapinput@ Transformation( [ 9, 7, 7, 8, 9, 6, 9, 7, 8 ] ), |
  !gapprompt@>| !gapinput@ Transformation( [ 9, 8, 8, 9, 6, 7, 6, 8, 9 ] ) );;|
  !gapprompt@gap>| !gapinput@IsRightSimple(S);|
  false
  !gapprompt@gap>| !gapinput@IsLeftSimple(S);|
  true
  !gapprompt@gap>| !gapinput@IsGroupAsSemigroup(S);|
  false
  !gapprompt@gap>| !gapinput@NrRClasses(S);|
  16
\end{Verbatim}
 }

 

\subsection{\textcolor{Chapter }{IsLeftZeroSemigroup}}
\logpage{[ 3, 6, 8 ]}\nobreak
\hyperdef{L}{X7E9261367C8C52C0}{}
{\noindent\textcolor{FuncColor}{$\triangleright$\ \ \texttt{IsLeftZeroSemigroup({\mdseries\slshape S})\index{IsLeftZeroSemigroup@\texttt{IsLeftZeroSemigroup}}
\label{IsLeftZeroSemigroup}
}\hfill{\scriptsize (property)}}\\
\textbf{\indent Returns:\ }
\texttt{true} or \texttt{false}. 



 \texttt{IsLeftZeroSemigroup} returns \texttt{true} if the semigroup \mbox{\texttt{\mdseries\slshape S}} is a left zero semigroup and \texttt{false} if it is not. 

 A semigroup is a \emph{left zero semigroup} if \texttt{x*y=x} for all \texttt{x,y}. An inverse semigroup is a left zero semigroup if and only if it is trivial. 
\begin{Verbatim}[commandchars=!@|,fontsize=\small,frame=single,label=Example]
  !gapprompt@gap>| !gapinput@gens:=[ Transformation( [ 2, 1, 4, 3, 5 ] ), |
  !gapprompt@>| !gapinput@ Transformation( [ 3, 2, 3, 1, 1 ] ) ];;|
  !gapprompt@gap>| !gapinput@S:=Semigroup(gens);;|
  !gapprompt@gap>| !gapinput@IsRightZeroSemigroup(S);|
  false
  !gapprompt@gap>| !gapinput@gens:=[Transformation( [ 1, 2, 3, 3, 1 ] ), |
  !gapprompt@>| !gapinput@Transformation( [ 1, 2, 3, 3, 3 ] ) ];;|
  !gapprompt@gap>| !gapinput@S:=Semigroup(gens);;|
  !gapprompt@gap>| !gapinput@IsLeftZeroSemigroup(S);|
  true
\end{Verbatim}
 }

 

\subsection{\textcolor{Chapter }{IsMonogenicSemigroup}}
\logpage{[ 3, 6, 9 ]}\nobreak
\hyperdef{L}{X79D46BAB7E327AD1}{}
{\noindent\textcolor{FuncColor}{$\triangleright$\ \ \texttt{IsMonogenicSemigroup({\mdseries\slshape S})\index{IsMonogenicSemigroup@\texttt{IsMonogenicSemigroup}}
\label{IsMonogenicSemigroup}
}\hfill{\scriptsize (property)}}\\
\textbf{\indent Returns:\ }
\texttt{true} or \texttt{false}. 



 \texttt{IsMonogenicSemigroup} returns \texttt{true} if the semigroup \mbox{\texttt{\mdseries\slshape S}} is monogenic and it returns \texttt{false} if it is not. 

 A semigroup is \emph{monogenic} if it is generated by a single element. See also \texttt{IsMonogenicInverseSemigroup} (\ref{IsMonogenicInverseSemigroup}) and \texttt{IndexPeriodOfTransformation} (\textbf{Reference: IndexPeriodOfTransformation}). 
\begin{Verbatim}[commandchars=!@|,fontsize=\small,frame=single,label=Example]
  !gapprompt@gap>| !gapinput@S:=Semigroup(|
  !gapprompt@>| !gapinput@Transformation( [ 2, 2, 2, 11, 10, 8, 10, 11, 2, 11, 10, 2, 11, 11, 10 ] ),|
  !gapprompt@>| !gapinput@Transformation( [ 2, 2, 2, 8, 11, 15, 11, 10, 2, 10, 11, 2, 10, 4, 7 ] ), |
  !gapprompt@>| !gapinput@Transformation( [ 2, 2, 2, 11, 10, 8, 10, 11, 2, 11, 10, 2, 11, 11, 10 ] ),|
  !gapprompt@>| !gapinput@Transformation( [ 2, 2, 12, 7, 8, 14, 8, 11, 2, 11, 10, 2, 11, 15, 4 ] ));;|
  !gapprompt@gap>| !gapinput@IsMonogenicSemigroup(S);|
  true
\end{Verbatim}
 }

 

\subsection{\textcolor{Chapter }{IsMonoidAsSemigroup}}
\logpage{[ 3, 6, 10 ]}\nobreak
\hyperdef{L}{X7E4DEECD7CD9886D}{}
{\noindent\textcolor{FuncColor}{$\triangleright$\ \ \texttt{IsMonoidAsSemigroup({\mdseries\slshape S})\index{IsMonoidAsSemigroup@\texttt{IsMonoidAsSemigroup}}
\label{IsMonoidAsSemigroup}
}\hfill{\scriptsize (property)}}\\
\textbf{\indent Returns:\ }
\texttt{true} or \texttt{false}. 



 \texttt{IsMonoidAsSemigroup} returns \texttt{true} if the semigroup \mbox{\texttt{\mdseries\slshape S}} is a monoid and \texttt{false} if it is not. It is possible that \mbox{\texttt{\mdseries\slshape S}} is a monoid but does not satisfy \texttt{IsMonoid} (\textbf{Reference: IsMonoid}) and so \mbox{\texttt{\mdseries\slshape S}} does not possess the attributes of a monoid (such as, \texttt{GeneratorsOfMonoid} (\textbf{Reference: GeneratorsOfMonoid})).

 A semigroup of transformations satisfies \texttt{IsMonoidAsSemigroup} if and only if it satisfies \texttt{IsTransformationMonoid} (\textbf{Reference: IsTransformationMonoid}). A semigroup of partial permutations satisfies \texttt{IsMonoidAsSemigroup} if and only if it satisfies \texttt{IsPartialPermMonoid} (\textbf{Reference: IsPartialPermMonoid}). 

 See also \texttt{One} (\textbf{Reference: One}), \texttt{IsInverseMonoid} (\textbf{Reference: IsInverseMonoid}) and \texttt{IsomorphismTransformationMonoid} (\ref{IsomorphismTransformationMonoid}). 
\begin{Verbatim}[commandchars=!@|,fontsize=\small,frame=single,label=Example]
  !gapprompt@gap>| !gapinput@S:=Semigroup( Transformation( [ 1, 4, 6, 2, 5, 3, 7, 8, 9, 9 ] ),|
  !gapprompt@>| !gapinput@Transformation( [ 6, 3, 2, 7, 5, 1, 8, 8, 9, 9 ] ) );;|
  !gapprompt@gap>| !gapinput@IsMonoidAsSemigroup(S);|
  true
  !gapprompt@gap>| !gapinput@MultiplicativeNeutralElement(S);|
  Transformation( [ 1, 2, 3, 4, 5, 6, 7, 8, 9, 9 ] )
  !gapprompt@gap>| !gapinput@S:=Monoid(Transformation( [ 8, 2, 8, 9, 10, 6, 2, 8, 7, 8 ] ),|
  !gapprompt@>| !gapinput@Transformation( [ 9, 2, 6, 3, 6, 4, 5, 5, 3, 2 ] ));;|
  !gapprompt@gap>| !gapinput@IsMonoidAsSemigroup(S);|
  false
\end{Verbatim}
 }

 

\subsection{\textcolor{Chapter }{IsOrthodoxSemigroup}}
\logpage{[ 3, 6, 11 ]}\nobreak
\hyperdef{L}{X7935C476808C8773}{}
{\noindent\textcolor{FuncColor}{$\triangleright$\ \ \texttt{IsOrthodoxSemigroup({\mdseries\slshape S})\index{IsOrthodoxSemigroup@\texttt{IsOrthodoxSemigroup}}
\label{IsOrthodoxSemigroup}
}\hfill{\scriptsize (property)}}\\
\textbf{\indent Returns:\ }
\texttt{true} or \texttt{false}. 



 \texttt{IsOrthodoxSemigroup} returns \texttt{true} if the semigroup \mbox{\texttt{\mdseries\slshape S}} is orthodox and \texttt{false} if it is not.

 A semigroup is \emph{orthodox} if it is regular and its idempotent elements form a subsemigroup. Every
inverse semigroup is also an orthodox semigroup. 

 See also \texttt{IsRegularSemigroup} (\ref{IsRegularSemigroup}) and \texttt{IsRegularSemigroup} (\textbf{Reference: IsRegularSemigroup}). 
\begin{Verbatim}[commandchars=!@|,fontsize=\small,frame=single,label=Example]
  !gapprompt@gap>| !gapinput@gens:=[ Transformation( [ 1, 1, 1, 4, 5, 4 ] ), |
  !gapprompt@>| !gapinput@ Transformation( [ 1, 2, 3, 1, 1, 2 ] ), |
  !gapprompt@>| !gapinput@ Transformation( [ 1, 2, 3, 1, 1, 3 ] ), |
  !gapprompt@>| !gapinput@ Transformation( [ 5, 5, 5, 5, 5, 5 ] ) ];;|
  !gapprompt@gap>| !gapinput@S:=Semigroup(gens);;|
  !gapprompt@gap>| !gapinput@IsOrthodoxSemigroup(S);|
  true
\end{Verbatim}
 }

 

\subsection{\textcolor{Chapter }{IsRectangularBand}}
\logpage{[ 3, 6, 12 ]}\nobreak
\hyperdef{L}{X7E9B674D781B072C}{}
{\noindent\textcolor{FuncColor}{$\triangleright$\ \ \texttt{IsRectangularBand({\mdseries\slshape S})\index{IsRectangularBand@\texttt{IsRectangularBand}}
\label{IsRectangularBand}
}\hfill{\scriptsize (property)}}\\
\textbf{\indent Returns:\ }
\texttt{true} or \texttt{false}. 



 \texttt{IsRectangularBand} returns \texttt{true} if the semigroup \mbox{\texttt{\mdseries\slshape S}} is a rectangular band and \texttt{false} if it is not.

 A semigroup \mbox{\texttt{\mdseries\slshape S}} is a \emph{rectangular band} if for all \texttt{x,y,z} in \mbox{\texttt{\mdseries\slshape S}} we have that \texttt{x\texttt{\symbol{94}}2=x} and \texttt{xyz=xz}. An inverse semigroup is a rectangular band if and only if it is a group. 
\begin{Verbatim}[commandchars=!@|,fontsize=\small,frame=single,label=Example]
  !gapprompt@gap>| !gapinput@gens:=[ Transformation( [ 1, 1, 1, 4, 4, 4, 7, 7, 7, 1 ] ), |
  !gapprompt@>| !gapinput@Transformation( [ 2, 2, 2, 5, 5, 5, 8, 8, 8, 2 ] ), |
  !gapprompt@>| !gapinput@Transformation( [ 3, 3, 3, 6, 6, 6, 9, 9, 9, 3 ] ), |
  !gapprompt@>| !gapinput@Transformation( [ 1, 1, 1, 4, 4, 4, 7, 7, 7, 4 ] ), |
  !gapprompt@>| !gapinput@Transformation( [ 1, 1, 1, 4, 4, 4, 7, 7, 7, 7 ] ) ];;|
  !gapprompt@gap>| !gapinput@S:=Semigroup(gens);;|
  !gapprompt@gap>| !gapinput@IsRectangularBand(S);|
  true
\end{Verbatim}
 }

 

\subsection{\textcolor{Chapter }{IsRegularSemigroup}}
\logpage{[ 3, 6, 13 ]}\nobreak
\hyperdef{L}{X7C4663827C5ACEF1}{}
{\noindent\textcolor{FuncColor}{$\triangleright$\ \ \texttt{IsRegularSemigroup({\mdseries\slshape S})\index{IsRegularSemigroup@\texttt{IsRegularSemigroup}}
\label{IsRegularSemigroup}
}\hfill{\scriptsize (property)}}\\
\textbf{\indent Returns:\ }
\texttt{true} or \texttt{false}. 



 \texttt{IsRegularSemigroup} returns \texttt{true} if the semigroup \mbox{\texttt{\mdseries\slshape S}} is regular and \texttt{false} if it is not. 

 A semigroup \texttt{S} is \emph{regular} if for all \texttt{x} in \texttt{S} there exists \texttt{y} in \texttt{S} such that \texttt{x*y*x=x}. Every inverse semigroup is regular, and a semigroup of partial permutations
is regular if and only if it is an inverse semigroup.

 See also \texttt{IsRegularDClass} (\textbf{Reference: IsRegularDClass}), \texttt{IsRegularClass} (\ref{IsRegularClass}), and \texttt{IsRegularSemigroupElement} (\textbf{Reference: IsRegularSemigroupElement}). 
\begin{Verbatim}[commandchars=!@|,fontsize=\small,frame=single,label=Example]
  !gapprompt@gap>| !gapinput@IsRegularSemigroup(FullTransformationSemigroup(5));|
  true
\end{Verbatim}
 }

 

\subsection{\textcolor{Chapter }{IsRightZeroSemigroup}}
\logpage{[ 3, 6, 14 ]}\nobreak
\hyperdef{L}{X7CB099958658F979}{}
{\noindent\textcolor{FuncColor}{$\triangleright$\ \ \texttt{IsRightZeroSemigroup({\mdseries\slshape S})\index{IsRightZeroSemigroup@\texttt{IsRightZeroSemigroup}}
\label{IsRightZeroSemigroup}
}\hfill{\scriptsize (property)}}\\
\textbf{\indent Returns:\ }
\texttt{true} or \texttt{false}. 



 \texttt{IsRightZeroSemigroup} returns \texttt{true} if the \mbox{\texttt{\mdseries\slshape S}} is a right zero semigroup and \texttt{false} if it is not.

 A semigroup \texttt{S} is a \emph{right zero semigroup} if \texttt{x*y=y} for all \texttt{x,y} in \texttt{S}. An inverse semigroup is a right zero semigroup if and only if it is trivial. 
\begin{Verbatim}[commandchars=!@|,fontsize=\small,frame=single,label=Example]
  !gapprompt@gap>| !gapinput@gens:=[ Transformation( [ 2, 1, 4, 3, 5 ] ), |
  !gapprompt@>| !gapinput@ Transformation( [ 3, 2, 3, 1, 1 ] ) ];;|
  !gapprompt@gap>| !gapinput@S:=Semigroup(gens);;|
  !gapprompt@gap>| !gapinput@IsRightZeroSemigroup(S);|
  false
  !gapprompt@gap>| !gapinput@gens:=[Transformation( [ 1, 2, 3, 3, 1 ] ), |
  !gapprompt@>| !gapinput@ Transformation( [ 1, 2, 4, 4, 1 ] )];;|
  !gapprompt@gap>| !gapinput@S:=Semigroup(gens);;|
  !gapprompt@gap>| !gapinput@IsRightZeroSemigroup(S);|
  true
\end{Verbatim}
 }

 
\subsection{\textcolor{Chapter }{IsXTrivial}}\logpage{[ 3, 6, 15 ]}
\hyperdef{L}{X8752642C7F7E512B}{}
{
\noindent\textcolor{FuncColor}{$\triangleright$\ \ \texttt{IsRTrivial({\mdseries\slshape S})\index{IsRTrivial@\texttt{IsRTrivial}}
\label{IsRTrivial}
}\hfill{\scriptsize (property)}}\\
\noindent\textcolor{FuncColor}{$\triangleright$\ \ \texttt{IsLTrivial({\mdseries\slshape S})\index{IsLTrivial@\texttt{IsLTrivial}}
\label{IsLTrivial}
}\hfill{\scriptsize (property)}}\\
\noindent\textcolor{FuncColor}{$\triangleright$\ \ \texttt{IsHTrivial({\mdseries\slshape S})\index{IsHTrivial@\texttt{IsHTrivial}}
\label{IsHTrivial}
}\hfill{\scriptsize (property)}}\\
\noindent\textcolor{FuncColor}{$\triangleright$\ \ \texttt{IsDTrivial({\mdseries\slshape S})\index{IsDTrivial@\texttt{IsDTrivial}}
\label{IsDTrivial}
}\hfill{\scriptsize (property)}}\\
\noindent\textcolor{FuncColor}{$\triangleright$\ \ \texttt{IsAperiodicSemigroup({\mdseries\slshape S})\index{IsAperiodicSemigroup@\texttt{IsAperiodicSemigroup}}
\label{IsAperiodicSemigroup}
}\hfill{\scriptsize (property)}}\\
\noindent\textcolor{FuncColor}{$\triangleright$\ \ \texttt{IsCombinatorialSemigroup({\mdseries\slshape S})\index{IsCombinatorialSemigroup@\texttt{IsCombinatorialSemigroup}}
\label{IsCombinatorialSemigroup}
}\hfill{\scriptsize (property)}}\\
\textbf{\indent Returns:\ }
\texttt{true} or \texttt{false}. 



 \texttt{IsXTrivial} returns \texttt{true} if Green's $\mathcal{R}$-relation, $\mathcal{L}$-relation, $\mathcal{H}$-relation, $\mathcal{D}$-relation, respectively, on the semigroup \mbox{\texttt{\mdseries\slshape S}} is trivial and \texttt{false} if it is not. These properties can also be applied to a Green's class instead
of a semigroup where applicable. 

 For inverse semigroups, the properties of being $\mathcal{R}$-trivial, $\mathcal{L}$-trivial, $\mathcal{D}$-trivial, and a semilattice are equivalent; see \texttt{IsSemilatticeAsSemigroup} (\ref{IsSemilatticeAsSemigroup}). 

 A semigroup is \emph{aperiodic} if its contains no non-trivial subgroups (equivalently, all of its group $\mathcal{H}$-classes are trivial). A finite semigroup is aperiodic if and only if it is $\mathcal{H}$-trivial. 

 \emph{Combinatorial} is a synonym for aperiodic in this context. 
\begin{Verbatim}[commandchars=!@|,fontsize=\small,frame=single,label=Example]
  !gapprompt@gap>| !gapinput@S:=Semigroup( Transformation( [ 1, 5, 1, 3, 7, 10, 6, 2, 7, 10 ] ), |
  !gapprompt@>| !gapinput@ Transformation( [ 4, 4, 5, 6, 7, 7, 7, 4, 3, 10 ] ) );;|
  !gapprompt@gap>| !gapinput@IsHTrivial(S);|
  true
  !gapprompt@gap>| !gapinput@Size(S);|
  108
  !gapprompt@gap>| !gapinput@IsRTrivial(S);|
  false
  !gapprompt@gap>| !gapinput@IsLTrivial(S);|
  false
\end{Verbatim}
 }

 

\subsection{\textcolor{Chapter }{IsSemilatticeAsSemigroup}}
\logpage{[ 3, 6, 16 ]}\nobreak
\hyperdef{L}{X7BF9F1BE87F0636D}{}
{\noindent\textcolor{FuncColor}{$\triangleright$\ \ \texttt{IsSemilatticeAsSemigroup({\mdseries\slshape S})\index{IsSemilatticeAsSemigroup@\texttt{IsSemilatticeAsSemigroup}}
\label{IsSemilatticeAsSemigroup}
}\hfill{\scriptsize (property)}}\\
\textbf{\indent Returns:\ }
\texttt{true} or \texttt{false}. 



 \texttt{IsSemilatticeAsSemigroup} returns \texttt{true} if the semigroup \mbox{\texttt{\mdseries\slshape S}} is a semilattice and \texttt{false} if it is not. 

 A semigroup is a \emph{semilattice} if it is commutative and every element is an idempotent. The idempotents of an
inverse semigroup form a semilattice. 
\begin{Verbatim}[commandchars=!@|,fontsize=\small,frame=single,label=Example]
  !gapprompt@gap>| !gapinput@S:=Semigroup(Transformation( [ 2, 5, 1, 7, 3, 7, 7 ] ), |
  !gapprompt@>| !gapinput@Transformation( [ 3, 6, 5, 7, 2, 1, 7 ] ) );;                    |
  !gapprompt@gap>| !gapinput@Size(S);|
  631
  !gapprompt@gap>| !gapinput@IsInverseSemigroup(S);|
  true
  !gapprompt@gap>| !gapinput@A:=Semigroup(Idempotents(S)); |
  <transformation semigroup on 7 pts with 32 generators>
  !gapprompt@gap>| !gapinput@IsSemilatticeAsSemigroup(A);|
  true
\end{Verbatim}
 }

 
\subsection{\textcolor{Chapter }{IsSimpleSemigroup}}\logpage{[ 3, 6, 17 ]}
\hyperdef{L}{X836F4692839F4874}{}
{
\noindent\textcolor{FuncColor}{$\triangleright$\ \ \texttt{IsSimpleSemigroup({\mdseries\slshape S})\index{IsSimpleSemigroup@\texttt{IsSimpleSemigroup}}
\label{IsSimpleSemigroup}
}\hfill{\scriptsize (property)}}\\
\noindent\textcolor{FuncColor}{$\triangleright$\ \ \texttt{IsCompletelySimpleSemigroup({\mdseries\slshape S})\index{IsCompletelySimpleSemigroup@\texttt{IsCompletelySimpleSemigroup}}
\label{IsCompletelySimpleSemigroup}
}\hfill{\scriptsize (property)}}\\
\textbf{\indent Returns:\ }
\texttt{true} or \texttt{false}. 



 \texttt{IsSimpleSemigroup} returns \texttt{true} if the semigroup of transformations or partial permutations \mbox{\texttt{\mdseries\slshape S}} is simple and \texttt{false} if it is not.

 A semigroup is \emph{simple} if it has no proper 2-sided ideals. A semigroup is \emph{completely simple} if it is simple and possesses minimal left and right ideals. A finite
semigroup is simple if and only if it is completely simple. An inverse
semigroup is simple if and only if it is a group. 
\begin{Verbatim}[commandchars=!@|,fontsize=\small,frame=single,label=Example]
  !gapprompt@gap>| !gapinput@gens:=[ Transformation( [ 2, 2, 4, 4, 6, 6, 8, 8, 10, 10, 12, 12, 2 ] ), |
  !gapprompt@>| !gapinput@ Transformation( [ 1, 1, 3, 3, 5, 5, 7, 7, 9, 9, 11, 11, 3 ] ), |
  !gapprompt@>| !gapinput@ Transformation( [ 1, 7, 3, 9, 5, 11, 7, 1, 9, 3, 11, 5, 5 ] ), |
  !gapprompt@>| !gapinput@ Transformation( [ 7, 7, 9, 9, 11, 11, 1, 1, 3, 3, 5, 5, 7 ] ) ];;|
  !gapprompt@gap>| !gapinput@S:=Semigroup(gens);;|
  !gapprompt@gap>| !gapinput@IsSimpleSemigroup(S);|
  true
  !gapprompt@gap>| !gapinput@IsCompletelySimpleSemigroup(S);|
  true
\end{Verbatim}
 }

 

\subsection{\textcolor{Chapter }{IsSynchronizingSemigroup}}
\logpage{[ 3, 6, 18 ]}\nobreak
\hyperdef{L}{X84A1B84180811785}{}
{\noindent\textcolor{FuncColor}{$\triangleright$\ \ \texttt{IsSynchronizingSemigroup({\mdseries\slshape S, n})\index{IsSynchronizingSemigroup@\texttt{IsSynchronizingSemigroup}}
\label{IsSynchronizingSemigroup}
}\hfill{\scriptsize (operation)}}\\
\textbf{\indent Returns:\ }
\texttt{true} or \texttt{false}. 



 \texttt{IsSynchronizingSemigroup} returns \texttt{true} if the semigroup of transformations \mbox{\texttt{\mdseries\slshape S}} contains a transformation with constant value on \texttt{[1..\mbox{\texttt{\mdseries\slshape n}}]}. See also \texttt{ConstantTransformation} (\textbf{Reference: ConstantTransformation}). 
\begin{Verbatim}[commandchars=!@|,fontsize=\small,frame=single,label=Example]
  !gapprompt@gap>| !gapinput@S:=Semigroup( Transformation( [ 1, 1, 8, 7, 6, 6, 4, 1, 8, 9 ] ), |
  !gapprompt@>| !gapinput@ Transformation( [ 5, 8, 7, 6, 10, 8, 7, 6, 9, 7 ] ) );;|
  !gapprompt@gap>| !gapinput@IsSynchronizingSemigroup(S, 10);|
  true
  !gapprompt@gap>| !gapinput@S:=Semigroup( Transformation( [ 3, 8, 1, 1, 9, 9, 8, 7, 9, 6 ] ), |
  !gapprompt@>| !gapinput@ Transformation( [ 7, 6, 8, 7, 5, 6, 8, 7, 8, 9 ] ) );;|
  !gapprompt@gap>| !gapinput@IsSynchronizingSemigroup(S, 10);|
  false
  !gapprompt@gap>| !gapinput@Representative(MinimalIdeal(S));|
  Transformation( [ 8, 7, 7, 8, 7, 7, 7, 8, 7, 8 ] )
\end{Verbatim}
 }

 

\subsection{\textcolor{Chapter }{IsZeroGroup}}
\logpage{[ 3, 6, 19 ]}\nobreak
\hyperdef{L}{X85F7E5CD86F0643B}{}
{\noindent\textcolor{FuncColor}{$\triangleright$\ \ \texttt{IsZeroGroup({\mdseries\slshape S})\index{IsZeroGroup@\texttt{IsZeroGroup}}
\label{IsZeroGroup}
}\hfill{\scriptsize (property)}}\\
\textbf{\indent Returns:\ }
\texttt{true} or \texttt{false}. 



 \texttt{IsZeroGroup} returns \texttt{true} if the semigroup \mbox{\texttt{\mdseries\slshape S}} is a zero group and \texttt{false} if it is not.

 A semigroup \texttt{S} is a \emph{zero group} if there exists an element \texttt{z} in \texttt{S} such that \texttt{S} without \texttt{z} is a group and \texttt{x*z=z*x=z} for all \texttt{x} in \texttt{S}. Every zero group is an inverse semigroup. 
\begin{Verbatim}[commandchars=!@|,fontsize=\small,frame=single,label=Example]
  !gapprompt@gap>| !gapinput@S:=Semigroup(Transformation( [ 2, 2, 3, 4, 6, 8, 5, 5, 9 ] ),|
  !gapprompt@>| !gapinput@Transformation( [ 3, 3, 8, 2, 5, 6, 4, 4, 9 ] ),|
  !gapprompt@>| !gapinput@ConstantTransformation(9, 9));;|
  !gapprompt@gap>| !gapinput@IsZeroGroup(S);|
  true
  !gapprompt@gap>| !gapinput@T:=Range(IsomorphismPartialPermSemigroup(S));;|
  !gapprompt@gap>| !gapinput@IsZeroGroup(T);|
  true
\end{Verbatim}
 }

 

\subsection{\textcolor{Chapter }{IsZeroRectangularBand}}
\logpage{[ 3, 6, 20 ]}\nobreak
\hyperdef{L}{X7C6787D07B95B450}{}
{\noindent\textcolor{FuncColor}{$\triangleright$\ \ \texttt{IsZeroRectangularBand({\mdseries\slshape S})\index{IsZeroRectangularBand@\texttt{IsZeroRectangularBand}}
\label{IsZeroRectangularBand}
}\hfill{\scriptsize (property)}}\\
\textbf{\indent Returns:\ }
\texttt{true} or \texttt{false}. 



 \texttt{IsZeroRectangularBand} returns \texttt{true} if the semigroup \mbox{\texttt{\mdseries\slshape S}} is a zero rectangular band and \texttt{false} if it is not.

 A semigroup is a \emph{zero rectangular band} if it is zero simple and $\mathcal{H}$-trivial; see also \texttt{IsZeroSimpleSemigroup} (\ref{IsZeroSimpleSemigroup}) and \texttt{IsHTrivial} (\ref{IsHTrivial}). An inverse semigroup is a zero rectangular band if and only if it is a zero
group; see \texttt{IsZeroGroup} (\ref{IsZeroGroup}). 
\begin{Verbatim}[commandchars=!@|,fontsize=\small,frame=single,label=Example]
  !gapprompt@gap>| !gapinput@S:=Semigroup( |
  !gapprompt@>| !gapinput@ Transformation( [ 1, 3, 7, 9, 1, 12, 13, 1, 15, 9, 1, 18, 1, 1, 13, 1, 1, |
  !gapprompt@>| !gapinput@     21, 1, 1, 1, 1, 1, 25, 26, 1 ] ),|
  !gapprompt@>| !gapinput@Transformation( [ 1, 5, 1, 5, 11, 1, 1, 14, 1, 16, 17, 1, 1, 19, 1, 11, 1,|
  !gapprompt@>| !gapinput@     1, 1, 23, 1, 16, 19, 1, 1, 1 ] ),|
  !gapprompt@>| !gapinput@Transformation( [ 1, 4, 8, 1, 10, 1, 8, 1, 1, 1, 10, 1, 8, 10, 1, 1, 20, 1,|
  !gapprompt@>| !gapinput@     22, 1, 8, 1, 1, 1, 1, 1 ] ),|
  !gapprompt@>| !gapinput@Transformation( [ 1, 6, 6, 1, 1, 1, 6, 1, 1, 1, 1, 1, 6, 1, 6, 1, 1, 6, 1,|
  !gapprompt@>| !gapinput@     1, 24, 1, 1, 1, 1, 6 ] ) );;|
  !gapprompt@gap>| !gapinput@IsZeroRectangularBand(Semigroup(Elements(GreensDClasses(S)[7]))); |
  true
  !gapprompt@gap>| !gapinput@IsZeroRectangularBand(Semigroup(Elements(GreensDClasses(S)[1])));|
  false
\end{Verbatim}
 }

 

\subsection{\textcolor{Chapter }{IsZeroSemigroup}}
\logpage{[ 3, 6, 21 ]}\nobreak
\hyperdef{L}{X81A1882181B75CC9}{}
{\noindent\textcolor{FuncColor}{$\triangleright$\ \ \texttt{IsZeroSemigroup({\mdseries\slshape S})\index{IsZeroSemigroup@\texttt{IsZeroSemigroup}}
\label{IsZeroSemigroup}
}\hfill{\scriptsize (property)}}\\
\textbf{\indent Returns:\ }
\texttt{true} or \texttt{false}. 



 \texttt{IsZeroSemigroup} returns \texttt{true} if the semigroup \mbox{\texttt{\mdseries\slshape S}} is a zero semigroup and \texttt{false} if it is not.

 A semigroup \texttt{S} is a \emph{zero semigroup} if there exists an element \texttt{z} in \texttt{S} such that \texttt{x*y=z} for all \texttt{x,y} in \texttt{S}. An inverse semigroup is a zero semigroup if and only if it is trivial. 
\begin{Verbatim}[commandchars=!@|,fontsize=\small,frame=single,label=Example]
  !gapprompt@gap>| !gapinput@S:=Semigroup( Transformation( [ 4, 7, 6, 3, 1, 5, 3, 6, 5, 9 ] ), |
  !gapprompt@>| !gapinput@Transformation( [ 5, 3, 5, 1, 9, 3, 8, 7, 4, 3 ] ) );;|
  !gapprompt@gap>| !gapinput@IsZeroSemigroup(S);|
  false
  !gapprompt@gap>| !gapinput@S:=Semigroup( Transformation( [ 7, 8, 8, 8, 5, 8, 8, 8 ] ), |
  !gapprompt@>| !gapinput@ Transformation( [ 8, 8, 8, 8, 5, 7, 8, 8 ] ), |
  !gapprompt@>| !gapinput@ Transformation( [ 8, 7, 8, 8, 5, 8, 8, 8 ] ), |
  !gapprompt@>| !gapinput@ Transformation( [ 8, 8, 8, 7, 5, 8, 8, 8 ] ), |
  !gapprompt@>| !gapinput@ Transformation( [ 8, 8, 7, 8, 5, 8, 8, 8 ] ) );;|
  !gapprompt@gap>| !gapinput@IsZeroSemigroup(S);|
  true
  !gapprompt@gap>| !gapinput@MultiplicativeZero(S);|
  Transformation( [ 8, 8, 8, 8, 5, 8, 8, 8 ] )
\end{Verbatim}
 }

 

\subsection{\textcolor{Chapter }{IsZeroSimpleSemigroup}}
\logpage{[ 3, 6, 22 ]}\nobreak
\hyperdef{L}{X8193A60F839C064E}{}
{\noindent\textcolor{FuncColor}{$\triangleright$\ \ \texttt{IsZeroSimpleSemigroup({\mdseries\slshape S})\index{IsZeroSimpleSemigroup@\texttt{IsZeroSimpleSemigroup}}
\label{IsZeroSimpleSemigroup}
}\hfill{\scriptsize (property)}}\\
\textbf{\indent Returns:\ }
\texttt{true} or \texttt{false}. 



 \texttt{IsZeroSimpleSemigroup} returns \texttt{true} if the semigroup \mbox{\texttt{\mdseries\slshape S}} is a zero simple semigroup and \texttt{false} if it is not.

 A semigroup is a \emph{zero simple semigroup} if it has no two-sided ideals other than itself and the set containing the
zero element; see also \texttt{MultiplicativeZero} (\ref{MultiplicativeZero}). An inverse semigroup is zero simple if and only if it is a Brandt semigroup;
see \texttt{IsBrandtSemigroup} (\ref{IsBrandtSemigroup}). 
\begin{Verbatim}[commandchars=!@|,fontsize=\small,frame=single,label=Example]
  !gapprompt@gap>| !gapinput@S:=Semigroup( |
  !gapprompt@>| !gapinput@ Transformation( [ 1, 17, 17, 17, 17, 17, 17, 17, 17, 17, 5, 17, |
  !gapprompt@>| !gapinput@ 17, 17, 17, 17, 17 ] ), |
  !gapprompt@>| !gapinput@ Transformation( [ 1, 17, 17, 17, 11, 17, 17, 17, 17, 17, 17, 17, |
  !gapprompt@>| !gapinput@ 17, 17, 17, 17, 17 ] ), |
  !gapprompt@>| !gapinput@ Transformation( [ 1, 17, 17, 17, 17, 17, 17, 17, 17, 17, 4, 17, |
  !gapprompt@>| !gapinput@ 17, 17, 17, 17, 17 ] ), |
  !gapprompt@>| !gapinput@ Transformation( [ 1, 17, 17, 5, 17, 17, 17, 17, 17, 17, 17, 17, |
  !gapprompt@>| !gapinput@ 17, 17, 17, 17, 17 ] ));;|
  !gapprompt@gap>| !gapinput@IsZeroSimpleSemigroup(S);|
  true
  !gapprompt@gap>| !gapinput@S:=Semigroup(|
  !gapprompt@>| !gapinput@Transformation( [ 2, 3, 4, 5, 1, 8, 7, 6, 2, 7 ] ),|
  !gapprompt@>| !gapinput@Transformation([ 2, 3, 4, 5, 6, 8, 7, 1, 2, 2 ] ));;|
  !gapprompt@gap>| !gapinput@IsZeroSimpleSemigroup(S);|
  false
\end{Verbatim}
 }

 }

 
\section{\textcolor{Chapter }{Properties and attributes of inverse semigroups}}\logpage{[ 3, 7, 0 ]}
\hyperdef{L}{X807F8A477A929076}{}
{
  In this section we describe several properties and attributes  (\textbf{Reference: Properties}) of inverse semigroup of transformations or partial permutations that can be
determined using \textsf{Semigroups}. The functions \texttt{IsJoinIrreducible} (\ref{IsJoinIrreducible}), \texttt{IsMajorantlyClosed} (\ref{IsMajorantlyClosed}), \texttt{JoinIrreducibleDClasses} (\ref{JoinIrreducibleDClasses}), \texttt{MajorantClosure} (\ref{MajorantClosure}), \texttt{Minorants} (\ref{Minorants}), \texttt{RightCosetsOfInverseSemigroup} (\ref{RightCosetsOfInverseSemigroup}), \texttt{SmallerDegreePartialPermRepresentation} (\ref{SmallerDegreePartialPermRepresentation}), and \texttt{VagnerPrestonRepresentation} (\ref{VagnerPrestonRepresentation}) were written by Wilf Wilson and Robert Hancock. 

\subsection{\textcolor{Chapter }{IsCliffordSemigroup}}
\logpage{[ 3, 7, 1 ]}\nobreak
\hyperdef{L}{X81DE11987BB81017}{}
{\noindent\textcolor{FuncColor}{$\triangleright$\ \ \texttt{IsCliffordSemigroup({\mdseries\slshape S})\index{IsCliffordSemigroup@\texttt{IsCliffordSemigroup}}
\label{IsCliffordSemigroup}
}\hfill{\scriptsize (property)}}\\
\textbf{\indent Returns:\ }
\texttt{true} or \texttt{false}. 



 \texttt{IsCliffordSemigroup} returns \texttt{true} if the \mbox{\texttt{\mdseries\slshape S}} is regular and its idempotents are central, and \texttt{false} if it is not. 
\begin{Verbatim}[commandchars=!@|,fontsize=\small,frame=single,label=Example]
  !gapprompt@gap>| !gapinput@S:=Semigroup( Transformation( [ 1, 2, 4, 5, 6, 3, 7, 8 ] ), |
  !gapprompt@>| !gapinput@Transformation( [ 3, 3, 4, 5, 6, 2, 7, 8 ] ), |
  !gapprompt@>| !gapinput@Transformation( [ 1, 2, 5, 3, 6, 8, 4, 4 ] ) );;|
  !gapprompt@gap>| !gapinput@IsCliffordSemigroup(S);|
  true
  !gapprompt@gap>| !gapinput@T:=Range(IsomorphismPartialPermSemigroup(S));;|
  !gapprompt@gap>| !gapinput@IsCliffordSemigroup(S);|
  true
\end{Verbatim}
 }

 

\subsection{\textcolor{Chapter }{IsBrandtSemigroup}}
\logpage{[ 3, 7, 2 ]}\nobreak
\hyperdef{L}{X7EFDBA687DCDA6FA}{}
{\noindent\textcolor{FuncColor}{$\triangleright$\ \ \texttt{IsBrandtSemigroup({\mdseries\slshape S})\index{IsBrandtSemigroup@\texttt{IsBrandtSemigroup}}
\label{IsBrandtSemigroup}
}\hfill{\scriptsize (property)}}\\
\textbf{\indent Returns:\ }
\texttt{true} or \texttt{false}. 



 \texttt{IsBrandtSemigroup} return \texttt{true} if the semigroup \mbox{\texttt{\mdseries\slshape S}} is a 0-simple inverse semigroup, and \texttt{false} if it is not. See also \texttt{IsZeroSimpleSemigroup} (\ref{IsZeroSimpleSemigroup}) and \texttt{IsInverseSemigroup} (\textbf{Reference: IsInverseSemigroup}). 
\begin{Verbatim}[commandchars=!@|,fontsize=\small,frame=single,label=Example]
  !gapprompt@gap>| !gapinput@S:=Semigroup(Transformation( [ 2, 8, 8, 8, 8, 8, 8, 8 ] ),|
  !gapprompt@>| !gapinput@Transformation( [ 5, 8, 8, 8, 8, 8, 8, 8 ] ),|
  !gapprompt@>| !gapinput@Transformation( [ 8, 3, 8, 8, 8, 8, 8, 8 ] ),|
  !gapprompt@>| !gapinput@Transformation( [ 8, 6, 8, 8, 8, 8, 8, 8 ] ),|
  !gapprompt@>| !gapinput@Transformation( [ 8, 8, 1, 8, 8, 8, 8, 8 ] ),|
  !gapprompt@>| !gapinput@Transformation( [ 8, 8, 8, 1, 8, 8, 8, 8 ] ),|
  !gapprompt@>| !gapinput@Transformation( [ 8, 8, 8, 8, 4, 8, 8, 8 ] ),|
  !gapprompt@>| !gapinput@Transformation( [ 8, 8, 8, 8, 8, 7, 8, 8 ] ),|
  !gapprompt@>| !gapinput@Transformation( [ 8, 8, 8, 8, 8, 8, 2, 8 ] ));;|
  !gapprompt@gap>| !gapinput@IsBrandtSemigroup(S);|
  true
  !gapprompt@gap>| !gapinput@T:=Range(IsomorphismPartialPermSemigroup(S));;|
  !gapprompt@gap>| !gapinput@IsBrandtSemigroup(T);|
  true
\end{Verbatim}
 }

 

\subsection{\textcolor{Chapter }{IsFactorisableSemigroup}}
\logpage{[ 3, 7, 3 ]}\nobreak
\hyperdef{L}{X862158348720781D}{}
{\noindent\textcolor{FuncColor}{$\triangleright$\ \ \texttt{IsFactorisableSemigroup({\mdseries\slshape S})\index{IsFactorisableSemigroup@\texttt{IsFactorisableSemigroup}}
\label{IsFactorisableSemigroup}
}\hfill{\scriptsize (property)}}\\
\textbf{\indent Returns:\ }
\texttt{true} or \texttt{false}.



 An inverse monoid is \emph{factorisable} if every element is the product of an element of the group of units and an
idempotent; see also \texttt{GroupOfUnits} (\ref{GroupOfUnits}) and \texttt{Idempotents} (\ref{Idempotents}). Hence an inverse semigroup of partial permutations is factorisable if and
only if each of its generators is the restriction of some element in the group
of units. 
\begin{Verbatim}[commandchars=!@|,fontsize=\small,frame=single,label=Example]
  !gapprompt@gap>| !gapinput@S:=InverseSemigroup( PartialPerm( [ 1, 2, 4 ], [ 3, 1, 4 ] ),|
  !gapprompt@>| !gapinput@PartialPerm( [ 1, 2, 3, 5 ], [ 4, 1, 5, 2 ] ) );;|
  !gapprompt@gap>| !gapinput@IsFactorisableSemigroup(S);|
  false
  !gapprompt@gap>| !gapinput@IsFactorisableSemigroup(SymmetricInverseSemigroup(5)); |
  true
\end{Verbatim}
 }

 

\subsection{\textcolor{Chapter }{IsJoinIrreducible}}
\logpage{[ 3, 7, 4 ]}\nobreak
\hyperdef{L}{X817F9F3984FC842C}{}
{\noindent\textcolor{FuncColor}{$\triangleright$\ \ \texttt{IsJoinIrreducible({\mdseries\slshape S, x})\index{IsJoinIrreducible@\texttt{IsJoinIrreducible}}
\label{IsJoinIrreducible}
}\hfill{\scriptsize (operation)}}\\
\textbf{\indent Returns:\ }
 \texttt{true} or \texttt{false}. 



 \texttt{IsJoinIrreducible} determines whether an element \mbox{\texttt{\mdseries\slshape x}} of an inverse semigroup \mbox{\texttt{\mdseries\slshape S}} of partial permutations is join irreducible.

 An element \mbox{\texttt{\mdseries\slshape x}} is \emph{join irreducible} when it is not the least upper bound (with respect to the natural partial
order \texttt{NaturalLeqPartialPerm} (\textbf{Reference: NaturalLeqPartialPerm})) of any subset of \mbox{\texttt{\mdseries\slshape S}} not containing \mbox{\texttt{\mdseries\slshape x}}. }

 
\begin{Verbatim}[commandchars=!@|,fontsize=\small,frame=single,label=Example]
  !gapprompt@gap>| !gapinput@S:=SymmetricInverseSemigroup(3);|
  <symmetric inverse semigroup on 3 pts>
  !gapprompt@gap>| !gapinput@x:=PartialPerm([1,2,3]);|
  <identity partial perm on [ 1, 2, 3 ]>
  !gapprompt@gap>| !gapinput@IsJoinIrreducible(S,x);|
  false
  !gapprompt@gap>| !gapinput@a:=PartialPerm([1,2,4,3]);; b:=PartialPerm([1]);; c:=PartialPerm([0,2]);;|
  !gapprompt@gap>| !gapinput@T:=InverseSemigroup(a,b,c);|
  <inverse partial perm semigroup on 4 pts with 3 generators>
  !gapprompt@gap>| !gapinput@y:=PartialPerm([1,2,3,4]);|
  <identity partial perm on [ 1, 2, 3, 4 ]>
  !gapprompt@gap>| !gapinput@IsJoinIrreducible(T,y);|
  true
\end{Verbatim}
 

\subsection{\textcolor{Chapter }{IsMajorantlyClosed}}
\logpage{[ 3, 7, 5 ]}\nobreak
\hyperdef{L}{X81E6D24F852A7937}{}
{\noindent\textcolor{FuncColor}{$\triangleright$\ \ \texttt{IsMajorantlyClosed({\mdseries\slshape S, T})\index{IsMajorantlyClosed@\texttt{IsMajorantlyClosed}}
\label{IsMajorantlyClosed}
}\hfill{\scriptsize (operation)}}\\
\textbf{\indent Returns:\ }
 \texttt{true} or \texttt{false}. 



 \texttt{IsMajorantlyClosed} determines whether the subset \mbox{\texttt{\mdseries\slshape T}} of the inverse semigroup of partial permutations \mbox{\texttt{\mdseries\slshape S}} is majorantly closed in \mbox{\texttt{\mdseries\slshape S}}.

 We say that \mbox{\texttt{\mdseries\slshape T}} is \emph{majorantly closed} in \mbox{\texttt{\mdseries\slshape S}} if it contains all elements of \mbox{\texttt{\mdseries\slshape S}} which are greater than or equal to any element of \mbox{\texttt{\mdseries\slshape T}}, with respect to the natural partial order. See \texttt{NaturalLeqPartialPerm} (\textbf{Reference: NaturalLeqPartialPerm}).

 Note that \mbox{\texttt{\mdseries\slshape T}} can be a subset of \mbox{\texttt{\mdseries\slshape S}} or an inverse subsemigroup of \mbox{\texttt{\mdseries\slshape S}}. }

 
\begin{Verbatim}[commandchars=!@|,fontsize=\small,frame=single,label=Example]
  !gapprompt@gap>| !gapinput@S:=SymmetricInverseSemigroup(2);|
  <symmetric inverse semigroup on 2 pts>
  !gapprompt@gap>| !gapinput@T:=[Elements(S)[2]];|
  [ <identity partial perm on [ 1 ]> ]
  !gapprompt@gap>| !gapinput@IsMajorantlyClosed(S,T);|
  false
  !gapprompt@gap>| !gapinput@U:=[Elements(S)[2],Elements(S)[6]];|
  [ <identity partial perm on [ 1 ]>, <identity partial perm on [ 1, 2 ]
      > ]
  !gapprompt@gap>| !gapinput@IsMajorantlyClosed(S,U);|
  true
\end{Verbatim}
 

\subsection{\textcolor{Chapter }{IsMonogenicInverseSemigroup}}
\logpage{[ 3, 7, 6 ]}\nobreak
\hyperdef{L}{X7D2641AD830DEC1C}{}
{\noindent\textcolor{FuncColor}{$\triangleright$\ \ \texttt{IsMonogenicInverseSemigroup({\mdseries\slshape S})\index{IsMonogenicInverseSemigroup@\texttt{IsMonogenicInverseSemigroup}}
\label{IsMonogenicInverseSemigroup}
}\hfill{\scriptsize (property)}}\\
\textbf{\indent Returns:\ }
\texttt{true} or \texttt{false}. 



 \texttt{IsMonogenicInverseSemigroup} returns \texttt{true} if the semigroup \mbox{\texttt{\mdseries\slshape S}} is an inverse monogenic semigroup and it returns \texttt{false} if it is not. 

 A inverse semigroup is \emph{monogenic} if it is generated as an inverse semigroup by a single element. See also \texttt{IsMonogenicSemigroup} (\ref{IsMonogenicSemigroup}) and \texttt{IndexPeriodOfTransformation} (\textbf{Reference: IndexPeriodOfTransformation}). 
\begin{Verbatim}[commandchars=!@|,fontsize=\small,frame=single,label=Example]
  !gapprompt@gap>| !gapinput@f:=PartialPerm( [ 1, 2, 3, 6, 8, 10 ], [ 2, 6, 7, 9, 1, 5 ] );;|
  !gapprompt@gap>| !gapinput@S:=InverseSemigroup(f, f^2, f^3);;|
  !gapprompt@gap>| !gapinput@IsMonogenicSemigroup(S);|
  false
  !gapprompt@gap>| !gapinput@IsMonogenicInverseSemigroup(S);|
  true
\end{Verbatim}
 }

 

\subsection{\textcolor{Chapter }{JoinIrreducibleDClasses}}
\logpage{[ 3, 7, 7 ]}\nobreak
\hyperdef{L}{X85CDF93C805AF82A}{}
{\noindent\textcolor{FuncColor}{$\triangleright$\ \ \texttt{JoinIrreducibleDClasses({\mdseries\slshape S})\index{JoinIrreducibleDClasses@\texttt{JoinIrreducibleDClasses}}
\label{JoinIrreducibleDClasses}
}\hfill{\scriptsize (attribute)}}\\
\textbf{\indent Returns:\ }
 A list of $\mathcal{D}$-classes. 



 \texttt{JoinIrreducibleDClasses} returns a list of the join irreducible $\mathcal{D}$-classes of the inverse semigroup of partial permutations \mbox{\texttt{\mdseries\slshape S}}.

 A \emph{join irreducible $\mathcal{D}$-class} is a $\mathcal{D}$-class containing only join irreducible elements. See \texttt{IsJoinIrreducible} (\ref{IsJoinIrreducible}). If a $\mathcal{D}$-class contains one join irreducible element, then all of the elements in the $\mathcal{D}$-class are join irreducible. 
\begin{Verbatim}[commandchars=!@|,fontsize=\small,frame=single,label=Example]
  !gapprompt@gap>| !gapinput@S:=SymmetricInverseSemigroup(3);|
  <symmetric inverse semigroup on 3 pts>
  !gapprompt@gap>| !gapinput@JoinIrreducibleDClasses(S);|
  [ {PartialPerm( [ 1 ], [ 1 ] )} ]
  !gapprompt@gap>| !gapinput@a:=PartialPerm([1,2,4,3]);; b:=PartialPerm([1]);; c:=PartialPerm([0,2]);;|
  !gapprompt@gap>| !gapinput@T:=InverseSemigroup(a,b,c);|
  <inverse partial perm semigroup on 4 pts with 3 generators>
  !gapprompt@gap>| !gapinput@JoinIrreducibleDClasses(T);|
  [ {PartialPerm( [ 1, 2, 3, 4 ], [ 1, 2, 3, 4 ] )}, 
    {PartialPerm( [ 1 ], [ 1 ] )}, {PartialPerm( [ 2 ], [ 2 ] )} ]
\end{Verbatim}
 }

 

\subsection{\textcolor{Chapter }{MajorantClosure}}
\logpage{[ 3, 7, 8 ]}\nobreak
\hyperdef{L}{X801CC67E80898608}{}
{\noindent\textcolor{FuncColor}{$\triangleright$\ \ \texttt{MajorantClosure({\mdseries\slshape S, T})\index{MajorantClosure@\texttt{MajorantClosure}}
\label{MajorantClosure}
}\hfill{\scriptsize (operation)}}\\
\textbf{\indent Returns:\ }
 A majorantly closed list of partial permutations. 



 \texttt{MajorantClosure} returns a majorantly closed subset of an inverse semigroup of partial
permutations, \mbox{\texttt{\mdseries\slshape S}}, as a list. See \texttt{IsMajorantlyClosed} (\ref{IsMajorantlyClosed}). The result contains all elements of \mbox{\texttt{\mdseries\slshape S}} which are greater than or equal to any element of \mbox{\texttt{\mdseries\slshape T}} (with respect to the natural partial order \texttt{NaturalLeqPartialPerm} (\textbf{Reference: NaturalLeqPartialPerm})). In particular, the result contains \mbox{\texttt{\mdseries\slshape T}}.

 Note that \mbox{\texttt{\mdseries\slshape T}} can be a subset of \mbox{\texttt{\mdseries\slshape S}} or an inverse subsemigroup of \mbox{\texttt{\mdseries\slshape S}}. }

 
\begin{Verbatim}[commandchars=!@|,fontsize=\small,frame=single,label=Example]
  !gapprompt@gap>| !gapinput@S:=SymmetricInverseSemigroup(4);|
  <symmetric inverse semigroup on 4 pts>
  !gapprompt@gap>| !gapinput@T:=[PartialPerm([1,0,3,0])];|
  [ <identity partial perm on [ 1, 3 ]> ]
  !gapprompt@gap>| !gapinput@U:=MajorantClosure(S,T);|
  [ <identity partial perm on [ 1, 3 ]>, 
    <identity partial perm on [ 1, 2, 3 ]>, [2,4](1)(3), [4,2](1)(3), 
    <identity partial perm on [ 1, 3, 4 ]>, 
    <identity partial perm on [ 1, 2, 3, 4 ]>, (1)(2,4)(3) ]
\end{Verbatim}
 

\subsection{\textcolor{Chapter }{Minorants}}
\logpage{[ 3, 7, 9 ]}\nobreak
\hyperdef{L}{X84A3DB79816374DB}{}
{\noindent\textcolor{FuncColor}{$\triangleright$\ \ \texttt{Minorants({\mdseries\slshape S, f})\index{Minorants@\texttt{Minorants}}
\label{Minorants}
}\hfill{\scriptsize (operation)}}\\
\textbf{\indent Returns:\ }
 A list of partial permutations. 



 \texttt{Minorants} takes an element \mbox{\texttt{\mdseries\slshape f}} from an inverse semigroup of partial permutations \mbox{\texttt{\mdseries\slshape S}} and returns a list of the minorants of \mbox{\texttt{\mdseries\slshape f}} in \mbox{\texttt{\mdseries\slshape S}}. 

 A \emph{minorant} of \mbox{\texttt{\mdseries\slshape f}} is an element of \mbox{\texttt{\mdseries\slshape S}} which is strictly less than \mbox{\texttt{\mdseries\slshape f}} in the natural partial order of \mbox{\texttt{\mdseries\slshape S}}. See \texttt{NaturalLeqPartialPerm} (\textbf{Reference: NaturalLeqPartialPerm}). }

 
\begin{Verbatim}[commandchars=!@|,fontsize=\small,frame=single,label=Example]
  !gapprompt@gap>| !gapinput@s:=SymmetricInverseSemigroup(3);|
  <symmetric inverse semigroup on 3 pts>
  !gapprompt@gap>| !gapinput@f:=Elements(s)[13];|
  [1,3](2)
  !gapprompt@gap>| !gapinput@Minorants(s,f);|
  [ <empty partial perm>, [1,3], <identity partial perm on [ 2 ]> ]
  !gapprompt@gap>| !gapinput@f:=PartialPerm([3,2,4,0]);|
  [1,3,4](2)
  !gapprompt@gap>| !gapinput@S:=InverseSemigroup(f);|
  <inverse partial perm semigroup on 4 pts with 1 generator>
  !gapprompt@gap>| !gapinput@Minorants(S,f);|
  [ <identity partial perm on [ 2 ]>, [1,3](2), [3,4](2) ]
\end{Verbatim}
 

\subsection{\textcolor{Chapter }{PrimitiveIdempotents}}
\logpage{[ 3, 7, 10 ]}\nobreak
\hyperdef{L}{X80C0C6C37C4A2ABD}{}
{\noindent\textcolor{FuncColor}{$\triangleright$\ \ \texttt{PrimitiveIdempotents({\mdseries\slshape S})\index{PrimitiveIdempotents@\texttt{PrimitiveIdempotents}}
\label{PrimitiveIdempotents}
}\hfill{\scriptsize (attribute)}}\\
\textbf{\indent Returns:\ }
A list of idempotent partial permutations.



 An idempotent in an inverse semigroup \mbox{\texttt{\mdseries\slshape S}} is \emph{primitive} if it is non-zero and minimal with respect to the \texttt{NaturalPartialOrder} (\textbf{Reference: NaturalPartialOrder}) on \mbox{\texttt{\mdseries\slshape S}}. \texttt{PrimitiveIdempotents} returns the list of primitive idempotents in the inverse semigroup of partial
permutations \mbox{\texttt{\mdseries\slshape S}}. 
\begin{Verbatim}[commandchars=!@|,fontsize=\small,frame=single,label=Example]
  !gapprompt@gap>| !gapinput@S:= InverseMonoid(|
  !gapprompt@>| !gapinput@PartialPerm( [ 1 ], [ 4 ] ),|
  !gapprompt@>| !gapinput@PartialPerm( [ 1, 2, 3 ], [ 2, 1, 3 ] ),|
  !gapprompt@>| !gapinput@PartialPerm( [ 1, 2, 3 ], [ 3, 1, 2 ] ) );;|
  !gapprompt@gap>| !gapinput@MultiplicativeZero(S);|
  <empty partial perm>
  !gapprompt@gap>| !gapinput@PrimitiveIdempotents(S);|
  [ <identity partial perm on [ 4 ]>, <identity partial perm on [ 1 ]>, 
    <identity partial perm on [ 2 ]>, <identity partial perm on [ 3 ]> ]
\end{Verbatim}
 }

 

\subsection{\textcolor{Chapter }{RightCosetsOfInverseSemigroup}}
\logpage{[ 3, 7, 11 ]}\nobreak
\hyperdef{L}{X7B5D89A585F8B5EA}{}
{\noindent\textcolor{FuncColor}{$\triangleright$\ \ \texttt{RightCosetsOfInverseSemigroup({\mdseries\slshape S, T})\index{RightCosetsOfInverseSemigroup@\texttt{RightCosetsOfInverseSemigroup}}
\label{RightCosetsOfInverseSemigroup}
}\hfill{\scriptsize (operation)}}\\
\textbf{\indent Returns:\ }
 A list of lists of partial permutations. 



 \texttt{RightCosetsOfInverseSemigroup} takes a majorantly closed inverse subsemigroup \mbox{\texttt{\mdseries\slshape T}} of an inverse semigroup of partial permutations \mbox{\texttt{\mdseries\slshape S}}. See \texttt{IsMajorantlyClosed} (\ref{IsMajorantlyClosed}). The result is a list of the right cosets of \mbox{\texttt{\mdseries\slshape T}} in \mbox{\texttt{\mdseries\slshape S}}.

 For $s \in S$, the right coset $\overline{Ts}$ is defined if and only if $ss^{-1} \in T$, in which case it is defined to be the majorant closure of the set $Ts$. See \texttt{MajorantClosure} (\ref{MajorantClosure}). Distinct cosets are disjoint but do not necessarily partition \mbox{\texttt{\mdseries\slshape S}}. }

 
\begin{Verbatim}[commandchars=!@|,fontsize=\small,frame=single,label=Example]
  !gapprompt@gap>| !gapinput@S:=SymmetricInverseSemigroup(3);|
  <symmetric inverse semigroup on 3 pts>
  !gapprompt@gap>| !gapinput@T:=InverseSemigroup(MajorantClosure(S,[PartialPerm([1])]));|
  <inverse partial perm semigroup on 3 pts with 7 generators>
  !gapprompt@gap>| !gapinput@IsMajorantlyClosed(S,T);|
  true
  !gapprompt@gap>| !gapinput@RC:=RightCosetsOfInverseSemigroup(S,T);|
  [ [ <identity partial perm on [ 1 ]>, 
        <identity partial perm on [ 1, 2 ]>, [2,3](1), [3,2](1), 
        <identity partial perm on [ 1, 3 ]>, 
        <identity partial perm on [ 1, 2, 3 ]>, (1)(2,3) ], 
    [ [1,3], [2,1,3], [1,3](2), (1,3), [1,3,2], (1,3,2), (1,3)(2) ], 
    [ [1,2], (1,2), [1,2,3], [3,1,2], [1,2](3), (1,2)(3), (1,2,3) ] ]
\end{Verbatim}
 

\subsection{\textcolor{Chapter }{SameMinorantsSubgroup}}
\logpage{[ 3, 7, 12 ]}\nobreak
\hyperdef{L}{X83298E9E86A343FF}{}
{\noindent\textcolor{FuncColor}{$\triangleright$\ \ \texttt{SameMinorantsSubgroup({\mdseries\slshape H})\index{SameMinorantsSubgroup@\texttt{SameMinorantsSubgroup}}
\label{SameMinorantsSubgroup}
}\hfill{\scriptsize (attribute)}}\\
\textbf{\indent Returns:\ }
 A list of elements of the group $\mathcal{H}$-class \mbox{\texttt{\mdseries\slshape H}}. 



 Given a group $\mathcal{H}$-class \mbox{\texttt{\mdseries\slshape H}} in an inverse semigroup of partial permutations \texttt{S}, \texttt{SameMinorantsSubgroup} returns a list of the elements of \mbox{\texttt{\mdseries\slshape h}} which have the same strict minorants as the identity element of \mbox{\texttt{\mdseries\slshape h}}. A \emph{strict minorant} of \texttt{x} in \mbox{\texttt{\mdseries\slshape H}} is an element of \texttt{S} which is less than \texttt{x} (with respect to the natural partial order), but is not equal to \texttt{x}.

 The returned list of elements of \mbox{\texttt{\mdseries\slshape H}} describe a subgroup of \mbox{\texttt{\mdseries\slshape H}}. }

 
\begin{Verbatim}[commandchars=!@|,fontsize=\small,frame=single,label=Example]
  !gapprompt@gap>| !gapinput@S:=SymmetricInverseSemigroup(3);|
  <symmetric inverse semigroup on 3 pts>
  !gapprompt@gap>| !gapinput@h:=GroupHClass(GreensDClasses(S)[1]);|
  {PartialPerm( [ 1, 2, 3 ], [ 1, 2, 3 ] )}
  !gapprompt@gap>| !gapinput@Elements(h);|
  [ <identity partial perm on [ 1, 2, 3 ]>, (1)(2,3), (1,2)(3), 
    (1,2,3), (1,3,2), (1,3)(2) ]
  !gapprompt@gap>| !gapinput@SameMinorantsSubgroup(h);|
  [ <identity partial perm on [ 1, 2, 3 ]> ]
  !gapprompt@gap>| !gapinput@a:=PartialPerm([1,2,4,3]);; b:=PartialPerm([1]);; c:=PartialPerm([0,2]);;|
  !gapprompt@gap>| !gapinput@T:=InverseSemigroup(a,b,c);|
  <inverse partial perm semigroup on 4 pts with 3 generators>
  !gapprompt@gap>| !gapinput@Elements(T);|
  [ <empty partial perm>, <identity partial perm on [ 1 ]>, 
    <identity partial perm on [ 2 ]>, 
    <identity partial perm on [ 1, 2, 3, 4 ]>, (1)(2)(3,4) ]
  !gapprompt@gap>| !gapinput@f:=GroupHClass(GreensDClasses(T)[1]);|
  {PartialPerm( [ 1, 2, 3, 4 ], [ 1, 2, 3, 4 ] )}
  !gapprompt@gap>| !gapinput@Elements(f);|
  [ <identity partial perm on [ 1, 2, 3, 4 ]>, (1)(2)(3,4) ]
  !gapprompt@gap>| !gapinput@SameMinorantsSubgroup(f);|
  [ <identity partial perm on [ 1, 2, 3, 4 ]>, (1)(2)(3,4) ]
\end{Verbatim}
 

\subsection{\textcolor{Chapter }{SmallerDegreePartialPermRepresentation}}
\logpage{[ 3, 7, 13 ]}\nobreak
\hyperdef{L}{X786D4E397EA4445D}{}
{\noindent\textcolor{FuncColor}{$\triangleright$\ \ \texttt{SmallerDegreePartialPermRepresentation({\mdseries\slshape S})\index{SmallerDegreePartialPermRepresentation@\texttt{Smaller}\-\texttt{Degree}\-\texttt{Partial}\-\texttt{Perm}\-\texttt{Representation}}
\label{SmallerDegreePartialPermRepresentation}
}\hfill{\scriptsize (attribute)}}\\
\textbf{\indent Returns:\ }
 An isomorphic mapping. 



 \texttt{SmallerDegreePartialPermRepresentation} attempts to reduce the degree of \mbox{\texttt{\mdseries\slshape S}}, an inverse semigroup of partial permutations, and returns an isomorphic
mapping to another inverse semigroup of partial permutations. If the function
can not reduce the degree, the identity mapping is returned. 

 Note that there is no guarantee that the smallest possible degree
representation is returned. For more information see \cite{Schein1992aa}. }

 
\begin{Verbatim}[commandchars=!@|,fontsize=\small,frame=single,label=Example]
  !gapprompt@gap>| !gapinput@S:=InverseSemigroup(PartialPerm([2,1,4,3,6,5,8,7]));|
  <commutative inverse partial perm semigroup on 8 pts with 1 generator>
  !gapprompt@gap>| !gapinput@Elements(S);|
  [ <identity partial perm on [ 1, 2, 3, 4, 5, 6, 7, 8 ]>, 
    (1,2)(3,4)(5,6)(7,8) ]
  !gapprompt@gap>| !gapinput@T:=SmallerDegreePartialPermRepresentation(S);|
  MappingByFunction( <commutative inverse partial perm semigroup 
  of size 2, on 8 pts
   with 1 generator>, <commutative inverse partial perm semigroup 
  on 2 pts
   with 1 generator>, function( x ) ... end, function( x ) ... end )
  !gapprompt@gap>| !gapinput@R:=Range(T);|
  <commutative inverse partial perm semigroup on 2 pts with 1 generator>
  !gapprompt@gap>| !gapinput@Elements(R);|
  [ <identity partial perm on [ 1, 2 ]>, (1,2) ]
\end{Verbatim}
 

\subsection{\textcolor{Chapter }{VagnerPrestonRepresentation}}
\logpage{[ 3, 7, 14 ]}\nobreak
\hyperdef{L}{X7BC49C3487384364}{}
{\noindent\textcolor{FuncColor}{$\triangleright$\ \ \texttt{VagnerPrestonRepresentation({\mdseries\slshape S})\index{VagnerPrestonRepresentation@\texttt{VagnerPrestonRepresentation}}
\label{VagnerPrestonRepresentation}
}\hfill{\scriptsize (function)}}\\
\textbf{\indent Returns:\ }
 An inverse semigroup of partial permutations. 



 \texttt{VagnerPrestonRepresentation} returns an isomorphism from the inverse semigroup \mbox{\texttt{\mdseries\slshape S}} to the inverse semigroup of partial permutations \texttt{T} of degree equal to the size of \mbox{\texttt{\mdseries\slshape S}}, which is obtained using the Vagner-Preston Representation Theorem. 

 More precisely, if $f:S\to T$ is the isomorphism returned by \texttt{VagnerPrestonRepresentation(\mbox{\texttt{\mdseries\slshape S}})} and $x$ is in \mbox{\texttt{\mdseries\slshape S}}, then $f(x)$ is the partial permutation with domain $Sx^{-1}$ and range $Sx^{-1}x$ defined by $f(x): sx^{-1}\mapsto sx^{-1}x$. 

 In many cases, it is possible to find a smaller degree representation than
that provided by \texttt{VagnerPrestonRepresentation} using \texttt{IsomorphismPartialPermSemigroup} (\textbf{Reference: IsomorphismPartialPermSemigroup}) or \texttt{SmallerDegreePartialPermRepresentation} (\ref{SmallerDegreePartialPermRepresentation}). }

 
\begin{Verbatim}[commandchars=!@|,fontsize=\small,frame=single,label=Example]
  !gapprompt@gap>| !gapinput@S:=SymmetricInverseSemigroup(2);|
  <symmetric inverse semigroup on 2 pts>
  !gapprompt@gap>| !gapinput@Size(S);|
  7
  !gapprompt@gap>| !gapinput@iso:=VagnerPrestonRepresentation(S);|
  MappingByFunction( <symmetric inverse semigroup on 2 pts>, <inverse pa\
  rtial perm semigroup on 7 pts
   with 3 generators>, function( x ) ... end, function( x ) ... end )
  !gapprompt@gap>| !gapinput@RespectsMultiplication(iso);|
  true
  !gapprompt@gap>| !gapinput@inv:=InverseGeneralMapping(iso);;|
  !gapprompt@gap>| !gapinput@ForAll(S, x-> (x^iso)^inv=x);|
  true
\end{Verbatim}
 }

 
\section{\textcolor{Chapter }{Visualising the $\mathcal{D}$-class structure}}\logpage{[ 3, 8, 0 ]}
\hyperdef{L}{X7C66D2377E2FB756}{}
{
 In this section, we describe a function for creating a picture of the $\mathcal{D}$-class structure of a semigroup of transformations, partial permutations, or a
subsemigroup of a Rees 0-matrix semigroup. 

\subsection{\textcolor{Chapter }{DotDClasses (for a semigroup)}}
\logpage{[ 3, 8, 1 ]}\nobreak
\hyperdef{L}{X7EDE2122810063D7}{}
{\noindent\textcolor{FuncColor}{$\triangleright$\ \ \texttt{DotDClasses({\mdseries\slshape S})\index{DotDClasses@\texttt{DotDClasses}!for a semigroup}
\label{DotDClasses:for a semigroup}
}\hfill{\scriptsize (attribute)}}\\
\noindent\textcolor{FuncColor}{$\triangleright$\ \ \texttt{DotDClasses({\mdseries\slshape S, record})\index{DotDClasses@\texttt{DotDClasses}}
\label{DotDClasses}
}\hfill{\scriptsize (operation)}}\\
\textbf{\indent Returns:\ }
A string.



 This function produces a graphical representation of the partial order of the $\mathcal{D}$-classes of the semigroup \mbox{\texttt{\mdseries\slshape S}} together with the eggbox diagram of each $\mathcal{D}$-class. The output is in \texttt{dot} format (also known as \texttt{GraphViz}) format. For details about this file format, and information about how to
display or edit this format see \href{http://www.graphviz.org} {\texttt{http://www.graphviz.org}}. 

 The string returned by \texttt{DotDClasses} can be written to a file using the command \texttt{FileString} (\textbf{GAPDoc: FileString}).

 The $\mathcal{D}$-classes are shown as eggbox diagrams with $\mathcal{L}$-classes as rows and $\mathcal{R}$-classes as columns; group $\mathcal{H}$-classes are shaded gray and contain an asterisk. The $\mathcal{D}$-classes are numbered according to their index in \texttt{GreensDClasses(\mbox{\texttt{\mdseries\slshape S}})}, so that a \texttt{i} appears next to the eggbox diagram of \texttt{GreensDClasses(\mbox{\texttt{\mdseries\slshape S}})[i]}. A red line from one $\mathcal{D}$-class to another indicates that higher $\mathcal{D}$-class is greater than the lower one in the $\mathcal{D}$-order on \mbox{\texttt{\mdseries\slshape S}}. 

 If the optional second argument \mbox{\texttt{\mdseries\slshape record}} is present, it can be used to specify some options for output. 
\begin{description}
\item[{number}]  if \texttt{\mbox{\texttt{\mdseries\slshape record}}.number} is \texttt{false}, then the $\mathcal{D}$-classes in the diagram are not numbered according to their index in the list
of $\mathcal{D}$-classes of \mbox{\texttt{\mdseries\slshape S}}. The default value for this option is \texttt{true}. 
\item[{maximal}]  if \texttt{\mbox{\texttt{\mdseries\slshape record}}.maximal} is \texttt{true}, then the structure description of the group $\mathcal{H}$-classes is displayed; see \texttt{StructureDescription} (\textbf{Reference: StructureDescription}). Setting this attribute to \texttt{true} can adversely affect the performance of \texttt{DotDClasses}. The default value for this option is \texttt{false}. 
\end{description}
 
\begin{Verbatim}[commandchars=!@|,fontsize=\small,frame=single,label=Example]
  !gapprompt@gap>| !gapinput@S:=FullTransformationSemigroup(3);|
  <full transformation semigroup on 3 pts>
  !gapprompt@gap>| !gapinput@DotDClasses(S);        |
  "digraph  DClasses {\nnode [shape=plaintext]\nedge [color=red,arrowhe\
  ad=none]\n1 [shape=box style=dotted label=<\n<TABLE BORDER=\"0\" CELL\
  BORDER=\"1\" CELLPADDING=\"10\" CELLSPACING=\"0\" PORT=\"1\">\n<TR BO\
  RDER=\"0\"><TD COLSPAN=\"1\" BORDER=\"0\" >1</TD></TR><TR><TD BGCOLOR\
  =\"grey\">*</TD></TR>\n</TABLE>>];\n2 [shape=box style=dotted label=<\
  \n<TABLE BORDER=\"0\" CELLBORDER=\"1\" CELLPADDING=\"10\" CELLSPACING\
  =\"0\" PORT=\"2\">\n<TR BORDER=\"0\"><TD COLSPAN=\"3\" BORDER=\"0\" >\
  2</TD></TR><TR><TD BGCOLOR=\"grey\">*</TD><TD BGCOLOR=\"grey\">*</TD>\
  <TD></TD></TR>\n<TR><TD BGCOLOR=\"grey\">*</TD><TD></TD><TD BGCOLOR=\
  \"grey\">*</TD></TR>\n<TR><TD></TD><TD BGCOLOR=\"grey\">*</TD><TD BGC\
  OLOR=\"grey\">*</TD></TR>\n</TABLE>>];\n3 [shape=box style=dotted lab\
  el=<\n<TABLE BORDER=\"0\" CELLBORDER=\"1\" CELLPADDING=\"10\" CELLSPA\
  CING=\"0\" PORT=\"3\">\n<TR BORDER=\"0\"><TD COLSPAN=\"1\" BORDER=\"0\
  \" >3</TD></TR><TR><TD BGCOLOR=\"grey\">*</TD></TR>\n<TR><TD BGCOLOR=\
  \"grey\">*</TD></TR>\n<TR><TD BGCOLOR=\"grey\">*</TD></TR>\n</TABLE>>\
  ];\n1 -> 2\n2 -> 3\n }"
  !gapprompt@gap>| !gapinput@FileString(DotDClasses(S), "t3.dot");|
  fail
  !gapprompt@gap>| !gapinput@FileString("t3.dot", DotDClasses(S));|
  966
\end{Verbatim}
 }

 }

 }

 
\chapter{\textcolor{Chapter }{Free inverse semigroups}}\label{Free inverse semigroups}
\logpage{[ 4, 0, 0 ]}
\hyperdef{L}{X7E51292C8755DCF2}{}
{
  This chapter describes the functions in \textsf{Semigroups} for dealing with free inverse semigroups. This part of the manual and the
functions described herein were written by Julius Jonu{\v s}as.

 A \emph{free inverse semigroup} is an inverse semigroup having the following universal property. An inverse
semigroup $F$ is a \emph{free inverse semigroup} on a non-empty set $X$ if $F$ is an inverse semigroup with a map $f$ from $F$ to $X$ such that for every inverse semigroup $S$ and a map $g$ from $X$ to $S$ there exists a unique morphism $g'$ from $F$ to $S$ such that $fg' = g$. This object is unique up to isomorphism. Moreover, by the universal
property, every inverse semigroup can be expressed as a quotient of a free
inverse semigroup.

 The internal representation of an element of a free inverse semigroup uses a
Munn tree. A \emph{Munn tree} is a directed tree with distinguished start and terminal vertices and where
the edges are labeled by generators so that two edges labeled by the same
generator are only incident to the same vertex if one of the edges is coming
in and the other is leaving the vertex. For more information regarding free
inverse semigroups and the Munn representations see Section 5.10 of \cite{howie}. See also  (\textbf{Reference: Inverse semigroups and monoids}),  (\textbf{Reference: Partial permutations}) and  (\textbf{Reference: Free Groups, Monoids and Semigroups}).

 An element of a free inverse semigroup in \textsf{Semigroups} is be displayed, by default, as a shortest word corresponding to the element.
However, there might be more than one word of the minimum length. For example,
if $x$ and $y$ are generators of a free inverse semigroups, then 
\[xyy^{-1}xx^{-1}x^{-1} = xxx^{-1}yy^{-1}x^{-1}.\]
 See \texttt{MinimalWord} (\ref{MinimalWord:for free inverse semigroup element}) Therefore we provide a another method for printing elements of a free inverse
semigroup: a unique canonical form. Suppose an element of a free inverse
semigroup is given as a Munn tree. Let $L$ be the set of words corresponding to the shortest paths from the start vertex
to the leaves of the tree. Also let $w$ be a word corresponding to the shortest path from start to terminal vertices.
The word $vv^{-1}$ is an idempotent for every $v$ in $L$. The canonical form is given by multiplying these idempotents, in shortlex
order, and then postmultiplying by $w$. For example, consider the word $xyy^{-1}xx^{-1}x^{-1}$ again. The words corresponding to the paths to the leaves are in this case $xx$ and $xy$. And $w$ is an empty word since start and terminal vertices are the same. Therefore,
the canonical form is 
\[xxx^{-1}x^{-1}xyy^{-1}x^{-1}.\]
 See \texttt{CanonicalForm} (\ref{CanonicalForm:for a free inverse semigroup element}). 
\section{\textcolor{Chapter }{ Free inverse semigroups}}\label{sect:Free inverse semigroups}
\logpage{[ 4, 1, 0 ]}
\hyperdef{L}{X7E51292C8755DCF2}{}
{
  

\subsection{\textcolor{Chapter }{FreeInverseSemigroup (for a given rank)}}
\logpage{[ 4, 1, 1 ]}\nobreak
\hyperdef{L}{X7F3F9DED8003CBD0}{}
{\noindent\textcolor{FuncColor}{$\triangleright$\ \ \texttt{FreeInverseSemigroup({\mdseries\slshape rank[, name]})\index{FreeInverseSemigroup@\texttt{FreeInverseSemigroup}!for a given rank}
\label{FreeInverseSemigroup:for a given rank}
}\hfill{\scriptsize (operation)}}\\
\noindent\textcolor{FuncColor}{$\triangleright$\ \ \texttt{FreeInverseSemigroup({\mdseries\slshape name1, name2, ...})\index{FreeInverseSemigroup@\texttt{FreeInverseSemigroup}!for a list of names}
\label{FreeInverseSemigroup:for a list of names}
}\hfill{\scriptsize (operation)}}\\
\noindent\textcolor{FuncColor}{$\triangleright$\ \ \texttt{FreeInverseSemigroup({\mdseries\slshape names})\index{FreeInverseSemigroup@\texttt{FreeInverseSemigroup}!for various names}
\label{FreeInverseSemigroup:for various names}
}\hfill{\scriptsize (operation)}}\\
\textbf{\indent Returns:\ }
 A free inverse semigroup. 



 Returns a free inverse semigroup on \mbox{\texttt{\mdseries\slshape rank}} generators, where \mbox{\texttt{\mdseries\slshape rank}} is a positive integer. If \mbox{\texttt{\mdseries\slshape rank}} is not specified, the number of \mbox{\texttt{\mdseries\slshape names}} is used. If \texttt{S} is a free inverse semigroup, then the generators can be accessed by \texttt{S.1}, \texttt{S.2} and so on. 
\begin{Verbatim}[commandchars=!@|,fontsize=\small,frame=single,label=Example]
  !gapprompt@gap>| !gapinput@S := FreeInverseSemigroup(7);|
  <free inverse semigroup on the generators 
  [ x1, x2, x3, x4, x5, x6, x7 ]>
  !gapprompt@gap>| !gapinput@S := FreeInverseSemigroup(7,"s");|
  <free inverse semigroup on the generators 
  [ s1, s2, s3, s4, s5, s6, s7 ]>
  !gapprompt@gap>| !gapinput@S := FreeInverseSemigroup("a", "b", "c");|
  <free inverse semigroup on the generators [ a, b, c ]>
  !gapprompt@gap>| !gapinput@S := FreeInverseSemigroup(["a", "b", "c"]);|
  <free inverse semigroup on the generators [ a, b, c ]>
  !gapprompt@gap>| !gapinput@S.1;|
  a
  !gapprompt@gap>| !gapinput@S.2;|
  b
\end{Verbatim}
 }

 

\subsection{\textcolor{Chapter }{IsFreeInverseSemigroup}}
\logpage{[ 4, 1, 2 ]}\nobreak
\hyperdef{L}{X7B91643B827DA6DB}{}
{\noindent\textcolor{FuncColor}{$\triangleright$\ \ \texttt{IsFreeInverseSemigroup({\mdseries\slshape obj})\index{IsFreeInverseSemigroup@\texttt{IsFreeInverseSemigroup}}
\label{IsFreeInverseSemigroup}
}\hfill{\scriptsize (Category)}}\\


 Every free inverse semigroup in \textsf{GAP} belongs to the category \texttt{IsFreeInverseSemigroup}. Basic operations for a free inverse semigroup are: \texttt{GeneratorsOfInverseSemigroup} (\textbf{Reference: GeneratorsOfInverseSemigroup}) and \texttt{GeneratorsOfSemigroup} (\textbf{Reference: GeneratorsOfSemigroup}). Elements of a free inverse semigroup belong to the category \texttt{IsFreeInverseSemigroupElement} (\ref{IsFreeInverseSemigroupElement}). }

 

\subsection{\textcolor{Chapter }{IsFreeInverseSemigroupElement}}
\logpage{[ 4, 1, 3 ]}\nobreak
\hyperdef{L}{X7999FE0286283CC2}{}
{\noindent\textcolor{FuncColor}{$\triangleright$\ \ \texttt{IsFreeInverseSemigroupElement\index{IsFreeInverseSemigroupElement@\texttt{IsFreeInverseSemigroupElement}}
\label{IsFreeInverseSemigroupElement}
}\hfill{\scriptsize (Category)}}\\


 Every element of a free inverse semigroup belongs to the category \texttt{IsFreeInverseSemigroupElement}. }

 }

 
\section{\textcolor{Chapter }{ Displaying free inverse semigroup elements }}\logpage{[ 4, 2, 0 ]}
\hyperdef{L}{X8073A2387A42B52D}{}
{
  There is a way to change how \textsf{GAP} displays free inverse semigroup elements using the user preference \texttt{FreeInverseSemigroupElementDisplay}. See \texttt{UserPreference} (\textbf{Reference: UserPreference}) for more information about user preferences.

 There are two possible values for \texttt{FreeInverseSemigroupElementDisplay}: 
\begin{description}
\item[{minimal }]  With this option selected, \textsf{GAP} will display a shortest word corresponding to the free inverse semigroup
element. However, this shortest word is not unique. This is a default setting. 
\item[{canonical}]  With this option selected, \textsf{GAP} will display a free inverse semigroup element in the canonical form. 
\end{description}
 
\begin{Verbatim}[commandchars=!@|,fontsize=\small,frame=single,label=Example]
  !gapprompt@gap>| !gapinput@SetUserPreference("semigroups", "FreeInverseSemigroupElementDisplay", "minimal");|
  !gapprompt@gap>| !gapinput@S:=FreeInverseSemigroup(2);|
  <free inverse semigroup on the generators [ x1, x2 ]>
  !gapprompt@gap>| !gapinput@S.1 * S.2;|
  x1*x2
  !gapprompt@gap>| !gapinput@SetUserPreference("semigroups", "FreeInverseSemigroupElementDisplay", "canonical");|
  !gapprompt@gap>| !gapinput@S.1 * S.2;|
  x1x2x2^-1x1^-1x1x2
\end{Verbatim}
 }

 
\section{\textcolor{Chapter }{Operators and operations for free inverse semigroup elements }}\logpage{[ 4, 3, 0 ]}
\hyperdef{L}{X7E93822179CD7602}{}
{
  
\begin{description}
\item[{\texttt{\mbox{\texttt{\mdseries\slshape w}} \texttt{\symbol{94}} -1}}]  returns the semigroup inverse of the free inverse semigroup element \mbox{\texttt{\mdseries\slshape w}}. 
\item[{\texttt{\mbox{\texttt{\mdseries\slshape u}} * \mbox{\texttt{\mdseries\slshape v}}}}]  returns the product of two free inverse semigroup elements \mbox{\texttt{\mdseries\slshape u}} and \mbox{\texttt{\mdseries\slshape v}}. 
\item[{\texttt{\mbox{\texttt{\mdseries\slshape u}} = \mbox{\texttt{\mdseries\slshape v}} }}]  checks if two free inverse semigroup elements are equal, by comparing their
canonical forms. 
\end{description}
 

\subsection{\textcolor{Chapter }{CanonicalForm (for a free inverse semigroup element)}}
\logpage{[ 4, 3, 1 ]}\nobreak
\hyperdef{L}{X7DB7DCEC7E0FE9A3}{}
{\noindent\textcolor{FuncColor}{$\triangleright$\ \ \texttt{CanonicalForm({\mdseries\slshape w})\index{CanonicalForm@\texttt{CanonicalForm}!for a free inverse semigroup element}
\label{CanonicalForm:for a free inverse semigroup element}
}\hfill{\scriptsize (operation)}}\\
\textbf{\indent Returns:\ }
 A string. 



 Every element of a free inverse semigroup has a unique canonical form. If \mbox{\texttt{\mdseries\slshape w}} is such an element, then \texttt{CanonicalForm} returns the canonical form of \mbox{\texttt{\mdseries\slshape w}} as a string. 
\begin{Verbatim}[commandchars=!@|,fontsize=\small,frame=single,label=Example]
  !gapprompt@gap>| !gapinput@S := FreeInverseSemigroup(3);|
  <free inverse semigroup on the generators [ x1, x2, x3 ]>
  !gapprompt@gap>| !gapinput@x := S.1; y := S.2;|
  x1
  x2
  !gapprompt@gap>| !gapinput@CanonicalForm(x^3*y^3);|
  "x1x1x1x2x2x2x2^-1x2^-1x2^-1x1^-1x1^-1x1^-1x1x1x1x2x2x2"
\end{Verbatim}
 }

 

\subsection{\textcolor{Chapter }{MinimalWord (for free inverse semigroup element)}}
\logpage{[ 4, 3, 2 ]}\nobreak
\hyperdef{L}{X87BB5D047EB7C2BF}{}
{\noindent\textcolor{FuncColor}{$\triangleright$\ \ \texttt{MinimalWord({\mdseries\slshape w})\index{MinimalWord@\texttt{MinimalWord}!for free inverse semigroup element}
\label{MinimalWord:for free inverse semigroup element}
}\hfill{\scriptsize (operation)}}\\
\textbf{\indent Returns:\ }
 A string. 



 For an element \mbox{\texttt{\mdseries\slshape w}} of a free inverse semigroup \texttt{S}, \texttt{MinimalWord} returns a word of minimal length equal to \mbox{\texttt{\mdseries\slshape w}} in \texttt{S} as a string.

 Note that there maybe more than one word of minimal length which is equal to \mbox{\texttt{\mdseries\slshape w}} in \texttt{S}. 
\begin{Verbatim}[commandchars=!@|,fontsize=\small,frame=single,label=Example]
  !gapprompt@gap>| !gapinput@S := FreeInverseSemigroup(3);|
  <free inverse semigroup on the generators [ x1, x2, x3 ]>
  !gapprompt@gap>| !gapinput@x := S.1;|
  x1
  !gapprompt@gap>| !gapinput@y := S.2;|
  x2
  !gapprompt@gap>| !gapinput@MinimalWord(x^3 * y^3);|
  "x1*x1*x1*x2*x2*x2"
\end{Verbatim}
 }

 }

 }

 
\chapter{\textcolor{Chapter }{ Orbits }}\label{orbits}
\logpage{[ 5, 0, 0 ]}
\hyperdef{L}{X81E0FF0587C54543}{}
{
  
\section{\textcolor{Chapter }{Looking for something in an orbit}}\logpage{[ 5, 1, 0 ]}
\hyperdef{L}{X7B4CFC5380ACB95F}{}
{
 The functions described in this section supplement the \href{ http://www-groups.mcs.st-and.ac.uk/~neunhoef/Computer/Software/Gap/orb.html } {Orb} package by providing methods for finding something in an orbit, with the
possibility of continuing the orbit enumeration at some later point. 

\subsection{\textcolor{Chapter }{EnumeratePosition}}
\logpage{[ 5, 1, 1 ]}\nobreak
\hyperdef{L}{X7C8B0A2A82C9E4D8}{}
{\noindent\textcolor{FuncColor}{$\triangleright$\ \ \texttt{EnumeratePosition({\mdseries\slshape o, val[, onlynew]})\index{EnumeratePosition@\texttt{EnumeratePosition}}
\label{EnumeratePosition}
}\hfill{\scriptsize (operation)}}\\
\textbf{\indent Returns:\ }
A positive integer or \texttt{fail}.



 This function returns the position of the value \mbox{\texttt{\mdseries\slshape val}} in the orbit \mbox{\texttt{\mdseries\slshape o}}. If \mbox{\texttt{\mdseries\slshape o}} is closed, then this is equivalent to doing \texttt{Position(\mbox{\texttt{\mdseries\slshape o}}, \mbox{\texttt{\mdseries\slshape val}})}. However, if \mbox{\texttt{\mdseries\slshape o}} is open, then the orbit is enumerated until \mbox{\texttt{\mdseries\slshape val}} is found, in which case the position of \mbox{\texttt{\mdseries\slshape val}} is returned, or the enumeration ends, in which case \texttt{fail} is returned. 

 If the optional argument \mbox{\texttt{\mdseries\slshape onlynew}} is present, it should be \texttt{true} or \texttt{false}. If \mbox{\texttt{\mdseries\slshape onlynew}} is \texttt{true}, then \mbox{\texttt{\mdseries\slshape val}} will only be checked against new points in \mbox{\texttt{\mdseries\slshape o}}. Otherwise, every point in the \mbox{\texttt{\mdseries\slshape o}}, not only the new ones, is considered. }

 

\subsection{\textcolor{Chapter }{LookForInOrb}}
\logpage{[ 5, 1, 2 ]}\nobreak
\hyperdef{L}{X7DEB890C7D547574}{}
{\noindent\textcolor{FuncColor}{$\triangleright$\ \ \texttt{LookForInOrb({\mdseries\slshape o, func, start})\index{LookForInOrb@\texttt{LookForInOrb}}
\label{LookForInOrb}
}\hfill{\scriptsize (operation)}}\\
\textbf{\indent Returns:\ }
\texttt{false} or a positive integer.



 The arguments of this function should be an orbit \mbox{\texttt{\mdseries\slshape o}}, a function \mbox{\texttt{\mdseries\slshape func}} which gets the orbit object and a point in the orbit as arguments, and a
positive integer \mbox{\texttt{\mdseries\slshape start}}. The function \mbox{\texttt{\mdseries\slshape func}} will be called for every point in \mbox{\texttt{\mdseries\slshape o}} starting from \mbox{\texttt{\mdseries\slshape start}} (inclusive) and the orbit will be enumerated until \mbox{\texttt{\mdseries\slshape func}} returns \texttt{true} or the enumeration ends. In the former case, the position of the first point
in \mbox{\texttt{\mdseries\slshape o}} for which \mbox{\texttt{\mdseries\slshape func}} returns \texttt{true} is returned, and in the latter \texttt{false} is returned. 
\begin{Verbatim}[commandchars=!@|,fontsize=\small,frame=single,label=Example]
  !gapprompt@gap>| !gapinput@o:=Orb(SymmetricGroup(100), 1, OnPoints);|
  <open Int-orbit, 1 points>
  !gapprompt@gap>| !gapinput@func:=function(o, x) return x=42; end;|
  function( o, x ) ... end
  !gapprompt@gap>| !gapinput@LookForInOrb(o, func, 1);|
  42
  !gapprompt@gap>| !gapinput@o;|
  <open Int-orbit, 42 points>
\end{Verbatim}
 }

 }

 
\section{\textcolor{Chapter }{Strongly connected components of orbits}}\logpage{[ 5, 2, 0 ]}
\hyperdef{L}{X80B19F9E8631D88B}{}
{
  The functions described in this section supplement the \href{ http://www-groups.mcs.st-and.ac.uk/~neunhoef/Computer/Software/Gap/orb.html } {Orb} package by providing methods for operations related to strongly connected
components of orbits.

 If any of the functions is applied to an open orbit, then the orbit is
completely enumerated before any further calculation is performed. It is not
possible to calculate the strongly connected components of an orbit of a
semigroup acting on a set until the entire orbit has been found. 

\subsection{\textcolor{Chapter }{OrbSCC}}
\logpage{[ 5, 2, 1 ]}\nobreak
\hyperdef{L}{X8178A420792E6AAC}{}
{\noindent\textcolor{FuncColor}{$\triangleright$\ \ \texttt{OrbSCC({\mdseries\slshape o})\index{OrbSCC@\texttt{OrbSCC}}
\label{OrbSCC}
}\hfill{\scriptsize (function)}}\\
\textbf{\indent Returns:\ }
The strongly connected components of an orbit.



 If \mbox{\texttt{\mdseries\slshape o}} is an orbit created by the \textsf{Orb} package with the option \texttt{orbitgraph=true}, then \texttt{OrbSCC} returns a set of sets of positions in \mbox{\texttt{\mdseries\slshape o}} corresponding to its strongly connected components. 

 See also \texttt{OrbSCCLookup} (\ref{OrbSCCLookup}) and \texttt{OrbSCCTruthTable} (\ref{OrbSCCTruthTable}). 
\begin{Verbatim}[commandchars=!@|,fontsize=\small,frame=single,label=Example]
  !gapprompt@gap>| !gapinput@S:=FullTransformationSemigroup(4);;|
  !gapprompt@gap>| !gapinput@o:=LambdaOrb(S);|
  <open orbit, 1 points with Schreier tree with log>
  !gapprompt@gap>| !gapinput@OrbSCC(o);|
  [ [ 1 ], [ 2 ], [ 3, 4, 5, 6 ], [ 7, 8, 9, 10, 11, 12 ], 
    [ 13, 14, 15, 16 ] ]
\end{Verbatim}
 }

 

\subsection{\textcolor{Chapter }{OrbSCCLookup}}
\logpage{[ 5, 2, 2 ]}\nobreak
\hyperdef{L}{X814337A47B773F50}{}
{\noindent\textcolor{FuncColor}{$\triangleright$\ \ \texttt{OrbSCCLookup({\mdseries\slshape o})\index{OrbSCCLookup@\texttt{OrbSCCLookup}}
\label{OrbSCCLookup}
}\hfill{\scriptsize (function)}}\\
\textbf{\indent Returns:\ }
A lookup table for the strongly connected components of an orbit. 



 If \mbox{\texttt{\mdseries\slshape o}} is an orbit created by the \textsf{Orb} package with the option \texttt{orbitgraph=true}, then \texttt{OrbSCCLookup} returns a lookup table for its strongly connected components. More precisely, \texttt{OrbSCCLookup(o)[i]} equals the index of the strongly connected component containing \texttt{o[i]}. 

 See also \texttt{OrbSCC} (\ref{OrbSCC}) and \texttt{OrbSCCTruthTable} (\ref{OrbSCCTruthTable}). 
\begin{Verbatim}[commandchars=!@|,fontsize=\small,frame=single,label=Example]
  !gapprompt@gap>| !gapinput@S:=FullTransformationSemigroup(4);;|
  !gapprompt@gap>| !gapinput@o:=LambdaOrb(S);;|
  !gapprompt@gap>| !gapinput@OrbSCC(o);|
  [ [ 1 ], [ 2 ], [ 3, 4, 5, 6 ], [ 7, 8, 9, 10, 11, 12 ], 
    [ 13, 14, 15, 16 ] ]
  !gapprompt@gap>| !gapinput@OrbSCCLookup(o);|
  [ 1, 2, 3, 3, 3, 3, 4, 4, 4, 4, 4, 4, 5, 5, 5, 5 ]
  !gapprompt@gap>| !gapinput@OrbSCCLookup(o)[1]; OrbSCCLookup(o)[4]; OrbSCCLookup(o)[7]; |
  1
  3
  4
\end{Verbatim}
 }

 

\subsection{\textcolor{Chapter }{OrbSCCTruthTable}}
\logpage{[ 5, 2, 3 ]}\nobreak
\hyperdef{L}{X78AFD003840823BD}{}
{\noindent\textcolor{FuncColor}{$\triangleright$\ \ \texttt{OrbSCCTruthTable({\mdseries\slshape o})\index{OrbSCCTruthTable@\texttt{OrbSCCTruthTable}}
\label{OrbSCCTruthTable}
}\hfill{\scriptsize (function)}}\\
\textbf{\indent Returns:\ }
Truth tables for strongly connected components of an orbit. 



 If \mbox{\texttt{\mdseries\slshape o}} is an orbit created by the \textsf{Orb} package with the option \texttt{orbitgraph=true}, then \texttt{OrbSCCTruthTable} returns a list of boolean lists such that \texttt{OrbSCCTruthTable(o)[i][j]} is \texttt{true} if \texttt{j} belongs to \texttt{OrbSCC(o)[i]}.

 See also \texttt{OrbSCC} (\ref{OrbSCC}) and \texttt{OrbSCCLookup} (\ref{OrbSCCLookup}). 
\begin{Verbatim}[commandchars=!@|,fontsize=\small,frame=single,label=Example]
  !gapprompt@gap>| !gapinput@S:=FullTransformationSemigroup(4);;|
  !gapprompt@gap>| !gapinput@o:=LambdaOrb(S);;|
  !gapprompt@gap>| !gapinput@OrbSCC(o);|
  [ [ 1 ], [ 2 ], [ 3, 4, 5, 6 ], [ 7, 8, 9, 10, 11, 12 ], 
    [ 13, 14, 15, 16 ] ]
  !gapprompt@gap>| !gapinput@OrbSCCTruthTable(o);|
  [ [ true, false, false, false, false, false, false, false, false, 
        false, false, false, false, false, false, false ], 
    [ false, true, false, false, false, false, false, false, false, 
        false, false, false, false, false, false, false ], 
    [ false, false, true, true, true, true, false, false, false, false, 
        false, false, false, false, false, false ], 
    [ false, false, false, false, false, false, true, true, true, true, 
        true, true, false, false, false, false ], 
    [ false, false, false, false, false, false, false, false, false, 
        false, false, false, true, true, true, true ] ]
\end{Verbatim}
 }

 

\subsection{\textcolor{Chapter }{ReverseSchreierTreeOfSCC}}
\logpage{[ 5, 2, 4 ]}\nobreak
\hyperdef{L}{X7D9A29B47D743213}{}
{\noindent\textcolor{FuncColor}{$\triangleright$\ \ \texttt{ReverseSchreierTreeOfSCC({\mdseries\slshape o, i})\index{ReverseSchreierTreeOfSCC@\texttt{ReverseSchreierTreeOfSCC}}
\label{ReverseSchreierTreeOfSCC}
}\hfill{\scriptsize (function)}}\\
\textbf{\indent Returns:\ }
The reverse Schreier tree corresponding to the \mbox{\texttt{\mdseries\slshape i}}th strongly connected component of an orbit. 



 If \mbox{\texttt{\mdseries\slshape o}} is an orbit created by the \textsf{Orb} package with the option \texttt{orbitgraph=true} and action \texttt{act}, and \mbox{\texttt{\mdseries\slshape i}} is a positive integer, then \texttt{ReverseSchreierTreeOfSCC(\mbox{\texttt{\mdseries\slshape o}}, \mbox{\texttt{\mdseries\slshape i}})} returns a pair \texttt{[ gen, pos ]} of lists with \texttt{Length(o)} entries such that 
\begin{Verbatim}[commandchars=@|A,fontsize=\small,frame=single,label=Example]
  act(o[j], o!.gens[gen[j]])=o[pos[j]].
\end{Verbatim}
 The pair \texttt{[ gen, pos ]} corresponds to a tree with root \texttt{OrbSCC(o)[i][1]} and a path from every element of \texttt{OrbSCC(o)[i]} to the root. 

 See also \texttt{OrbSCC} (\ref{OrbSCC}), \texttt{TraceSchreierTreeOfSCCBack} (\ref{TraceSchreierTreeOfSCCBack}), \texttt{SchreierTreeOfSCC} (\ref{SchreierTreeOfSCC}), and \texttt{TraceSchreierTreeOfSCCForward} (\ref{TraceSchreierTreeOfSCCForward}). 
\begin{Verbatim}[commandchars=!@|,fontsize=\small,frame=single,label=Example]
  !gapprompt@gap>| !gapinput@S:=Semigroup(Transformation( [ 2, 2, 1, 4, 4 ] ), |
  !gapprompt@>| !gapinput@Transformation( [ 3, 3, 3, 4, 5 ] ),|
  !gapprompt@>| !gapinput@Transformation( [ 5, 1, 4, 5, 5 ] ) );;|
  !gapprompt@gap>| !gapinput@o:=Orb(S, [1..4], OnSets, rec(orbitgraph:=true, schreier:=true));;|
  !gapprompt@gap>| !gapinput@OrbSCC(o);|
  [ [ 1 ], [ 2 ], [ 3, 5, 6, 7, 11 ], [ 4 ], [ 8 ], [ 9 ], [ 10, 12 ] ]
  !gapprompt@gap>| !gapinput@ReverseSchreierTreeOfSCC(o, 3);|
  [ [ ,, fail,, 2, 1, 2,,,, 1 ], [ ,, fail,, 3, 5, 3,,,, 7 ] ]
  !gapprompt@gap>| !gapinput@ReverseSchreierTreeOfSCC(o, 7);|
  [ [ ,,,,,,,,, fail,, 3 ], [ ,,,,,,,,, fail,, 10 ] ]
  !gapprompt@gap>| !gapinput@OnSets(o[11], Generators(S)[1]);|
  [ 1, 4 ]
  !gapprompt@gap>| !gapinput@Position(o, last);|
  7
\end{Verbatim}
 }

 

\subsection{\textcolor{Chapter }{SchreierTreeOfSCC}}
\logpage{[ 5, 2, 5 ]}\nobreak
\hyperdef{L}{X8071C7148255D0DB}{}
{\noindent\textcolor{FuncColor}{$\triangleright$\ \ \texttt{SchreierTreeOfSCC({\mdseries\slshape o, i})\index{SchreierTreeOfSCC@\texttt{SchreierTreeOfSCC}}
\label{SchreierTreeOfSCC}
}\hfill{\scriptsize (function)}}\\
\textbf{\indent Returns:\ }
The Schreier tree corresponding to the \mbox{\texttt{\mdseries\slshape i}}th strongly connected component of an orbit. 



 If \mbox{\texttt{\mdseries\slshape o}} is an orbit created by the \textsf{Orb} package with the option \texttt{orbitgraph=true} and action \texttt{act}, and \mbox{\texttt{\mdseries\slshape i}} is a positive integer, then \texttt{SchreierTreeOfSCC(\mbox{\texttt{\mdseries\slshape o}}, \mbox{\texttt{\mdseries\slshape i}})} returns a pair \texttt{[ gen, pos ]} of lists with \texttt{Length(o)} entries such that 
\begin{Verbatim}[commandchars=@|A,fontsize=\small,frame=single,label=Example]
  act(o[pos[j]], o!.gens[gen[j]])=o[j].
\end{Verbatim}
 The pair \texttt{[ gen, pos ]} corresponds to a tree with root \texttt{OrbSCC(o)[i][1]} and a path from the root to every element of \texttt{OrbSCC(o)[i]}. 

 See also \texttt{OrbSCC} (\ref{OrbSCC}), \texttt{TraceSchreierTreeOfSCCBack} (\ref{TraceSchreierTreeOfSCCBack}), \texttt{ReverseSchreierTreeOfSCC} (\ref{ReverseSchreierTreeOfSCC}), and \texttt{TraceSchreierTreeOfSCCForward} (\ref{TraceSchreierTreeOfSCCForward}). 
\begin{Verbatim}[commandchars=!@|,fontsize=\small,frame=single,label=Example]
  !gapprompt@gap>| !gapinput@S:=Semigroup(Transformation( [ 2, 2, 1, 4, 4 ] ), |
  !gapprompt@>| !gapinput@Transformation( [ 3, 3, 3, 4, 5 ] ),|
  !gapprompt@>| !gapinput@Transformation( [ 5, 1, 4, 5, 5 ] ) );;|
  !gapprompt@gap>| !gapinput@o:=Orb(S, [1..4], OnSets, rec(orbitgraph:=true, schreier:=true));;|
  !gapprompt@gap>| !gapinput@OrbSCC(o);|
  [ [ 1 ], [ 2 ], [ 3, 5, 6, 7, 11 ], [ 4 ], [ 8 ], [ 9 ], [ 10, 12 ] ]
  !gapprompt@gap>| !gapinput@SchreierTreeOfSCC(o, 3);|
  [ [ ,, fail,, 1, 3, 1,,,, 2 ], [ ,, fail,, 7, 5, 3,,,, 6 ] ]
  !gapprompt@gap>| !gapinput@SchreierTreeOfSCC(o, 7);|
  [ [ ,,,,,,,,, fail,, 1 ], [ ,,,,,,,,, fail,, 10 ] ]
  !gapprompt@gap>| !gapinput@OnSets(o[6], Generators(S)[2]);|
  [ 3, 5 ]
  !gapprompt@gap>| !gapinput@Position(o, last);|
  11
\end{Verbatim}
 }

 

\subsection{\textcolor{Chapter }{TraceSchreierTreeOfSCCBack}}
\logpage{[ 5, 2, 6 ]}\nobreak
\hyperdef{L}{X7853DC817C3102A4}{}
{\noindent\textcolor{FuncColor}{$\triangleright$\ \ \texttt{TraceSchreierTreeOfSCCBack({\mdseries\slshape orb, m, nr})\index{TraceSchreierTreeOfSCCBack@\texttt{TraceSchreierTreeOfSCCBack}}
\label{TraceSchreierTreeOfSCCBack}
}\hfill{\scriptsize (function)}}\\
\textbf{\indent Returns:\ }
A word in the generators.



 \mbox{\texttt{\mdseries\slshape orb}} must be an orbit object with a Schreier tree and orbit graph, that is, the
options \texttt{schreier} and \texttt{orbitgraph} must have been set to \texttt{true} during the creation of the orbit, \mbox{\texttt{\mdseries\slshape m}} must be the number of a strongly connected component of \mbox{\texttt{\mdseries\slshape orb}}, and \texttt{nr} must be the number of a point in the \mbox{\texttt{\mdseries\slshape m}}th strongly connect component of \mbox{\texttt{\mdseries\slshape orb}}. 

 This operation traces the result of \texttt{ReverseSchreierTreeOfSCC} (\ref{ReverseSchreierTreeOfSCC}) and with arguments \mbox{\texttt{\mdseries\slshape orb}} and \mbox{\texttt{\mdseries\slshape m}} and returns a word in the generators that maps the point with number \mbox{\texttt{\mdseries\slshape nr}} to the first point in the \mbox{\texttt{\mdseries\slshape m}}th strongly connected component of \mbox{\texttt{\mdseries\slshape orb}}. Here, a word is a list of integers, where positive integers are numbers of
generators. See also \texttt{OrbSCC} (\ref{OrbSCC}), \texttt{ReverseSchreierTreeOfSCC} (\ref{ReverseSchreierTreeOfSCC}), \texttt{SchreierTreeOfSCC} (\ref{SchreierTreeOfSCC}), and \texttt{TraceSchreierTreeOfSCCForward} (\ref{TraceSchreierTreeOfSCCForward}). 
\begin{Verbatim}[commandchars=!@|,fontsize=\small,frame=single,label=Example]
  !gapprompt@gap>| !gapinput@S:=Semigroup(Transformation( [ 1, 3, 4, 1 ] ), |
  !gapprompt@>| !gapinput@Transformation( [ 2, 4, 1, 2 ] ),|
  !gapprompt@>| !gapinput@Transformation( [ 3, 1, 1, 3 ] ), |
  !gapprompt@>| !gapinput@Transformation( [ 3, 3, 4, 1 ] ) );;|
  !gapprompt@gap>| !gapinput@o:=Orb(S, [1..4], OnSets, rec(orbitgraph:=true, schreier:=true));;|
  !gapprompt@gap>| !gapinput@OrbSCC(o);|
  [ [ 1 ], [ 2 ], [ 3 ], [ 4, 5, 6, 7, 8 ], [ 9, 10, 11, 12 ] ]
  !gapprompt@gap>| !gapinput@ReverseSchreierTreeOfSCC(o, 4);               |
  [ [ ,,, fail, 4, 1, 1, 3 ], [ ,,, fail, 4, 4, 4, 4 ] ]
  !gapprompt@gap>| !gapinput@TraceSchreierTreeOfSCCBack(o, 4, 7);|
  [ 1 ]
  !gapprompt@gap>| !gapinput@TraceSchreierTreeOfSCCBack(o, 4, 8);|
  [ 3 ]
\end{Verbatim}
 }

 

\subsection{\textcolor{Chapter }{TraceSchreierTreeOfSCCForward}}
\logpage{[ 5, 2, 7 ]}\nobreak
\hyperdef{L}{X7D2E200A7B2D5946}{}
{\noindent\textcolor{FuncColor}{$\triangleright$\ \ \texttt{TraceSchreierTreeOfSCCForward({\mdseries\slshape orb, m, nr})\index{TraceSchreierTreeOfSCCForward@\texttt{TraceSchreierTreeOfSCCForward}}
\label{TraceSchreierTreeOfSCCForward}
}\hfill{\scriptsize (function)}}\\
\textbf{\indent Returns:\ }
A word in the generators.



 \mbox{\texttt{\mdseries\slshape orb}} must be an orbit object with a Schreier tree and orbit graph, that is, the
options \texttt{schreier} and \texttt{orbitgraph} must have been set to \texttt{true} during the creation of the orbit, \mbox{\texttt{\mdseries\slshape m}} must be the number of a strongly connected component of \mbox{\texttt{\mdseries\slshape orb}}, and \texttt{nr} must be the number of a point in the \mbox{\texttt{\mdseries\slshape m}}th strongly connect component of \mbox{\texttt{\mdseries\slshape orb}}. 

 This operation traces the result of \texttt{SchreierTreeOfSCC} (\ref{SchreierTreeOfSCC}) and with arguments \mbox{\texttt{\mdseries\slshape orb}} and \mbox{\texttt{\mdseries\slshape m}} and returns a word in the generators that maps the first point in the \mbox{\texttt{\mdseries\slshape m}}th strongly connected component of \mbox{\texttt{\mdseries\slshape orb}} to the point with number \mbox{\texttt{\mdseries\slshape nr}}. Here, a word is a list of integers, where positive integers are numbers of
generators. See also \texttt{OrbSCC} (\ref{OrbSCC}), \texttt{ReverseSchreierTreeOfSCC} (\ref{ReverseSchreierTreeOfSCC}), \texttt{SchreierTreeOfSCC} (\ref{SchreierTreeOfSCC}), and \texttt{TraceSchreierTreeOfSCCBack} (\ref{TraceSchreierTreeOfSCCBack}). 
\begin{Verbatim}[commandchars=!@|,fontsize=\small,frame=single,label=Example]
  !gapprompt@gap>| !gapinput@S:=Semigroup(Transformation( [ 1, 3, 4, 1 ] ), |
  !gapprompt@>| !gapinput@Transformation( [ 2, 4, 1, 2 ] ),|
  !gapprompt@>| !gapinput@Transformation( [ 3, 1, 1, 3 ] ), |
  !gapprompt@>| !gapinput@Transformation( [ 3, 3, 4, 1 ] ) );;|
  !gapprompt@gap>| !gapinput@o:=Orb(S, [1..4], OnSets, rec(orbitgraph:=true, schreier:=true));;|
  !gapprompt@gap>| !gapinput@OrbSCC(o);|
  [ [ 1 ], [ 2 ], [ 3 ], [ 4, 5, 6, 7, 8 ], [ 9, 10, 11, 12 ] ]
  !gapprompt@gap>| !gapinput@SchreierTreeOfSCC(o, 4);|
  [ [ ,,, fail, 1, 2, 2, 4 ], [ ,,, fail, 4, 4, 6, 4 ] ]
  !gapprompt@gap>| !gapinput@TraceSchreierTreeOfSCCForward(o, 4, 8);|
  [ 4 ]
  !gapprompt@gap>| !gapinput@TraceSchreierTreeOfSCCForward(o, 4, 7);|
  [ 2, 2 ]
\end{Verbatim}
 }

 }

 }

 \def\bibname{References\logpage{[ "Bib", 0, 0 ]}
\hyperdef{L}{X7A6F98FD85F02BFE}{}
}

\bibliographystyle{alpha}
\bibliography{semigroups}

\addcontentsline{toc}{chapter}{References}

\def\indexname{Index\logpage{[ "Ind", 0, 0 ]}
\hyperdef{L}{X83A0356F839C696F}{}
}

\cleardoublepage
\phantomsection
\addcontentsline{toc}{chapter}{Index}


\printindex

\newpage
\immediate\write\pagenrlog{["End"], \arabic{page}];}
\immediate\closeout\pagenrlog
\end{document}
