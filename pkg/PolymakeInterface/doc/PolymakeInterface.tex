% generated by GAPDoc2LaTeX from XML source (Frank Luebeck)
\documentclass[a4paper,11pt]{report}

\usepackage{a4wide}
\sloppy
\pagestyle{myheadings}
\usepackage{amssymb}
\usepackage[utf8]{inputenc}
\usepackage{makeidx}
\makeindex
\usepackage{color}
\definecolor{FireBrick}{rgb}{0.5812,0.0074,0.0083}
\definecolor{RoyalBlue}{rgb}{0.0236,0.0894,0.6179}
\definecolor{RoyalGreen}{rgb}{0.0236,0.6179,0.0894}
\definecolor{RoyalRed}{rgb}{0.6179,0.0236,0.0894}
\definecolor{LightBlue}{rgb}{0.8544,0.9511,1.0000}
\definecolor{Black}{rgb}{0.0,0.0,0.0}

\definecolor{linkColor}{rgb}{0.0,0.0,0.554}
\definecolor{citeColor}{rgb}{0.0,0.0,0.554}
\definecolor{fileColor}{rgb}{0.0,0.0,0.554}
\definecolor{urlColor}{rgb}{0.0,0.0,0.554}
\definecolor{promptColor}{rgb}{0.0,0.0,0.589}
\definecolor{brkpromptColor}{rgb}{0.589,0.0,0.0}
\definecolor{gapinputColor}{rgb}{0.589,0.0,0.0}
\definecolor{gapoutputColor}{rgb}{0.0,0.0,0.0}

%%  for a long time these were red and blue by default,
%%  now black, but keep variables to overwrite
\definecolor{FuncColor}{rgb}{0.0,0.0,0.0}
%% strange name because of pdflatex bug:
\definecolor{Chapter }{rgb}{0.0,0.0,0.0}
\definecolor{DarkOlive}{rgb}{0.1047,0.2412,0.0064}


\usepackage{fancyvrb}

\usepackage{mathptmx,helvet}
\usepackage[T1]{fontenc}
\usepackage{textcomp}


\usepackage[
            pdftex=true,
            bookmarks=true,        
            a4paper=true,
            pdftitle={Written with GAPDoc},
            pdfcreator={LaTeX with hyperref package / GAPDoc},
            colorlinks=true,
            backref=page,
            breaklinks=true,
            linkcolor=linkColor,
            citecolor=citeColor,
            filecolor=fileColor,
            urlcolor=urlColor,
            pdfpagemode={UseNone}, 
           ]{hyperref}

\newcommand{\maintitlesize}{\fontsize{50}{55}\selectfont}

% write page numbers to a .pnr log file for online help
\newwrite\pagenrlog
\immediate\openout\pagenrlog =\jobname.pnr
\immediate\write\pagenrlog{PAGENRS := [}
\newcommand{\logpage}[1]{\protect\write\pagenrlog{#1, \thepage,}}
%% were never documented, give conflicts with some additional packages

\newcommand{\GAP}{\textsf{GAP}}

%% nicer description environments, allows long labels
\usepackage{enumitem}
\setdescription{style=nextline}

%% depth of toc
\setcounter{tocdepth}{1}





%% command for ColorPrompt style examples
\newcommand{\gapprompt}[1]{\color{promptColor}{\bfseries #1}}
\newcommand{\gapbrkprompt}[1]{\color{brkpromptColor}{\bfseries #1}}
\newcommand{\gapinput}[1]{\color{gapinputColor}{#1}}


\begin{document}

\logpage{[ 0, 0, 0 ]}
\begin{titlepage}
\mbox{}\vfill

\begin{center}{\maintitlesize \textbf{\textsf{PolymakeInterface}\mbox{}}}\\
\vfill

\hypersetup{pdftitle=\textsf{PolymakeInterface}}
\markright{\scriptsize \mbox{}\hfill \textsf{PolymakeInterface} \hfill\mbox{}}
{\Huge \textbf{A package to provide algorithms for fans and cones of polymake to other
packages\mbox{}}}\\
\vfill

{\Huge Version 2013.01.15\mbox{}}\\[1cm]
{15/01/2013\mbox{}}\\[1cm]
\mbox{}\\[2cm]
{\Large \textbf{Thomas Baechler\\
    \mbox{}}}\\
{\Large \textbf{Sebastian Gutsche\\
    \mbox{}}}\\
\hypersetup{pdfauthor=Thomas Baechler\\
    ; Sebastian Gutsche\\
    }
\mbox{}\\[2cm]
\begin{minipage}{12cm}\noindent
(\emph{this manual is still under construction}) \\
\\
 This manual is best viewed as an \textsc{HTML} document. An \textsc{offline} version should be included in the documentation subfolder of the package. \\
\\
 \end{minipage}

\end{center}\vfill

\mbox{}\\
{\mbox{}\\
\small \noindent \textbf{Thomas Baechler\\
    }  Email: \href{mailto://thomas@momo.math.rwth-aachen.de} {\texttt{thomas@momo.math.rwth-aachen.de}}\\
  Homepage: \href{http://wwwb.math.rwth-aachen.de/~thomas/} {\texttt{http://wwwb.math.rwth-aachen.de/\texttt{\symbol{126}}thomas/}}\\
  Address: \begin{minipage}[t]{8cm}\noindent
Thomas Baechler Lehrstuhl B fuer Mathematik RWTH Aachen Templergraben 64 52062
Aachen Germany\end{minipage}
}\\
{\mbox{}\\
\small \noindent \textbf{Sebastian Gutsche\\
    }  Email: \href{mailto://sebastian.gutsche@rwth-aachen.de} {\texttt{sebastian.gutsche@rwth-aachen.de}}\\
  Homepage: \href{http://wwwb.math.rwth-aachen.de/~gutsche/} {\texttt{http://wwwb.math.rwth-aachen.de/\texttt{\symbol{126}}gutsche/}}\\
  Address: \begin{minipage}[t]{8cm}\noindent
Sebastian Gutsche Lehrstuhl B fuer Mathematik, RWTH Aachen Templergraben 64
52062 Aachen Germany\end{minipage}
}\\
\end{titlepage}

\newpage\setcounter{page}{2}
{\small 
\section*{Copyright}
\logpage{[ 0, 0, 1 ]}
 This package may be distributed under the terms and conditions of the GNU
Public License Version 2. \mbox{}}\\[1cm]
{\small 
\section*{Acknowledgements}
\logpage{[ 0, 0, 2 ]}
 \mbox{}}\\[1cm]
\newpage

\def\contentsname{Contents\logpage{[ 0, 0, 3 ]}}

\tableofcontents
\newpage

 \index{\textsf{PolymakeInterface}}     
\chapter{\textcolor{Chapter }{Introduction}}\label{Introduction_automatically_generated_documentation_parts}
\logpage{[ 1, 0, 0 ]}
\hyperdef{L}{X7DFB63A97E67C0A1}{}
{
    
\section{\textcolor{Chapter }{What is the idea of PolymakeInterface}}\label{What_is_the_idea_of_PolymakeInterface_automatically_generated_documentation_parts}
\logpage{[ 1, 1, 0 ]}
\hyperdef{L}{X7B287B7A86173E0D}{}
{
  PolymakeInterface is an GAP-Package that provides a link to the callable
library of the CAS polymake. It is not supposed to do any work by itself, but
to provide the methods in polymake to GAP. All the functions in this package
are supposed to be capsuled by functions in the Convex package, which provides
needed structures and datatypes. Also the functions the have nicer names. This
fact also causes that there are no doumentations for functions in this
package. To get an overview about the supported functions, one might look at
the polymake{\textunderscore}main.cpp file or simply message the author.
Working with this package alone without Convex is not recommended.}

 }

  
\chapter{\textcolor{Chapter }{Installation}}\label{Installation_automatically_generated_documentation_parts}
\logpage{[ 2, 0, 0 ]}
\hyperdef{L}{X8360C04082558A12}{}
{
    
\section{\textcolor{Chapter }{Install polymake}}\label{Install_polymake_automatically_generated_documentation_parts}
\logpage{[ 2, 1, 0 ]}
\hyperdef{L}{X800C29157C1742BE}{}
{
  [ "To make GAP and polymake work together porperly, one has to make sure that
the two systems", "are using the same GMP library.\texttt{\symbol{92}}n", "You
can choose the GMP which polymake uses by the flag --with-gmp=", "in the
polymake configure skript.\texttt{\symbol{92}}n", "However, having BOTH
systems using your systems GMP is HIGHLY recommended." ]}

   
\section{\textcolor{Chapter }{How to install this package}}\label{How_to_install_this_package_automatically_generated_documentation_parts}
\logpage{[ 2, 2, 0 ]}
\hyperdef{L}{X81A5946683F0AD7D}{}
{
  This package can only be compiled on a system that has polymake correctly
installed, like it is said in the polymake wiki itself. For more information
about this please visit \href{http://www.polymake.org} {www.polymake.org}. For installing this package, first make sure you have polymake installed.
Copy it in your GAP pkg-directory and run the configure script (./configure)
with your GAP root-directory as argument. The default is ../../... Then run
make. After this, the package can be loaded via LoadPackage(
"PolymakeInterface" );.}

 }

 \def\indexname{Index\logpage{[ "Ind", 0, 0 ]}
\hyperdef{L}{X83A0356F839C696F}{}
}

\cleardoublepage
\phantomsection
\addcontentsline{toc}{chapter}{Index}


\printindex

\newpage
\immediate\write\pagenrlog{["End"], \arabic{page}];}
\immediate\closeout\pagenrlog
\end{document}
